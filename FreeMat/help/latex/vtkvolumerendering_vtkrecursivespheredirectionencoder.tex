\section{vtkRecursiveSphereDirectionEncoder}

\subsection{Usage}

 vtkRecursiveSphereDirectionEncoder is a direction encoder which uses the 
 vertices of a recursive subdivision of an octahedron (with the vertices
 pushed out onto the surface of an enclosing sphere) to encode directions
 into a two byte value. 


To create an instance of class vtkRecursiveSphereDirectionEncoder, simply
invoke its constructor as follows
\begin{verbatim}
  obj = vtkRecursiveSphereDirectionEncoder
\end{verbatim}
\subsection{Methods}

The class vtkRecursiveSphereDirectionEncoder has several methods that can be used.
  They are listed below.
Note that the documentation is translated automatically from the VTK sources,
and may not be completely intelligible.  When in doubt, consult the VTK website.
In the methods listed below, \verb|obj| is an instance of the vtkRecursiveSphereDirectionEncoder class.
\begin{itemize}
\item  \verb|string = obj.GetClassName ()|

\item  \verb|int = obj.IsA (string name)|

\item  \verb|vtkRecursiveSphereDirectionEncoder = obj.NewInstance ()|

\item  \verb|vtkRecursiveSphereDirectionEncoder = obj.SafeDownCast (vtkObject o)|

\item  \verb|int = obj.GetEncodedDirection (float n[3])| -  Given a normal vector n, return the encoded direction  

\item  \verb|float = obj.GetDecodedGradient (int value)| - / Given an encoded value, return a pointer to the normal vector

\item  \verb|int = obj.GetNumberOfEncodedDirections (void )| -  Return the number of encoded directions

\item  \verb|obj.SetRecursionDepth (int )| -  Set / Get the recursion depth for the subdivision. This
 indicates how many time one triangle on the initial 8-sided
 sphere model is replaced by four triangles formed by connecting
 triangle edge midpoints. A recursion level of 0 yields 8 triangles
 with 6 unique vertices. The normals are the vectors from the
 sphere center through the vertices. The number of directions
 will be 11 since the four normals with 0 z values will be
 duplicated in the table - once with +0 values and the other
 time with -0 values, and an addition index will be used to
 represent the (0,0,0) normal. If we instead choose a recursion 
 level of 6 (the maximum that can fit within 2 bytes) the number
 of directions is 16643, with 16386 unique directions and a 
 zero normal.

\item  \verb|int = obj.GetRecursionDepthMinValue ()| -  Set / Get the recursion depth for the subdivision. This
 indicates how many time one triangle on the initial 8-sided
 sphere model is replaced by four triangles formed by connecting
 triangle edge midpoints. A recursion level of 0 yields 8 triangles
 with 6 unique vertices. The normals are the vectors from the
 sphere center through the vertices. The number of directions
 will be 11 since the four normals with 0 z values will be
 duplicated in the table - once with +0 values and the other
 time with -0 values, and an addition index will be used to
 represent the (0,0,0) normal. If we instead choose a recursion 
 level of 6 (the maximum that can fit within 2 bytes) the number
 of directions is 16643, with 16386 unique directions and a 
 zero normal.

\item  \verb|int = obj.GetRecursionDepthMaxValue ()| -  Set / Get the recursion depth for the subdivision. This
 indicates how many time one triangle on the initial 8-sided
 sphere model is replaced by four triangles formed by connecting
 triangle edge midpoints. A recursion level of 0 yields 8 triangles
 with 6 unique vertices. The normals are the vectors from the
 sphere center through the vertices. The number of directions
 will be 11 since the four normals with 0 z values will be
 duplicated in the table - once with +0 values and the other
 time with -0 values, and an addition index will be used to
 represent the (0,0,0) normal. If we instead choose a recursion 
 level of 6 (the maximum that can fit within 2 bytes) the number
 of directions is 16643, with 16386 unique directions and a 
 zero normal.

\item  \verb|int = obj.GetRecursionDepth ()| -  Set / Get the recursion depth for the subdivision. This
 indicates how many time one triangle on the initial 8-sided
 sphere model is replaced by four triangles formed by connecting
 triangle edge midpoints. A recursion level of 0 yields 8 triangles
 with 6 unique vertices. The normals are the vectors from the
 sphere center through the vertices. The number of directions
 will be 11 since the four normals with 0 z values will be
 duplicated in the table - once with +0 values and the other
 time with -0 values, and an addition index will be used to
 represent the (0,0,0) normal. If we instead choose a recursion 
 level of 6 (the maximum that can fit within 2 bytes) the number
 of directions is 16643, with 16386 unique directions and a 
 zero normal.

\end{itemize}
