\section{AXIS Setup Axis Behavior}

\subsection{Usage}

Control the axis behavior.  There are several versions of the
axis command based on what you would like the axis command to
do.  The first versions set scalings for the current plot.
The general syntax for its use is
\begin{verbatim}
  axis([xmin xmax ymin ymax zmin zmax cmin cmax])
\end{verbatim}
which sets the limits in the X, Y, Z and color axes.  You can
also set only the X, Y and Z axes:
\begin{verbatim}
  axis([xmin xmax ymin ymax zmin zmax])
\end{verbatim}
or only the X and Y axes:
\begin{verbatim}
  axis([xmin xmax ymin ymax])
\end{verbatim}
To retrieve the current axis limits, use the syntax
\begin{verbatim}
  x = axis
\end{verbatim}
where \verb|x| is a 4-vector for 2D plots, and a 6-vector for
3D plots.

There are a number of axis options supported by FreeMat.
The first version sets the axis limits to be automatically
selected by FreeMat for each dimension.  This state is the
default one for new axes created by FreeMat.
\begin{verbatim}
  axis auto
\end{verbatim}
The next option sets all of the axis limits to \verb|manual|
mode.  This state turns off automatic scaling of the axis
based on the children of the current axis object.  
\begin{verbatim}
  axis manual
\end{verbatim}
The next option sets the axis limits to fit tightly around
the data.
\begin{verbatim}
  axis tight
\end{verbatim}
The next option adjusts the axis limits and plotbox aspect
ratio so that the axis fills the position rectangle.
\begin{verbatim}
  axis fill
\end{verbatim}
The next option puts the axis in matrix mode.  This mode
is equivalent to the standard mode, but with the vertical
axis reversed.  Thus, the origin of the coordinate system
is at the top left corner of the plot.  This mode makes
plots of matrix elements look normal (i.e., an identity
matrix goes from upper left to lower right).
\begin{verbatim}
  axis ij
\end{verbatim}
The next option puts the axis in normal mode, with the origin
at the lower left corner.
\begin{verbatim}
  axis xy
\end{verbatim}
The next option sets the axis parameters (specifically the
data aspect ratio) so that equal ticks on each axis represent
equal length.  In this mode, spheres look spherical insteal of
ellipsoidal.
\begin{verbatim}
  axis equal
\end{verbatim}
The next option is the same as \verb|axis equal|, but sets the
plot box to fit tightly around the data (so no background shows
through).  It is the best option to use when displaying images.
\begin{verbatim}
  axis image
\end{verbatim}
The next option makes the axis box square.
\begin{verbatim}
  axis square
\end{verbatim}
The next option restores many of the normal characteristics of
the axis.  In particular, it undoes the effects of \verb|square|
\verb|image| and \verb|equal| modes.
\begin{verbatim}
  axis normal
\end{verbatim}
The next mode freezes axis properties so that 3D objects can
be rotated properly.
\begin{verbatim}
  axis vis3d
\end{verbatim}
The next mode turns off all labels, tick marks and background.
\begin{verbatim}
  axis on
\end{verbatim}
The next mode turns on all labels, tick marks and background.
\begin{verbatim}
  axis off
\end{verbatim}
The next mode is similar to \verb|axis off|, but also repacks the
figure as tightly as possible.  The result is a plot box that
takes up the entire \verb|outerposition| vector.
\begin{verbatim}
  axis maximal
\end{verbatim}
The \verb|axis| command can also be applied to a particular axis
(as opposed to the current axis as returned by \verb|gca|) handle
\begin{verbatim}
  axis(M,...)
\end{verbatim}
