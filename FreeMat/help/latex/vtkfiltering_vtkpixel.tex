\section{vtkPixel}

\subsection{Usage}

 vtkPixel is a concrete implementation of vtkCell to represent a 2D
 orthogonal quadrilateral. Unlike vtkQuad, the corners are at right angles,
 and aligned along x-y-z coordinate axes leading to large increases in
 computational efficiency.

To create an instance of class vtkPixel, simply
invoke its constructor as follows
\begin{verbatim}
  obj = vtkPixel
\end{verbatim}
\subsection{Methods}

The class vtkPixel has several methods that can be used.
  They are listed below.
Note that the documentation is translated automatically from the VTK sources,
and may not be completely intelligible.  When in doubt, consult the VTK website.
In the methods listed below, \verb|obj| is an instance of the vtkPixel class.
\begin{itemize}
\item  \verb|string = obj.GetClassName ()|

\item  \verb|int = obj.IsA (string name)|

\item  \verb|vtkPixel = obj.NewInstance ()|

\item  \verb|vtkPixel = obj.SafeDownCast (vtkObject o)|

\item  \verb|int = obj.GetCellType ()| -  See the vtkCell API for descriptions of these methods.

\item  \verb|int = obj.GetCellDimension ()| -  See the vtkCell API for descriptions of these methods.

\item  \verb|int = obj.GetNumberOfEdges ()| -  See the vtkCell API for descriptions of these methods.

\item  \verb|int = obj.GetNumberOfFaces ()| -  See the vtkCell API for descriptions of these methods.

\item  \verb|vtkCell = obj.GetEdge (int edgeId)| -  See the vtkCell API for descriptions of these methods.

\item  \verb|vtkCell = obj.GetFace (int )| -  See the vtkCell API for descriptions of these methods.

\item  \verb|int = obj.CellBoundary (int subId, double pcoords[3], vtkIdList pts)| -  See the vtkCell API for descriptions of these methods.

\item  \verb|obj.Contour (double value, vtkDataArray cellScalars, vtkIncrementalPointLocator locator, vtkCellArray verts, vtkCellArray lines, vtkCellArray polys, vtkPointData inPd, vtkPointData outPd, vtkCellData inCd, vtkIdType cellId, vtkCellData outCd)| -  See the vtkCell API for descriptions of these methods.

\item  \verb|obj.Clip (double value, vtkDataArray cellScalars, vtkIncrementalPointLocator locator, vtkCellArray polys, vtkPointData inPd, vtkPointData outPd, vtkCellData inCd, vtkIdType cellId, vtkCellData outCd, int insideOut)| -  See the vtkCell API for descriptions of these methods.

\item  \verb|int = obj.GetParametricCenter (double pcoords[3])| -  Return the center of the triangle in parametric coordinates.

\item  \verb|int = obj.Triangulate (int index, vtkIdList ptIds, vtkPoints pts)|

\item  \verb|obj.Derivatives (int subId, double pcoords[3], double values, int dim, double derivs)|

\item  \verb|obj.InterpolateFunctions (double pcoords[3], double weights[4])| -  Compute the interpolation functions/derivatives
 (aka shape functions/derivatives)

\item  \verb|obj.InterpolateDerivs (double pcoords[3], double derivs[8])|

\end{itemize}
