\section{vtkRandomLayoutStrategy}

\subsection{Usage}

 Assigns points to the vertices of a graph randomly within a bounded range.

 .SECION Thanks
 Thanks to Brian Wylie from Sandia National Laboratories for adding incremental
 layout capabilities.

To create an instance of class vtkRandomLayoutStrategy, simply
invoke its constructor as follows
\begin{verbatim}
  obj = vtkRandomLayoutStrategy
\end{verbatim}
\subsection{Methods}

The class vtkRandomLayoutStrategy has several methods that can be used.
  They are listed below.
Note that the documentation is translated automatically from the VTK sources,
and may not be completely intelligible.  When in doubt, consult the VTK website.
In the methods listed below, \verb|obj| is an instance of the vtkRandomLayoutStrategy class.
\begin{itemize}
\item  \verb|string = obj.GetClassName ()|

\item  \verb|int = obj.IsA (string name)|

\item  \verb|vtkRandomLayoutStrategy = obj.NewInstance ()|

\item  \verb|vtkRandomLayoutStrategy = obj.SafeDownCast (vtkObject o)|

\item  \verb|obj.SetRandomSeed (int )| -  Seed the random number generator used to compute point positions.
 This has a significant effect on their final positions when
 the layout is complete.

\item  \verb|int = obj.GetRandomSeedMinValue ()| -  Seed the random number generator used to compute point positions.
 This has a significant effect on their final positions when
 the layout is complete.

\item  \verb|int = obj.GetRandomSeedMaxValue ()| -  Seed the random number generator used to compute point positions.
 This has a significant effect on their final positions when
 the layout is complete.

\item  \verb|int = obj.GetRandomSeed ()| -  Seed the random number generator used to compute point positions.
 This has a significant effect on their final positions when
 the layout is complete.

\item  \verb|obj.SetGraphBounds (double , double , double , double , double , double )| -  Set / get the region in space in which to place the final graph.
 The GraphBounds only affects the results if AutomaticBoundsComputation
 is off.

\item  \verb|obj.SetGraphBounds (double  a[6])| -  Set / get the region in space in which to place the final graph.
 The GraphBounds only affects the results if AutomaticBoundsComputation
 is off.

\item  \verb|double = obj. GetGraphBounds ()| -  Set / get the region in space in which to place the final graph.
 The GraphBounds only affects the results if AutomaticBoundsComputation
 is off.

\item  \verb|obj.SetAutomaticBoundsComputation (int )| -  Turn on/off automatic graph bounds calculation. If this
 boolean is off, then the manually specified GraphBounds is used.
 If on, then the input's bounds us used as the graph bounds.

\item  \verb|int = obj.GetAutomaticBoundsComputation ()| -  Turn on/off automatic graph bounds calculation. If this
 boolean is off, then the manually specified GraphBounds is used.
 If on, then the input's bounds us used as the graph bounds.

\item  \verb|obj.AutomaticBoundsComputationOn ()| -  Turn on/off automatic graph bounds calculation. If this
 boolean is off, then the manually specified GraphBounds is used.
 If on, then the input's bounds us used as the graph bounds.

\item  \verb|obj.AutomaticBoundsComputationOff ()| -  Turn on/off automatic graph bounds calculation. If this
 boolean is off, then the manually specified GraphBounds is used.
 If on, then the input's bounds us used as the graph bounds.

\item  \verb|obj.SetThreeDimensionalLayout (int )| -  Turn on/off layout of graph in three dimensions. If off, graph
 layout occurs in two dimensions. By default, three dimensional
 layout is on.

\item  \verb|int = obj.GetThreeDimensionalLayout ()| -  Turn on/off layout of graph in three dimensions. If off, graph
 layout occurs in two dimensions. By default, three dimensional
 layout is on.

\item  \verb|obj.ThreeDimensionalLayoutOn ()| -  Turn on/off layout of graph in three dimensions. If off, graph
 layout occurs in two dimensions. By default, three dimensional
 layout is on.

\item  \verb|obj.ThreeDimensionalLayoutOff ()| -  Turn on/off layout of graph in three dimensions. If off, graph
 layout occurs in two dimensions. By default, three dimensional
 layout is on.

\item  \verb|obj.SetGraph (vtkGraph graph)| -  Set the graph to layout.

\item  \verb|obj.Layout ()| -  Perform the random layout.

\end{itemize}
