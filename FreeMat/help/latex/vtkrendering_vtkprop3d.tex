\section{vtkProp3D}

\subsection{Usage}

 vtkProp3D is an abstract class used to represent an entity in a rendering
 scene (i.e., vtkProp3D is a vtkProp with an associated transformation
 matrix).  It handles functions related to the position, orientation and
 scaling. It combines these instance variables into one 4x4 transformation
 matrix as follows: [x y z 1] = [x y z 1] Translate(-origin) Scale(scale)
 Rot(y) Rot(x) Rot (z) Trans(origin) Trans(position). Both vtkActor and
 vtkVolume are specializations of class vtkProp.  The constructor defaults
 to: origin(0,0,0) position=(0,0,0) orientation=(0,0,0), no user defined 
 matrix or transform, and no texture map.

To create an instance of class vtkProp3D, simply
invoke its constructor as follows
\begin{verbatim}
  obj = vtkProp3D
\end{verbatim}
\subsection{Methods}

The class vtkProp3D has several methods that can be used.
  They are listed below.
Note that the documentation is translated automatically from the VTK sources,
and may not be completely intelligible.  When in doubt, consult the VTK website.
In the methods listed below, \verb|obj| is an instance of the vtkProp3D class.
\begin{itemize}
\item  \verb|string = obj.GetClassName ()|

\item  \verb|int = obj.IsA (string name)|

\item  \verb|vtkProp3D = obj.NewInstance ()|

\item  \verb|vtkProp3D = obj.SafeDownCast (vtkObject o)|

\item  \verb|obj.ShallowCopy (vtkProp prop)| -  Shallow copy of this vtkProp3D.

\item  \verb|obj.SetPosition (double \_arg1, double \_arg2, double \_arg3)| -  Set/Get/Add the position of the Prop3D in world coordinates.

\item  \verb|obj.SetPosition (double \_arg[3])| -  Set/Get/Add the position of the Prop3D in world coordinates.

\item  \verb|double = obj. GetPosition ()| -  Set/Get/Add the position of the Prop3D in world coordinates.

\item  \verb|obj.AddPosition (double deltaPosition[3])| -  Set/Get/Add the position of the Prop3D in world coordinates.

\item  \verb|obj.AddPosition (double deltaX, double deltaY, double deltaZ)| -  Set/Get/Add the position of the Prop3D in world coordinates.

\item  \verb|obj.SetOrigin (double \_arg1, double \_arg2, double \_arg3)| -  Set/Get the origin of the Prop3D. This is the point about which all 
 rotations take place.

\item  \verb|obj.SetOrigin (double \_arg[3])| -  Set/Get the origin of the Prop3D. This is the point about which all 
 rotations take place.

\item  \verb|double = obj. GetOrigin ()| -  Set/Get the origin of the Prop3D. This is the point about which all 
 rotations take place.

\item  \verb|obj.SetScale (double \_arg1, double \_arg2, double \_arg3)| -  Set/Get the scale of the actor. Scaling in performed independently on the
 X, Y and Z axis. A scale of zero is illegal and will be replaced with one.

\item  \verb|obj.SetScale (double \_arg[3])| -  Set/Get the scale of the actor. Scaling in performed independently on the
 X, Y and Z axis. A scale of zero is illegal and will be replaced with one.

\item  \verb|double = obj. GetScale ()| -  Set/Get the scale of the actor. Scaling in performed independently on the
 X, Y and Z axis. A scale of zero is illegal and will be replaced with one.

\item  \verb|obj.SetScale (double s)| -  Method to set the scale isotropically

\item  \verb|obj.SetUserTransform (vtkLinearTransform transform)| -  In addition to the instance variables such as position and orientation,
 you can add an additional transformation for your own use.  This
 transformation is concatenated with the actor's internal transformation,
 which you implicitly create through the use of SetPosition(),
 SetOrigin() and SetOrientation().
 <p>If the internal transformation
 is identity (i.e. if you don't set the Position, Origin, or
 Orientation) then the actors final transformation will be the
 UserTransform, concatenated with the UserMatrix if the UserMatrix
 is present.

\item  \verb|vtkLinearTransform = obj.GetUserTransform ()| -  In addition to the instance variables such as position and orientation,
 you can add an additional transformation for your own use.  This
 transformation is concatenated with the actor's internal transformation,
 which you implicitly create through the use of SetPosition(),
 SetOrigin() and SetOrientation().
 <p>If the internal transformation
 is identity (i.e. if you don't set the Position, Origin, or
 Orientation) then the actors final transformation will be the
 UserTransform, concatenated with the UserMatrix if the UserMatrix
 is present.

\item  \verb|obj.SetUserMatrix (vtkMatrix4x4 matrix)| -  The UserMatrix can be used in place of UserTransform.

\item  \verb|vtkMatrix4x4 = obj.GetUserMatrix ()| -  The UserMatrix can be used in place of UserTransform.

\item  \verb|obj.GetMatrix (vtkMatrix4x4 m)| -  Return a reference to the Prop3D's 4x4 composite matrix.
 Get the matrix from the position, origin, scale and orientation This
 matrix is cached, so multiple GetMatrix() calls will be efficient.

\item  \verb|obj.GetMatrix (double m[16])| -  Return a reference to the Prop3D's 4x4 composite matrix.
 Get the matrix from the position, origin, scale and orientation This
 matrix is cached, so multiple GetMatrix() calls will be efficient.

\item  \verb|obj.GetBounds (double bounds[6])| -  Get the bounds for this Prop3D as (Xmin,Xmax,Ymin,Ymax,Zmin,Zmax).

\item  \verb|double = obj.GetBounds ()| -  Get the bounds for this Prop3D as (Xmin,Xmax,Ymin,Ymax,Zmin,Zmax).

\item  \verb|double = obj.GetCenter ()| -  Get the center of the bounding box in world coordinates.

\item  \verb|double = obj.GetXRange ()| -  Get the Prop3D's x range in world coordinates.

\item  \verb|double = obj.GetYRange ()| -  Get the Prop3D's y range in world coordinates.

\item  \verb|double = obj.GetZRange ()| -  Get the Prop3D's z range in world coordinates.

\item  \verb|double = obj.GetLength ()| -  Get the length of the diagonal of the bounding box.

\item  \verb|obj.RotateX (double )| -  Rotate the Prop3D in degrees about the X axis using the right hand
 rule. The axis is the Prop3D's X axis, which can change as other
 rotations are performed.  To rotate about the world X axis use
 RotateWXYZ (angle, 1, 0, 0). This rotation is applied before all
 others in the current transformation matrix.

\item  \verb|obj.RotateY (double )| -  Rotate the Prop3D in degrees about the Y axis using the right hand
 rule. The axis is the Prop3D's Y axis, which can change as other
 rotations are performed.  To rotate about the world Y axis use
 RotateWXYZ (angle, 0, 1, 0). This rotation is applied before all
 others in the current transformation matrix.

\item  \verb|obj.RotateZ (double )| -  Rotate the Prop3D in degrees about the Z axis using the right hand
 rule. The axis is the Prop3D's Z axis, which can change as other
 rotations are performed.  To rotate about the world Z axis use
 RotateWXYZ (angle, 0, 0, 1). This rotation is applied before all
 others in the current transformation matrix.

\item  \verb|obj.RotateWXYZ (double , double , double , double )| -  Rotate the Prop3D in degrees about an arbitrary axis specified by
 the last three arguments. The axis is specified in world
 coordinates. To rotate an about its model axes, use RotateX,
 RotateY, RotateZ.

\item  \verb|obj.SetOrientation (double , double , double )| -  Sets the orientation of the Prop3D.  Orientation is specified as
 X,Y and Z rotations in that order, but they are performed as
 RotateZ, RotateX, and finally RotateY.

\item  \verb|obj.SetOrientation (double a[3])| -  Sets the orientation of the Prop3D.  Orientation is specified as
 X,Y and Z rotations in that order, but they are performed as
 RotateZ, RotateX, and finally RotateY.

\item  \verb|double = obj.GetOrientation ()| -  Returns the orientation of the Prop3D as s vector of X,Y and Z rotation.
 The ordering in which these rotations must be done to generate the 
 same matrix is RotateZ, RotateX, and finally RotateY. See also 
 SetOrientation.

\item  \verb|obj.GetOrientation (double o[3])| -  Returns the orientation of the Prop3D as s vector of X,Y and Z rotation.
 The ordering in which these rotations must be done to generate the 
 same matrix is RotateZ, RotateX, and finally RotateY. See also 
 SetOrientation.

\item  \verb|double = obj.GetOrientationWXYZ ()| -  Returns the WXYZ orientation of the Prop3D. 

\item  \verb|obj.AddOrientation (double , double , double )| -  Add to the current orientation. See SetOrientation and
 GetOrientation for more details. This basically does a
 GetOrientation, adds the passed in arguments, and then calls
 SetOrientation.

\item  \verb|obj.AddOrientation (double a[3])| -  Add to the current orientation. See SetOrientation and
 GetOrientation for more details. This basically does a
 GetOrientation, adds the passed in arguments, and then calls
 SetOrientation.

\item  \verb|obj.PokeMatrix (vtkMatrix4x4 matrix)| -  This method modifies the vtkProp3D so that its transformation
 state is set to the matrix specified. The method does this by
 setting appropriate transformation-related ivars to initial
 values (i.e., not transformed), and placing the user-supplied
 matrix into the UserMatrix of this vtkProp3D. If the method is
 called again with a NULL matrix, then the original state of the
 vtkProp3D will be restored. This method is used to support
 picking and assembly structures.

\item  \verb|obj.InitPathTraversal ()| -  Overload vtkProp's method for setting up assembly paths. See
 the documentation for vtkProp.

\item  \verb|long = obj.GetMTime ()| -  Get the vtkProp3D's mtime 

\item  \verb|long = obj.GetUserTransformMatrixMTime ()| -  Get the modified time of the user matrix or user transform.

\item  \verb|obj.ComputeMatrix ()| -  Generate the matrix based on ivars

\item  \verb|vtkMatrix4x4 = obj.GetMatrix ()| -  Is the matrix for this actor identity

\item  \verb|int = obj.GetIsIdentity ()| -  Is the matrix for this actor identity

\end{itemize}
