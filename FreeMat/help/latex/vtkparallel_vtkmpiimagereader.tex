\section{vtkMPIImageReader}

\subsection{Usage}


 vtkMPIImageReader provides the mechanism to read a brick of bytes (or shorts,
 or ints, or floats, or doubles, ...) from a file or series of files.  You can
 use it to read raw image data from files.  You may also be able to subclass
 this to read simple file formats.

 What distinguishes this class from vtkImageReader and vtkImageReader2 is that
 it performs synchronized parallel I/O using the MPIIO layer.  This can make a
 huge difference in file read times, especially when reading in parallel from
 a parallel file system.

 Dispite the name of this class, vtkMPIImageReader will work even if MPI is
 not available.  If MPI is not available or MPIIO is not available or the
 given Controller is not a vtkMPIController (or NULL), then this class will
 silently work exactly like its superclass.  The point is that you can safely
 use this class in applications that may or may not be compiled with MPI (or
 may or may not actually be run with MPI).


To create an instance of class vtkMPIImageReader, simply
invoke its constructor as follows
\begin{verbatim}
  obj = vtkMPIImageReader
\end{verbatim}
\subsection{Methods}

The class vtkMPIImageReader has several methods that can be used.
  They are listed below.
Note that the documentation is translated automatically from the VTK sources,
and may not be completely intelligible.  When in doubt, consult the VTK website.
In the methods listed below, \verb|obj| is an instance of the vtkMPIImageReader class.
\begin{itemize}
\item  \verb|string = obj.GetClassName ()|

\item  \verb|int = obj.IsA (string name)|

\item  \verb|vtkMPIImageReader = obj.NewInstance ()|

\item  \verb|vtkMPIImageReader = obj.SafeDownCast (vtkObject o)|

\item  \verb|vtkMultiProcessController = obj.GetController ()| -  Get/set the multi process controller to use for coordinated reads.  By
 default, set to the global controller.

\item  \verb|obj.SetController (vtkMultiProcessController )| -  Get/set the multi process controller to use for coordinated reads.  By
 default, set to the global controller.

\end{itemize}
