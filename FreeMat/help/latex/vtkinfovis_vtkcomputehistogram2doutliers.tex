\section{vtkComputeHistogram2DOutliers}

\subsection{Usage}

  This class takes a table and one or more vtkImageData histograms as input
  and computes the outliers in that data.  In general it does so by 
  identifying histogram bins that are removed by a median (salt and pepper)
  filter and below a threshold.  This threshold is automatically identified
  to retrieve a number of outliers close to a user-determined value.  This
  value is set by calling SetPreferredNumberOfOutliers(int).

  The image data input can come either as a multiple vtkImageData via the 
  repeatable INPUT\_HISTOGRAM\_IMAGE\_DATA port, or as a single 
  vtkMultiBlockDataSet containing vtkImageData objects as blocks.  One
  or the other must be set, not both (or neither).
 
  The output can be retrieved as a set of row ids in a vtkSelection or
  as a vtkTable containing the actual outlier row data.


To create an instance of class vtkComputeHistogram2DOutliers, simply
invoke its constructor as follows
\begin{verbatim}
  obj = vtkComputeHistogram2DOutliers
\end{verbatim}
\subsection{Methods}

The class vtkComputeHistogram2DOutliers has several methods that can be used.
  They are listed below.
Note that the documentation is translated automatically from the VTK sources,
and may not be completely intelligible.  When in doubt, consult the VTK website.
In the methods listed below, \verb|obj| is an instance of the vtkComputeHistogram2DOutliers class.
\begin{itemize}
\item  \verb|string = obj.GetClassName ()|

\item  \verb|int = obj.IsA (string name)|

\item  \verb|vtkComputeHistogram2DOutliers = obj.NewInstance ()|

\item  \verb|vtkComputeHistogram2DOutliers = obj.SafeDownCast (vtkObject o)|

\item  \verb|obj.SetPreferredNumberOfOutliers (int )|

\item  \verb|int = obj.GetPreferredNumberOfOutliers ()|

\item  \verb|vtkTable = obj.GetOutputTable ()|

\item  \verb|obj.SetInputTableConnection (vtkAlgorithmOutput cxn)| -  Set the input histogram data as a (repeatable) vtkImageData

\item  \verb|obj.SetInputHistogramImageDataConnection (vtkAlgorithmOutput cxn)| -  Set the input histogram data as a vtkMultiBlockData set
 containing multiple vtkImageData objects.

\item  \verb|obj.SetInputHistogramMultiBlockConnection (vtkAlgorithmOutput cxn)|

\end{itemize}
