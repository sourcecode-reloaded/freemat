\section{vtkPExodusIIReader}

\subsection{Usage}

 vtkPExodusIIReader is a unstructured grid source object that reads
 ExodusII files. Most of the meta data associated with the
 file is loaded when UpdateInformation is called. This includes
 information like Title, number of blocks, number and names of
 arrays. This data can be retrieved from methods in this
 reader. Separate arrays that are meant to be a single vector, are
 combined internally for convenience. To be combined, the array
 names have to be identical except for a trailing X,Y and Z (or
 x,y,z). By default all cell and point arrays are loaded. However,
 the user can flag arrays not to load with the methods
 ''SetPointDataArrayLoadFlag'' and ''SetCellDataArrayLoadFlag''. The
 reader responds to piece requests by loading only a range of the
 possible blocks. Unused points are filtered out internally.

To create an instance of class vtkPExodusIIReader, simply
invoke its constructor as follows
\begin{verbatim}
  obj = vtkPExodusIIReader
\end{verbatim}
\subsection{Methods}

The class vtkPExodusIIReader has several methods that can be used.
  They are listed below.
Note that the documentation is translated automatically from the VTK sources,
and may not be completely intelligible.  When in doubt, consult the VTK website.
In the methods listed below, \verb|obj| is an instance of the vtkPExodusIIReader class.
\begin{itemize}
\item  \verb|string = obj.GetClassName ()|

\item  \verb|int = obj.IsA (string name)|

\item  \verb|vtkPExodusIIReader = obj.NewInstance ()|

\item  \verb|vtkPExodusIIReader = obj.SafeDownCast (vtkObject o)|

\item  \verb|obj.SetController (vtkMultiProcessController c)| -  Set/get the communication object used to relay a list of files
 from the rank 0 process to all others. This is the only interprocess
 communication required by vtkPExodusIIReader.

\item  \verb|vtkMultiProcessController = obj.GetController ()| -  Set/get the communication object used to relay a list of files
 from the rank 0 process to all others. This is the only interprocess
 communication required by vtkPExodusIIReader.

\item  \verb|obj.SetFilePattern (string )| -  These methods tell the reader that the data is ditributed across
 multiple files. This is for distributed execution. It this case,
 pieces are mapped to files. The pattern should have one %d to
 format the file number. FileNumberRange is used to generate file
 numbers. I was thinking of having an arbitrary list of file
 numbers. This may happen in the future. (That is why there is no
 GetFileNumberRange method.

\item  \verb|string = obj.GetFilePattern ()| -  These methods tell the reader that the data is ditributed across
 multiple files. This is for distributed execution. It this case,
 pieces are mapped to files. The pattern should have one %d to
 format the file number. FileNumberRange is used to generate file
 numbers. I was thinking of having an arbitrary list of file
 numbers. This may happen in the future. (That is why there is no
 GetFileNumberRange method.

\item  \verb|obj.SetFilePrefix (string )| -  These methods tell the reader that the data is ditributed across
 multiple files. This is for distributed execution. It this case,
 pieces are mapped to files. The pattern should have one %d to
 format the file number. FileNumberRange is used to generate file
 numbers. I was thinking of having an arbitrary list of file
 numbers. This may happen in the future. (That is why there is no
 GetFileNumberRange method.

\item  \verb|string = obj.GetFilePrefix ()| -  These methods tell the reader that the data is ditributed across
 multiple files. This is for distributed execution. It this case,
 pieces are mapped to files. The pattern should have one %d to
 format the file number. FileNumberRange is used to generate file
 numbers. I was thinking of having an arbitrary list of file
 numbers. This may happen in the future. (That is why there is no
 GetFileNumberRange method.

\item  \verb|obj.SetFileRange (int , int )| -  Set the range of files that are being loaded. The range for single
 file should add to 0.

\item  \verb|obj.SetFileRange (int r)| -  Set the range of files that are being loaded. The range for single
 file should add to 0.

\item  \verb|int = obj. GetFileRange ()| -  Set the range of files that are being loaded. The range for single
 file should add to 0.

\item  \verb|obj.SetFileName (string name)|

\item  \verb|int = obj.GetNumberOfFileNames ()| -    Return the number of files to be read.

\item  \verb|int = obj.GetNumberOfFiles ()| -    Return the number of files to be read.

\item  \verb|vtkIdType = obj.GetTotalNumberOfElements ()|

\item  \verb|vtkIdType = obj.GetTotalNumberOfNodes ()|

\item  \verb|obj.UpdateTimeInformation ()| -  Calls UpdateTimeInformation() on all serial readers so they'll re-read
 their time info from the file. 
 The last time step that they all have in common
 is stored in LastCommonTimeStep, which is used in RequestInformation()
 to override the output time-specific information keys with the range
 of times that ALL readers can actually read.

\item  \verb|obj.Broadcast (vtkMultiProcessController ctrl)| -  Sends metadata (that read from the input file, not settings modified
 through this API) from the rank 0 node to all other processes in a job.

\end{itemize}
