\section{vtkMergeFields}

\subsection{Usage}

 vtkMergeFields is used to merge mutliple field into one.
 The new field is put in the same field data as the original field.
 For example
 @verbatim
 mf->SetOutputField(''foo'', vtkMergeFields::POINT\_DATA);
 mf->SetNumberOfComponents(2);
 mf->Merge(0, ''array1'', 1);
 mf->Merge(1, ''array2'', 0);
 @endverbatim
 will tell vtkMergeFields to use the 2nd component of array1 and
 the 1st component of array2 to create a 2 component field called foo.
 The same can be done using Tcl:
 @verbatim
 mf SetOutputField foo POINT\_DATA
 mf Merge 0 array1 1
 mf Merge 1 array2 0

 Field locations: DATA\_OBJECT, POINT\_DATA, CELL\_DATA
 @endverbatim

To create an instance of class vtkMergeFields, simply
invoke its constructor as follows
\begin{verbatim}
  obj = vtkMergeFields
\end{verbatim}
\subsection{Methods}

The class vtkMergeFields has several methods that can be used.
  They are listed below.
Note that the documentation is translated automatically from the VTK sources,
and may not be completely intelligible.  When in doubt, consult the VTK website.
In the methods listed below, \verb|obj| is an instance of the vtkMergeFields class.
\begin{itemize}
\item  \verb|string = obj.GetClassName ()|

\item  \verb|int = obj.IsA (string name)|

\item  \verb|vtkMergeFields = obj.NewInstance ()|

\item  \verb|vtkMergeFields = obj.SafeDownCast (vtkObject o)|

\item  \verb|obj.SetOutputField (string name, int fieldLoc)| -  The output field will have the given name and it will be in
 fieldLoc (the input fields also have to be in fieldLoc).

\item  \verb|obj.SetOutputField (string name, string fieldLoc)| -  Helper method used by the other language bindings. Allows the caller to
 specify arguments as strings instead of enums.Returns an operation id 
 which can later be used to remove the operation.

\item  \verb|obj.Merge (int component, string arrayName, int sourceComp)| -  Add a component (arrayName,sourceComp) to the output field.

\item  \verb|obj.SetNumberOfComponents (int )| -  Set the number of the components in the output field.
 This has to be set before execution. Default value is 0.

\item  \verb|int = obj.GetNumberOfComponents ()| -  Set the number of the components in the output field.
 This has to be set before execution. Default value is 0.

\end{itemize}
