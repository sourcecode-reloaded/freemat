\section{vtkObject}

\subsection{Usage}

 vtkObject is the base class for most objects in the visualization
 toolkit. vtkObject provides methods for tracking modification time,
 debugging, printing, and event callbacks. Most objects created
 within the VTK framework should be a subclass of vtkObject or one
 of its children.  The few exceptions tend to be very small helper
 classes that usually never get instantiated or situations where
 multiple inheritance gets in the way.  vtkObject also performs
 reference counting: objects that are reference counted exist as
 long as another object uses them. Once the last reference to a
 reference counted object is removed, the object will spontaneously
 destruct.

To create an instance of class vtkObject, simply
invoke its constructor as follows
\begin{verbatim}
  obj = vtkObject
\end{verbatim}
\subsection{Methods}

The class vtkObject has several methods that can be used.
  They are listed below.
Note that the documentation is translated automatically from the VTK sources,
and may not be completely intelligible.  When in doubt, consult the VTK website.
In the methods listed below, \verb|obj| is an instance of the vtkObject class.
\begin{itemize}
\item  \verb|string = obj.GetClassName ()|

\item  \verb|int = obj.IsA (string name)|

\item  \verb|vtkObject = obj.NewInstance ()|

\item  \verb|vtkObject = obj.SafeDownCast (vtkObject o)|

\item  \verb|obj.DebugOn ()| -  Turn debugging output on.

\item  \verb|obj.DebugOff ()| -  Turn debugging output off.

\item  \verb|char = obj.GetDebug ()| -  Get the value of the debug flag.

\item  \verb|obj.SetDebug (char debugFlag)| -  Set the value of the debug flag. A non-zero value turns debugging on.

\item  \verb|obj.Modified ()| -  Update the modification time for this object. Many filters rely on
 the modification time to determine if they need to recompute their
 data. The modification time is a unique monotonically increasing
 unsigned long integer.

\item  \verb|long = obj.GetMTime ()| -  Return this object's modified time.

\item  \verb|obj.RemoveObserver (long tag)| -  Allow people to add/remove/invoke observers (callbacks) to any VTK
 object.  This is an implementation of the subject/observer design
 pattern. An observer is added by specifying an event to respond to
 and a vtkCommand to execute. It returns an unsigned long tag which
 can be used later to remove the event or retrieve the command.
 When events are invoked, the observers are called in the order they
 were added. If a priority value is specified, then the higher 
 priority commands are called first. A command may set an abort
 flag to stop processing of the event. (See vtkCommand.h for more
 information.)

\item  \verb|obj.RemoveObservers (long event)| -  Allow people to add/remove/invoke observers (callbacks) to any VTK
 object.  This is an implementation of the subject/observer design
 pattern. An observer is added by specifying an event to respond to
 and a vtkCommand to execute. It returns an unsigned long tag which
 can be used later to remove the event or retrieve the command.
 When events are invoked, the observers are called in the order they
 were added. If a priority value is specified, then the higher 
 priority commands are called first. A command may set an abort
 flag to stop processing of the event. (See vtkCommand.h for more
 information.)

\item  \verb|obj.RemoveObservers (string event)| -  Allow people to add/remove/invoke observers (callbacks) to any VTK
 object.  This is an implementation of the subject/observer design
 pattern. An observer is added by specifying an event to respond to
 and a vtkCommand to execute. It returns an unsigned long tag which
 can be used later to remove the event or retrieve the command.
 When events are invoked, the observers are called in the order they
 were added. If a priority value is specified, then the higher 
 priority commands are called first. A command may set an abort
 flag to stop processing of the event. (See vtkCommand.h for more
 information.)

\item  \verb|obj.RemoveAllObservers ()| -  Allow people to add/remove/invoke observers (callbacks) to any VTK
 object.  This is an implementation of the subject/observer design
 pattern. An observer is added by specifying an event to respond to
 and a vtkCommand to execute. It returns an unsigned long tag which
 can be used later to remove the event or retrieve the command.
 When events are invoked, the observers are called in the order they
 were added. If a priority value is specified, then the higher 
 priority commands are called first. A command may set an abort
 flag to stop processing of the event. (See vtkCommand.h for more
 information.)

\item  \verb|int = obj.HasObserver (long event)| -  Allow people to add/remove/invoke observers (callbacks) to any VTK
 object.  This is an implementation of the subject/observer design
 pattern. An observer is added by specifying an event to respond to
 and a vtkCommand to execute. It returns an unsigned long tag which
 can be used later to remove the event or retrieve the command.
 When events are invoked, the observers are called in the order they
 were added. If a priority value is specified, then the higher 
 priority commands are called first. A command may set an abort
 flag to stop processing of the event. (See vtkCommand.h for more
 information.)

\item  \verb|int = obj.HasObserver (string event)| -  Allow people to add/remove/invoke observers (callbacks) to any VTK
 object.  This is an implementation of the subject/observer design
 pattern. An observer is added by specifying an event to respond to
 and a vtkCommand to execute. It returns an unsigned long tag which
 can be used later to remove the event or retrieve the command.
 When events are invoked, the observers are called in the order they
 were added. If a priority value is specified, then the higher 
 priority commands are called first. A command may set an abort
 flag to stop processing of the event. (See vtkCommand.h for more
 information.)

\item  \verb|int = obj.InvokeEvent (long event)|

\item  \verb|int = obj.InvokeEvent (string event)|

\end{itemize}
