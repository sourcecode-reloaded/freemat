\section{vtkClosedSurfacePointPlacer}

\subsection{Usage}

 This placer takes a set of boudning planes and constraints the validity
 within the supplied convex planes. It is used by the 
 ParallelopPipedRepresentation to place constraints on the motion the 
 handles within the parallelopiped.
 

To create an instance of class vtkClosedSurfacePointPlacer, simply
invoke its constructor as follows
\begin{verbatim}
  obj = vtkClosedSurfacePointPlacer
\end{verbatim}
\subsection{Methods}

The class vtkClosedSurfacePointPlacer has several methods that can be used.
  They are listed below.
Note that the documentation is translated automatically from the VTK sources,
and may not be completely intelligible.  When in doubt, consult the VTK website.
In the methods listed below, \verb|obj| is an instance of the vtkClosedSurfacePointPlacer class.
\begin{itemize}
\item  \verb|string = obj.GetClassName ()| -  Standard methods for instances of this class.

\item  \verb|int = obj.IsA (string name)| -  Standard methods for instances of this class.

\item  \verb|vtkClosedSurfacePointPlacer = obj.NewInstance ()| -  Standard methods for instances of this class.

\item  \verb|vtkClosedSurfacePointPlacer = obj.SafeDownCast (vtkObject o)| -  Standard methods for instances of this class.

\item  \verb|obj.AddBoundingPlane (vtkPlane plane)| -  A collection of plane equations used to bound the position of the point.
 This is in addition to confining the point to a plane - these contraints
 are meant to, for example, keep a point within the extent of an image.
 Using a set of plane equations allows for more complex bounds (such as
 bounding a point to an oblique reliced image that has hexagonal shape)
 than a simple extent.

\item  \verb|obj.RemoveBoundingPlane (vtkPlane plane)| -  A collection of plane equations used to bound the position of the point.
 This is in addition to confining the point to a plane - these contraints
 are meant to, for example, keep a point within the extent of an image.
 Using a set of plane equations allows for more complex bounds (such as
 bounding a point to an oblique reliced image that has hexagonal shape)
 than a simple extent.

\item  \verb|obj.RemoveAllBoundingPlanes ()| -  A collection of plane equations used to bound the position of the point.
 This is in addition to confining the point to a plane - these contraints
 are meant to, for example, keep a point within the extent of an image.
 Using a set of plane equations allows for more complex bounds (such as
 bounding a point to an oblique reliced image that has hexagonal shape)
 than a simple extent.

\item  \verb|obj.SetBoundingPlanes (vtkPlaneCollection )| -  A collection of plane equations used to bound the position of the point.
 This is in addition to confining the point to a plane - these contraints
 are meant to, for example, keep a point within the extent of an image.
 Using a set of plane equations allows for more complex bounds (such as
 bounding a point to an oblique reliced image that has hexagonal shape)
 than a simple extent.

\item  \verb|vtkPlaneCollection = obj.GetBoundingPlanes ()| -  A collection of plane equations used to bound the position of the point.
 This is in addition to confining the point to a plane - these contraints
 are meant to, for example, keep a point within the extent of an image.
 Using a set of plane equations allows for more complex bounds (such as
 bounding a point to an oblique reliced image that has hexagonal shape)
 than a simple extent.

\item  \verb|obj.SetBoundingPlanes (vtkPlanes planes)| -  A collection of plane equations used to bound the position of the point.
 This is in addition to confining the point to a plane - these contraints
 are meant to, for example, keep a point within the extent of an image.
 Using a set of plane equations allows for more complex bounds (such as
 bounding a point to an oblique reliced image that has hexagonal shape)
 than a simple extent.

\item  \verb|int = obj.ComputeWorldPosition (vtkRenderer ren, double displayPos[2], double worldPos[3], double worldOrient[9])| -  Given a renderer and a display position, compute the 
 world position and world orientation for this point. 
 A plane is defined by a combination of the
 ProjectionNormal, ProjectionOrigin, and ObliquePlane
 ivars. The display position is projected onto this
 plane to determine a world position, and the 
 orientation is set to the normal of the plane. If
 the point cannot project onto the plane or if it
 falls outside the bounds imposed by the
 BoundingPlanes, then 0 is returned, otherwise 1 is
 returned to indicate a valid return position and
 orientation.

\item  \verb|int = obj.ComputeWorldPosition (vtkRenderer ren, double displayPos[2], double refWorldPos[2], double worldPos[3], double worldOrient[9])| -  Given a renderer, a display position and a reference position, ''worldPos''
 is calculated as :
   Consider the line ''L'' that passes through the supplied ''displayPos'' and
 is parallel to the direction of projection of the camera. Clip this line
 segment with the parallelopiped, let's call it ''L\_segment''. The computed 
 world position, ''worldPos'' will be the point on ''L\_segment'' that is 
 closest to refWorldPos.
 NOTE: Note that a set of bounding planes must be supplied. The Oblique
       plane, if supplied is ignored.

\item  \verb|int = obj.ValidateWorldPosition (double worldPos[3])| -  Give a world position check if it is valid - does
 it lie on the plane and within the bounds? Returns
 1 if it is valid, 0 otherwise.

\item  \verb|int = obj.ValidateWorldPosition (double worldPos[3], double worldOrient[9])|

\item  \verb|obj.SetMinimumDistance (double )|

\item  \verb|double = obj.GetMinimumDistanceMinValue ()|

\item  \verb|double = obj.GetMinimumDistanceMaxValue ()|

\item  \verb|double = obj.GetMinimumDistance ()|

\end{itemize}
