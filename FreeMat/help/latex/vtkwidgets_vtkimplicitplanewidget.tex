\section{vtkImplicitPlaneWidget}

\subsection{Usage}

 This 3D widget defines an infinite plane that can be interactively placed
 in a scene. The widget is represented by a plane with a normal vector; the
 plane is contained by a bounding box, and where the plane intersects the
 bounding box the edges are shown (possibly tubed). The normal can be
 selected and moved to rotate the plane; the plane itself can be selected
 and translated in various directions. As the plane is moved, the implicit
 plane function and polygon (representing the plane cut against the bounding
 box) is updated.

 To use this object, just invoke SetInteractor() with the argument of the
 method a vtkRenderWindowInteractor.  You may also wish to invoke
 ''PlaceWidget()'' to initially position the widget. If the ''i'' key (for
 ''interactor'') is pressed, the vtkImplicitPlaneWidget will appear. (See
 superclass documentation for information about changing this behavior.)
 If you select the normal vector, the plane can be arbitrarily rotated. The
 plane can be translated along the normal by selecting the plane and moving
 it. The plane (the plane origin) can also be arbitrary moved by selecting
 the plane with the middle mouse button. The right mouse button can be used
 to uniformly scale the bounding box (moving ''up'' the box scales larger;
 moving ''down'' the box scales smaller). Events that occur outside of the
 widget (i.e., no part of the widget is picked) are propagated to any other
 registered obsevers (such as the interaction style).  Turn off the widget
 by pressing the ''i'' key again (or invoke the Off() method).

 The vtkImplicitPlaneWidget has several methods that can be used in
 conjunction with other VTK objects.  The GetPolyData() method can be used
 to get a polygonal representation (the single polygon clipped by the
 bounding box).  Typical usage of the widget is to make use of the
 StartInteractionEvent, InteractionEvent, and EndInteractionEvent
 events. The InteractionEvent is called on mouse motion; the other two
 events are called on button down and button up (either left or right
 button). (Note: there is also a PlaceWidgetEvent that is invoked when
 the widget is placed with PlaceWidget().)

 Some additional features of this class include the ability to control the
 properties of the widget. You do this by setting property values on the
 normal vector (selected and unselected properties); the plane (selected
 and unselected properties); the outline (selected and unselected
 properties); and the edges. The edges may also be tubed or not.

To create an instance of class vtkImplicitPlaneWidget, simply
invoke its constructor as follows
\begin{verbatim}
  obj = vtkImplicitPlaneWidget
\end{verbatim}
\subsection{Methods}

The class vtkImplicitPlaneWidget has several methods that can be used.
  They are listed below.
Note that the documentation is translated automatically from the VTK sources,
and may not be completely intelligible.  When in doubt, consult the VTK website.
In the methods listed below, \verb|obj| is an instance of the vtkImplicitPlaneWidget class.
\begin{itemize}
\item  \verb|string = obj.GetClassName ()|

\item  \verb|int = obj.IsA (string name)|

\item  \verb|vtkImplicitPlaneWidget = obj.NewInstance ()|

\item  \verb|vtkImplicitPlaneWidget = obj.SafeDownCast (vtkObject o)|

\item  \verb|obj.SetEnabled (int )| -  Methods that satisfy the superclass' API.

\item  \verb|obj.PlaceWidget (double bounds[6])| -  Methods that satisfy the superclass' API.

\item  \verb|obj.PlaceWidget ()| -  Methods that satisfy the superclass' API.

\item  \verb|obj.PlaceWidget (double xmin, double xmax, double ymin, double ymax, double zmin, double zmax)| -  Get the origin of the plane.

\item  \verb|obj.SetOrigin (double x, double y, double z)| -  Get the origin of the plane.

\item  \verb|obj.SetOrigin (double x[3])| -  Get the origin of the plane.

\item  \verb|double = obj.GetOrigin ()| -  Get the origin of the plane.

\item  \verb|obj.GetOrigin (double xyz[3])| -  Get the origin of the plane.

\item  \verb|obj.SetNormal (double x, double y, double z)| -  Get the normal to the plane.

\item  \verb|obj.SetNormal (double x[3])| -  Get the normal to the plane.

\item  \verb|double = obj.GetNormal ()| -  Get the normal to the plane.

\item  \verb|obj.GetNormal (double xyz[3])| -  Get the normal to the plane.

\item  \verb|obj.SetNormalToXAxis (int )| -  Force the plane widget to be aligned with one of the x-y-z axes.
 If one axis is set on, the other two will be set off.
 Remember that when the state changes, a ModifiedEvent is invoked.
 This can be used to snap the plane to the axes if it is orginally
 not aligned.

\item  \verb|int = obj.GetNormalToXAxis ()| -  Force the plane widget to be aligned with one of the x-y-z axes.
 If one axis is set on, the other two will be set off.
 Remember that when the state changes, a ModifiedEvent is invoked.
 This can be used to snap the plane to the axes if it is orginally
 not aligned.

\item  \verb|obj.NormalToXAxisOn ()| -  Force the plane widget to be aligned with one of the x-y-z axes.
 If one axis is set on, the other two will be set off.
 Remember that when the state changes, a ModifiedEvent is invoked.
 This can be used to snap the plane to the axes if it is orginally
 not aligned.

\item  \verb|obj.NormalToXAxisOff ()| -  Force the plane widget to be aligned with one of the x-y-z axes.
 If one axis is set on, the other two will be set off.
 Remember that when the state changes, a ModifiedEvent is invoked.
 This can be used to snap the plane to the axes if it is orginally
 not aligned.

\item  \verb|obj.SetNormalToYAxis (int )| -  Force the plane widget to be aligned with one of the x-y-z axes.
 If one axis is set on, the other two will be set off.
 Remember that when the state changes, a ModifiedEvent is invoked.
 This can be used to snap the plane to the axes if it is orginally
 not aligned.

\item  \verb|int = obj.GetNormalToYAxis ()| -  Force the plane widget to be aligned with one of the x-y-z axes.
 If one axis is set on, the other two will be set off.
 Remember that when the state changes, a ModifiedEvent is invoked.
 This can be used to snap the plane to the axes if it is orginally
 not aligned.

\item  \verb|obj.NormalToYAxisOn ()| -  Force the plane widget to be aligned with one of the x-y-z axes.
 If one axis is set on, the other two will be set off.
 Remember that when the state changes, a ModifiedEvent is invoked.
 This can be used to snap the plane to the axes if it is orginally
 not aligned.

\item  \verb|obj.NormalToYAxisOff ()| -  Force the plane widget to be aligned with one of the x-y-z axes.
 If one axis is set on, the other two will be set off.
 Remember that when the state changes, a ModifiedEvent is invoked.
 This can be used to snap the plane to the axes if it is orginally
 not aligned.

\item  \verb|obj.SetNormalToZAxis (int )| -  Force the plane widget to be aligned with one of the x-y-z axes.
 If one axis is set on, the other two will be set off.
 Remember that when the state changes, a ModifiedEvent is invoked.
 This can be used to snap the plane to the axes if it is orginally
 not aligned.

\item  \verb|int = obj.GetNormalToZAxis ()| -  Force the plane widget to be aligned with one of the x-y-z axes.
 If one axis is set on, the other two will be set off.
 Remember that when the state changes, a ModifiedEvent is invoked.
 This can be used to snap the plane to the axes if it is orginally
 not aligned.

\item  \verb|obj.NormalToZAxisOn ()| -  Force the plane widget to be aligned with one of the x-y-z axes.
 If one axis is set on, the other two will be set off.
 Remember that when the state changes, a ModifiedEvent is invoked.
 This can be used to snap the plane to the axes if it is orginally
 not aligned.

\item  \verb|obj.NormalToZAxisOff ()| -  Force the plane widget to be aligned with one of the x-y-z axes.
 If one axis is set on, the other two will be set off.
 Remember that when the state changes, a ModifiedEvent is invoked.
 This can be used to snap the plane to the axes if it is orginally
 not aligned.

\item  \verb|obj.SetTubing (int )| -  Turn on/off tubing of the wire outline of the plane. The tube thickens
 the line by wrapping with a vtkTubeFilter.

\item  \verb|int = obj.GetTubing ()| -  Turn on/off tubing of the wire outline of the plane. The tube thickens
 the line by wrapping with a vtkTubeFilter.

\item  \verb|obj.TubingOn ()| -  Turn on/off tubing of the wire outline of the plane. The tube thickens
 the line by wrapping with a vtkTubeFilter.

\item  \verb|obj.TubingOff ()| -  Turn on/off tubing of the wire outline of the plane. The tube thickens
 the line by wrapping with a vtkTubeFilter.

\item  \verb|obj.SetDrawPlane (int plane)| -  Enable/disable the drawing of the plane. In some cases the plane
 interferes with the object that it is operating on (i.e., the
 plane interferes with the cut surface it produces producing
 z-buffer artifacts.)

\item  \verb|int = obj.GetDrawPlane ()| -  Enable/disable the drawing of the plane. In some cases the plane
 interferes with the object that it is operating on (i.e., the
 plane interferes with the cut surface it produces producing
 z-buffer artifacts.)

\item  \verb|obj.DrawPlaneOn ()| -  Enable/disable the drawing of the plane. In some cases the plane
 interferes with the object that it is operating on (i.e., the
 plane interferes with the cut surface it produces producing
 z-buffer artifacts.)

\item  \verb|obj.DrawPlaneOff ()| -  Enable/disable the drawing of the plane. In some cases the plane
 interferes with the object that it is operating on (i.e., the
 plane interferes with the cut surface it produces producing
 z-buffer artifacts.)

\item  \verb|obj.SetOutlineTranslation (int )| -  Turn on/off the ability to translate the bounding box by grabbing it
 with the left mouse button.

\item  \verb|int = obj.GetOutlineTranslation ()| -  Turn on/off the ability to translate the bounding box by grabbing it
 with the left mouse button.

\item  \verb|obj.OutlineTranslationOn ()| -  Turn on/off the ability to translate the bounding box by grabbing it
 with the left mouse button.

\item  \verb|obj.OutlineTranslationOff ()| -  Turn on/off the ability to translate the bounding box by grabbing it
 with the left mouse button.

\item  \verb|obj.SetOutsideBounds (int )| -  Turn on/off the ability to move the widget outside of the input's bound

\item  \verb|int = obj.GetOutsideBounds ()| -  Turn on/off the ability to move the widget outside of the input's bound

\item  \verb|obj.OutsideBoundsOn ()| -  Turn on/off the ability to move the widget outside of the input's bound

\item  \verb|obj.OutsideBoundsOff ()| -  Turn on/off the ability to move the widget outside of the input's bound

\item  \verb|obj.SetScaleEnabled (int )| -  Turn on/off the ability to scale with the mouse

\item  \verb|int = obj.GetScaleEnabled ()| -  Turn on/off the ability to scale with the mouse

\item  \verb|obj.ScaleEnabledOn ()| -  Turn on/off the ability to scale with the mouse

\item  \verb|obj.ScaleEnabledOff ()| -  Turn on/off the ability to scale with the mouse

\item  \verb|obj.SetOriginTranslation (int )| -  Turn on/off the ability to translate the origin (sphere)
 with the left mouse button.

\item  \verb|int = obj.GetOriginTranslation ()| -  Turn on/off the ability to translate the origin (sphere)
 with the left mouse button.

\item  \verb|obj.OriginTranslationOn ()| -  Turn on/off the ability to translate the origin (sphere)
 with the left mouse button.

\item  \verb|obj.OriginTranslationOff ()| -  Turn on/off the ability to translate the origin (sphere)
 with the left mouse button.

\item  \verb|obj.SetDiagonalRatio (double )| -  By default the arrow is 30% of the diagonal length. DiagonalRatio control
 this ratio in the interval [0-2]

\item  \verb|double = obj.GetDiagonalRatioMinValue ()| -  By default the arrow is 30% of the diagonal length. DiagonalRatio control
 this ratio in the interval [0-2]

\item  \verb|double = obj.GetDiagonalRatioMaxValue ()| -  By default the arrow is 30% of the diagonal length. DiagonalRatio control
 this ratio in the interval [0-2]

\item  \verb|double = obj.GetDiagonalRatio ()| -  By default the arrow is 30% of the diagonal length. DiagonalRatio control
 this ratio in the interval [0-2]

\item  \verb|obj.GetPolyData (vtkPolyData pd)| -  Grab the polydata that defines the plane. The polydata contains a single
 polygon that is clipped by the bounding box.

\item  \verb|vtkPolyDataAlgorithm = obj.GetPolyDataAlgorithm ()| -  Satisfies superclass API.  This returns a pointer to the underlying
 PolyData (which represents the plane).

\item  \verb|obj.GetPlane (vtkPlane plane)| -  Get the implicit function for the plane. The user must provide the
 instance of the class vtkPlane. Note that vtkPlane is a subclass of
 vtkImplicitFunction, meaning that it can be used by a variety of filters
 to perform clipping, cutting, and selection of data.

\item  \verb|obj.UpdatePlacement ()| -  Satisfies the superclass API.  This will change the state of the widget
 to match changes that have been made to the underlying PolyDataSource

\item  \verb|obj.SizeHandles ()| -  Control widget appearance

\item  \verb|vtkProperty = obj.GetNormalProperty ()| -  Get the properties on the normal (line and cone).

\item  \verb|vtkProperty = obj.GetSelectedNormalProperty ()| -  Get the properties on the normal (line and cone).

\item  \verb|vtkProperty = obj.GetPlaneProperty ()| -  Get the plane properties. The properties of the plane when selected
 and unselected can be manipulated.

\item  \verb|vtkProperty = obj.GetSelectedPlaneProperty ()| -  Get the plane properties. The properties of the plane when selected
 and unselected can be manipulated.

\item  \verb|vtkProperty = obj.GetOutlineProperty ()| -  Get the property of the outline.

\item  \verb|vtkProperty = obj.GetSelectedOutlineProperty ()| -  Get the property of the outline.

\item  \verb|vtkProperty = obj.GetEdgesProperty ()| -  Get the property of the intersection edges. (This property also
 applies to the edges when tubed.)

\end{itemize}
