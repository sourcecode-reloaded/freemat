\section{vtkBoxClipDataSet}

\subsection{Usage}

 Clipping means that is actually 'cuts' through the cells of the dataset,
 returning tetrahedral cells inside of the box.
 The output of this filter is an unstructured grid.

 This filter can be configured to compute a second output. The
 second output is the part of the cell that is clipped away. Set the
 GenerateClippedData boolean on if you wish to access this output data.

 The vtkBoxClipDataSet will triangulate all types of 3D cells (i.e, create tetrahedra).
 This is necessary to preserve compatibility across face neighbors.

 To use this filter,you can decide if you will be clipping with a box or a hexahedral box.
 1) Set orientation 
    if(SetOrientation(0)): box (parallel with coordinate axis)
       SetBoxClip(xmin,xmax,ymin,ymax,zmin,zmax)  
    if(SetOrientation(1)): hexahedral box (Default)
       SetBoxClip(n[0],o[0],n[1],o[1],n[2],o[2],n[3],o[3],n[4],o[4],n[5],o[5])  
       PlaneNormal[] normal of each plane
       PlanePoint[] point on the plane 
 2) Apply the GenerateClipScalarsOn() 
 3) Execute clipping  Update();

To create an instance of class vtkBoxClipDataSet, simply
invoke its constructor as follows
\begin{verbatim}
  obj = vtkBoxClipDataSet
\end{verbatim}
\subsection{Methods}

The class vtkBoxClipDataSet has several methods that can be used.
  They are listed below.
Note that the documentation is translated automatically from the VTK sources,
and may not be completely intelligible.  When in doubt, consult the VTK website.
In the methods listed below, \verb|obj| is an instance of the vtkBoxClipDataSet class.
\begin{itemize}
\item  \verb|string = obj.GetClassName ()|

\item  \verb|int = obj.IsA (string name)|

\item  \verb|vtkBoxClipDataSet = obj.NewInstance ()|

\item  \verb|vtkBoxClipDataSet = obj.SafeDownCast (vtkObject o)|

\item  \verb|obj.SetBoxClip (double xmin, double xmax, double ymin, double ymax, double zmin, double zmax)|

\item  \verb|obj.SetBoxClip (double n0, double o0, double n1, double o1, double n2, double o2, double n3, double o3, double n4, double o4, double n5, double o5)|

\item  \verb|obj.SetGenerateClipScalars (int )| -  If this flag is enabled, then the output scalar values will be 
 interpolated, and not the input scalar data.

\item  \verb|int = obj.GetGenerateClipScalars ()| -  If this flag is enabled, then the output scalar values will be 
 interpolated, and not the input scalar data.

\item  \verb|obj.GenerateClipScalarsOn ()| -  If this flag is enabled, then the output scalar values will be 
 interpolated, and not the input scalar data.

\item  \verb|obj.GenerateClipScalarsOff ()| -  If this flag is enabled, then the output scalar values will be 
 interpolated, and not the input scalar data.

\item  \verb|obj.SetGenerateClippedOutput (int )| -  Control whether a second output is generated. The second output
 contains the polygonal data that's been clipped away.

\item  \verb|int = obj.GetGenerateClippedOutput ()| -  Control whether a second output is generated. The second output
 contains the polygonal data that's been clipped away.

\item  \verb|obj.GenerateClippedOutputOn ()| -  Control whether a second output is generated. The second output
 contains the polygonal data that's been clipped away.

\item  \verb|obj.GenerateClippedOutputOff ()| -  Control whether a second output is generated. The second output
 contains the polygonal data that's been clipped away.

\item  \verb|vtkUnstructuredGrid = obj.GetClippedOutput ()| -  Return the Clipped output.

\item  \verb|int = obj.GetNumberOfOutputs ()| -  Return the Clipped output.

\item  \verb|obj.SetLocator (vtkIncrementalPointLocator locator)| -  Specify a spatial locator for merging points. By default, an
 instance of vtkMergePoints is used.

\item  \verb|vtkIncrementalPointLocator = obj.GetLocator ()| -  Specify a spatial locator for merging points. By default, an
 instance of vtkMergePoints is used.

\item  \verb|obj.CreateDefaultLocator ()| -  Create default locator. Used to create one when none is specified. The 
 locator is used to merge coincident points.

\item  \verb|long = obj.GetMTime ()| -  Return the mtime also considering the locator.

\item  \verb|int = obj.GetOrientation ()| -  Tells if clipping happens with a box parallel with coordinate axis
 (0) or with an hexahedral box (1). Initial value is 1.

\item  \verb|obj.SetOrientation (int )| -  Tells if clipping happens with a box parallel with coordinate axis
 (0) or with an hexahedral box (1). Initial value is 1.

\item  \verb|obj.ClipBox (vtkPoints newPoints, vtkGenericCell cell, vtkIncrementalPointLocator locator, vtkCellArray tets, vtkPointData inPD, vtkPointData outPD, vtkCellData inCD, vtkIdType cellId, vtkCellData outCD)|

\item  \verb|obj.ClipHexahedron (vtkPoints newPoints, vtkGenericCell cell, vtkIncrementalPointLocator locator, vtkCellArray tets, vtkPointData inPD, vtkPointData outPD, vtkCellData inCD, vtkIdType cellId, vtkCellData outCD)|

\item  \verb|obj.ClipBox2D (vtkPoints newPoints, vtkGenericCell cell, vtkIncrementalPointLocator locator, vtkCellArray tets, vtkPointData inPD, vtkPointData outPD, vtkCellData inCD, vtkIdType cellId, vtkCellData outCD)|

\item  \verb|obj.ClipHexahedron2D (vtkPoints newPoints, vtkGenericCell cell, vtkIncrementalPointLocator locator, vtkCellArray tets, vtkPointData inPD, vtkPointData outPD, vtkCellData inCD, vtkIdType cellId, vtkCellData outCD)|

\item  \verb|obj.ClipBox1D (vtkPoints newPoints, vtkGenericCell cell, vtkIncrementalPointLocator locator, vtkCellArray lines, vtkPointData inPD, vtkPointData outPD, vtkCellData inCD, vtkIdType cellId, vtkCellData outCD)|

\item  \verb|obj.ClipHexahedron1D (vtkPoints newPoints, vtkGenericCell cell, vtkIncrementalPointLocator locator, vtkCellArray lines, vtkPointData inPD, vtkPointData outPD, vtkCellData inCD, vtkIdType cellId, vtkCellData outCD)|

\item  \verb|obj.ClipBox0D (vtkGenericCell cell, vtkIncrementalPointLocator locator, vtkCellArray verts, vtkPointData inPD, vtkPointData outPD, vtkCellData inCD, vtkIdType cellId, vtkCellData outCD)|

\item  \verb|obj.ClipHexahedron0D (vtkGenericCell cell, vtkIncrementalPointLocator locator, vtkCellArray verts, vtkPointData inPD, vtkPointData outPD, vtkCellData inCD, vtkIdType cellId, vtkCellData outCD)|

\end{itemize}
