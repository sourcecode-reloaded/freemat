\section{vtkOBBTree}

\subsection{Usage}

 vtkOBBTree is an object to generate oriented bounding box (OBB) trees.
 An oriented bounding box is a bounding box that does not necessarily line 
 up along coordinate axes. The OBB tree is a hierarchical tree structure 
 of such boxes, where deeper levels of OBB confine smaller regions of space.

 To build the OBB, a recursive, top-down process is used. First, the root OBB
 is constructed by finding the mean and covariance matrix of the cells (and
 their points) that define the dataset. The eigenvectors of the covariance
 matrix are extracted, giving a set of three orthogonal vectors that define 
 the tightest-fitting OBB. To create the two children OBB's, a split plane 
 is found that (approximately) divides the number cells in half. These are 
 then assigned to the children OBB's. This process then continues until
 the MaxLevel ivar limits the recursion, or no split plane can be found.

 A good reference for OBB-trees is Gottschalk \& Manocha in Proceedings of 
 Siggraph `96.

To create an instance of class vtkOBBTree, simply
invoke its constructor as follows
\begin{verbatim}
  obj = vtkOBBTree
\end{verbatim}
\subsection{Methods}

The class vtkOBBTree has several methods that can be used.
  They are listed below.
Note that the documentation is translated automatically from the VTK sources,
and may not be completely intelligible.  When in doubt, consult the VTK website.
In the methods listed below, \verb|obj| is an instance of the vtkOBBTree class.
\begin{itemize}
\item  \verb|string = obj.GetClassName ()|

\item  \verb|int = obj.IsA (string name)|

\item  \verb|vtkOBBTree = obj.NewInstance ()|

\item  \verb|vtkOBBTree = obj.SafeDownCast (vtkObject o)|

\item  \verb|int = obj.IntersectWithLine (double a0[3], double a1[3], vtkPoints points, vtkIdList cellIds)| -  Take the passed line segment and intersect it with the data set.
 This method assumes that the data set is a vtkPolyData that describes
 a closed surface, and the intersection points that are returned in
 'points' alternate between entrance points and exit points.
 The return value of the function is 0 if no intersections were found,
 -1 if point 'a0' lies inside the closed surface, or +1 if point 'a0'
 lies outside the closed surface.
 Either 'points' or 'cellIds' can be set to NULL if you don't want
 to receive that information.

\item  \verb|obj.ComputeOBB (vtkDataSet input, double corner[3], double max[3], double mid[3], double min[3], double size[3])| -  Compute an OBB for the input dataset using the cells in the data.
 Return the corner point and the three axes defining the orientation
 of the OBB. Also return a sorted list of relative ''sizes'' of axes for
 comparison purposes.

\item  \verb|int = obj.InsideOrOutside (double point[3])| -  Determine whether a point is inside or outside the data used to build
 this OBB tree.  The data must be a closed surface vtkPolyData data set.
 The return value is +1 if outside, -1 if inside, and 0 if undecided.

\item  \verb|obj.FreeSearchStructure ()| -  Satisfy locator's abstract interface, see vtkLocator.

\item  \verb|obj.BuildLocator ()| -  Satisfy locator's abstract interface, see vtkLocator.

\item  \verb|obj.GenerateRepresentation (int level, vtkPolyData pd)| -  Create polygonal representation for OBB tree at specified level. If
 level < 0, then the leaf OBB nodes will be gathered. The aspect ratio (ar)
 and line diameter (d) are used to control the building of the
 representation. If a OBB node edge ratio's are greater than ar, then the
 dimension of the OBB is collapsed (OBB->plane->line). A ''line'' OBB will be
 represented either as two crossed polygons, or as a line, depending on
 the relative diameter of the OBB compared to the diameter (d).

\end{itemize}
