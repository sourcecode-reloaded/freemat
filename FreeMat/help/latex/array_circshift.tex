\section{CIRCSHIFT Circularly Shift an Array}

\subsection{USAGE}

Applies a circular shift along each dimension of a given array.  The
syntax for its use is
\begin{verbatim}
   y = circshift(x,shiftvec)
\end{verbatim}
where \verb|x| is an n-dimensional array, and \verb|shiftvec| is a vector of
integers, each of which specify how much to shift \verb|x| along the
corresponding dimension.  
\subsection{Example}

The following examples show some uses of \verb|circshift| on N-dimensional
arrays.
\begin{verbatim}
--> x = int32(rand(4,5)*10)

x = 
  4  8  3  2  9 
  0  8  0  5  3 
  9  1  5  8  2 
  4  5 10  3  7 

--> circshift(x,[1,0])

ans = 
  4  5 10  3  7 
  4  8  3  2  9 
  0  8  0  5  3 
  9  1  5  8  2 

--> circshift(x,[0,-1])

ans = 
  8  3  2  9  4 
  8  0  5  3  0 
  1  5  8  2  9 
  5 10  3  7  4 

--> circshift(x,[2,2])

ans = 
  8  2  9  1  5 
  3  7  4  5 10 
  2  9  4  8  3 
  5  3  0  8  0 

--> x = int32(rand(4,5,3)*10)

x = 

(:,:,1) = 
  2  7  7  3 10 
  2  2  3  7  0 
  4  8  1  4  0 
 10  2  7  8  9 

(:,:,2) = 
  5  7 10  9  4 
  0  3  5  0  4 
  4  5  1  3  6 
  9  1  5  1  5 

(:,:,3) = 
  1  5  6  9  2 
  8 10  6  5  7 
  6  2  1  6  8 
  1  9  6  5  3 

--> circshift(x,[1,0,0])

ans = 

(:,:,1) = 
 10  2  7  8  9 
  2  7  7  3 10 
  2  2  3  7  0 
  4  8  1  4  0 

(:,:,2) = 
  9  1  5  1  5 
  5  7 10  9  4 
  0  3  5  0  4 
  4  5  1  3  6 

(:,:,3) = 
  1  9  6  5  3 
  1  5  6  9  2 
  8 10  6  5  7 
  6  2  1  6  8 

--> circshift(x,[0,-1,0])

ans = 

(:,:,1) = 
  7  7  3 10  2 
  2  3  7  0  2 
  8  1  4  0  4 
  2  7  8  9 10 

(:,:,2) = 
  7 10  9  4  5 
  3  5  0  4  0 
  5  1  3  6  4 
  1  5  1  5  9 

(:,:,3) = 
  5  6  9  2  1 
 10  6  5  7  8 
  2  1  6  8  6 
  9  6  5  3  1 

--> circshift(x,[0,0,-1])

ans = 

(:,:,1) = 
  5  7 10  9  4 
  0  3  5  0  4 
  4  5  1  3  6 
  9  1  5  1  5 

(:,:,2) = 
  1  5  6  9  2 
  8 10  6  5  7 
  6  2  1  6  8 
  1  9  6  5  3 

(:,:,3) = 
  2  7  7  3 10 
  2  2  3  7  0 
  4  8  1  4  0 
 10  2  7  8  9 

--> circshift(x,[2,-3,1])

ans = 

(:,:,1) = 
  6  8  6  2  1 
  5  3  1  9  6 
  9  2  1  5  6 
  5  7  8 10  6 

(:,:,2) = 
  4  0  4  8  1 
  8  9 10  2  7 
  3 10  2  7  7 
  7  0  2  2  3 

(:,:,3) = 
  3  6  4  5  1 
  1  5  9  1  5 
  9  4  5  7 10 
  0  4  0  3  5 
\end{verbatim}
