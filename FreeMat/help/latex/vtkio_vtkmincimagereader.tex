\section{vtkMINCImageReader}

\subsection{Usage}

 MINC is a NetCDF-based medical image file format that was developed
 at the Montreal Neurological Institute in 1992. 
 This class will read a MINC file into VTK, rearranging the data to
 match the VTK x, y, and z dimensions, and optionally rescaling
 real-valued data to VTK\_FLOAT if RescaleRealValuesOn() is set.
 If RescaleRealValues is off, then the data will be stored in its
 original data type and the GetRescaleSlope(), GetRescaleIntercept()
 method can be used to retrieve global rescaling parameters.
 If the original file had a time dimension, the SetTimeStep() method
 can be used to specify a time step to read.
 All of the original header information can be accessed though the
 GetImageAttributes() method.

To create an instance of class vtkMINCImageReader, simply
invoke its constructor as follows
\begin{verbatim}
  obj = vtkMINCImageReader
\end{verbatim}
\subsection{Methods}

The class vtkMINCImageReader has several methods that can be used.
  They are listed below.
Note that the documentation is translated automatically from the VTK sources,
and may not be completely intelligible.  When in doubt, consult the VTK website.
In the methods listed below, \verb|obj| is an instance of the vtkMINCImageReader class.
\begin{itemize}
\item  \verb|string = obj.GetClassName ()|

\item  \verb|int = obj.IsA (string name)|

\item  \verb|vtkMINCImageReader = obj.NewInstance ()|

\item  \verb|vtkMINCImageReader = obj.SafeDownCast (vtkObject o)|

\item  \verb|obj.SetFileName (string name)| -  Set the file name.

\item  \verb|string = obj.GetFileExtensions ()| -  Get the name of this file format.

\item  \verb|string = obj.GetDescriptiveName ()| -  Test whether the specified file can be read.

\item  \verb|int = obj.CanReadFile (string name)| -  Test whether the specified file can be read.

\item  \verb|vtkMatrix4x4 = obj.GetDirectionCosines ()| -  Get a matrix that describes the orientation of the data.
 The three columns of the matrix are the direction cosines
 for the x, y and z dimensions respectively.

\item  \verb|double = obj.GetRescaleSlope ()| -  Get the slope and intercept for rescaling the scalar values
 to real data values.  To convert scalar values to real values,
 use the equation y = x*RescaleSlope + RescaleIntercept.

\item  \verb|double = obj.GetRescaleIntercept ()| -  Get the slope and intercept for rescaling the scalar values
 to real data values.  To convert scalar values to real values,
 use the equation y = x*RescaleSlope + RescaleIntercept.

\item  \verb|obj.SetRescaleRealValues (int )| -  Rescale real data values to float.  If this is done, the
 RescaleSlope and RescaleIntercept will be set to 1 and 0
 respectively.  This is off by default.

\item  \verb|obj.RescaleRealValuesOn ()| -  Rescale real data values to float.  If this is done, the
 RescaleSlope and RescaleIntercept will be set to 1 and 0
 respectively.  This is off by default.

\item  \verb|obj.RescaleRealValuesOff ()| -  Rescale real data values to float.  If this is done, the
 RescaleSlope and RescaleIntercept will be set to 1 and 0
 respectively.  This is off by default.

\item  \verb|int = obj.GetRescaleRealValues ()| -  Rescale real data values to float.  If this is done, the
 RescaleSlope and RescaleIntercept will be set to 1 and 0
 respectively.  This is off by default.

\item  \verb|double = obj.GetDataRange ()| -  Get the scalar range of the output from the information in
 the file header.  This is more efficient that computing the
 scalar range, but in some cases the MINC file stores an
 incorrect valid\_range and the DataRange will be incorrect.

\item  \verb|obj.GetDataRange (double range[2])| -  Get the scalar range of the output from the information in
 the file header.  This is more efficient that computing the
 scalar range, but in some cases the MINC file stores an
 incorrect valid\_range and the DataRange will be incorrect.

\item  \verb|int = obj.GetNumberOfTimeSteps ()| -  Get the number of time steps in the file.

\item  \verb|obj.SetTimeStep (int )| -  Set the time step to read.

\item  \verb|int = obj.GetTimeStep ()| -  Set the time step to read.

\item  \verb|vtkMINCImageAttributes = obj.GetImageAttributes ()| -  Get the image attributes, which contain patient information and
 other useful metadata.

\end{itemize}
