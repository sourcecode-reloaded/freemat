\section{vtkImageFFT}

\subsection{Usage}

 vtkImageFFT implements a  fast Fourier transform.  The input
 can have real or complex data in any components and data types, but
 the output is always complex doubles with real values in component0, and
 imaginary values in component1.  The filter is fastest for images that
 have power of two sizes.  The filter uses a butterfly fitlers for each
 prime factor of the dimension.  This makes images with prime number dimensions 
 (i.e. 17x17) much slower to compute.  Multi dimensional (i.e volumes) 
 FFT's are decomposed so that each axis executes in series.

To create an instance of class vtkImageFFT, simply
invoke its constructor as follows
\begin{verbatim}
  obj = vtkImageFFT
\end{verbatim}
\subsection{Methods}

The class vtkImageFFT has several methods that can be used.
  They are listed below.
Note that the documentation is translated automatically from the VTK sources,
and may not be completely intelligible.  When in doubt, consult the VTK website.
In the methods listed below, \verb|obj| is an instance of the vtkImageFFT class.
\begin{itemize}
\item  \verb|string = obj.GetClassName ()|

\item  \verb|int = obj.IsA (string name)|

\item  \verb|vtkImageFFT = obj.NewInstance ()|

\item  \verb|vtkImageFFT = obj.SafeDownCast (vtkObject o)|

\item  \verb|int = obj.SplitExtent (int splitExt[6], int startExt[6], int num, int total)| -  Used internally for streaming and threads.  
 Splits output update extent into num pieces.
 This method needs to be called num times.  Results must not overlap for
 consistent starting extent.  Subclass can override this method.
 This method returns the number of pieces resulting from a
 successful split.  This can be from 1 to ''total''.  
 If 1 is returned, the extent cannot be split.

\end{itemize}
