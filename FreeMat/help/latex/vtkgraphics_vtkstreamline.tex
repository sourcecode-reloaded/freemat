\section{vtkStreamLine}

\subsection{Usage}

 vtkStreamLine is a filter that generates a streamline for an arbitrary 
 dataset. A streamline is a line that is everywhere tangent to the vector
 field. Scalar values also are calculated along the streamline and can be 
 used to color the line. Streamlines are calculated by integrating from
 a starting point through the vector field. Integration can be performed
 forward in time (see where the line goes), backward in time (see where the
 line came from), or in both directions. It also is possible to compute
 vorticity along the streamline. Vorticity is the projection (i.e., dot
 product) of the flow rotation on the velocity vector, i.e., the rotation
 of flow around the streamline.

 vtkStreamLine defines the instance variable StepLength. This parameter 
 controls the time increment used to generate individual points along
 the streamline(s). Smaller values result in more line 
 primitives but smoother streamlines. The StepLength instance variable is 
 defined in terms of time (i.e., the distance that the particle travels in
 the specified time period). Thus, the line segments will be smaller in areas
 of low velocity and larger in regions of high velocity. (NOTE: This is
 different than the IntegrationStepLength defined by the superclass
 vtkStreamer. IntegrationStepLength is used to control integration step 
 size and is expressed as a fraction of the cell length.) The StepLength
 instance variable is important because subclasses of vtkStreamLine (e.g.,
 vtkDashedStreamLine) depend on this value to build their representation.

To create an instance of class vtkStreamLine, simply
invoke its constructor as follows
\begin{verbatim}
  obj = vtkStreamLine
\end{verbatim}
\subsection{Methods}

The class vtkStreamLine has several methods that can be used.
  They are listed below.
Note that the documentation is translated automatically from the VTK sources,
and may not be completely intelligible.  When in doubt, consult the VTK website.
In the methods listed below, \verb|obj| is an instance of the vtkStreamLine class.
\begin{itemize}
\item  \verb|string = obj.GetClassName ()|

\item  \verb|int = obj.IsA (string name)|

\item  \verb|vtkStreamLine = obj.NewInstance ()|

\item  \verb|vtkStreamLine = obj.SafeDownCast (vtkObject o)|

\item  \verb|obj.SetStepLength (double )| -  Specify the length of a line segment. The length is expressed in terms of
 elapsed time. Smaller values result in smoother appearing streamlines, but
 greater numbers of line primitives.

\item  \verb|double = obj.GetStepLengthMinValue ()| -  Specify the length of a line segment. The length is expressed in terms of
 elapsed time. Smaller values result in smoother appearing streamlines, but
 greater numbers of line primitives.

\item  \verb|double = obj.GetStepLengthMaxValue ()| -  Specify the length of a line segment. The length is expressed in terms of
 elapsed time. Smaller values result in smoother appearing streamlines, but
 greater numbers of line primitives.

\item  \verb|double = obj.GetStepLength ()| -  Specify the length of a line segment. The length is expressed in terms of
 elapsed time. Smaller values result in smoother appearing streamlines, but
 greater numbers of line primitives.

\end{itemize}
