\section{vtkPointSetAlgorithm}

\subsection{Usage}

 vtkPointSetAlgorithm is a convenience class to make writing algorithms
 easier. It is also designed to help transition old algorithms to the new
 pipeline architecture. Ther are some assumptions and defaults made by this
 class you should be aware of. This class defaults such that your filter
 will have one input port and one output port. If that is not the case
 simply change it with SetNumberOfInputPorts etc. See this classes
 contstructor for the default. This class also provides a FillInputPortInfo
 method that by default says that all inputs will be PointSet. If that
 isn't the case then please override this method in your subclass. This
 class breaks out the downstream requests into seperate functions such as
 RequestDataObject RequestData and ExecuteInformation. The default 
 implementation of RequestDataObject will create an output data of the 
 same type as the input. 

To create an instance of class vtkPointSetAlgorithm, simply
invoke its constructor as follows
\begin{verbatim}
  obj = vtkPointSetAlgorithm
\end{verbatim}
\subsection{Methods}

The class vtkPointSetAlgorithm has several methods that can be used.
  They are listed below.
Note that the documentation is translated automatically from the VTK sources,
and may not be completely intelligible.  When in doubt, consult the VTK website.
In the methods listed below, \verb|obj| is an instance of the vtkPointSetAlgorithm class.
\begin{itemize}
\item  \verb|string = obj.GetClassName ()|

\item  \verb|int = obj.IsA (string name)|

\item  \verb|vtkPointSetAlgorithm = obj.NewInstance ()|

\item  \verb|vtkPointSetAlgorithm = obj.SafeDownCast (vtkObject o)|

\item  \verb|vtkPointSet = obj.GetOutput ()| -  Get the output data object for a port on this algorithm.

\item  \verb|vtkPointSet = obj.GetOutput (int )| -  Get the output data object for a port on this algorithm.

\item  \verb|vtkPolyData = obj.GetPolyDataOutput ()| -  Get the output as vtkPolyData.

\item  \verb|vtkStructuredGrid = obj.GetStructuredGridOutput ()| -  Get the output as vtkStructuredGrid.

\item  \verb|vtkUnstructuredGrid = obj.GetUnstructuredGridOutput ()| -  Get the output as vtkUnstructuredGrid.

\item  \verb|obj.SetInput (vtkDataObject )| -  Set an input of this algorithm. You should not override these
 methods because they are not the only way to connect a pipeline.
 Note that these methods support old-style pipeline connections.
 When writing new code you should use the more general
 vtkAlgorithm::SetInputConnection().  These methods transform the
 input index to the input port index, not an index of a connection
 within a single port.

\item  \verb|obj.SetInput (int , vtkDataObject )| -  Set an input of this algorithm. You should not override these
 methods because they are not the only way to connect a pipeline.
 Note that these methods support old-style pipeline connections.
 When writing new code you should use the more general
 vtkAlgorithm::SetInputConnection().  These methods transform the
 input index to the input port index, not an index of a connection
 within a single port.

\item  \verb|obj.SetInput (vtkPointSet )| -  Set an input of this algorithm. You should not override these
 methods because they are not the only way to connect a pipeline.
 Note that these methods support old-style pipeline connections.
 When writing new code you should use the more general
 vtkAlgorithm::SetInputConnection().  These methods transform the
 input index to the input port index, not an index of a connection
 within a single port.

\item  \verb|obj.SetInput (int , vtkPointSet )| -  Set an input of this algorithm. You should not override these
 methods because they are not the only way to connect a pipeline.
 Note that these methods support old-style pipeline connections.
 When writing new code you should use the more general
 vtkAlgorithm::SetInputConnection().  These methods transform the
 input index to the input port index, not an index of a connection
 within a single port.

\item  \verb|obj.AddInput (vtkDataObject )| -  Add an input of this algorithm.  Note that these methods support
 old-style pipeline connections.  When writing new code you should
 use the more general vtkAlgorithm::AddInputConnection().  See
 SetInput() for details.

\item  \verb|obj.AddInput (vtkPointSet )| -  Add an input of this algorithm.  Note that these methods support
 old-style pipeline connections.  When writing new code you should
 use the more general vtkAlgorithm::AddInputConnection().  See
 SetInput() for details.

\item  \verb|obj.AddInput (int , vtkPointSet )| -  Add an input of this algorithm.  Note that these methods support
 old-style pipeline connections.  When writing new code you should
 use the more general vtkAlgorithm::AddInputConnection().  See
 SetInput() for details.

\item  \verb|obj.AddInput (int , vtkDataObject )| -  Add an input of this algorithm.  Note that these methods support
 old-style pipeline connections.  When writing new code you should
 use the more general vtkAlgorithm::AddInputConnection().  See
 SetInput() for details.

\item  \verb|vtkDataObject = obj.GetInput ()|

\end{itemize}
