\section{vtkTerrainContourLineInterpolator}

\subsection{Usage}

 vtkTerrainContourLineInterpolator interpolates nodes on height field data.
 The class is meant to be used in conjunciton with a vtkContourWidget, 
 enabling you to draw paths on terrain data. The class internally uses a 
 vtkProjectedTerrainPath. Users can set kind of interpolation
 desired between two node points by setting the modes of the this filter.
 For instance:

 \begin{verbatim}
 contourRepresentation->SetLineInterpolator(interpolator);
 interpolator->SetImageData( demDataFile );
 interpolator->GetProjector()->SetProjectionModeToHug();
 interpolator->SetHeightOffset(25.0);
 \end{verbatim}

 You are required to set the ImageData to this class as the height-field 
 image.


To create an instance of class vtkTerrainContourLineInterpolator, simply
invoke its constructor as follows
\begin{verbatim}
  obj = vtkTerrainContourLineInterpolator
\end{verbatim}
\subsection{Methods}

The class vtkTerrainContourLineInterpolator has several methods that can be used.
  They are listed below.
Note that the documentation is translated automatically from the VTK sources,
and may not be completely intelligible.  When in doubt, consult the VTK website.
In the methods listed below, \verb|obj| is an instance of the vtkTerrainContourLineInterpolator class.
\begin{itemize}
\item  \verb|string = obj.GetClassName ()| -  Standard methods for instances of this class.

\item  \verb|int = obj.IsA (string name)| -  Standard methods for instances of this class.

\item  \verb|vtkTerrainContourLineInterpolator = obj.NewInstance ()| -  Standard methods for instances of this class.

\item  \verb|vtkTerrainContourLineInterpolator = obj.SafeDownCast (vtkObject o)| -  Standard methods for instances of this class.

\item  \verb|int = obj.InterpolateLine (vtkRenderer ren, vtkContourRepresentation rep, int idx1, int idx2)| -  Interpolate to create lines between contour nodes idx1 and idx2.
 Depending on the projection mode, the interpolated line may either 
 hug the terrain, just connect the two points with a straight line or 
 a non-occluded interpolation. 
 Used internally by vtkContourRepresentation.

\item  \verb|int = obj.UpdateNode (vtkRenderer , vtkContourRepresentation , double , int )| -  The interpolator is given a chance to update the node.
 Used internally by vtkContourRepresentation
 Returns 0 if the node (world position) is unchanged.

\item  \verb|obj.SetImageData (vtkImageData )| -  Set the height field data. The height field data is a 2D image. The
 scalars in the image represent the height field. This must be set.

\item  \verb|vtkImageData = obj.GetImageData ()| -  Set the height field data. The height field data is a 2D image. The
 scalars in the image represent the height field. This must be set.

\item  \verb|vtkProjectedTerrainPath = obj.GetProjector ()| -  Get the vtkProjectedTerrainPath operator used to project the terrain
 onto the data. This operator has several modes, See the documentation
 of vtkProjectedTerrainPath. The default mode is to hug the terrain
 data at 0 height offset.

\end{itemize}
