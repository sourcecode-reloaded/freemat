\section{vtkBiDimensionalWidget}

\subsection{Usage}

 The vtkBiDimensionalWidget is used to measure the bi-dimensional length of
 an object. The bi-dimensional measure is defined by two finite,
 orthogonal lines that intersect within the finite extent of both lines.
 The lengths of these two lines gives the bi-dimensional measure. Each line
 is defined by two handle widgets at the end points of each line.

 The orthognal constraint on the two lines limits how the four end points
 can be positioned. The first two points can be placed arbitrarily to
 define the first line (similar to vtkDistanceWidget). The placement of the
 third point is limited by the finite extent of the first line. As the
 third point is placed, the fourth point is placed on the opposite side of
 the first line. Once the third point is placed, the second line is defined
 since the fourth point is defined at the same time, but the fourth point
 can be moved along the second line (i.e., maintaining the orthogonal
 relationship between the two lines). Onced defined, any of the four points
 can be moved along their constraint lines. Also, each line can be translated
 along the other line (in an orthogonal direction), and the whole
 bi-dimensional widget can be rotated about its center point (see the description
 of the event bindings). Finally, selecting the point where the two orthogonal
 axes intersect, the entire widget can be translated in any direction.

 Placement of any point results in a special PlacePointEvent invocation so
 that special operations may be performed to reposition the point. Motion
 of any point, moving the lines, or rotating the widget cause
 InteractionEvents to be invoked. Note that the widget has two fundamental
 modes: a define mode (when initially placing the points) and a manipulate
 mode (after the points are placed). Line translation and rotation are only
 possible in manipulate mode.
 
 To use this widget, specify an instance of vtkBiDimensionalWidget and a
 representation (e.g., vtkBiDimensionalRepresentation2D). The widget is
 implemented using four instances of vtkHandleWidget which are used to
 position the end points of the two intersecting lines. The representations
 for these handle widgets are provided by the
 vtkBiDimensionalRepresentation2D class.

 .SECTION Event Bindings
 By default, the widget responds to the following VTK events (i.e., it
 watches the vtkRenderWindowInteractor for these events):
 \begin{verbatim}
   LeftButtonPressEvent - define a point or manipulate a handle, line,
                          perform rotation or translate the widget.
   MouseMoveEvent - position the points, move a line, rotate or translate the widget
   LeftButtonReleaseEvent - release the selected handle and end interaction
 \end{verbatim}

 Note that the event bindings described above can be changed using this
 class's vtkWidgetEventTranslator. This class translates VTK events 
 into the vtkBiDimensionalWidget's widget events:
 \begin{verbatim}
   vtkWidgetEvent::AddPoint -- (In Define mode:) Add one point; depending on the 
                               state it may the first, second, third or fourth 
                               point added. (In Manipulate mode:) If near a handle, 
                               select the handle. Or if near a line, select the line.
   vtkWidgetEvent::Move -- (In Define mode:) Position the second, third or fourth 
                           point. (In Manipulate mode:) Move the handle, line or widget.
   vtkWidgetEvent::EndSelect -- the manipulation process has completed.
 \end{verbatim}

 This widget invokes the following VTK events on itself (which observers
 can listen for):
 \begin{verbatim}
   vtkCommand::StartInteractionEvent (beginning to interact)
   vtkCommand::EndInteractionEvent (completing interaction)
   vtkCommand::InteractionEvent (moving a handle, line or performing rotation)
   vtkCommand::PlacePointEvent (after a point is positioned; 
                                call data includes handle id (0,1,2,4))
 \end{verbatim}

To create an instance of class vtkBiDimensionalWidget, simply
invoke its constructor as follows
\begin{verbatim}
  obj = vtkBiDimensionalWidget
\end{verbatim}
\subsection{Methods}

The class vtkBiDimensionalWidget has several methods that can be used.
  They are listed below.
Note that the documentation is translated automatically from the VTK sources,
and may not be completely intelligible.  When in doubt, consult the VTK website.
In the methods listed below, \verb|obj| is an instance of the vtkBiDimensionalWidget class.
\begin{itemize}
\item  \verb|string = obj.GetClassName ()| -  Standard methods for a VTK class.

\item  \verb|int = obj.IsA (string name)| -  Standard methods for a VTK class.

\item  \verb|vtkBiDimensionalWidget = obj.NewInstance ()| -  Standard methods for a VTK class.

\item  \verb|vtkBiDimensionalWidget = obj.SafeDownCast (vtkObject o)| -  Standard methods for a VTK class.

\item  \verb|obj.SetEnabled (int )| -  The method for activiating and deactiviating this widget. This method
 must be overridden because it is a composite widget and does more than
 its superclasses' vtkAbstractWidget::SetEnabled() method.

\item  \verb|obj.SetRepresentation (vtkBiDimensionalRepresentation2D r)| -  Create the default widget representation if one is not set. 

\item  \verb|obj.CreateDefaultRepresentation ()| -  Create the default widget representation if one is not set. 

\item  \verb|int = obj.IsMeasureValid ()| -  A flag indicates whether the bi-dimensional measure is valid. The widget
 becomes valid after two of the four points are placed.

\item  \verb|obj.SetProcessEvents (int )| -  Methods to change the whether the widget responds to interaction.
 Overridden to pass the state to component widgets.

\end{itemize}
