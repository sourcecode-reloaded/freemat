\section{vtkPOpenFOAMReader}

\subsection{Usage}

 vtkPOpenFOAMReader creates a multiblock dataset. It reads
 parallel-decomposed mesh information and time dependent data.  The
 polyMesh folders contain mesh information. The time folders contain
 transient data for the cells. Each folder can contain any number of
 data files.

To create an instance of class vtkPOpenFOAMReader, simply
invoke its constructor as follows
\begin{verbatim}
  obj = vtkPOpenFOAMReader
\end{verbatim}
\subsection{Methods}

The class vtkPOpenFOAMReader has several methods that can be used.
  They are listed below.
Note that the documentation is translated automatically from the VTK sources,
and may not be completely intelligible.  When in doubt, consult the VTK website.
In the methods listed below, \verb|obj| is an instance of the vtkPOpenFOAMReader class.
\begin{itemize}
\item  \verb|string = obj.GetClassName ()|

\item  \verb|int = obj.IsA (string name)|

\item  \verb|vtkPOpenFOAMReader = obj.NewInstance ()|

\item  \verb|vtkPOpenFOAMReader = obj.SafeDownCast (vtkObject o)|

\item  \verb|obj.SetCaseType (int t)| -  Set and get case type. 0 = decomposed case, 1 = reconstructed case.

\item  \verb|obj.SetController (vtkMultiProcessController )| -  Set and get the controller.

\item  \verb|vtkMultiProcessController = obj.GetController ()| -  Set and get the controller.

\end{itemize}
