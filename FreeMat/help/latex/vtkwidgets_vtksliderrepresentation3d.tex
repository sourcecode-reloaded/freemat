\section{vtkSliderRepresentation3D}

\subsection{Usage}

 This class is used to represent and render a vtkSliderWidget. To use this
 class, you must at a minimum specify the end points of the
 slider. Optional instance variable can be used to modify the appearance of
 the widget.


To create an instance of class vtkSliderRepresentation3D, simply
invoke its constructor as follows
\begin{verbatim}
  obj = vtkSliderRepresentation3D
\end{verbatim}
\subsection{Methods}

The class vtkSliderRepresentation3D has several methods that can be used.
  They are listed below.
Note that the documentation is translated automatically from the VTK sources,
and may not be completely intelligible.  When in doubt, consult the VTK website.
In the methods listed below, \verb|obj| is an instance of the vtkSliderRepresentation3D class.
\begin{itemize}
\item  \verb|string = obj.GetClassName ()| -  Standard methods for the class.

\item  \verb|int = obj.IsA (string name)| -  Standard methods for the class.

\item  \verb|vtkSliderRepresentation3D = obj.NewInstance ()| -  Standard methods for the class.

\item  \verb|vtkSliderRepresentation3D = obj.SafeDownCast (vtkObject o)| -  Standard methods for the class.

\item  \verb|vtkCoordinate = obj.GetPoint1Coordinate ()| -  Position the first end point of the slider. Note that this point is an
 instance of vtkCoordinate, meaning that Point 1 can be specified in a
 variety of coordinate systems, and can even be relative to another
 point. To set the point, you'll want to get the Point1Coordinate and
 then invoke the necessary methods to put it into the correct coordinate
 system and set the correct initial value.

\item  \verb|obj.SetPoint1InWorldCoordinates (double x, double y, double z)| -  Position the first end point of the slider. Note that this point is an
 instance of vtkCoordinate, meaning that Point 1 can be specified in a
 variety of coordinate systems, and can even be relative to another
 point. To set the point, you'll want to get the Point1Coordinate and
 then invoke the necessary methods to put it into the correct coordinate
 system and set the correct initial value.

\item  \verb|vtkCoordinate = obj.GetPoint2Coordinate ()| -  Position the second end point of the slider. Note that this point is an
 instance of vtkCoordinate, meaning that Point 1 can be specified in a
 variety of coordinate systems, and can even be relative to another
 point. To set the point, you'll want to get the Point2Coordinate and
 then invoke the necessary methods to put it into the correct coordinate
 system and set the correct initial value.

\item  \verb|obj.SetPoint2InWorldCoordinates (double x, double y, double z)| -  Position the second end point of the slider. Note that this point is an
 instance of vtkCoordinate, meaning that Point 1 can be specified in a
 variety of coordinate systems, and can even be relative to another
 point. To set the point, you'll want to get the Point2Coordinate and
 then invoke the necessary methods to put it into the correct coordinate
 system and set the correct initial value.

\item  \verb|obj.SetTitleText (string )| -  Specify the title text for this widget. If the value is not set, or set
 to the empty string '''', then the title text is not displayed.

\item  \verb|string = obj.GetTitleText ()| -  Specify the title text for this widget. If the value is not set, or set
 to the empty string '''', then the title text is not displayed.

\item  \verb|obj.SetSliderShape (int )| -  Specify whether to use a sphere or cylinder slider shape. By default, a
 sphere shape is used.

\item  \verb|int = obj.GetSliderShapeMinValue ()| -  Specify whether to use a sphere or cylinder slider shape. By default, a
 sphere shape is used.

\item  \verb|int = obj.GetSliderShapeMaxValue ()| -  Specify whether to use a sphere or cylinder slider shape. By default, a
 sphere shape is used.

\item  \verb|int = obj.GetSliderShape ()| -  Specify whether to use a sphere or cylinder slider shape. By default, a
 sphere shape is used.

\item  \verb|obj.SetSliderShapeToSphere ()| -  Specify whether to use a sphere or cylinder slider shape. By default, a
 sphere shape is used.

\item  \verb|obj.SetSliderShapeToCylinder ()| -  Set the rotation of the slider widget around the axis of the widget. This is
 used to control which way the widget is initially oriented. (This is especially
 important for the label and title.)

\item  \verb|obj.SetRotation (double )| -  Set the rotation of the slider widget around the axis of the widget. This is
 used to control which way the widget is initially oriented. (This is especially
 important for the label and title.)

\item  \verb|double = obj.GetRotation ()| -  Set the rotation of the slider widget around the axis of the widget. This is
 used to control which way the widget is initially oriented. (This is especially
 important for the label and title.)

\item  \verb|vtkProperty = obj.GetSliderProperty ()| -  Get the slider properties. The properties of the slider when selected 
 and unselected can be manipulated.

\item  \verb|vtkProperty = obj.GetTubeProperty ()| -  Get the properties for the tube and end caps. 

\item  \verb|vtkProperty = obj.GetCapProperty ()| -  Get the properties for the tube and end caps. 

\item  \verb|vtkProperty = obj.GetSelectedProperty ()| -  Get the selection property. This property is used to modify the appearance of
 selected objects (e.g., the slider).

\item  \verb|obj.PlaceWidget (double bounds[6])| -  Methods to interface with the vtkSliderWidget.

\item  \verb|obj.BuildRepresentation ()| -  Methods to interface with the vtkSliderWidget.

\item  \verb|obj.StartWidgetInteraction (double eventPos[2])| -  Methods to interface with the vtkSliderWidget.

\item  \verb|obj.WidgetInteraction (double newEventPos[2])| -  Methods to interface with the vtkSliderWidget.

\item  \verb|obj.Highlight (int )| -  Methods to interface with the vtkSliderWidget.

\item  \verb|double = obj.GetBounds ()|

\item  \verb|obj.GetActors (vtkPropCollection )|

\item  \verb|obj.ReleaseGraphicsResources (vtkWindow )|

\item  \verb|int = obj.RenderOpaqueGeometry (vtkViewport )|

\item  \verb|int = obj.RenderTranslucentPolygonalGeometry (vtkViewport )|

\item  \verb|int = obj.HasTranslucentPolygonalGeometry ()|

\item  \verb|long = obj.GetMTime ()| -  Override GetMTime to include point coordinates

\end{itemize}
