\section{vtkGeoTransform}

\subsection{Usage}

 This class takes two geographic projections and transforms point
 coordinates between them.

To create an instance of class vtkGeoTransform, simply
invoke its constructor as follows
\begin{verbatim}
  obj = vtkGeoTransform
\end{verbatim}
\subsection{Methods}

The class vtkGeoTransform has several methods that can be used.
  They are listed below.
Note that the documentation is translated automatically from the VTK sources,
and may not be completely intelligible.  When in doubt, consult the VTK website.
In the methods listed below, \verb|obj| is an instance of the vtkGeoTransform class.
\begin{itemize}
\item  \verb|string = obj.GetClassName ()|

\item  \verb|int = obj.IsA (string name)|

\item  \verb|vtkGeoTransform = obj.NewInstance ()|

\item  \verb|vtkGeoTransform = obj.SafeDownCast (vtkObject o)|

\item  \verb|obj.SetSourceProjection (vtkGeoProjection source)| -  The source geographic projection.

\item  \verb|vtkGeoProjection = obj.GetSourceProjection ()| -  The source geographic projection.

\item  \verb|obj.SetDestinationProjection (vtkGeoProjection dest)| -  The target geographic projection.

\item  \verb|vtkGeoProjection = obj.GetDestinationProjection ()| -  The target geographic projection.

\item  \verb|obj.TransformPoints (vtkPoints src, vtkPoints dst)| -  Transform many points at once.

\item  \verb|obj.Inverse ()| -  Invert the transformation.

\item  \verb|obj.InternalTransformPoint (float in[3], float out[3])| -  This will calculate the transformation without calling Update.
 Meant for use only within other VTK classes.

\item  \verb|obj.InternalTransformPoint (double in[3], double out[3])| -  This will calculate the transformation without calling Update.
 Meant for use only within other VTK classes.

\item  \verb|vtkAbstractTransform = obj.MakeTransform ()| -  Make another transform of the same type.

\end{itemize}
