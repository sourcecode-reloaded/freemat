\section{vtkBitArray}

\subsection{Usage}

 vtkBitArray is an array of bits (0/1 data value). The array is packed 
 so that each byte stores eight bits. vtkBitArray provides methods
 for insertion and retrieval of bits, and will automatically resize 
 itself to hold new data.

To create an instance of class vtkBitArray, simply
invoke its constructor as follows
\begin{verbatim}
  obj = vtkBitArray
\end{verbatim}
\subsection{Methods}

The class vtkBitArray has several methods that can be used.
  They are listed below.
Note that the documentation is translated automatically from the VTK sources,
and may not be completely intelligible.  When in doubt, consult the VTK website.
In the methods listed below, \verb|obj| is an instance of the vtkBitArray class.
\begin{itemize}
\item  \verb|string = obj.GetClassName ()|

\item  \verb|int = obj.IsA (string name)|

\item  \verb|vtkBitArray = obj.NewInstance ()|

\item  \verb|vtkBitArray = obj.SafeDownCast (vtkObject o)|

\item  \verb|int = obj.Allocate (vtkIdType sz, vtkIdType ext)| -  Allocate memory for this array. Delete old storage only if necessary.
 Note that ext is no longer used.

\item  \verb|obj.Initialize ()| -  Release storage and reset array to initial state.

\item  \verb|int = obj.GetDataType ()|

\item  \verb|int = obj.GetDataTypeSize ()| -  Set the number of n-tuples in the array.

\item  \verb|obj.SetNumberOfTuples (vtkIdType number)| -  Set the number of n-tuples in the array.

\item  \verb|obj.SetTuple (vtkIdType i, vtkIdType j, vtkAbstractArray source)| -  Set the tuple at the ith location using the jth tuple in the source array.
 This method assumes that the two arrays have the same type
 and structure. Note that range checking and memory allocation is not 
 performed; use in conjunction with SetNumberOfTuples() to allocate space.

\item  \verb|obj.InsertTuple (vtkIdType i, vtkIdType j, vtkAbstractArray source)| -  Insert the jth tuple in the source array, at ith location in this array. 
 Note that memory allocation is performed as necessary to hold the data.

\item  \verb|vtkIdType = obj.InsertNextTuple (vtkIdType j, vtkAbstractArray source)| -  Insert the jth tuple in the source array, at the end in this array. 
 Note that memory allocation is performed as necessary to hold the data.
 Returns the location at which the data was inserted.

\item  \verb|obj.GetTuple (vtkIdType i, double tuple)| -  Copy the tuple value into a user-provided array.

\item  \verb|obj.SetTuple (vtkIdType i, float tuple)| -  Set the tuple value at the ith location in the array.

\item  \verb|obj.SetTuple (vtkIdType i, double tuple)| -  Set the tuple value at the ith location in the array.

\item  \verb|obj.InsertTuple (vtkIdType i, float tuple)| -  Insert (memory allocation performed) the tuple into the ith location
 in the array.

\item  \verb|obj.InsertTuple (vtkIdType i, double tuple)| -  Insert (memory allocation performed) the tuple into the ith location
 in the array.

\item  \verb|vtkIdType = obj.InsertNextTuple (float tuple)| -  Insert (memory allocation performed) the tuple onto the end of the array.

\item  \verb|vtkIdType = obj.InsertNextTuple (double tuple)| -  Insert (memory allocation performed) the tuple onto the end of the array.

\item  \verb|obj.RemoveTuple (vtkIdType id)| -  These methods remove tuples from the data array. They shift data and
 resize array, so the data array is still valid after this operation. Note,
 this operation is fairly slow.

\item  \verb|obj.RemoveFirstTuple ()| -  These methods remove tuples from the data array. They shift data and
 resize array, so the data array is still valid after this operation. Note,
 this operation is fairly slow.

\item  \verb|obj.RemoveLastTuple ()| -  These methods remove tuples from the data array. They shift data and
 resize array, so the data array is still valid after this operation. Note,
 this operation is fairly slow.

\item  \verb|obj.SetComponent (vtkIdType i, int j, double c)| -  Set the data component at the ith tuple and jth component location.
 Note that i is less then NumberOfTuples and j is less then 
 NumberOfComponents. Make sure enough memory has been allocated (use 
 SetNumberOfTuples() and  SetNumberOfComponents()).

\item  \verb|obj.Squeeze ()| -  Free any unneeded memory.

\item  \verb|int = obj.Resize (vtkIdType numTuples)| -  Resize the array while conserving the data.

\item  \verb|int = obj.GetValue (vtkIdType id)| -  Get the data at a particular index.

\item  \verb|obj.SetNumberOfValues (vtkIdType number)| -  Fast method based setting of values without memory checks. First
 use SetNumberOfValues then use SetValue to actually set them.
 Specify the number of values for this object to hold. Does an
 allocation as well as setting the MaxId ivar. Used in conjunction with
 SetValue() method for fast insertion.

\item  \verb|obj.SetValue (vtkIdType id, int value)| -  Set the data at a particular index. Does not do range checking. Make sure
 you use the method SetNumberOfValues() before inserting data.

\item  \verb|obj.InsertValue (vtkIdType id, int i)| -  Insets values and checks to make sure there is enough memory

\item  \verb|vtkIdType = obj.InsertNextValue (int i)|

\item  \verb|obj.InsertComponent (vtkIdType i, int j, double c)| -  Insert the data component at ith tuple and jth component location. 
 Note that memory allocation is performed as necessary to hold the data.

\item  \verb|obj.DeepCopy (vtkDataArray da)| -  Deep copy of another bit array.

\item  \verb|obj.DeepCopy (vtkAbstractArray aa)| -  This method lets the user specify data to be held by the array.  The 
 array argument is a pointer to the data.  size is the size of 
 the array supplied by the user.  Set save to 1 to keep the class
 from deleting the array when it cleans up or reallocates memory.
 The class uses the actual array provided; it does not copy the data 
 from the suppled array. If save 0, the array must have been allocated
 with new[] not malloc.

\item  \verb|obj.SetArray (string array, vtkIdType size, int save)| -  This method lets the user specify data to be held by the array.  The 
 array argument is a pointer to the data.  size is the size of 
 the array supplied by the user.  Set save to 1 to keep the class
 from deleting the array when it cleans up or reallocates memory.
 The class uses the actual array provided; it does not copy the data 
 from the suppled array. If save 0, the array must have been allocated
 with new[] not malloc.

\item  \verb|vtkArrayIterator = obj.NewIterator ()| -  Returns a new vtkBitArrayIterator instance.

\item  \verb|vtkIdType = obj.LookupValue (int value)|

\item  \verb|obj.LookupValue (int value, vtkIdList ids)|

\item  \verb|obj.DataChanged ()| -  Tell the array explicitly that the data has changed.
 This is only necessary to call when you modify the array contents
 without using the array's API (i.e. you retrieve a pointer to the
 data and modify the array contents).  You need to call this so that
 the fast lookup will know to rebuild itself.  Otherwise, the lookup
 functions will give incorrect results.

\item  \verb|obj.ClearLookup ()| -  Delete the associated fast lookup data structure on this array,
 if it exists.  The lookup will be rebuilt on the next call to a lookup
 function.

\end{itemize}
