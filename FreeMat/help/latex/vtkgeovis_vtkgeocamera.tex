\section{vtkGeoCamera}

\subsection{Usage}

 I wanted to hide the normal vtkCamera API
 so I did not make this a subclass.  The camera is a helper object.
 You can get a pointer to the camera, but it should be treated like
 a const.

To create an instance of class vtkGeoCamera, simply
invoke its constructor as follows
\begin{verbatim}
  obj = vtkGeoCamera
\end{verbatim}
\subsection{Methods}

The class vtkGeoCamera has several methods that can be used.
  They are listed below.
Note that the documentation is translated automatically from the VTK sources,
and may not be completely intelligible.  When in doubt, consult the VTK website.
In the methods listed below, \verb|obj| is an instance of the vtkGeoCamera class.
\begin{itemize}
\item  \verb|string = obj.GetClassName ()|

\item  \verb|int = obj.IsA (string name)|

\item  \verb|vtkGeoCamera = obj.NewInstance ()|

\item  \verb|vtkGeoCamera = obj.SafeDownCast (vtkObject o)|

\item  \verb|double = obj. GetPosition ()| -  Get the world position without the origin shift.

\item  \verb|obj.SetLongitude (double longitude)| -  Longitude is in degrees: (-180->180)
   Relative to absolute coordinates.
  Rotate Longitude  around z axis (earth axis),

\item  \verb|double = obj.GetLongitude ()| -  Longitude is in degrees: (-180->180)
   Relative to absolute coordinates.
  Rotate Longitude  around z axis (earth axis),

\item  \verb|obj.SetLatitude (double latitude)| -   Latitude is in degrees: (-90->90)
    Relative to Longitude.
  Rotate Latitude   around x axis by Latitude, 

\item  \verb|double = obj.GetLatitude ()| -   Latitude is in degrees: (-90->90)
    Relative to Longitude.
  Rotate Latitude   around x axis by Latitude, 

\item  \verb|obj.SetDistance (double Distance)| -  Distance is in Meters
   Relative to Longitude and Latitude.
   above sea level   ???? should we make this from center of earth ????
                    ???? what about equatorial bulge ????

\item  \verb|double = obj.GetDistance ()| -  Distance is in Meters
   Relative to Longitude and Latitude.
   above sea level   ???? should we make this from center of earth ????
                    ???? what about equatorial bulge ????

\item  \verb|obj.SetHeading (double heading)| -  Heading is in degrees:  (-180->180)
   Relative to Logitude and Latitude.
   0 is north. 
   90 is east.       ???? what is the standard ????
   180 is south.
   -90 is west.  
  Rotate Heading    around -y axis Center,     

\item  \verb|double = obj.GetHeading ()| -  Heading is in degrees:  (-180->180)
   Relative to Logitude and Latitude.
   0 is north. 
   90 is east.       ???? what is the standard ????
   180 is south.
   -90 is west.  
  Rotate Heading    around -y axis Center,     

\item  \verb|obj.SetTilt (double tilt)| -  Tilt is also know as pitch.
 Tilt is in degrees: (0->90)
   Relative to Longitude, Latitude, and Heading.
 Rotate Tilt       around x axis,             

\item  \verb|double = obj.GetTilt ()| -  Tilt is also know as pitch.
 Tilt is in degrees: (0->90)
   Relative to Longitude, Latitude, and Heading.
 Rotate Tilt       around x axis,             

\item  \verb|vtkCamera = obj.GetVTKCamera ()| -  This vtk camera is updated to match this geo cameras state.
 It should be treated as a const and should not be modified.

\item  \verb|obj.InitializeNodeAnalysis (int rendererSize[2])| -  We precompute some values to speed up update of the terrain.
 Unfortunately, they have to be manually/explicitely updated
 when the camera or renderer size changes.

\item  \verb|double = obj.GetNodeCoverage (vtkGeoTerrainNode node)| -  This method estimates how much of the view is covered by the sphere.
 Returns a value from 0 to 1.

\item  \verb|bool = obj.GetLockHeading ()|

\item  \verb|obj.SetLockHeading (bool )|

\item  \verb|obj.LockHeadingOn ()|

\item  \verb|obj.LockHeadingOff ()|

\item  \verb|obj.SetOriginLatitude (double oLat)| -  This point is shifted to 0,0,0 to avoid openGL issues.

\item  \verb|double = obj.GetOriginLatitude ()| -  This point is shifted to 0,0,0 to avoid openGL issues.

\item  \verb|obj.SetOriginLongitude (double oLat)| -  This point is shifted to 0,0,0 to avoid openGL issues.

\item  \verb|double = obj.GetOriginLongitude ()| -  This point is shifted to 0,0,0 to avoid openGL issues.

\item  \verb|double = obj. GetOrigin ()| -  Get the rectilinear cooridinate location of the origin.
 This is used to shift the terrain points.

\item  \verb|obj.SetOrigin (double ox, double oy, double oz)|

\end{itemize}
