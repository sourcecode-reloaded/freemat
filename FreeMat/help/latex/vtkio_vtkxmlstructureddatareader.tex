\section{vtkXMLStructuredDataReader}

\subsection{Usage}

 vtkXMLStructuredDataReader provides functionality common to all
 structured data format readers.

To create an instance of class vtkXMLStructuredDataReader, simply
invoke its constructor as follows
\begin{verbatim}
  obj = vtkXMLStructuredDataReader
\end{verbatim}
\subsection{Methods}

The class vtkXMLStructuredDataReader has several methods that can be used.
  They are listed below.
Note that the documentation is translated automatically from the VTK sources,
and may not be completely intelligible.  When in doubt, consult the VTK website.
In the methods listed below, \verb|obj| is an instance of the vtkXMLStructuredDataReader class.
\begin{itemize}
\item  \verb|string = obj.GetClassName ()|

\item  \verb|int = obj.IsA (string name)|

\item  \verb|vtkXMLStructuredDataReader = obj.NewInstance ()|

\item  \verb|vtkXMLStructuredDataReader = obj.SafeDownCast (vtkObject o)|

\item  \verb|vtkIdType = obj.GetNumberOfPoints ()| -  Get the number of points in the output.

\item  \verb|vtkIdType = obj.GetNumberOfCells ()| -  Get the number of cells in the output.

\item  \verb|obj.SetWholeSlices (int )| -  Get/Set whether the reader gets a whole slice from disk when only
 a rectangle inside it is needed.  This mode reads more data than
 necessary, but prevents many short reads from interacting poorly
 with the compression and encoding schemes.

\item  \verb|int = obj.GetWholeSlices ()| -  Get/Set whether the reader gets a whole slice from disk when only
 a rectangle inside it is needed.  This mode reads more data than
 necessary, but prevents many short reads from interacting poorly
 with the compression and encoding schemes.

\item  \verb|obj.WholeSlicesOn ()| -  Get/Set whether the reader gets a whole slice from disk when only
 a rectangle inside it is needed.  This mode reads more data than
 necessary, but prevents many short reads from interacting poorly
 with the compression and encoding schemes.

\item  \verb|obj.WholeSlicesOff ()| -  Get/Set whether the reader gets a whole slice from disk when only
 a rectangle inside it is needed.  This mode reads more data than
 necessary, but prevents many short reads from interacting poorly
 with the compression and encoding schemes.

\item  \verb|obj.CopyOutputInformation (vtkInformation outInfo, int port)| -  For the specified port, copy the information this reader sets up in
 SetupOutputInformation to outInfo

\end{itemize}
