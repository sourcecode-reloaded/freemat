\section{vtkExtractDataOverTime}

\subsection{Usage}

 This filter extracts the point data from a time sequence and specified index
 and creates an output of the same type as the input but with Points 
 containing ''number of time steps'' points; the point and PointData 
 corresponding to the PointIndex are extracted at each time step and added to
 the output.  A PointData array is added called ''Time'' (or ''TimeData'' if
 there is already an array called ''Time''), which is the time at each index.

To create an instance of class vtkExtractDataOverTime, simply
invoke its constructor as follows
\begin{verbatim}
  obj = vtkExtractDataOverTime
\end{verbatim}
\subsection{Methods}

The class vtkExtractDataOverTime has several methods that can be used.
  They are listed below.
Note that the documentation is translated automatically from the VTK sources,
and may not be completely intelligible.  When in doubt, consult the VTK website.
In the methods listed below, \verb|obj| is an instance of the vtkExtractDataOverTime class.
\begin{itemize}
\item  \verb|string = obj.GetClassName ()|

\item  \verb|int = obj.IsA (string name)|

\item  \verb|vtkExtractDataOverTime = obj.NewInstance ()|

\item  \verb|vtkExtractDataOverTime = obj.SafeDownCast (vtkObject o)|

\item  \verb|obj.SetPointIndex (int )| -  Index of point to extract at each time step

\item  \verb|int = obj.GetPointIndex ()| -  Index of point to extract at each time step

\item  \verb|int = obj.GetNumberOfTimeSteps ()| -  Get the number of time steps

\end{itemize}
