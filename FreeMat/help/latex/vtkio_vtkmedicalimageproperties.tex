\section{vtkMedicalImageProperties}

\subsection{Usage}

 vtkMedicalImageProperties is a helper class that can be used by medical
 image readers and applications to encapsulate medical image/acquisition
 properties. Later on, this should probably be extended to add
 any user-defined property.

To create an instance of class vtkMedicalImageProperties, simply
invoke its constructor as follows
\begin{verbatim}
  obj = vtkMedicalImageProperties
\end{verbatim}
\subsection{Methods}

The class vtkMedicalImageProperties has several methods that can be used.
  They are listed below.
Note that the documentation is translated automatically from the VTK sources,
and may not be completely intelligible.  When in doubt, consult the VTK website.
In the methods listed below, \verb|obj| is an instance of the vtkMedicalImageProperties class.
\begin{itemize}
\item  \verb|string = obj.GetClassName ()|

\item  \verb|int = obj.IsA (string name)|

\item  \verb|vtkMedicalImageProperties = obj.NewInstance ()|

\item  \verb|vtkMedicalImageProperties = obj.SafeDownCast (vtkObject o)|

\item  \verb|obj.Clear ()| -  Convenience method to reset all fields to an emptry string/value

\item  \verb|obj.SetPatientName (string )| -  Patient name
 For ex: DICOM (0010,0010) = DOE,JOHN

\item  \verb|string = obj.GetPatientName ()| -  Patient name
 For ex: DICOM (0010,0010) = DOE,JOHN

\item  \verb|obj.SetPatientID (string )| -  Patient ID
 For ex: DICOM (0010,0020) = 1933197

\item  \verb|string = obj.GetPatientID ()| -  Patient ID
 For ex: DICOM (0010,0020) = 1933197

\item  \verb|obj.SetPatientAge (string )| -  Patient age
 Format: nnnD, nnW, nnnM or nnnY (eventually nnD, nnW, nnY)
         with D (day), M (month), W (week), Y (year)
 For ex: DICOM (0010,1010) = 031Y

\item  \verb|string = obj.GetPatientAge ()| -  Patient age
 Format: nnnD, nnW, nnnM or nnnY (eventually nnD, nnW, nnY)
         with D (day), M (month), W (week), Y (year)
 For ex: DICOM (0010,1010) = 031Y

\item  \verb|int = obj.GetPatientAgeYear ()|

\item  \verb|int = obj.GetPatientAgeMonth ()|

\item  \verb|int = obj.GetPatientAgeWeek ()|

\item  \verb|int = obj.GetPatientAgeDay ()|

\item  \verb|obj.SetPatientSex (string )| -  Patient sex
 For ex: DICOM (0010,0040) = M

\item  \verb|string = obj.GetPatientSex ()| -  Patient sex
 For ex: DICOM (0010,0040) = M

\item  \verb|obj.SetPatientBirthDate (string )| -  Patient birth date
 Format: yyyymmdd
 For ex: DICOM (0010,0030) = 19680427

\item  \verb|string = obj.GetPatientBirthDate ()| -  Patient birth date
 Format: yyyymmdd
 For ex: DICOM (0010,0030) = 19680427

\item  \verb|int = obj.GetPatientBirthDateYear ()|

\item  \verb|int = obj.GetPatientBirthDateMonth ()|

\item  \verb|int = obj.GetPatientBirthDateDay ()|

\item  \verb|obj.SetStudyDate (string )| -  Study Date
 Format: yyyymmdd
 For ex: DICOM (0008,0020) = 20030617

\item  \verb|string = obj.GetStudyDate ()| -  Study Date
 Format: yyyymmdd
 For ex: DICOM (0008,0020) = 20030617

\item  \verb|obj.SetAcquisitionDate (string )| -  Acquisition Date
 Format: yyyymmdd
 For ex: DICOM (0008,0022) = 20030617

\item  \verb|string = obj.GetAcquisitionDate ()| -  Acquisition Date
 Format: yyyymmdd
 For ex: DICOM (0008,0022) = 20030617

\item  \verb|int = obj.GetAcquisitionDateYear ()|

\item  \verb|int = obj.GetAcquisitionDateMonth ()|

\item  \verb|int = obj.GetAcquisitionDateDay ()|

\item  \verb|obj.SetStudyTime (string )| -  Study Time
 Format: hhmmss.frac (any trailing component(s) can be ommited)
 For ex: DICOM (0008,0030) = 162552.0705 or 230012, or 0012

\item  \verb|string = obj.GetStudyTime ()| -  Study Time
 Format: hhmmss.frac (any trailing component(s) can be ommited)
 For ex: DICOM (0008,0030) = 162552.0705 or 230012, or 0012

\item  \verb|obj.SetAcquisitionTime (string )| -  Acquisition time
 Format: hhmmss.frac (any trailing component(s) can be ommited)
 For ex: DICOM (0008,0032) = 162552.0705 or 230012, or 0012

\item  \verb|string = obj.GetAcquisitionTime ()| -  Acquisition time
 Format: hhmmss.frac (any trailing component(s) can be ommited)
 For ex: DICOM (0008,0032) = 162552.0705 or 230012, or 0012

\item  \verb|obj.SetImageDate (string )| -  Image Date aka Content Date
 Format: yyyymmdd
 For ex: DICOM (0008,0023) = 20030617

\item  \verb|string = obj.GetImageDate ()| -  Image Date aka Content Date
 Format: yyyymmdd
 For ex: DICOM (0008,0023) = 20030617

\item  \verb|int = obj.GetImageDateYear ()|

\item  \verb|int = obj.GetImageDateMonth ()|

\item  \verb|int = obj.GetImageDateDay ()|

\item  \verb|obj.SetImageTime (string )| -  Image Time
 Format: hhmmss.frac (any trailing component(s) can be ommited)
 For ex: DICOM (0008,0033) = 162552.0705 or 230012, or 0012

\item  \verb|string = obj.GetImageTime ()| -  Image Time
 Format: hhmmss.frac (any trailing component(s) can be ommited)
 For ex: DICOM (0008,0033) = 162552.0705 or 230012, or 0012

\item  \verb|obj.SetImageNumber (string )| -  Image number
 For ex: DICOM (0020,0013) = 1

\item  \verb|string = obj.GetImageNumber ()| -  Image number
 For ex: DICOM (0020,0013) = 1

\item  \verb|obj.SetSeriesNumber (string )| -  Series number
 For ex: DICOM (0020,0011) = 902

\item  \verb|string = obj.GetSeriesNumber ()| -  Series number
 For ex: DICOM (0020,0011) = 902

\item  \verb|obj.SetSeriesDescription (string )| -  Series Description
 User provided description of the Series
 For ex: DICOM (0008,103e) = SCOUT

\item  \verb|string = obj.GetSeriesDescription ()| -  Series Description
 User provided description of the Series
 For ex: DICOM (0008,103e) = SCOUT

\item  \verb|obj.SetStudyID (string )| -  Study ID
 For ex: DICOM (0020,0010) = 37481

\item  \verb|string = obj.GetStudyID ()| -  Study ID
 For ex: DICOM (0020,0010) = 37481

\item  \verb|obj.SetStudyDescription (string )| -  Study description
 For ex: DICOM (0008,1030) = BRAIN/C-SP/FACIAL

\item  \verb|string = obj.GetStudyDescription ()| -  Study description
 For ex: DICOM (0008,1030) = BRAIN/C-SP/FACIAL

\item  \verb|obj.SetModality (string )| -  Modality
 For ex: DICOM (0008,0060)= CT

\item  \verb|string = obj.GetModality ()| -  Modality
 For ex: DICOM (0008,0060)= CT

\item  \verb|obj.SetManufacturer (string )| -  Manufacturer
 For ex: DICOM (0008,0070) = Siemens

\item  \verb|string = obj.GetManufacturer ()| -  Manufacturer
 For ex: DICOM (0008,0070) = Siemens

\item  \verb|obj.SetManufacturerModelName (string )| -  Manufacturer's Model Name
 For ex: DICOM (0008,1090) = LightSpeed QX/i

\item  \verb|string = obj.GetManufacturerModelName ()| -  Manufacturer's Model Name
 For ex: DICOM (0008,1090) = LightSpeed QX/i

\item  \verb|obj.SetStationName (string )| -  Station Name
 For ex: DICOM (0008,1010) = LSPD\_OC8

\item  \verb|string = obj.GetStationName ()| -  Station Name
 For ex: DICOM (0008,1010) = LSPD\_OC8

\item  \verb|obj.SetInstitutionName (string )| -  Institution Name
 For ex: DICOM (0008,0080) = FooCity Medical Center

\item  \verb|string = obj.GetInstitutionName ()| -  Institution Name
 For ex: DICOM (0008,0080) = FooCity Medical Center

\item  \verb|obj.SetConvolutionKernel (string )| -  Convolution Kernel (or algorithm used to reconstruct the data)
 For ex: DICOM (0018,1210) = Bone

\item  \verb|string = obj.GetConvolutionKernel ()| -  Convolution Kernel (or algorithm used to reconstruct the data)
 For ex: DICOM (0018,1210) = Bone

\item  \verb|obj.SetSliceThickness (string )| -  Slice Thickness (Nominal reconstructed slice thickness, in mm)
 For ex: DICOM (0018,0050) = 0.273438

\item  \verb|string = obj.GetSliceThickness ()| -  Slice Thickness (Nominal reconstructed slice thickness, in mm)
 For ex: DICOM (0018,0050) = 0.273438

\item  \verb|double = obj.GetSliceThicknessAsDouble ()| -  Slice Thickness (Nominal reconstructed slice thickness, in mm)
 For ex: DICOM (0018,0050) = 0.273438

\item  \verb|obj.SetKVP (string )| -  Peak kilo voltage output of the (x-ray) generator used
 For ex: DICOM (0018,0060) = 120

\item  \verb|string = obj.GetKVP ()| -  Peak kilo voltage output of the (x-ray) generator used
 For ex: DICOM (0018,0060) = 120

\item  \verb|obj.SetGantryTilt (string )| -  Gantry/Detector tilt (Nominal angle of tilt in degrees of the scanning
 gantry.)
 For ex: DICOM (0018,1120) = 15

\item  \verb|string = obj.GetGantryTilt ()| -  Gantry/Detector tilt (Nominal angle of tilt in degrees of the scanning
 gantry.)
 For ex: DICOM (0018,1120) = 15

\item  \verb|double = obj.GetGantryTiltAsDouble ()| -  Gantry/Detector tilt (Nominal angle of tilt in degrees of the scanning
 gantry.)
 For ex: DICOM (0018,1120) = 15

\item  \verb|obj.SetEchoTime (string )| -  Echo Time
 (Time in ms between the middle of the excitation pulse and the peak of
 the echo produced)
 For ex: DICOM (0018,0081) = 105

\item  \verb|string = obj.GetEchoTime ()| -  Echo Time
 (Time in ms between the middle of the excitation pulse and the peak of
 the echo produced)
 For ex: DICOM (0018,0081) = 105

\item  \verb|obj.SetEchoTrainLength (string )| -  Echo Train Length
 (Number of lines in k-space acquired per excitation per image)
 For ex: DICOM (0018,0091) = 35

\item  \verb|string = obj.GetEchoTrainLength ()| -  Echo Train Length
 (Number of lines in k-space acquired per excitation per image)
 For ex: DICOM (0018,0091) = 35

\item  \verb|obj.SetRepetitionTime (string )| -  Repetition Time
 The period of time in msec between the beginning of a pulse sequence and
 the beginning of the succeeding (essentially identical) pulse sequence.
 For ex: DICOM (0018,0080) = 2040

\item  \verb|string = obj.GetRepetitionTime ()| -  Repetition Time
 The period of time in msec between the beginning of a pulse sequence and
 the beginning of the succeeding (essentially identical) pulse sequence.
 For ex: DICOM (0018,0080) = 2040

\item  \verb|obj.SetExposureTime (string )| -  Exposure time (time of x-ray exposure in msec)
 For ex: DICOM (0018,1150) = 5

\item  \verb|string = obj.GetExposureTime ()| -  Exposure time (time of x-ray exposure in msec)
 For ex: DICOM (0018,1150) = 5

\item  \verb|obj.SetXRayTubeCurrent (string )| -  X-ray tube current (in mA)
 For ex: DICOM (0018,1151) = 400

\item  \verb|string = obj.GetXRayTubeCurrent ()| -  X-ray tube current (in mA)
 For ex: DICOM (0018,1151) = 400

\item  \verb|obj.SetExposure (string )| -  Exposure (The exposure expressed in mAs, for example calculated
 from Exposure Time and X-ray Tube Current)
 For ex: DICOM (0018,1152) = 114

\item  \verb|string = obj.GetExposure ()| -  Exposure (The exposure expressed in mAs, for example calculated
 from Exposure Time and X-ray Tube Current)
 For ex: DICOM (0018,1152) = 114

\item  \verb|obj.SetDirectionCosine (double , double , double , double , double , double )| -  Get the direction cosine (default to 1,0,0,0,1,0)

\item  \verb|obj.SetDirectionCosine (double  a[6])| -  Get the direction cosine (default to 1,0,0,0,1,0)

\item  \verb|double = obj. GetDirectionCosine ()| -  Get the direction cosine (default to 1,0,0,0,1,0)

\item  \verb|obj.AddUserDefinedValue (string name, string value)|

\item  \verb|string = obj.GetUserDefinedValue (string name)|

\item  \verb|int = obj.GetNumberOfUserDefinedValues ()|

\item  \verb|string = obj.GetUserDefinedNameByIndex (int idx)|

\item  \verb|string = obj.GetUserDefinedValueByIndex (int idx)|

\item  \verb|obj.RemoveAllUserDefinedValues ()|

\item  \verb|int = obj.AddWindowLevelPreset (double w, double l)| -  Add/Remove/Query the window/level presets that may have been associated
 to a medical image. Window is also known as 'width', level is also known
 as 'center'. The same window/level pair can not be added twice.
 As a convenience, a comment (aka Explanation) can be associated to
 a preset.
 For ex:
 \begin{verbatim}
         DICOM Window Center (0028,1050) = 00045\000470
         DICOM Window Width  (0028,1051) = 0106\03412
         DICOM Window Center Width Explanation (0028,1055) = WINDOW1\WINDOW2
 \end{verbatim}

\item  \verb|obj.RemoveWindowLevelPreset (double w, double l)| -  Add/Remove/Query the window/level presets that may have been associated
 to a medical image. Window is also known as 'width', level is also known
 as 'center'. The same window/level pair can not be added twice.
 As a convenience, a comment (aka Explanation) can be associated to
 a preset.
 For ex:
 \begin{verbatim}
         DICOM Window Center (0028,1050) = 00045\000470
         DICOM Window Width  (0028,1051) = 0106\03412
         DICOM Window Center Width Explanation (0028,1055) = WINDOW1\WINDOW2
 \end{verbatim}

\item  \verb|obj.RemoveAllWindowLevelPresets ()| -  Add/Remove/Query the window/level presets that may have been associated
 to a medical image. Window is also known as 'width', level is also known
 as 'center'. The same window/level pair can not be added twice.
 As a convenience, a comment (aka Explanation) can be associated to
 a preset.
 For ex:
 \begin{verbatim}
         DICOM Window Center (0028,1050) = 00045\000470
         DICOM Window Width  (0028,1051) = 0106\03412
         DICOM Window Center Width Explanation (0028,1055) = WINDOW1\WINDOW2
 \end{verbatim}

\item  \verb|int = obj.GetNumberOfWindowLevelPresets ()| -  Add/Remove/Query the window/level presets that may have been associated
 to a medical image. Window is also known as 'width', level is also known
 as 'center'. The same window/level pair can not be added twice.
 As a convenience, a comment (aka Explanation) can be associated to
 a preset.
 For ex:
 \begin{verbatim}
         DICOM Window Center (0028,1050) = 00045\000470
         DICOM Window Width  (0028,1051) = 0106\03412
         DICOM Window Center Width Explanation (0028,1055) = WINDOW1\WINDOW2
 \end{verbatim}

\item  \verb|int = obj.HasWindowLevelPreset (double w, double l)| -  Add/Remove/Query the window/level presets that may have been associated
 to a medical image. Window is also known as 'width', level is also known
 as 'center'. The same window/level pair can not be added twice.
 As a convenience, a comment (aka Explanation) can be associated to
 a preset.
 For ex:
 \begin{verbatim}
         DICOM Window Center (0028,1050) = 00045\000470
         DICOM Window Width  (0028,1051) = 0106\03412
         DICOM Window Center Width Explanation (0028,1055) = WINDOW1\WINDOW2
 \end{verbatim}

\item  \verb|int = obj.GetWindowLevelPresetIndex (double w, double l)| -  Add/Remove/Query the window/level presets that may have been associated
 to a medical image. Window is also known as 'width', level is also known
 as 'center'. The same window/level pair can not be added twice.
 As a convenience, a comment (aka Explanation) can be associated to
 a preset.
 For ex:
 \begin{verbatim}
         DICOM Window Center (0028,1050) = 00045\000470
         DICOM Window Width  (0028,1051) = 0106\03412
         DICOM Window Center Width Explanation (0028,1055) = WINDOW1\WINDOW2
 \end{verbatim}

\item  \verb|int = obj.GetNthWindowLevelPreset (int idx, double w, double l)| -  Add/Remove/Query the window/level presets that may have been associated
 to a medical image. Window is also known as 'width', level is also known
 as 'center'. The same window/level pair can not be added twice.
 As a convenience, a comment (aka Explanation) can be associated to
 a preset.
 For ex:
 \begin{verbatim}
         DICOM Window Center (0028,1050) = 00045\000470
         DICOM Window Width  (0028,1051) = 0106\03412
         DICOM Window Center Width Explanation (0028,1055) = WINDOW1\WINDOW2
 \end{verbatim}

\item  \verb|double = obj.GetNthWindowLevelPreset (int idx)| -  Add/Remove/Query the window/level presets that may have been associated
 to a medical image. Window is also known as 'width', level is also known
 as 'center'. The same window/level pair can not be added twice.
 As a convenience, a comment (aka Explanation) can be associated to
 a preset.
 For ex:
 \begin{verbatim}
         DICOM Window Center (0028,1050) = 00045\000470
         DICOM Window Width  (0028,1051) = 0106\03412
         DICOM Window Center Width Explanation (0028,1055) = WINDOW1\WINDOW2
 \end{verbatim}

\item  \verb|obj.SetNthWindowLevelPresetComment (int idx, string comment)| -  Add/Remove/Query the window/level presets that may have been associated
 to a medical image. Window is also known as 'width', level is also known
 as 'center'. The same window/level pair can not be added twice.
 As a convenience, a comment (aka Explanation) can be associated to
 a preset.
 For ex:
 \begin{verbatim}
         DICOM Window Center (0028,1050) = 00045\000470
         DICOM Window Width  (0028,1051) = 0106\03412
         DICOM Window Center Width Explanation (0028,1055) = WINDOW1\WINDOW2
 \end{verbatim}

\item  \verb|string = obj.GetNthWindowLevelPresetComment (int idx)| -  Add/Remove/Query the window/level presets that may have been associated
 to a medical image. Window is also known as 'width', level is also known
 as 'center'. The same window/level pair can not be added twice.
 As a convenience, a comment (aka Explanation) can be associated to
 a preset.
 For ex:
 \begin{verbatim}
         DICOM Window Center (0028,1050) = 00045\000470
         DICOM Window Width  (0028,1051) = 0106\03412
         DICOM Window Center Width Explanation (0028,1055) = WINDOW1\WINDOW2
 \end{verbatim}

\item  \verb|string = obj.GetInstanceUIDFromSliceID (int volumeidx, int sliceid)| -  Mapping from a sliceidx within a volumeidx into a DICOM Instance UID
 Some DICOM reader can populate this structure so that later on from 
 a slice index in a vtkImageData volume we can backtrack and find out
 which 2d slice it was coming from

\item  \verb|obj.SetInstanceUIDFromSliceID (int volumeidx, int sliceid, string uid)| -  Mapping from a sliceidx within a volumeidx into a DICOM Instance UID
 Some DICOM reader can populate this structure so that later on from 
 a slice index in a vtkImageData volume we can backtrack and find out
 which 2d slice it was coming from

\item  \verb|int = obj.GetOrientationType (int volumeidx)|

\item  \verb|obj.SetOrientationType (int volumeidx, int orientation)|

\item  \verb|obj.DeepCopy (vtkMedicalImageProperties p)| -  Copy the contents of p to this instance.

\end{itemize}
