\section{vtkBMPReader}

\subsection{Usage}

 vtkBMPReader is a source object that reads Windows BMP files.
 This includes indexed and 24bit bitmaps
 Usually, all BMPs are converted to 24bit RGB, but BMPs may be output
 as 8bit images with a LookupTable if the Allow8BitBMP flag is set.

 BMPReader creates structured point datasets. The dimension of the 
 dataset depends upon the number of files read. Reading a single file 
 results in a 2D image, while reading more than one file results in a 
 3D volume.

 To read a volume, files must be of the form ''FileName.<number>''
 (e.g., foo.bmp.0, foo.bmp.1, ...). You must also specify the image 
 range. This range specifies the beginning and ending files to read (range
 can be any pair of non-negative numbers). 

 The default behavior is to read a single file. In this case, the form
 of the file is simply ''FileName'' (e.g., foo.bmp).

To create an instance of class vtkBMPReader, simply
invoke its constructor as follows
\begin{verbatim}
  obj = vtkBMPReader
\end{verbatim}
\subsection{Methods}

The class vtkBMPReader has several methods that can be used.
  They are listed below.
Note that the documentation is translated automatically from the VTK sources,
and may not be completely intelligible.  When in doubt, consult the VTK website.
In the methods listed below, \verb|obj| is an instance of the vtkBMPReader class.
\begin{itemize}
\item  \verb|string = obj.GetClassName ()|

\item  \verb|int = obj.IsA (string name)|

\item  \verb|vtkBMPReader = obj.NewInstance ()|

\item  \verb|vtkBMPReader = obj.SafeDownCast (vtkObject o)|

\item  \verb|int = obj.GetDepth ()| -  Returns the depth of the BMP, either 8 or 24.

\item  \verb|int = obj.CanReadFile (string fname)| -  Is the given file a BMP file?

\item  \verb|string = obj.GetFileExtensions ()| -  Return a descriptive name for the file format that might be useful in a GUI.

\item  \verb|string = obj.GetDescriptiveName ()| -  If this flag is set and the BMP reader encounters an 8bit file,
 the data will be kept as unsigned chars and a lookuptable will be
 exported

\item  \verb|obj.SetAllow8BitBMP (int )| -  If this flag is set and the BMP reader encounters an 8bit file,
 the data will be kept as unsigned chars and a lookuptable will be
 exported

\item  \verb|int = obj.GetAllow8BitBMP ()| -  If this flag is set and the BMP reader encounters an 8bit file,
 the data will be kept as unsigned chars and a lookuptable will be
 exported

\item  \verb|obj.Allow8BitBMPOn ()| -  If this flag is set and the BMP reader encounters an 8bit file,
 the data will be kept as unsigned chars and a lookuptable will be
 exported

\item  \verb|obj.Allow8BitBMPOff ()| -  If this flag is set and the BMP reader encounters an 8bit file,
 the data will be kept as unsigned chars and a lookuptable will be
 exported

\item  \verb|vtkLookupTable = obj.GetLookupTable ()|

\end{itemize}
