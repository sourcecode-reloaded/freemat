\section{vtkImageThreshold}

\subsection{Usage}

 vtkImageThreshold can do binary or continuous thresholding for lower, upper
 or a range of data.  The output data type may be different than the
 output, but defaults to the same type.

To create an instance of class vtkImageThreshold, simply
invoke its constructor as follows
\begin{verbatim}
  obj = vtkImageThreshold
\end{verbatim}
\subsection{Methods}

The class vtkImageThreshold has several methods that can be used.
  They are listed below.
Note that the documentation is translated automatically from the VTK sources,
and may not be completely intelligible.  When in doubt, consult the VTK website.
In the methods listed below, \verb|obj| is an instance of the vtkImageThreshold class.
\begin{itemize}
\item  \verb|string = obj.GetClassName ()|

\item  \verb|int = obj.IsA (string name)|

\item  \verb|vtkImageThreshold = obj.NewInstance ()|

\item  \verb|vtkImageThreshold = obj.SafeDownCast (vtkObject o)|

\item  \verb|obj.ThresholdByUpper (double thresh)| -  The values greater than or equal to the value match.

\item  \verb|obj.ThresholdByLower (double thresh)| -  The values less than or equal to the value match.

\item  \verb|obj.ThresholdBetween (double lower, double upper)| -  The values in a range (inclusive) match

\item  \verb|obj.SetReplaceIn (int )| -  Determines whether to replace the pixel in range with InValue

\item  \verb|int = obj.GetReplaceIn ()| -  Determines whether to replace the pixel in range with InValue

\item  \verb|obj.ReplaceInOn ()| -  Determines whether to replace the pixel in range with InValue

\item  \verb|obj.ReplaceInOff ()| -  Determines whether to replace the pixel in range with InValue

\item  \verb|obj.SetInValue (double val)| -  Replace the in range pixels with this value.

\item  \verb|double = obj.GetInValue ()| -  Replace the in range pixels with this value.

\item  \verb|obj.SetReplaceOut (int )| -  Determines whether to replace the pixel out of range with OutValue

\item  \verb|int = obj.GetReplaceOut ()| -  Determines whether to replace the pixel out of range with OutValue

\item  \verb|obj.ReplaceOutOn ()| -  Determines whether to replace the pixel out of range with OutValue

\item  \verb|obj.ReplaceOutOff ()| -  Determines whether to replace the pixel out of range with OutValue

\item  \verb|obj.SetOutValue (double val)| -  Replace the in range pixels with this value.

\item  \verb|double = obj.GetOutValue ()| -  Replace the in range pixels with this value.

\item  \verb|double = obj.GetUpperThreshold ()| -  Get the Upper and Lower thresholds.

\item  \verb|double = obj.GetLowerThreshold ()| -  Get the Upper and Lower thresholds.

\item  \verb|obj.SetOutputScalarType (int )| -  Set the desired output scalar type to cast to

\item  \verb|int = obj.GetOutputScalarType ()| -  Set the desired output scalar type to cast to

\item  \verb|obj.SetOutputScalarTypeToDouble ()| -  Set the desired output scalar type to cast to

\item  \verb|obj.SetOutputScalarTypeToFloat ()| -  Set the desired output scalar type to cast to

\item  \verb|obj.SetOutputScalarTypeToLong ()| -  Set the desired output scalar type to cast to

\item  \verb|obj.SetOutputScalarTypeToUnsignedLong ()| -  Set the desired output scalar type to cast to

\item  \verb|obj.SetOutputScalarTypeToInt ()| -  Set the desired output scalar type to cast to

\item  \verb|obj.SetOutputScalarTypeToUnsignedInt ()| -  Set the desired output scalar type to cast to

\item  \verb|obj.SetOutputScalarTypeToShort ()| -  Set the desired output scalar type to cast to

\item  \verb|obj.SetOutputScalarTypeToUnsignedShort ()| -  Set the desired output scalar type to cast to

\item  \verb|obj.SetOutputScalarTypeToChar ()| -  Set the desired output scalar type to cast to

\item  \verb|obj.SetOutputScalarTypeToSignedChar ()| -  Set the desired output scalar type to cast to

\item  \verb|obj.SetOutputScalarTypeToUnsignedChar ()|

\end{itemize}
