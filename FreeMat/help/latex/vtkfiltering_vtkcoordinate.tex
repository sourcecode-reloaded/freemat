\section{vtkCoordinate}

\subsection{Usage}

 vtkCoordinate represents position in a variety of coordinate systems, and
 converts position to other coordinate systems. It also supports relative
 positioning, so you can create a cascade of vtkCoordinate objects (no loops
 please!) that refer to each other. The typical usage of this object is
 to set the coordinate system in which to represent a position (e.g., 
 SetCoordinateSystemToNormalizedDisplay()), set the value of the coordinate
 (e.g., SetValue()), and then invoke the appropriate method to convert to 
 another coordinate system (e.g., GetComputedWorldValue()).
 
 The coordinate systems in vtk are as follows:
<PRE>
  DISPLAY -             x-y pixel values in window
  NORMALIZED DISPLAY -  x-y (0,1) normalized values
  VIEWPORT -            x-y pixel values in viewport
  NORMALIZED VIEWPORT - x-y (0,1) normalized value in viewport
  VIEW -                x-y-z (-1,1) values in camera coordinates. (z is depth)
  WORLD -               x-y-z global coordinate values
  USERDEFINED -         x-y-z in User defined space
</PRE>

 If you cascade vtkCoordinate objects, you refer to another vtkCoordinate
 object which in turn can refer to others, and so on. This allows you to
 create composite groups of things like vtkActor2D that are positioned
 relative to one another. Note that in cascaded sequences, each
 vtkCoordinate object may be specified in different coordinate systems!

To create an instance of class vtkCoordinate, simply
invoke its constructor as follows
\begin{verbatim}
  obj = vtkCoordinate
\end{verbatim}
\subsection{Methods}

The class vtkCoordinate has several methods that can be used.
  They are listed below.
Note that the documentation is translated automatically from the VTK sources,
and may not be completely intelligible.  When in doubt, consult the VTK website.
In the methods listed below, \verb|obj| is an instance of the vtkCoordinate class.
\begin{itemize}
\item  \verb|string = obj.GetClassName ()|

\item  \verb|int = obj.IsA (string name)|

\item  \verb|vtkCoordinate = obj.NewInstance ()|

\item  \verb|vtkCoordinate = obj.SafeDownCast (vtkObject o)|

\item  \verb|obj.SetCoordinateSystem (int )| -  Set/get the coordinate system which this coordinate
 is defined in. The options are Display, Normalized Display,
 Viewport, Normalized Viewport, View, and World.

\item  \verb|int = obj.GetCoordinateSystem ()| -  Set/get the coordinate system which this coordinate
 is defined in. The options are Display, Normalized Display,
 Viewport, Normalized Viewport, View, and World.

\item  \verb|obj.SetCoordinateSystemToDisplay ()| -  Set/get the coordinate system which this coordinate
 is defined in. The options are Display, Normalized Display,
 Viewport, Normalized Viewport, View, and World.

\item  \verb|obj.SetCoordinateSystemToNormalizedDisplay ()| -  Set/get the coordinate system which this coordinate
 is defined in. The options are Display, Normalized Display,
 Viewport, Normalized Viewport, View, and World.

\item  \verb|obj.SetCoordinateSystemToViewport ()| -  Set/get the coordinate system which this coordinate
 is defined in. The options are Display, Normalized Display,
 Viewport, Normalized Viewport, View, and World.

\item  \verb|obj.SetCoordinateSystemToNormalizedViewport ()| -  Set/get the coordinate system which this coordinate
 is defined in. The options are Display, Normalized Display,
 Viewport, Normalized Viewport, View, and World.

\item  \verb|obj.SetCoordinateSystemToView ()| -  Set/get the coordinate system which this coordinate
 is defined in. The options are Display, Normalized Display,
 Viewport, Normalized Viewport, View, and World.

\item  \verb|obj.SetCoordinateSystemToWorld ()|

\item  \verb|string = obj.GetCoordinateSystemAsString ()|

\item  \verb|obj.SetValue (double , double , double )| -  Set/get the value of this coordinate. This can be thought of as
 the position of this coordinate in its coordinate system.

\item  \verb|obj.SetValue (double  a[3])| -  Set/get the value of this coordinate. This can be thought of as
 the position of this coordinate in its coordinate system.

\item  \verb|double = obj. GetValue ()| -  Set/get the value of this coordinate. This can be thought of as
 the position of this coordinate in its coordinate system.

\item  \verb|obj.SetValue (double a, double b)| -  If this coordinate is relative to another coordinate,
 then specify that coordinate as the ReferenceCoordinate.
 If this is NULL the coordinate is assumed to be absolute.

\item  \verb|obj.SetReferenceCoordinate (vtkCoordinate )| -  If this coordinate is relative to another coordinate,
 then specify that coordinate as the ReferenceCoordinate.
 If this is NULL the coordinate is assumed to be absolute.

\item  \verb|vtkCoordinate = obj.GetReferenceCoordinate ()| -  If this coordinate is relative to another coordinate,
 then specify that coordinate as the ReferenceCoordinate.
 If this is NULL the coordinate is assumed to be absolute.

\item  \verb|obj.SetViewport (vtkViewport viewport)| -  If you want this coordinate to be relative to a specific
 vtkViewport (vtkRenderer) then you can specify
 that here.
 NOTE: this is a raw pointer, not a weak pointer not a reference counted
 object to avoid reference cycle loop between rendering classes and filter
 classes.

\item  \verb|vtkViewport = obj.GetViewport ()| -  If you want this coordinate to be relative to a specific
 vtkViewport (vtkRenderer) then you can specify
 that here.
 NOTE: this is a raw pointer, not a weak pointer not a reference counted
 object to avoid reference cycle loop between rendering classes and filter
 classes.

\item  \verb|double = obj.GetComputedWorldValue (vtkViewport )| -  Return the computed value in a specified coordinate system.

\item  \verb|int = obj.GetComputedViewportValue (vtkViewport )| -  Return the computed value in a specified coordinate system.

\item  \verb|int = obj.GetComputedDisplayValue (vtkViewport )| -  Return the computed value in a specified coordinate system.

\item  \verb|int = obj.GetComputedLocalDisplayValue (vtkViewport )| -  Return the computed value in a specified coordinate system.

\item  \verb|double = obj.GetComputedDoubleViewportValue (vtkViewport )|

\item  \verb|double = obj.GetComputedDoubleDisplayValue (vtkViewport )|

\end{itemize}
