\section{vtkImageGradient}

\subsection{Usage}

 vtkImageGradient computes the gradient vector of an image.  The
 vector results are stored as scalar components. The Dimensionality
 determines whether to perform a 2d or 3d gradient. The default is
 two dimensional XY gradient.  OutputScalarType is always
 double. Gradient is computed using central differences.

To create an instance of class vtkImageGradient, simply
invoke its constructor as follows
\begin{verbatim}
  obj = vtkImageGradient
\end{verbatim}
\subsection{Methods}

The class vtkImageGradient has several methods that can be used.
  They are listed below.
Note that the documentation is translated automatically from the VTK sources,
and may not be completely intelligible.  When in doubt, consult the VTK website.
In the methods listed below, \verb|obj| is an instance of the vtkImageGradient class.
\begin{itemize}
\item  \verb|string = obj.GetClassName ()|

\item  \verb|int = obj.IsA (string name)|

\item  \verb|vtkImageGradient = obj.NewInstance ()|

\item  \verb|vtkImageGradient = obj.SafeDownCast (vtkObject o)|

\item  \verb|obj.SetDimensionality (int )| -  Determines how the input is interpreted (set of 2d slices ...)

\item  \verb|int = obj.GetDimensionalityMinValue ()| -  Determines how the input is interpreted (set of 2d slices ...)

\item  \verb|int = obj.GetDimensionalityMaxValue ()| -  Determines how the input is interpreted (set of 2d slices ...)

\item  \verb|int = obj.GetDimensionality ()| -  Determines how the input is interpreted (set of 2d slices ...)

\item  \verb|obj.SetHandleBoundaries (int )| -  Get/Set whether to handle boundaries.  If enabled, boundary
 pixels are treated as duplicated so that central differencing
 works for the boundary pixels.  If disabled, the output whole
 extent of the image is reduced by one pixel.

\item  \verb|int = obj.GetHandleBoundaries ()| -  Get/Set whether to handle boundaries.  If enabled, boundary
 pixels are treated as duplicated so that central differencing
 works for the boundary pixels.  If disabled, the output whole
 extent of the image is reduced by one pixel.

\item  \verb|obj.HandleBoundariesOn ()| -  Get/Set whether to handle boundaries.  If enabled, boundary
 pixels are treated as duplicated so that central differencing
 works for the boundary pixels.  If disabled, the output whole
 extent of the image is reduced by one pixel.

\item  \verb|obj.HandleBoundariesOff ()| -  Get/Set whether to handle boundaries.  If enabled, boundary
 pixels are treated as duplicated so that central differencing
 works for the boundary pixels.  If disabled, the output whole
 extent of the image is reduced by one pixel.

\end{itemize}
