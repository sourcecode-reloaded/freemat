\section{vtkStructuredPoints}

\subsection{Usage}

 StructuredPoints is a subclass of ImageData that requires the data extent
 to exactly match the update extent. Normall image data allows that the
 data extent may be larger than the update extent.
 StructuredPoints also defines the origin differently that vtkImageData.
 For structured points the origin is the location of first point. 
 Whereas images define the origin as the location of point 0, 0, 0.
 Image Origin is stored in ivar, and structured points
 have special methods for setting/getting the origin/extents.

To create an instance of class vtkStructuredPoints, simply
invoke its constructor as follows
\begin{verbatim}
  obj = vtkStructuredPoints
\end{verbatim}
\subsection{Methods}

The class vtkStructuredPoints has several methods that can be used.
  They are listed below.
Note that the documentation is translated automatically from the VTK sources,
and may not be completely intelligible.  When in doubt, consult the VTK website.
In the methods listed below, \verb|obj| is an instance of the vtkStructuredPoints class.
\begin{itemize}
\item  \verb|string = obj.GetClassName ()|

\item  \verb|int = obj.IsA (string name)|

\item  \verb|vtkStructuredPoints = obj.NewInstance ()|

\item  \verb|vtkStructuredPoints = obj.SafeDownCast (vtkObject o)|

\item  \verb|int = obj.GetDataObjectType ()|

\end{itemize}
