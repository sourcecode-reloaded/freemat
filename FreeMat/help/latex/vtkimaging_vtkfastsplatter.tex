\section{vtkFastSplatter}

\subsection{Usage}


 vtkFastSplatter takes any vtkPointSet as input (of which vtkPolyData and
 vtkUnstructuredGrid inherit).  Each point in the data set is considered to be
 an impulse.  These impulses are convolved with a given splat image.  In other
 words, the splat image is added to the final image at every place where there
 is an input point.

 Note that point and cell data are thrown away.  If you want a sampling
 of unstructured points consider vtkGaussianSplatter or vtkShepardMethod.

 Use input port 0 for the impulse data (vtkPointSet), and input port 1 for
 the splat image (vtkImageData)

 .SECTION Bugs

 Any point outside of the extents of the image is thrown away, even if it is
 close enough such that it's convolution with the splat image would overlap
 the extents.


To create an instance of class vtkFastSplatter, simply
invoke its constructor as follows
\begin{verbatim}
  obj = vtkFastSplatter
\end{verbatim}
\subsection{Methods}

The class vtkFastSplatter has several methods that can be used.
  They are listed below.
Note that the documentation is translated automatically from the VTK sources,
and may not be completely intelligible.  When in doubt, consult the VTK website.
In the methods listed below, \verb|obj| is an instance of the vtkFastSplatter class.
\begin{itemize}
\item  \verb|string = obj.GetClassName ()|

\item  \verb|int = obj.IsA (string name)|

\item  \verb|vtkFastSplatter = obj.NewInstance ()|

\item  \verb|vtkFastSplatter = obj.SafeDownCast (vtkObject o)|

\item  \verb|obj.SetModelBounds (double , double , double , double , double , double )| -  Set / get the (xmin,xmax, ymin,ymax, zmin,zmax) bounding box in which
 the sampling is performed. If any of the (min,max) bounds values are
 min >= max, then the bounds will be computed automatically from the input
 data. Otherwise, the user-specified bounds will be used.

\item  \verb|obj.SetModelBounds (double  a[6])| -  Set / get the (xmin,xmax, ymin,ymax, zmin,zmax) bounding box in which
 the sampling is performed. If any of the (min,max) bounds values are
 min >= max, then the bounds will be computed automatically from the input
 data. Otherwise, the user-specified bounds will be used.

\item  \verb|double = obj. GetModelBounds ()| -  Set / get the (xmin,xmax, ymin,ymax, zmin,zmax) bounding box in which
 the sampling is performed. If any of the (min,max) bounds values are
 min >= max, then the bounds will be computed automatically from the input
 data. Otherwise, the user-specified bounds will be used.

\item  \verb|obj.SetOutputDimensions (int , int , int )| -  Set/get the dimensions of the output image

\item  \verb|obj.SetOutputDimensions (int  a[3])| -  Set/get the dimensions of the output image

\item  \verb|int = obj. GetOutputDimensions ()| -  Set/get the dimensions of the output image

\item  \verb|obj.SetLimitMode (int )| -  Set/get the way voxel values will be limited.  If this is set to None (the
 default), the output can have arbitrarily large values.  If set to clamp,
 the output will be clamped to [MinValue,MaxValue].  If set to scale, the
 output will be linearly scaled between MinValue and MaxValue.

\item  \verb|int = obj.GetLimitMode ()| -  Set/get the way voxel values will be limited.  If this is set to None (the
 default), the output can have arbitrarily large values.  If set to clamp,
 the output will be clamped to [MinValue,MaxValue].  If set to scale, the
 output will be linearly scaled between MinValue and MaxValue.

\item  \verb|obj.SetLimitModeToNone ()| -  Set/get the way voxel values will be limited.  If this is set to None (the
 default), the output can have arbitrarily large values.  If set to clamp,
 the output will be clamped to [MinValue,MaxValue].  If set to scale, the
 output will be linearly scaled between MinValue and MaxValue.

\item  \verb|obj.SetLimitModeToClamp ()| -  Set/get the way voxel values will be limited.  If this is set to None (the
 default), the output can have arbitrarily large values.  If set to clamp,
 the output will be clamped to [MinValue,MaxValue].  If set to scale, the
 output will be linearly scaled between MinValue and MaxValue.

\item  \verb|obj.SetLimitModeToScale ()| -  Set/get the way voxel values will be limited.  If this is set to None (the
 default), the output can have arbitrarily large values.  If set to clamp,
 the output will be clamped to [MinValue,MaxValue].  If set to scale, the
 output will be linearly scaled between MinValue and MaxValue.

\item  \verb|obj.SetLimitModeToFreezeScale ()| -  See the LimitMode method.

\item  \verb|obj.SetMinValue (double )| -  See the LimitMode method.

\item  \verb|double = obj.GetMinValue ()| -  See the LimitMode method.

\item  \verb|obj.SetMaxValue (double )| -  See the LimitMode method.

\item  \verb|double = obj.GetMaxValue ()| -  See the LimitMode method.

\item  \verb|int = obj.GetNumberOfPointsSplatted ()| -  This returns the number of points splatted (as opposed to
 discarded for being outside the image) during the previous pass.

\item  \verb|obj.SetSplatConnection (vtkAlgorithmOutput )| -  Convenience function for connecting the splat algorithm source.
 This is provided mainly for convenience using the filter with
 ParaView, VTK users should prefer SetInputConnection(1, splat) instead.

\end{itemize}
