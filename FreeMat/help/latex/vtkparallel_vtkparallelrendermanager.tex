\section{vtkParallelRenderManager}

\subsection{Usage}

 vtkParallelRenderManager operates in multiple processes.  It provides
 proper renderers and render windows for performing the parallel
 rendering correctly.  It can also attach itself to render windows and
 propagate rendering events and camera views.

 .SECTION Note:
 Many parallel rendering schemes do not correctly handle transparency.
 Unless otherwise documented, assume a sub class does not.

 .SECTION ToDo:
 Synchronization/barrier primitives.

 Query ranges of scalar values of objects in addition to the boundry in
 three-space


To create an instance of class vtkParallelRenderManager, simply
invoke its constructor as follows
\begin{verbatim}
  obj = vtkParallelRenderManager
\end{verbatim}
\subsection{Methods}

The class vtkParallelRenderManager has several methods that can be used.
  They are listed below.
Note that the documentation is translated automatically from the VTK sources,
and may not be completely intelligible.  When in doubt, consult the VTK website.
In the methods listed below, \verb|obj| is an instance of the vtkParallelRenderManager class.
\begin{itemize}
\item  \verb|string = obj.GetClassName ()|

\item  \verb|int = obj.IsA (string name)|

\item  \verb|vtkParallelRenderManager = obj.NewInstance ()|

\item  \verb|vtkParallelRenderManager = obj.SafeDownCast (vtkObject o)|

\item  \verb|vtkRenderWindow = obj.MakeRenderWindow ()| -  Builds a vtkRenderWindow compatible with this render manager.  The
 user program is responsible for registering the render window with the
 SetRenderWindow method and calling Delete.  It is not advisable to use
 a parallel render manager with a render window that was not built with
 this method.

\item  \verb|vtkRenderer = obj.MakeRenderer ()| -  Builds a vtkRenderer compatible with this render manager.  (Should we
 also register it?)  The user program is responsible for calling
 Delete.  It is not advisable to use a parallel render manager with a
 renderer that was not built with this method.

\item  \verb|vtkRenderWindow = obj.GetRenderWindow ()| -  Set/Get the RenderWindow to use for compositing.
 We add a start and end observer to the window.

\item  \verb|obj.SetRenderWindow (vtkRenderWindow renWin)| -  Set/Get the RenderWindow to use for compositing.
 We add a start and end observer to the window.

\item  \verb|vtkMultiProcessController = obj.GetController ()| -  Set/Get the vtkMultiProcessController which will handle communications
 for the parallel rendering.

\item  \verb|obj.SetController (vtkMultiProcessController controller)| -  Set/Get the vtkMultiProcessController which will handle communications
 for the parallel rendering.

\item  \verb|obj.InitializePieces ()| -  This method sets the piece and number of pieces for each
 actor with a polydata mapper.

\item  \verb|obj.InitializeOffScreen ()| -  Make all rendering windows not viewable set as off screen rendering.
 To make all renderwindows on screen rendering again, call
 OffScreenRenderingOff on all the render windows.  This class assumes
 the window on root node is the only one viewable.  Subclasses should
 change this as necessary.

\item  \verb|obj.StartInteractor ()| -  Initializes the RMIs and then, if on root node, starts the interactor
 on the attached render window.  Otherwise, starts processing RMIs.
 When the interactor returns, it breaks the RMI listening on all other
 processors.

\item  \verb|obj.StartServices ()| -  If on node other than root, starts serving RMI requests for parallel
 renders.

\item  \verb|obj.StopServices ()| -  If on root node, stops the RMI processing on all service nodes.

\item  \verb|obj.StartRender ()| -  Callbacks that initialize and finish rendering and other tasks.

\item  \verb|obj.EndRender ()| -  Callbacks that initialize and finish rendering and other tasks.

\item  \verb|obj.SatelliteStartRender ()| -  Callbacks that initialize and finish rendering and other tasks.

\item  \verb|obj.SatelliteEndRender ()| -  Callbacks that initialize and finish rendering and other tasks.

\item  \verb|obj.RenderRMI ()| -  Callbacks that initialize and finish rendering and other tasks.

\item  \verb|obj.ResetCamera (vtkRenderer ren)| -  Callbacks that initialize and finish rendering and other tasks.

\item  \verb|obj.ResetCameraClippingRange (vtkRenderer ren)| -  Callbacks that initialize and finish rendering and other tasks.

\item  \verb|obj.ComputeVisiblePropBoundsRMI (int renderId)| -  Callbacks that initialize and finish rendering and other tasks.

\item  \verb|obj.InitializeRMIs ()|

\item  \verb|obj.ResetAllCameras ()| -  Resets the camera of each renderer contained in the RenderWindow.
 Should only be called in the ''root'' process, and all remote processes
 must be processing RMIs for this method to complete.

\item  \verb|obj.ComputeVisiblePropBounds (vtkRenderer ren, double bounds[6])| -  Calculates the bounds by gathering information from all processes.

\item  \verb|obj.SetParallelRendering (int )| -  Turns on/off parallel rendering.  When on (the default) the object
 responds to render events of the attached window, propagates the
 render event to other processors, and otherwise enables the parallel
 rendering process.

\item  \verb|int = obj.GetParallelRendering ()| -  Turns on/off parallel rendering.  When on (the default) the object
 responds to render events of the attached window, propagates the
 render event to other processors, and otherwise enables the parallel
 rendering process.

\item  \verb|obj.ParallelRenderingOn ()| -  Turns on/off parallel rendering.  When on (the default) the object
 responds to render events of the attached window, propagates the
 render event to other processors, and otherwise enables the parallel
 rendering process.

\item  \verb|obj.ParallelRenderingOff ()| -  Turns on/off parallel rendering.  When on (the default) the object
 responds to render events of the attached window, propagates the
 render event to other processors, and otherwise enables the parallel
 rendering process.

\item  \verb|obj.SetRenderEventPropagation (int )| -  Turns on/off render event propagation.  When on (the default) and
 ParallelRendering is on, process 0 will send an RMI call to all remote
 processes to perform a synchronized render.  When off, render must be
 manually called on each process.

\item  \verb|int = obj.GetRenderEventPropagation ()| -  Turns on/off render event propagation.  When on (the default) and
 ParallelRendering is on, process 0 will send an RMI call to all remote
 processes to perform a synchronized render.  When off, render must be
 manually called on each process.

\item  \verb|obj.RenderEventPropagationOn ()| -  Turns on/off render event propagation.  When on (the default) and
 ParallelRendering is on, process 0 will send an RMI call to all remote
 processes to perform a synchronized render.  When off, render must be
 manually called on each process.

\item  \verb|obj.RenderEventPropagationOff ()| -  Turns on/off render event propagation.  When on (the default) and
 ParallelRendering is on, process 0 will send an RMI call to all remote
 processes to perform a synchronized render.  When off, render must be
 manually called on each process.

\item  \verb|obj.SetUseCompositing (int )| -  This is used for tiled display rendering.  When data has been
 duplicated on all processes, then we do not need to compositing.
 Cameras and renders are still propagated though.

\item  \verb|int = obj.GetUseCompositing ()| -  This is used for tiled display rendering.  When data has been
 duplicated on all processes, then we do not need to compositing.
 Cameras and renders are still propagated though.

\item  \verb|obj.UseCompositingOn ()| -  This is used for tiled display rendering.  When data has been
 duplicated on all processes, then we do not need to compositing.
 Cameras and renders are still propagated though.

\item  \verb|obj.UseCompositingOff ()| -  This is used for tiled display rendering.  When data has been
 duplicated on all processes, then we do not need to compositing.
 Cameras and renders are still propagated though.

\item  \verb|obj.SetImageReductionFactor (double factor)| -  Set/Get the reduction factor (for sort-last based parallel renderers).
 The size of rendered image is divided by the reduction factor and then
 is blown up to the size of the current vtkRenderWindow.  Setting
 higher reduction factors enables shorter image transfer times (which
 is often the bottleneck) but will greatly reduce image quality.  A
 reduction factor of 2 or greater should only be used for intermediate
 images in interactive applications.  A reduction factor of 1 (or less)
 will result in no change in image quality.  A parallel render manager
 may ignore the image reduction factor if it will result in little or
 no performance enhancements (eg. it does not do image space
 manipulations).

\item  \verb|double = obj.GetImageReductionFactor ()| -  Set/Get the reduction factor (for sort-last based parallel renderers).
 The size of rendered image is divided by the reduction factor and then
 is blown up to the size of the current vtkRenderWindow.  Setting
 higher reduction factors enables shorter image transfer times (which
 is often the bottleneck) but will greatly reduce image quality.  A
 reduction factor of 2 or greater should only be used for intermediate
 images in interactive applications.  A reduction factor of 1 (or less)
 will result in no change in image quality.  A parallel render manager
 may ignore the image reduction factor if it will result in little or
 no performance enhancements (eg. it does not do image space
 manipulations).

\item  \verb|obj.SetMaxImageReductionFactor (double )|

\item  \verb|double = obj.GetMaxImageReductionFactor ()|

\item  \verb|obj.SetImageReductionFactorForUpdateRate (double DesiredUpdateRate)| -  Sets the ReductionFactor based on the given desired update rate and
 the rendering metrics taken from the last time UpdateServerInfo was
 called.  Note that if AutoReductionFactor is on, this function is called
 with the desired update rate of the render window automatically.

\item  \verb|obj.SetAutoImageReductionFactor (int )| -  If on, the ReductionFactor is automatically adjusted to best meet the
 the DesiredUpdateRate in the current RenderWindow based on metrics
 from the last render.

\item  \verb|int = obj.GetAutoImageReductionFactor ()| -  If on, the ReductionFactor is automatically adjusted to best meet the
 the DesiredUpdateRate in the current RenderWindow based on metrics
 from the last render.

\item  \verb|obj.AutoImageReductionFactorOn ()| -  If on, the ReductionFactor is automatically adjusted to best meet the
 the DesiredUpdateRate in the current RenderWindow based on metrics
 from the last render.

\item  \verb|obj.AutoImageReductionFactorOff ()| -  If on, the ReductionFactor is automatically adjusted to best meet the
 the DesiredUpdateRate in the current RenderWindow based on metrics
 from the last render.

\item  \verb|double = obj.GetRenderTime ()| -  Get rendering metrics.

\item  \verb|double = obj.GetImageProcessingTime ()| -  Get rendering metrics.

\item  \verb|int = obj.GetSyncRenderWindowRenderers ()| -  By default, the state of all renderers in the root's render window is
 propagated to the rest of the processes.  In order for this to work, all
 render windows must have the same renderers in the same order.  If this is
 not the case, you can turn off the SyncRenderWindowRenderers.  When this
 flag is off, the list of renderers held by this parallel render manager
 (initially empty) is synced.  You can modify the list of renderers with the
 AddRenderer, RemoveRenderer, and RemoveAllRenderers methods.

\item  \verb|obj.SetSyncRenderWindowRenderers (int )| -  By default, the state of all renderers in the root's render window is
 propagated to the rest of the processes.  In order for this to work, all
 render windows must have the same renderers in the same order.  If this is
 not the case, you can turn off the SyncRenderWindowRenderers.  When this
 flag is off, the list of renderers held by this parallel render manager
 (initially empty) is synced.  You can modify the list of renderers with the
 AddRenderer, RemoveRenderer, and RemoveAllRenderers methods.

\item  \verb|obj.SyncRenderWindowRenderersOn ()| -  By default, the state of all renderers in the root's render window is
 propagated to the rest of the processes.  In order for this to work, all
 render windows must have the same renderers in the same order.  If this is
 not the case, you can turn off the SyncRenderWindowRenderers.  When this
 flag is off, the list of renderers held by this parallel render manager
 (initially empty) is synced.  You can modify the list of renderers with the
 AddRenderer, RemoveRenderer, and RemoveAllRenderers methods.

\item  \verb|obj.SyncRenderWindowRenderersOff ()| -  By default, the state of all renderers in the root's render window is
 propagated to the rest of the processes.  In order for this to work, all
 render windows must have the same renderers in the same order.  If this is
 not the case, you can turn off the SyncRenderWindowRenderers.  When this
 flag is off, the list of renderers held by this parallel render manager
 (initially empty) is synced.  You can modify the list of renderers with the
 AddRenderer, RemoveRenderer, and RemoveAllRenderers methods.

\item  \verb|obj.AddRenderer (vtkRenderer )| -  By default, the state of all renderers in the root's render window is
 propagated to the rest of the processes.  In order for this to work, all
 render windows must have the same renderers in the same order.  If this is
 not the case, you can turn off the SyncRenderWindowRenderers.  When this
 flag is off, the list of renderers held by this parallel render manager
 (initially empty) is synced.  You can modify the list of renderers with the
 AddRenderer, RemoveRenderer, and RemoveAllRenderers methods.

\item  \verb|obj.RemoveRenderer (vtkRenderer )| -  By default, the state of all renderers in the root's render window is
 propagated to the rest of the processes.  In order for this to work, all
 render windows must have the same renderers in the same order.  If this is
 not the case, you can turn off the SyncRenderWindowRenderers.  When this
 flag is off, the list of renderers held by this parallel render manager
 (initially empty) is synced.  You can modify the list of renderers with the
 AddRenderer, RemoveRenderer, and RemoveAllRenderers methods.

\item  \verb|obj.RemoveAllRenderers ()| -  By default, the state of all renderers in the root's render window is
 propagated to the rest of the processes.  In order for this to work, all
 render windows must have the same renderers in the same order.  If this is
 not the case, you can turn off the SyncRenderWindowRenderers.  When this
 flag is off, the list of renderers held by this parallel render manager
 (initially empty) is synced.  You can modify the list of renderers with the
 AddRenderer, RemoveRenderer, and RemoveAllRenderers methods.

\item  \verb|obj.SetWriteBackImages (int )| -  If on (the default), the result of any image space manipulations are
 written back to the render window frame buffer.  If off, the image
 stored in the frame buffer may not be correct.  Either way, the
 correct frame buffer images may be read with
 vtkParallelRenderManager::GetPixelData.  Turning WriteBackImages off
 may result in a speedup if the render window is not visible to the user
 and images are read back for further processing or transit.

\item  \verb|int = obj.GetWriteBackImages ()| -  If on (the default), the result of any image space manipulations are
 written back to the render window frame buffer.  If off, the image
 stored in the frame buffer may not be correct.  Either way, the
 correct frame buffer images may be read with
 vtkParallelRenderManager::GetPixelData.  Turning WriteBackImages off
 may result in a speedup if the render window is not visible to the user
 and images are read back for further processing or transit.

\item  \verb|obj.WriteBackImagesOn ()| -  If on (the default), the result of any image space manipulations are
 written back to the render window frame buffer.  If off, the image
 stored in the frame buffer may not be correct.  Either way, the
 correct frame buffer images may be read with
 vtkParallelRenderManager::GetPixelData.  Turning WriteBackImages off
 may result in a speedup if the render window is not visible to the user
 and images are read back for further processing or transit.

\item  \verb|obj.WriteBackImagesOff ()| -  If on (the default), the result of any image space manipulations are
 written back to the render window frame buffer.  If off, the image
 stored in the frame buffer may not be correct.  Either way, the
 correct frame buffer images may be read with
 vtkParallelRenderManager::GetPixelData.  Turning WriteBackImages off
 may result in a speedup if the render window is not visible to the user
 and images are read back for further processing or transit.

\item  \verb|obj.SetMagnifyImages (int )| -  If on (the default), when the ImageReductionFactor is greater than 1
 and WriteBackImages is on, the image will be magnified to fill the
 entire render window.

\item  \verb|int = obj.GetMagnifyImages ()| -  If on (the default), when the ImageReductionFactor is greater than 1
 and WriteBackImages is on, the image will be magnified to fill the
 entire render window.

\item  \verb|obj.MagnifyImagesOn ()| -  If on (the default), when the ImageReductionFactor is greater than 1
 and WriteBackImages is on, the image will be magnified to fill the
 entire render window.

\item  \verb|obj.MagnifyImagesOff ()| -  If on (the default), when the ImageReductionFactor is greater than 1
 and WriteBackImages is on, the image will be magnified to fill the
 entire render window.

\item  \verb|obj.SetMagnifyImageMethod (int method)| -  Sets the method used to magnify images.  Nearest simply replicates
 each pixel enough times to fill the image.  Linear performs linear
 interpolation between the pixels.

\item  \verb|int = obj.GetMagnifyImageMethod ()| -  Sets the method used to magnify images.  Nearest simply replicates
 each pixel enough times to fill the image.  Linear performs linear
 interpolation between the pixels.

\item  \verb|obj.SetMagnifyImageMethodToNearest ()| -  Sets the method used to magnify images.  Nearest simply replicates
 each pixel enough times to fill the image.  Linear performs linear
 interpolation between the pixels.

\item  \verb|obj.SetMagnifyImageMethodToLinear ()| -  Convenience functions for magnifying images.

\item  \verb|obj.MagnifyImage (vtkUnsignedCharArray fullImage, int fullImageSize[2], vtkUnsignedCharArray reducedImage, int reducedImageSize[2], int fullImageViewport[4]NULL, int reducedImageViewport[4]NULL)| -  Convenience functions for magnifying images.

\item  \verb|obj.GetPixelData (vtkUnsignedCharArray data)| -  The most appropriate way to retrieve full size image data after a
 render.  Will work regardless of whether WriteBackImages or
 MagnifyImage is on or off.  The data returned may be a shallow copy of
 an internal array.  Therefore, the data may be invalid after the next
 render or if the ParallelRenderManager is destroyed.

\item  \verb|obj.GetPixelData (int x1, int y1, int x2, int y2, vtkUnsignedCharArray data)| -  The most appropriate way to retrieve full size image data after a
 render.  Will work regardless of whether WriteBackImages or
 MagnifyImage is on or off.  The data returned may be a shallow copy of
 an internal array.  Therefore, the data may be invalid after the next
 render or if the ParallelRenderManager is destroyed.

\item  \verb|obj.GetReducedPixelData (vtkUnsignedCharArray data)| -  The most appropriate way to retrieve reduced size image data after a
 render.  Will work regardless of whether WriteBackImages or
 MagnifyImage is on or off.  The data returned may be a shallow copy of
 an internal array.  Therefore, the data may be invalid after the next
 render or if the ParallelRenderManager is destroyed.

\item  \verb|obj.GetReducedPixelData (int x1, int y1, int x2, int y2, vtkUnsignedCharArray data)| -  The most appropriate way to retrieve reduced size image data after a
 render.  Will work regardless of whether WriteBackImages or
 MagnifyImage is on or off.  The data returned may be a shallow copy of
 an internal array.  Therefore, the data may be invalid after the next
 render or if the ParallelRenderManager is destroyed.

\item  \verb|int = obj. GetFullImageSize ()| -  Returns the full image size calculated at the last render.

\item  \verb|int = obj. GetReducedImageSize ()| -  Returns the reduced image size calculated at the last render.

\item  \verb|obj.TileWindows (int xsize, int ysize, int nColumns)| -  Given the x and y size of the render windows, reposition them
 in a tile of n columns.

\item  \verb|obj.SetUseRGBA (int )| -  Get/Set if all Images must use RGBA instead of RGB. By default,
 this flag is on.

\item  \verb|int = obj.GetUseRGBA ()| -  Get/Set if all Images must use RGBA instead of RGB. By default,
 this flag is on.

\item  \verb|obj.SetForceRenderWindowSize (int )| -  If ForceRenderWindowSize is set to true, the render manager will use
 the RenderWindowSize ivar instead of getting the size from the render window.

\item  \verb|int = obj.GetForceRenderWindowSize ()| -  If ForceRenderWindowSize is set to true, the render manager will use
 the RenderWindowSize ivar instead of getting the size from the render window.

\item  \verb|obj.SetForcedRenderWindowSize (int , int )| -  If ForceRenderWindowSize is set to true, the render manager will use
 the Size ivar instead of getting the size from the render window.

\item  \verb|obj.SetForcedRenderWindowSize (int  a[2])| -  If ForceRenderWindowSize is set to true, the render manager will use
 the Size ivar instead of getting the size from the render window.

\item  \verb|int = obj. GetForcedRenderWindowSize ()| -  If ForceRenderWindowSize is set to true, the render manager will use
 the Size ivar instead of getting the size from the render window.

\item  \verb|obj.StartService ()| -  @deprecated Replaced by vtkParallelRenderManager::StartServices()
 as of VTK 5.0.

\item  \verb|obj.SetUseBackBuffer (int )|

\item  \verb|int = obj.GetUseBackBuffer ()|

\item  \verb|obj.UseBackBufferOn ()|

\item  \verb|obj.UseBackBufferOff ()|

\item  \verb|obj.SetSynchronizeTileProperties (int )| -  When set the render manager will synchronize the TileViewport and TileScale
 properties. This may not be desirable in cases where there's some other
 mechanism to set the tile dimensions eg. Tile displays.

\item  \verb|int = obj.GetSynchronizeTileProperties ()| -  When set the render manager will synchronize the TileViewport and TileScale
 properties. This may not be desirable in cases where there's some other
 mechanism to set the tile dimensions eg. Tile displays.

\item  \verb|obj.SynchronizeTilePropertiesOn ()| -  When set the render manager will synchronize the TileViewport and TileScale
 properties. This may not be desirable in cases where there's some other
 mechanism to set the tile dimensions eg. Tile displays.

\item  \verb|obj.SynchronizeTilePropertiesOff ()| -  When set the render manager will synchronize the TileViewport and TileScale
 properties. This may not be desirable in cases where there's some other
 mechanism to set the tile dimensions eg. Tile displays.

\item  \verb|obj.GenericStartRenderCallback ()| -  INTERNAL METHODS (DON NOT USE).
 There are internal methods made public so that they can be called from
 callback functions.

\item  \verb|obj.GenericEndRenderCallback ()| -  INTERNAL METHODS (DON NOT USE).
 There are internal methods made public so that they can be called from
 callback functions.

\end{itemize}
