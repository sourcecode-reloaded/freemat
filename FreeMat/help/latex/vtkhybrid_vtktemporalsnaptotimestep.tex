\section{vtkTemporalSnapToTimeStep}

\subsection{Usage}

 vtkTemporalSnapToTimeStep  modify the time range or time steps of
 the data without changing the data itself. The data is not resampled
 by this filter, only the information accompanying the data is modified.

To create an instance of class vtkTemporalSnapToTimeStep, simply
invoke its constructor as follows
\begin{verbatim}
  obj = vtkTemporalSnapToTimeStep
\end{verbatim}
\subsection{Methods}

The class vtkTemporalSnapToTimeStep has several methods that can be used.
  They are listed below.
Note that the documentation is translated automatically from the VTK sources,
and may not be completely intelligible.  When in doubt, consult the VTK website.
In the methods listed below, \verb|obj| is an instance of the vtkTemporalSnapToTimeStep class.
\begin{itemize}
\item  \verb|string = obj.GetClassName ()|

\item  \verb|int = obj.IsA (string name)|

\item  \verb|vtkTemporalSnapToTimeStep = obj.NewInstance ()|

\item  \verb|vtkTemporalSnapToTimeStep = obj.SafeDownCast (vtkObject o)|

\item  \verb|obj.SetSnapMode (int )|

\item  \verb|int = obj.GetSnapMode ()|

\item  \verb|obj.SetSnapModeToNearest ()|

\item  \verb|obj.SetSnapModeToNextBelowOrEqual ()|

\item  \verb|obj.SetSnapModeToNextAboveOrEqual ()|

\end{itemize}
