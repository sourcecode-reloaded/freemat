\section{vtkCenteredSliderRepresentation}

\subsection{Usage}

 This class is used to represent and render a vtkCenteredSliderWidget. To use this
 class, you must at a minimum specify the end points of the
 slider. Optional instance variable can be used to modify the appearance of
 the widget.


To create an instance of class vtkCenteredSliderRepresentation, simply
invoke its constructor as follows
\begin{verbatim}
  obj = vtkCenteredSliderRepresentation
\end{verbatim}
\subsection{Methods}

The class vtkCenteredSliderRepresentation has several methods that can be used.
  They are listed below.
Note that the documentation is translated automatically from the VTK sources,
and may not be completely intelligible.  When in doubt, consult the VTK website.
In the methods listed below, \verb|obj| is an instance of the vtkCenteredSliderRepresentation class.
\begin{itemize}
\item  \verb|string = obj.GetClassName ()| -  Standard methods for the class.

\item  \verb|int = obj.IsA (string name)| -  Standard methods for the class.

\item  \verb|vtkCenteredSliderRepresentation = obj.NewInstance ()| -  Standard methods for the class.

\item  \verb|vtkCenteredSliderRepresentation = obj.SafeDownCast (vtkObject o)| -  Standard methods for the class.

\item  \verb|vtkCoordinate = obj.GetPoint1Coordinate ()| -  Position the first end point of the slider. Note that this point is an
 instance of vtkCoordinate, meaning that Point 1 can be specified in a
 variety of coordinate systems, and can even be relative to another
 point. To set the point, you'll want to get the Point1Coordinate and
 then invoke the necessary methods to put it into the correct coordinate
 system and set the correct initial value.

\item  \verb|vtkCoordinate = obj.GetPoint2Coordinate ()| -  Position the second end point of the slider. Note that this point is an
 instance of vtkCoordinate, meaning that Point 1 can be specified in a
 variety of coordinate systems, and can even be relative to another
 point. To set the point, you'll want to get the Point2Coordinate and
 then invoke the necessary methods to put it into the correct coordinate
 system and set the correct initial value.

\item  \verb|obj.SetTitleText (string )| -  Specify the label text for this widget. If the value is not set, or set
 to the empty string '''', then the label text is not displayed.

\item  \verb|string = obj.GetTitleText ()| -  Specify the label text for this widget. If the value is not set, or set
 to the empty string '''', then the label text is not displayed.

\item  \verb|vtkProperty2D = obj.GetTubeProperty ()| -  Get the properties for the tube and slider

\item  \verb|vtkProperty2D = obj.GetSliderProperty ()| -  Get the properties for the tube and slider

\item  \verb|vtkProperty2D = obj.GetSelectedProperty ()| -  Get the selection property. This property is used to modify the
 appearance of selected objects (e.g., the slider).

\item  \verb|vtkTextProperty = obj.GetLabelProperty ()| -  Set/Get the properties for the label and title text.

\item  \verb|obj.PlaceWidget (double bounds[6])| -  Methods to interface with the vtkSliderWidget. The PlaceWidget() method
 assumes that the parameter bounds[6] specifies the location in display
 space where the widget should be placed.

\item  \verb|obj.BuildRepresentation ()| -  Methods to interface with the vtkSliderWidget. The PlaceWidget() method
 assumes that the parameter bounds[6] specifies the location in display
 space where the widget should be placed.

\item  \verb|obj.StartWidgetInteraction (double eventPos[2])| -  Methods to interface with the vtkSliderWidget. The PlaceWidget() method
 assumes that the parameter bounds[6] specifies the location in display
 space where the widget should be placed.

\item  \verb|int = obj.ComputeInteractionState (int X, int Y, int modify)| -  Methods to interface with the vtkSliderWidget. The PlaceWidget() method
 assumes that the parameter bounds[6] specifies the location in display
 space where the widget should be placed.

\item  \verb|obj.WidgetInteraction (double eventPos[2])| -  Methods to interface with the vtkSliderWidget. The PlaceWidget() method
 assumes that the parameter bounds[6] specifies the location in display
 space where the widget should be placed.

\item  \verb|obj.Highlight (int )| -  Methods to interface with the vtkSliderWidget. The PlaceWidget() method
 assumes that the parameter bounds[6] specifies the location in display
 space where the widget should be placed.

\item  \verb|obj.GetActors (vtkPropCollection )|

\item  \verb|obj.ReleaseGraphicsResources (vtkWindow )|

\item  \verb|int = obj.RenderOverlay (vtkViewport )|

\item  \verb|int = obj.RenderOpaqueGeometry (vtkViewport )|

\end{itemize}
