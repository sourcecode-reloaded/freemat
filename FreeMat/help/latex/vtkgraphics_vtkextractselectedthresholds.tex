\section{vtkExtractSelectedThresholds}

\subsection{Usage}

 vtkExtractSelectedThresholds extracts all cells and points with attribute 
 values that lie within a vtkSelection's THRESHOLD contents. The selecion
 can specify to threshold a particular array within either the point or cell
 attribute data of the input. This is similar to vtkThreshold
 but allows mutliple thresholds ranges.
 This filter adds a scalar array called vtkOriginalCellIds that says what 
 input cell produced each output cell. This is an example of a Pedigree ID 
 which helps to trace back results.

To create an instance of class vtkExtractSelectedThresholds, simply
invoke its constructor as follows
\begin{verbatim}
  obj = vtkExtractSelectedThresholds
\end{verbatim}
\subsection{Methods}

The class vtkExtractSelectedThresholds has several methods that can be used.
  They are listed below.
Note that the documentation is translated automatically from the VTK sources,
and may not be completely intelligible.  When in doubt, consult the VTK website.
In the methods listed below, \verb|obj| is an instance of the vtkExtractSelectedThresholds class.
\begin{itemize}
\item  \verb|string = obj.GetClassName ()|

\item  \verb|int = obj.IsA (string name)|

\item  \verb|vtkExtractSelectedThresholds = obj.NewInstance ()|

\item  \verb|vtkExtractSelectedThresholds = obj.SafeDownCast (vtkObject o)|

\end{itemize}
