\section{vtkExtractVOI}

\subsection{Usage}

 vtkExtractVOI is a filter that selects a portion of an input structured
 points dataset, or subsamples an input dataset. (The selected portion of
 interested is referred to as the Volume Of Interest, or VOI.) The output of
 this filter is a structured points dataset. The filter treats input data
 of any topological dimension (i.e., point, line, image, or volume) and can
 generate output data of any topological dimension.

 To use this filter set the VOI ivar which are i-j-k min/max indices that
 specify a rectangular region in the data. (Note that these are 0-offset.)
 You can also specify a sampling rate to subsample the data.

 Typical applications of this filter are to extract a slice from a volume
 for image processing, subsampling large volumes to reduce data size, or
 extracting regions of a volume with interesting data.

To create an instance of class vtkExtractVOI, simply
invoke its constructor as follows
\begin{verbatim}
  obj = vtkExtractVOI
\end{verbatim}
\subsection{Methods}

The class vtkExtractVOI has several methods that can be used.
  They are listed below.
Note that the documentation is translated automatically from the VTK sources,
and may not be completely intelligible.  When in doubt, consult the VTK website.
In the methods listed below, \verb|obj| is an instance of the vtkExtractVOI class.
\begin{itemize}
\item  \verb|string = obj.GetClassName ()|

\item  \verb|int = obj.IsA (string name)|

\item  \verb|vtkExtractVOI = obj.NewInstance ()|

\item  \verb|vtkExtractVOI = obj.SafeDownCast (vtkObject o)|

\item  \verb|obj.SetVOI (int , int , int , int , int , int )| -  Specify i-j-k (min,max) pairs to extract. The resulting structured points
 dataset can be of any topological dimension (i.e., point, line, image, 
 or volume). 

\item  \verb|obj.SetVOI (int  a[6])| -  Specify i-j-k (min,max) pairs to extract. The resulting structured points
 dataset can be of any topological dimension (i.e., point, line, image, 
 or volume). 

\item  \verb|int = obj. GetVOI ()| -  Specify i-j-k (min,max) pairs to extract. The resulting structured points
 dataset can be of any topological dimension (i.e., point, line, image, 
 or volume). 

\item  \verb|obj.SetSampleRate (int , int , int )| -  Set the sampling rate in the i, j, and k directions. If the rate is >
 1, then the resulting VOI will be subsampled representation of the
 input.  For example, if the SampleRate=(2,2,2), every other point will
 be selected, resulting in a volume 1/8th the original size.

\item  \verb|obj.SetSampleRate (int  a[3])| -  Set the sampling rate in the i, j, and k directions. If the rate is >
 1, then the resulting VOI will be subsampled representation of the
 input.  For example, if the SampleRate=(2,2,2), every other point will
 be selected, resulting in a volume 1/8th the original size.

\item  \verb|int = obj. GetSampleRate ()| -  Set the sampling rate in the i, j, and k directions. If the rate is >
 1, then the resulting VOI will be subsampled representation of the
 input.  For example, if the SampleRate=(2,2,2), every other point will
 be selected, resulting in a volume 1/8th the original size.

\end{itemize}
