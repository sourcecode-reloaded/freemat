\section{VARARGOUT Variable Output Arguments}

\subsection{Usage}

FreeMat functions can return a variable number of output arguments
by setting the last argument in the argument list to \verb|varargout|.
This special keyword indicates that the number of return values
is variable.  The general syntax for a function that returns
a variable number of outputs is
\begin{verbatim}
  function [out_1,...,out_M,varargout] = fname(in_1,...,in_M)
\end{verbatim}
The function is responsible for ensuring that \verb|varargout| is
a cell array that contains the values to assign to the outputs
beyond \verb|out_M|.  Generally, variable output functions use
\verb|nargout| to figure out how many outputs have been requested.
\subsection{Example}

This is a function that returns a varying number of values
depending on the value of the argument.
\begin{verbatim}
    varoutfunc.m
function [varargout] = varoutfunc
  switch(nargout)
    case 1
      varargout = {'one of one'};
    case 2
      varargout = {'one of two','two of two'};
    case 3
      varargout = {'one of three','two of three','three of three'};
  end
\end{verbatim}
Here are some examples of exercising \verb|varoutfunc|:
\begin{verbatim}
--> [c1] = varoutfunc
c1 = 
one of one
--> [c1,c2] = varoutfunc
c1 = 
one of two
c2 = 
two of two
--> [c1,c2,c3] = varoutfunc
c1 = 
one of three
c2 = 
two of three
c3 = 
three of three
\end{verbatim}
