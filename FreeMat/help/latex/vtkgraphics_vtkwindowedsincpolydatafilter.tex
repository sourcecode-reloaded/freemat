\section{vtkWindowedSincPolyDataFilter}

\subsection{Usage}

 vtkWindowedSincPolyDataFiler adjust point coordinate using a windowed
 sinc function interpolation kernel.  The effect is to ''relax'' the mesh,
 making the cells better shaped and the vertices more evenly distributed.
 Note that this filter operates the lines, polygons, and triangle strips
 composing an instance of vtkPolyData.  Vertex or poly-vertex cells are
 never modified.

 The algorithm proceeds as follows. For each vertex v, a topological and
 geometric analysis is performed to determine which vertices are connected
 to v, and which cells are connected to v. Then, a connectivity array is
 constructed for each vertex. (The connectivity array is a list of lists
 of vertices that directly attach to each vertex.) Next, an iteration
 phase begins over all vertices. For each vertex v, the coordinates of v
 are modified using a windowed sinc function interpolation kernel.
 Taubin describes this methodology is the IBM tech report RC-20404
 (\#90237, dated 3/12/96) ''Optimal Surface Smoothing as Filter Design''
 G. Taubin, T. Zhang and G. Golub. (Zhang and Golub are at Stanford
 University).

 This report discusses using standard signal processing low-pass filters
 (in particular windowed sinc functions) to smooth polyhedra. The
 transfer functions of the low-pass filters are approximated by
 Chebyshev polynomials. This facilitates applying the filters in an
 iterative diffusion process (as opposed to a kernel convolution).  The
 more smoothing iterations applied, the higher the degree of polynomial
 approximating the low-pass filter transfer function. Each smoothing
 iteration, therefore, applies the next higher term of the Chebyshev
 filter approximation to the polyhedron. This decoupling of the filter
 into an iteratively applied polynomial is possible since the Chebyshev
 polynomials are orthogonal, i.e. increasing the order of the
 approximation to the filter transfer function does not alter the
 previously calculated coefficients for the low order terms. 

 Note: Care must be taken to avoid smoothing with too few iterations.
 A Chebyshev approximation with too few terms is an poor approximation.
 The first few smoothing iterations represent a severe scaling and
 translation of the data.  Subsequent iterations cause the smoothed
 polyhedron to converge to the true location and scale of the object.
 We have attempted to protect against this by automatically adjusting
 the filter, effectively widening the pass band. This adjustment is only
 possible if the number of iterations is greater than 1.  Note that this
 sacrifices some degree of smoothing for model integrity. For those
 interested, the filter is adjusted by searching for a value sigma
 such that the actual pass band is k\_pb + sigma and such that the
 filter transfer function evaluates to unity at k\_pb, i.e. f(k\_pb) = 1

 To improve the numerical stability of the solution and minimize the
 scaling the translation effects, the algorithm can translate and
 scale the position coordinates to within the unit cube [-1, 1],
 perform the smoothing, and translate and scale the position
 coordinates back to the original coordinate frame.  This mode is
 controlled with the NormalizeCoordinatesOn() /
 NormalizeCoordinatesOff() methods.  For legacy reasons, the default
 is NormalizeCoordinatesOff.

 This implementation is currently limited to using an interpolation
 kernel based on Hamming windows.  Other windows (such as Hann, Blackman,
 Kaiser, Lanczos, Gaussian, and exponential windows) could be used
 instead.

 There are some special instance variables used to control the execution
 of this filter. (These ivars basically control what vertices can be
 smoothed, and the creation of the connectivity array.) The
 BoundarySmoothing ivar enables/disables the smoothing operation on
 vertices that are on the ''boundary'' of the mesh. A boundary vertex is one
 that is surrounded by a semi-cycle of polygons (or used by a single
 line).
 
 Another important ivar is FeatureEdgeSmoothing. If this ivar is
 enabled, then interior vertices are classified as either ''simple'',
 ''interior edge'', or ''fixed'', and smoothed differently. (Interior
 vertices are manifold vertices surrounded by a cycle of polygons; or used
 by two line cells.) The classification is based on the number of feature 
 edges attached to v. A feature edge occurs when the angle between the two
 surface normals of a polygon sharing an edge is greater than the
 FeatureAngle ivar. Then, vertices used by no feature edges are classified
 ''simple'', vertices used by exactly two feature edges are classified
 ''interior edge'', and all others are ''fixed'' vertices.

 Once the classification is known, the vertices are smoothed
 differently. Corner (i.e., fixed) vertices are not smoothed at all. 
 Simple vertices are smoothed as before . Interior edge vertices are
 smoothed only along their two connected edges, and only if the angle
 between the edges is less than the EdgeAngle ivar.

 The total smoothing can be controlled by using two ivars. The 
 NumberOfIterations determines the maximum number of smoothing passes.
 The NumberOfIterations corresponds to the degree of the polynomial that
 is used to approximate the windowed sinc function. Ten or twenty
 iterations is all the is usually necessary. Contrast this with
 vtkSmoothPolyDataFilter which usually requires 100 to 200 smoothing
 iterations. vtkSmoothPolyDataFilter is also not an approximation to
 an ideal low-pass filter, which can cause the geometry to shrink as the
 amount of smoothing increases.

 The second ivar is the specification of the PassBand for the windowed
 sinc filter.  By design, the PassBand is specified as a doubleing point
 number between 0 and 2.  Lower PassBand values produce more smoothing.
 A good default value for the PassBand is 0.1 (for those interested, the
 PassBand (and frequencies) for PolyData are based on the valence of the
 vertices, this limits all the frequency modes in a polyhedral mesh to
 between 0 and 2.)

 There are two instance variables that control the generation of error
 data. If the ivar GenerateErrorScalars is on, then a scalar value indicating
 the distance of each vertex from its original position is computed. If the
 ivar GenerateErrorVectors is on, then a vector representing change in 
 position is computed.


To create an instance of class vtkWindowedSincPolyDataFilter, simply
invoke its constructor as follows
\begin{verbatim}
  obj = vtkWindowedSincPolyDataFilter
\end{verbatim}
\subsection{Methods}

The class vtkWindowedSincPolyDataFilter has several methods that can be used.
  They are listed below.
Note that the documentation is translated automatically from the VTK sources,
and may not be completely intelligible.  When in doubt, consult the VTK website.
In the methods listed below, \verb|obj| is an instance of the vtkWindowedSincPolyDataFilter class.
\begin{itemize}
\item  \verb|string = obj.GetClassName ()|

\item  \verb|int = obj.IsA (string name)|

\item  \verb|vtkWindowedSincPolyDataFilter = obj.NewInstance ()|

\item  \verb|vtkWindowedSincPolyDataFilter = obj.SafeDownCast (vtkObject o)|

\item  \verb|obj.SetNumberOfIterations (int )| -  Specify the number of iterations (or degree of the polynomial
 approximating the windowed sinc function).

\item  \verb|int = obj.GetNumberOfIterationsMinValue ()| -  Specify the number of iterations (or degree of the polynomial
 approximating the windowed sinc function).

\item  \verb|int = obj.GetNumberOfIterationsMaxValue ()| -  Specify the number of iterations (or degree of the polynomial
 approximating the windowed sinc function).

\item  \verb|int = obj.GetNumberOfIterations ()| -  Specify the number of iterations (or degree of the polynomial
 approximating the windowed sinc function).

\item  \verb|obj.SetPassBand (double )| -  Set the passband value for the windowed sinc filter

\item  \verb|double = obj.GetPassBandMinValue ()| -  Set the passband value for the windowed sinc filter

\item  \verb|double = obj.GetPassBandMaxValue ()| -  Set the passband value for the windowed sinc filter

\item  \verb|double = obj.GetPassBand ()| -  Set the passband value for the windowed sinc filter

\item  \verb|obj.SetNormalizeCoordinates (int )| -  Turn on/off coordinate normalization.  The positions can be
 translated and scaled such that they fit within a [-1, 1] prior
 to the smoothing computation. The default is off.  The numerical
 stability of the solution can be improved by turning
 normalization on.  If normalization is on, the coordinates will
 be rescaled to the original coordinate system after smoothing has
 completed.

\item  \verb|int = obj.GetNormalizeCoordinates ()| -  Turn on/off coordinate normalization.  The positions can be
 translated and scaled such that they fit within a [-1, 1] prior
 to the smoothing computation. The default is off.  The numerical
 stability of the solution can be improved by turning
 normalization on.  If normalization is on, the coordinates will
 be rescaled to the original coordinate system after smoothing has
 completed.

\item  \verb|obj.NormalizeCoordinatesOn ()| -  Turn on/off coordinate normalization.  The positions can be
 translated and scaled such that they fit within a [-1, 1] prior
 to the smoothing computation. The default is off.  The numerical
 stability of the solution can be improved by turning
 normalization on.  If normalization is on, the coordinates will
 be rescaled to the original coordinate system after smoothing has
 completed.

\item  \verb|obj.NormalizeCoordinatesOff ()| -  Turn on/off coordinate normalization.  The positions can be
 translated and scaled such that they fit within a [-1, 1] prior
 to the smoothing computation. The default is off.  The numerical
 stability of the solution can be improved by turning
 normalization on.  If normalization is on, the coordinates will
 be rescaled to the original coordinate system after smoothing has
 completed.

\item  \verb|obj.SetFeatureEdgeSmoothing (int )| -  Turn on/off smoothing along sharp interior edges.

\item  \verb|int = obj.GetFeatureEdgeSmoothing ()| -  Turn on/off smoothing along sharp interior edges.

\item  \verb|obj.FeatureEdgeSmoothingOn ()| -  Turn on/off smoothing along sharp interior edges.

\item  \verb|obj.FeatureEdgeSmoothingOff ()| -  Turn on/off smoothing along sharp interior edges.

\item  \verb|obj.SetFeatureAngle (double )| -  Specify the feature angle for sharp edge identification.

\item  \verb|double = obj.GetFeatureAngleMinValue ()| -  Specify the feature angle for sharp edge identification.

\item  \verb|double = obj.GetFeatureAngleMaxValue ()| -  Specify the feature angle for sharp edge identification.

\item  \verb|double = obj.GetFeatureAngle ()| -  Specify the feature angle for sharp edge identification.

\item  \verb|obj.SetEdgeAngle (double )| -  Specify the edge angle to control smoothing along edges (either interior
 or boundary).

\item  \verb|double = obj.GetEdgeAngleMinValue ()| -  Specify the edge angle to control smoothing along edges (either interior
 or boundary).

\item  \verb|double = obj.GetEdgeAngleMaxValue ()| -  Specify the edge angle to control smoothing along edges (either interior
 or boundary).

\item  \verb|double = obj.GetEdgeAngle ()| -  Specify the edge angle to control smoothing along edges (either interior
 or boundary).

\item  \verb|obj.SetBoundarySmoothing (int )| -  Turn on/off the smoothing of vertices on the boundary of the mesh.

\item  \verb|int = obj.GetBoundarySmoothing ()| -  Turn on/off the smoothing of vertices on the boundary of the mesh.

\item  \verb|obj.BoundarySmoothingOn ()| -  Turn on/off the smoothing of vertices on the boundary of the mesh.

\item  \verb|obj.BoundarySmoothingOff ()| -  Turn on/off the smoothing of vertices on the boundary of the mesh.

\item  \verb|obj.SetNonManifoldSmoothing (int )| -  Smooth non-manifold vertices.

\item  \verb|int = obj.GetNonManifoldSmoothing ()| -  Smooth non-manifold vertices.

\item  \verb|obj.NonManifoldSmoothingOn ()| -  Smooth non-manifold vertices.

\item  \verb|obj.NonManifoldSmoothingOff ()| -  Smooth non-manifold vertices.

\item  \verb|obj.SetGenerateErrorScalars (int )| -  Turn on/off the generation of scalar distance values.

\item  \verb|int = obj.GetGenerateErrorScalars ()| -  Turn on/off the generation of scalar distance values.

\item  \verb|obj.GenerateErrorScalarsOn ()| -  Turn on/off the generation of scalar distance values.

\item  \verb|obj.GenerateErrorScalarsOff ()| -  Turn on/off the generation of scalar distance values.

\item  \verb|obj.SetGenerateErrorVectors (int )| -  Turn on/off the generation of error vectors.

\item  \verb|int = obj.GetGenerateErrorVectors ()| -  Turn on/off the generation of error vectors.

\item  \verb|obj.GenerateErrorVectorsOn ()| -  Turn on/off the generation of error vectors.

\item  \verb|obj.GenerateErrorVectorsOff ()| -  Turn on/off the generation of error vectors.

\end{itemize}
