\section{vtkRotationFilter}

\subsection{Usage}

 The vtkRotationFilter duplicates a data set by rotation about one of the
 3 axis of the dataset's reference.
 Since it converts data sets into unstructured grids, it is not efficient
 for structured data sets.

 .SECTION Thanks
 Theophane Foggia of The Swiss National Supercomputing Centre (CSCS)
 for creating and contributing this filter

To create an instance of class vtkRotationFilter, simply
invoke its constructor as follows
\begin{verbatim}
  obj = vtkRotationFilter
\end{verbatim}
\subsection{Methods}

The class vtkRotationFilter has several methods that can be used.
  They are listed below.
Note that the documentation is translated automatically from the VTK sources,
and may not be completely intelligible.  When in doubt, consult the VTK website.
In the methods listed below, \verb|obj| is an instance of the vtkRotationFilter class.
\begin{itemize}
\item  \verb|string = obj.GetClassName ()|

\item  \verb|int = obj.IsA (string name)|

\item  \verb|vtkRotationFilter = obj.NewInstance ()|

\item  \verb|vtkRotationFilter = obj.SafeDownCast (vtkObject o)|

\item  \verb|obj.SetAxis (int )| -  Set the axis of rotation to use. It is set by default to Z.

\item  \verb|int = obj.GetAxisMinValue ()| -  Set the axis of rotation to use. It is set by default to Z.

\item  \verb|int = obj.GetAxisMaxValue ()| -  Set the axis of rotation to use. It is set by default to Z.

\item  \verb|int = obj.GetAxis ()| -  Set the axis of rotation to use. It is set by default to Z.

\item  \verb|obj.SetAxisToX ()| -  Set the axis of rotation to use. It is set by default to Z.

\item  \verb|obj.SetAxisToY ()| -  Set the axis of rotation to use. It is set by default to Z.

\item  \verb|obj.SetAxisToZ ()| -  Set the axis of rotation to use. It is set by default to Z.

\item  \verb|obj.SetAngle (double )| -  Set the rotation angle to use.

\item  \verb|double = obj.GetAngle ()| -  Set the rotation angle to use.

\item  \verb|obj.SetCenter (double , double , double )| -  Set the rotation center coordinates.

\item  \verb|obj.SetCenter (double  a[3])| -  Set the rotation center coordinates.

\item  \verb|double = obj. GetCenter ()| -  Set the rotation center coordinates.

\item  \verb|obj.SetNumberOfCopies (int )| -  Set the number of copies to create. The source will be rotated N times
 and a new polydata copy of the original created at each angular position
 All copies will be appended to form a single output

\item  \verb|int = obj.GetNumberOfCopies ()| -  Set the number of copies to create. The source will be rotated N times
 and a new polydata copy of the original created at each angular position
 All copies will be appended to form a single output

\item  \verb|obj.SetCopyInput (int )| -  If on (the default), copy the input geometry to the output. If off,
 the output will only contain the rotation.

\item  \verb|int = obj.GetCopyInput ()| -  If on (the default), copy the input geometry to the output. If off,
 the output will only contain the rotation.

\item  \verb|obj.CopyInputOn ()| -  If on (the default), copy the input geometry to the output. If off,
 the output will only contain the rotation.

\item  \verb|obj.CopyInputOff ()| -  If on (the default), copy the input geometry to the output. If off,
 the output will only contain the rotation.

\end{itemize}
