\section{vtkGeoAlignedImageRepresentation}

\subsection{Usage}

 vtkGeoAlignedImageRepresentation represents a high resolution image
 over the globle. It has an associated vtkGeoSource which is responsible
 for fetching new data. This class keeps the fetched data in a quad-tree
 structure organized by latitude and longitude.

To create an instance of class vtkGeoAlignedImageRepresentation, simply
invoke its constructor as follows
\begin{verbatim}
  obj = vtkGeoAlignedImageRepresentation
\end{verbatim}
\subsection{Methods}

The class vtkGeoAlignedImageRepresentation has several methods that can be used.
  They are listed below.
Note that the documentation is translated automatically from the VTK sources,
and may not be completely intelligible.  When in doubt, consult the VTK website.
In the methods listed below, \verb|obj| is an instance of the vtkGeoAlignedImageRepresentation class.
\begin{itemize}
\item  \verb|string = obj.GetClassName ()|

\item  \verb|int = obj.IsA (string name)|

\item  \verb|vtkGeoAlignedImageRepresentation = obj.NewInstance ()|

\item  \verb|vtkGeoAlignedImageRepresentation = obj.SafeDownCast (vtkObject o)|

\item  \verb|vtkGeoImageNode = obj.GetBestImageForBounds (double bounds[4])| -  Retrieve the most refined image patch that covers the specified
 latitude and longitude bounds (lat-min, lat-max, long-min, long-max).

\item  \verb|vtkGeoSource = obj.GetSource ()| -  The source for this representation. This must be set first before
 calling GetBestImageForBounds.

\item  \verb|obj.SetSource (vtkGeoSource source)| -  The source for this representation. This must be set first before
 calling GetBestImageForBounds.

\item  \verb|obj.SaveDatabase (string path)| -  Serialize the database to the specified directory.
 Each image is stored as a .vti file.
 The Origin and Spacing of the saved image contain (lat-min, long-min)
 and (lat-max, long-max), respectively.
 Files are named based on their level and id within that level.

\end{itemize}
