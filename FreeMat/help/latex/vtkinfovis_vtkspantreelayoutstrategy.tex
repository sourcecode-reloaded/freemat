\section{vtkSpanTreeLayoutStrategy}

\subsection{Usage}

 vtkSpanTreeLayout is a strategy for drawing directed graphs that
 works by first extracting a spanning tree (more accurately, a 
 spanning forest), and using this both to position graph vertices
 and to plan the placement of non-tree edges.  The latter are drawn
 with the aid of edge points to produce a tidy drawing.

 The approach is best suited to ''quasi-trees'', graphs where the number
 of edges is of the same order as the number of nodes; it is less well
 suited to denser graphs.  The boolean flag DepthFirstSpanningTree
 determines whether a depth-first or breadth-first strategy is used to
 construct the underlying forest, and the choice of strategy affects
 the output layout significantly.  Informal experiments suggest that
 the breadth-first strategy is better for denser graphs.

 Different layouts could also be produced by plugging in alternative
 tree layout strategies.  To work with the method of routing non-tree
 edges, any strategy should draw a tree so that levels are equally
 spaced along the z-axis, precluding for example the use of a radial
 or balloon layout.

 vtkSpanTreeLayout is based on an approach to 3D graph layout first 
 developed as part of the ''tulip'' tool by Dr. David Auber at LaBRI, 
 U.Bordeaux: see www.tulip-software.org   

 This implementation departs from the original version in that:
 (a) it is reconstructed to use Titan/VTK data structures;
 (b) it uses a faster method for dealing with non-tree edges,
     requiring at most two edge points per edge
 (c) allows for plugging in different tree layout methods
 (d) allows selection of two different strategies for building
     the underlying layout tree, which can yield significantly
     different results depending on the data.

 .SECTION Thanks
 Thanks to David Duke from the University of Leeds for providing this 
 implementation.

To create an instance of class vtkSpanTreeLayoutStrategy, simply
invoke its constructor as follows
\begin{verbatim}
  obj = vtkSpanTreeLayoutStrategy
\end{verbatim}
\subsection{Methods}

The class vtkSpanTreeLayoutStrategy has several methods that can be used.
  They are listed below.
Note that the documentation is translated automatically from the VTK sources,
and may not be completely intelligible.  When in doubt, consult the VTK website.
In the methods listed below, \verb|obj| is an instance of the vtkSpanTreeLayoutStrategy class.
\begin{itemize}
\item  \verb|string = obj.GetClassName ()|

\item  \verb|int = obj.IsA (string name)|

\item  \verb|vtkSpanTreeLayoutStrategy = obj.NewInstance ()|

\item  \verb|vtkSpanTreeLayoutStrategy = obj.SafeDownCast (vtkObject o)|

\item  \verb|obj.SetDepthFirstSpanningTree (bool )| -  If set, base the layout on a depth-first spanning tree,
 rather than the default breadth-first spanning tree.
 Switching between DFT and BFT may significantly change
 the layout, and choice must be made on a per-graph basis.
 Default value is off.

\item  \verb|bool = obj.GetDepthFirstSpanningTree ()| -  If set, base the layout on a depth-first spanning tree,
 rather than the default breadth-first spanning tree.
 Switching between DFT and BFT may significantly change
 the layout, and choice must be made on a per-graph basis.
 Default value is off.

\item  \verb|obj.DepthFirstSpanningTreeOn ()| -  If set, base the layout on a depth-first spanning tree,
 rather than the default breadth-first spanning tree.
 Switching between DFT and BFT may significantly change
 the layout, and choice must be made on a per-graph basis.
 Default value is off.

\item  \verb|obj.DepthFirstSpanningTreeOff ()| -  If set, base the layout on a depth-first spanning tree,
 rather than the default breadth-first spanning tree.
 Switching between DFT and BFT may significantly change
 the layout, and choice must be made on a per-graph basis.
 Default value is off.

\item  \verb|obj.Layout ()| -  Perform the layout.

\end{itemize}
