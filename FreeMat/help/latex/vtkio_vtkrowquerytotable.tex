\section{vtkRowQueryToTable}

\subsection{Usage}

 vtkRowQueryToTable creates a vtkTable with the results of an arbitrary SQL
 query.  To use this filter, you first need an instance of a vtkSQLDatabase
 subclass.  You may use the database class to obtain a vtkRowQuery instance.
 Set that query on this filter to extract the query as a table.

 .SECTION Thanks
 Thanks to Andrew Wilson from Sandia National Laboratories for his work
 on the database classes.


To create an instance of class vtkRowQueryToTable, simply
invoke its constructor as follows
\begin{verbatim}
  obj = vtkRowQueryToTable
\end{verbatim}
\subsection{Methods}

The class vtkRowQueryToTable has several methods that can be used.
  They are listed below.
Note that the documentation is translated automatically from the VTK sources,
and may not be completely intelligible.  When in doubt, consult the VTK website.
In the methods listed below, \verb|obj| is an instance of the vtkRowQueryToTable class.
\begin{itemize}
\item  \verb|string = obj.GetClassName ()|

\item  \verb|int = obj.IsA (string name)|

\item  \verb|vtkRowQueryToTable = obj.NewInstance ()|

\item  \verb|vtkRowQueryToTable = obj.SafeDownCast (vtkObject o)|

\item  \verb|obj.SetQuery (vtkRowQuery query)| -  The query to execute.

\item  \verb|vtkRowQuery = obj.GetQuery ()| -  The query to execute.

\item  \verb|long = obj.GetMTime ()| -  Update the modified time based on the query.

\end{itemize}
