\section{vtkConditionVariable}

\subsection{Usage}

 vtkConditionVariable allows the locking of variables which are accessed 
 through different threads.  This header file also defines 
 vtkSimpleConditionVariable which is not a subclass of vtkObject.

 The win32 implementation is based on notes provided by
 Douglas C. Schmidt and Irfan Pyarali,
 Department of Computer Science,
 Washington University, St. Louis, Missouri.
 http://www.cs.wustl.edu/~schmidt/win32-cv-1.html

To create an instance of class vtkConditionVariable, simply
invoke its constructor as follows
\begin{verbatim}
  obj = vtkConditionVariable
\end{verbatim}
\subsection{Methods}

The class vtkConditionVariable has several methods that can be used.
  They are listed below.
Note that the documentation is translated automatically from the VTK sources,
and may not be completely intelligible.  When in doubt, consult the VTK website.
In the methods listed below, \verb|obj| is an instance of the vtkConditionVariable class.
\begin{itemize}
\item  \verb|string = obj.GetClassName ()|

\item  \verb|int = obj.IsA (string name)|

\item  \verb|vtkConditionVariable = obj.NewInstance ()|

\item  \verb|vtkConditionVariable = obj.SafeDownCast (vtkObject o)|

\item  \verb|obj.Signal ()| -  Wake one thread waiting for the condition to change.

\item  \verb|obj.Broadcast ()| -  Wake all threads waiting for the condition to change.

\item  \verb|int = obj.Wait (vtkMutexLock mutex)| -  Wait for the condition to change.
 Upon entry, the mutex must be locked and the lock held by the calling thread.
 Upon exit, the mutex will be locked and held by the calling thread.
 Between entry and exit, the mutex will be unlocked and may be held by other threads.
 
 @param mutex The mutex that should be locked on entry and will be locked on exit (but not in between)
 @retval Normally, this function returns 0. Should a thread be interrupted by a signal, a non-zero value may be returned.

\end{itemize}
