\section{vtkImageOpenClose3D}

\subsection{Usage}

 vtkImageOpenClose3D performs opening or closing by having two 
 vtkImageErodeDilates in series.  The size of operation
 is determined by the method SetKernelSize, and the operator is an ellipse.
 OpenValue and CloseValue determine how the filter behaves.  For binary
 images Opening and closing behaves as expected.
 Close value is first dilated, and then eroded.
 Open value is first eroded, and then dilated.
 Degenerate two dimensional opening/closing can be achieved by setting the
 one axis the 3D KernelSize to 1.
 Values other than open value and close value are not touched.
 This enables the filter to processes segmented images containing more than
 two tags.

To create an instance of class vtkImageOpenClose3D, simply
invoke its constructor as follows
\begin{verbatim}
  obj = vtkImageOpenClose3D
\end{verbatim}
\subsection{Methods}

The class vtkImageOpenClose3D has several methods that can be used.
  They are listed below.
Note that the documentation is translated automatically from the VTK sources,
and may not be completely intelligible.  When in doubt, consult the VTK website.
In the methods listed below, \verb|obj| is an instance of the vtkImageOpenClose3D class.
\begin{itemize}
\item  \verb|string = obj.GetClassName ()| -  Default open value is 0, and default close value is 255.

\item  \verb|int = obj.IsA (string name)| -  Default open value is 0, and default close value is 255.

\item  \verb|vtkImageOpenClose3D = obj.NewInstance ()| -  Default open value is 0, and default close value is 255.

\item  \verb|vtkImageOpenClose3D = obj.SafeDownCast (vtkObject o)| -  Default open value is 0, and default close value is 255.

\item  \verb|long = obj.GetMTime ()| -  This method considers the sub filters MTimes when computing this objects
 modified time.

\item  \verb|obj.DebugOn ()| -  Turn debugging output on. (in sub filters also)

\item  \verb|obj.DebugOff ()| -  Turn debugging output on. (in sub filters also)

\item  \verb|obj.Modified ()| -  Pass modified message to sub filters.

\item  \verb|obj.SetKernelSize (int size0, int size1, int size2)| -  Selects the size of gaps or objects removed.

\item  \verb|obj.SetOpenValue (double value)| -  Determines the value that will opened.  
 Open value is first eroded, and then dilated.

\item  \verb|double = obj.GetOpenValue ()| -  Determines the value that will opened.  
 Open value is first eroded, and then dilated.

\item  \verb|obj.SetCloseValue (double value)| -  Determines the value that will closed.
 Close value is first dilated, and then eroded

\item  \verb|double = obj.GetCloseValue ()| -  Determines the value that will closed.
 Close value is first dilated, and then eroded

\item  \verb|vtkImageDilateErode3D = obj.GetFilter0 ()| -  Needed for Progress functions

\item  \verb|vtkImageDilateErode3D = obj.GetFilter1 ()| -  Needed for Progress functions

\end{itemize}
