\section{vtkBiQuadraticTriangle}

\subsection{Usage}

 vtkBiQuadraticTriangle is a concrete implementation of vtkNonLinearCell to
 represent a two-dimensional, 7-node, isoparametric parabolic triangle. The
 interpolation is the standard finite element, bi-quadratic isoparametric
 shape function. The cell includes three mid-edge nodes besides the three
 triangle vertices and a center node. The ordering of the three points defining the cell is 
 point ids (0-2,3-6) where id \#3 is the midedge node between points
 (0,1); id \#4 is the midedge node between points (1,2); and id \#5 is the 
 midedge node between points (2,0). id \#6 is the center node of the cell.

To create an instance of class vtkBiQuadraticTriangle, simply
invoke its constructor as follows
\begin{verbatim}
  obj = vtkBiQuadraticTriangle
\end{verbatim}
\subsection{Methods}

The class vtkBiQuadraticTriangle has several methods that can be used.
  They are listed below.
Note that the documentation is translated automatically from the VTK sources,
and may not be completely intelligible.  When in doubt, consult the VTK website.
In the methods listed below, \verb|obj| is an instance of the vtkBiQuadraticTriangle class.
\begin{itemize}
\item  \verb|string = obj.GetClassName ()|

\item  \verb|int = obj.IsA (string name)|

\item  \verb|vtkBiQuadraticTriangle = obj.NewInstance ()|

\item  \verb|vtkBiQuadraticTriangle = obj.SafeDownCast (vtkObject o)|

\item  \verb|int = obj.GetCellType ()| -  Implement the vtkCell API. See the vtkCell API for descriptions
 of these methods.

\item  \verb|int = obj.GetCellDimension ()| -  Implement the vtkCell API. See the vtkCell API for descriptions
 of these methods.

\item  \verb|int = obj.GetNumberOfEdges ()| -  Implement the vtkCell API. See the vtkCell API for descriptions
 of these methods.

\item  \verb|int = obj.GetNumberOfFaces ()| -  Implement the vtkCell API. See the vtkCell API for descriptions
 of these methods.

\item  \verb|vtkCell = obj.GetEdge (int edgeId)| -  Implement the vtkCell API. See the vtkCell API for descriptions
 of these methods.

\item  \verb|vtkCell = obj.GetFace (int )|

\item  \verb|int = obj.CellBoundary (int subId, double pcoords[3], vtkIdList pts)|

\item  \verb|obj.Contour (double value, vtkDataArray cellScalars, vtkIncrementalPointLocator locator, vtkCellArray verts, vtkCellArray lines, vtkCellArray polys, vtkPointData inPd, vtkPointData outPd, vtkCellData inCd, vtkIdType cellId, vtkCellData outCd)|

\item  \verb|int = obj.Triangulate (int index, vtkIdList ptIds, vtkPoints pts)|

\item  \verb|obj.Derivatives (int subId, double pcoords[3], double values, int dim, double derivs)|

\item  \verb|obj.Clip (double value, vtkDataArray cellScalars, vtkIncrementalPointLocator locator, vtkCellArray polys, vtkPointData inPd, vtkPointData outPd, vtkCellData inCd, vtkIdType cellId, vtkCellData outCd, int insideOut)| -  Clip this quadratic triangle using scalar value provided. Like
 contouring, except that it cuts the triangle to produce linear
 triangles.

\item  \verb|int = obj.GetParametricCenter (double pcoords[3])| -  Return the center of the quadratic triangle in parametric coordinates.

\item  \verb|double = obj.GetParametricDistance (double pcoords[3])| -  Return the distance of the parametric coordinate provided to the
 cell. If inside the cell, a distance of zero is returned.

\item  \verb|obj.InterpolateFunctions (double pcoords[3], double weights[7])| -  Compute the interpolation functions/derivatives
 (aka shape functions/derivatives)

\item  \verb|obj.InterpolateDerivs (double pcoords[3], double derivs[14])|

\end{itemize}
