\section{vtkOrderedTriangulator}

\subsection{Usage}

 This class is used to generate unique triangulations of points. The
 uniqueness of the triangulation is controlled by the id of the inserted
 points in combination with a Delaunay criterion. The class is designed to
 be as fast as possible (since the algorithm can be slow) and uses block
 memory allocations to support rapid triangulation generation. Also, the
 assumption behind the class is that a maximum of hundreds of points are to
 be triangulated. If you desire more robust triangulation methods use
 vtkPolygon::Triangulate(), vtkDelaunay2D, or vtkDelaunay3D.

 .SECTION Background
 This work is documented in the technical paper: W.J. Schroeder, B. Geveci,
 M. Malaterre. Compatible Triangulations of Spatial Decompositions. In
 Proceedings of Visualization 2004, IEEE Press October 2004.

 Delaunay triangulations are unique assuming a random distribution of input
 points. The 3D Delaunay criterion is as follows: the circumsphere of each
 tetrahedron contains no other points of the triangulation except for the
 four points defining the tetrahedron.  In application this property is
 hard to satisfy because objects like cubes are defined by eight points all
 sharing the same circumsphere (center and radius); hence the Delaunay
 triangulation is not unique.  These so-called degenerate situations are
 typically resolved by arbitrary selecting a triangulation. This code does
 something different: it resolves degenerate triangulations by modifying
 the ''InCircumsphere'' method to use a slightly smaller radius. Hence,
 degenerate points are always considered ''out'' of the circumsphere. This,
 in combination with an ordering (based on id) of the input points,
 guarantees a unique triangulation.

 There is another related characteristic of Delaunay triangulations. Given
 a N-dimensional Delaunay triangulation, points lying on a (N-1) dimensional
 plane also form a (N-1) Delaunay triangulation. This means for example,
 that if a 3D cell is defined by a set of (2D) planar faces, then the
 face triangulations are Delaunay. Combining this with the method to
 generate unique triangulations described previously, the triangulations
 on the face are guaranteed unique. This fact can be used to triangulate
 3D objects in such a way to guarantee compatible face triangulations.
 This is a very useful fact for parallel processing, or performing
 operations like clipping that require compatible triangulations across
 3D cell faces. (See vtkClipVolume for an example.)

 A special feature of this class is that it can generate triangulation 
 templates on the fly. If template triangulation is enabled, then the
 ordered triangulator will first triangulate the cell using the slower
 ordered Delaunay approach, and then store the result as a template.
 Later, if the same cell type and cell configuration is encountered,
 then the template is reused which greatly speeds the triangulation.

To create an instance of class vtkOrderedTriangulator, simply
invoke its constructor as follows
\begin{verbatim}
  obj = vtkOrderedTriangulator
\end{verbatim}
\subsection{Methods}

The class vtkOrderedTriangulator has several methods that can be used.
  They are listed below.
Note that the documentation is translated automatically from the VTK sources,
and may not be completely intelligible.  When in doubt, consult the VTK website.
In the methods listed below, \verb|obj| is an instance of the vtkOrderedTriangulator class.
\begin{itemize}
\item  \verb|string = obj.GetClassName ()|

\item  \verb|int = obj.IsA (string name)|

\item  \verb|vtkOrderedTriangulator = obj.NewInstance ()|

\item  \verb|vtkOrderedTriangulator = obj.SafeDownCast (vtkObject o)|

\item  \verb|obj.InitTriangulation (double xmin, double xmax, double ymin, double ymax, double zmin, double zmax, int numPts)| -  Initialize the triangulation process. Provide a bounding box and
 the maximum number of points to be inserted. Note that since the
 triangulation is performed using parametric coordinates (see
 InsertPoint()) the bounds should be represent the range of the
 parametric coordinates inserted.
 

\item  \verb|obj.InitTriangulation (double bounds[6], int numPts)| -  Initialize the triangulation process. Provide a bounding box and
 the maximum number of points to be inserted. Note that since the
 triangulation is performed using parametric coordinates (see
 InsertPoint()) the bounds should be represent the range of the
 parametric coordinates inserted.
 

\item  \verb|vtkIdType = obj.InsertPoint (vtkIdType id, double x[3], double p[3], int type)| -  For each point to be inserted, provide an id, a position x, parametric
 coordinate p, and whether the point is inside (type=0), outside
 (type=1), or on the boundary (type=2). You must call InitTriangulation()
 prior to invoking this method. Make sure that the number of points
 inserted does not exceed the numPts specified in
 InitTriangulation(). Also note that the ''id'' can be any integer and can
 be greater than numPts. It is used to create tetras (in AddTetras()) with
 the appropriate connectivity ids. The method returns an internal id that
 can be used prior to the Triangulate() method to update the type of the
 point with UpdatePointType(). (Note: the algorithm triangulated with the
 parametric coordinate p[3] and creates tetras with the global coordinate
 x[3]. The parametric coordinates and global coordinates may be the same.)

\item  \verb|vtkIdType = obj.InsertPoint (vtkIdType id, vtkIdType sortid, double x[3], double p[3], int type)| -  For each point to be inserted, provide an id, a position x, parametric
 coordinate p, and whether the point is inside (type=0), outside
 (type=1), or on the boundary (type=2). You must call InitTriangulation()
 prior to invoking this method. Make sure that the number of points
 inserted does not exceed the numPts specified in
 InitTriangulation(). Also note that the ''id'' can be any integer and can
 be greater than numPts. It is used to create tetras (in AddTetras()) with
 the appropriate connectivity ids. The method returns an internal id that
 can be used prior to the Triangulate() method to update the type of the
 point with UpdatePointType(). (Note: the algorithm triangulated with the
 parametric coordinate p[3] and creates tetras with the global coordinate
 x[3]. The parametric coordinates and global coordinates may be the same.)

\item  \verb|vtkIdType = obj.InsertPoint (vtkIdType id, vtkIdType sortid, vtkIdType sortid2, double x[3], double p[3], int type)| -  For each point to be inserted, provide an id, a position x, parametric
 coordinate p, and whether the point is inside (type=0), outside
 (type=1), or on the boundary (type=2). You must call InitTriangulation()
 prior to invoking this method. Make sure that the number of points
 inserted does not exceed the numPts specified in
 InitTriangulation(). Also note that the ''id'' can be any integer and can
 be greater than numPts. It is used to create tetras (in AddTetras()) with
 the appropriate connectivity ids. The method returns an internal id that
 can be used prior to the Triangulate() method to update the type of the
 point with UpdatePointType(). (Note: the algorithm triangulated with the
 parametric coordinate p[3] and creates tetras with the global coordinate
 x[3]. The parametric coordinates and global coordinates may be the same.)

\item  \verb|obj.Triangulate ()| -  Perform the triangulation. (Complete all calls to InsertPoint() prior
 to invoking this method.) A special version is available when templates
 should be used.

\item  \verb|obj.TemplateTriangulate (int cellType, int numPts, int numEdges)| -  Perform the triangulation. (Complete all calls to InsertPoint() prior
 to invoking this method.) A special version is available when templates
 should be used.

\item  \verb|obj.UpdatePointType (vtkIdType internalId, int type)| -  Update the point type. This is useful when the merging of nearly 
 coincident points is performed. The id is the internal id returned
 from InsertPoint(). The method should be invoked prior to the
 Triangulate method. The type is specified as inside (type=0), 
 outside (type=1), or on the boundary (type=2).
 

\item  \verb|vtkIdType = obj.GetPointId (vtkIdType internalId)| -  Return the Id of point `internalId'. This id is the one passed in
 argument of InsertPoint.
 It assumes that the point has already been inserted.
 The method should be invoked prior to the Triangulate method.
 

\item  \verb|int = obj.GetNumberOfPoints ()| -  Return the number of inserted points.

\item  \verb|obj.SetUseTemplates (int )| -  If this flag is set, then the ordered triangulator will create
 and use templates for the triangulation. To use templates, the
 TemplateTriangulate() method should be called when appropriate.
 (Note: the TemplateTriangulate() method works for complete 
 (interior) cells without extra points due to intersection, etc.)

\item  \verb|int = obj.GetUseTemplates ()| -  If this flag is set, then the ordered triangulator will create
 and use templates for the triangulation. To use templates, the
 TemplateTriangulate() method should be called when appropriate.
 (Note: the TemplateTriangulate() method works for complete 
 (interior) cells without extra points due to intersection, etc.)

\item  \verb|obj.UseTemplatesOn ()| -  If this flag is set, then the ordered triangulator will create
 and use templates for the triangulation. To use templates, the
 TemplateTriangulate() method should be called when appropriate.
 (Note: the TemplateTriangulate() method works for complete 
 (interior) cells without extra points due to intersection, etc.)

\item  \verb|obj.UseTemplatesOff ()| -  If this flag is set, then the ordered triangulator will create
 and use templates for the triangulation. To use templates, the
 TemplateTriangulate() method should be called when appropriate.
 (Note: the TemplateTriangulate() method works for complete 
 (interior) cells without extra points due to intersection, etc.)

\item  \verb|obj.SetPreSorted (int )| -  Boolean indicates whether the points have been pre-sorted. If 
 pre-sorted is enabled, the points are not sorted on point id.
 By default, presorted is off. (The point id is defined in
 InsertPoint().)

\item  \verb|int = obj.GetPreSorted ()| -  Boolean indicates whether the points have been pre-sorted. If 
 pre-sorted is enabled, the points are not sorted on point id.
 By default, presorted is off. (The point id is defined in
 InsertPoint().)

\item  \verb|obj.PreSortedOn ()| -  Boolean indicates whether the points have been pre-sorted. If 
 pre-sorted is enabled, the points are not sorted on point id.
 By default, presorted is off. (The point id is defined in
 InsertPoint().)

\item  \verb|obj.PreSortedOff ()| -  Boolean indicates whether the points have been pre-sorted. If 
 pre-sorted is enabled, the points are not sorted on point id.
 By default, presorted is off. (The point id is defined in
 InsertPoint().)

\item  \verb|obj.SetUseTwoSortIds (int )| -  Tells the triangulator that a second sort id is provided
 for each point and should also be considered when sorting.

\item  \verb|int = obj.GetUseTwoSortIds ()| -  Tells the triangulator that a second sort id is provided
 for each point and should also be considered when sorting.

\item  \verb|obj.UseTwoSortIdsOn ()| -  Tells the triangulator that a second sort id is provided
 for each point and should also be considered when sorting.

\item  \verb|obj.UseTwoSortIdsOff ()| -  Tells the triangulator that a second sort id is provided
 for each point and should also be considered when sorting.

\item  \verb|vtkIdType = obj.GetTetras (int classification, vtkUnstructuredGrid ugrid)| -  Initialize and add the tetras and points from the triangulation to the
 unstructured grid provided.  New points are created and the mesh is
 allocated. (This method differs from AddTetras() in that it inserts
 points and cells; AddTetras only adds the tetra cells.) The tetrahdera
 added are of the type specified (0=inside,1=outside,2=all). Inside
 tetrahedron are those whose points are classified ''inside'' or on the
 ''boundary.''  Outside tetrahedron have at least one point classified
 ''outside.''  The method returns the number of tetrahedrahedron of the
 type requested.

\item  \verb|vtkIdType = obj.AddTetras (int classification, vtkUnstructuredGrid ugrid)| -  Add the tetras to the unstructured grid provided. The unstructured
 grid is assumed to have been initialized (with Allocate()) and
 points set (with SetPoints()). The tetrahdera added are of the type
 specified (0=inside,1=outside,2=all). Inside tetrahedron are 
 those whose points are classified ''inside'' or on the ''boundary.'' 
 Outside tetrahedron have at least one point classified ''outside.'' 
 The method returns the number of tetrahedrahedron of the type 
 requested.

\item  \verb|vtkIdType = obj.AddTetras (int classification, vtkCellArray connectivity)| -  Add the tetrahedra classified (0=inside,1=outside) to the connectivity
 list provided. Inside tetrahedron are those whose points are all
 classified ''inside.'' Outside tetrahedron have at least one point
 classified ''outside.'' The method returns the number of tetrahedron
 of the type requested.    

\item  \verb|vtkIdType = obj.AddTetras (int classification, vtkIncrementalPointLocator locator, vtkCellArray outConnectivity, vtkPointData inPD, vtkPointData outPD, vtkCellData inCD, vtkIdType cellId, vtkCellData outCD)| -  Assuming that all the inserted points come from a cell `cellId' to
 triangulate, get the tetrahedra in outConnectivity, the points in locator
 and copy point data and cell data. Return the number of added tetras.
 
 
 
 
 
 

\item  \verb|vtkIdType = obj.AddTetras (int classification, vtkIdList ptIds, vtkPoints pts)| -  Add the tetrahedra classified (0=inside,1=outside) to the list
 of ids and coordinates provided. These assume that the first four points
 form a tetrahedron, the next four the next, and so on.

\item  \verb|vtkIdType = obj.AddTriangles (vtkCellArray connectivity)| -  Add the triangle faces classified (2=boundary) to the connectivity
 list provided. The method returns the number of triangles.

\item  \verb|vtkIdType = obj.AddTriangles (vtkIdType id, vtkCellArray connectivity)| -  Add the triangle faces classified (2=boundary) and attached to the
 specified point id to the connectivity list provided. (The id is the
 same as that specified in InsertPoint().)  

\item  \verb|obj.InitTetraTraversal ()| -  Methods to get one tetra at a time. Start with InitTetraTraversal()
 and then invoke GetNextTetra() until the method returns 0.

\item  \verb|int = obj.GetNextTetra (int classification, vtkTetra tet, vtkDataArray cellScalars, vtkDoubleArray tetScalars)| -  Methods to get one tetra at a time. Start with InitTetraTraversal()
 and then invoke GetNextTetra() until the method returns 0.
 cellScalars are point-centered scalars on the original cell.
 tetScalars are point-centered scalars on the tetra: the values will be
 copied from cellScalars.
 
 
 
 

\end{itemize}
