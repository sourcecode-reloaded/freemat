\section{vtkAbstractWidget}

\subsection{Usage}

 The vtkAbstractWidget defines an API and implements methods common to all
 widgets using the interaction/representation design. In this design, the
 term interaction means that part of the widget that performs event
 handling, while the representation corresponds to a vtkProp (or the
 subclass vtkWidgetRepresentation) used to represent the
 widget. vtkAbstractWidget also implements some methods common to all
 subclasses.

 Note that vtkAbstractWidget provides access to the
 vtkWidgetEventTranslator.  This class is responsible for translating VTK
 events (defined in vtkCommand.h) into widget events (defined in
 vtkWidgetEvent.h).  This class can be manipulated so that different VTK
 events can be mapped into widget events, thereby allowing the modification
 of event bindings. Each subclass of vtkAbstractWidget defines the events
 to which it responds.


To create an instance of class vtkAbstractWidget, simply
invoke its constructor as follows
\begin{verbatim}
  obj = vtkAbstractWidget
\end{verbatim}
\subsection{Methods}

The class vtkAbstractWidget has several methods that can be used.
  They are listed below.
Note that the documentation is translated automatically from the VTK sources,
and may not be completely intelligible.  When in doubt, consult the VTK website.
In the methods listed below, \verb|obj| is an instance of the vtkAbstractWidget class.
\begin{itemize}
\item  \verb|string = obj.GetClassName ()| -  Standard macros implementing standard VTK methods.

\item  \verb|int = obj.IsA (string name)| -  Standard macros implementing standard VTK methods.

\item  \verb|vtkAbstractWidget = obj.NewInstance ()| -  Standard macros implementing standard VTK methods.

\item  \verb|vtkAbstractWidget = obj.SafeDownCast (vtkObject o)| -  Standard macros implementing standard VTK methods.

\item  \verb|obj.SetEnabled (int )| -  Methods for activating this widget. Note that the widget representation
 must be specified or the widget will not appear.
 ProcessEvents (On by default) must be On for Enabled widget to respond 
 to interaction. If ProcessEvents is Off, enabling/disabling a widget 
 merely affects the visibility of the representation.

\item  \verb|obj.SetProcessEvents (int )| -  Methods to change the whether the widget responds to interaction.
 Set this to Off to disable interaction. On by default.
 Subclasses must overide SetProcessEvents() to make sure
 that they pass on the flag to all component widgets.

\item  \verb|int = obj.GetProcessEventsMinValue ()| -  Methods to change the whether the widget responds to interaction.
 Set this to Off to disable interaction. On by default.
 Subclasses must overide SetProcessEvents() to make sure
 that they pass on the flag to all component widgets.

\item  \verb|int = obj.GetProcessEventsMaxValue ()| -  Methods to change the whether the widget responds to interaction.
 Set this to Off to disable interaction. On by default.
 Subclasses must overide SetProcessEvents() to make sure
 that they pass on the flag to all component widgets.

\item  \verb|int = obj.GetProcessEvents ()| -  Methods to change the whether the widget responds to interaction.
 Set this to Off to disable interaction. On by default.
 Subclasses must overide SetProcessEvents() to make sure
 that they pass on the flag to all component widgets.

\item  \verb|obj.ProcessEventsOn ()| -  Methods to change the whether the widget responds to interaction.
 Set this to Off to disable interaction. On by default.
 Subclasses must overide SetProcessEvents() to make sure
 that they pass on the flag to all component widgets.

\item  \verb|obj.ProcessEventsOff ()| -  Methods to change the whether the widget responds to interaction.
 Set this to Off to disable interaction. On by default.
 Subclasses must overide SetProcessEvents() to make sure
 that they pass on the flag to all component widgets.

\item  \verb|vtkWidgetEventTranslator = obj.GetEventTranslator ()| -  Create the default widget representation if one is not set. The
 representation defines the geometry of the widget (i.e., how it appears)
 as well as providing special methods for manipulting the state and
 appearance of the widget.

\item  \verb|obj.CreateDefaultRepresentation ()| -  Create the default widget representation if one is not set. The
 representation defines the geometry of the widget (i.e., how it appears)
 as well as providing special methods for manipulting the state and
 appearance of the widget.

\item  \verb|obj.Render ()| -  This method is called by subclasses when a render method is to be
 invoked on the vtkRenderWindowInteractor. This method should be called
 (instead of vtkRenderWindow::Render() because it has built into it
 optimizations for minimizing renders and/or speeding renders.

\item  \verb|obj.SetParent (vtkAbstractWidget parent)| -  Specifying a parent to this widget is used when creating composite
 widgets. It is an internal method not meant to be used by the public.
 When a widget has a parent, it defers the rendering to the parent. It
 may also defer managing the cursor (see ManagesCursor ivar).

\item  \verb|vtkAbstractWidget = obj.GetParent ()| -  Specifying a parent to this widget is used when creating composite
 widgets. It is an internal method not meant to be used by the public.
 When a widget has a parent, it defers the rendering to the parent. It
 may also defer managing the cursor (see ManagesCursor ivar).

\item  \verb|vtkWidgetRepresentation = obj.GetRepresentation ()| -  Turn on or off the management of the cursor. Cursor management is
 typically disabled for subclasses when composite widgets are
 created. For example, vtkHandleWidgets are often used to create
 composite widgets, and the parent widget takes over the cursor
 management.

\item  \verb|obj.SetManagesCursor (int )| -  Turn on or off the management of the cursor. Cursor management is
 typically disabled for subclasses when composite widgets are
 created. For example, vtkHandleWidgets are often used to create
 composite widgets, and the parent widget takes over the cursor
 management.

\item  \verb|int = obj.GetManagesCursor ()| -  Turn on or off the management of the cursor. Cursor management is
 typically disabled for subclasses when composite widgets are
 created. For example, vtkHandleWidgets are often used to create
 composite widgets, and the parent widget takes over the cursor
 management.

\item  \verb|obj.ManagesCursorOn ()| -  Turn on or off the management of the cursor. Cursor management is
 typically disabled for subclasses when composite widgets are
 created. For example, vtkHandleWidgets are often used to create
 composite widgets, and the parent widget takes over the cursor
 management.

\item  \verb|obj.ManagesCursorOff ()| -  Turn on or off the management of the cursor. Cursor management is
 typically disabled for subclasses when composite widgets are
 created. For example, vtkHandleWidgets are often used to create
 composite widgets, and the parent widget takes over the cursor
 management.

\item  \verb|obj.SetPriority (float )| -  Override the superclass method. This will automatically change the
 priority of the widget. Unlike the superclass documentation, no 
 methods such as SetInteractor to null and reset it etc. are necessary

\end{itemize}
