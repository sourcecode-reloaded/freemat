\section{vtkSphericalTransform}

\subsection{Usage}

 vtkSphericalTransform will convert (r,phi,theta) coordinates to 
 (x,y,z) coordinates and back again.  The angles are given in radians.
 By default, it converts spherical coordinates to rectangular, but
 GetInverse() returns a transform that will do the opposite.  The equation
 that is used is x = r*sin(phi)*cos(theta), y = r*sin(phi)*sin(theta),
 z = r*cos(phi).

To create an instance of class vtkSphericalTransform, simply
invoke its constructor as follows
\begin{verbatim}
  obj = vtkSphericalTransform
\end{verbatim}
\subsection{Methods}

The class vtkSphericalTransform has several methods that can be used.
  They are listed below.
Note that the documentation is translated automatically from the VTK sources,
and may not be completely intelligible.  When in doubt, consult the VTK website.
In the methods listed below, \verb|obj| is an instance of the vtkSphericalTransform class.
\begin{itemize}
\item  \verb|string = obj.GetClassName ()|

\item  \verb|int = obj.IsA (string name)|

\item  \verb|vtkSphericalTransform = obj.NewInstance ()|

\item  \verb|vtkSphericalTransform = obj.SafeDownCast (vtkObject o)|

\item  \verb|vtkAbstractTransform = obj.MakeTransform ()| -  Make another transform of the same type.

\end{itemize}
