\section{vtkGraphAlgorithm}

\subsection{Usage}

 vtkGraphAlgorithm is a convenience class to make writing algorithms
 easier. It is also designed to help transition old algorithms to the new
 pipeline architecture. There are some assumptions and defaults made by this
 class you should be aware of. This class defaults such that your filter
 will have one input port and one output port. If that is not the case
 simply change it with SetNumberOfInputPorts etc. See this class
 constructor for the default. This class also provides a FillInputPortInfo
 method that by default says that all inputs will be Graph. If that
 isn't the case then please override this method in your subclass. This
 class breaks out the downstream requests into separate functions such as
 ExecuteData and ExecuteInformation.  For new algorithms you should
 implement RequestData( request, inputVec, outputVec) but for older filters
 there is a default implementation that calls the old ExecuteData(output)
 signature. For even older filters that don't implement ExecuteData the
 default implementation calls the even older Execute() signature.

 .SECTION Thanks
 Thanks to Patricia Crossno, Ken Moreland, Andrew Wilson and Brian Wylie from
 Sandia National Laboratories for their help in developing this class.

To create an instance of class vtkGraphAlgorithm, simply
invoke its constructor as follows
\begin{verbatim}
  obj = vtkGraphAlgorithm
\end{verbatim}
\subsection{Methods}

The class vtkGraphAlgorithm has several methods that can be used.
  They are listed below.
Note that the documentation is translated automatically from the VTK sources,
and may not be completely intelligible.  When in doubt, consult the VTK website.
In the methods listed below, \verb|obj| is an instance of the vtkGraphAlgorithm class.
\begin{itemize}
\item  \verb|string = obj.GetClassName ()|

\item  \verb|int = obj.IsA (string name)|

\item  \verb|vtkGraphAlgorithm = obj.NewInstance ()|

\item  \verb|vtkGraphAlgorithm = obj.SafeDownCast (vtkObject o)|

\item  \verb|vtkGraph = obj.GetOutput ()| -  Get the output data object for a port on this algorithm.

\item  \verb|vtkGraph = obj.GetOutput (int index)| -  Get the output data object for a port on this algorithm.

\item  \verb|obj.SetInput (vtkDataObject obj)| -  Set an input of this algorithm. You should not override these
 methods because they are not the only way to connect a pipeline.
 Note that these methods support old-style pipeline connections.
 When writing new code you should use the more general
 vtkAlgorithm::SetInputConnection().  These methods transform the
 input index to the input port index, not an index of a connection
 within a single port.

\item  \verb|obj.SetInput (int index, vtkDataObject obj)| -  Set an input of this algorithm. You should not override these
 methods because they are not the only way to connect a pipeline.
 Note that these methods support old-style pipeline connections.
 When writing new code you should use the more general
 vtkAlgorithm::SetInputConnection().  These methods transform the
 input index to the input port index, not an index of a connection
 within a single port.

\end{itemize}
