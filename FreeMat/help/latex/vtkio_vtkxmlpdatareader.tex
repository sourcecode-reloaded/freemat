\section{vtkXMLPDataReader}

\subsection{Usage}

 vtkXMLPDataReader provides functionality common to all PVTK XML
 file readers.  Concrete subclasses call upon this functionality
 when needed.

To create an instance of class vtkXMLPDataReader, simply
invoke its constructor as follows
\begin{verbatim}
  obj = vtkXMLPDataReader
\end{verbatim}
\subsection{Methods}

The class vtkXMLPDataReader has several methods that can be used.
  They are listed below.
Note that the documentation is translated automatically from the VTK sources,
and may not be completely intelligible.  When in doubt, consult the VTK website.
In the methods listed below, \verb|obj| is an instance of the vtkXMLPDataReader class.
\begin{itemize}
\item  \verb|string = obj.GetClassName ()|

\item  \verb|int = obj.IsA (string name)|

\item  \verb|vtkXMLPDataReader = obj.NewInstance ()|

\item  \verb|vtkXMLPDataReader = obj.SafeDownCast (vtkObject o)|

\item  \verb|int = obj.GetNumberOfPieces ()| -  Get the number of pieces from the summary file being read.

\item  \verb|obj.CopyOutputInformation (vtkInformation outInfo, int port)|

\end{itemize}
