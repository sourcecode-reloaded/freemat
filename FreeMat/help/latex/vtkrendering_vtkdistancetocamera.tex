\section{vtkDistanceToCamera}

\subsection{Usage}

 This filter adds a double array containing the distance from each point
 to the camera. If Scaling is on, it will use the values in the input
 array to process in order to scale the size of the points. ScreenSize
 sets the size in screen pixels that you would want a rendered rectangle
 at that point to be, if it was scaled by the output array.

To create an instance of class vtkDistanceToCamera, simply
invoke its constructor as follows
\begin{verbatim}
  obj = vtkDistanceToCamera
\end{verbatim}
\subsection{Methods}

The class vtkDistanceToCamera has several methods that can be used.
  They are listed below.
Note that the documentation is translated automatically from the VTK sources,
and may not be completely intelligible.  When in doubt, consult the VTK website.
In the methods listed below, \verb|obj| is an instance of the vtkDistanceToCamera class.
\begin{itemize}
\item  \verb|string = obj.GetClassName ()|

\item  \verb|int = obj.IsA (string name)|

\item  \verb|vtkDistanceToCamera = obj.NewInstance ()|

\item  \verb|vtkDistanceToCamera = obj.SafeDownCast (vtkObject o)|

\item  \verb|obj.SetRenderer (vtkRenderer ren)| -  The renderer which will ultimately render these points.

\item  \verb|vtkRenderer = obj.GetRenderer ()| -  The renderer which will ultimately render these points.

\item  \verb|obj.SetScreenSize (double )| -  The desired screen size obtained by scaling glyphs by the distance
 array. It assumes the glyph at each point will be unit size.

\item  \verb|double = obj.GetScreenSize ()| -  The desired screen size obtained by scaling glyphs by the distance
 array. It assumes the glyph at each point will be unit size.

\item  \verb|obj.SetScaling (bool )| -  Whether to scale the distance by the input array to process.

\item  \verb|bool = obj.GetScaling ()| -  Whether to scale the distance by the input array to process.

\item  \verb|obj.ScalingOn ()| -  Whether to scale the distance by the input array to process.

\item  \verb|obj.ScalingOff ()| -  Whether to scale the distance by the input array to process.

\item  \verb|long = obj.GetMTime ()| -  The modified time of this filter.

\end{itemize}
