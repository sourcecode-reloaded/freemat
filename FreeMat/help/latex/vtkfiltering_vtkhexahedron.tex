\section{vtkHexahedron}

\subsection{Usage}

 vtkHexahedron is a concrete implementation of vtkCell to represent a
 linear, 3D rectangular hexahedron (e.g., ''brick'' topology). vtkHexahedron
 uses the standard isoparametric shape functions for a linear
 hexahedron. The hexahedron is defined by the eight points (0-7) where
 (0,1,2,3) is the base of the hexahedron which, using the right hand rule,
 forms a quadrilaterial whose normal points in the direction of the
 opposite face (4,5,6,7).

To create an instance of class vtkHexahedron, simply
invoke its constructor as follows
\begin{verbatim}
  obj = vtkHexahedron
\end{verbatim}
\subsection{Methods}

The class vtkHexahedron has several methods that can be used.
  They are listed below.
Note that the documentation is translated automatically from the VTK sources,
and may not be completely intelligible.  When in doubt, consult the VTK website.
In the methods listed below, \verb|obj| is an instance of the vtkHexahedron class.
\begin{itemize}
\item  \verb|string = obj.GetClassName ()|

\item  \verb|int = obj.IsA (string name)|

\item  \verb|vtkHexahedron = obj.NewInstance ()|

\item  \verb|vtkHexahedron = obj.SafeDownCast (vtkObject o)|

\item  \verb|int = obj.GetCellType ()| -  See the vtkCell API for descriptions of these methods.

\item  \verb|int = obj.GetNumberOfEdges ()| -  See the vtkCell API for descriptions of these methods.

\item  \verb|int = obj.GetNumberOfFaces ()| -  See the vtkCell API for descriptions of these methods.

\item  \verb|vtkCell = obj.GetEdge (int edgeId)| -  See the vtkCell API for descriptions of these methods.

\item  \verb|vtkCell = obj.GetFace (int faceId)| -  See the vtkCell API for descriptions of these methods.

\item  \verb|int = obj.CellBoundary (int subId, double pcoords[3], vtkIdList pts)| -  See the vtkCell API for descriptions of these methods.

\item  \verb|obj.Contour (double value, vtkDataArray cellScalars, vtkIncrementalPointLocator locator, vtkCellArray verts, vtkCellArray lines, vtkCellArray polys, vtkPointData inPd, vtkPointData outPd, vtkCellData inCd, vtkIdType cellId, vtkCellData outCd)| -  See the vtkCell API for descriptions of these methods.

\item  \verb|int = obj.Triangulate (int index, vtkIdList ptIds, vtkPoints pts)|

\item  \verb|obj.Derivatives (int subId, double pcoords[3], double values, int dim, double derivs)|

\item  \verb|obj.InterpolateFunctions (double pcoords[3], double weights[8])| -  Compute the interpolation functions/derivatives
 (aka shape functions/derivatives)

\item  \verb|obj.InterpolateDerivs (double pcoords[3], double derivs[24])| -  Return the ids of the vertices defining edge/face (`edgeId`/`faceId').
 Ids are related to the cell, not to the dataset.

\end{itemize}
