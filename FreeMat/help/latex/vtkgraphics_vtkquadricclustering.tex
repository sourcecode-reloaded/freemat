\section{vtkQuadricClustering}

\subsection{Usage}

 vtkQuadricClustering is a filter to reduce the number of triangles in a
 triangle mesh, forming a good approximation to the original geometry.  The
 input to vtkQuadricClustering is a vtkPolyData object, and all types of
 polygonal data are handled.

 The algorithm used is the one described by Peter Lindstrom in his Siggraph
 2000 paper, ''Out-of-Core Simplification of Large Polygonal Models.''  The
 general approach of the algorithm is to cluster vertices in a uniform
 binning of space, accumulating the quadric of each triangle (pushed out to
 the triangles vertices) within each bin, and then determining an optimal
 position for a single vertex in a bin by using the accumulated quadric. In
 more detail, the algorithm first gets the bounds of the input poly data.
 It then breaks this bounding volume into a user-specified number of
 spatial bins.  It then reads each triangle from the input and hashes its
 vertices into these bins.  (If this is the first time a bin has been
 visited, initialize its quadric to the 0 matrix.) The algorithm computes
 the error quadric for this triangle and adds it to the existing quadric of
 the bin in which each vertex is contained. Then, if 2 or more vertices of
 the triangle fall in the same bin, the triangle is dicarded.  If the
 triangle is not discarded, it adds the triangle to the list of output
 triangles as a list of vertex identifiers.  (There is one vertex id per
 bin.)  After all the triangles have been read, the representative vertex
 for each bin is computed (an optimal location is found) using the quadric
 for that bin.  This determines the spatial location of the vertices of
 each of the triangles in the output.

 To use this filter, specify the divisions defining the spatial subdivision
 in the x, y, and z directions. You must also specify an input vtkPolyData.
 Then choose to either 1) use the original points that minimize the quadric
 error to produce the output triangles or 2) compute an optimal position in
 each bin to produce the output triangles (recommended and default behavior).

 This filter can take multiple inputs.  To do this, the user must explicity
 call StartAppend, Append (once for each input), and EndAppend.  StartAppend
 sets up the data structure to hold the quadric matrices.  Append processes
 each triangle in the input poly data it was called on, hashes its vertices
 to the appropriate bins, determines whether to keep this triangle, and
 updates the appropriate quadric matrices.  EndAppend determines the spatial
 location of each of the representative vertices for the visited bins. While
 this approach does not fit into the visualization architecture and requires
 manual control, it has the advantage that extremely large data can be 
 processed in pieces and appended to the filter piece-by-piece.

To create an instance of class vtkQuadricClustering, simply
invoke its constructor as follows
\begin{verbatim}
  obj = vtkQuadricClustering
\end{verbatim}
\subsection{Methods}

The class vtkQuadricClustering has several methods that can be used.
  They are listed below.
Note that the documentation is translated automatically from the VTK sources,
and may not be completely intelligible.  When in doubt, consult the VTK website.
In the methods listed below, \verb|obj| is an instance of the vtkQuadricClustering class.
\begin{itemize}
\item  \verb|string = obj.GetClassName ()| -  Standard instantition, type and print methods.

\item  \verb|int = obj.IsA (string name)| -  Standard instantition, type and print methods.

\item  \verb|vtkQuadricClustering = obj.NewInstance ()| -  Standard instantition, type and print methods.

\item  \verb|vtkQuadricClustering = obj.SafeDownCast (vtkObject o)| -  Standard instantition, type and print methods.

\item  \verb|obj.SetNumberOfXDivisions (int num)| -  Set/Get the number of divisions along each axis for the spatial bins.
 The number of spatial bins is NumberOfXDivisions*NumberOfYDivisions*
 NumberOfZDivisions.  The filter may choose to ignore large numbers of
 divisions if the input has few points and AutoAdjustNumberOfDivisions
 is enabled.

\item  \verb|obj.SetNumberOfYDivisions (int num)| -  Set/Get the number of divisions along each axis for the spatial bins.
 The number of spatial bins is NumberOfXDivisions*NumberOfYDivisions*
 NumberOfZDivisions.  The filter may choose to ignore large numbers of
 divisions if the input has few points and AutoAdjustNumberOfDivisions
 is enabled.

\item  \verb|obj.SetNumberOfZDivisions (int num)| -  Set/Get the number of divisions along each axis for the spatial bins.
 The number of spatial bins is NumberOfXDivisions*NumberOfYDivisions*
 NumberOfZDivisions.  The filter may choose to ignore large numbers of
 divisions if the input has few points and AutoAdjustNumberOfDivisions
 is enabled.

\item  \verb|int = obj.GetNumberOfXDivisions ()| -  Set/Get the number of divisions along each axis for the spatial bins.
 The number of spatial bins is NumberOfXDivisions*NumberOfYDivisions*
 NumberOfZDivisions.  The filter may choose to ignore large numbers of
 divisions if the input has few points and AutoAdjustNumberOfDivisions
 is enabled.

\item  \verb|int = obj.GetNumberOfYDivisions ()| -  Set/Get the number of divisions along each axis for the spatial bins.
 The number of spatial bins is NumberOfXDivisions*NumberOfYDivisions*
 NumberOfZDivisions.  The filter may choose to ignore large numbers of
 divisions if the input has few points and AutoAdjustNumberOfDivisions
 is enabled.

\item  \verb|int = obj.GetNumberOfZDivisions ()| -  Set/Get the number of divisions along each axis for the spatial bins.
 The number of spatial bins is NumberOfXDivisions*NumberOfYDivisions*
 NumberOfZDivisions.  The filter may choose to ignore large numbers of
 divisions if the input has few points and AutoAdjustNumberOfDivisions
 is enabled.

\item  \verb|obj.SetNumberOfDivisions (int div[3])| -  Set/Get the number of divisions along each axis for the spatial bins.
 The number of spatial bins is NumberOfXDivisions*NumberOfYDivisions*
 NumberOfZDivisions.  The filter may choose to ignore large numbers of
 divisions if the input has few points and AutoAdjustNumberOfDivisions
 is enabled.

\item  \verb|obj.SetNumberOfDivisions (int div0, int div1, int div2)| -  Set/Get the number of divisions along each axis for the spatial bins.
 The number of spatial bins is NumberOfXDivisions*NumberOfYDivisions*
 NumberOfZDivisions.  The filter may choose to ignore large numbers of
 divisions if the input has few points and AutoAdjustNumberOfDivisions
 is enabled.

\item  \verb|int = obj.GetNumberOfDivisions ()| -  Set/Get the number of divisions along each axis for the spatial bins.
 The number of spatial bins is NumberOfXDivisions*NumberOfYDivisions*
 NumberOfZDivisions.  The filter may choose to ignore large numbers of
 divisions if the input has few points and AutoAdjustNumberOfDivisions
 is enabled.

\item  \verb|obj.GetNumberOfDivisions (int div[3])| -  Set/Get the number of divisions along each axis for the spatial bins.
 The number of spatial bins is NumberOfXDivisions*NumberOfYDivisions*
 NumberOfZDivisions.  The filter may choose to ignore large numbers of
 divisions if the input has few points and AutoAdjustNumberOfDivisions
 is enabled.

\item  \verb|obj.SetAutoAdjustNumberOfDivisions (int )| -  Enable automatic adjustment of number of divisions. If off, the number
 of divisions specified by the user is always used (as long as it is valid).
 The default is On

\item  \verb|int = obj.GetAutoAdjustNumberOfDivisions ()| -  Enable automatic adjustment of number of divisions. If off, the number
 of divisions specified by the user is always used (as long as it is valid).
 The default is On

\item  \verb|obj.AutoAdjustNumberOfDivisionsOn ()| -  Enable automatic adjustment of number of divisions. If off, the number
 of divisions specified by the user is always used (as long as it is valid).
 The default is On

\item  \verb|obj.AutoAdjustNumberOfDivisionsOff ()| -  Enable automatic adjustment of number of divisions. If off, the number
 of divisions specified by the user is always used (as long as it is valid).
 The default is On

\item  \verb|obj.SetDivisionOrigin (double x, double y, double z)| -  This is an alternative way to set up the bins.  If you are trying to match
 boundaries between pieces, then you should use these methods rather than
 SetNumberOfDivisions. To use these methods, specify the origin and spacing
 of the spatial binning.

\item  \verb|obj.SetDivisionOrigin (double o[3])| -  This is an alternative way to set up the bins.  If you are trying to match
 boundaries between pieces, then you should use these methods rather than
 SetNumberOfDivisions. To use these methods, specify the origin and spacing
 of the spatial binning.

\item  \verb|double = obj. GetDivisionOrigin ()| -  This is an alternative way to set up the bins.  If you are trying to match
 boundaries between pieces, then you should use these methods rather than
 SetNumberOfDivisions. To use these methods, specify the origin and spacing
 of the spatial binning.

\item  \verb|obj.SetDivisionSpacing (double x, double y, double z)| -  This is an alternative way to set up the bins.  If you are trying to match
 boundaries between pieces, then you should use these methods rather than
 SetNumberOfDivisions. To use these methods, specify the origin and spacing
 of the spatial binning.

\item  \verb|obj.SetDivisionSpacing (double s[3])| -  This is an alternative way to set up the bins.  If you are trying to match
 boundaries between pieces, then you should use these methods rather than
 SetNumberOfDivisions. To use these methods, specify the origin and spacing
 of the spatial binning.

\item  \verb|double = obj. GetDivisionSpacing ()| -  This is an alternative way to set up the bins.  If you are trying to match
 boundaries between pieces, then you should use these methods rather than
 SetNumberOfDivisions. To use these methods, specify the origin and spacing
 of the spatial binning.

\item  \verb|obj.SetUseInputPoints (int )| -  Normally the point that minimizes the quadric error function is used as
 the output of the bin.  When this flag is on, the bin point is forced to
 be one of the points from the input (the one with the smallest
 error). This option does not work (i.e., input points cannot be used)
 when the append methods (StartAppend(), Append(), EndAppend()) are being
 called directly.

\item  \verb|int = obj.GetUseInputPoints ()| -  Normally the point that minimizes the quadric error function is used as
 the output of the bin.  When this flag is on, the bin point is forced to
 be one of the points from the input (the one with the smallest
 error). This option does not work (i.e., input points cannot be used)
 when the append methods (StartAppend(), Append(), EndAppend()) are being
 called directly.

\item  \verb|obj.UseInputPointsOn ()| -  Normally the point that minimizes the quadric error function is used as
 the output of the bin.  When this flag is on, the bin point is forced to
 be one of the points from the input (the one with the smallest
 error). This option does not work (i.e., input points cannot be used)
 when the append methods (StartAppend(), Append(), EndAppend()) are being
 called directly.

\item  \verb|obj.UseInputPointsOff ()| -  Normally the point that minimizes the quadric error function is used as
 the output of the bin.  When this flag is on, the bin point is forced to
 be one of the points from the input (the one with the smallest
 error). This option does not work (i.e., input points cannot be used)
 when the append methods (StartAppend(), Append(), EndAppend()) are being
 called directly.

\item  \verb|obj.SetUseFeatureEdges (int )| -  By default, this flag is off.  When ''UseFeatureEdges'' is on, then
 quadrics are computed for boundary edges/feature edges.  They influence
 the quadrics (position of points), but not the mesh.  Which features to
 use can be controlled by the filter ''FeatureEdges''.

\item  \verb|int = obj.GetUseFeatureEdges ()| -  By default, this flag is off.  When ''UseFeatureEdges'' is on, then
 quadrics are computed for boundary edges/feature edges.  They influence
 the quadrics (position of points), but not the mesh.  Which features to
 use can be controlled by the filter ''FeatureEdges''.

\item  \verb|obj.UseFeatureEdgesOn ()| -  By default, this flag is off.  When ''UseFeatureEdges'' is on, then
 quadrics are computed for boundary edges/feature edges.  They influence
 the quadrics (position of points), but not the mesh.  Which features to
 use can be controlled by the filter ''FeatureEdges''.

\item  \verb|obj.UseFeatureEdgesOff ()| -  By default, this flag is off.  When ''UseFeatureEdges'' is on, then
 quadrics are computed for boundary edges/feature edges.  They influence
 the quadrics (position of points), but not the mesh.  Which features to
 use can be controlled by the filter ''FeatureEdges''.

\item  \verb|vtkFeatureEdges = obj.GetFeatureEdges ()| -  By default, this flag is off.  It only has an effect when
 ''UseFeatureEdges'' is also on.  When ''UseFeaturePoints'' is on, then
 quadrics are computed for boundary / feature points used in the boundary
 / feature edges.  They influence the quadrics (position of points), but
 not the mesh.

\item  \verb|obj.SetUseFeaturePoints (int )| -  By default, this flag is off.  It only has an effect when
 ''UseFeatureEdges'' is also on.  When ''UseFeaturePoints'' is on, then
 quadrics are computed for boundary / feature points used in the boundary
 / feature edges.  They influence the quadrics (position of points), but
 not the mesh.

\item  \verb|int = obj.GetUseFeaturePoints ()| -  By default, this flag is off.  It only has an effect when
 ''UseFeatureEdges'' is also on.  When ''UseFeaturePoints'' is on, then
 quadrics are computed for boundary / feature points used in the boundary
 / feature edges.  They influence the quadrics (position of points), but
 not the mesh.

\item  \verb|obj.UseFeaturePointsOn ()| -  By default, this flag is off.  It only has an effect when
 ''UseFeatureEdges'' is also on.  When ''UseFeaturePoints'' is on, then
 quadrics are computed for boundary / feature points used in the boundary
 / feature edges.  They influence the quadrics (position of points), but
 not the mesh.

\item  \verb|obj.UseFeaturePointsOff ()| -  By default, this flag is off.  It only has an effect when
 ''UseFeatureEdges'' is also on.  When ''UseFeaturePoints'' is on, then
 quadrics are computed for boundary / feature points used in the boundary
 / feature edges.  They influence the quadrics (position of points), but
 not the mesh.

\item  \verb|obj.SetFeaturePointsAngle (double )| -  Set/Get the angle to use in determining whether a point on a boundary /
 feature edge is a feature point.

\item  \verb|double = obj.GetFeaturePointsAngleMinValue ()| -  Set/Get the angle to use in determining whether a point on a boundary /
 feature edge is a feature point.

\item  \verb|double = obj.GetFeaturePointsAngleMaxValue ()| -  Set/Get the angle to use in determining whether a point on a boundary /
 feature edge is a feature point.

\item  \verb|double = obj.GetFeaturePointsAngle ()| -  Set/Get the angle to use in determining whether a point on a boundary /
 feature edge is a feature point.

\item  \verb|obj.SetUseInternalTriangles (int )| -  When this flag is on (and it is on by default), then triangles that are 
 completely contained in a bin are added to the bin quadrics.  When the
 the flag is off the filter operates faster, but the surface may not be
 as well behaved.

\item  \verb|int = obj.GetUseInternalTriangles ()| -  When this flag is on (and it is on by default), then triangles that are 
 completely contained in a bin are added to the bin quadrics.  When the
 the flag is off the filter operates faster, but the surface may not be
 as well behaved.

\item  \verb|obj.UseInternalTrianglesOn ()| -  When this flag is on (and it is on by default), then triangles that are 
 completely contained in a bin are added to the bin quadrics.  When the
 the flag is off the filter operates faster, but the surface may not be
 as well behaved.

\item  \verb|obj.UseInternalTrianglesOff ()| -  When this flag is on (and it is on by default), then triangles that are 
 completely contained in a bin are added to the bin quadrics.  When the
 the flag is off the filter operates faster, but the surface may not be
 as well behaved.

\item  \verb|obj.StartAppend (double bounds)| -  These methods provide an alternative way of executing the filter.
 PolyData can be added to the result in pieces (append).
 In this mode, the user must specify the bounds of the entire model
 as an argument to the ''StartAppend'' method.

\item  \verb|obj.StartAppend (double x0, double x1, double y0, double y1, double z0, double z1)| -  These methods provide an alternative way of executing the filter.
 PolyData can be added to the result in pieces (append).
 In this mode, the user must specify the bounds of the entire model
 as an argument to the ''StartAppend'' method.

\item  \verb|obj.Append (vtkPolyData piece)| -  These methods provide an alternative way of executing the filter.
 PolyData can be added to the result in pieces (append).
 In this mode, the user must specify the bounds of the entire model
 as an argument to the ''StartAppend'' method.

\item  \verb|obj.EndAppend ()| -  These methods provide an alternative way of executing the filter.
 PolyData can be added to the result in pieces (append).
 In this mode, the user must specify the bounds of the entire model
 as an argument to the ''StartAppend'' method.

\item  \verb|obj.SetCopyCellData (int )| -  This flag makes the filter copy cell data from input to output 
 (the best it can).  It uses input cells that trigger the addition
 of output cells (no averaging).  This is off by default, and does
 not work when append is being called explicitely (non-pipeline usage).

\item  \verb|int = obj.GetCopyCellData ()| -  This flag makes the filter copy cell data from input to output 
 (the best it can).  It uses input cells that trigger the addition
 of output cells (no averaging).  This is off by default, and does
 not work when append is being called explicitely (non-pipeline usage).

\item  \verb|obj.CopyCellDataOn ()| -  This flag makes the filter copy cell data from input to output 
 (the best it can).  It uses input cells that trigger the addition
 of output cells (no averaging).  This is off by default, and does
 not work when append is being called explicitely (non-pipeline usage).

\item  \verb|obj.CopyCellDataOff ()| -  This flag makes the filter copy cell data from input to output 
 (the best it can).  It uses input cells that trigger the addition
 of output cells (no averaging).  This is off by default, and does
 not work when append is being called explicitely (non-pipeline usage).

\item  \verb|obj.SetPreventDuplicateCells (int )| -  Specify a boolean indicating whether to remove duplicate cells
 (i.e. triangles).  This is a little slower, and takes more memory, but
 in some cases can reduce the number of cells produced by an order of
 magnitude. By default, this flag is true.

\item  \verb|int = obj.GetPreventDuplicateCells ()| -  Specify a boolean indicating whether to remove duplicate cells
 (i.e. triangles).  This is a little slower, and takes more memory, but
 in some cases can reduce the number of cells produced by an order of
 magnitude. By default, this flag is true.

\item  \verb|obj.PreventDuplicateCellsOn ()| -  Specify a boolean indicating whether to remove duplicate cells
 (i.e. triangles).  This is a little slower, and takes more memory, but
 in some cases can reduce the number of cells produced by an order of
 magnitude. By default, this flag is true.

\item  \verb|obj.PreventDuplicateCellsOff ()| -  Specify a boolean indicating whether to remove duplicate cells
 (i.e. triangles).  This is a little slower, and takes more memory, but
 in some cases can reduce the number of cells produced by an order of
 magnitude. By default, this flag is true.

\end{itemize}
