\section{vtkSQLDatabaseTableSource}

\subsection{Usage}

 This class combines vtkSQLDatabase, vtkSQLQuery, and vtkQueryToTable to
 provide a convenience class for generating tables from databases. 
 Also this class can be easily wrapped and used within ParaView / OverView.

To create an instance of class vtkSQLDatabaseTableSource, simply
invoke its constructor as follows
\begin{verbatim}
  obj = vtkSQLDatabaseTableSource
\end{verbatim}
\subsection{Methods}

The class vtkSQLDatabaseTableSource has several methods that can be used.
  They are listed below.
Note that the documentation is translated automatically from the VTK sources,
and may not be completely intelligible.  When in doubt, consult the VTK website.
In the methods listed below, \verb|obj| is an instance of the vtkSQLDatabaseTableSource class.
\begin{itemize}
\item  \verb|string = obj.GetClassName ()|

\item  \verb|int = obj.IsA (string name)|

\item  \verb|vtkSQLDatabaseTableSource = obj.NewInstance ()|

\item  \verb|vtkSQLDatabaseTableSource = obj.SafeDownCast (vtkObject o)|

\item  \verb|vtkStdString = obj.GetURL ()|

\item  \verb|obj.SetURL (vtkStdString \&url)|

\item  \verb|obj.SetPassword (vtkStdString \&password)|

\item  \verb|vtkStdString = obj.GetQuery ()|

\item  \verb|obj.SetQuery (vtkStdString \&query)|

\item  \verb|obj.SetPedigreeIdArrayName (string )| -  The name of the array for generating or assigning pedigree ids
 (default ''id'').

\item  \verb|string = obj.GetPedigreeIdArrayName ()| -  The name of the array for generating or assigning pedigree ids
 (default ''id'').

\item  \verb|obj.SetGeneratePedigreeIds (bool )| -  If on (default), generates pedigree ids automatically.
 If off, assign one of the arrays to be the pedigree id.

\item  \verb|bool = obj.GetGeneratePedigreeIds ()| -  If on (default), generates pedigree ids automatically.
 If off, assign one of the arrays to be the pedigree id.

\item  \verb|obj.GeneratePedigreeIdsOn ()| -  If on (default), generates pedigree ids automatically.
 If off, assign one of the arrays to be the pedigree id.

\item  \verb|obj.GeneratePedigreeIdsOff ()| -  If on (default), generates pedigree ids automatically.
 If off, assign one of the arrays to be the pedigree id.

\end{itemize}
