\section{vtkObjectBase}

\subsection{Usage}

 vtkObjectBase is the base class for all reference counted classes
 in the VTK. These classes include vtkCommand classes, vtkInformationKey
 classes, and vtkObject classes.

 vtkObjectBase performs reference counting: objects that are
 reference counted exist as long as another object uses them. Once
 the last reference to a reference counted object is removed, the
 object will spontaneously destruct.

 Constructor and destructor of the subclasses of vtkObjectBase
 should be protected, so that only New() and UnRegister() actually
 call them. Debug leaks can be used to see if there are any objects
 left with nonzero reference count.


To create an instance of class vtkObjectBase, simply
invoke its constructor as follows
\begin{verbatim}
  obj = vtkObjectBase
\end{verbatim}
\subsection{Methods}

The class vtkObjectBase has several methods that can be used.
  They are listed below.
Note that the documentation is translated automatically from the VTK sources,
and may not be completely intelligible.  When in doubt, consult the VTK website.
In the methods listed below, \verb|obj| is an instance of the vtkObjectBase class.
\begin{itemize}
\item  \verb|string = obj.GetClassName () const| -  Return the class name as a string. This method is defined
 in all subclasses of vtkObjectBase with the vtkTypeRevisionMacro found
 in vtkSetGet.h.

\item  \verb|int = obj.IsA (string name)| -  Return 1 if this class is the same type of (or a subclass of)
 the named class. Returns 0 otherwise. This method works in
 combination with vtkTypeRevisionMacro found in vtkSetGet.h.

\item  \verb|obj.Delete ()| -  Delete a VTK object.  This method should always be used to delete
 an object when the New() method was used to create it. Using the
 C++ delete method will not work with reference counting.

\item  \verb|obj.FastDelete ()| -  Delete a reference to this object.  This version will not invoke
 garbage collection and can potentially leak the object if it is
 part of a reference loop.  Use this method only when it is known
 that the object has another reference and would not be collected
 if a full garbage collection check were done.

\item  \verb|obj.Register (vtkObjectBase o)| -  Increase the reference count (mark as used by another object).

\item  \verb|obj.UnRegister (vtkObjectBase o)| -  Decrease the reference count (release by another object). This
 has the same effect as invoking Delete() (i.e., it reduces the
 reference count by 1).

\item  \verb|int = obj.GetReferenceCount ()| -  Sets the reference count. (This is very dangerous, use with care.)

\item  \verb|obj.SetReferenceCount (int )| -  Sets the reference count. (This is very dangerous, use with care.)

\end{itemize}
