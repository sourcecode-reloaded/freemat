\section{vtkCubeAxesActor2D}

\subsection{Usage}

 vtkCubeAxesActor2D is a composite actor that draws three axes of the 
 bounding box of an input dataset. The axes include labels and titles
 for the x-y-z axes. The algorithm selects the axes that are on the 
 ''exterior'' of the bounding box, exterior as determined from examining
 outer edges of the bounding box in projection (display) space. Alternatively,
 the edges closest to the viewer (i.e., camera position) can be drawn.
 
 To use this object you must define a bounding box and the camera used
 to render the vtkCubeAxesActor2D. The camera is used to control the 
 scaling and position of the vtkCubeAxesActor2D so that it fits in the
 viewport and always remains visible.)

 The font property of the axes titles and labels can be modified through the
 AxisTitleTextProperty and AxisLabelTextProperty attributes. You may also
 use the GetXAxisActor2D, GetYAxisActor2D or GetZAxisActor2D methods 
 to access each individual axis actor to modify their font properties.

 The bounding box to use is defined in one of three ways. First, if the Input
 ivar is defined, then the input dataset's bounds is used. If the Input is 
 not defined, and the Prop (superclass of all actors) is defined, then the
 Prop's bounds is used. If neither the Input or Prop is defined, then the
 Bounds instance variable (an array of six doubles) is used.
 

To create an instance of class vtkCubeAxesActor2D, simply
invoke its constructor as follows
\begin{verbatim}
  obj = vtkCubeAxesActor2D
\end{verbatim}
\subsection{Methods}

The class vtkCubeAxesActor2D has several methods that can be used.
  They are listed below.
Note that the documentation is translated automatically from the VTK sources,
and may not be completely intelligible.  When in doubt, consult the VTK website.
In the methods listed below, \verb|obj| is an instance of the vtkCubeAxesActor2D class.
\begin{itemize}
\item  \verb|string = obj.GetClassName ()|

\item  \verb|int = obj.IsA (string name)|

\item  \verb|vtkCubeAxesActor2D = obj.NewInstance ()|

\item  \verb|vtkCubeAxesActor2D = obj.SafeDownCast (vtkObject o)|

\item  \verb|int = obj.RenderOverlay (vtkViewport )| -  Draw the axes as per the vtkProp superclass' API.

\item  \verb|int = obj.RenderOpaqueGeometry (vtkViewport )| -  Draw the axes as per the vtkProp superclass' API.

\item  \verb|int = obj.RenderTranslucentPolygonalGeometry (vtkViewport )| -  Does this prop have some translucent polygonal geometry?

\item  \verb|int = obj.HasTranslucentPolygonalGeometry ()| -  Does this prop have some translucent polygonal geometry?

\item  \verb|obj.SetInput (vtkDataSet )| -  Use the bounding box of this input dataset to draw the cube axes. If this
 is not specified, then the class will attempt to determine the bounds from
 the defined Prop or Bounds.

\item  \verb|vtkDataSet = obj.GetInput ()| -  Use the bounding box of this input dataset to draw the cube axes. If this
 is not specified, then the class will attempt to determine the bounds from
 the defined Prop or Bounds.

\item  \verb|obj.SetViewProp (vtkProp prop)| -  Use the bounding box of this prop to draw the cube axes. The
 ViewProp is used to determine the bounds only if the Input is not
 defined.

\item  \verb|vtkProp = obj.GetViewProp ()| -  Use the bounding box of this prop to draw the cube axes. The
 ViewProp is used to determine the bounds only if the Input is not
 defined.

\item  \verb|obj.SetBounds (double , double , double , double , double , double )| -  Explicitly specify the region in space around which to draw the bounds.
 The bounds is used only when no Input or Prop is specified. The bounds
 are specified according to (xmin,xmax, ymin,ymax, zmin,zmax), making
 sure that the min's are less than the max's.

\item  \verb|obj.SetBounds (double  a[6])| -  Explicitly specify the region in space around which to draw the bounds.
 The bounds is used only when no Input or Prop is specified. The bounds
 are specified according to (xmin,xmax, ymin,ymax, zmin,zmax), making
 sure that the min's are less than the max's.

\item  \verb|double = obj.GetBounds ()| -  Explicitly specify the region in space around which to draw the bounds.
 The bounds is used only when no Input or Prop is specified. The bounds
 are specified according to (xmin,xmax, ymin,ymax, zmin,zmax), making
 sure that the min's are less than the max's.

\item  \verb|obj.GetBounds (double bounds[6])| -  Explicitly specify the region in space around which to draw the bounds.
 The bounds is used only when no Input or Prop is specified. The bounds
 are specified according to (xmin,xmax, ymin,ymax, zmin,zmax), making
 sure that the min's are less than the max's.

\item  \verb|obj.SetRanges (double , double , double , double , double , double )| -  Explicitly specify the range of values used on the bounds.
 The ranges are specified according to (xmin,xmax, ymin,ymax, zmin,zmax), 
 making sure that the min's are less than the max's.

\item  \verb|obj.SetRanges (double  a[6])| -  Explicitly specify the range of values used on the bounds.
 The ranges are specified according to (xmin,xmax, ymin,ymax, zmin,zmax), 
 making sure that the min's are less than the max's.

\item  \verb|double = obj.GetRanges ()| -  Explicitly specify the range of values used on the bounds.
 The ranges are specified according to (xmin,xmax, ymin,ymax, zmin,zmax), 
 making sure that the min's are less than the max's.

\item  \verb|obj.GetRanges (double ranges[6])| -  Explicitly specify the range of values used on the bounds.
 The ranges are specified according to (xmin,xmax, ymin,ymax, zmin,zmax), 
 making sure that the min's are less than the max's.

\item  \verb|obj.SetXOrigin (double )| -  Explicitly specify an origin for the axes. These usually intersect at one of the
 corners of the bounding box, however users have the option to override this if
 necessary

\item  \verb|obj.SetYOrigin (double )| -  Explicitly specify an origin for the axes. These usually intersect at one of the
 corners of the bounding box, however users have the option to override this if
 necessary

\item  \verb|obj.SetZOrigin (double )| -  Explicitly specify an origin for the axes. These usually intersect at one of the
 corners of the bounding box, however users have the option to override this if
 necessary

\item  \verb|obj.SetUseRanges (int )| -  Set/Get a flag that controls whether the axes use the data ranges
 or the ranges set by SetRanges. By default the axes use the data
 ranges.

\item  \verb|int = obj.GetUseRanges ()| -  Set/Get a flag that controls whether the axes use the data ranges
 or the ranges set by SetRanges. By default the axes use the data
 ranges.

\item  \verb|obj.UseRangesOn ()| -  Set/Get a flag that controls whether the axes use the data ranges
 or the ranges set by SetRanges. By default the axes use the data
 ranges.

\item  \verb|obj.UseRangesOff ()| -  Set/Get a flag that controls whether the axes use the data ranges
 or the ranges set by SetRanges. By default the axes use the data
 ranges.

\item  \verb|obj.SetCamera (vtkCamera )| -  Set/Get the camera to perform scaling and translation of the 
 vtkCubeAxesActor2D.

\item  \verb|vtkCamera = obj.GetCamera ()| -  Set/Get the camera to perform scaling and translation of the 
 vtkCubeAxesActor2D.

\item  \verb|obj.SetFlyMode (int )| -  Specify a mode to control how the axes are drawn: either outer edges
 or closest triad to the camera position, or you may also disable flying 
 of the axes.

\item  \verb|int = obj.GetFlyModeMinValue ()| -  Specify a mode to control how the axes are drawn: either outer edges
 or closest triad to the camera position, or you may also disable flying 
 of the axes.

\item  \verb|int = obj.GetFlyModeMaxValue ()| -  Specify a mode to control how the axes are drawn: either outer edges
 or closest triad to the camera position, or you may also disable flying 
 of the axes.

\item  \verb|int = obj.GetFlyMode ()| -  Specify a mode to control how the axes are drawn: either outer edges
 or closest triad to the camera position, or you may also disable flying 
 of the axes.

\item  \verb|obj.SetFlyModeToOuterEdges ()| -  Specify a mode to control how the axes are drawn: either outer edges
 or closest triad to the camera position, or you may also disable flying 
 of the axes.

\item  \verb|obj.SetFlyModeToClosestTriad ()| -  Specify a mode to control how the axes are drawn: either outer edges
 or closest triad to the camera position, or you may also disable flying 
 of the axes.

\item  \verb|obj.SetFlyModeToNone ()| -  Specify a mode to control how the axes are drawn: either outer edges
 or closest triad to the camera position, or you may also disable flying 
 of the axes.

\item  \verb|obj.SetScaling (int )| -  Set/Get a flag that controls whether the axes are scaled to fit in
 the viewport. If off, the axes size remains constant (i.e., stay the
 size of the bounding box). By default scaling is on so the axes are
 scaled to fit inside the viewport.

\item  \verb|int = obj.GetScaling ()| -  Set/Get a flag that controls whether the axes are scaled to fit in
 the viewport. If off, the axes size remains constant (i.e., stay the
 size of the bounding box). By default scaling is on so the axes are
 scaled to fit inside the viewport.

\item  \verb|obj.ScalingOn ()| -  Set/Get a flag that controls whether the axes are scaled to fit in
 the viewport. If off, the axes size remains constant (i.e., stay the
 size of the bounding box). By default scaling is on so the axes are
 scaled to fit inside the viewport.

\item  \verb|obj.ScalingOff ()| -  Set/Get a flag that controls whether the axes are scaled to fit in
 the viewport. If off, the axes size remains constant (i.e., stay the
 size of the bounding box). By default scaling is on so the axes are
 scaled to fit inside the viewport.

\item  \verb|obj.SetNumberOfLabels (int )| -  Set/Get the number of annotation labels to show along the x, y, and 
 z axes. This values is a suggestion: the number of labels may vary
 depending on the particulars of the data.

\item  \verb|int = obj.GetNumberOfLabelsMinValue ()| -  Set/Get the number of annotation labels to show along the x, y, and 
 z axes. This values is a suggestion: the number of labels may vary
 depending on the particulars of the data.

\item  \verb|int = obj.GetNumberOfLabelsMaxValue ()| -  Set/Get the number of annotation labels to show along the x, y, and 
 z axes. This values is a suggestion: the number of labels may vary
 depending on the particulars of the data.

\item  \verb|int = obj.GetNumberOfLabels ()| -  Set/Get the number of annotation labels to show along the x, y, and 
 z axes. This values is a suggestion: the number of labels may vary
 depending on the particulars of the data.

\item  \verb|obj.SetXLabel (string )| -  Set/Get the labels for the x, y, and z axes. By default, 
 use ''X'', ''Y'' and ''Z''.

\item  \verb|string = obj.GetXLabel ()| -  Set/Get the labels for the x, y, and z axes. By default, 
 use ''X'', ''Y'' and ''Z''.

\item  \verb|obj.SetYLabel (string )| -  Set/Get the labels for the x, y, and z axes. By default, 
 use ''X'', ''Y'' and ''Z''.

\item  \verb|string = obj.GetYLabel ()| -  Set/Get the labels for the x, y, and z axes. By default, 
 use ''X'', ''Y'' and ''Z''.

\item  \verb|obj.SetZLabel (string )| -  Set/Get the labels for the x, y, and z axes. By default, 
 use ''X'', ''Y'' and ''Z''.

\item  \verb|string = obj.GetZLabel ()| -  Set/Get the labels for the x, y, and z axes. By default, 
 use ''X'', ''Y'' and ''Z''.

\item  \verb|vtkAxisActor2D = obj.GetXAxisActor2D ()| -  Retrieve handles to the X, Y and Z axis (so that you can set their text
 properties for example)

\item  \verb|vtkAxisActor2D = obj.GetYAxisActor2D ()| -  Retrieve handles to the X, Y and Z axis (so that you can set their text
 properties for example)

\item  \verb|vtkAxisActor2D = obj.GetZAxisActor2D ()| -  Set/Get the title text property of all axes. Note that each axis can
 be controlled individually through the GetX/Y/ZAxisActor2D() methods.

\item  \verb|obj.SetAxisTitleTextProperty (vtkTextProperty p)| -  Set/Get the title text property of all axes. Note that each axis can
 be controlled individually through the GetX/Y/ZAxisActor2D() methods.

\item  \verb|vtkTextProperty = obj.GetAxisTitleTextProperty ()| -  Set/Get the title text property of all axes. Note that each axis can
 be controlled individually through the GetX/Y/ZAxisActor2D() methods.

\item  \verb|obj.SetAxisLabelTextProperty (vtkTextProperty p)| -  Set/Get the labels text property of all axes. Note that each axis can
 be controlled individually through the GetX/Y/ZAxisActor2D() methods.

\item  \verb|vtkTextProperty = obj.GetAxisLabelTextProperty ()| -  Set/Get the labels text property of all axes. Note that each axis can
 be controlled individually through the GetX/Y/ZAxisActor2D() methods.

\item  \verb|obj.SetLabelFormat (string )| -  Set/Get the format with which to print the labels on each of the
 x-y-z axes.

\item  \verb|string = obj.GetLabelFormat ()| -  Set/Get the format with which to print the labels on each of the
 x-y-z axes.

\item  \verb|obj.SetFontFactor (double )| -  Set/Get the factor that controls the overall size of the fonts used
 to label and title the axes. 

\item  \verb|double = obj.GetFontFactorMinValue ()| -  Set/Get the factor that controls the overall size of the fonts used
 to label and title the axes. 

\item  \verb|double = obj.GetFontFactorMaxValue ()| -  Set/Get the factor that controls the overall size of the fonts used
 to label and title the axes. 

\item  \verb|double = obj.GetFontFactor ()| -  Set/Get the factor that controls the overall size of the fonts used
 to label and title the axes. 

\item  \verb|obj.SetInertia (int )| -  Set/Get the inertial factor that controls how often (i.e, how
 many renders) the axes can switch position (jump from one axes 
 to another).

\item  \verb|int = obj.GetInertiaMinValue ()| -  Set/Get the inertial factor that controls how often (i.e, how
 many renders) the axes can switch position (jump from one axes 
 to another).

\item  \verb|int = obj.GetInertiaMaxValue ()| -  Set/Get the inertial factor that controls how often (i.e, how
 many renders) the axes can switch position (jump from one axes 
 to another).

\item  \verb|int = obj.GetInertia ()| -  Set/Get the inertial factor that controls how often (i.e, how
 many renders) the axes can switch position (jump from one axes 
 to another).

\item  \verb|obj.SetShowActualBounds (int )| -  Set/Get the variable that controls whether the actual
 bounds of the dataset are always shown. Setting this variable
 to 1 means that clipping is disabled and that the actual
 value of the bounds is displayed even with corner offsets
 Setting this variable to 0 means these axis will clip
 themselves and show variable bounds (legacy mode)

\item  \verb|int = obj.GetShowActualBoundsMinValue ()| -  Set/Get the variable that controls whether the actual
 bounds of the dataset are always shown. Setting this variable
 to 1 means that clipping is disabled and that the actual
 value of the bounds is displayed even with corner offsets
 Setting this variable to 0 means these axis will clip
 themselves and show variable bounds (legacy mode)

\item  \verb|int = obj.GetShowActualBoundsMaxValue ()| -  Set/Get the variable that controls whether the actual
 bounds of the dataset are always shown. Setting this variable
 to 1 means that clipping is disabled and that the actual
 value of the bounds is displayed even with corner offsets
 Setting this variable to 0 means these axis will clip
 themselves and show variable bounds (legacy mode)

\item  \verb|int = obj.GetShowActualBounds ()| -  Set/Get the variable that controls whether the actual
 bounds of the dataset are always shown. Setting this variable
 to 1 means that clipping is disabled and that the actual
 value of the bounds is displayed even with corner offsets
 Setting this variable to 0 means these axis will clip
 themselves and show variable bounds (legacy mode)

\item  \verb|obj.SetCornerOffset (double )| -  Specify an offset value to ''pull back'' the axes from the corner at
 which they are joined to avoid overlap of axes labels. The 
 ''CornerOffset'' is the fraction of the axis length to pull back.

\item  \verb|double = obj.GetCornerOffset ()| -  Specify an offset value to ''pull back'' the axes from the corner at
 which they are joined to avoid overlap of axes labels. The 
 ''CornerOffset'' is the fraction of the axis length to pull back.

\item  \verb|obj.ReleaseGraphicsResources (vtkWindow )| -  Release any graphics resources that are being consumed by this actor.
 The parameter window could be used to determine which graphic
 resources to release.

\item  \verb|obj.SetXAxisVisibility (int )| -  Turn on and off the visibility of each axis.

\item  \verb|int = obj.GetXAxisVisibility ()| -  Turn on and off the visibility of each axis.

\item  \verb|obj.XAxisVisibilityOn ()| -  Turn on and off the visibility of each axis.

\item  \verb|obj.XAxisVisibilityOff ()| -  Turn on and off the visibility of each axis.

\item  \verb|obj.SetYAxisVisibility (int )| -  Turn on and off the visibility of each axis.

\item  \verb|int = obj.GetYAxisVisibility ()| -  Turn on and off the visibility of each axis.

\item  \verb|obj.YAxisVisibilityOn ()| -  Turn on and off the visibility of each axis.

\item  \verb|obj.YAxisVisibilityOff ()| -  Turn on and off the visibility of each axis.

\item  \verb|obj.SetZAxisVisibility (int )| -  Turn on and off the visibility of each axis.

\item  \verb|int = obj.GetZAxisVisibility ()| -  Turn on and off the visibility of each axis.

\item  \verb|obj.ZAxisVisibilityOn ()| -  Turn on and off the visibility of each axis.

\item  \verb|obj.ZAxisVisibilityOff ()| -  Turn on and off the visibility of each axis.

\item  \verb|obj.ShallowCopy (vtkCubeAxesActor2D actor)| -  Shallow copy of a CubeAxesActor2D.

\item  \verb|obj.SetProp (vtkProp prop)| -  @deprecated Replaced by vtkCubeAxesActor2D::SetViewProp() as of VTK 5.0.

\item  \verb|vtkProp = obj.GetProp ()| -  @deprecated Replaced by vtkCubeAxesActor2D::GetViewProp() as of VTK 5.0.

\end{itemize}
