\section{vtkImageAccumulate}

\subsection{Usage}

 vtkImageAccumulate - This filter divides component space into
 discrete bins.  It then counts the number of pixels associated
 with each bin.  The output is this ''scatter plot'' (histogram values for 1D).
 The dimensionality of the output depends on how many components the
 input pixels have.  Input pixels with one component generate a 1D histogram.
 This filter can only handle images with 1 to 3 scalar components.
 The input can be any type, but the output is always int.
 Some statistics are computed on the pixel values at the same time.
 The SetStencil and ReverseStencil
 functions allow the statistics to be computed on an arbitrary
 portion of the input data.
 See the documentation for vtkImageStencilData for more information.

 This filter also support ignoring pixel with value equal to 0. Using this
 option with vtkImageMask may result in results being slightly off since 0
 could be a valid value from your input.


To create an instance of class vtkImageAccumulate, simply
invoke its constructor as follows
\begin{verbatim}
  obj = vtkImageAccumulate
\end{verbatim}
\subsection{Methods}

The class vtkImageAccumulate has several methods that can be used.
  They are listed below.
Note that the documentation is translated automatically from the VTK sources,
and may not be completely intelligible.  When in doubt, consult the VTK website.
In the methods listed below, \verb|obj| is an instance of the vtkImageAccumulate class.
\begin{itemize}
\item  \verb|string = obj.GetClassName ()|

\item  \verb|int = obj.IsA (string name)|

\item  \verb|vtkImageAccumulate = obj.NewInstance ()|

\item  \verb|vtkImageAccumulate = obj.SafeDownCast (vtkObject o)|

\item  \verb|obj.SetComponentSpacing (double , double , double )| -  Set/Get - The component spacing is the dimension of each bin.
 This ends up being the spacing of the output ''image''.
 If the number of input scalar components are less than three,
 then some of these spacing values are ignored.
 For a 1D histogram with 10 bins spanning the values 1000 to 2000,
 this spacing should be set to 100, 0, 0

\item  \verb|obj.SetComponentSpacing (double  a[3])| -  Set/Get - The component spacing is the dimension of each bin.
 This ends up being the spacing of the output ''image''.
 If the number of input scalar components are less than three,
 then some of these spacing values are ignored.
 For a 1D histogram with 10 bins spanning the values 1000 to 2000,
 this spacing should be set to 100, 0, 0

\item  \verb|double = obj. GetComponentSpacing ()| -  Set/Get - The component spacing is the dimension of each bin.
 This ends up being the spacing of the output ''image''.
 If the number of input scalar components are less than three,
 then some of these spacing values are ignored.
 For a 1D histogram with 10 bins spanning the values 1000 to 2000,
 this spacing should be set to 100, 0, 0

\item  \verb|obj.SetComponentOrigin (double , double , double )| -  Set/Get - The component origin is the location of bin (0, 0, 0).
 Note that if the Component extent does not include the value (0,0,0),
 then this origin bin will not actually be in the output.
 The origin of the output ends up being the same as the componenet origin.
 For a 1D histogram with 10 bins spanning the values 1000 to 2000,
 this origin should be set to 1000, 0, 0

\item  \verb|obj.SetComponentOrigin (double  a[3])| -  Set/Get - The component origin is the location of bin (0, 0, 0).
 Note that if the Component extent does not include the value (0,0,0),
 then this origin bin will not actually be in the output.
 The origin of the output ends up being the same as the componenet origin.
 For a 1D histogram with 10 bins spanning the values 1000 to 2000,
 this origin should be set to 1000, 0, 0

\item  \verb|double = obj. GetComponentOrigin ()| -  Set/Get - The component origin is the location of bin (0, 0, 0).
 Note that if the Component extent does not include the value (0,0,0),
 then this origin bin will not actually be in the output.
 The origin of the output ends up being the same as the componenet origin.
 For a 1D histogram with 10 bins spanning the values 1000 to 2000,
 this origin should be set to 1000, 0, 0

\item  \verb|obj.SetComponentExtent (int extent[6])| -  Set/Get - The component extent sets the number/extent of the bins.
 For a 1D histogram with 10 bins spanning the values 1000 to 2000,
 this extent should be set to 0, 9, 0, 0, 0, 0.
 The extent specifies inclusive min/max values.
 This implies that the top extent should be set to the number of bins - 1.

\item  \verb|obj.SetComponentExtent (int minX, int maxX, int minY, int maxY, int minZ, int maxZ)| -  Set/Get - The component extent sets the number/extent of the bins.
 For a 1D histogram with 10 bins spanning the values 1000 to 2000,
 this extent should be set to 0, 9, 0, 0, 0, 0.
 The extent specifies inclusive min/max values.
 This implies that the top extent should be set to the number of bins - 1.

\item  \verb|obj.GetComponentExtent (int extent[6])| -  Set/Get - The component extent sets the number/extent of the bins.
 For a 1D histogram with 10 bins spanning the values 1000 to 2000,
 this extent should be set to 0, 9, 0, 0, 0, 0.
 The extent specifies inclusive min/max values.
 This implies that the top extent should be set to the number of bins - 1.

\item  \verb|int = obj.GetComponentExtent ()| -  Use a stencil to specify which voxels to accumulate.

\item  \verb|obj.SetStencil (vtkImageStencilData stencil)| -  Use a stencil to specify which voxels to accumulate.

\item  \verb|vtkImageStencilData = obj.GetStencil ()| -  Use a stencil to specify which voxels to accumulate.

\item  \verb|obj.SetReverseStencil (int )| -  Reverse the stencil.

\item  \verb|int = obj.GetReverseStencilMinValue ()| -  Reverse the stencil.

\item  \verb|int = obj.GetReverseStencilMaxValue ()| -  Reverse the stencil.

\item  \verb|obj.ReverseStencilOn ()| -  Reverse the stencil.

\item  \verb|obj.ReverseStencilOff ()| -  Reverse the stencil.

\item  \verb|int = obj.GetReverseStencil ()| -  Reverse the stencil.

\item  \verb|double = obj. GetMin ()| -  Get the statistics information for the data.

\item  \verb|double = obj. GetMax ()| -  Get the statistics information for the data.

\item  \verb|double = obj. GetMean ()| -  Get the statistics information for the data.

\item  \verb|double = obj. GetStandardDeviation ()| -  Get the statistics information for the data.

\item  \verb|long = obj.GetVoxelCount ()| -  Get the statistics information for the data.

\item  \verb|obj.SetIgnoreZero (int )| -  Should the data with value 0 be ignored?

\item  \verb|int = obj.GetIgnoreZeroMinValue ()| -  Should the data with value 0 be ignored?

\item  \verb|int = obj.GetIgnoreZeroMaxValue ()| -  Should the data with value 0 be ignored?

\item  \verb|int = obj.GetIgnoreZero ()| -  Should the data with value 0 be ignored?

\item  \verb|obj.IgnoreZeroOn ()| -  Should the data with value 0 be ignored?

\item  \verb|obj.IgnoreZeroOff ()| -  Should the data with value 0 be ignored?

\end{itemize}
