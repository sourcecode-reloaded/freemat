\section{vtkInformationInformationKey}

\subsection{Usage}

 vtkInformationInformationKey is used to represent keys in vtkInformation
 for other information objects.

To create an instance of class vtkInformationInformationKey, simply
invoke its constructor as follows
\begin{verbatim}
  obj = vtkInformationInformationKey
\end{verbatim}
\subsection{Methods}

The class vtkInformationInformationKey has several methods that can be used.
  They are listed below.
Note that the documentation is translated automatically from the VTK sources,
and may not be completely intelligible.  When in doubt, consult the VTK website.
In the methods listed below, \verb|obj| is an instance of the vtkInformationInformationKey class.
\begin{itemize}
\item  \verb|string = obj.GetClassName ()|

\item  \verb|int = obj.IsA (string name)|

\item  \verb|vtkInformationInformationKey = obj.NewInstance ()|

\item  \verb|vtkInformationInformationKey = obj.SafeDownCast (vtkObject o)|

\item  \verb|vtkInformationInformationKey = obj.(string name, string location)|

\item  \verb|~vtkInformationInformationKey = obj.()|

\item  \verb|obj.Set (vtkInformation info, vtkInformation )| -  Get/Set the value associated with this key in the given
 information object.

\item  \verb|vtkInformation = obj.Get (vtkInformation info)| -  Get/Set the value associated with this key in the given
 information object.

\item  \verb|obj.ShallowCopy (vtkInformation from, vtkInformation to)| -  Copy the entry associated with this key from one information
 object to another.  If there is no entry in the first information
 object for this key, the value is removed from the second.

\item  \verb|obj.DeepCopy (vtkInformation from, vtkInformation to)| -  Duplicate (new instance created) the entry associated with this key from
 one information object to another (new instances of any contained
 vtkInformation and vtkInformationVector objects are created).  

\end{itemize}
