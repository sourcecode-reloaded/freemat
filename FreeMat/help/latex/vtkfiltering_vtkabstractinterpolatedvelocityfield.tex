\section{vtkAbstractInterpolatedVelocityField}

\subsection{Usage}

  vtkAbstractInterpolatedVelocityField acts as a continuous velocity field
  by performing cell interpolation on the underlying vtkDataSet. This is an
  abstract sub-class of vtkFunctionSet, NumberOfIndependentVariables = 4
  (x,y,z,t) and NumberOfFunctions = 3 (u,v,w). With a brute-force scheme,
  every time an evaluation is performed, the target cell containing point 
  (x,y,z) needs to be found by calling FindCell(), via either vtkDataSet or
  vtkAbstractCelllocator's sub-classes (vtkCellLocator \& vtkModifiedBSPTree). 
  As it incurs a large cost, one (for vtkCellLocatorInterpolatedVelocityField
  via vtkAbstractCellLocator) or two (for vtkInterpolatedVelocityField via 
  vtkDataSet that involves vtkPointLocator in addressing vtkPointSet) levels
  of cell caching may be exploited to increase the performance.

  For vtkInterpolatedVelocityField, level \#0 begins with intra-cell caching. 
  Specifically if the previous cell is valid and the next point is still in
  it ( i.e., vtkCell::EvaluatePosition() returns 1, coupled with newly created 
  parametric coordinates \& weights ), the function values can be interpolated
  and only vtkCell::EvaluatePosition() is invoked. If this fails, then level \#1 
  follows by inter-cell search for the target cell that contains the next point. 
  By an inter-cell search, the previous cell provides an important clue or serves 
  as an immediate neighbor to aid in locating the target cell via vtkPointSet::
  FindCell(). If this still fails, a global cell location / search is invoked via
  vtkPointSet::FindCell(). Here regardless of either inter-cell or global search, 
  vtkPointLocator is in fact employed (for datasets of type vtkPointSet only, note
  vtkImageData and vtkRectilinearGrid are able to provide rapid and robust cell 
  location due to the simple mesh topology) as a crucial tool underlying the cell 
  locator. However, the use of vtkPointLocator makes vtkInterpolatedVelocityField 
  non-robust in cell location for vtkPointSet.
  
  For vtkCellLocatorInterpolatedVelocityField, the only caching (level \#0) works
  by intra-cell trial. In case of failure, a global search for the target cell is 
  invoked via vtkAbstractCellLocator::FindCell() and the actual work is done by 
  either vtkCellLocator or vtkModifiedBSPTree (for datasets of type vtkPointSet 
  only, while vtkImageData and vtkRectilinearGrid themselves are able to provide 
  fast robust cell location). Without the involvement of vtkPointLocator, robust 
  cell location is achieved for vtkPointSet. 


To create an instance of class vtkAbstractInterpolatedVelocityField, simply
invoke its constructor as follows
\begin{verbatim}
  obj = vtkAbstractInterpolatedVelocityField
\end{verbatim}
\subsection{Methods}

The class vtkAbstractInterpolatedVelocityField has several methods that can be used.
  They are listed below.
Note that the documentation is translated automatically from the VTK sources,
and may not be completely intelligible.  When in doubt, consult the VTK website.
In the methods listed below, \verb|obj| is an instance of the vtkAbstractInterpolatedVelocityField class.
\begin{itemize}
\item  \verb|string = obj.GetClassName ()|

\item  \verb|int = obj.IsA (string name)|

\item  \verb|vtkAbstractInterpolatedVelocityField = obj.NewInstance ()|

\item  \verb|vtkAbstractInterpolatedVelocityField = obj.SafeDownCast (vtkObject o)|

\item  \verb|obj.SetCaching (bool )| -  Set/Get the caching flag. If this flag is turned ON, there are two levels
 of caching for derived concrete class vtkInterpolatedVelocityField and one
 level of caching for derived concrete class vtkCellLocatorInterpolatedVelocityField.
 Otherwise a global cell location is always invoked for evaluating the 
 function values at any point.

\item  \verb|bool = obj.GetCaching ()| -  Set/Get the caching flag. If this flag is turned ON, there are two levels
 of caching for derived concrete class vtkInterpolatedVelocityField and one
 level of caching for derived concrete class vtkCellLocatorInterpolatedVelocityField.
 Otherwise a global cell location is always invoked for evaluating the 
 function values at any point.

\item  \verb|int = obj.GetCacheHit ()| -  Get the caching statistics. CacheHit refers to the number of level \#0 cache
 hits while CacheMiss is the number of level \#0 cache misses.

\item  \verb|int = obj.GetCacheMiss ()| -  Get the caching statistics. CacheHit refers to the number of level \#0 cache
 hits while CacheMiss is the number of level \#0 cache misses.

\item  \verb|int = obj.GetLastDataSetIndex ()| -  Get the most recently visited dataset and it id. The dataset is used 
 for a guess regarding where the next point will be, without searching
 through all datasets. When setting the last dataset, care is needed as
 no reference counting or checks are performed. This feature is intended
 for custom interpolators only that cache datasets independently.

\item  \verb|vtkDataSet = obj.GetLastDataSet ()| -  Get the most recently visited dataset and it id. The dataset is used 
 for a guess regarding where the next point will be, without searching
 through all datasets. When setting the last dataset, care is needed as
 no reference counting or checks are performed. This feature is intended
 for custom interpolators only that cache datasets independently.

\item  \verb|vtkIdType = obj.GetLastCellId ()| -  Get/Set the id of the cell cached from last evaluation.

\item  \verb|obj.SetLastCellId (vtkIdType c)| -  Set the id of the most recently visited cell of a dataset.

\item  \verb|obj.SetLastCellId (vtkIdType c, int dataindex)| -  Set the id of the most recently visited cell of a dataset.

\item  \verb|string = obj.GetVectorsSelection ()| -  Get/Set the name of a spcified vector array. By default it is NULL, with
 the active vector array for use.

\item  \verb|obj.SelectVectors (string fieldName)| -  Set/Get the flag indicating vector post-normalization (following vector
 interpolation). Vector post-normalization is required to avoid the 
 'curve-overshooting' problem (caused by high velocity magnitude) that
 occurs when Cell-Length is used as the step size unit (particularly the
 Minimum step size unit). Furthermore, it is required by RK45 to achieve,
 as expected, high numerical accuracy (or high smoothness of flow lines)
 through adaptive step sizing. Note this operation is performed (when
 NormalizeVector TRUE) right after vector interpolation such that the
 differing amount of contribution of each node (of a cell) to the
 resulting direction of the interpolated vector, due to the possibly
 significantly-differing velocity magnitude values at the nodes (which is
 the case with large cells), can be reflected as is. Also note that this
 flag needs to be turned to FALSE after vtkInitialValueProblemSolver::
 ComputeNextStep() as subsequent operations, e.g., vorticity computation, 
 may need non-normalized vectors.

\item  \verb|obj.SetNormalizeVector (bool )| -  Set/Get the flag indicating vector post-normalization (following vector
 interpolation). Vector post-normalization is required to avoid the 
 'curve-overshooting' problem (caused by high velocity magnitude) that
 occurs when Cell-Length is used as the step size unit (particularly the
 Minimum step size unit). Furthermore, it is required by RK45 to achieve,
 as expected, high numerical accuracy (or high smoothness of flow lines)
 through adaptive step sizing. Note this operation is performed (when
 NormalizeVector TRUE) right after vector interpolation such that the
 differing amount of contribution of each node (of a cell) to the
 resulting direction of the interpolated vector, due to the possibly
 significantly-differing velocity magnitude values at the nodes (which is
 the case with large cells), can be reflected as is. Also note that this
 flag needs to be turned to FALSE after vtkInitialValueProblemSolver::
 ComputeNextStep() as subsequent operations, e.g., vorticity computation, 
 may need non-normalized vectors.

\item  \verb|bool = obj.GetNormalizeVector ()| -  Set/Get the flag indicating vector post-normalization (following vector
 interpolation). Vector post-normalization is required to avoid the 
 'curve-overshooting' problem (caused by high velocity magnitude) that
 occurs when Cell-Length is used as the step size unit (particularly the
 Minimum step size unit). Furthermore, it is required by RK45 to achieve,
 as expected, high numerical accuracy (or high smoothness of flow lines)
 through adaptive step sizing. Note this operation is performed (when
 NormalizeVector TRUE) right after vector interpolation such that the
 differing amount of contribution of each node (of a cell) to the
 resulting direction of the interpolated vector, due to the possibly
 significantly-differing velocity magnitude values at the nodes (which is
 the case with large cells), can be reflected as is. Also note that this
 flag needs to be turned to FALSE after vtkInitialValueProblemSolver::
 ComputeNextStep() as subsequent operations, e.g., vorticity computation, 
 may need non-normalized vectors.

\item  \verb|obj.CopyParameters (vtkAbstractInterpolatedVelocityField from)| -  Add a dataset for implicit velocity function evaluation. If more than 
 one dataset is added, the evaluation point is searched in all until a
 match is found. THIS FUNCTION DOES NOT CHANGE THE REFERENCE COUNT OF 
 dataset FOR THREAD SAFETY REASONS.

\item  \verb|obj.AddDataSet (vtkDataSet dataset)| -  Add a dataset for implicit velocity function evaluation. If more than 
 one dataset is added, the evaluation point is searched in all until a
 match is found. THIS FUNCTION DOES NOT CHANGE THE REFERENCE COUNT OF 
 dataset FOR THREAD SAFETY REASONS.

\item  \verb|int = obj.FunctionValues (double x, double f)| -  Evaluate the velocity field f at point (x, y, z).

\item  \verb|obj.ClearLastCellId ()| -  Get the interpolation weights cached from last evaluation. Return 1 if the
 cached cell is valid and 0 otherwise.

\item  \verb|int = obj.GetLastWeights (double w)| -  Get the interpolation weights cached from last evaluation. Return 1 if the
 cached cell is valid and 0 otherwise.

\item  \verb|int = obj.GetLastLocalCoordinates (double pcoords[3])| -  Get the interpolation weights cached from last evaluation. Return 1 if the
 cached cell is valid and 0 otherwise.

\end{itemize}
