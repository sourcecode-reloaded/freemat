\section{TYPE Type Contents of File or Function}

\subsection{Usage}

Displays the contents of a file or a function to the screen
or console.  The syntax for its use is
\begin{verbatim}
   type filename
   type('filename')
\end{verbatim}
or
\begin{verbatim}
   type function
   type('function')
\end{verbatim}
in which case the function named \verb|'function.m'| will be displayed.
\subsection{Example}

Here we use \verb|type| to display the contents of itself 
\begin{verbatim}
--> type('type')
%!
%@Module TYPE Type Contents of File or Function
%@@Section IO
%@@Usage
%Displays the contents of a file or a function to the screen
%or console.  The syntax for its use is
%@[
%   type filename
%   type('filename')
%@]
%or
%@[
%   type function
%   type('function')
%@]
%in which case the function named @|'function.m'| will be displayed.
%@@Example
%Here we use @|type| to display the contents of itself 
%@<
%type('type')
%@>
%!

function type(filename)
fp = fopen(filename,'r');
if (fp == -1)
  f = which(filename);
  if (isempty(f)), return; end
  filename = f;
  fp = fopen(filename,'r');
end
if (fp == -1), return; end
while (~feof(fp))
  printf('%s',fgetline(fp));
end
fclose(fp);
\end{verbatim}
