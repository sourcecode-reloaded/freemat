\section{vtkProcess}

\subsection{Usage}

 vtkProcess is an abstract class representing a process that can be launched
 by a vtkMultiProcessController. Concrete classes just have to implement
 Execute() method and make sure it set the proper value in ReturnValue.

 .SECTION Example
  class MyProcess: public vtkProcess
  ...
  vtkMultiProcessController *c;
  MyProcess *p=new MyProcess::New();
  p->SetArgs(argc,argv); // some parameters specific to the process
  p->SetX(10.0); // ...
  c->SetSingleProcess(p);
  c->SingleMethodExecute();
  int returnValue=p->GetReturnValue();


To create an instance of class vtkProcess, simply
invoke its constructor as follows
\begin{verbatim}
  obj = vtkProcess
\end{verbatim}
\subsection{Methods}

The class vtkProcess has several methods that can be used.
  They are listed below.
Note that the documentation is translated automatically from the VTK sources,
and may not be completely intelligible.  When in doubt, consult the VTK website.
In the methods listed below, \verb|obj| is an instance of the vtkProcess class.
\begin{itemize}
\item  \verb|string = obj.GetClassName ()|

\item  \verb|int = obj.IsA (string name)|

\item  \verb|vtkProcess = obj.NewInstance ()|

\item  \verb|vtkProcess = obj.SafeDownCast (vtkObject o)|

\item  \verb|obj.Execute ()|

\item  \verb|vtkMultiProcessController = obj.GetController ()| -  Give access to the controller that launched the process.
 Initial value is NULL.

\item  \verb|obj.SetController (vtkMultiProcessController aController)| -  This method should not be called directly but set by the controller
 itself.

\item  \verb|int = obj.GetReturnValue ()| -  Value set at the end of a call to Execute.

\end{itemize}
