\section{vtkConstrainedPointHandleRepresentation}

\subsection{Usage}

 This class is used to represent a vtkHandleWidget. It represents a
 position in 3D world coordinates that is constrained to a specified plane.
 The default look is to draw a white point when this widget is not selected
 or active, a thin green circle when it is highlighted, and a thicker cyan
 circle when it is active (being positioned). Defaults can be adjusted - but
 take care to define cursor geometry that makes sense for this widget.
 The geometry will be aligned on the constraining plane, with the plane
 normal aligned with the X axis of the geometry (similar behavior to
 vtkGlyph3D). 

 TODO: still need to work on 
 1) translation when mouse is outside bounding planes
 2) size of the widget


To create an instance of class vtkConstrainedPointHandleRepresentation, simply
invoke its constructor as follows
\begin{verbatim}
  obj = vtkConstrainedPointHandleRepresentation
\end{verbatim}
\subsection{Methods}

The class vtkConstrainedPointHandleRepresentation has several methods that can be used.
  They are listed below.
Note that the documentation is translated automatically from the VTK sources,
and may not be completely intelligible.  When in doubt, consult the VTK website.
In the methods listed below, \verb|obj| is an instance of the vtkConstrainedPointHandleRepresentation class.
\begin{itemize}
\item  \verb|string = obj.GetClassName ()| -  Standard methods for instances of this class.

\item  \verb|int = obj.IsA (string name)| -  Standard methods for instances of this class.

\item  \verb|vtkConstrainedPointHandleRepresentation = obj.NewInstance ()| -  Standard methods for instances of this class.

\item  \verb|vtkConstrainedPointHandleRepresentation = obj.SafeDownCast (vtkObject o)| -  Standard methods for instances of this class.

\item  \verb|obj.SetCursorShape (vtkPolyData cursorShape)| -  Specify the cursor shape. Keep in mind that the shape will be
 aligned with the  constraining plane by orienting it such that
 the x axis of the geometry lies along the normal of the plane.

\item  \verb|vtkPolyData = obj.GetCursorShape ()| -  Specify the cursor shape. Keep in mind that the shape will be
 aligned with the  constraining plane by orienting it such that
 the x axis of the geometry lies along the normal of the plane.

\item  \verb|obj.SetActiveCursorShape (vtkPolyData activeShape)| -  Specify the shape of the cursor (handle) when it is active.
 This is the geometry that will be used when the mouse is
 close to the handle or if the user is manipulating the handle.

\item  \verb|vtkPolyData = obj.GetActiveCursorShape ()| -  Specify the shape of the cursor (handle) when it is active.
 This is the geometry that will be used when the mouse is
 close to the handle or if the user is manipulating the handle.

\item  \verb|obj.SetProjectionNormal (int )| -  Set the projection normal to lie along the x, y, or z axis,
 or to be oblique. If it is oblique, then the plane is 
 defined in the ObliquePlane ivar.

\item  \verb|int = obj.GetProjectionNormalMinValue ()| -  Set the projection normal to lie along the x, y, or z axis,
 or to be oblique. If it is oblique, then the plane is 
 defined in the ObliquePlane ivar.

\item  \verb|int = obj.GetProjectionNormalMaxValue ()| -  Set the projection normal to lie along the x, y, or z axis,
 or to be oblique. If it is oblique, then the plane is 
 defined in the ObliquePlane ivar.

\item  \verb|int = obj.GetProjectionNormal ()| -  Set the projection normal to lie along the x, y, or z axis,
 or to be oblique. If it is oblique, then the plane is 
 defined in the ObliquePlane ivar.

\item  \verb|obj.SetProjectionNormalToXAxis ()|

\item  \verb|obj.SetProjectionNormalToYAxis ()|

\item  \verb|obj.SetProjectionNormalToZAxis ()|

\item  \verb|obj.SetProjectionNormalToOblique ()| -  If the ProjectionNormal is set to Oblique, then this is the 
 oblique plane used to constrain the handle position

\item  \verb|obj.SetObliquePlane (vtkPlane )| -  If the ProjectionNormal is set to Oblique, then this is the 
 oblique plane used to constrain the handle position

\item  \verb|vtkPlane = obj.GetObliquePlane ()| -  If the ProjectionNormal is set to Oblique, then this is the 
 oblique plane used to constrain the handle position

\item  \verb|obj.SetProjectionPosition (double position)| -  The position of the bounding plane from the origin along the
 normal. The origin and normal are defined in the oblique plane
 when the ProjectionNormal is Oblique. For the X, Y, and Z
 axes projection normals, the normal is the axis direction, and
 the origin is (0,0,0).

\item  \verb|double = obj.GetProjectionPosition ()| -  The position of the bounding plane from the origin along the
 normal. The origin and normal are defined in the oblique plane
 when the ProjectionNormal is Oblique. For the X, Y, and Z
 axes projection normals, the normal is the axis direction, and
 the origin is (0,0,0).

\item  \verb|obj.AddBoundingPlane (vtkPlane plane)| -  A collection of plane equations used to bound the position of the point.
 This is in addition to confining the point to a plane - these contraints
 are meant to, for example, keep a point within the extent of an image.
 Using a set of plane equations allows for more complex bounds (such as
 bounding a point to an oblique reliced image that has hexagonal shape)
 than a simple extent.

\item  \verb|obj.RemoveBoundingPlane (vtkPlane plane)| -  A collection of plane equations used to bound the position of the point.
 This is in addition to confining the point to a plane - these contraints
 are meant to, for example, keep a point within the extent of an image.
 Using a set of plane equations allows for more complex bounds (such as
 bounding a point to an oblique reliced image that has hexagonal shape)
 than a simple extent.

\item  \verb|obj.RemoveAllBoundingPlanes ()| -  A collection of plane equations used to bound the position of the point.
 This is in addition to confining the point to a plane - these contraints
 are meant to, for example, keep a point within the extent of an image.
 Using a set of plane equations allows for more complex bounds (such as
 bounding a point to an oblique reliced image that has hexagonal shape)
 than a simple extent.

\item  \verb|obj.SetBoundingPlanes (vtkPlaneCollection )| -  A collection of plane equations used to bound the position of the point.
 This is in addition to confining the point to a plane - these contraints
 are meant to, for example, keep a point within the extent of an image.
 Using a set of plane equations allows for more complex bounds (such as
 bounding a point to an oblique reliced image that has hexagonal shape)
 than a simple extent.

\item  \verb|vtkPlaneCollection = obj.GetBoundingPlanes ()| -  A collection of plane equations used to bound the position of the point.
 This is in addition to confining the point to a plane - these contraints
 are meant to, for example, keep a point within the extent of an image.
 Using a set of plane equations allows for more complex bounds (such as
 bounding a point to an oblique reliced image that has hexagonal shape)
 than a simple extent.

\item  \verb|obj.SetBoundingPlanes (vtkPlanes planes)| -  A collection of plane equations used to bound the position of the point.
 This is in addition to confining the point to a plane - these contraints
 are meant to, for example, keep a point within the extent of an image.
 Using a set of plane equations allows for more complex bounds (such as
 bounding a point to an oblique reliced image that has hexagonal shape)
 than a simple extent.

\item  \verb|int = obj.CheckConstraint (vtkRenderer renderer, double pos[2])| -  Overridden from the base class. It converts the display
 co-ordinates to world co-ordinates. It returns 1 if the point lies
 within the constrained region, otherwise return 0

\item  \verb|obj.SetPosition (double x, double y, double z)| -  Set/Get the position of the point in display coordinates.  These are
 convenience methods that extend the superclasses' GetHandlePosition()
 method. Note that only the x-y coordinate values are used

\item  \verb|obj.SetPosition (double xyz[3])| -  Set/Get the position of the point in display coordinates.  These are
 convenience methods that extend the superclasses' GetHandlePosition()
 method. Note that only the x-y coordinate values are used

\item  \verb|obj.GetPosition (double xyz[3])| -  Set/Get the position of the point in display coordinates.  These are
 convenience methods that extend the superclasses' GetHandlePosition()
 method. Note that only the x-y coordinate values are used

\item  \verb|vtkProperty = obj.GetProperty ()| -  This is the property used when the handle is not active 
 (the mouse is not near the handle)

\item  \verb|vtkProperty = obj.GetSelectedProperty ()| -  This is the property used when the mouse is near the
 handle (but the user is not yet interacting with it)

\item  \verb|vtkProperty = obj.GetActiveProperty ()| -  This is the property used when the user is interacting
 with the handle.

\item  \verb|obj.SetRenderer (vtkRenderer ren)| -  Subclasses of vtkConstrainedPointHandleRepresentation must implement these methods. These
 are the methods that the widget and its representation use to
 communicate with each other.

\item  \verb|obj.BuildRepresentation ()| -  Subclasses of vtkConstrainedPointHandleRepresentation must implement these methods. These
 are the methods that the widget and its representation use to
 communicate with each other.

\item  \verb|obj.StartWidgetInteraction (double eventPos[2])| -  Subclasses of vtkConstrainedPointHandleRepresentation must implement these methods. These
 are the methods that the widget and its representation use to
 communicate with each other.

\item  \verb|obj.WidgetInteraction (double eventPos[2])| -  Subclasses of vtkConstrainedPointHandleRepresentation must implement these methods. These
 are the methods that the widget and its representation use to
 communicate with each other.

\item  \verb|int = obj.ComputeInteractionState (int X, int Y, int modify)| -  Subclasses of vtkConstrainedPointHandleRepresentation must implement these methods. These
 are the methods that the widget and its representation use to
 communicate with each other.

\item  \verb|obj.SetDisplayPosition (double pos[3])| -  Method overridden from Superclass. computes the world
 co-ordinates using GetIntersectionPosition()

\item  \verb|obj.GetActors (vtkPropCollection )| -  Methods to make this class behave as a vtkProp.

\item  \verb|obj.ReleaseGraphicsResources (vtkWindow )| -  Methods to make this class behave as a vtkProp.

\item  \verb|int = obj.RenderOverlay (vtkViewport viewport)| -  Methods to make this class behave as a vtkProp.

\item  \verb|int = obj.RenderOpaqueGeometry (vtkViewport viewport)| -  Methods to make this class behave as a vtkProp.

\item  \verb|int = obj.RenderTranslucentPolygonalGeometry (vtkViewport viewport)| -  Methods to make this class behave as a vtkProp.

\item  \verb|int = obj.HasTranslucentPolygonalGeometry ()| -  Methods to make this class behave as a vtkProp.

\item  \verb|obj.ShallowCopy (vtkProp prop)| -  Methods to make this class behave as a vtkProp.

\end{itemize}
