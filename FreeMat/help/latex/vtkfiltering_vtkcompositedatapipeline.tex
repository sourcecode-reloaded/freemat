\section{vtkCompositeDataPipeline}

\subsection{Usage}

 vtkCompositeDataPipeline is an executive that supports the processing of
 composite dataset. It supports algorithms that are aware of composite
 dataset as well as those that are not. Type checking is performed at run
 time. Algorithms that are not composite dataset-aware have to support
 all dataset types contained in the composite dataset. The pipeline
 execution can be summarized as follows: 

 * REQUEST\_INFORMATION: The producers have to provide information about
 the contents of the composite dataset in this pass. 
 Sources that can produce more than one piece (note that a piece is
 different than a block; each piece consistes of 0 or more blocks) should
 set MAXIMUM\_NUMBER\_OF\_PIECES to -1.

 * REQUEST\_UPDATE\_EXTENT: This pass is identical to the one implemented
 in vtkStreamingDemandDrivenPipeline

 * REQUEST\_DATA: This is where the algorithms execute. If the 
 vtkCompositeDataPipeline is assigned to a simple filter, 
 it will invoke the  vtkStreamingDemandDrivenPipeline passes in a loop, 
 passing a different block each time and will collect the results in a 
 composite dataset. 
 .SECTION See also
  vtkCompositeDataSet

To create an instance of class vtkCompositeDataPipeline, simply
invoke its constructor as follows
\begin{verbatim}
  obj = vtkCompositeDataPipeline
\end{verbatim}
\subsection{Methods}

The class vtkCompositeDataPipeline has several methods that can be used.
  They are listed below.
Note that the documentation is translated automatically from the VTK sources,
and may not be completely intelligible.  When in doubt, consult the VTK website.
In the methods listed below, \verb|obj| is an instance of the vtkCompositeDataPipeline class.
\begin{itemize}
\item  \verb|string = obj.GetClassName ()|

\item  \verb|int = obj.IsA (string name)|

\item  \verb|vtkCompositeDataPipeline = obj.NewInstance ()|

\item  \verb|vtkCompositeDataPipeline = obj.SafeDownCast (vtkObject o)|

\item  \verb|vtkDataObject = obj.GetCompositeOutputData (int port)| -  Returns the data object stored with the DATA\_OBJECT() in the
 output port

\end{itemize}
