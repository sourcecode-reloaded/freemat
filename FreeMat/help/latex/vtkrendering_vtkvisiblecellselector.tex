\section{vtkVisibleCellSelector}

\subsection{Usage}

 DEPRECATED: Please refer to vtkHardwareSelector instead.
 This class can be used to determine what cells are visible within a given
 rectangle of the RenderWindow. To use it, call in order, SetRenderer(), 
 SetArea(), SetProcessorId(), SetRenderPasses(), and then Select(). Select 
 will cause the attached vtkRenderer to render in a special color mode, 
 where each cell is given it own color so that later inspection of the 
 Rendered Pixels can determine what cells are visible. In practice up to 
 five different rendering passes may occur depending on your choices in 
 SetRenderPasses. After Select(), a list of the visible cells can be 
 obtained by calling GetSelectedIds().

 Limitations:
 Antialiasing will break this class. If your graphics card settings force
 their use this class will return invalid results.

 Currently only cells from PolyDataMappers can be selected from. When 
 vtkRenderer is put into a SelectMode, it temporarily swaps in a new 
 vtkIdentColoredPainter to do the color index rendering of each cell in 
 each vtkProp that it renders. Until alternatives to vtkIdentColoredPainter
 exist that can do a similar coloration of other vtkDataSet types, only
 polygonal data can be selected. If you need to select other data types,
 consider using vtkDataSetMapper and turning on it's PassThroughCellIds 
 feature, or using vtkFrustumExtractor.

 Only Opaque geometry in Actors is selected from. Assemblies and LODMappers 
 are not currently supported. 

 During selection, visible datasets that can not be selected from are
 temporarily hidden so as not to produce invalid indices from their colors.


To create an instance of class vtkVisibleCellSelector, simply
invoke its constructor as follows
\begin{verbatim}
  obj = vtkVisibleCellSelector
\end{verbatim}
\subsection{Methods}

The class vtkVisibleCellSelector has several methods that can be used.
  They are listed below.
Note that the documentation is translated automatically from the VTK sources,
and may not be completely intelligible.  When in doubt, consult the VTK website.
In the methods listed below, \verb|obj| is an instance of the vtkVisibleCellSelector class.
\begin{itemize}
\item  \verb|string = obj.GetClassName ()|

\item  \verb|int = obj.IsA (string name)|

\item  \verb|vtkVisibleCellSelector = obj.NewInstance ()|

\item  \verb|vtkVisibleCellSelector = obj.SafeDownCast (vtkObject o)|

\item  \verb|obj.SetRenderer (vtkRenderer )| -  Call to let this know where to select within.

\item  \verb|obj.SetArea (int x0, int y0, int x1, int y1)| -  Call to set the selection area region.
 This crops the selected area to the renderers pixel limits.

\item  \verb|obj.SetProcessorId (int pid)| -  Call to let this know what processor number to render as in the processor
 select pass. Internally this adds 1 to pid because 0 is reserved for miss.

\item  \verb|int = obj.GetProcessorId ()| -  Call to let this know what processor number to render as in the processor
 select pass. Internally this adds 1 to pid because 0 is reserved for miss.

\item  \verb|obj.SetRenderPasses (int DoProcessor, int DoActor, int DoCellIdHi, int DoCellIdMid, int DoCellIdLo, int DoVertexId)| -  Call to let this know what selection render passes to do.
 If you have only one processor or one actor, you can leave DoProcessor 
 and DoActor as false (the default). If you have less than 2\^48 cells in 
 any actor, you do not need the CellIdHi pass, or similarly if you have 
 less than 2\^24 cells, you do not need DoCellIdMid.
 The DoPointId will enable another render pass for determining visible
 vertices.

\item  \verb|obj.Select ()| -  Execute the selection algorithm.

\item  \verb|obj.GetSelectedIds (vtkIdTypeArray ToCopyInto)| -  After Select(), this will return the list of selected Ids.
 The ProcessorId and Actor Id are returned in the first two components.
 The CellId is returned in the last two components (only 64 bits total).

\item  \verb|obj.GetSelectedIds (vtkSelection ToCopyInto)| -  After Select(), this will return the list of selected Ids.

\item  \verb|obj.GetSelectedVertices (vtkIdTypeArray VertexPointers, vtkIdTypeArray VertexIds)| -  After Select(), (assuming DoVertexId is on), the will return arrays that
 describe which cell vertices are visible.
 The VertexPointers array contains one index into the VertexIds array for
 every visible cell. Any index may be -1 in which case no vertices were 
 visible for that cell. The VertexIds array contains a set of integers for
 each cell that has visible vertices. The first entry in the set is
 the number of visible vertices. The rest are visible vertex ranks. 
 A set such at 2,0,4, means that a particular polygon's first and fifth
 vertices were visible.

\item  \verb|vtkProp = obj.GetActorFromId (vtkIdType id)| -  After a select, this will return a pointer to the actor corresponding to
 a particular id. This will return NULL if id is out of range.

\item  \verb|obj.PrintSelectedIds (vtkIdTypeArray IdsToPrint)| -  For debugging - prints out the list of selected ids.

\end{itemize}
