\section{vtkVolumeCollection}

\subsection{Usage}

 vtkVolumeCollection represents and provides methods to manipulate a 
 list of volumes (i.e., vtkVolume and subclasses). The list is unsorted 
 and duplicate entries are not prevented.

To create an instance of class vtkVolumeCollection, simply
invoke its constructor as follows
\begin{verbatim}
  obj = vtkVolumeCollection
\end{verbatim}
\subsection{Methods}

The class vtkVolumeCollection has several methods that can be used.
  They are listed below.
Note that the documentation is translated automatically from the VTK sources,
and may not be completely intelligible.  When in doubt, consult the VTK website.
In the methods listed below, \verb|obj| is an instance of the vtkVolumeCollection class.
\begin{itemize}
\item  \verb|string = obj.GetClassName ()|

\item  \verb|int = obj.IsA (string name)|

\item  \verb|vtkVolumeCollection = obj.NewInstance ()|

\item  \verb|vtkVolumeCollection = obj.SafeDownCast (vtkObject o)|

\item  \verb|obj.AddItem (vtkVolume a)| -  Get the next Volume in the list. Return NULL when at the end of the 
 list.

\item  \verb|vtkVolume = obj.GetNextVolume ()| -  Get the next Volume in the list. Return NULL when at the end of the 
 list.

\item  \verb|vtkVolume = obj.GetNextItem ()| -  Access routine provided for compatibility with previous
 versions of VTK.  Please use the GetNextVolume() variant
 where possible.

\end{itemize}
