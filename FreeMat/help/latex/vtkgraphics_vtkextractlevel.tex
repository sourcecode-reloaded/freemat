\section{vtkExtractLevel}

\subsection{Usage}

 vtkExtractLevel filter extracts the levels between (and including) the user
 specified min and max levels.

To create an instance of class vtkExtractLevel, simply
invoke its constructor as follows
\begin{verbatim}
  obj = vtkExtractLevel
\end{verbatim}
\subsection{Methods}

The class vtkExtractLevel has several methods that can be used.
  They are listed below.
Note that the documentation is translated automatically from the VTK sources,
and may not be completely intelligible.  When in doubt, consult the VTK website.
In the methods listed below, \verb|obj| is an instance of the vtkExtractLevel class.
\begin{itemize}
\item  \verb|string = obj.GetClassName ()|

\item  \verb|int = obj.IsA (string name)|

\item  \verb|vtkExtractLevel = obj.NewInstance ()|

\item  \verb|vtkExtractLevel = obj.SafeDownCast (vtkObject o)|

\item  \verb|obj.AddLevel (int level)| -  Select the levels that should be extracted. All other levels will have no
 datasets in them.

\item  \verb|obj.RemoveLevel (int level)| -  Select the levels that should be extracted. All other levels will have no
 datasets in them.

\item  \verb|obj.RemoveAllLevels ()| -  Select the levels that should be extracted. All other levels will have no
 datasets in them.

\end{itemize}
