\section{vtkHyperOctreeCursor}

\subsection{Usage}

 Objects that can traverse hyperoctree nodes. It is an abstract class.
 Cursors are created by the hyperoctree.

To create an instance of class vtkHyperOctreeCursor, simply
invoke its constructor as follows
\begin{verbatim}
  obj = vtkHyperOctreeCursor
\end{verbatim}
\subsection{Methods}

The class vtkHyperOctreeCursor has several methods that can be used.
  They are listed below.
Note that the documentation is translated automatically from the VTK sources,
and may not be completely intelligible.  When in doubt, consult the VTK website.
In the methods listed below, \verb|obj| is an instance of the vtkHyperOctreeCursor class.
\begin{itemize}
\item  \verb|string = obj.GetClassName ()|

\item  \verb|int = obj.IsA (string name)|

\item  \verb|vtkHyperOctreeCursor = obj.NewInstance ()|

\item  \verb|vtkHyperOctreeCursor = obj.SafeDownCast (vtkObject o)|

\item  \verb|int = obj.GetLeafId ()| -  Return the index of the current leaf in the data arrays.
 

\item  \verb|int = obj.CurrentIsLeaf ()| -  Is the node pointed by the cursor a leaf?

\item  \verb|int = obj.CurrentIsRoot ()| -  Is the node pointed by the cursor the root?

\item  \verb|int = obj.GetCurrentLevel ()| -  Return the level of the node pointed by the cursor.
 

\item  \verb|int = obj.GetChildIndex ()| -  Return the child number of the current node relative to its parent.
 
 

\item  \verb|int = obj.CurrentIsTerminalNode ()|

\item  \verb|obj.ToRoot ()| -  Move the cursor the root node.
 
 

\item  \verb|obj.ToParent ()| -  Move the cursor to the parent of the current node.
 

\item  \verb|obj.ToChild (int child)| -  Move the cursor to child `child' of the current node.
 
 

\item  \verb|obj.ToSameNode (vtkHyperOctreeCursor other)| -  Move the cursor to the same node pointed by `other'.
 
 
 

\item  \verb|int = obj.IsEqual (vtkHyperOctreeCursor other)| -  Is `this' equal to `other'?
 
 

\item  \verb|vtkHyperOctreeCursor = obj.Clone ()| -  Create a copy of `this'.
 
 

\item  \verb|int = obj.SameTree (vtkHyperOctreeCursor other)| -  Are `this' and `other' pointing on the same hyperoctree?
 

\item  \verb|int = obj.GetIndex (int d)| -  Return the index in dimension `d', as if the node was a cell of a
 uniform grid of 1<<GetCurrentLevel() cells in each dimension.
 
 

\item  \verb|int = obj.GetNumberOfChildren ()| -  Return the number of children for each node of the tree.
 

\item  \verb|int = obj.GetDimension ()| -  Return the dimension of the tree.
 

\item  \verb|obj.MoveToNode (int indices, int level)| -  Move to the node described by its indices in each dimension and
 at a given level. If there is actually a node or a leaf at this
 location, Found() returns true. Otherwise, Found() returns false and the
 cursor moves to the closest parent of the query. It can be the root in the
 worst case.
 
 
 

\item  \verb|int = obj.Found ()|

\end{itemize}
