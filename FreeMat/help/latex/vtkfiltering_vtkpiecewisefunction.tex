\section{vtkPiecewiseFunction}

\subsection{Usage}

 Defines a piecewise function mapping. This mapping allows the addition
 of control points, and allows the user to control the function between
 the control points. A piecewise hermite curve is used between control 
 points, based on the sharpness and midpoint parameters. A sharpness of
 0 yields a piecewise linear function and a sharpness of 1 yields a
 piecewise constant function. The midpoint is the normalized distance
 between control points at which the curve reaches the median Y value.
 The midpoint and sharpness values specified when adding a node are used
 to control the transition to the next node (the last node's values are
 ignored) Outside the range of nodes, the values are 0 if Clamping is off,
 or the nearest node point if Clamping is on. Using the legacy methods for
 adding points  (which do not have Sharpness and Midpoint parameters) 
 will default to Midpoint = 0.5 (halfway between the control points) and
 Sharpness = 0.0 (linear).

To create an instance of class vtkPiecewiseFunction, simply
invoke its constructor as follows
\begin{verbatim}
  obj = vtkPiecewiseFunction
\end{verbatim}
\subsection{Methods}

The class vtkPiecewiseFunction has several methods that can be used.
  They are listed below.
Note that the documentation is translated automatically from the VTK sources,
and may not be completely intelligible.  When in doubt, consult the VTK website.
In the methods listed below, \verb|obj| is an instance of the vtkPiecewiseFunction class.
\begin{itemize}
\item  \verb|string = obj.GetClassName ()|

\item  \verb|int = obj.IsA (string name)|

\item  \verb|vtkPiecewiseFunction = obj.NewInstance ()|

\item  \verb|vtkPiecewiseFunction = obj.SafeDownCast (vtkObject o)|

\item  \verb|obj.DeepCopy (vtkDataObject f)|

\item  \verb|obj.ShallowCopy (vtkDataObject f)|

\item  \verb|int = obj.GetDataObjectType ()| -  Return what type of dataset this is.

\item  \verb|int = obj.GetSize ()| -  Get the number of points used to specify the function

\item  \verb|int = obj.AddPoint (double x, double y)| -  Add/Remove points to/from the function. If a duplicate point is added
 then the function value is changed at that location.
 Return the index of the point (0 based), or -1 on error.

\item  \verb|int = obj.AddPoint (double x, double y, double midpoint, double sharpness)| -  Add/Remove points to/from the function. If a duplicate point is added
 then the function value is changed at that location.
 Return the index of the point (0 based), or -1 on error.

\item  \verb|int = obj.RemovePoint (double x)| -  Add/Remove points to/from the function. If a duplicate point is added
 then the function value is changed at that location.
 Return the index of the point (0 based), or -1 on error.

\item  \verb|obj.RemoveAllPoints ()| -  Removes all points from the function. 

\item  \verb|obj.AddSegment (double x1, double y1, double x2, double y2)| -  Add a line segment to the function. All points defined between the
 two points specified are removed from the function. This is a legacy
 method that does not allow the specification of the sharpness and
 midpoint values for the two nodes.

\item  \verb|double = obj.GetValue (double x)| -  Returns the value of the function at the specified location using
 the specified interpolation. 

\item  \verb|int = obj.GetNodeValue (int index, double val[4])| -  For the node specified by index, set/get the
 location (X), value (Y), midpoint, and sharpness 
 values at the node.

\item  \verb|int = obj.SetNodeValue (int index, double val[4])| -  For the node specified by index, set/get the
 location (X), value (Y), midpoint, and sharpness 
 values at the node.

\item  \verb|obj.FillFromDataPointer (int , double )| -  Returns a pointer to the data stored in the table.
 Fills from a pointer to data stored in a similar table. These are
 legacy methods which will be maintained for compatibility - however,
 note that the vtkPiecewiseFunction no longer stores the nodes 
 in a double array internally.

\item  \verb|double = obj. GetRange ()| -  Returns the min and max node locations of the function.

\item  \verb|int = obj.AdjustRange (double range[2])| -  Remove all points out of the new range, and make sure there is a point
 at each end of that range.
 Return 1 on success, 0 otherwise.

\item  \verb|obj.GetTable (double x1, double x2, int size, float table, int stride)| -  Fills in an array of function values evaluated at regular intervals.
 Parameter ''stride'' is used to step through the output ''table''. 

\item  \verb|obj.GetTable (double x1, double x2, int size, double table, int stride)| -  Fills in an array of function values evaluated at regular intervals.
 Parameter ''stride'' is used to step through the output ''table''. 

\item  \verb|obj.BuildFunctionFromTable (double x1, double x2, int size, double table, int stride)| -  Constructs a piecewise function from a table.  Function range is
 is set to [x1, x2], function size is set to size, and function points
 are regularly spaced between x1 and x2.  Parameter ''stride'' is
 is step through the input table.  

\item  \verb|obj.SetClamping (int )| -  When zero range clamping is Off, GetValue() returns 0.0 when a
 value is requested outside of the points specified.
 When zero range clamping is On, GetValue() returns the value at
 the value at the lowest point for a request below all points 
 specified and returns the value at the highest point for a request 
 above all points specified. On is the default.

\item  \verb|int = obj.GetClamping ()| -  When zero range clamping is Off, GetValue() returns 0.0 when a
 value is requested outside of the points specified.
 When zero range clamping is On, GetValue() returns the value at
 the value at the lowest point for a request below all points 
 specified and returns the value at the highest point for a request 
 above all points specified. On is the default.

\item  \verb|obj.ClampingOn ()| -  When zero range clamping is Off, GetValue() returns 0.0 when a
 value is requested outside of the points specified.
 When zero range clamping is On, GetValue() returns the value at
 the value at the lowest point for a request below all points 
 specified and returns the value at the highest point for a request 
 above all points specified. On is the default.

\item  \verb|obj.ClampingOff ()| -  When zero range clamping is Off, GetValue() returns 0.0 when a
 value is requested outside of the points specified.
 When zero range clamping is On, GetValue() returns the value at
 the value at the lowest point for a request below all points 
 specified and returns the value at the highest point for a request 
 above all points specified. On is the default.

\item  \verb|string = obj.GetType ()| -  Return the type of function:
 Function Types:
    0 : Constant        (No change in slope between end points)
    1 : NonDecreasing   (Always increasing or zero slope)
    2 : NonIncreasing   (Always decreasing or zero slope)
    3 : Varied          (Contains both decreasing and increasing slopes)

\item  \verb|double = obj.GetFirstNonZeroValue ()| -  Returns the first point location which precedes a non-zero segment of the
 function. Note that the value at this point may be zero.

\item  \verb|obj.Initialize ()| -  Clears out the current function. A newly created vtkPiecewiseFunction
 is alreay initialized, so there is no need to call this method which
 in turn simply calls RemoveAllPoints() 

\item  \verb|obj.SetAllowDuplicateScalars (int )| -  Toggle whether to allow duplicate scalar values in the piecewise
 function (off by default).

\item  \verb|int = obj.GetAllowDuplicateScalars ()| -  Toggle whether to allow duplicate scalar values in the piecewise
 function (off by default).

\item  \verb|obj.AllowDuplicateScalarsOn ()| -  Toggle whether to allow duplicate scalar values in the piecewise
 function (off by default).

\item  \verb|obj.AllowDuplicateScalarsOff ()| -  Toggle whether to allow duplicate scalar values in the piecewise
 function (off by default).

\end{itemize}
