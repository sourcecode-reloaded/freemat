\section{vtkProbeFilter}

\subsection{Usage}

 vtkProbeFilter is a filter that computes point attributes (e.g., scalars,
 vectors, etc.) at specified point positions. The filter has two inputs:
 the Input and Source. The Input geometric structure is passed through the
 filter. The point attributes are computed at the Input point positions
 by interpolating into the source data. For example, we can compute data
 values on a plane (plane specified as Input) from a volume (Source).
 The cell data of the source data is copied to the output based on in
 which source cell each input point is. If an array of the same name exists
 both in source's point and cell data, only the one from the point data is
 probed.

 This filter can be used to resample data, or convert one dataset form into
 another. For example, an unstructured grid (vtkUnstructuredGrid) can be
 probed with a volume (three-dimensional vtkImageData), and then volume
 rendering techniques can be used to visualize the results. Another example:
 a line or curve can be used to probe data to produce x-y plots along
 that line or curve.

To create an instance of class vtkProbeFilter, simply
invoke its constructor as follows
\begin{verbatim}
  obj = vtkProbeFilter
\end{verbatim}
\subsection{Methods}

The class vtkProbeFilter has several methods that can be used.
  They are listed below.
Note that the documentation is translated automatically from the VTK sources,
and may not be completely intelligible.  When in doubt, consult the VTK website.
In the methods listed below, \verb|obj| is an instance of the vtkProbeFilter class.
\begin{itemize}
\item  \verb|string = obj.GetClassName ()|

\item  \verb|int = obj.IsA (string name)|

\item  \verb|vtkProbeFilter = obj.NewInstance ()|

\item  \verb|vtkProbeFilter = obj.SafeDownCast (vtkObject o)|

\item  \verb|obj.SetSource (vtkDataObject source)| -  Specify the point locations used to probe input. Any geometry
 can be used. Old style. Do not use unless for backwards compatibility.

\item  \verb|vtkDataObject = obj.GetSource ()| -  Specify the point locations used to probe input. Any geometry
 can be used. Old style. Do not use unless for backwards compatibility.

\item  \verb|obj.SetSourceConnection (vtkAlgorithmOutput algOutput)| -  Specify the point locations used to probe input. Any geometry
 can be used. New style. Equivalent to SetInputConnection(1, algOutput).

\item  \verb|obj.SetSpatialMatch (int )| -  This flag is used only when a piece is requested to update.  By default
 the flag is off.  Because no spatial correspondence between input pieces
 and source pieces is known, all of the source has to be requested no
 matter what piece of the output is requested.  When there is a spatial 
 correspondence, the user/application can set this flag.  This hint allows
 the breakup of the probe operation to be much more efficient.  When piece
 m of n is requested for update by the user, then only n of m needs to
 be requested of the source. 

\item  \verb|int = obj.GetSpatialMatch ()| -  This flag is used only when a piece is requested to update.  By default
 the flag is off.  Because no spatial correspondence between input pieces
 and source pieces is known, all of the source has to be requested no
 matter what piece of the output is requested.  When there is a spatial 
 correspondence, the user/application can set this flag.  This hint allows
 the breakup of the probe operation to be much more efficient.  When piece
 m of n is requested for update by the user, then only n of m needs to
 be requested of the source. 

\item  \verb|obj.SpatialMatchOn ()| -  This flag is used only when a piece is requested to update.  By default
 the flag is off.  Because no spatial correspondence between input pieces
 and source pieces is known, all of the source has to be requested no
 matter what piece of the output is requested.  When there is a spatial 
 correspondence, the user/application can set this flag.  This hint allows
 the breakup of the probe operation to be much more efficient.  When piece
 m of n is requested for update by the user, then only n of m needs to
 be requested of the source. 

\item  \verb|obj.SpatialMatchOff ()| -  This flag is used only when a piece is requested to update.  By default
 the flag is off.  Because no spatial correspondence between input pieces
 and source pieces is known, all of the source has to be requested no
 matter what piece of the output is requested.  When there is a spatial 
 correspondence, the user/application can set this flag.  This hint allows
 the breakup of the probe operation to be much more efficient.  When piece
 m of n is requested for update by the user, then only n of m needs to
 be requested of the source. 

\item  \verb|vtkIdTypeArray = obj.GetValidPoints ()| -  Get the list of point ids in the output that contain attribute data
 interpolated from the source.

\item  \verb|obj.SetValidPointMaskArrayName (string )| -  Returns the name of the char array added to the output with values 1 for
 valid points and 0 for invalid points.
 Set to ''vtkValidPointMask'' by default.

\item  \verb|string = obj.GetValidPointMaskArrayName ()| -  Returns the name of the char array added to the output with values 1 for
 valid points and 0 for invalid points.
 Set to ''vtkValidPointMask'' by default.

\end{itemize}
