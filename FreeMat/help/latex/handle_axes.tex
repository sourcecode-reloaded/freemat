\section{AXES Create Handle Axes}

\subsection{Usage}

This function has three different syntaxes.  The first takes
no arguments,
\begin{verbatim}
  h = axes
\end{verbatim}
and creates a new set of axes that are parented to the current
figure (see \verb|gcf|).  The newly created axes are made the current
axes (see \verb|gca|) and are added to the end of the list of children 
for the current figure.
The second form takes a set of property names and values
\begin{verbatim}
  h = axes(propertyname,value,propertyname,value,...)
\end{verbatim}
Creates a new set of axes, and then sets the specified properties
to the given value.  This is a shortcut for calling 
\verb|set(h,propertyname,value)| for each pair.
The third form takes a handle as an argument
\begin{verbatim}
  axes(handle)
\end{verbatim}
and makes \verb|handle| the current axes, placing it at the head of
the list of children for the current figure.
