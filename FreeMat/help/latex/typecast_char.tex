\section{CHAR Convert to character array or string}

\subsection{Usage}

The \verb|char| function can be used to convert an array
into a string.  It has several forms.  The first form
is
\begin{verbatim}
   y = char(x)
\end{verbatim}
where \verb|x| is a numeric array containing character codes.
FreeMat does not currently support Unicode, so the
character codes must be in the range of \verb|[0,255]|.  The
output is a string of the same size as \verb|x|.  A second
form is
\begin{verbatim}
   y = char(c)
\end{verbatim}
where \verb|c| is a cell array of strings, creates a matrix string
where each row contains a string from the corresponding cell array.
The third form is
\begin{verbatim}
   y = char(s1, s2, s3, ...)
\end{verbatim}
where \verb|si| are a character arrays.  The result is a matrix string
where each row contains a string from the corresponding argument.
\subsection{Example}

Here is an example of the first technique being used to generate
a string containing some ASCII characters
\begin{verbatim}
--> char([32:64;65:97])

ans = 
 !"#$%&'()*+,-./0123456789:;<=>?@
ABCDEFGHIJKLMNOPQRSTUVWXYZ[\]^_`a
\end{verbatim}
In the next example, we form a character array from a set of
strings in a cell array.  Note that the character array is padded
with spaces to make the rows all have the same length.
\begin{verbatim}
--> char({'hello','to','the','world'})

ans = 
hello
to   
the  
world
\end{verbatim}
In the last example, we pass the individual strings as explicit
arguments to \verb|char|
\begin{verbatim}
--> char('hello','to','the','world')

ans = 
hello
to   
the  
world
\end{verbatim}
