\section{vtkAbstractMapper3D}

\subsection{Usage}

 vtkAbstractMapper3D is an abstract class to specify interface between 3D
 data and graphics primitives or software rendering techniques. Subclasses
 of vtkAbstractMapper3D can be used for rendering geometry or rendering
 volumetric data.

 This class also defines an API to support hardware clipping planes (at most
 six planes can be defined). It also provides geometric data about the input
 data it maps, such as the bounding box and center.


To create an instance of class vtkAbstractMapper3D, simply
invoke its constructor as follows
\begin{verbatim}
  obj = vtkAbstractMapper3D
\end{verbatim}
\subsection{Methods}

The class vtkAbstractMapper3D has several methods that can be used.
  They are listed below.
Note that the documentation is translated automatically from the VTK sources,
and may not be completely intelligible.  When in doubt, consult the VTK website.
In the methods listed below, \verb|obj| is an instance of the vtkAbstractMapper3D class.
\begin{itemize}
\item  \verb|string = obj.GetClassName ()|

\item  \verb|int = obj.IsA (string name)|

\item  \verb|vtkAbstractMapper3D = obj.NewInstance ()|

\item  \verb|vtkAbstractMapper3D = obj.SafeDownCast (vtkObject o)|

\item  \verb|double = obj.GetBounds ()| -  Return bounding box (array of six doubles) of data expressed as
 (xmin,xmax, ymin,ymax, zmin,zmax).
 Update this->Bounds as a side effect.

\item  \verb|obj.GetBounds (double bounds[6])| -  Get the bounds for this mapper as (Xmin,Xmax,Ymin,Ymax,Zmin,Zmax).

\item  \verb|double = obj.GetCenter ()| -  Return the Center of this mapper's data.

\item  \verb|obj.GetCenter (double center[3])| -  Return the diagonal length of this mappers bounding box.

\item  \verb|double = obj.GetLength ()| -  Return the diagonal length of this mappers bounding box.

\item  \verb|int = obj.IsARayCastMapper ()| -  Is this a ''render into image'' mapper? A subclass would return 1 if the
 mapper produces an image by rendering into a software image buffer.

\item  \verb|int = obj.IsARenderIntoImageMapper ()|

\end{itemize}
