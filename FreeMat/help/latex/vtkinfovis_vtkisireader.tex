\section{vtkISIReader}

\subsection{Usage}

 ISI is a tagged format for expressing bibliographic citations.  Data is
 structured as a collection of records with each record composed of
 one-to-many fields.  See

 http://isibasic.com/help/helpprn.html\#dialog\_export\_format

 for details.  vtkISIReader will convert an ISI file into a vtkTable, with
 the set of table columns determined dynamically from the contents of the
 file.

To create an instance of class vtkISIReader, simply
invoke its constructor as follows
\begin{verbatim}
  obj = vtkISIReader
\end{verbatim}
\subsection{Methods}

The class vtkISIReader has several methods that can be used.
  They are listed below.
Note that the documentation is translated automatically from the VTK sources,
and may not be completely intelligible.  When in doubt, consult the VTK website.
In the methods listed below, \verb|obj| is an instance of the vtkISIReader class.
\begin{itemize}
\item  \verb|string = obj.GetClassName ()|

\item  \verb|int = obj.IsA (string name)|

\item  \verb|vtkISIReader = obj.NewInstance ()|

\item  \verb|vtkISIReader = obj.SafeDownCast (vtkObject o)|

\item  \verb|string = obj.GetFileName ()| -  Set/get the file to load

\item  \verb|obj.SetFileName (string )| -  Set/get the file to load

\item  \verb|string = obj.GetDelimiter ()| -  Set/get the delimiter to be used for concatenating field data (default: '';'')

\item  \verb|obj.SetDelimiter (string )| -  Set/get the delimiter to be used for concatenating field data (default: '';'')

\item  \verb|int = obj.GetMaxRecords ()| -  Set/get the maximum number of records to read from the file (zero = unlimited)

\item  \verb|obj.SetMaxRecords (int )| -  Set/get the maximum number of records to read from the file (zero = unlimited)

\end{itemize}
