\section{vtkMultiBlockDataGroupFilter}

\subsection{Usage}

 vtkMultiBlockDataGroupFilter is an M to 1 filter that merges multiple
 input into one multi-group dataset. It will assign each input to
 one group of the multi-group dataset and will assign each update piece
 as a sub-block. For example, if there are two inputs and four update
 pieces, the output contains two groups with four datasets each.

To create an instance of class vtkMultiBlockDataGroupFilter, simply
invoke its constructor as follows
\begin{verbatim}
  obj = vtkMultiBlockDataGroupFilter
\end{verbatim}
\subsection{Methods}

The class vtkMultiBlockDataGroupFilter has several methods that can be used.
  They are listed below.
Note that the documentation is translated automatically from the VTK sources,
and may not be completely intelligible.  When in doubt, consult the VTK website.
In the methods listed below, \verb|obj| is an instance of the vtkMultiBlockDataGroupFilter class.
\begin{itemize}
\item  \verb|string = obj.GetClassName ()|

\item  \verb|int = obj.IsA (string name)|

\item  \verb|vtkMultiBlockDataGroupFilter = obj.NewInstance ()|

\item  \verb|vtkMultiBlockDataGroupFilter = obj.SafeDownCast (vtkObject o)|

\item  \verb|obj.AddInput (vtkDataObject )| -  Add an input of this algorithm.  Note that these methods support
 old-style pipeline connections.  When writing new code you should
 use the more general vtkAlgorithm::AddInputConnection().  See
 SetInput() for details.

\item  \verb|obj.AddInput (int , vtkDataObject )| -  Add an input of this algorithm.  Note that these methods support
 old-style pipeline connections.  When writing new code you should
 use the more general vtkAlgorithm::AddInputConnection().  See
 SetInput() for details.

\end{itemize}
