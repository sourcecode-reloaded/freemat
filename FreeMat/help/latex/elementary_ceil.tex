\section{CEIL Ceiling Function}

\subsection{Usage}

Computes the ceiling of an n-dimensional array elementwise.  The
ceiling of a number is defined as the smallest integer that is
larger than or equal to that number. The general syntax for its use
is
\begin{verbatim}
   y = ceil(x)
\end{verbatim}
where \verb|x| is a multidimensional array of numerical type.  The \verb|ceil| 
function preserves the type of the argument.  So integer arguments 
are not modified, and \verb|float| arrays return \verb|float| arrays as 
outputs, and similarly for \verb|double| arrays.  The \verb|ceil| function 
is not defined for \verb|complex| or \verb|dcomplex| types.
\subsection{Example}

The following demonstrates the \verb|ceil| function applied to various
(numerical) arguments.  For integer arguments, the ceil function has
no effect:
\begin{verbatim}
--> ceil(3)

ans = 
 3 

--> ceil(-3)

ans = 
 -3 
\end{verbatim}
Next, we take the \verb|ceil| of a floating point value:
\begin{verbatim}
--> ceil(float(3.023))

ans = 
 4 

--> ceil(float(-2.341))

ans = 
 -2 
\end{verbatim}
Note that the return type is a \verb|float| also.  Finally, for a \verb|double|
type:
\begin{verbatim}
--> ceil(4.312)

ans = 
 5 

--> ceil(-5.32)

ans = 
 -5 
\end{verbatim}
