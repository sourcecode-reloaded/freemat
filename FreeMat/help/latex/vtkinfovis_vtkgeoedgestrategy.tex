\section{vtkGeoEdgeStrategy}

\subsection{Usage}

 vtkGeoEdgeStrategy produces arcs for each edge in the input graph.
 This is useful for viewing lines on a sphere (e.g. the earth).
 The arcs may ''jump'' above the sphere's surface using ExplodeFactor.

To create an instance of class vtkGeoEdgeStrategy, simply
invoke its constructor as follows
\begin{verbatim}
  obj = vtkGeoEdgeStrategy
\end{verbatim}
\subsection{Methods}

The class vtkGeoEdgeStrategy has several methods that can be used.
  They are listed below.
Note that the documentation is translated automatically from the VTK sources,
and may not be completely intelligible.  When in doubt, consult the VTK website.
In the methods listed below, \verb|obj| is an instance of the vtkGeoEdgeStrategy class.
\begin{itemize}
\item  \verb|string = obj.GetClassName ()|

\item  \verb|int = obj.IsA (string name)|

\item  \verb|vtkGeoEdgeStrategy = obj.NewInstance ()|

\item  \verb|vtkGeoEdgeStrategy = obj.SafeDownCast (vtkObject o)|

\item  \verb|obj.SetGlobeRadius (double )| -  The base radius used to determine the earth's surface.
 Default is the earth's radius in meters.
 TODO: Change this to take in a vtkGeoTerrain to get altitude.

\item  \verb|double = obj.GetGlobeRadius ()| -  The base radius used to determine the earth's surface.
 Default is the earth's radius in meters.
 TODO: Change this to take in a vtkGeoTerrain to get altitude.

\item  \verb|obj.SetExplodeFactor (double )| -  Factor on which to ''explode'' the arcs away from the surface.
 A value of 0.0 keeps the values on the surface.
 Values larger than 0.0 push the arcs away from the surface by a distance
 proportional to the distance between the points.
 The default is 0.2.

\item  \verb|double = obj.GetExplodeFactor ()| -  Factor on which to ''explode'' the arcs away from the surface.
 A value of 0.0 keeps the values on the surface.
 Values larger than 0.0 push the arcs away from the surface by a distance
 proportional to the distance between the points.
 The default is 0.2.

\item  \verb|obj.SetNumberOfSubdivisions (int )| -  The number of subdivisions in the arc.
 The default is 20.

\item  \verb|int = obj.GetNumberOfSubdivisions ()| -  The number of subdivisions in the arc.
 The default is 20.

\item  \verb|obj.Layout ()| -  Perform the layout.

\end{itemize}
