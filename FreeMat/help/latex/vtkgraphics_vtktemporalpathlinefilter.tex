\section{vtkTemporalPathLineFilter}

\subsection{Usage}

 vtkTemporalPathLineFilter takes any dataset as input, it extracts the point
 locations of all cells over time to build up a polyline trail.
 The point number (index) is used as the 'key' if the points are randomly 
 changing their respective order in the points list, then you should specify
 a scalar that represents the unique ID. This is intended to handle the output
 of a filter such as the TemporalStreamTracer.
 

To create an instance of class vtkTemporalPathLineFilter, simply
invoke its constructor as follows
\begin{verbatim}
  obj = vtkTemporalPathLineFilter
\end{verbatim}
\subsection{Methods}

The class vtkTemporalPathLineFilter has several methods that can be used.
  They are listed below.
Note that the documentation is translated automatically from the VTK sources,
and may not be completely intelligible.  When in doubt, consult the VTK website.
In the methods listed below, \verb|obj| is an instance of the vtkTemporalPathLineFilter class.
\begin{itemize}
\item  \verb|string = obj.GetClassName ()| -  Standard Type-Macro

\item  \verb|int = obj.IsA (string name)| -  Standard Type-Macro

\item  \verb|vtkTemporalPathLineFilter = obj.NewInstance ()| -  Standard Type-Macro

\item  \verb|vtkTemporalPathLineFilter = obj.SafeDownCast (vtkObject o)| -  Standard Type-Macro

\item  \verb|obj.SetMaskPoints (int )| -  Set the number of particles to track as a ratio of the input
 example: setting MaskPoints to 10 will track every 10th point

\item  \verb|int = obj.GetMaskPoints ()| -  Set the number of particles to track as a ratio of the input
 example: setting MaskPoints to 10 will track every 10th point

\item  \verb|obj.SetMaxTrackLength (int )| -  If the Particles being traced animate for a long time, the
 trails or traces will become long and stringy. Setting
 the MaxTraceTimeLength will limit how much of the trace
 is displayed. Tracks longer then the Max will disappear
 and the trace will apppear like a snake of fixed length
 which progresses as the particle moves

\item  \verb|int = obj.GetMaxTrackLength ()| -  If the Particles being traced animate for a long time, the
 trails or traces will become long and stringy. Setting
 the MaxTraceTimeLength will limit how much of the trace
 is displayed. Tracks longer then the Max will disappear
 and the trace will apppear like a snake of fixed length
 which progresses as the particle moves

\item  \verb|obj.SetIdChannelArray (string )| -  Specify the name of a scalar array which will be used to fetch
 the index of each point. This is necessary only if the particles
 change position (Id order) on each time step. The Id can be used
 to identify particles at each step and hence track them properly.
 If this array is NULL, the global point ids are used.  If an Id
 array cannot otherwise be found, the point index is used as the ID.

\item  \verb|string = obj.GetIdChannelArray ()| -  Specify the name of a scalar array which will be used to fetch
 the index of each point. This is necessary only if the particles
 change position (Id order) on each time step. The Id can be used
 to identify particles at each step and hence track them properly.
 If this array is NULL, the global point ids are used.  If an Id
 array cannot otherwise be found, the point index is used as the ID.

\item  \verb|obj.SetScalarArray (string )|

\item  \verb|string = obj.GetScalarArray ()|

\item  \verb|obj.SetMaxStepDistance (double , double , double )| -  If a particle disappears from one end of a simulation and reappears
 on the other side, the track left will be unrepresentative.
 Set a MaxStepDistance{x,y,z} which acts as a threshold above which
 if a step occurs larger than the value (for the dimension), the track will
 be dropped and restarted after the step. (ie the part before the wrap 
 around will be dropped and the newer part kept).

\item  \verb|obj.SetMaxStepDistance (double  a[3])| -  If a particle disappears from one end of a simulation and reappears
 on the other side, the track left will be unrepresentative.
 Set a MaxStepDistance{x,y,z} which acts as a threshold above which
 if a step occurs larger than the value (for the dimension), the track will
 be dropped and restarted after the step. (ie the part before the wrap 
 around will be dropped and the newer part kept).

\item  \verb|double = obj. GetMaxStepDistance ()| -  If a particle disappears from one end of a simulation and reappears
 on the other side, the track left will be unrepresentative.
 Set a MaxStepDistance{x,y,z} which acts as a threshold above which
 if a step occurs larger than the value (for the dimension), the track will
 be dropped and restarted after the step. (ie the part before the wrap 
 around will be dropped and the newer part kept).

\item  \verb|obj.SetKeepDeadTrails (int )| -  When a particle 'disappears', the trail belonging to it is removed from
 the list. When this flag is enabled, dead trails will persist
 until the next time the list is cleared. Use carefully as it may cause
 excessive memory consumption if left on by mistake.

\item  \verb|int = obj.GetKeepDeadTrails ()| -  When a particle 'disappears', the trail belonging to it is removed from
 the list. When this flag is enabled, dead trails will persist
 until the next time the list is cleared. Use carefully as it may cause
 excessive memory consumption if left on by mistake.

\item  \verb|obj.Flush ()| -  Flush will wipe any existing data so that traces can be restarted from
 whatever time step is next supplied.

\item  \verb|obj.SetSelectionConnection (vtkAlgorithmOutput algOutput)| -  Set a second input which is a selection. Particles with the same
 Id in the selection as the primary input will be chosen for pathlines
 Note that you must have the same IdChannelArray in the selection as the input

\item  \verb|obj.SetSelection (vtkDataSet input)| -  Set a second input which is a selection. Particles with the same
 Id in the selection as the primary input will be chosen for pathlines
 Note that you must have the same IdChannelArray in the selection as the input

\end{itemize}
