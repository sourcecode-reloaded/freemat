\section{vtkParametricFunctionSource}

\subsection{Usage}

 This class tessellates parametric functions. The user must specify how
 many points in the parametric coordinate directions are required (i.e.,
 the resolution), and the mode to use to generate scalars. 

 .SECTION Thanks
 Andrew Maclean a.maclean@cas.edu.au for creating and contributing the
 class.


To create an instance of class vtkParametricFunctionSource, simply
invoke its constructor as follows
\begin{verbatim}
  obj = vtkParametricFunctionSource
\end{verbatim}
\subsection{Methods}

The class vtkParametricFunctionSource has several methods that can be used.
  They are listed below.
Note that the documentation is translated automatically from the VTK sources,
and may not be completely intelligible.  When in doubt, consult the VTK website.
In the methods listed below, \verb|obj| is an instance of the vtkParametricFunctionSource class.
\begin{itemize}
\item  \verb|string = obj.GetClassName ()|

\item  \verb|int = obj.IsA (string name)|

\item  \verb|vtkParametricFunctionSource = obj.NewInstance ()|

\item  \verb|vtkParametricFunctionSource = obj.SafeDownCast (vtkObject o)|

\item  \verb|obj.SetParametricFunction (vtkParametricFunction )| -  Specify the parametric function to use to generate the tessellation.

\item  \verb|vtkParametricFunction = obj.GetParametricFunction ()| -  Specify the parametric function to use to generate the tessellation.

\item  \verb|obj.SetUResolution (int )| -  Set/Get the number of subdivisions / tessellations in the u parametric
 direction. Note that the number of tessellant points in the u 
 direction is the UResolution + 1.

\item  \verb|int = obj.GetUResolution ()| -  Set/Get the number of subdivisions / tessellations in the u parametric
 direction. Note that the number of tessellant points in the u 
 direction is the UResolution + 1.

\item  \verb|obj.SetVResolution (int )| -  Set/Get the number of subdivisions / tessellations in the v parametric
 direction. Note that the number of tessellant points in the v 
 direction is the VResolution + 1.

\item  \verb|int = obj.GetVResolution ()| -  Set/Get the number of subdivisions / tessellations in the v parametric
 direction. Note that the number of tessellant points in the v 
 direction is the VResolution + 1.

\item  \verb|obj.SetWResolution (int )| -  Set/Get the number of subdivisions / tessellations in the w parametric
 direction. Note that the number of tessellant points in the w 
 direction is the WResolution + 1.

\item  \verb|int = obj.GetWResolution ()| -  Set/Get the number of subdivisions / tessellations in the w parametric
 direction. Note that the number of tessellant points in the w 
 direction is the WResolution + 1.

\item  \verb|obj.GenerateTextureCoordinatesOn ()| -  Set/Get the generation of texture coordinates. This is off by
 default.
 Note that this is only applicable to parametric surfaces 
 whose parametric dimension is 2.
 Note that texturing may fail in some cases.

\item  \verb|obj.GenerateTextureCoordinatesOff ()| -  Set/Get the generation of texture coordinates. This is off by
 default.
 Note that this is only applicable to parametric surfaces 
 whose parametric dimension is 2.
 Note that texturing may fail in some cases.

\item  \verb|obj.SetGenerateTextureCoordinates (int )| -  Set/Get the generation of texture coordinates. This is off by
 default.
 Note that this is only applicable to parametric surfaces 
 whose parametric dimension is 2.
 Note that texturing may fail in some cases.

\item  \verb|int = obj.GetGenerateTextureCoordinates ()| -  Set/Get the generation of texture coordinates. This is off by
 default.
 Note that this is only applicable to parametric surfaces 
 whose parametric dimension is 2.
 Note that texturing may fail in some cases.

\item  \verb|obj.SetScalarMode (int )| -  Get/Set the mode used for the scalar data.  The options are:
 SCALAR\_NONE, (default) scalars are not generated.
 SCALAR\_U, the scalar is set to the u-value. 
 SCALAR\_V, the scalar is set to the v-value.
 SCALAR\_U0, the scalar is set to 1 if u = (u\_max - u\_min)/2 = u\_avg, 0 otherwise.
 SCALAR\_V0, the scalar is set to 1 if v = (v\_max - v\_min)/2 = v\_avg, 0 otherwise.
 SCALAR\_U0V0, the scalar is 
   set to 1 if u == u\_avg, 2 if v == v\_avg, 3 if u = u\_avg \&\& v = v\_avg, 0 otherwise.
 SCALAR\_MODULUS, the scalar is set to (sqrt(u*u+v*v)), this is measured relative to (u\_avg,v\_avg).
 SCALAR\_PHASE, the scalar is set to (atan2(v,u)) (in degrees, 0 to 360), this is measured relative to (u\_avg,v\_avg).
 SCALAR\_QUADRANT, the scalar is set to 1, 2, 3 or 4 
   depending upon the quadrant of the point (u,v).
 SCALAR\_X, the scalar is set to the x-value. 
 SCALAR\_Y, the scalar is set to the y-value. 
 SCALAR\_Z, the scalar is set to the z-value. 
 SCALAR\_DISTANCE, the scalar is set to (sqrt(x*x+y*y+z*z)). I.e. distance from the origin.
 SCALAR\_FUNCTION\_DEFINED, the scalar is set to the value returned from EvaluateScalar().

\item  \verb|int = obj.GetScalarModeMinValue ()| -  Get/Set the mode used for the scalar data.  The options are:
 SCALAR\_NONE, (default) scalars are not generated.
 SCALAR\_U, the scalar is set to the u-value. 
 SCALAR\_V, the scalar is set to the v-value.
 SCALAR\_U0, the scalar is set to 1 if u = (u\_max - u\_min)/2 = u\_avg, 0 otherwise.
 SCALAR\_V0, the scalar is set to 1 if v = (v\_max - v\_min)/2 = v\_avg, 0 otherwise.
 SCALAR\_U0V0, the scalar is 
   set to 1 if u == u\_avg, 2 if v == v\_avg, 3 if u = u\_avg \&\& v = v\_avg, 0 otherwise.
 SCALAR\_MODULUS, the scalar is set to (sqrt(u*u+v*v)), this is measured relative to (u\_avg,v\_avg).
 SCALAR\_PHASE, the scalar is set to (atan2(v,u)) (in degrees, 0 to 360), this is measured relative to (u\_avg,v\_avg).
 SCALAR\_QUADRANT, the scalar is set to 1, 2, 3 or 4 
   depending upon the quadrant of the point (u,v).
 SCALAR\_X, the scalar is set to the x-value. 
 SCALAR\_Y, the scalar is set to the y-value. 
 SCALAR\_Z, the scalar is set to the z-value. 
 SCALAR\_DISTANCE, the scalar is set to (sqrt(x*x+y*y+z*z)). I.e. distance from the origin.
 SCALAR\_FUNCTION\_DEFINED, the scalar is set to the value returned from EvaluateScalar().

\item  \verb|int = obj.GetScalarModeMaxValue ()| -  Get/Set the mode used for the scalar data.  The options are:
 SCALAR\_NONE, (default) scalars are not generated.
 SCALAR\_U, the scalar is set to the u-value. 
 SCALAR\_V, the scalar is set to the v-value.
 SCALAR\_U0, the scalar is set to 1 if u = (u\_max - u\_min)/2 = u\_avg, 0 otherwise.
 SCALAR\_V0, the scalar is set to 1 if v = (v\_max - v\_min)/2 = v\_avg, 0 otherwise.
 SCALAR\_U0V0, the scalar is 
   set to 1 if u == u\_avg, 2 if v == v\_avg, 3 if u = u\_avg \&\& v = v\_avg, 0 otherwise.
 SCALAR\_MODULUS, the scalar is set to (sqrt(u*u+v*v)), this is measured relative to (u\_avg,v\_avg).
 SCALAR\_PHASE, the scalar is set to (atan2(v,u)) (in degrees, 0 to 360), this is measured relative to (u\_avg,v\_avg).
 SCALAR\_QUADRANT, the scalar is set to 1, 2, 3 or 4 
   depending upon the quadrant of the point (u,v).
 SCALAR\_X, the scalar is set to the x-value. 
 SCALAR\_Y, the scalar is set to the y-value. 
 SCALAR\_Z, the scalar is set to the z-value. 
 SCALAR\_DISTANCE, the scalar is set to (sqrt(x*x+y*y+z*z)). I.e. distance from the origin.
 SCALAR\_FUNCTION\_DEFINED, the scalar is set to the value returned from EvaluateScalar().

\item  \verb|int = obj.GetScalarMode ()| -  Get/Set the mode used for the scalar data.  The options are:
 SCALAR\_NONE, (default) scalars are not generated.
 SCALAR\_U, the scalar is set to the u-value. 
 SCALAR\_V, the scalar is set to the v-value.
 SCALAR\_U0, the scalar is set to 1 if u = (u\_max - u\_min)/2 = u\_avg, 0 otherwise.
 SCALAR\_V0, the scalar is set to 1 if v = (v\_max - v\_min)/2 = v\_avg, 0 otherwise.
 SCALAR\_U0V0, the scalar is 
   set to 1 if u == u\_avg, 2 if v == v\_avg, 3 if u = u\_avg \&\& v = v\_avg, 0 otherwise.
 SCALAR\_MODULUS, the scalar is set to (sqrt(u*u+v*v)), this is measured relative to (u\_avg,v\_avg).
 SCALAR\_PHASE, the scalar is set to (atan2(v,u)) (in degrees, 0 to 360), this is measured relative to (u\_avg,v\_avg).
 SCALAR\_QUADRANT, the scalar is set to 1, 2, 3 or 4 
   depending upon the quadrant of the point (u,v).
 SCALAR\_X, the scalar is set to the x-value. 
 SCALAR\_Y, the scalar is set to the y-value. 
 SCALAR\_Z, the scalar is set to the z-value. 
 SCALAR\_DISTANCE, the scalar is set to (sqrt(x*x+y*y+z*z)). I.e. distance from the origin.
 SCALAR\_FUNCTION\_DEFINED, the scalar is set to the value returned from EvaluateScalar().

\item  \verb|obj.SetScalarModeToNone (void )| -  Get/Set the mode used for the scalar data.  The options are:
 SCALAR\_NONE, (default) scalars are not generated.
 SCALAR\_U, the scalar is set to the u-value. 
 SCALAR\_V, the scalar is set to the v-value.
 SCALAR\_U0, the scalar is set to 1 if u = (u\_max - u\_min)/2 = u\_avg, 0 otherwise.
 SCALAR\_V0, the scalar is set to 1 if v = (v\_max - v\_min)/2 = v\_avg, 0 otherwise.
 SCALAR\_U0V0, the scalar is 
   set to 1 if u == u\_avg, 2 if v == v\_avg, 3 if u = u\_avg \&\& v = v\_avg, 0 otherwise.
 SCALAR\_MODULUS, the scalar is set to (sqrt(u*u+v*v)), this is measured relative to (u\_avg,v\_avg).
 SCALAR\_PHASE, the scalar is set to (atan2(v,u)) (in degrees, 0 to 360), this is measured relative to (u\_avg,v\_avg).
 SCALAR\_QUADRANT, the scalar is set to 1, 2, 3 or 4 
   depending upon the quadrant of the point (u,v).
 SCALAR\_X, the scalar is set to the x-value. 
 SCALAR\_Y, the scalar is set to the y-value. 
 SCALAR\_Z, the scalar is set to the z-value. 
 SCALAR\_DISTANCE, the scalar is set to (sqrt(x*x+y*y+z*z)). I.e. distance from the origin.
 SCALAR\_FUNCTION\_DEFINED, the scalar is set to the value returned from EvaluateScalar().

\item  \verb|obj.SetScalarModeToU (void )| -  Get/Set the mode used for the scalar data.  The options are:
 SCALAR\_NONE, (default) scalars are not generated.
 SCALAR\_U, the scalar is set to the u-value. 
 SCALAR\_V, the scalar is set to the v-value.
 SCALAR\_U0, the scalar is set to 1 if u = (u\_max - u\_min)/2 = u\_avg, 0 otherwise.
 SCALAR\_V0, the scalar is set to 1 if v = (v\_max - v\_min)/2 = v\_avg, 0 otherwise.
 SCALAR\_U0V0, the scalar is 
   set to 1 if u == u\_avg, 2 if v == v\_avg, 3 if u = u\_avg \&\& v = v\_avg, 0 otherwise.
 SCALAR\_MODULUS, the scalar is set to (sqrt(u*u+v*v)), this is measured relative to (u\_avg,v\_avg).
 SCALAR\_PHASE, the scalar is set to (atan2(v,u)) (in degrees, 0 to 360), this is measured relative to (u\_avg,v\_avg).
 SCALAR\_QUADRANT, the scalar is set to 1, 2, 3 or 4 
   depending upon the quadrant of the point (u,v).
 SCALAR\_X, the scalar is set to the x-value. 
 SCALAR\_Y, the scalar is set to the y-value. 
 SCALAR\_Z, the scalar is set to the z-value. 
 SCALAR\_DISTANCE, the scalar is set to (sqrt(x*x+y*y+z*z)). I.e. distance from the origin.
 SCALAR\_FUNCTION\_DEFINED, the scalar is set to the value returned from EvaluateScalar().

\item  \verb|obj.SetScalarModeToV (void )| -  Get/Set the mode used for the scalar data.  The options are:
 SCALAR\_NONE, (default) scalars are not generated.
 SCALAR\_U, the scalar is set to the u-value. 
 SCALAR\_V, the scalar is set to the v-value.
 SCALAR\_U0, the scalar is set to 1 if u = (u\_max - u\_min)/2 = u\_avg, 0 otherwise.
 SCALAR\_V0, the scalar is set to 1 if v = (v\_max - v\_min)/2 = v\_avg, 0 otherwise.
 SCALAR\_U0V0, the scalar is 
   set to 1 if u == u\_avg, 2 if v == v\_avg, 3 if u = u\_avg \&\& v = v\_avg, 0 otherwise.
 SCALAR\_MODULUS, the scalar is set to (sqrt(u*u+v*v)), this is measured relative to (u\_avg,v\_avg).
 SCALAR\_PHASE, the scalar is set to (atan2(v,u)) (in degrees, 0 to 360), this is measured relative to (u\_avg,v\_avg).
 SCALAR\_QUADRANT, the scalar is set to 1, 2, 3 or 4 
   depending upon the quadrant of the point (u,v).
 SCALAR\_X, the scalar is set to the x-value. 
 SCALAR\_Y, the scalar is set to the y-value. 
 SCALAR\_Z, the scalar is set to the z-value. 
 SCALAR\_DISTANCE, the scalar is set to (sqrt(x*x+y*y+z*z)). I.e. distance from the origin.
 SCALAR\_FUNCTION\_DEFINED, the scalar is set to the value returned from EvaluateScalar().

\item  \verb|obj.SetScalarModeToU0 (void )| -  Get/Set the mode used for the scalar data.  The options are:
 SCALAR\_NONE, (default) scalars are not generated.
 SCALAR\_U, the scalar is set to the u-value. 
 SCALAR\_V, the scalar is set to the v-value.
 SCALAR\_U0, the scalar is set to 1 if u = (u\_max - u\_min)/2 = u\_avg, 0 otherwise.
 SCALAR\_V0, the scalar is set to 1 if v = (v\_max - v\_min)/2 = v\_avg, 0 otherwise.
 SCALAR\_U0V0, the scalar is 
   set to 1 if u == u\_avg, 2 if v == v\_avg, 3 if u = u\_avg \&\& v = v\_avg, 0 otherwise.
 SCALAR\_MODULUS, the scalar is set to (sqrt(u*u+v*v)), this is measured relative to (u\_avg,v\_avg).
 SCALAR\_PHASE, the scalar is set to (atan2(v,u)) (in degrees, 0 to 360), this is measured relative to (u\_avg,v\_avg).
 SCALAR\_QUADRANT, the scalar is set to 1, 2, 3 or 4 
   depending upon the quadrant of the point (u,v).
 SCALAR\_X, the scalar is set to the x-value. 
 SCALAR\_Y, the scalar is set to the y-value. 
 SCALAR\_Z, the scalar is set to the z-value. 
 SCALAR\_DISTANCE, the scalar is set to (sqrt(x*x+y*y+z*z)). I.e. distance from the origin.
 SCALAR\_FUNCTION\_DEFINED, the scalar is set to the value returned from EvaluateScalar().

\item  \verb|obj.SetScalarModeToV0 (void )| -  Get/Set the mode used for the scalar data.  The options are:
 SCALAR\_NONE, (default) scalars are not generated.
 SCALAR\_U, the scalar is set to the u-value. 
 SCALAR\_V, the scalar is set to the v-value.
 SCALAR\_U0, the scalar is set to 1 if u = (u\_max - u\_min)/2 = u\_avg, 0 otherwise.
 SCALAR\_V0, the scalar is set to 1 if v = (v\_max - v\_min)/2 = v\_avg, 0 otherwise.
 SCALAR\_U0V0, the scalar is 
   set to 1 if u == u\_avg, 2 if v == v\_avg, 3 if u = u\_avg \&\& v = v\_avg, 0 otherwise.
 SCALAR\_MODULUS, the scalar is set to (sqrt(u*u+v*v)), this is measured relative to (u\_avg,v\_avg).
 SCALAR\_PHASE, the scalar is set to (atan2(v,u)) (in degrees, 0 to 360), this is measured relative to (u\_avg,v\_avg).
 SCALAR\_QUADRANT, the scalar is set to 1, 2, 3 or 4 
   depending upon the quadrant of the point (u,v).
 SCALAR\_X, the scalar is set to the x-value. 
 SCALAR\_Y, the scalar is set to the y-value. 
 SCALAR\_Z, the scalar is set to the z-value. 
 SCALAR\_DISTANCE, the scalar is set to (sqrt(x*x+y*y+z*z)). I.e. distance from the origin.
 SCALAR\_FUNCTION\_DEFINED, the scalar is set to the value returned from EvaluateScalar().

\item  \verb|obj.SetScalarModeToU0V0 (void )| -  Get/Set the mode used for the scalar data.  The options are:
 SCALAR\_NONE, (default) scalars are not generated.
 SCALAR\_U, the scalar is set to the u-value. 
 SCALAR\_V, the scalar is set to the v-value.
 SCALAR\_U0, the scalar is set to 1 if u = (u\_max - u\_min)/2 = u\_avg, 0 otherwise.
 SCALAR\_V0, the scalar is set to 1 if v = (v\_max - v\_min)/2 = v\_avg, 0 otherwise.
 SCALAR\_U0V0, the scalar is 
   set to 1 if u == u\_avg, 2 if v == v\_avg, 3 if u = u\_avg \&\& v = v\_avg, 0 otherwise.
 SCALAR\_MODULUS, the scalar is set to (sqrt(u*u+v*v)), this is measured relative to (u\_avg,v\_avg).
 SCALAR\_PHASE, the scalar is set to (atan2(v,u)) (in degrees, 0 to 360), this is measured relative to (u\_avg,v\_avg).
 SCALAR\_QUADRANT, the scalar is set to 1, 2, 3 or 4 
   depending upon the quadrant of the point (u,v).
 SCALAR\_X, the scalar is set to the x-value. 
 SCALAR\_Y, the scalar is set to the y-value. 
 SCALAR\_Z, the scalar is set to the z-value. 
 SCALAR\_DISTANCE, the scalar is set to (sqrt(x*x+y*y+z*z)). I.e. distance from the origin.
 SCALAR\_FUNCTION\_DEFINED, the scalar is set to the value returned from EvaluateScalar().

\item  \verb|obj.SetScalarModeToModulus (void )| -  Get/Set the mode used for the scalar data.  The options are:
 SCALAR\_NONE, (default) scalars are not generated.
 SCALAR\_U, the scalar is set to the u-value. 
 SCALAR\_V, the scalar is set to the v-value.
 SCALAR\_U0, the scalar is set to 1 if u = (u\_max - u\_min)/2 = u\_avg, 0 otherwise.
 SCALAR\_V0, the scalar is set to 1 if v = (v\_max - v\_min)/2 = v\_avg, 0 otherwise.
 SCALAR\_U0V0, the scalar is 
   set to 1 if u == u\_avg, 2 if v == v\_avg, 3 if u = u\_avg \&\& v = v\_avg, 0 otherwise.
 SCALAR\_MODULUS, the scalar is set to (sqrt(u*u+v*v)), this is measured relative to (u\_avg,v\_avg).
 SCALAR\_PHASE, the scalar is set to (atan2(v,u)) (in degrees, 0 to 360), this is measured relative to (u\_avg,v\_avg).
 SCALAR\_QUADRANT, the scalar is set to 1, 2, 3 or 4 
   depending upon the quadrant of the point (u,v).
 SCALAR\_X, the scalar is set to the x-value. 
 SCALAR\_Y, the scalar is set to the y-value. 
 SCALAR\_Z, the scalar is set to the z-value. 
 SCALAR\_DISTANCE, the scalar is set to (sqrt(x*x+y*y+z*z)). I.e. distance from the origin.
 SCALAR\_FUNCTION\_DEFINED, the scalar is set to the value returned from EvaluateScalar().

\item  \verb|obj.SetScalarModeToPhase (void )| -  Get/Set the mode used for the scalar data.  The options are:
 SCALAR\_NONE, (default) scalars are not generated.
 SCALAR\_U, the scalar is set to the u-value. 
 SCALAR\_V, the scalar is set to the v-value.
 SCALAR\_U0, the scalar is set to 1 if u = (u\_max - u\_min)/2 = u\_avg, 0 otherwise.
 SCALAR\_V0, the scalar is set to 1 if v = (v\_max - v\_min)/2 = v\_avg, 0 otherwise.
 SCALAR\_U0V0, the scalar is 
   set to 1 if u == u\_avg, 2 if v == v\_avg, 3 if u = u\_avg \&\& v = v\_avg, 0 otherwise.
 SCALAR\_MODULUS, the scalar is set to (sqrt(u*u+v*v)), this is measured relative to (u\_avg,v\_avg).
 SCALAR\_PHASE, the scalar is set to (atan2(v,u)) (in degrees, 0 to 360), this is measured relative to (u\_avg,v\_avg).
 SCALAR\_QUADRANT, the scalar is set to 1, 2, 3 or 4 
   depending upon the quadrant of the point (u,v).
 SCALAR\_X, the scalar is set to the x-value. 
 SCALAR\_Y, the scalar is set to the y-value. 
 SCALAR\_Z, the scalar is set to the z-value. 
 SCALAR\_DISTANCE, the scalar is set to (sqrt(x*x+y*y+z*z)). I.e. distance from the origin.
 SCALAR\_FUNCTION\_DEFINED, the scalar is set to the value returned from EvaluateScalar().

\item  \verb|obj.SetScalarModeToQuadrant (void )| -  Get/Set the mode used for the scalar data.  The options are:
 SCALAR\_NONE, (default) scalars are not generated.
 SCALAR\_U, the scalar is set to the u-value. 
 SCALAR\_V, the scalar is set to the v-value.
 SCALAR\_U0, the scalar is set to 1 if u = (u\_max - u\_min)/2 = u\_avg, 0 otherwise.
 SCALAR\_V0, the scalar is set to 1 if v = (v\_max - v\_min)/2 = v\_avg, 0 otherwise.
 SCALAR\_U0V0, the scalar is 
   set to 1 if u == u\_avg, 2 if v == v\_avg, 3 if u = u\_avg \&\& v = v\_avg, 0 otherwise.
 SCALAR\_MODULUS, the scalar is set to (sqrt(u*u+v*v)), this is measured relative to (u\_avg,v\_avg).
 SCALAR\_PHASE, the scalar is set to (atan2(v,u)) (in degrees, 0 to 360), this is measured relative to (u\_avg,v\_avg).
 SCALAR\_QUADRANT, the scalar is set to 1, 2, 3 or 4 
   depending upon the quadrant of the point (u,v).
 SCALAR\_X, the scalar is set to the x-value. 
 SCALAR\_Y, the scalar is set to the y-value. 
 SCALAR\_Z, the scalar is set to the z-value. 
 SCALAR\_DISTANCE, the scalar is set to (sqrt(x*x+y*y+z*z)). I.e. distance from the origin.
 SCALAR\_FUNCTION\_DEFINED, the scalar is set to the value returned from EvaluateScalar().

\item  \verb|obj.SetScalarModeToX (void )| -  Get/Set the mode used for the scalar data.  The options are:
 SCALAR\_NONE, (default) scalars are not generated.
 SCALAR\_U, the scalar is set to the u-value. 
 SCALAR\_V, the scalar is set to the v-value.
 SCALAR\_U0, the scalar is set to 1 if u = (u\_max - u\_min)/2 = u\_avg, 0 otherwise.
 SCALAR\_V0, the scalar is set to 1 if v = (v\_max - v\_min)/2 = v\_avg, 0 otherwise.
 SCALAR\_U0V0, the scalar is 
   set to 1 if u == u\_avg, 2 if v == v\_avg, 3 if u = u\_avg \&\& v = v\_avg, 0 otherwise.
 SCALAR\_MODULUS, the scalar is set to (sqrt(u*u+v*v)), this is measured relative to (u\_avg,v\_avg).
 SCALAR\_PHASE, the scalar is set to (atan2(v,u)) (in degrees, 0 to 360), this is measured relative to (u\_avg,v\_avg).
 SCALAR\_QUADRANT, the scalar is set to 1, 2, 3 or 4 
   depending upon the quadrant of the point (u,v).
 SCALAR\_X, the scalar is set to the x-value. 
 SCALAR\_Y, the scalar is set to the y-value. 
 SCALAR\_Z, the scalar is set to the z-value. 
 SCALAR\_DISTANCE, the scalar is set to (sqrt(x*x+y*y+z*z)). I.e. distance from the origin.
 SCALAR\_FUNCTION\_DEFINED, the scalar is set to the value returned from EvaluateScalar().

\item  \verb|obj.SetScalarModeToY (void )| -  Get/Set the mode used for the scalar data.  The options are:
 SCALAR\_NONE, (default) scalars are not generated.
 SCALAR\_U, the scalar is set to the u-value. 
 SCALAR\_V, the scalar is set to the v-value.
 SCALAR\_U0, the scalar is set to 1 if u = (u\_max - u\_min)/2 = u\_avg, 0 otherwise.
 SCALAR\_V0, the scalar is set to 1 if v = (v\_max - v\_min)/2 = v\_avg, 0 otherwise.
 SCALAR\_U0V0, the scalar is 
   set to 1 if u == u\_avg, 2 if v == v\_avg, 3 if u = u\_avg \&\& v = v\_avg, 0 otherwise.
 SCALAR\_MODULUS, the scalar is set to (sqrt(u*u+v*v)), this is measured relative to (u\_avg,v\_avg).
 SCALAR\_PHASE, the scalar is set to (atan2(v,u)) (in degrees, 0 to 360), this is measured relative to (u\_avg,v\_avg).
 SCALAR\_QUADRANT, the scalar is set to 1, 2, 3 or 4 
   depending upon the quadrant of the point (u,v).
 SCALAR\_X, the scalar is set to the x-value. 
 SCALAR\_Y, the scalar is set to the y-value. 
 SCALAR\_Z, the scalar is set to the z-value. 
 SCALAR\_DISTANCE, the scalar is set to (sqrt(x*x+y*y+z*z)). I.e. distance from the origin.
 SCALAR\_FUNCTION\_DEFINED, the scalar is set to the value returned from EvaluateScalar().

\item  \verb|obj.SetScalarModeToZ (void )| -  Get/Set the mode used for the scalar data.  The options are:
 SCALAR\_NONE, (default) scalars are not generated.
 SCALAR\_U, the scalar is set to the u-value. 
 SCALAR\_V, the scalar is set to the v-value.
 SCALAR\_U0, the scalar is set to 1 if u = (u\_max - u\_min)/2 = u\_avg, 0 otherwise.
 SCALAR\_V0, the scalar is set to 1 if v = (v\_max - v\_min)/2 = v\_avg, 0 otherwise.
 SCALAR\_U0V0, the scalar is 
   set to 1 if u == u\_avg, 2 if v == v\_avg, 3 if u = u\_avg \&\& v = v\_avg, 0 otherwise.
 SCALAR\_MODULUS, the scalar is set to (sqrt(u*u+v*v)), this is measured relative to (u\_avg,v\_avg).
 SCALAR\_PHASE, the scalar is set to (atan2(v,u)) (in degrees, 0 to 360), this is measured relative to (u\_avg,v\_avg).
 SCALAR\_QUADRANT, the scalar is set to 1, 2, 3 or 4 
   depending upon the quadrant of the point (u,v).
 SCALAR\_X, the scalar is set to the x-value. 
 SCALAR\_Y, the scalar is set to the y-value. 
 SCALAR\_Z, the scalar is set to the z-value. 
 SCALAR\_DISTANCE, the scalar is set to (sqrt(x*x+y*y+z*z)). I.e. distance from the origin.
 SCALAR\_FUNCTION\_DEFINED, the scalar is set to the value returned from EvaluateScalar().

\item  \verb|obj.SetScalarModeToDistance (void )| -  Get/Set the mode used for the scalar data.  The options are:
 SCALAR\_NONE, (default) scalars are not generated.
 SCALAR\_U, the scalar is set to the u-value. 
 SCALAR\_V, the scalar is set to the v-value.
 SCALAR\_U0, the scalar is set to 1 if u = (u\_max - u\_min)/2 = u\_avg, 0 otherwise.
 SCALAR\_V0, the scalar is set to 1 if v = (v\_max - v\_min)/2 = v\_avg, 0 otherwise.
 SCALAR\_U0V0, the scalar is 
   set to 1 if u == u\_avg, 2 if v == v\_avg, 3 if u = u\_avg \&\& v = v\_avg, 0 otherwise.
 SCALAR\_MODULUS, the scalar is set to (sqrt(u*u+v*v)), this is measured relative to (u\_avg,v\_avg).
 SCALAR\_PHASE, the scalar is set to (atan2(v,u)) (in degrees, 0 to 360), this is measured relative to (u\_avg,v\_avg).
 SCALAR\_QUADRANT, the scalar is set to 1, 2, 3 or 4 
   depending upon the quadrant of the point (u,v).
 SCALAR\_X, the scalar is set to the x-value. 
 SCALAR\_Y, the scalar is set to the y-value. 
 SCALAR\_Z, the scalar is set to the z-value. 
 SCALAR\_DISTANCE, the scalar is set to (sqrt(x*x+y*y+z*z)). I.e. distance from the origin.
 SCALAR\_FUNCTION\_DEFINED, the scalar is set to the value returned from EvaluateScalar().

\item  \verb|obj.SetScalarModeToFunctionDefined (void )| -  Return the MTime also considering the parametric function.

\item  \verb|long = obj.GetMTime ()| -  Return the MTime also considering the parametric function.

\end{itemize}
