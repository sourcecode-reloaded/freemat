\section{vtkTransformPolyDataFilter}

\subsection{Usage}

 vtkTransformPolyDataFilter is a filter to transform point
 coordinates and associated point and cell normals and
 vectors. Other point and cell data is passed through the filter
 unchanged. This filter is specialized for polygonal data. See
 vtkTransformFilter for more general data.

 An alternative method of transformation is to use vtkActor's methods
 to scale, rotate, and translate objects. The difference between the
 two methods is that vtkActor's transformation simply effects where
 objects are rendered (via the graphics pipeline), whereas
 vtkTransformPolyDataFilter actually modifies point coordinates in the 
 visualization pipeline. This is necessary for some objects 
 (e.g., vtkProbeFilter) that require point coordinates as input.

To create an instance of class vtkTransformPolyDataFilter, simply
invoke its constructor as follows
\begin{verbatim}
  obj = vtkTransformPolyDataFilter
\end{verbatim}
\subsection{Methods}

The class vtkTransformPolyDataFilter has several methods that can be used.
  They are listed below.
Note that the documentation is translated automatically from the VTK sources,
and may not be completely intelligible.  When in doubt, consult the VTK website.
In the methods listed below, \verb|obj| is an instance of the vtkTransformPolyDataFilter class.
\begin{itemize}
\item  \verb|string = obj.GetClassName ()|

\item  \verb|int = obj.IsA (string name)|

\item  \verb|vtkTransformPolyDataFilter = obj.NewInstance ()|

\item  \verb|vtkTransformPolyDataFilter = obj.SafeDownCast (vtkObject o)|

\item  \verb|long = obj.GetMTime ()| -  Return the MTime also considering the transform.

\item  \verb|obj.SetTransform (vtkAbstractTransform )| -  Specify the transform object used to transform points.

\item  \verb|vtkAbstractTransform = obj.GetTransform ()| -  Specify the transform object used to transform points.

\end{itemize}
