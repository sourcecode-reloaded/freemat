\section{vtkMutableDirectedGraph}

\subsection{Usage}

 vtkMutableDirectedGraph is a directed graph which has additional methods
 for adding edges and vertices. AddChild() is a convenience method for
 constructing trees. ShallowCopy(), DeepCopy(), CheckedShallowCopy() and
 CheckedDeepCopy() will succeed for instances of vtkDirectedGraph,
 vtkMutableDirectedGraph and vtkTree.


To create an instance of class vtkMutableDirectedGraph, simply
invoke its constructor as follows
\begin{verbatim}
  obj = vtkMutableDirectedGraph
\end{verbatim}
\subsection{Methods}

The class vtkMutableDirectedGraph has several methods that can be used.
  They are listed below.
Note that the documentation is translated automatically from the VTK sources,
and may not be completely intelligible.  When in doubt, consult the VTK website.
In the methods listed below, \verb|obj| is an instance of the vtkMutableDirectedGraph class.
\begin{itemize}
\item  \verb|string = obj.GetClassName ()|

\item  \verb|int = obj.IsA (string name)|

\item  \verb|vtkMutableDirectedGraph = obj.NewInstance ()|

\item  \verb|vtkMutableDirectedGraph = obj.SafeDownCast (vtkObject o)|

\item  \verb|vtkIdType = obj.AddVertex ()| -  Adds a vertex to the graph and returns the index of the new vertex.

 Note: In a distributed graph (i.e. a graph whose DistributedHelper
 is non-null), this routine cannot be used to add a vertex
 if the vertices in the graph have pedigree IDs, because this routine
 will always add the vertex locally, which may conflict with the
 proper location of the vertex based on the distribution of the
 pedigree IDs.

\item  \verb|vtkIdType = obj.AddVertex (vtkVariantArray propertyArr)| -  Adds a vertex to the graph with associated properties defined in
  propertyArr and returns the index of the new vertex.
 The number and order of values in  propertyArr must match up with the
 arrays in the vertex data retrieved by GetVertexData().
 
 If a vertex with the given pedigree ID already exists, its properties will be
 overwritten with the properties in  propertyArr and the existing
 vertex index will be returned.

 Note: In a distributed graph (i.e. a graph whose DistributedHelper
 is non-null) the vertex added or found might not be local. In this case,
 AddVertex will wait until the vertex can be added or found
 remotely, so that the proper vertex index can be returned. If you
 don't actually need to use the vertex index, consider calling
 LazyAddVertex, which provides better performance by eliminating
 the delays associated with returning the vertex index.

\item  \verb|obj.LazyAddVertex ()| -  Adds a vertex to the graph.
 
 This method is lazily evaluated for distributed graphs (i.e. graphs
 whose DistributedHelper is non-null) the next time Synchronize is
 called on the helper.

\item  \verb|obj.LazyAddVertex (vtkVariantArray propertyArr)| -  Adds a vertex to the graph with associated properties defined in
  propertyArr.
 The number and order of values in  propertyArr must match up with the
 arrays in the vertex data retrieved by GetVertexData().
 
 If a vertex with the given pedigree ID already exists, its properties will be
 overwritten with the properties in  propertyArr.
 
 This method is lazily evaluated for distributed graphs (i.e. graphs
 whose DistributedHelper is non-null) the next time Synchronize is
 called on the helper.

\item  \verb|vtkGraphEdge = obj.AddGraphEdge (vtkIdType u, vtkIdType v)| -  Variant of AddEdge() that returns a heavyweight  vtkGraphEdge object.
 The graph owns the reference of the edge and will replace
 its contents on the next call to AddGraphEdge().
 
 Note: This is a less efficient method for use with wrappers.
 In C++ you should use the faster AddEdge().

\item  \verb|vtkIdType = obj.AddChild (vtkIdType parent, vtkVariantArray propertyArr)| -  Convenience method for creating trees.
 Returns the newly created vertex id.
 Shortcut for
 \begin{verbatim}
 vtkIdType v = g->AddVertex();
 g->AddEdge(parent, v);
 \end{verbatim}
 If non-null,  propertyArr provides edge properties
 for the newly-created edge. The values in  propertyArr must match
 up with the arrays in the edge data returned by GetEdgeData().

\item  \verb|vtkIdType = obj.AddChild (vtkIdType parent)| -  Removes the vertex from the graph along with any connected edges.
 Note: This invalidates the last vertex index, which is reassigned to v.

\item  \verb|obj.RemoveVertex (vtkIdType v)| -  Removes the vertex from the graph along with any connected edges.
 Note: This invalidates the last vertex index, which is reassigned to v.

\item  \verb|obj.RemoveEdge (vtkIdType e)| -  Removes the edge from the graph.
 Note: This invalidates the last edge index, which is reassigned to e.

\item  \verb|obj.RemoveVertices (vtkIdTypeArray arr)| -  Removes a collection of vertices from the graph along with any connected edges.

\item  \verb|obj.RemoveEdges (vtkIdTypeArray arr)| -  Removes a collection of edges from the graph.

\end{itemize}
