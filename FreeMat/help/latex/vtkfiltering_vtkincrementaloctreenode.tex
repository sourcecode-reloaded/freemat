\section{vtkIncrementalOctreeNode}

\subsection{Usage}

  Octree nodes serve as spatial sub-division primitives to build the search
  structure of an incremental octree in a recursive top-down manner. The
  hierarchy takes the form of a tree-like representation by which a parent
  node contains eight mutually non-overlapping child nodes. Each child is 
  assigned with an axis-aligned rectangular volume (Spatial Bounding Box) 
  and the eight children together cover exactly the same region as governed
  by their parent. The eight child nodes / octants are ordered as

  { (xBBoxMin, xBBoxMid] \& (yBBoxMin, yBBoxMid] \& (zBBoxMin, zBBoxMid] },
  { (xBBoxMid, xBBoxMax] \& (yBBoxMin, yBBoxMid] \& (zBBoxMin, zBBoxMid] },
  { (xBBoxMin, xBBoxMid] \& (yBBoxMid, yBBoxMax] \& (zBBoxMin, zBBoxMid] }, 
  { (xBBoxMid, xBBoxMax] \& (yBBoxMid, yBBoxMax] \& (zBBoxMin, zBBoxMid] },
  { (xBBoxMin, xBBoxMid] \& (yBBoxMin, yBBoxMid] \& (zBBoxMid, zBBoxMax] },
  { (xBBoxMid, xBBoxMax] \& (yBBoxMin, yBBoxMid] \& (zBBoxMid, zBBoxMax] },
  { (xBBoxMin, xBBoxMid] \& (yBBoxMid, yBBoxMax] \& (zBBoxMid, zBBoxMax] }, 
  { (xBBoxMid, xBBoxMax] \& (yBBoxMid, yBBoxMax] \& (zBBoxMid, zBBoxMax] },

  where { xrange \& yRange \& zRange } defines the region of each 3D octant. 
  In addition, the points falling within and registered, by means of point
  indices, in the parent node are distributed to the child nodes for delegated
  maintenance. In fact, only leaf nodes, i.e., those without any descendants,
  actually store point indices while each node, regardless of a leaf or non-
  leaf node, keeps a dynamically updated Data Bounding Box of the inhabitant 
  points, if any. Given a maximum number of points per leaf node, an octree 
  is initialized with an empty leaf node that is then recursively sub-divided, 
  but only on demand as points are incrementally inserted, to construct a 
  populated tree.

  Please note that this octree node class is able to handle a large number
  of EXACTLY duplicate points that is greater than the specified maximum
  number of points per leaf node. In other words, as an exception, a leaf
  node may maintain an arbitrary number of exactly duplicate points to deal
  with possible extreme cases.
 

To create an instance of class vtkIncrementalOctreeNode, simply
invoke its constructor as follows
\begin{verbatim}
  obj = vtkIncrementalOctreeNode
\end{verbatim}
\subsection{Methods}

The class vtkIncrementalOctreeNode has several methods that can be used.
  They are listed below.
Note that the documentation is translated automatically from the VTK sources,
and may not be completely intelligible.  When in doubt, consult the VTK website.
In the methods listed below, \verb|obj| is an instance of the vtkIncrementalOctreeNode class.
\begin{itemize}
\item  \verb|string = obj.GetClassName ()|

\item  \verb|int = obj.IsA (string name)|

\item  \verb|vtkIncrementalOctreeNode = obj.NewInstance ()|

\item  \verb|vtkIncrementalOctreeNode = obj.SafeDownCast (vtkObject o)|

\item  \verb|int = obj.GetNumberOfPoints ()| -  Get the number of points inside or under this node.

\item  \verb|vtkIdList = obj.GetPointIdSet ()| -  Get the list of point indices, NULL for a non-leaf node.

\item  \verb|obj.DeleteChildNodes ()| -  Delete the eight child nodes.

\item  \verb|obj.SetBounds (double x1, double x2, double y1, double y2, double z1, double z2)| -  Set the spatial bounding box of the node. This function sets a default
 data bounding box.

\item  \verb|obj.GetBounds (double bounds[6]) const| -  Get the spatial bounding box of the node. The values are returned via
 an array in order of: x\_min, x\_max, y\_min, y\_max, z\_min, z\_max.   

\item  \verb|double = obj. GetMinBounds ()| -  Get access to MinBounds. Do not free this pointer.

\item  \verb|double = obj. GetMaxBounds ()| -  Get access to MaxBounds. Do not free this pointer.

\item  \verb|int = obj.IsLeaf ()| -  Determine which specific child / octant contains a given point. Note that
 the point is assumed to be inside this node and no checking is performed 
 on the inside issue.

\item  \verb|int = obj.GetChildIndex (double point[3])| -  Determine which specific child / octant contains a given point. Note that
 the point is assumed to be inside this node and no checking is performed 
 on the inside issue.

\item  \verb|vtkIncrementalOctreeNode = obj.GetChild (int i)| -  A point is in a node if and only if MinBounds[i] < p[i] <= MaxBounds[i],
 which allows a node to be divided into eight non-overlapping children.

\item  \verb|int = obj.ContainsPoint (double pnt[3])| -  A point is in a node if and only if MinBounds[i] < p[i] <= MaxBounds[i],
 which allows a node to be divided into eight non-overlapping children.

\item  \verb|int = obj.ContainsPointByData (double pnt[3])| -  A point is in a node, in terms of data, if and only if MinDataBounds[i]
 <= p[i] <= MaxDataBounds[i].

\item  \verb|double = obj.GetDistance2ToInnerBoundary (double point[3], vtkIncrementalOctreeNode rootNode)| -  Given a point inside this node, get the minimum squared distance to all 
 inner boundaries. An inner boundary is a node's face that is shared by
 another non-root node. 

\item  \verb|double = obj.GetDistance2ToBoundary (double point[3], vtkIncrementalOctreeNode rootNode, int checkData)| -  Compute the minimum squared distance from a point to this node, with all
 six boundaries considered. The data bounding box is checked if checkData
 is non-zero. 

\item  \verb|double = obj.GetDistance2ToBoundary (double point[3], double closest[3], vtkIncrementalOctreeNode rootNode, int checkData)| -  Compute the minimum squared distance from a point to this node, with all
 six boundaries considered. The data bounding box is checked if checkData
 is non-zero. The closest on-boundary point is returned via closest.

\item  \verb|obj.ExportAllPointIdsByInsertion (vtkIdList idList)| -  Export all the indices of the points (contained in or under this node) by
 inserting them to an allocated vtkIdList via vtkIdList::InsertNextId().

\end{itemize}
