\section{vtkVisibilitySort}

\subsection{Usage}

 vtkVisibilitySort encapsulates a method for depth sorting the cells of a
 vtkDataSet for a given viewpoint.  It should be noted that subclasses
 are not required to give an absolutely correct sorting.  Many types of
 unstructured grids may have sorting cycles, meaning that there is no
 possible correct sorting.  Some subclasses also only give an approximate
 sorting in the interest of speed.

 .SECTION Note
 The Input field of this class tends to causes reference cycles.  To help
 break these cycles, garbage collection is enabled on this object and the
 input parameter is traced.  For this to work, though, an object in the
 loop holding the visibility sort should also report that to the garbage
 collector.


To create an instance of class vtkVisibilitySort, simply
invoke its constructor as follows
\begin{verbatim}
  obj = vtkVisibilitySort
\end{verbatim}
\subsection{Methods}

The class vtkVisibilitySort has several methods that can be used.
  They are listed below.
Note that the documentation is translated automatically from the VTK sources,
and may not be completely intelligible.  When in doubt, consult the VTK website.
In the methods listed below, \verb|obj| is an instance of the vtkVisibilitySort class.
\begin{itemize}
\item  \verb|string = obj.GetClassName ()|

\item  \verb|int = obj.IsA (string name)|

\item  \verb|vtkVisibilitySort = obj.NewInstance ()|

\item  \verb|vtkVisibilitySort = obj.SafeDownCast (vtkObject o)|

\item  \verb|obj.InitTraversal ()| -  To facilitate incremental sorting algorithms, the cells are retrieved
 in an iteration process.  That is, call InitTraversal to start the
 iteration and call GetNextCells to get the cell IDs in order.
 However, for efficiencies sake, GetNextCells returns an ordered list
 of several id's in once call (but not necessarily all).  GetNextCells
 will return NULL once the entire sorted list is output.  The
 vtkIdTypeArray returned from GetNextCells is a cached array, so do not
 delete it.  At the same note, do not expect the array to be valid
 after subsequent calls to GetNextCells.

\item  \verb|vtkIdTypeArray = obj.GetNextCells ()| -  To facilitate incremental sorting algorithms, the cells are retrieved
 in an iteration process.  That is, call InitTraversal to start the
 iteration and call GetNextCells to get the cell IDs in order.
 However, for efficiencies sake, GetNextCells returns an ordered list
 of several id's in once call (but not necessarily all).  GetNextCells
 will return NULL once the entire sorted list is output.  The
 vtkIdTypeArray returned from GetNextCells is a cached array, so do not
 delete it.  At the same note, do not expect the array to be valid
 after subsequent calls to GetNextCells.

\item  \verb|obj.SetMaxCellsReturned (int )| -  Set/Get the maximum number of cells that GetNextCells will return
 in one invocation.

\item  \verb|int = obj.GetMaxCellsReturnedMinValue ()| -  Set/Get the maximum number of cells that GetNextCells will return
 in one invocation.

\item  \verb|int = obj.GetMaxCellsReturnedMaxValue ()| -  Set/Get the maximum number of cells that GetNextCells will return
 in one invocation.

\item  \verb|int = obj.GetMaxCellsReturned ()| -  Set/Get the maximum number of cells that GetNextCells will return
 in one invocation.

\item  \verb|obj.SetModelTransform (vtkMatrix4x4 mat)| -  Set/Get the matrix that transforms from object space to world space.
 Generally, you get this matrix from a call to GetMatrix of a vtkProp3D
 (vtkActor).

\item  \verb|vtkMatrix4x4 = obj.GetModelTransform ()| -  Set/Get the matrix that transforms from object space to world space.
 Generally, you get this matrix from a call to GetMatrix of a vtkProp3D
 (vtkActor).

\item  \verb|vtkMatrix4x4 = obj.GetInverseModelTransform ()|

\item  \verb|obj.SetCamera (vtkCamera camera)| -  Set/Get the camera that specifies the viewing parameters.

\item  \verb|vtkCamera = obj.GetCamera ()| -  Set/Get the camera that specifies the viewing parameters.

\item  \verb|obj.SetInput (vtkDataSet data)| -  Set/Get the data set containing the cells to sort.

\item  \verb|vtkDataSet = obj.GetInput ()| -  Set/Get the data set containing the cells to sort.

\item  \verb|int = obj.GetDirection ()| -  Set/Get the sorting direction.  Be default, the direction is set
 to back to front.

\item  \verb|obj.SetDirection (int )| -  Set/Get the sorting direction.  Be default, the direction is set
 to back to front.

\item  \verb|obj.SetDirectionToBackToFront ()| -  Set/Get the sorting direction.  Be default, the direction is set
 to back to front.

\item  \verb|obj.SetDirectionToFrontToBack ()| -  Overwritten to enable garbage collection.

\item  \verb|obj.Register (vtkObjectBase o)| -  Overwritten to enable garbage collection.

\item  \verb|obj.UnRegister (vtkObjectBase o)| -  Overwritten to enable garbage collection.

\end{itemize}
