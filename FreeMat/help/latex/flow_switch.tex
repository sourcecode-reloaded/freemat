\section{SWITCH Switch statement}

\subsection{Usage}

The \verb|switch| statement is used to selective execute code
based on the value of either scalar value or a string.
The general syntax for a \verb|switch| statement is
\begin{verbatim}
  switch(expression)
    case test_expression_1
      statements
    case test_expression_2
      statements
    otherwise
      statements
  end
\end{verbatim}
The \verb|otherwise| clause is optional.  Note that each test
expression can either be a scalar value, a string to test
against (if the switch expression is a string), or a
\verb|cell-array| of expressions to test against.  Note that
unlike \verb|C| \verb|switch| statements, the FreeMat \verb|switch|
does not have fall-through, meaning that the statements
associated with the first matching case are executed, and
then the \verb|switch| ends.  Also, if the \verb|switch| expression
matches multiple \verb|case| expressions, only the first one
is executed.
\subsection{Examples}

Here is an example of a \verb|switch| expression that tests
against a string input:
\begin{verbatim}
    switch_test.m
function c = switch_test(a)
  switch(a)
    case {'lima beans','root beer'}
      c = 'food';
    case {'red','green','blue'}
      c = 'color';
    otherwise
      c = 'not sure';
  end
\end{verbatim}
Now we exercise the switch statements
\begin{verbatim}
--> switch_test('root beer')

ans = 
food
--> switch_test('red')

ans = 
color
--> switch_test('carpet')

ans = 
not sure
\end{verbatim}
