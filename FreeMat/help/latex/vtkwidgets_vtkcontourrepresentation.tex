\section{vtkContourRepresentation}

\subsection{Usage}

 The vtkContourRepresentation is a superclass for various types of
 representations for the vtkContourWidget.

 .SECTION Managing contour points
 The classes vtkContourRepresentationNode, vtkContourRepresentationInternals, 
 vtkContourRepresentationPoint manage the data structure used to represent 
 nodes and points on a contour. A contour may contain several nodes and 
 several more points. Nodes are usually the result of user clicked points on
 the contour. Additional points are created between nodes to generate a 
 smooth curve using some Interpolator. See the method \c SetLineInterpolator.
 ar
 The data structure stores both the world and display positions for every 
 point. (This may seem like a duplication.) The default behaviour of this 
 class is to use the WorldPosition to do all the math. Typically a point is
 added at a given display position. Its corresponding world position is 
 computed using the point placer and stored. Any query of the display 
 position of a stored point is done via the Renderer, which computes the 
 display position given a world position. 
 
 ar 
 So why maintain the display position ? Consider drawing a contour on a 
 volume widget. You might want the contour to be located at a certain world
 position in the volume or you might want to be overlayed over the window 
 like an Actor2D. The default behaviour of this class is to provide the 
 former behaviour. 

 ar
 To achieve the latter behaviour override the methods that return the display 
 position (to return the set display position instead of computing it from
 the world positions) and the method \c BuildLines() to interpolate lines 
 using their display positions intead of world positions.


To create an instance of class vtkContourRepresentation, simply
invoke its constructor as follows
\begin{verbatim}
  obj = vtkContourRepresentation
\end{verbatim}
\subsection{Methods}

The class vtkContourRepresentation has several methods that can be used.
  They are listed below.
Note that the documentation is translated automatically from the VTK sources,
and may not be completely intelligible.  When in doubt, consult the VTK website.
In the methods listed below, \verb|obj| is an instance of the vtkContourRepresentation class.
\begin{itemize}
\item  \verb|string = obj.GetClassName ()| -  Standard VTK methods.

\item  \verb|int = obj.IsA (string name)| -  Standard VTK methods.

\item  \verb|vtkContourRepresentation = obj.NewInstance ()| -  Standard VTK methods.

\item  \verb|vtkContourRepresentation = obj.SafeDownCast (vtkObject o)| -  Standard VTK methods.

\item  \verb|int = obj.AddNodeAtWorldPosition (double x, double y, double z)| -  Add a node at a specific world position. Returns 0 if the
 node could not be added, 1 otherwise.

\item  \verb|int = obj.AddNodeAtWorldPosition (double worldPos[3])| -  Add a node at a specific world position. Returns 0 if the
 node could not be added, 1 otherwise.

\item  \verb|int = obj.AddNodeAtWorldPosition (double worldPos[3], double worldOrient[9])| -  Add a node at a specific world position. Returns 0 if the
 node could not be added, 1 otherwise.

\item  \verb|int = obj.AddNodeAtDisplayPosition (double displayPos[2])| -  Add a node at a specific display position. This will be
 converted into a world position according to the current
 constraints of the point placer. Return 0 if a point could
 not be added, 1 otherwise.

\item  \verb|int = obj.AddNodeAtDisplayPosition (int displayPos[2])| -  Add a node at a specific display position. This will be
 converted into a world position according to the current
 constraints of the point placer. Return 0 if a point could
 not be added, 1 otherwise.

\item  \verb|int = obj.AddNodeAtDisplayPosition (int X, int Y)| -  Add a node at a specific display position. This will be
 converted into a world position according to the current
 constraints of the point placer. Return 0 if a point could
 not be added, 1 otherwise.

\item  \verb|int = obj.ActivateNode (double displayPos[2])| -  Given a display position, activate a node. The closest
 node within tolerance will be activated. If a node is
 activated, 1 will be returned, otherwise 0 will be
 returned.

\item  \verb|int = obj.ActivateNode (int displayPos[2])| -  Given a display position, activate a node. The closest
 node within tolerance will be activated. If a node is
 activated, 1 will be returned, otherwise 0 will be
 returned.

\item  \verb|int = obj.ActivateNode (int X, int Y)| -  Given a display position, activate a node. The closest
 node within tolerance will be activated. If a node is
 activated, 1 will be returned, otherwise 0 will be
 returned.

\item  \verb|int = obj.SetActiveNodeToWorldPosition (double pos[3])|

\item  \verb|int = obj.SetActiveNodeToWorldPosition (double pos[3], double orient[9])|

\item  \verb|int = obj.SetActiveNodeToDisplayPosition (double pos[2])| -  Move the active node based on a specified display position.
 The display position will be converted into a world
 position. If the new position is not valid or there is
 no active node, a 0 will be returned. Otherwise, on
 success a 1 will be returned.

\item  \verb|int = obj.SetActiveNodeToDisplayPosition (int pos[2])| -  Move the active node based on a specified display position.
 The display position will be converted into a world
 position. If the new position is not valid or there is
 no active node, a 0 will be returned. Otherwise, on
 success a 1 will be returned.

\item  \verb|int = obj.SetActiveNodeToDisplayPosition (int X, int Y)| -  Move the active node based on a specified display position.
 The display position will be converted into a world
 position. If the new position is not valid or there is
 no active node, a 0 will be returned. Otherwise, on
 success a 1 will be returned.

\item  \verb|int = obj.ToggleActiveNodeSelected ()| -  Set/Get whether the active or nth node is selected. 

\item  \verb|int = obj.GetActiveNodeSelected ()| -  Set/Get whether the active or nth node is selected. 

\item  \verb|int = obj.GetNthNodeSelected (int )| -  Set/Get whether the active or nth node is selected. 

\item  \verb|int = obj.SetNthNodeSelected (int )| -  Set/Get whether the active or nth node is selected. 

\item  \verb|int = obj.GetActiveNodeWorldPosition (double pos[3])| -  Get the world position of the active node. Will return
 0 if there is no active node, or 1 otherwise.

\item  \verb|int = obj.GetActiveNodeWorldOrientation (double orient[9])| -  Get the world orientation of the active node. Will return
 0 if there is no active node, or 1 otherwise.  

\item  \verb|int = obj.GetActiveNodeDisplayPosition (double pos[2])| -  Get the display position of the active node. Will return
 0 if there is no active node, or 1 otherwise.

\item  \verb|int = obj.GetNumberOfNodes ()| -  Get the number of nodes.

\item  \verb|int = obj.GetNthNodeDisplayPosition (int n, double pos[2])| -  Get the nth node's display position. Will return
 1 on success, or 0 if there are not at least 
 (n+1) nodes (0 based counting).

\item  \verb|int = obj.GetNthNodeWorldPosition (int n, double pos[3])| -  Get the nth node's world position. Will return
 1 on success, or 0 if there are not at least 
 (n+1) nodes (0 based counting).

\item  \verb|int = obj.GetNthNodeWorldOrientation (int n, double orient[9])| -  Get the nth node's world orientation. Will return
 1 on success, or 0 if there are not at least 
 (n+1) nodes (0 based counting).

\item  \verb|int = obj.SetNthNodeDisplayPosition (int n, int X, int Y)| -  Set the nth node's display position. Display position
 will be converted into world position according to the
 constraints of the point placer. Will return
 1 on success, or 0 if there are not at least 
 (n+1) nodes (0 based counting) or the world position
 is not valid.

\item  \verb|int = obj.SetNthNodeDisplayPosition (int n, int pos[2])| -  Set the nth node's display position. Display position
 will be converted into world position according to the
 constraints of the point placer. Will return
 1 on success, or 0 if there are not at least 
 (n+1) nodes (0 based counting) or the world position
 is not valid.

\item  \verb|int = obj.SetNthNodeDisplayPosition (int n, double pos[2])| -  Set the nth node's display position. Display position
 will be converted into world position according to the
 constraints of the point placer. Will return
 1 on success, or 0 if there are not at least 
 (n+1) nodes (0 based counting) or the world position
 is not valid.

\item  \verb|int = obj.SetNthNodeWorldPosition (int n, double pos[3])| -  Set the nth node's world position. Will return
 1 on success, or 0 if there are not at least 
 (n+1) nodes (0 based counting) or the world
 position is not valid according to the point
 placer.

\item  \verb|int = obj.SetNthNodeWorldPosition (int n, double pos[3], double orient[9])| -  Set the nth node's world position. Will return
 1 on success, or 0 if there are not at least 
 (n+1) nodes (0 based counting) or the world
 position is not valid according to the point
 placer.

\item  \verb|int = obj.GetNthNodeSlope (int idx, double slope[3])| -  Get the nth node's slope. Will return
 1 on success, or 0 if there are not at least 
 (n+1) nodes (0 based counting).

\item  \verb|int = obj.GetNumberOfIntermediatePoints (int n)|

\item  \verb|int = obj.GetIntermediatePointWorldPosition (int n, int idx, double point[3])| -  Get the world position of the intermediate point at
 index idx between nodes n and (n+1) (or n and 0 if
 n is the last node and the loop is closed). Returns
 1 on success or 0 if n or idx are out of range.

\item  \verb|int = obj.AddIntermediatePointWorldPosition (int n, double point[3])| -  Add an intermediate point between node n and n+1
 (or n and 0 if n is the last node and the loop is closed).
 Returns 1 on success or 0 if n is out of range.

\item  \verb|int = obj.DeleteLastNode ()| -  Delete the last node. Returns 1 on success or 0 if 
 there were not any nodes.

\item  \verb|int = obj.DeleteActiveNode ()| -  Delete the active node. Returns 1 on success or 0 if
 the active node did not indicate a valid node.

\item  \verb|int = obj.DeleteNthNode (int n)| -  Delete the nth node. Return 1 on success or 0 if n
 is out of range.

\item  \verb|obj.ClearAllNodes ()| -  Delete all nodes. 

\item  \verb|int = obj.AddNodeOnContour (int X, int Y)| -  Given a specific X, Y pixel location, add a new node 
 on the contour at this location. 

\item  \verb|obj.SetPixelTolerance (int )| -  The tolerance to use when calculations are performed in 
 display coordinates

\item  \verb|int = obj.GetPixelToleranceMinValue ()| -  The tolerance to use when calculations are performed in 
 display coordinates

\item  \verb|int = obj.GetPixelToleranceMaxValue ()| -  The tolerance to use when calculations are performed in 
 display coordinates

\item  \verb|int = obj.GetPixelTolerance ()| -  The tolerance to use when calculations are performed in 
 display coordinates

\item  \verb|obj.SetWorldTolerance (double )| -  The tolerance to use when calculations are performed in
 world coordinates

\item  \verb|double = obj.GetWorldToleranceMinValue ()| -  The tolerance to use when calculations are performed in
 world coordinates

\item  \verb|double = obj.GetWorldToleranceMaxValue ()| -  The tolerance to use when calculations are performed in
 world coordinates

\item  \verb|double = obj.GetWorldTolerance ()| -  The tolerance to use when calculations are performed in
 world coordinates

\item  \verb|int = obj.GetCurrentOperation ()| -  Set / get the current operation. The widget is either
 inactive, or it is being translated.

\item  \verb|obj.SetCurrentOperation (int )| -  Set / get the current operation. The widget is either
 inactive, or it is being translated.

\item  \verb|int = obj.GetCurrentOperationMinValue ()| -  Set / get the current operation. The widget is either
 inactive, or it is being translated.

\item  \verb|int = obj.GetCurrentOperationMaxValue ()| -  Set / get the current operation. The widget is either
 inactive, or it is being translated.

\item  \verb|obj.SetCurrentOperationToInactive ()| -  Set / get the current operation. The widget is either
 inactive, or it is being translated.

\item  \verb|obj.SetCurrentOperationToTranslate ()| -  Set / get the current operation. The widget is either
 inactive, or it is being translated.

\item  \verb|obj.SetCurrentOperationToShift ()| -  Set / get the current operation. The widget is either
 inactive, or it is being translated.

\item  \verb|obj.SetCurrentOperationToScale ()|

\item  \verb|obj.SetPointPlacer (vtkPointPlacer )|

\item  \verb|vtkPointPlacer = obj.GetPointPlacer ()|

\item  \verb|obj.SetLineInterpolator (vtkContourLineInterpolator )| -  Set / Get the Line Interpolator. The line interpolator
 is responsible for generating the line segments connecting
 nodes.

\item  \verb|vtkContourLineInterpolator = obj.GetLineInterpolator ()| -  Set / Get the Line Interpolator. The line interpolator
 is responsible for generating the line segments connecting
 nodes.

\item  \verb|obj.BuildRepresentation ()| -  These are methods that satisfy vtkWidgetRepresentation's API.

\item  \verb|int = obj.ComputeInteractionState (int X, int Y, int modified)| -  These are methods that satisfy vtkWidgetRepresentation's API.

\item  \verb|obj.StartWidgetInteraction (double e[2])| -  These are methods that satisfy vtkWidgetRepresentation's API.

\item  \verb|obj.WidgetInteraction (double e[2])| -  These are methods that satisfy vtkWidgetRepresentation's API.

\item  \verb|obj.ReleaseGraphicsResources (vtkWindow w)| -  Methods required by vtkProp superclass.

\item  \verb|int = obj.RenderOverlay (vtkViewport viewport)| -  Methods required by vtkProp superclass.

\item  \verb|int = obj.RenderOpaqueGeometry (vtkViewport viewport)| -  Methods required by vtkProp superclass.

\item  \verb|int = obj.RenderTranslucentPolygonalGeometry (vtkViewport viewport)| -  Methods required by vtkProp superclass.

\item  \verb|int = obj.HasTranslucentPolygonalGeometry ()| -  Methods required by vtkProp superclass.

\item  \verb|obj.SetClosedLoop (int val)| -  Set / Get the ClosedLoop value. This ivar indicates whether the contour
 forms a closed loop. 

\item  \verb|int = obj.GetClosedLoop ()| -  Set / Get the ClosedLoop value. This ivar indicates whether the contour
 forms a closed loop. 

\item  \verb|obj.ClosedLoopOn ()| -  Set / Get the ClosedLoop value. This ivar indicates whether the contour
 forms a closed loop. 

\item  \verb|obj.ClosedLoopOff ()| -  Set / Get the ClosedLoop value. This ivar indicates whether the contour
 forms a closed loop. 

\item  \verb|obj.SetShowSelectedNodes (int )| -  A flag to indicate whether to show the Selected nodes
 Default is to set it to false.

\item  \verb|int = obj.GetShowSelectedNodes ()| -  A flag to indicate whether to show the Selected nodes
 Default is to set it to false.

\item  \verb|obj.ShowSelectedNodesOn ()| -  A flag to indicate whether to show the Selected nodes
 Default is to set it to false.

\item  \verb|obj.ShowSelectedNodesOff ()| -  A flag to indicate whether to show the Selected nodes
 Default is to set it to false.

\item  \verb|obj.GetNodePolyData (vtkPolyData poly)| -  Get the nodes and not the intermediate points in this 
 contour as a vtkPolyData.

\end{itemize}
