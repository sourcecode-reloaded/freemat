\section{vtkDistanceRepresentation}

\subsection{Usage}

 The vtkDistanceRepresentation is a superclass for various types of
 representations for the vtkDistanceWidget. Logically subclasses consist of
 an axis and two handles for placing/manipulating the end points.

To create an instance of class vtkDistanceRepresentation, simply
invoke its constructor as follows
\begin{verbatim}
  obj = vtkDistanceRepresentation
\end{verbatim}
\subsection{Methods}

The class vtkDistanceRepresentation has several methods that can be used.
  They are listed below.
Note that the documentation is translated automatically from the VTK sources,
and may not be completely intelligible.  When in doubt, consult the VTK website.
In the methods listed below, \verb|obj| is an instance of the vtkDistanceRepresentation class.
\begin{itemize}
\item  \verb|string = obj.GetClassName ()| -  Standard VTK methods.

\item  \verb|int = obj.IsA (string name)| -  Standard VTK methods.

\item  \verb|vtkDistanceRepresentation = obj.NewInstance ()| -  Standard VTK methods.

\item  \verb|vtkDistanceRepresentation = obj.SafeDownCast (vtkObject o)| -  Standard VTK methods.

\item  \verb|double = obj.GetDistance ()| -  This representation and all subclasses must keep a distance
 consistent with the state of the widget.

\item  \verb|obj.GetPoint1WorldPosition (double pos[3])| -  Methods to Set/Get the coordinates of the two points defining
 this representation. Note that methods are available for both
 display and world coordinates.

\item  \verb|obj.GetPoint2WorldPosition (double pos[3])| -  Methods to Set/Get the coordinates of the two points defining
 this representation. Note that methods are available for both
 display and world coordinates.

\item  \verb|double = obj.GetPoint1WorldPosition ()| -  Methods to Set/Get the coordinates of the two points defining
 this representation. Note that methods are available for both
 display and world coordinates.

\item  \verb|double = obj.GetPoint2WorldPosition ()| -  Methods to Set/Get the coordinates of the two points defining
 this representation. Note that methods are available for both
 display and world coordinates.

\item  \verb|obj.SetPoint1DisplayPosition (double pos[3])| -  Methods to Set/Get the coordinates of the two points defining
 this representation. Note that methods are available for both
 display and world coordinates.

\item  \verb|obj.SetPoint2DisplayPosition (double pos[3])| -  Methods to Set/Get the coordinates of the two points defining
 this representation. Note that methods are available for both
 display and world coordinates.

\item  \verb|obj.GetPoint1DisplayPosition (double pos[3])| -  Methods to Set/Get the coordinates of the two points defining
 this representation. Note that methods are available for both
 display and world coordinates.

\item  \verb|obj.GetPoint2DisplayPosition (double pos[3])| -  Methods to Set/Get the coordinates of the two points defining
 this representation. Note that methods are available for both
 display and world coordinates.

\item  \verb|obj.SetPoint1WorldPosition (double pos[3])| -  Methods to Set/Get the coordinates of the two points defining
 this representation. Note that methods are available for both
 display and world coordinates.

\item  \verb|obj.SetPoint2WorldPosition (double pos[3])| -  Methods to Set/Get the coordinates of the two points defining
 this representation. Note that methods are available for both
 display and world coordinates.

\item  \verb|obj.SetHandleRepresentation (vtkHandleRepresentation handle)| -  This method is used to specify the type of handle representation to
 use for the two internal vtkHandleWidgets within vtkDistanceWidget.
 To use this method, create a dummy vtkHandleWidget (or subclass),
 and then invoke this method with this dummy. Then the 
 vtkDistanceRepresentation uses this dummy to clone two vtkHandleWidgets
 of the same type. Make sure you set the handle representation before
 the widget is enabled. (The method InstantiateHandleRepresentation()
 is invoked by the vtkDistance widget.)

\item  \verb|obj.InstantiateHandleRepresentation ()| -  This method is used to specify the type of handle representation to
 use for the two internal vtkHandleWidgets within vtkDistanceWidget.
 To use this method, create a dummy vtkHandleWidget (or subclass),
 and then invoke this method with this dummy. Then the 
 vtkDistanceRepresentation uses this dummy to clone two vtkHandleWidgets
 of the same type. Make sure you set the handle representation before
 the widget is enabled. (The method InstantiateHandleRepresentation()
 is invoked by the vtkDistance widget.)

\item  \verb|vtkHandleRepresentation = obj.GetPoint1Representation ()| -  Set/Get the two handle representations used for the vtkDistanceWidget. (Note:
 properties can be set by grabbing these representations and setting the
 properties appropriately.)

\item  \verb|vtkHandleRepresentation = obj.GetPoint2Representation ()| -  Set/Get the two handle representations used for the vtkDistanceWidget. (Note:
 properties can be set by grabbing these representations and setting the
 properties appropriately.)

\item  \verb|obj.SetTolerance (int )| -  The tolerance representing the distance to the widget (in pixels) in
 which the cursor is considered near enough to the end points of
 the widget to be active.

\item  \verb|int = obj.GetToleranceMinValue ()| -  The tolerance representing the distance to the widget (in pixels) in
 which the cursor is considered near enough to the end points of
 the widget to be active.

\item  \verb|int = obj.GetToleranceMaxValue ()| -  The tolerance representing the distance to the widget (in pixels) in
 which the cursor is considered near enough to the end points of
 the widget to be active.

\item  \verb|int = obj.GetTolerance ()| -  The tolerance representing the distance to the widget (in pixels) in
 which the cursor is considered near enough to the end points of
 the widget to be active.

\item  \verb|obj.SetLabelFormat (string )| -  Specify the format to use for labelling the distance. Note that an empty
 string results in no label, or a format string without a ''%'' character
 will not print the distance value.

\item  \verb|string = obj.GetLabelFormat ()| -  Specify the format to use for labelling the distance. Note that an empty
 string results in no label, or a format string without a ''%'' character
 will not print the distance value.

\item  \verb|obj.BuildRepresentation ()| -  These are methods that satisfy vtkWidgetRepresentation's API.

\item  \verb|int = obj.ComputeInteractionState (int X, int Y, int modify)| -  These are methods that satisfy vtkWidgetRepresentation's API.

\item  \verb|obj.StartWidgetInteraction (double e[2])| -  These are methods that satisfy vtkWidgetRepresentation's API.

\item  \verb|obj.WidgetInteraction (double e[2])| -  These are methods that satisfy vtkWidgetRepresentation's API.

\end{itemize}
