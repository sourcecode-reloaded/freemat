\section{vtkCompositeDataSet}

\subsection{Usage}

 vtkCompositeDataSet is an abstract class that represents a collection
 of datasets (including other composite datasets). It
 provides an interface to access the datasets through iterators.
 vtkCompositeDataSet provides methods that are used by subclasses to store the
 datasets.
 vtkCompositeDataSet provides the datastructure for a full tree 
 representation. Subclasses provide the semantics for it and control how
 this tree is built.

To create an instance of class vtkCompositeDataSet, simply
invoke its constructor as follows
\begin{verbatim}
  obj = vtkCompositeDataSet
\end{verbatim}
\subsection{Methods}

The class vtkCompositeDataSet has several methods that can be used.
  They are listed below.
Note that the documentation is translated automatically from the VTK sources,
and may not be completely intelligible.  When in doubt, consult the VTK website.
In the methods listed below, \verb|obj| is an instance of the vtkCompositeDataSet class.
\begin{itemize}
\item  \verb|string = obj.GetClassName ()|

\item  \verb|int = obj.IsA (string name)|

\item  \verb|vtkCompositeDataSet = obj.NewInstance ()|

\item  \verb|vtkCompositeDataSet = obj.SafeDownCast (vtkObject o)|

\item  \verb|vtkCompositeDataIterator = obj.NewIterator ()| -  Return a new iterator (the iterator has to be deleted by user).

\item  \verb|int = obj.GetDataObjectType ()| -  Get the port currently producing this object.

\item  \verb|vtkAlgorithmOutput = obj.GetProducerPort ()| -  Get the port currently producing this object.

\item  \verb|obj.CopyStructure (vtkCompositeDataSet input)| -  Copies the tree structure from the input. All pointers to non-composite
 data objects are intialized to NULL. This also shallow copies the meta data
 associated with all the nodes.

\item  \verb|obj.SetDataSet (vtkCompositeDataIterator iter, vtkDataObject dataObj)| -  Sets the data set at the location pointed by the iterator.
 The iterator does not need to be iterating over this dataset itself. It can
 be any composite datasite with similar structure (achieved by using
 CopyStructure).

\item  \verb|vtkDataObject = obj.GetDataSet (vtkCompositeDataIterator iter)| -  Returns the dataset located at the positiong pointed by the iterator.
 The iterator does not need to be iterating over this dataset itself. It can
 be an iterator for composite dataset with similar structure (achieved by
 using CopyStructure).

\item  \verb|vtkInformation = obj.GetMetaData (vtkCompositeDataIterator iter)| -  Returns the meta-data associated with the position pointed by the iterator.
 This will create a new vtkInformation object if none already exists. Use
 HasMetaData to avoid creating the vtkInformation object unnecessarily.
 The iterator does not need to be iterating over this dataset itself. It can
 be an iterator for composite dataset with similar structure (achieved by
 using CopyStructure).

\item  \verb|int = obj.HasMetaData (vtkCompositeDataIterator iter)| -  Returns if any meta-data associated with the position pointed by the iterator.
 The iterator does not need to be iterating over this dataset itself. It can
 be an iterator for composite dataset with similar structure (achieved by
 using CopyStructure).

\item  \verb|obj.Initialize ()| -  Restore data object to initial state,

\item  \verb|obj.ShallowCopy (vtkDataObject src)| -  Shallow and Deep copy.

\item  \verb|obj.DeepCopy (vtkDataObject src)| -  Shallow and Deep copy.

\item  \verb|vtkIdType = obj.GetNumberOfPoints ()| -  Returns the total number of points of all blocks. This will
 iterate over all blocks and call GetNumberOfPoints() so it
 might be expansive.

\end{itemize}
