\section{vtkUnstructuredGridVolumeZSweepMapper}

\subsection{Usage}

 This is a volume mapper for unstructured grid implemented with the ZSweep
 algorithm. This is a software projective method.

To create an instance of class vtkUnstructuredGridVolumeZSweepMapper, simply
invoke its constructor as follows
\begin{verbatim}
  obj = vtkUnstructuredGridVolumeZSweepMapper
\end{verbatim}
\subsection{Methods}

The class vtkUnstructuredGridVolumeZSweepMapper has several methods that can be used.
  They are listed below.
Note that the documentation is translated automatically from the VTK sources,
and may not be completely intelligible.  When in doubt, consult the VTK website.
In the methods listed below, \verb|obj| is an instance of the vtkUnstructuredGridVolumeZSweepMapper class.
\begin{itemize}
\item  \verb|string = obj.GetClassName ()|

\item  \verb|int = obj.IsA (string name)|

\item  \verb|vtkUnstructuredGridVolumeZSweepMapper = obj.NewInstance ()|

\item  \verb|vtkUnstructuredGridVolumeZSweepMapper = obj.SafeDownCast (vtkObject o)|

\item  \verb|obj.SetImageSampleDistance (float )| -  Sampling distance in the XY image dimensions. Default value of 1 meaning
 1 ray cast per pixel. If set to 0.5, 4 rays will be cast per pixel. If
 set to 2.0, 1 ray will be cast for every 4 (2 by 2) pixels.

\item  \verb|float = obj.GetImageSampleDistanceMinValue ()| -  Sampling distance in the XY image dimensions. Default value of 1 meaning
 1 ray cast per pixel. If set to 0.5, 4 rays will be cast per pixel. If
 set to 2.0, 1 ray will be cast for every 4 (2 by 2) pixels.

\item  \verb|float = obj.GetImageSampleDistanceMaxValue ()| -  Sampling distance in the XY image dimensions. Default value of 1 meaning
 1 ray cast per pixel. If set to 0.5, 4 rays will be cast per pixel. If
 set to 2.0, 1 ray will be cast for every 4 (2 by 2) pixels.

\item  \verb|float = obj.GetImageSampleDistance ()| -  Sampling distance in the XY image dimensions. Default value of 1 meaning
 1 ray cast per pixel. If set to 0.5, 4 rays will be cast per pixel. If
 set to 2.0, 1 ray will be cast for every 4 (2 by 2) pixels.

\item  \verb|obj.SetMinimumImageSampleDistance (float )| -  This is the minimum image sample distance allow when the image
 sample distance is being automatically adjusted

\item  \verb|float = obj.GetMinimumImageSampleDistanceMinValue ()| -  This is the minimum image sample distance allow when the image
 sample distance is being automatically adjusted

\item  \verb|float = obj.GetMinimumImageSampleDistanceMaxValue ()| -  This is the minimum image sample distance allow when the image
 sample distance is being automatically adjusted

\item  \verb|float = obj.GetMinimumImageSampleDistance ()| -  This is the minimum image sample distance allow when the image
 sample distance is being automatically adjusted

\item  \verb|obj.SetMaximumImageSampleDistance (float )| -  This is the maximum image sample distance allow when the image
 sample distance is being automatically adjusted

\item  \verb|float = obj.GetMaximumImageSampleDistanceMinValue ()| -  This is the maximum image sample distance allow when the image
 sample distance is being automatically adjusted

\item  \verb|float = obj.GetMaximumImageSampleDistanceMaxValue ()| -  This is the maximum image sample distance allow when the image
 sample distance is being automatically adjusted

\item  \verb|float = obj.GetMaximumImageSampleDistance ()| -  This is the maximum image sample distance allow when the image
 sample distance is being automatically adjusted

\item  \verb|obj.SetAutoAdjustSampleDistances (int )| -  If AutoAdjustSampleDistances is on, the the ImageSampleDistance
 will be varied to achieve the allocated render time of this 
 prop (controlled by the desired update rate and any culling in
 use). 

\item  \verb|int = obj.GetAutoAdjustSampleDistancesMinValue ()| -  If AutoAdjustSampleDistances is on, the the ImageSampleDistance
 will be varied to achieve the allocated render time of this 
 prop (controlled by the desired update rate and any culling in
 use). 

\item  \verb|int = obj.GetAutoAdjustSampleDistancesMaxValue ()| -  If AutoAdjustSampleDistances is on, the the ImageSampleDistance
 will be varied to achieve the allocated render time of this 
 prop (controlled by the desired update rate and any culling in
 use). 

\item  \verb|int = obj.GetAutoAdjustSampleDistances ()| -  If AutoAdjustSampleDistances is on, the the ImageSampleDistance
 will be varied to achieve the allocated render time of this 
 prop (controlled by the desired update rate and any culling in
 use). 

\item  \verb|obj.AutoAdjustSampleDistancesOn ()| -  If AutoAdjustSampleDistances is on, the the ImageSampleDistance
 will be varied to achieve the allocated render time of this 
 prop (controlled by the desired update rate and any culling in
 use). 

\item  \verb|obj.AutoAdjustSampleDistancesOff ()| -  If AutoAdjustSampleDistances is on, the the ImageSampleDistance
 will be varied to achieve the allocated render time of this 
 prop (controlled by the desired update rate and any culling in
 use). 

\item  \verb|obj.SetIntermixIntersectingGeometry (int )| -  If IntermixIntersectingGeometry is turned on, the zbuffer will be
 captured and used to limit the traversal of the rays.

\item  \verb|int = obj.GetIntermixIntersectingGeometryMinValue ()| -  If IntermixIntersectingGeometry is turned on, the zbuffer will be
 captured and used to limit the traversal of the rays.

\item  \verb|int = obj.GetIntermixIntersectingGeometryMaxValue ()| -  If IntermixIntersectingGeometry is turned on, the zbuffer will be
 captured and used to limit the traversal of the rays.

\item  \verb|int = obj.GetIntermixIntersectingGeometry ()| -  If IntermixIntersectingGeometry is turned on, the zbuffer will be
 captured and used to limit the traversal of the rays.

\item  \verb|obj.IntermixIntersectingGeometryOn ()| -  If IntermixIntersectingGeometry is turned on, the zbuffer will be
 captured and used to limit the traversal of the rays.

\item  \verb|obj.IntermixIntersectingGeometryOff ()| -  If IntermixIntersectingGeometry is turned on, the zbuffer will be
 captured and used to limit the traversal of the rays.

\item  \verb|int = obj.GetMaxPixelListSize ()| -  Maximum size allowed for a pixel list. Default is 32.
 During the rendering, if a list of pixel is full, incremental compositing
 is performed. Even if it is a user setting, it is an advanced parameter.
 You have to understand how the algorithm works to change this value.

\item  \verb|obj.SetMaxPixelListSize (int size)| -  Change the maximum size allowed for a pixel list. It is an advanced
 parameter.
 

\item  \verb|obj.SetRayIntegrator (vtkUnstructuredGridVolumeRayIntegrator ri)| -  Set/Get the helper class for integrating rays.  If set to NULL, a
 default integrator will be assigned.

\item  \verb|vtkUnstructuredGridVolumeRayIntegrator = obj.GetRayIntegrator ()| -  Set/Get the helper class for integrating rays.  If set to NULL, a
 default integrator will be assigned.

\end{itemize}
