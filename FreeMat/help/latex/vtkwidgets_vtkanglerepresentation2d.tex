\section{vtkAngleRepresentation2D}

\subsection{Usage}

 The vtkAngleRepresentation2D is a representation for the
 vtkAngleWidget. This representation consists of two rays and three
 vtkHandleRepresentations to place and manipulate the three points defining
 the angle representation. (Note: the three points are referred to as Point1,
 Center, and Point2, at the two end points (Point1 and Point2) and Center
 (around which the angle is measured). This particular implementation is a 
 2D representation, meaning that it draws in the overlay plane.

To create an instance of class vtkAngleRepresentation2D, simply
invoke its constructor as follows
\begin{verbatim}
  obj = vtkAngleRepresentation2D
\end{verbatim}
\subsection{Methods}

The class vtkAngleRepresentation2D has several methods that can be used.
  They are listed below.
Note that the documentation is translated automatically from the VTK sources,
and may not be completely intelligible.  When in doubt, consult the VTK website.
In the methods listed below, \verb|obj| is an instance of the vtkAngleRepresentation2D class.
\begin{itemize}
\item  \verb|string = obj.GetClassName ()| -  Standard VTK methods.

\item  \verb|int = obj.IsA (string name)| -  Standard VTK methods.

\item  \verb|vtkAngleRepresentation2D = obj.NewInstance ()| -  Standard VTK methods.

\item  \verb|vtkAngleRepresentation2D = obj.SafeDownCast (vtkObject o)| -  Standard VTK methods.

\item  \verb|double = obj.GetAngle ()| -  Satisfy the superclasses API.

\item  \verb|obj.GetPoint1WorldPosition (double pos[3])| -  Methods to Set/Get the coordinates of the two points defining
 this representation. Note that methods are available for both
 display and world coordinates.

\item  \verb|obj.GetCenterWorldPosition (double pos[3])| -  Methods to Set/Get the coordinates of the two points defining
 this representation. Note that methods are available for both
 display and world coordinates.

\item  \verb|obj.GetPoint2WorldPosition (double pos[3])| -  Methods to Set/Get the coordinates of the two points defining
 this representation. Note that methods are available for both
 display and world coordinates.

\item  \verb|obj.SetPoint1DisplayPosition (double pos[3])| -  Methods to Set/Get the coordinates of the two points defining
 this representation. Note that methods are available for both
 display and world coordinates.

\item  \verb|obj.SetCenterDisplayPosition (double pos[3])| -  Methods to Set/Get the coordinates of the two points defining
 this representation. Note that methods are available for both
 display and world coordinates.

\item  \verb|obj.SetPoint2DisplayPosition (double pos[3])| -  Methods to Set/Get the coordinates of the two points defining
 this representation. Note that methods are available for both
 display and world coordinates.

\item  \verb|obj.GetPoint1DisplayPosition (double pos[3])| -  Methods to Set/Get the coordinates of the two points defining
 this representation. Note that methods are available for both
 display and world coordinates.

\item  \verb|obj.GetCenterDisplayPosition (double pos[3])| -  Methods to Set/Get the coordinates of the two points defining
 this representation. Note that methods are available for both
 display and world coordinates.

\item  \verb|obj.GetPoint2DisplayPosition (double pos[3])| -  Methods to Set/Get the coordinates of the two points defining
 this representation. Note that methods are available for both
 display and world coordinates.

\item  \verb|vtkLeaderActor2D = obj.GetRay1 ()| -  Set/Get the three leaders used to create this representation.
 By obtaining these leaders the user can set the appropriate
 properties, etc.

\item  \verb|vtkLeaderActor2D = obj.GetRay2 ()| -  Set/Get the three leaders used to create this representation.
 By obtaining these leaders the user can set the appropriate
 properties, etc.

\item  \verb|vtkLeaderActor2D = obj.GetArc ()| -  Set/Get the three leaders used to create this representation.
 By obtaining these leaders the user can set the appropriate
 properties, etc.

\item  \verb|obj.BuildRepresentation ()| -  Method defined by vtkWidgetRepresentation superclass and
 needed here.

\item  \verb|obj.ReleaseGraphicsResources (vtkWindow w)| -  Methods required by vtkProp superclass.

\item  \verb|int = obj.RenderOverlay (vtkViewport viewport)| -  Methods required by vtkProp superclass.

\end{itemize}
