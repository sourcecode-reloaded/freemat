\section{vtkImageToPolyDataFilter}

\subsection{Usage}

 vtkImageToPolyDataFilter converts raster data (i.e., an image) into
 polygonal data (i.e., quads or n-sided polygons), with each polygon
 assigned a constant color. This is useful for writers that generate vector
 formats (i.e., CGM or PostScript). To use this filter, you specify how to
 quantize the color (or whether to use an image with a lookup table), and
 what style the output should be. The output is always polygons, but the
 choice is n x m quads (where n and m define the input image dimensions)
 ''Pixelize'' option; arbitrary polygons ''Polygonalize'' option; or variable
 number of quads of constant color generated along scan lines ''RunLength''
 option.

 The algorithm quantizes color in order to create coherent regions that the
 polygons can represent with good compression. By default, the input image
 is quantized to 256 colors using a 3-3-2 bits for red-green-blue. However,
 you can also supply a single component image and a lookup table, with the
 single component assumed to be an index into the table.  (Note: a quantized
 image can be generated with the filter vtkImageQuantizeRGBToIndex.) The
 number of colors on output is equal to the number of colors in the input
 lookup table (or 256 if the built in linear ramp is used).

 The output of the filter is polygons with a single color per polygon cell.
 If the output style is set to ''Polygonalize'', the polygons may have an
 large number of points (bounded by something like 2*(n+m)); and the
 polygon may not be convex which may cause rendering problems on some
 systems (use vtkTriangleFilter). Otherwise, each polygon will have four
 vertices. The output also contains scalar data defining RGB color in
 unsigned char form.


To create an instance of class vtkImageToPolyDataFilter, simply
invoke its constructor as follows
\begin{verbatim}
  obj = vtkImageToPolyDataFilter
\end{verbatim}
\subsection{Methods}

The class vtkImageToPolyDataFilter has several methods that can be used.
  They are listed below.
Note that the documentation is translated automatically from the VTK sources,
and may not be completely intelligible.  When in doubt, consult the VTK website.
In the methods listed below, \verb|obj| is an instance of the vtkImageToPolyDataFilter class.
\begin{itemize}
\item  \verb|string = obj.GetClassName ()|

\item  \verb|int = obj.IsA (string name)|

\item  \verb|vtkImageToPolyDataFilter = obj.NewInstance ()|

\item  \verb|vtkImageToPolyDataFilter = obj.SafeDownCast (vtkObject o)|

\item  \verb|obj.SetOutputStyle (int )| -  Specify how to create the output. Pixelize means converting the image
 to quad polygons with a constant color per quad. Polygonalize means
 merging colors together into polygonal regions, and then smoothing
 the regions (if smoothing is turned on). RunLength means creating
 quad polygons that may encompass several pixels on a scan line. The
 default behavior is Polygonalize.

\item  \verb|int = obj.GetOutputStyleMinValue ()| -  Specify how to create the output. Pixelize means converting the image
 to quad polygons with a constant color per quad. Polygonalize means
 merging colors together into polygonal regions, and then smoothing
 the regions (if smoothing is turned on). RunLength means creating
 quad polygons that may encompass several pixels on a scan line. The
 default behavior is Polygonalize.

\item  \verb|int = obj.GetOutputStyleMaxValue ()| -  Specify how to create the output. Pixelize means converting the image
 to quad polygons with a constant color per quad. Polygonalize means
 merging colors together into polygonal regions, and then smoothing
 the regions (if smoothing is turned on). RunLength means creating
 quad polygons that may encompass several pixels on a scan line. The
 default behavior is Polygonalize.

\item  \verb|int = obj.GetOutputStyle ()| -  Specify how to create the output. Pixelize means converting the image
 to quad polygons with a constant color per quad. Polygonalize means
 merging colors together into polygonal regions, and then smoothing
 the regions (if smoothing is turned on). RunLength means creating
 quad polygons that may encompass several pixels on a scan line. The
 default behavior is Polygonalize.

\item  \verb|obj.SetOutputStyleToPixelize ()| -  Specify how to create the output. Pixelize means converting the image
 to quad polygons with a constant color per quad. Polygonalize means
 merging colors together into polygonal regions, and then smoothing
 the regions (if smoothing is turned on). RunLength means creating
 quad polygons that may encompass several pixels on a scan line. The
 default behavior is Polygonalize.

\item  \verb|obj.SetOutputStyleToPolygonalize ()| -  Specify how to create the output. Pixelize means converting the image
 to quad polygons with a constant color per quad. Polygonalize means
 merging colors together into polygonal regions, and then smoothing
 the regions (if smoothing is turned on). RunLength means creating
 quad polygons that may encompass several pixels on a scan line. The
 default behavior is Polygonalize.

\item  \verb|obj.SetOutputStyleToRunLength ()| -  Specify how to create the output. Pixelize means converting the image
 to quad polygons with a constant color per quad. Polygonalize means
 merging colors together into polygonal regions, and then smoothing
 the regions (if smoothing is turned on). RunLength means creating
 quad polygons that may encompass several pixels on a scan line. The
 default behavior is Polygonalize.

\item  \verb|obj.SetColorMode (int )| -  Specify how to quantize color.

\item  \verb|int = obj.GetColorModeMinValue ()| -  Specify how to quantize color.

\item  \verb|int = obj.GetColorModeMaxValue ()| -  Specify how to quantize color.

\item  \verb|int = obj.GetColorMode ()| -  Specify how to quantize color.

\item  \verb|obj.SetColorModeToLUT ()| -  Specify how to quantize color.

\item  \verb|obj.SetColorModeToLinear256 ()| -  Specify how to quantize color.

\item  \verb|obj.SetLookupTable (vtkScalarsToColors )| -  Set/Get the vtkLookupTable to use. The lookup table is used when the
 color mode is set to LUT and a single component scalar is input.

\item  \verb|vtkScalarsToColors = obj.GetLookupTable ()| -  Set/Get the vtkLookupTable to use. The lookup table is used when the
 color mode is set to LUT and a single component scalar is input.

\item  \verb|obj.SetSmoothing (int )| -  If the output style is set to polygonalize, then you can control
 whether to smooth boundaries.

\item  \verb|int = obj.GetSmoothing ()| -  If the output style is set to polygonalize, then you can control
 whether to smooth boundaries.

\item  \verb|obj.SmoothingOn ()| -  If the output style is set to polygonalize, then you can control
 whether to smooth boundaries.

\item  \verb|obj.SmoothingOff ()| -  If the output style is set to polygonalize, then you can control
 whether to smooth boundaries.

\item  \verb|obj.SetNumberOfSmoothingIterations (int )| -  Specify the number of smoothing iterations to smooth polygons. (Only
 in effect if output style is Polygonalize and smoothing is on.)

\item  \verb|int = obj.GetNumberOfSmoothingIterationsMinValue ()| -  Specify the number of smoothing iterations to smooth polygons. (Only
 in effect if output style is Polygonalize and smoothing is on.)

\item  \verb|int = obj.GetNumberOfSmoothingIterationsMaxValue ()| -  Specify the number of smoothing iterations to smooth polygons. (Only
 in effect if output style is Polygonalize and smoothing is on.)

\item  \verb|int = obj.GetNumberOfSmoothingIterations ()| -  Specify the number of smoothing iterations to smooth polygons. (Only
 in effect if output style is Polygonalize and smoothing is on.)

\item  \verb|obj.SetDecimation (int )| -  Turn on/off whether the final polygons should be decimated.
 whether to smooth boundaries.

\item  \verb|int = obj.GetDecimation ()| -  Turn on/off whether the final polygons should be decimated.
 whether to smooth boundaries.

\item  \verb|obj.DecimationOn ()| -  Turn on/off whether the final polygons should be decimated.
 whether to smooth boundaries.

\item  \verb|obj.DecimationOff ()| -  Turn on/off whether the final polygons should be decimated.
 whether to smooth boundaries.

\item  \verb|obj.SetDecimationError (double )| -  Specify the error to use for decimation (if decimation is on).
 The error is an absolute number--the image spacing and
 dimensions are used to create points so the error should be
 consistent with the image size.

\item  \verb|double = obj.GetDecimationErrorMinValue ()| -  Specify the error to use for decimation (if decimation is on).
 The error is an absolute number--the image spacing and
 dimensions are used to create points so the error should be
 consistent with the image size.

\item  \verb|double = obj.GetDecimationErrorMaxValue ()| -  Specify the error to use for decimation (if decimation is on).
 The error is an absolute number--the image spacing and
 dimensions are used to create points so the error should be
 consistent with the image size.

\item  \verb|double = obj.GetDecimationError ()| -  Specify the error to use for decimation (if decimation is on).
 The error is an absolute number--the image spacing and
 dimensions are used to create points so the error should be
 consistent with the image size.

\item  \verb|obj.SetError (int )| -  Specify the error value between two colors where the colors are 
 considered the same. Only use this if the color mode uses the
 default 256 table.

\item  \verb|int = obj.GetErrorMinValue ()| -  Specify the error value between two colors where the colors are 
 considered the same. Only use this if the color mode uses the
 default 256 table.

\item  \verb|int = obj.GetErrorMaxValue ()| -  Specify the error value between two colors where the colors are 
 considered the same. Only use this if the color mode uses the
 default 256 table.

\item  \verb|int = obj.GetError ()| -  Specify the error value between two colors where the colors are 
 considered the same. Only use this if the color mode uses the
 default 256 table.

\item  \verb|obj.SetSubImageSize (int )| -  Specify the size (n by n pixels) of the largest region to 
 polygonalize. When the OutputStyle is set to VTK\_STYLE\_POLYGONALIZE,
 large amounts of memory are used. In order to process large images,
 the image is broken into pieces that are at most Size pixels in
 width and height.

\item  \verb|int = obj.GetSubImageSizeMinValue ()| -  Specify the size (n by n pixels) of the largest region to 
 polygonalize. When the OutputStyle is set to VTK\_STYLE\_POLYGONALIZE,
 large amounts of memory are used. In order to process large images,
 the image is broken into pieces that are at most Size pixels in
 width and height.

\item  \verb|int = obj.GetSubImageSizeMaxValue ()| -  Specify the size (n by n pixels) of the largest region to 
 polygonalize. When the OutputStyle is set to VTK\_STYLE\_POLYGONALIZE,
 large amounts of memory are used. In order to process large images,
 the image is broken into pieces that are at most Size pixels in
 width and height.

\item  \verb|int = obj.GetSubImageSize ()| -  Specify the size (n by n pixels) of the largest region to 
 polygonalize. When the OutputStyle is set to VTK\_STYLE\_POLYGONALIZE,
 large amounts of memory are used. In order to process large images,
 the image is broken into pieces that are at most Size pixels in
 width and height.

\end{itemize}
