\section{vtkFacetReader}

\subsection{Usage}

 vtkFacetReader creates a poly data dataset. It reads ASCII files
 stored in Facet format

 The facet format looks like this:
 FACET FILE ...
 nparts
 Part 1 name
 0
 npoints 0 0
 p1x p1y p1z
 p2x p2y p2z
 ...
 1
 Part 1 name
 ncells npointspercell
 p1c1 p2c1 p3c1 ... pnc1 materialnum partnum
 p1c2 p2c2 p3c2 ... pnc2 materialnum partnum
 ...

To create an instance of class vtkFacetReader, simply
invoke its constructor as follows
\begin{verbatim}
  obj = vtkFacetReader
\end{verbatim}
\subsection{Methods}

The class vtkFacetReader has several methods that can be used.
  They are listed below.
Note that the documentation is translated automatically from the VTK sources,
and may not be completely intelligible.  When in doubt, consult the VTK website.
In the methods listed below, \verb|obj| is an instance of the vtkFacetReader class.
\begin{itemize}
\item  \verb|string = obj.GetClassName ()|

\item  \verb|int = obj.IsA (string name)|

\item  \verb|vtkFacetReader = obj.NewInstance ()|

\item  \verb|vtkFacetReader = obj.SafeDownCast (vtkObject o)|

\item  \verb|obj.SetFileName (string )| -  Specify file name of Facet datafile to read

\item  \verb|string = obj.GetFileName ()| -  Specify file name of Facet datafile to read

\end{itemize}
