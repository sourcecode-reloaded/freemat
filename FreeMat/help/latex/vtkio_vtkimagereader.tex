\section{vtkImageReader}

\subsection{Usage}

 vtkImageReader provides methods needed to read a region from a file.
 It supports both transforms and masks on the input data, but as a result
 is more complicated and slower than its parent class vtkImageReader2.

To create an instance of class vtkImageReader, simply
invoke its constructor as follows
\begin{verbatim}
  obj = vtkImageReader
\end{verbatim}
\subsection{Methods}

The class vtkImageReader has several methods that can be used.
  They are listed below.
Note that the documentation is translated automatically from the VTK sources,
and may not be completely intelligible.  When in doubt, consult the VTK website.
In the methods listed below, \verb|obj| is an instance of the vtkImageReader class.
\begin{itemize}
\item  \verb|string = obj.GetClassName ()|

\item  \verb|int = obj.IsA (string name)|

\item  \verb|vtkImageReader = obj.NewInstance ()|

\item  \verb|vtkImageReader = obj.SafeDownCast (vtkObject o)|

\item  \verb|obj.SetDataVOI (int , int , int , int , int , int )| -  Set/get the data VOI. You can limit the reader to only
 read a subset of the data. 

\item  \verb|obj.SetDataVOI (int  a[6])| -  Set/get the data VOI. You can limit the reader to only
 read a subset of the data. 

\item  \verb|int = obj. GetDataVOI ()| -  Set/get the data VOI. You can limit the reader to only
 read a subset of the data. 

\item  \verb|vtkTypeUInt64 = obj.GetDataMask ()| -  Set/Get the Data mask.  The data mask is a simply integer whose bits are
 treated as a mask to the bits read from disk.  That is, the data mask is
 bitwise-and'ed to the numbers read from disk.  This ivar is stored as 64
 bits, the largest mask you will need.  The mask will be truncated to the
 data size required to be read (using the least significant bits).

\item  \verb|obj.SetDataMask (vtkTypeUInt64 )| -  Set/Get the Data mask.  The data mask is a simply integer whose bits are
 treated as a mask to the bits read from disk.  That is, the data mask is
 bitwise-and'ed to the numbers read from disk.  This ivar is stored as 64
 bits, the largest mask you will need.  The mask will be truncated to the
 data size required to be read (using the least significant bits).

\item  \verb|obj.SetTransform (vtkTransform )| -  Set/Get transformation matrix to transform the data from slice space
 into world space. This matrix must be a permutation matrix. To qualify,
 the sums of the rows must be + or - 1.

\item  \verb|vtkTransform = obj.GetTransform ()| -  Set/Get transformation matrix to transform the data from slice space
 into world space. This matrix must be a permutation matrix. To qualify,
 the sums of the rows must be + or - 1.

\item  \verb|obj.ComputeInverseTransformedExtent (int inExtent[6], int outExtent[6])|

\item  \verb|int = obj.OpenAndSeekFile (int extent[6], int slice)|

\item  \verb|obj.SetScalarArrayName (string )| -  Set/get the scalar array name for this data set.

\item  \verb|string = obj.GetScalarArrayName ()| -  Set/get the scalar array name for this data set.

\end{itemize}
