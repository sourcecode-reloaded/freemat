\section{vtkInitialValueProblemSolver}

\subsection{Usage}

 Given a vtkFunctionSet which returns dF\_i(x\_j, t)/dt given x\_j and
 t, vtkInitialValueProblemSolver computes the value of F\_i at t+deltat.

To create an instance of class vtkInitialValueProblemSolver, simply
invoke its constructor as follows
\begin{verbatim}
  obj = vtkInitialValueProblemSolver
\end{verbatim}
\subsection{Methods}

The class vtkInitialValueProblemSolver has several methods that can be used.
  They are listed below.
Note that the documentation is translated automatically from the VTK sources,
and may not be completely intelligible.  When in doubt, consult the VTK website.
In the methods listed below, \verb|obj| is an instance of the vtkInitialValueProblemSolver class.
\begin{itemize}
\item  \verb|string = obj.GetClassName ()|

\item  \verb|int = obj.IsA (string name)|

\item  \verb|vtkInitialValueProblemSolver = obj.NewInstance ()|

\item  \verb|vtkInitialValueProblemSolver = obj.SafeDownCast (vtkObject o)|

\item  \verb|obj.SetFunctionSet (vtkFunctionSet functionset)| -  Set / get the dataset used for the implicit function evaluation.

\item  \verb|vtkFunctionSet = obj.GetFunctionSet ()| -  Set / get the dataset used for the implicit function evaluation.

\item  \verb|int = obj.IsAdaptive ()|

\end{itemize}
