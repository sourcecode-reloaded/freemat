\section{vtkBandedPolyDataContourFilter}

\subsection{Usage}

 vtkBandedPolyDataContourFilter is a filter that takes as input vtkPolyData
 and produces as output filled contours (also represented as vtkPolyData).
 Filled contours are bands of cells that all have the same cell scalar
 value, and can therefore be colored the same. The method is also referred
 to as filled contour generation.

 To use this filter you must specify one or more contour values.  You can
 either use the method SetValue() to specify each contour value, or use
 GenerateValues() to generate a series of evenly spaced contours.  Each
 contour value divides (or clips) the data into two pieces, values below
 the contour value, and values above it. The scalar values of each
 band correspond to the specified contour value.  Note that if the first and
 last contour values are not the minimum/maximum contour range, then two
 extra contour values are added corresponding to the minimum and maximum
 range values. These extra contour bands can be prevented from being output
 by turning clipping on.


To create an instance of class vtkBandedPolyDataContourFilter, simply
invoke its constructor as follows
\begin{verbatim}
  obj = vtkBandedPolyDataContourFilter
\end{verbatim}
\subsection{Methods}

The class vtkBandedPolyDataContourFilter has several methods that can be used.
  They are listed below.
Note that the documentation is translated automatically from the VTK sources,
and may not be completely intelligible.  When in doubt, consult the VTK website.
In the methods listed below, \verb|obj| is an instance of the vtkBandedPolyDataContourFilter class.
\begin{itemize}
\item  \verb|string = obj.GetClassName ()|

\item  \verb|int = obj.IsA (string name)|

\item  \verb|vtkBandedPolyDataContourFilter = obj.NewInstance ()|

\item  \verb|vtkBandedPolyDataContourFilter = obj.SafeDownCast (vtkObject o)|

\item  \verb|obj.SetValue (int i, double value)| -  Methods to set / get contour values. A single value at a time can be
 set with SetValue(). Multiple contour values can be set with
 GenerateValues(). Note that GenerateValues() generates n values
 inclusive of the start and end range values.

\item  \verb|double = obj.GetValue (int i)| -  Methods to set / get contour values. A single value at a time can be
 set with SetValue(). Multiple contour values can be set with
 GenerateValues(). Note that GenerateValues() generates n values
 inclusive of the start and end range values.

\item  \verb|obj.GetValues (double contourValues)| -  Methods to set / get contour values. A single value at a time can be
 set with SetValue(). Multiple contour values can be set with
 GenerateValues(). Note that GenerateValues() generates n values
 inclusive of the start and end range values.

\item  \verb|obj.SetNumberOfContours (int number)| -  Methods to set / get contour values. A single value at a time can be
 set with SetValue(). Multiple contour values can be set with
 GenerateValues(). Note that GenerateValues() generates n values
 inclusive of the start and end range values.

\item  \verb|int = obj.GetNumberOfContours ()| -  Methods to set / get contour values. A single value at a time can be
 set with SetValue(). Multiple contour values can be set with
 GenerateValues(). Note that GenerateValues() generates n values
 inclusive of the start and end range values.

\item  \verb|obj.GenerateValues (int numContours, double range[2])| -  Methods to set / get contour values. A single value at a time can be
 set with SetValue(). Multiple contour values can be set with
 GenerateValues(). Note that GenerateValues() generates n values
 inclusive of the start and end range values.

\item  \verb|obj.GenerateValues (int numContours, double rangeStart, double rangeEnd)| -  Methods to set / get contour values. A single value at a time can be
 set with SetValue(). Multiple contour values can be set with
 GenerateValues(). Note that GenerateValues() generates n values
 inclusive of the start and end range values.

\item  \verb|obj.SetClipping (int )| -  Indicate whether to clip outside the range specified by the user.
 (The range is contour value[0] to contour value[numContours-1].)
 Clipping means all cells outside of the range specified are not
 sent to the output.

\item  \verb|int = obj.GetClipping ()| -  Indicate whether to clip outside the range specified by the user.
 (The range is contour value[0] to contour value[numContours-1].)
 Clipping means all cells outside of the range specified are not
 sent to the output.

\item  \verb|obj.ClippingOn ()| -  Indicate whether to clip outside the range specified by the user.
 (The range is contour value[0] to contour value[numContours-1].)
 Clipping means all cells outside of the range specified are not
 sent to the output.

\item  \verb|obj.ClippingOff ()| -  Indicate whether to clip outside the range specified by the user.
 (The range is contour value[0] to contour value[numContours-1].)
 Clipping means all cells outside of the range specified are not
 sent to the output.

\item  \verb|obj.SetScalarMode (int )| -  Control whether the cell scalars are output as an integer index or
 a scalar value. If an index, the index refers to the bands produced
 by the clipping range. If a value, then a scalar value which is a 
 value between clip values is used.

\item  \verb|int = obj.GetScalarModeMinValue ()| -  Control whether the cell scalars are output as an integer index or
 a scalar value. If an index, the index refers to the bands produced
 by the clipping range. If a value, then a scalar value which is a 
 value between clip values is used.

\item  \verb|int = obj.GetScalarModeMaxValue ()| -  Control whether the cell scalars are output as an integer index or
 a scalar value. If an index, the index refers to the bands produced
 by the clipping range. If a value, then a scalar value which is a 
 value between clip values is used.

\item  \verb|int = obj.GetScalarMode ()| -  Control whether the cell scalars are output as an integer index or
 a scalar value. If an index, the index refers to the bands produced
 by the clipping range. If a value, then a scalar value which is a 
 value between clip values is used.

\item  \verb|obj.SetScalarModeToIndex ()| -  Control whether the cell scalars are output as an integer index or
 a scalar value. If an index, the index refers to the bands produced
 by the clipping range. If a value, then a scalar value which is a 
 value between clip values is used.

\item  \verb|obj.SetScalarModeToValue ()| -  Turn on/off a flag to control whether contour edges are generated.
 Contour edges are the edges between bands. If enabled, they are 
 generated from polygons/triangle strips and placed into the second
 output (the ContourEdgesOutput).

\item  \verb|obj.SetGenerateContourEdges (int )| -  Turn on/off a flag to control whether contour edges are generated.
 Contour edges are the edges between bands. If enabled, they are 
 generated from polygons/triangle strips and placed into the second
 output (the ContourEdgesOutput).

\item  \verb|int = obj.GetGenerateContourEdges ()| -  Turn on/off a flag to control whether contour edges are generated.
 Contour edges are the edges between bands. If enabled, they are 
 generated from polygons/triangle strips and placed into the second
 output (the ContourEdgesOutput).

\item  \verb|obj.GenerateContourEdgesOn ()| -  Turn on/off a flag to control whether contour edges are generated.
 Contour edges are the edges between bands. If enabled, they are 
 generated from polygons/triangle strips and placed into the second
 output (the ContourEdgesOutput).

\item  \verb|obj.GenerateContourEdgesOff ()| -  Turn on/off a flag to control whether contour edges are generated.
 Contour edges are the edges between bands. If enabled, they are 
 generated from polygons/triangle strips and placed into the second
 output (the ContourEdgesOutput).

\item  \verb|vtkPolyData = obj.GetContourEdgesOutput ()| -  Get the second output which contains the edges dividing the contour 
 bands. This output is empty unless GenerateContourEdges is enabled.

\item  \verb|long = obj.GetMTime ()| -  Overload GetMTime because we delegate to vtkContourValues so its
 modified time must be taken into account.

\end{itemize}
