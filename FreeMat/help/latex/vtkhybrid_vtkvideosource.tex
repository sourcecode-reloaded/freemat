\section{vtkVideoSource}

\subsection{Usage}

 vtkVideoSource is a superclass for video input interfaces for VTK.
 The goal is to provide an interface which is very similar to the
 interface of a VCR, where the 'tape' is an internal frame buffer
 capable of holding a preset number of video frames.  Specialized
 versions of this class record input from various video input sources.
 This base class records input from a noise source.

To create an instance of class vtkVideoSource, simply
invoke its constructor as follows
\begin{verbatim}
  obj = vtkVideoSource
\end{verbatim}
\subsection{Methods}

The class vtkVideoSource has several methods that can be used.
  They are listed below.
Note that the documentation is translated automatically from the VTK sources,
and may not be completely intelligible.  When in doubt, consult the VTK website.
In the methods listed below, \verb|obj| is an instance of the vtkVideoSource class.
\begin{itemize}
\item  \verb|string = obj.GetClassName ()|

\item  \verb|int = obj.IsA (string name)|

\item  \verb|vtkVideoSource = obj.NewInstance ()|

\item  \verb|vtkVideoSource = obj.SafeDownCast (vtkObject o)|

\item  \verb|obj.Record ()| -  Record incoming video at the specified FrameRate.  The recording
 continues indefinitely until Stop() is called. 

\item  \verb|obj.Play ()| -  Play through the 'tape' sequentially at the specified frame rate.
 If you have just finished Recoding, you should call Rewind() first.

\item  \verb|obj.Stop ()| -  Stop recording or playing.

\item  \verb|obj.Rewind ()| -  Rewind to the frame with the earliest timestamp.  Record operations
 will start on the following frame, therefore if you want to re-record
 over this frame you must call Seek(-1) before calling Grab() or Record().

\item  \verb|obj.FastForward ()| -  FastForward to the last frame that was recorded (i.e. to the frame
 that has the most recent timestamp).

\item  \verb|obj.Seek (int n)| -  Seek forwards or backwards by the specified number of frames
 (positive is forward, negative is backward).

\item  \verb|obj.Grab ()| -  Grab a single video frame.

\item  \verb|int = obj.GetRecording ()| -  Are we in record mode? (record mode and play mode are mutually
 exclusive).

\item  \verb|int = obj.GetPlaying ()| -  Are we in play mode? (record mode and play mode are mutually
 exclusive).

\item  \verb|obj.SetFrameSize (int x, int y, int z)| -  Set the full-frame size.  This must be an allowed size for the device,
 the device may either refuse a request for an illegal frame size or
 automatically choose a new frame size.
 The default is usually 320x240x1, but can be device specific.  
 The 'depth' should always be 1 (unless you have a device that
 can handle 3D acquisition).

\item  \verb|obj.SetFrameSize (int dim[3])| -  Set the full-frame size.  This must be an allowed size for the device,
 the device may either refuse a request for an illegal frame size or
 automatically choose a new frame size.
 The default is usually 320x240x1, but can be device specific.  
 The 'depth' should always be 1 (unless you have a device that
 can handle 3D acquisition).

\item  \verb|int = obj. GetFrameSize ()| -  Set the full-frame size.  This must be an allowed size for the device,
 the device may either refuse a request for an illegal frame size or
 automatically choose a new frame size.
 The default is usually 320x240x1, but can be device specific.  
 The 'depth' should always be 1 (unless you have a device that
 can handle 3D acquisition).

\item  \verb|obj.SetFrameRate (float rate)| -  Request a particular frame rate (default 30 frames per second).

\item  \verb|float = obj.GetFrameRate ()| -  Request a particular frame rate (default 30 frames per second).

\item  \verb|obj.SetOutputFormat (int format)| -  Set the output format.  This must be appropriate for device,
 usually only VTK\_LUMINANCE, VTK\_RGB, and VTK\_RGBA are supported.

\item  \verb|obj.SetOutputFormatToLuminance ()| -  Set the output format.  This must be appropriate for device,
 usually only VTK\_LUMINANCE, VTK\_RGB, and VTK\_RGBA are supported.

\item  \verb|obj.SetOutputFormatToRGB ()| -  Set the output format.  This must be appropriate for device,
 usually only VTK\_LUMINANCE, VTK\_RGB, and VTK\_RGBA are supported.

\item  \verb|obj.SetOutputFormatToRGBA ()| -  Set the output format.  This must be appropriate for device,
 usually only VTK\_LUMINANCE, VTK\_RGB, and VTK\_RGBA are supported.

\item  \verb|int = obj.GetOutputFormat ()| -  Set the output format.  This must be appropriate for device,
 usually only VTK\_LUMINANCE, VTK\_RGB, and VTK\_RGBA are supported.

\item  \verb|obj.SetFrameBufferSize (int FrameBufferSize)| -  Set size of the frame buffer, i.e. the number of frames that
 the 'tape' can store.

\item  \verb|int = obj.GetFrameBufferSize ()| -  Set size of the frame buffer, i.e. the number of frames that
 the 'tape' can store.

\item  \verb|obj.SetNumberOfOutputFrames (int )| -  Set the number of frames to copy to the output on each execute.
 The frames will be concatenated along the Z dimension, with the 
 most recent frame first.
 Default: 1

\item  \verb|int = obj.GetNumberOfOutputFrames ()| -  Set the number of frames to copy to the output on each execute.
 The frames will be concatenated along the Z dimension, with the 
 most recent frame first.
 Default: 1

\item  \verb|obj.AutoAdvanceOn ()| -  Set whether to automatically advance the buffer before each grab. 
 Default: on

\item  \verb|obj.AutoAdvanceOff ()| -  Set whether to automatically advance the buffer before each grab. 
 Default: on

\item  \verb|obj.SetAutoAdvance (int )| -  Set whether to automatically advance the buffer before each grab. 
 Default: on

\item  \verb|int = obj.GetAutoAdvance ()| -  Set whether to automatically advance the buffer before each grab. 
 Default: on

\item  \verb|obj.SetClipRegion (int r[6])| -  Set the clip rectangle for the frames.  The video will be clipped 
 before it is copied into the framebuffer.  Changing the ClipRegion
 will destroy the current contents of the framebuffer.
 The default ClipRegion is (0,VTK\_INT\_MAX,0,VTK\_INT\_MAX,0,VTK\_INT\_MAX).

\item  \verb|obj.SetClipRegion (int x0, int x1, int y0, int y1, int z0, int z1)| -  Set the clip rectangle for the frames.  The video will be clipped 
 before it is copied into the framebuffer.  Changing the ClipRegion
 will destroy the current contents of the framebuffer.
 The default ClipRegion is (0,VTK\_INT\_MAX,0,VTK\_INT\_MAX,0,VTK\_INT\_MAX).

\item  \verb|int = obj. GetClipRegion ()| -  Set the clip rectangle for the frames.  The video will be clipped 
 before it is copied into the framebuffer.  Changing the ClipRegion
 will destroy the current contents of the framebuffer.
 The default ClipRegion is (0,VTK\_INT\_MAX,0,VTK\_INT\_MAX,0,VTK\_INT\_MAX).

\item  \verb|obj.SetOutputWholeExtent (int , int , int , int , int , int )| -  Get/Set the WholeExtent of the output.  This can be used to either
 clip or pad the video frame.  This clipping/padding is done when
 the frame is copied to the output, and does not change the contents
 of the framebuffer.  This is useful e.g. for expanding 
 the output size to a power of two for texture mapping.  The
 default is (0,-1,0,-1,0,-1) which causes the entire frame to be
 copied to the output.

\item  \verb|obj.SetOutputWholeExtent (int  a[6])| -  Get/Set the WholeExtent of the output.  This can be used to either
 clip or pad the video frame.  This clipping/padding is done when
 the frame is copied to the output, and does not change the contents
 of the framebuffer.  This is useful e.g. for expanding 
 the output size to a power of two for texture mapping.  The
 default is (0,-1,0,-1,0,-1) which causes the entire frame to be
 copied to the output.

\item  \verb|int = obj. GetOutputWholeExtent ()| -  Get/Set the WholeExtent of the output.  This can be used to either
 clip or pad the video frame.  This clipping/padding is done when
 the frame is copied to the output, and does not change the contents
 of the framebuffer.  This is useful e.g. for expanding 
 the output size to a power of two for texture mapping.  The
 default is (0,-1,0,-1,0,-1) which causes the entire frame to be
 copied to the output.

\item  \verb|obj.SetDataSpacing (double , double , double )| -  Set/Get the pixel spacing. 
 Default: (1.0,1.0,1.0)

\item  \verb|obj.SetDataSpacing (double  a[3])| -  Set/Get the pixel spacing. 
 Default: (1.0,1.0,1.0)

\item  \verb|double = obj. GetDataSpacing ()| -  Set/Get the pixel spacing. 
 Default: (1.0,1.0,1.0)

\item  \verb|obj.SetDataOrigin (double , double , double )| -  Set/Get the coordinates of the lower, left corner of the frame. 
 Default: (0.0,0.0,0.0)

\item  \verb|obj.SetDataOrigin (double  a[3])| -  Set/Get the coordinates of the lower, left corner of the frame. 
 Default: (0.0,0.0,0.0)

\item  \verb|double = obj. GetDataOrigin ()| -  Set/Get the coordinates of the lower, left corner of the frame. 
 Default: (0.0,0.0,0.0)

\item  \verb|obj.SetOpacity (float )| -  For RGBA output only (4 scalar components), set the opacity.  This
 will not modify the existing contents of the framebuffer, only
 subsequently grabbed frames.

\item  \verb|float = obj.GetOpacity ()| -  For RGBA output only (4 scalar components), set the opacity.  This
 will not modify the existing contents of the framebuffer, only
 subsequently grabbed frames.

\item  \verb|int = obj.GetFrameCount ()| -  This value is incremented each time a frame is grabbed.
 reset it to zero (or any other value) at any time.

\item  \verb|obj.SetFrameCount (int )| -  This value is incremented each time a frame is grabbed.
 reset it to zero (or any other value) at any time.

\item  \verb|int = obj.GetFrameIndex ()| -  Get the frame index relative to the 'beginning of the tape'.  This
 value wraps back to zero if it increases past the FrameBufferSize.

\item  \verb|double = obj.GetFrameTimeStamp (int frame)| -  Get a time stamp in seconds (resolution of milliseconds) for
 a video frame.   Time began on Jan 1, 1970.  You can specify
 a number (negative or positive) to specify the position of the
 video frame relative to the current frame.

\item  \verb|double = obj.GetFrameTimeStamp ()| -  Get a time stamp in seconds (resolution of milliseconds) for
 the Output.  Time began on Jan 1, 1970.  This timestamp is only
 valid after the Output has been Updated.

\item  \verb|obj.Initialize ()| -  Initialize the hardware.  This is called automatically
 on the first Update or Grab.

\item  \verb|int = obj.GetInitialized ()| -  Initialize the hardware.  This is called automatically
 on the first Update or Grab.

\item  \verb|obj.ReleaseSystemResources ()| -  Release the video driver.  This method must be called before
 application exit, or else the application might hang during
 exit.  

\item  \verb|obj.InternalGrab ()| -  The internal function which actually does the grab.  You will
 definitely want to override this if you develop a vtkVideoSource
 subclass. 

\item  \verb|obj.SetStartTimeStamp (double t)| -  And internal variable which marks the beginning of a Record session.
 These methods are for internal use only.

\item  \verb|double = obj.GetStartTimeStamp ()| -  And internal variable which marks the beginning of a Record session.
 These methods are for internal use only.

\end{itemize}
