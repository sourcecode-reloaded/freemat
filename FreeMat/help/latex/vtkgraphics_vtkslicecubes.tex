\section{vtkSliceCubes}

\subsection{Usage}

 vtkSliceCubes is a special version of the marching cubes filter. Instead
 of ingesting an entire volume at once it processes only four slices at
 a time. This way, it can generate isosurfaces from huge volumes. Also, the 
 output of this object is written to a marching cubes triangle file. That
 way, output triangles do not need to be held in memory.
 
 To use vtkSliceCubes you must specify an instance of vtkVolumeReader to
 read the data. Set this object up with the proper file prefix, image range,
 data origin, data dimensions, header size, data mask, and swap bytes flag. 
 The vtkSliceCubes object will then take over and read slices as necessary. 
 You also will need to specify the name of an output marching cubes triangle 
 file.


To create an instance of class vtkSliceCubes, simply
invoke its constructor as follows
\begin{verbatim}
  obj = vtkSliceCubes
\end{verbatim}
\subsection{Methods}

The class vtkSliceCubes has several methods that can be used.
  They are listed below.
Note that the documentation is translated automatically from the VTK sources,
and may not be completely intelligible.  When in doubt, consult the VTK website.
In the methods listed below, \verb|obj| is an instance of the vtkSliceCubes class.
\begin{itemize}
\item  \verb|string = obj.GetClassName ()|

\item  \verb|int = obj.IsA (string name)|

\item  \verb|vtkSliceCubes = obj.NewInstance ()|

\item  \verb|vtkSliceCubes = obj.SafeDownCast (vtkObject o)|

\item  \verb|obj.Write ()|

\item  \verb|obj.Update ()|

\item  \verb|obj.SetReader (vtkVolumeReader )| -  Set/get object to read slices.

\item  \verb|vtkVolumeReader = obj.GetReader ()| -  Set/get object to read slices.

\item  \verb|obj.SetFileName (string )| -  Specify file name of marching cubes output file.

\item  \verb|string = obj.GetFileName ()| -  Specify file name of marching cubes output file.

\item  \verb|obj.SetValue (double )| -  Set/get isosurface contour value.

\item  \verb|double = obj.GetValue ()| -  Set/get isosurface contour value.

\item  \verb|obj.SetLimitsFileName (string )| -  Specify file name of marching cubes limits file. The limits file
 speeds up subsequent reading of output triangle file.

\item  \verb|string = obj.GetLimitsFileName ()| -  Specify file name of marching cubes limits file. The limits file
 speeds up subsequent reading of output triangle file.

\end{itemize}
