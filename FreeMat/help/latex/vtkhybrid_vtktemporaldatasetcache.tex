\section{vtkTemporalDataSetCache}

\subsection{Usage}

 vtkTemporalDataSetCache cache time step requests of a temporal dataset,
 when cached data is requested it is returned using a shallow copy.
 .SECTION Thanks
 Ken Martin (Kitware) and John Bidiscombe of 
 CSCS - Swiss National Supercomputing Centre
 for creating and contributing this class.
 For related material, please refer to : 
 John Biddiscombe, Berk Geveci, Ken Martin, Kenneth Moreland, David Thompson,
 ''Time Dependent Processing in a Parallel Pipeline Architecture'', 
 IEEE Visualization 2007. 

To create an instance of class vtkTemporalDataSetCache, simply
invoke its constructor as follows
\begin{verbatim}
  obj = vtkTemporalDataSetCache
\end{verbatim}
\subsection{Methods}

The class vtkTemporalDataSetCache has several methods that can be used.
  They are listed below.
Note that the documentation is translated automatically from the VTK sources,
and may not be completely intelligible.  When in doubt, consult the VTK website.
In the methods listed below, \verb|obj| is an instance of the vtkTemporalDataSetCache class.
\begin{itemize}
\item  \verb|string = obj.GetClassName ()|

\item  \verb|int = obj.IsA (string name)|

\item  \verb|vtkTemporalDataSetCache = obj.NewInstance ()|

\item  \verb|vtkTemporalDataSetCache = obj.SafeDownCast (vtkObject o)|

\item  \verb|obj.SetCacheSize (int size)| -  This is the maximum number of time steps that can be retained in memory.
 it defaults to 10.

\item  \verb|int = obj.GetCacheSize ()| -  This is the maximum number of time steps that can be retained in memory.
 it defaults to 10.

\end{itemize}
