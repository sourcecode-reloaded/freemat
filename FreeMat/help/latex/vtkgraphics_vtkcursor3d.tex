\section{vtkCursor3D}

\subsection{Usage}

 vtkCursor3D is an object that generates a 3D representation of a cursor.
 The cursor consists of a wireframe bounding box, three intersecting 
 axes lines that meet at the cursor focus, and ''shadows'' or projections
 of the axes against the sides of the bounding box. Each of these
 components can be turned on/off.

 This filter generates two output datasets. The first (Output) is just the 
 geometric representation of the cursor. The second (Focus) is a single
 point at the focal point.

To create an instance of class vtkCursor3D, simply
invoke its constructor as follows
\begin{verbatim}
  obj = vtkCursor3D
\end{verbatim}
\subsection{Methods}

The class vtkCursor3D has several methods that can be used.
  They are listed below.
Note that the documentation is translated automatically from the VTK sources,
and may not be completely intelligible.  When in doubt, consult the VTK website.
In the methods listed below, \verb|obj| is an instance of the vtkCursor3D class.
\begin{itemize}
\item  \verb|string = obj.GetClassName ()|

\item  \verb|int = obj.IsA (string name)|

\item  \verb|vtkCursor3D = obj.NewInstance ()|

\item  \verb|vtkCursor3D = obj.SafeDownCast (vtkObject o)|

\item  \verb|obj.SetModelBounds (double xmin, double xmax, double ymin, double ymax, double zmin, double zmax)| -  Set / get the boundary of the 3D cursor.

\item  \verb|obj.SetModelBounds (double bounds[6])| -  Set / get the boundary of the 3D cursor.

\item  \verb|double = obj. GetModelBounds ()| -  Set / get the boundary of the 3D cursor.

\item  \verb|obj.SetFocalPoint (double x[3])| -  Set/Get the position of cursor focus. If translation mode is on,
 then the entire cursor (including bounding box, cursor, and shadows)
 is translated. Otherwise, the focal point will either be clamped to the
 bounding box, or wrapped, if Wrap is on. (Note: this behavior requires
 that the bounding box is set prior to the focal point.)

\item  \verb|obj.SetFocalPoint (double x, double y, double z)| -  Set/Get the position of cursor focus. If translation mode is on,
 then the entire cursor (including bounding box, cursor, and shadows)
 is translated. Otherwise, the focal point will either be clamped to the
 bounding box, or wrapped, if Wrap is on. (Note: this behavior requires
 that the bounding box is set prior to the focal point.)

\item  \verb|double = obj. GetFocalPoint ()| -  Set/Get the position of cursor focus. If translation mode is on,
 then the entire cursor (including bounding box, cursor, and shadows)
 is translated. Otherwise, the focal point will either be clamped to the
 bounding box, or wrapped, if Wrap is on. (Note: this behavior requires
 that the bounding box is set prior to the focal point.)

\item  \verb|obj.SetOutline (int )| -  Turn on/off the wireframe bounding box.

\item  \verb|int = obj.GetOutline ()| -  Turn on/off the wireframe bounding box.

\item  \verb|obj.OutlineOn ()| -  Turn on/off the wireframe bounding box.

\item  \verb|obj.OutlineOff ()| -  Turn on/off the wireframe bounding box.

\item  \verb|obj.SetAxes (int )| -  Turn on/off the wireframe axes.

\item  \verb|int = obj.GetAxes ()| -  Turn on/off the wireframe axes.

\item  \verb|obj.AxesOn ()| -  Turn on/off the wireframe axes.

\item  \verb|obj.AxesOff ()| -  Turn on/off the wireframe axes.

\item  \verb|obj.SetXShadows (int )| -  Turn on/off the wireframe x-shadows.

\item  \verb|int = obj.GetXShadows ()| -  Turn on/off the wireframe x-shadows.

\item  \verb|obj.XShadowsOn ()| -  Turn on/off the wireframe x-shadows.

\item  \verb|obj.XShadowsOff ()| -  Turn on/off the wireframe x-shadows.

\item  \verb|obj.SetYShadows (int )| -  Turn on/off the wireframe y-shadows.

\item  \verb|int = obj.GetYShadows ()| -  Turn on/off the wireframe y-shadows.

\item  \verb|obj.YShadowsOn ()| -  Turn on/off the wireframe y-shadows.

\item  \verb|obj.YShadowsOff ()| -  Turn on/off the wireframe y-shadows.

\item  \verb|obj.SetZShadows (int )| -  Turn on/off the wireframe z-shadows.

\item  \verb|int = obj.GetZShadows ()| -  Turn on/off the wireframe z-shadows.

\item  \verb|obj.ZShadowsOn ()| -  Turn on/off the wireframe z-shadows.

\item  \verb|obj.ZShadowsOff ()| -  Turn on/off the wireframe z-shadows.

\item  \verb|obj.SetTranslationMode (int )| -  Enable/disable the translation mode. If on, changes in cursor position
 cause the entire widget to translate along with the cursor.
 By default, translation mode is off.

\item  \verb|int = obj.GetTranslationMode ()| -  Enable/disable the translation mode. If on, changes in cursor position
 cause the entire widget to translate along with the cursor.
 By default, translation mode is off.

\item  \verb|obj.TranslationModeOn ()| -  Enable/disable the translation mode. If on, changes in cursor position
 cause the entire widget to translate along with the cursor.
 By default, translation mode is off.

\item  \verb|obj.TranslationModeOff ()| -  Enable/disable the translation mode. If on, changes in cursor position
 cause the entire widget to translate along with the cursor.
 By default, translation mode is off.

\item  \verb|obj.SetWrap (int )| -  Turn on/off cursor wrapping. If the cursor focus moves outside the
 specified bounds, the cursor will either be restrained against the
 nearest ''wall'' (Wrap=off), or it will wrap around (Wrap=on).

\item  \verb|int = obj.GetWrap ()| -  Turn on/off cursor wrapping. If the cursor focus moves outside the
 specified bounds, the cursor will either be restrained against the
 nearest ''wall'' (Wrap=off), or it will wrap around (Wrap=on).

\item  \verb|obj.WrapOn ()| -  Turn on/off cursor wrapping. If the cursor focus moves outside the
 specified bounds, the cursor will either be restrained against the
 nearest ''wall'' (Wrap=off), or it will wrap around (Wrap=on).

\item  \verb|obj.WrapOff ()| -  Turn on/off cursor wrapping. If the cursor focus moves outside the
 specified bounds, the cursor will either be restrained against the
 nearest ''wall'' (Wrap=off), or it will wrap around (Wrap=on).

\item  \verb|vtkPolyData = obj.GetFocus ()| -  Get the focus for this filter.

\item  \verb|obj.AllOn ()| -  Turn every part of the 3D cursor on or off.

\item  \verb|obj.AllOff ()| -  Turn every part of the 3D cursor on or off.

\end{itemize}
