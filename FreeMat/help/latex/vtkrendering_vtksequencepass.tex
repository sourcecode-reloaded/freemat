\section{vtkSequencePass}

\subsection{Usage}

 vtkSequencePass executes a list of render passes sequentially.
 This class allows to define a sequence of render passes at run time.
 The other solution to write a sequence of render passes is to write an
 effective subclass of vtkRenderPass.

 As vtkSequencePass is a vtkRenderPass itself, it is possible to have a
 hierarchy of render passes built at runtime.

To create an instance of class vtkSequencePass, simply
invoke its constructor as follows
\begin{verbatim}
  obj = vtkSequencePass
\end{verbatim}
\subsection{Methods}

The class vtkSequencePass has several methods that can be used.
  They are listed below.
Note that the documentation is translated automatically from the VTK sources,
and may not be completely intelligible.  When in doubt, consult the VTK website.
In the methods listed below, \verb|obj| is an instance of the vtkSequencePass class.
\begin{itemize}
\item  \verb|string = obj.GetClassName ()|

\item  \verb|int = obj.IsA (string name)|

\item  \verb|vtkSequencePass = obj.NewInstance ()|

\item  \verb|vtkSequencePass = obj.SafeDownCast (vtkObject o)|

\item  \verb|obj.ReleaseGraphicsResources (vtkWindow w)| -  Release graphics resources and ask components to release their own
 resources.
 

\item  \verb|vtkRenderPassCollection = obj.GetPasses ()| -  The ordered list of render passes to execute sequentially.
 If the pointer is NULL or the list is empty, it is silently ignored.
 There is no warning.
 Initial value is a NULL pointer.

\item  \verb|obj.SetPasses (vtkRenderPassCollection passes)| -  The ordered list of render passes to execute sequentially.
 If the pointer is NULL or the list is empty, it is silently ignored.
 There is no warning.
 Initial value is a NULL pointer.

\end{itemize}
