\section{vtkQuadraticLinearQuad}

\subsection{Usage}

 vtkQuadraticQuad is a concrete implementation of vtkNonLinearCell to
 represent a two-dimensional, 6-node isoparametric quadratic-linear quadrilateral
 element. The interpolation is the standard finite element, quadratic-linear
 isoparametric shape function. The cell includes a mid-edge node for two
 of the four edges. The ordering of the six points defining
 the cell are point ids (0-3,4-5) where ids 0-3 define the four corner
 vertices of the quad; ids 4-7 define the midedge nodes (0,1) and (2,3) .


To create an instance of class vtkQuadraticLinearQuad, simply
invoke its constructor as follows
\begin{verbatim}
  obj = vtkQuadraticLinearQuad
\end{verbatim}
\subsection{Methods}

The class vtkQuadraticLinearQuad has several methods that can be used.
  They are listed below.
Note that the documentation is translated automatically from the VTK sources,
and may not be completely intelligible.  When in doubt, consult the VTK website.
In the methods listed below, \verb|obj| is an instance of the vtkQuadraticLinearQuad class.
\begin{itemize}
\item  \verb|string = obj.GetClassName ()|

\item  \verb|int = obj.IsA (string name)|

\item  \verb|vtkQuadraticLinearQuad = obj.NewInstance ()|

\item  \verb|vtkQuadraticLinearQuad = obj.SafeDownCast (vtkObject o)|

\item  \verb|int = obj.GetCellType ()| -  Implement the vtkCell API. See the vtkCell API for descriptions
 of these methods.

\item  \verb|int = obj.GetCellDimension ()| -  Implement the vtkCell API. See the vtkCell API for descriptions
 of these methods.

\item  \verb|int = obj.GetNumberOfEdges ()| -  Implement the vtkCell API. See the vtkCell API for descriptions
 of these methods.

\item  \verb|int = obj.GetNumberOfFaces ()| -  Implement the vtkCell API. See the vtkCell API for descriptions
 of these methods.

\item  \verb|vtkCell = obj.GetEdge (int )| -  Implement the vtkCell API. See the vtkCell API for descriptions
 of these methods.

\item  \verb|vtkCell = obj.GetFace (int )|

\item  \verb|int = obj.CellBoundary (int subId, double pcoords[3], vtkIdList pts)|

\item  \verb|obj.Contour (double value, vtkDataArray cellScalars, vtkIncrementalPointLocator locator, vtkCellArray verts, vtkCellArray lines, vtkCellArray polys, vtkPointData inPd, vtkPointData outPd, vtkCellData inCd, vtkIdType cellId, vtkCellData outCd)|

\item  \verb|int = obj.Triangulate (int index, vtkIdList ptIds, vtkPoints pts)|

\item  \verb|obj.Derivatives (int subId, double pcoords[3], double values, int dim, double derivs)|

\item  \verb|obj.Clip (double value, vtkDataArray cellScalars, vtkIncrementalPointLocator locator, vtkCellArray polys, vtkPointData inPd, vtkPointData outPd, vtkCellData inCd, vtkIdType cellId, vtkCellData outCd, int insideOut)| -  Clip this quadratic linear quad using scalar value provided. Like
 contouring, except that it cuts the quad to produce linear triangles.

\item  \verb|int = obj.GetParametricCenter (double pcoords[3])| -  Return the center of the pyramid in parametric coordinates.

\item  \verb|obj.InterpolateFunctions (double pcoords[3], double weights[6])| -  Compute the interpolation functions/derivatives
 (aka shape functions/derivatives)

\item  \verb|obj.InterpolateDerivs (double pcoords[3], double derivs[12])| -  Return the ids of the vertices defining edge (`edgeId`).
 Ids are related to the cell, not to the dataset.

\end{itemize}
