\section{NORM Norm Calculation}

\subsection{Usage}

Calculates the norm of a matrix.  There are two ways to
use the \verb|norm| function.  The general syntax is
\begin{verbatim}
   y = norm(A,p)
\end{verbatim}
where \verb|A| is the matrix to analyze, and \verb|p| is the
type norm to compute.  The following choices of \verb|p|
are supported
\begin{itemize}
\item  \verb|p = 1| returns the 1-norm, or the max column sum of A

\item  \verb|p = 2| returns the 2-norm (largest singular value of A)

\item  \verb|p = inf| returns the infinity norm, or the max row sum of A

\item  \verb|p = 'fro'| returns the Frobenius-norm (vector Euclidean norm, or RMS value)

\end{itemize}
For a vector, the regular norm calculations are performed:
\begin{itemize}
\item  \verb|1 <= p < inf| returns \verb|sum(abs(A).^p)^(1/p)|

\item  \verb|p| unspecified returns \verb|norm(A,2)|

\item  \verb|p = inf| returns max(abs(A))

\item  \verb|p = -inf| returns min(abs(A))

\end{itemize}
\subsection{Examples}

Here are the various norms calculated for a sample matrix
\begin{verbatim}
--> A = float(rand(3,4))

A = 
    0.0063    0.2224    0.7574    0.9848 
    0.7319    0.1965    0.7191    0.7010 
    0.8319    0.6392    0.8905    0.9280 

--> norm(A,1)

ans = 
    2.6138 

--> norm(A,2)

ans = 
    2.3403 

--> norm(A,inf)

ans = 
    3.2896 

--> norm(A,'fro')

ans = 
    2.4353 
\end{verbatim}
Next, we calculate some vector norms.
\begin{verbatim}
--> A = float(rand(4,1))

A = 
    0.5011 
    0.3269 
    0.8192 
    0.7321 

--> norm(A,1)

ans = 
    2.3792 

--> norm(A,2)

ans = 
    1.2510 

--> norm(A,7)

ans = 
    0.8671 

--> norm(A,inf)

ans = 
    0.8192 

--> norm(A,-inf)

ans = 
    0.3269 
\end{verbatim}
