\section{IPERMUTE Array Inverse Permutation Function}

\subsection{Usage}

The \verb|ipermute| function rearranges the contents of an array according
to the inverse of the specified permutation vector.  The syntax for 
its use is
\begin{verbatim}
   y = ipermute(x,p)
\end{verbatim}
where \verb|p| is a permutation vector - i.e., a vector containing the 
integers \verb|1...ndims(x)| each occuring exactly once.  The resulting
array \verb|y| contains the same data as the array \verb|x|, but ordered
according to the inverse of the given permutation.  This function and
the \verb|permute| function are inverses of each other.
\subsection{Example}

First we create a large multi-dimensional array, then permute it
 and then inverse permute it, to retrieve the original array:
\begin{verbatim}
--> A = randn(13,5,7,2);
--> size(A)

ans = 
 13  5  7  2 

--> B = permute(A,[3,4,2,1]);
--> size(B)

ans = 
  7  2  5 13 

--> C = ipermute(B,[3,4,2,1]);
--> size(C)

ans = 
 13  5  7  2 

--> any(A~=C)

ans = 

(:,:,1,1) = 
 0 0 0 0 0 

(:,:,2,1) = 
 0 0 0 0 0 

(:,:,3,1) = 
 0 0 0 0 0 

(:,:,4,1) = 
 0 0 0 0 0 

(:,:,5,1) = 
 0 0 0 0 0 

(:,:,6,1) = 
 0 0 0 0 0 

(:,:,7,1) = 
 0 0 0 0 0 

(:,:,1,2) = 
 0 0 0 0 0 

(:,:,2,2) = 
 0 0 0 0 0 

(:,:,3,2) = 
 0 0 0 0 0 

(:,:,4,2) = 
 0 0 0 0 0 

(:,:,5,2) = 
 0 0 0 0 0 

(:,:,6,2) = 
 0 0 0 0 0 

(:,:,7,2) = 
 0 0 0 0 0 
\end{verbatim}
