\section{vtkTreeRingView}

\subsection{Usage}

 Accepts a graph and a hierarchy - currently
 a tree - and provides a hierarchy-aware display.  Currently, this means
 displaying the hierarchy using a tree ring layout, then rendering the graph
 vertices as leaves of the tree with curved graph edges between leaves.

 .SEE ALSO
 vtkGraphLayoutView

 .SECTION Thanks
 Thanks to Jason Shepherd for implementing this class

To create an instance of class vtkTreeRingView, simply
invoke its constructor as follows
\begin{verbatim}
  obj = vtkTreeRingView
\end{verbatim}
\subsection{Methods}

The class vtkTreeRingView has several methods that can be used.
  They are listed below.
Note that the documentation is translated automatically from the VTK sources,
and may not be completely intelligible.  When in doubt, consult the VTK website.
In the methods listed below, \verb|obj| is an instance of the vtkTreeRingView class.
\begin{itemize}
\item  \verb|string = obj.GetClassName ()|

\item  \verb|int = obj.IsA (string name)|

\item  \verb|vtkTreeRingView = obj.NewInstance ()|

\item  \verb|vtkTreeRingView = obj.SafeDownCast (vtkObject o)|

\item  \verb|obj.SetRootAngles (double start, double end)| -  Set the root angles for laying out the hierarchy.

\item  \verb|obj.SetRootAtCenter (bool value)| -  Sets whether the root is at the center or around the outside.

\item  \verb|bool = obj.GetRootAtCenter ()| -  Sets whether the root is at the center or around the outside.

\item  \verb|obj.RootAtCenterOn ()| -  Sets whether the root is at the center or around the outside.

\item  \verb|obj.RootAtCenterOff ()| -  Sets whether the root is at the center or around the outside.

\item  \verb|obj.SetLayerThickness (double thickness)| -  Set the thickness of each layer.

\item  \verb|double = obj.GetLayerThickness ()| -  Set the thickness of each layer.

\item  \verb|obj.SetInteriorRadius (double thickness)| -  Set the interior radius of the tree
 (i.e. the size of the ''hole'' in the center).

\item  \verb|double = obj.GetInteriorRadius ()| -  Set the interior radius of the tree
 (i.e. the size of the ''hole'' in the center).

\item  \verb|obj.SetInteriorLogSpacingValue (double thickness)| -  Set the log spacing factor for the invisible interior tree
 used for routing edges of the overlaid graph.

\item  \verb|double = obj.GetInteriorLogSpacingValue ()| -  Set the log spacing factor for the invisible interior tree
 used for routing edges of the overlaid graph.

\end{itemize}
