\section{vtkParallelCoordinatesActor}

\subsection{Usage}

 vtkParallelCoordinatesActor generates a parallel coordinates plot from an
 input field (i.e., vtkDataObject). Parallel coordinates represent
 N-dimensional data by using a set of N parallel axes (not orthogonal like
 the usual x-y-z Cartesian axes). Each N-dimensional point is plotted as a
 polyline, were each of the N components of the point lie on one of the
 N axes, and the components are connected by straight lines.

 To use this class, you must specify an input data object. You'll probably
 also want to specify the position of the plot be setting the Position and
 Position2 instance variables, which define a rectangle in which the plot
 lies. Another important parameter is the IndependentVariables ivar, which
 tells the instance how to interpret the field data (independent variables
 as the rows or columns of the field). There are also many other instance
 variables that control the look of the plot includes its title, 
 attributes, number of ticks on the axes, etc.

 Set the text property/attributes of the title and the labels through the 
 vtkTextProperty objects associated to this actor.

To create an instance of class vtkParallelCoordinatesActor, simply
invoke its constructor as follows
\begin{verbatim}
  obj = vtkParallelCoordinatesActor
\end{verbatim}
\subsection{Methods}

The class vtkParallelCoordinatesActor has several methods that can be used.
  They are listed below.
Note that the documentation is translated automatically from the VTK sources,
and may not be completely intelligible.  When in doubt, consult the VTK website.
In the methods listed below, \verb|obj| is an instance of the vtkParallelCoordinatesActor class.
\begin{itemize}
\item  \verb|string = obj.GetClassName ()|

\item  \verb|int = obj.IsA (string name)|

\item  \verb|vtkParallelCoordinatesActor = obj.NewInstance ()|

\item  \verb|vtkParallelCoordinatesActor = obj.SafeDownCast (vtkObject o)|

\item  \verb|obj.SetIndependentVariables (int )| -  Specify whether to use the rows or columns as independent variables.
 If columns, then each row represents a separate point. If rows, then 
 each column represents a separate point.

\item  \verb|int = obj.GetIndependentVariablesMinValue ()| -  Specify whether to use the rows or columns as independent variables.
 If columns, then each row represents a separate point. If rows, then 
 each column represents a separate point.

\item  \verb|int = obj.GetIndependentVariablesMaxValue ()| -  Specify whether to use the rows or columns as independent variables.
 If columns, then each row represents a separate point. If rows, then 
 each column represents a separate point.

\item  \verb|int = obj.GetIndependentVariables ()| -  Specify whether to use the rows or columns as independent variables.
 If columns, then each row represents a separate point. If rows, then 
 each column represents a separate point.

\item  \verb|obj.SetIndependentVariablesToColumns ()| -  Specify whether to use the rows or columns as independent variables.
 If columns, then each row represents a separate point. If rows, then 
 each column represents a separate point.

\item  \verb|obj.SetIndependentVariablesToRows ()| -  Specify whether to use the rows or columns as independent variables.
 If columns, then each row represents a separate point. If rows, then 
 each column represents a separate point.

\item  \verb|obj.SetTitle (string )| -  Set/Get the title of the parallel coordinates plot.

\item  \verb|string = obj.GetTitle ()| -  Set/Get the title of the parallel coordinates plot.

\item  \verb|obj.SetNumberOfLabels (int )| -  Set/Get the number of annotation labels to show along each axis.
 This values is a suggestion: the number of labels may vary depending
 on the particulars of the data.

\item  \verb|int = obj.GetNumberOfLabelsMinValue ()| -  Set/Get the number of annotation labels to show along each axis.
 This values is a suggestion: the number of labels may vary depending
 on the particulars of the data.

\item  \verb|int = obj.GetNumberOfLabelsMaxValue ()| -  Set/Get the number of annotation labels to show along each axis.
 This values is a suggestion: the number of labels may vary depending
 on the particulars of the data.

\item  \verb|int = obj.GetNumberOfLabels ()| -  Set/Get the number of annotation labels to show along each axis.
 This values is a suggestion: the number of labels may vary depending
 on the particulars of the data.

\item  \verb|obj.SetLabelFormat (string )| -  Set/Get the format with which to print the labels on the axes.

\item  \verb|string = obj.GetLabelFormat ()| -  Set/Get the format with which to print the labels on the axes.

\item  \verb|obj.SetTitleTextProperty (vtkTextProperty p)| -  Set/Get the title text property.

\item  \verb|vtkTextProperty = obj.GetTitleTextProperty ()| -  Set/Get the title text property.

\item  \verb|obj.SetLabelTextProperty (vtkTextProperty p)| -  Set/Get the labels text property.

\item  \verb|vtkTextProperty = obj.GetLabelTextProperty ()| -  Set/Get the labels text property.

\item  \verb|int = obj.RenderOpaqueGeometry (vtkViewport )| -  Draw the parallel coordinates plot.

\item  \verb|int = obj.RenderOverlay (vtkViewport )| -  Draw the parallel coordinates plot.

\item  \verb|int = obj.RenderTranslucentPolygonalGeometry (vtkViewport )| -  Does this prop have some translucent polygonal geometry?

\item  \verb|int = obj.HasTranslucentPolygonalGeometry ()| -  Does this prop have some translucent polygonal geometry?

\item  \verb|obj.SetInput (vtkDataObject )| -  Set the input to the parallel coordinates actor.

\item  \verb|vtkDataObject = obj.GetInput ()| -  Remove a dataset from the list of data to append.

\item  \verb|obj.ReleaseGraphicsResources (vtkWindow )| -  Release any graphics resources that are being consumed by this actor.
 The parameter window could be used to determine which graphic
 resources to release.

\end{itemize}
