\section{vtkXMLParser}

\subsection{Usage}

 vtkXMLParser reads a stream and parses XML element tags and corresponding
 attributes.  Each element begin tag and its attributes are sent to
 the StartElement method.  Each element end tag is sent to the
 EndElement method.  Subclasses should replace these methods to actually
 use the tags.
 .SECTION ToDo: Add commands for parsing in Tcl.

To create an instance of class vtkXMLParser, simply
invoke its constructor as follows
\begin{verbatim}
  obj = vtkXMLParser
\end{verbatim}
\subsection{Methods}

The class vtkXMLParser has several methods that can be used.
  They are listed below.
Note that the documentation is translated automatically from the VTK sources,
and may not be completely intelligible.  When in doubt, consult the VTK website.
In the methods listed below, \verb|obj| is an instance of the vtkXMLParser class.
\begin{itemize}
\item  \verb|string = obj.GetClassName ()|

\item  \verb|int = obj.IsA (string name)|

\item  \verb|vtkXMLParser = obj.NewInstance ()|

\item  \verb|vtkXMLParser = obj.SafeDownCast (vtkObject o)|

\item  \verb|long = obj.TellG ()| -  Used by subclasses and their supporting classes.  These methods
 wrap around the tellg and seekg methods of the input stream to
 work-around stream bugs on various platforms.

\item  \verb|obj.SeekG (long position)| -  Used by subclasses and their supporting classes.  These methods
 wrap around the tellg and seekg methods of the input stream to
 work-around stream bugs on various platforms.

\item  \verb|int = obj.Parse ()| -  Parse the XML input.

\item  \verb|int = obj.Parse (string inputString)| -  Parse the XML message. If length is specified, parse only the
 first ''length'' characters

\item  \verb|int = obj.Parse (string inputString, int length)| -  Parse the XML message. If length is specified, parse only the
 first ''length'' characters

\item  \verb|int = obj.InitializeParser ()| -  When parsing fragments of XML or streaming XML, use the following
 three methods.  InitializeParser method initialize parser but
 does not perform any actual parsing.  ParseChunk parses framgent
 of XML. This has to match to what was already
 parsed. CleanupParser finishes parsing. If there were errors,
 CleanupParser will report them.

\item  \verb|int = obj.ParseChunk (string inputString, int length)| -  When parsing fragments of XML or streaming XML, use the following
 three methods.  InitializeParser method initialize parser but
 does not perform any actual parsing.  ParseChunk parses framgent
 of XML. This has to match to what was already
 parsed. CleanupParser finishes parsing. If there were errors,
 CleanupParser will report them.

\item  \verb|int = obj.CleanupParser ()| -  When parsing fragments of XML or streaming XML, use the following
 three methods.  InitializeParser method initialize parser but
 does not perform any actual parsing.  ParseChunk parses framgent
 of XML. This has to match to what was already
 parsed. CleanupParser finishes parsing. If there were errors,
 CleanupParser will report them.

\item  \verb|obj.SetFileName (string )| -  Set and get file name.

\item  \verb|string = obj.GetFileName ()| -  Set and get file name.

\item  \verb|obj.SetIgnoreCharacterData (int )| -  If this is off (the default), CharacterDataHandler will be called to 
 process text within XML Elements. If this is on, the text will be 
 ignored.

\item  \verb|int = obj.GetIgnoreCharacterData ()| -  If this is off (the default), CharacterDataHandler will be called to 
 process text within XML Elements. If this is on, the text will be 
 ignored.

\item  \verb|obj.SetEncoding (string )| -  Set and get the encoding the parser should expect (NULL defaults to
 Expat's own default encoder, i.e UTF-8).
 This should be set before parsing (i.e. a call to Parse()) or
 even initializing the parser (i.e. a call to InitializeParser())

\item  \verb|string = obj.GetEncoding ()| -  Set and get the encoding the parser should expect (NULL defaults to
 Expat's own default encoder, i.e UTF-8).
 This should be set before parsing (i.e. a call to Parse()) or
 even initializing the parser (i.e. a call to InitializeParser())

\end{itemize}
