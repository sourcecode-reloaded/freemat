\section{vtkTecplotReader}

\subsection{Usage}

  vtkTecplotReader parses an ASCII Tecplot file to get a vtkMultiBlockDataSet
  object made up of several vtkDataSet objects, of which each is of type
  either vtkStructuredGrid or vtkUnstructuredGrid. Each vtkDataSet object 
  maintains the geometry, topology, and some associated attributes describing
  physical properties.

  Tecplot treats 3D coordinates (only one or two coordinates might be
  explicitly specified in a file) as varaibles too, whose names (e.g.,
  'X' / 'x' / 'I', 'Y' / 'y' / 'J', 'Z' / 'z' / 'K') are provided in the
  variables list (the 'VARIABLES' section). These names are then followed
  in the list by those of other traditional variables or attributes (node-
  based and / or cell-based data with the mode specified via token 'VAR
  LOCATION', to be extracted to create vtkPointData and / or vtkCellData). 
  Each zone described afterwards (in the 'ZONE's section) provides the 
  specific values of the aforementioned variables (including 3D coordinates), 
  in the same order as indicated by the variable-names list, through either
  POINT-packing (i.e., tuple-based storage) or BLOCK-packing (component-based
  storage). In particular, the first / description line of  each zone tells 
  the type of all the  constituent cells as the connectivity / topology 
  information. In other words, the entire dataset is made up of multiple zones
  (blocks), of which each maintains a set of cells of the same type ('BRICK', 
  'TRIANGLE', 'QUADRILATERAL', 'TETRAHEDRON', and 'POINT' in Tecplot terms). 
  In addition, the description line of each zone specifies the zone name, 
  dimensionality information (size of each dimension for a structured zone), 
  number of nodes, and number of cells. Information about the file format is 
  available at http://download.tecplot.com/360/dataformat.pdf.
  

To create an instance of class vtkTecplotReader, simply
invoke its constructor as follows
\begin{verbatim}
  obj = vtkTecplotReader
\end{verbatim}
\subsection{Methods}

The class vtkTecplotReader has several methods that can be used.
  They are listed below.
Note that the documentation is translated automatically from the VTK sources,
and may not be completely intelligible.  When in doubt, consult the VTK website.
In the methods listed below, \verb|obj| is an instance of the vtkTecplotReader class.
\begin{itemize}
\item  \verb|string = obj.GetClassName ()|

\item  \verb|int = obj.IsA (string name)|

\item  \verb|vtkTecplotReader = obj.NewInstance ()|

\item  \verb|vtkTecplotReader = obj.SafeDownCast (vtkObject o)|

\item  \verb|int = obj.GetNumberOfVariables ()| -  Get the number of all variables (including 3D coordinates).

\item  \verb|obj.SetFileName (string fileName)| -  Specify a Tecplot ASCII file for data loading.

\item  \verb|string = obj.GetDataTitle ()| -  Get the Tecplot data title.

\item  \verb|int = obj.GetNumberOfBlocks ()| -  Get the number of blocks (i.e., zones in Tecplot terms).

\item  \verb|string = obj.GetBlockName (int blockIdx)| -  Get the name of a block specified by a zero-based index. NULL is returned
 for an invalid block index.

\item  \verb|int = obj.GetNumberOfDataAttributes ()| -  Get the number of standard data attributes (node-based and cell-based),
 excluding 3D coordinates.

\item  \verb|string = obj.GetDataAttributeName (int attrIndx)| -  Get the name of a zero-based data attribute (not 3D coordinates). NULL is
 returned for an invalid attribute index.

\item  \verb|int = obj.IsDataAttributeCellBased (string attrName)| -  Get the type (0 for node-based and 1 for cell-based) of a specified data 
 attribute (not 3D coordinates). -1 is returned for an invalid attribute 
 name.

\item  \verb|int = obj.IsDataAttributeCellBased (int attrIndx)| -  Get the type (0 for node-based and 1 for cell-based) of a specified data 
 attribute (not 3D coordinates). -1 is returned for an invalid attribute
 index.

\item  \verb|int = obj.GetNumberOfDataArrays ()| -  Get the number of all data attributes (point data and cell data).

\item  \verb|string = obj.GetDataArrayName (int arrayIdx)| -  Get the name of a data array specified by the zero-based index (arrayIdx).

\item  \verb|int = obj.GetDataArrayStatus (string arayName)| -  Get the status of a specific data array (0: un-selected; 1: selected).

\item  \verb|obj.SetDataArrayStatus (string arayName, int bChecked)| -  Set the status of a specific data array (0: de-select; 1: select) specified
 by the name.

\end{itemize}
