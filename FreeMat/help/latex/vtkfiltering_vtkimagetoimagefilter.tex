\section{vtkImageToImageFilter}

\subsection{Usage}

 vtkImageToImageFilter is a filter superclass that hides much of the
 pipeline  complexity. It handles breaking the pipeline execution
 into smaller extents so that the vtkImageData limits are observed. It
 also provides support for multithreading. If you don't need any of this
 functionality, consider using vtkSimpleImageToImageFilter instead.
 .SECTION Warning
 This used to be the parent class for most imaging filter in VTK4.x, now
 this role has been replaced by vtkImageAlgorithm. You should consider
 using vtkImageAlgorithm instead, when writing filter for VTK5 and above.
 This class was kept to ensure full backward compatibility.
 .SECTION See also
 vtkSimpleImageToImageFilter vtkImageSpatialFilter vtkImageAlgorithm

To create an instance of class vtkImageToImageFilter, simply
invoke its constructor as follows
\begin{verbatim}
  obj = vtkImageToImageFilter
\end{verbatim}
\subsection{Methods}

The class vtkImageToImageFilter has several methods that can be used.
  They are listed below.
Note that the documentation is translated automatically from the VTK sources,
and may not be completely intelligible.  When in doubt, consult the VTK website.
In the methods listed below, \verb|obj| is an instance of the vtkImageToImageFilter class.
\begin{itemize}
\item  \verb|string = obj.GetClassName ()|

\item  \verb|int = obj.IsA (string name)|

\item  \verb|vtkImageToImageFilter = obj.NewInstance ()|

\item  \verb|vtkImageToImageFilter = obj.SafeDownCast (vtkObject o)|

\item  \verb|obj.SetInput (vtkImageData input)| -  Set the Input of a filter. 

\item  \verb|vtkImageData = obj.GetInput ()| -  Set the Input of a filter. 

\item  \verb|obj.SetBypass (int )| -  Obsolete feature - do not use.

\item  \verb|obj.BypassOn ()| -  Obsolete feature - do not use.

\item  \verb|obj.BypassOff ()| -  Obsolete feature - do not use.

\item  \verb|int = obj.GetBypass ()| -  Obsolete feature - do not use.

\item  \verb|obj.ThreadedExecute (vtkImageData inData, vtkImageData outData, int extent[6], int threadId)| -  If the subclass does not define an Execute method, then the task
 will be broken up, multiple threads will be spawned, and each thread
 will call this method. It is public so that the thread functions
 can call this method.

\item  \verb|obj.SetNumberOfThreads (int )| -  Get/Set the number of threads to create when rendering

\item  \verb|int = obj.GetNumberOfThreadsMinValue ()| -  Get/Set the number of threads to create when rendering

\item  \verb|int = obj.GetNumberOfThreadsMaxValue ()| -  Get/Set the number of threads to create when rendering

\item  \verb|int = obj.GetNumberOfThreads ()| -  Get/Set the number of threads to create when rendering

\item  \verb|obj.SetInputMemoryLimit (int )|

\item  \verb|long = obj.GetInputMemoryLimit ()|

\item  \verb|int = obj.SplitExtent (int splitExt[6], int startExt[6], int num, int total)| -  Putting this here until I merge graphics and imaging streaming.

\end{itemize}
