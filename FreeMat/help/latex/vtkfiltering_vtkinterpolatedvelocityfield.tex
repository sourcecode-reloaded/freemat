\section{vtkInterpolatedVelocityField}

\subsection{Usage}

  vtkInterpolatedVelocityField acts as a continuous velocity field via
  cell interpolation on a vtkDataSet, NumberOfIndependentVariables = 4
  (x,y,z,t) and NumberOfFunctions = 3 (u,v,w). As a concrete sub-class 
  of vtkAbstractInterpolatedVelocityField, this class adopts two levels 
  of cell caching for faster though less robust cell location than its 
  sibling class vtkCellLocatorInterpolatedVelocityField. Level \#0 begins 
  with intra-cell caching. Specifically, if the previous cell is valid 
  and the nex point is still within it, ( vtkCell::EvaluatePosition() 
  returns 1, coupled with the new parametric coordinates and weights ), 
  the function values are interpolated and vtkCell::EvaluatePosition() 
  is invoked only. If it fails, level \#1 follows by inter-cell location 
  of the target cell (that contains the next point). By inter-cell, the 
  previous cell gives an important clue / guess or serves as an immediate
  neighbor to aid in the location of the target cell (as is typically the 
  case with integrating a streamline across cells) by means of vtkDataSet::
  FindCell(). If this still fails, a global cell search is invoked via 
  vtkDataSet::FindCell(). 
  
  Regardless of inter-cell or global search, vtkPointLocator is employed 
  as a crucial tool underlying the cell locator. The use of vtkPointLocator
  casues vtkInterpolatedVelocityField to return false target cells for 
  datasets defined on complex grids.


To create an instance of class vtkInterpolatedVelocityField, simply
invoke its constructor as follows
\begin{verbatim}
  obj = vtkInterpolatedVelocityField
\end{verbatim}
\subsection{Methods}

The class vtkInterpolatedVelocityField has several methods that can be used.
  They are listed below.
Note that the documentation is translated automatically from the VTK sources,
and may not be completely intelligible.  When in doubt, consult the VTK website.
In the methods listed below, \verb|obj| is an instance of the vtkInterpolatedVelocityField class.
\begin{itemize}
\item  \verb|string = obj.GetClassName ()|

\item  \verb|int = obj.IsA (string name)|

\item  \verb|vtkInterpolatedVelocityField = obj.NewInstance ()|

\item  \verb|vtkInterpolatedVelocityField = obj.SafeDownCast (vtkObject o)|

\item  \verb|obj.AddDataSet (vtkDataSet dataset)| -  Add a dataset used for the implicit function evaluation. If more than
 one dataset is added, the evaluation point is searched in all until a 
 match is found. THIS FUNCTION DOES NOT CHANGE THE REFERENCE COUNT OF 
 DATASET FOR THREAD SAFETY REASONS.

\item  \verb|int = obj.FunctionValues (double x, double f)| -  Evaluate the velocity field f at point (x, y, z).

\item  \verb|obj.SetLastCellId (vtkIdType c, int dataindex)| -  Set the cell id cached by the last evaluation within a specified dataset.

\item  \verb|obj.SetLastCellId (vtkIdType c)|

\end{itemize}
