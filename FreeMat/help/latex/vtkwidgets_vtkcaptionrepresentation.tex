\section{vtkCaptionRepresentation}

\subsection{Usage}

 This class represents vtkCaptionWidget. A caption is defined by some text
 with a leader (e.g., arrow) that points from the text to a point in the
 scene. The caption is defined by an instance of vtkCaptionActor2D. It uses
 the event bindings of its superclass (vtkBorderWidget) to control the
 placement of the text, and adds the ability to move the attachment point
 around. In addition, when the caption text is selected, the widget emits a
 ActivateEvent that observers can watch for. This is useful for opening GUI
 dialogoues to adjust font characteristics, etc. (Please see the superclass
 for a description of event bindings.)

 Note that this widget extends the behavior of its superclass 
 vtkBorderRepresentation. 

To create an instance of class vtkCaptionRepresentation, simply
invoke its constructor as follows
\begin{verbatim}
  obj = vtkCaptionRepresentation
\end{verbatim}
\subsection{Methods}

The class vtkCaptionRepresentation has several methods that can be used.
  They are listed below.
Note that the documentation is translated automatically from the VTK sources,
and may not be completely intelligible.  When in doubt, consult the VTK website.
In the methods listed below, \verb|obj| is an instance of the vtkCaptionRepresentation class.
\begin{itemize}
\item  \verb|string = obj.GetClassName ()| -  Standard VTK class methods.

\item  \verb|int = obj.IsA (string name)| -  Standard VTK class methods.

\item  \verb|vtkCaptionRepresentation = obj.NewInstance ()| -  Standard VTK class methods.

\item  \verb|vtkCaptionRepresentation = obj.SafeDownCast (vtkObject o)| -  Standard VTK class methods.

\item  \verb|obj.SetAnchorPosition (double pos[3])| -  Specify the position of the anchor (i.e., the point that the caption is anchored to).
 Note that the position should be specified in world coordinates.

\item  \verb|obj.GetAnchorPosition (double pos[3])| -  Specify the position of the anchor (i.e., the point that the caption is anchored to).
 Note that the position should be specified in world coordinates.

\item  \verb|obj.SetCaptionActor2D (vtkCaptionActor2D captionActor)| -  Specify the vtkCaptionActor2D to manage. If not specified, then one
 is automatically created.

\item  \verb|vtkCaptionActor2D = obj.GetCaptionActor2D ()| -  Specify the vtkCaptionActor2D to manage. If not specified, then one
 is automatically created.

\item  \verb|obj.SetAnchorRepresentation (vtkPointHandleRepresentation3D )| -  Set and get the instances of vtkPointHandleRepresention3D used to implement this
 representation. Normally default representations are created, but you can
 specify the ones you want to use.

\item  \verb|vtkPointHandleRepresentation3D = obj.GetAnchorRepresentation ()| -  Set and get the instances of vtkPointHandleRepresention3D used to implement this
 representation. Normally default representations are created, but you can
 specify the ones you want to use.

\item  \verb|obj.BuildRepresentation ()| -  Satisfy the superclasses API.

\item  \verb|obj.GetSize (double size[2])| -  These methods are necessary to make this representation behave as
 a vtkProp.

\item  \verb|obj.GetActors2D (vtkPropCollection )| -  These methods are necessary to make this representation behave as
 a vtkProp.

\item  \verb|obj.ReleaseGraphicsResources (vtkWindow )| -  These methods are necessary to make this representation behave as
 a vtkProp.

\item  \verb|int = obj.RenderOverlay (vtkViewport )| -  These methods are necessary to make this representation behave as
 a vtkProp.

\item  \verb|int = obj.RenderOpaqueGeometry (vtkViewport )| -  These methods are necessary to make this representation behave as
 a vtkProp.

\item  \verb|int = obj.RenderTranslucentPolygonalGeometry (vtkViewport )| -  These methods are necessary to make this representation behave as
 a vtkProp.

\item  \verb|int = obj.HasTranslucentPolygonalGeometry ()| -  These methods are necessary to make this representation behave as
 a vtkProp.

\item  \verb|obj.SetFontFactor (double )| -  Set/Get the factor that controls the overall size of the fonts 
 of the caption when the text actor's ScaledText is OFF 

\item  \verb|double = obj.GetFontFactorMinValue ()| -  Set/Get the factor that controls the overall size of the fonts 
 of the caption when the text actor's ScaledText is OFF 

\item  \verb|double = obj.GetFontFactorMaxValue ()| -  Set/Get the factor that controls the overall size of the fonts 
 of the caption when the text actor's ScaledText is OFF 

\item  \verb|double = obj.GetFontFactor ()| -  Set/Get the factor that controls the overall size of the fonts 
 of the caption when the text actor's ScaledText is OFF 

\end{itemize}
