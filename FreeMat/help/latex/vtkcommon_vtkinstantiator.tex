\section{vtkInstantiator}

\subsection{Usage}

 vtkInstantiator provides an interface to create an instance of any
 VTK class from its name.  Instances are created through registered
 pointers to functions returning the objects.  New classes can also be
 registered with the creator.  VTK libraries automatically register
 their classes with the creator when they are loaded.  Instances are
 created using the static New() method, so the normal vtkObjectFactory
 mechanism is still invoked.

 When using this class from language wrappers (Tcl, Python, or Java),
 the vtkInstantiator should be able to create any class from any kit
 that has been loaded.

 In C++ code, one should include the header for each kit from which
 one wishes to create instances through vtkInstantiator.  This is
 necessary to ensure proper linking when building static libraries.
 Be careful, though, because including each kit's header means every
 class from that kit will be linked into your executable whether or
 not the class is used.  The headers are:

   vtkCommon          - vtkCommonInstantiator.h
   vtkFiltering       - vtkFilteringInstantiator.h
   vtkIO              - vtkIOInstantiator.h
   vtkImaging         - vtkImagingInstantiator.h
   vtkGraphics        - vtkGraphicsInstantiator.h
   vtkRendering       - vtkRenderingInstantiator.h
   vtkVolumeRendering - vtkVolumeRenderingInstantiator.h
   vtkHybrid          - vtkHybridInstantiator.h
   vtkParallel        - vtkParallelInstantiator.h

 The VTK\_MAKE\_INSTANTIATOR() command in CMake is used to automatically
 generate the creator registration for each VTK library.  It can also
 be used to create registration code for VTK-style user libraries
 that are linked to vtkCommon.  After using this command to register
 classes from a new library, the generated header must be included.


To create an instance of class vtkInstantiator, simply
invoke its constructor as follows
\begin{verbatim}
  obj = vtkInstantiator
\end{verbatim}
\subsection{Methods}

The class vtkInstantiator has several methods that can be used.
  They are listed below.
Note that the documentation is translated automatically from the VTK sources,
and may not be completely intelligible.  When in doubt, consult the VTK website.
In the methods listed below, \verb|obj| is an instance of the vtkInstantiator class.
\begin{itemize}
\item  \verb|string = obj.GetClassName ()|

\item  \verb|int = obj.IsA (string name)|

\item  \verb|vtkInstantiator = obj.NewInstance ()|

\item  \verb|vtkInstantiator = obj.SafeDownCast (vtkObject o)|

\end{itemize}
