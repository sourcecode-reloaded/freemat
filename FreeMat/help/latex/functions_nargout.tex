\section{NARGOUT Number of Output Arguments}

\subsection{Usage}

The \verb|nargout| function computes the number of return values requested from
a function when it was called.  The general syntax for its use
\begin{verbatim}
   y = nargout
\end{verbatim}
FreeMat allows for
fewer return values to be requested from a function than were declared,
and \verb|nargout| can be used to determine exactly what subset of 
the functions outputs are required.  

You can also use \verb|nargout| on a function handle to return the
number of input arguments expected by the function
\begin{verbatim}
  y = nargout(fun)
\end{verbatim}
where \verb|fun| is the name of the function (e.g. \verb|'sin'|) or 
a function handle.
\subsection{Example}

Here is a function that is declared to return five 
values, and that simply prints the value of \verb|nargout|
each time it is called.
\begin{verbatim}
    nargouttest.m
function [a1,a2,a3,a4,a5] = nargouttest
  printf('nargout = %d\n',nargout);
  a1 = 1; a2 = 2; a3 = 3; a4 = 4; a5 = 5;
\end{verbatim}
\begin{verbatim}
--> a1 = nargouttest
nargout = 1

a1 = 
 1 

--> [a1,a2] = nargouttest
nargout = 2
a1 = 
 1 

a2 = 
 2 

--> [a1,a2,a3] = nargouttest
nargout = 3
a1 = 
 1 

a2 = 
 2 

a3 = 
 3 

--> [a1,a2,a3,a4,a5] = nargouttest
nargout = 5
a1 = 
 1 

a2 = 
 2 

a3 = 
 3 

a4 = 
 4 

a5 = 
 5 

--> nargout('sin')

ans = 
 1 

--> y = @sin

y = 
 @sin
--> nargout(y)

ans = 
 1 
\end{verbatim}
