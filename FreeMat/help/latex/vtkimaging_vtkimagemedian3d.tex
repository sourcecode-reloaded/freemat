\section{vtkImageMedian3D}

\subsection{Usage}

 vtkImageMedian3D a Median filter that replaces each pixel with the 
 median value from a rectangular neighborhood around that pixel.
 Neighborhoods can be no more than 3 dimensional.  Setting one
 axis of the neighborhood kernelSize to 1 changes the filter
 into a 2D median.  

To create an instance of class vtkImageMedian3D, simply
invoke its constructor as follows
\begin{verbatim}
  obj = vtkImageMedian3D
\end{verbatim}
\subsection{Methods}

The class vtkImageMedian3D has several methods that can be used.
  They are listed below.
Note that the documentation is translated automatically from the VTK sources,
and may not be completely intelligible.  When in doubt, consult the VTK website.
In the methods listed below, \verb|obj| is an instance of the vtkImageMedian3D class.
\begin{itemize}
\item  \verb|string = obj.GetClassName ()|

\item  \verb|int = obj.IsA (string name)|

\item  \verb|vtkImageMedian3D = obj.NewInstance ()|

\item  \verb|vtkImageMedian3D = obj.SafeDownCast (vtkObject o)|

\item  \verb|obj.SetKernelSize (int size0, int size1, int size2)| -  This method sets the size of the neighborhood.  It also sets the 
 default middle of the neighborhood 

\item  \verb|int = obj.GetNumberOfElements ()| -  Return the number of elements in the median mask

\end{itemize}
