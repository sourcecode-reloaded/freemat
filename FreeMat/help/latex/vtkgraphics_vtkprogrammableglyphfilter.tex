\section{vtkProgrammableGlyphFilter}

\subsection{Usage}

 vtkProgrammableGlyphFilter is a filter that allows you to place a glyph at
 each input point in the dataset. In addition, the filter is programmable
 which means the user has control over the generation of the glyph. The
 glyphs can be controlled via the point data attributes (e.g., scalars,
 vectors, etc.) or any other information in the input dataset.

 This is the way the filter works. You must define an input dataset which
 at a minimum contains points with associated attribute values. Also, the
 Source instance variable must be set which is of type vtkPolyData. Then,
 for each point in the input, the PointId is set to the current point id,
 and a user-defined function is called (i.e., GlyphMethod). In this method
 you can manipulate the Source data (including changing to a different
 Source object). After the GlyphMethod is called,
 vtkProgrammableGlyphFilter will invoke an Update() on its Source object,
 and then copy its data to the output of the
 vtkProgrammableGlyphFilter. Therefore the output of this filter is of type
 vtkPolyData.

 Another option to this filter is the way you color the glyphs. You can use
 the scalar data from the input or the source. The instance variable
 ColorMode controls this behavior.

To create an instance of class vtkProgrammableGlyphFilter, simply
invoke its constructor as follows
\begin{verbatim}
  obj = vtkProgrammableGlyphFilter
\end{verbatim}
\subsection{Methods}

The class vtkProgrammableGlyphFilter has several methods that can be used.
  They are listed below.
Note that the documentation is translated automatically from the VTK sources,
and may not be completely intelligible.  When in doubt, consult the VTK website.
In the methods listed below, \verb|obj| is an instance of the vtkProgrammableGlyphFilter class.
\begin{itemize}
\item  \verb|string = obj.GetClassName ()|

\item  \verb|int = obj.IsA (string name)|

\item  \verb|vtkProgrammableGlyphFilter = obj.NewInstance ()|

\item  \verb|vtkProgrammableGlyphFilter = obj.SafeDownCast (vtkObject o)|

\item  \verb|obj.SetSource (vtkPolyData source)| -  Set/Get the source to use for this glyph. 
 Note: you can change the source during execution of this filter.

\item  \verb|vtkPolyData = obj.GetSource ()| -  Set/Get the source to use for this glyph. 
 Note: you can change the source during execution of this filter.

\item  \verb|vtkIdType = obj.GetPointId ()| -  Get the current point id during processing. Value only valid during the
 Execute() method of this filter. (Meant to be called by the GlyphMethod().)

\item  \verb|double = obj. GetPoint ()| -  Get the current point coordinates during processing. Value only valid during the
 Execute() method of this filter. (Meant to be called by the GlyphMethod().)

\item  \verb|vtkPointData = obj.GetPointData ()| -  Get the set of point data attributes for the input. A convenience to the
 programmer to be used in the GlyphMethod(). Only valid during the Execute()
 method of this filter.

\item  \verb|obj.SetColorMode (int )| -  Either color by the input or source scalar data.

\item  \verb|int = obj.GetColorMode ()| -  Either color by the input or source scalar data.

\item  \verb|obj.SetColorModeToColorByInput ()| -  Either color by the input or source scalar data.

\item  \verb|obj.SetColorModeToColorBySource ()| -  Either color by the input or source scalar data.

\item  \verb|string = obj.GetColorModeAsString ()| -  Either color by the input or source scalar data.

\end{itemize}
