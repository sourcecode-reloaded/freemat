\section{vtkDataObjectSource}

\subsection{Usage}

 vtkDataObjectSource is an abstract object that specifies behavior and
 interface of field source objects. Field source objects are source objects
 that create vtkFieldData (field data) on output.

 Concrete subclasses of vtkDataObjectSource must define Update() and
 Execute() methods. The public method Update() invokes network execution
 and will bring the network up-to-date. The protected Execute() method
 actually does the work of data creation/generation. The difference between
 the two methods is that Update() implements input consistency checks and
 modified time comparisons and then invokes the Execute() which is an
 implementation of a particular algorithm.

 vtkDataObjectSource provides a mechanism for invoking the methods
 StartMethod() and EndMethod() before and after object execution (via
 Execute()). These are convenience methods you can use for any purpose
 (e.g., debugging info, highlighting/notifying user interface, etc.) These
 methods accept a single void* pointer that can be used to send data to the
 methods. It is also possible to specify a function to delete the argument
 via StartMethodArgDelete and EndMethodArgDelete.

 Another method, ProgressMethod() can be specified. Some filters invoke this 
 method periodically during their execution. The use is similar to that of 
 StartMethod() and EndMethod().

 An important feature of subclasses of vtkDataObjectSource is that it is
 possible to control the memory-management model (i.e., retain output
 versus delete output data). If enabled the ReleaseDataFlag enables the
 deletion of the output data once the downstream process object finishes
 processing the data (please see text).

To create an instance of class vtkDataObjectSource, simply
invoke its constructor as follows
\begin{verbatim}
  obj = vtkDataObjectSource
\end{verbatim}
\subsection{Methods}

The class vtkDataObjectSource has several methods that can be used.
  They are listed below.
Note that the documentation is translated automatically from the VTK sources,
and may not be completely intelligible.  When in doubt, consult the VTK website.
In the methods listed below, \verb|obj| is an instance of the vtkDataObjectSource class.
\begin{itemize}
\item  \verb|string = obj.GetClassName ()|

\item  \verb|int = obj.IsA (string name)|

\item  \verb|vtkDataObjectSource = obj.NewInstance ()|

\item  \verb|vtkDataObjectSource = obj.SafeDownCast (vtkObject o)|

\item  \verb|vtkDataObject = obj.GetOutput ()| -  Get the output field of this source.

\item  \verb|vtkDataObject = obj.GetOutput (int idx)|

\item  \verb|obj.SetOutput (vtkDataObject )|

\end{itemize}
