\section{vtkProcessObject}

\subsection{Usage}

 vtkProcessObject is an abstract object that specifies behavior and
 interface of visualization network process objects (sources, filters,
 mappers). Source objects are creators of visualization data; filters
 input, process, and output visualization data; and mappers transform data
 into another form (like rendering primitives or write data to a file).

 vtkProcessObject fires events for Start and End events before and after
 object execution (via Execute()). These events can be used for any purpose
 (e.g., debugging info, highlighting/notifying user interface, etc.)

 Another event, Progress, can be observed. Some filters fire this 
 event periodically during their execution. The use is similar to that of 
 Start and End events. Filters may also check their AbortExecute
 flag to determine whether to prematurely end their execution.

 An important feature of subclasses of vtkProcessObject is that it is
 possible to control the memory-management model (i.e., retain output
 versus delete output data). If enabled the ReleaseDataFlag enables the
 deletion of the output data once the downstream process object finishes
 processing the data (please see text).  


To create an instance of class vtkProcessObject, simply
invoke its constructor as follows
\begin{verbatim}
  obj = vtkProcessObject
\end{verbatim}
\subsection{Methods}

The class vtkProcessObject has several methods that can be used.
  They are listed below.
Note that the documentation is translated automatically from the VTK sources,
and may not be completely intelligible.  When in doubt, consult the VTK website.
In the methods listed below, \verb|obj| is an instance of the vtkProcessObject class.
\begin{itemize}
\item  \verb|string = obj.GetClassName ()|

\item  \verb|int = obj.IsA (string name)|

\item  \verb|vtkProcessObject = obj.NewInstance ()|

\item  \verb|vtkProcessObject = obj.SafeDownCast (vtkObject o)|

\item  \verb|int = obj.GetNumberOfInputs ()| -  Return an array with all the inputs of this process object.
 This is useful for tracing back in the pipeline to construct
 graphs etc.

\item  \verb|obj.SqueezeInputArray ()| -  This method will rearrange the input array so that all NULL entries 
 are removed.

\item  \verb|obj.RemoveAllInputs ()| -  Remove all the input data.

\item  \verb|obj.SetInputConnection (vtkAlgorithmOutput input)| -  Reimplemented from vtkAlgorithm to maintain backward
 compatibility for vtkProcessObject.

\item  \verb|obj.SetInputConnection (int port, vtkAlgorithmOutput input)| -  Reimplemented from vtkAlgorithm to maintain backward
 compatibility for vtkProcessObject.

\item  \verb|obj.AddInputConnection (int port, vtkAlgorithmOutput input)| -  Reimplemented from vtkAlgorithm to maintain backward
 compatibility for vtkProcessObject.

\item  \verb|obj.AddInputConnection (vtkAlgorithmOutput input)| -  Reimplemented from vtkAlgorithm to maintain backward
 compatibility for vtkProcessObject.

\item  \verb|obj.RemoveInputConnection (int port, vtkAlgorithmOutput input)| -  Reimplemented from vtkAlgorithm to maintain backward
 compatibility for vtkProcessObject.

\item  \verb|obj.SetNthInputConnection (int port, int index, vtkAlgorithmOutput input)| -  Reimplemented from vtkAlgorithm to maintain backward
 compatibility for vtkProcessObject.

\item  \verb|obj.SetNumberOfInputConnections (int port, int n)| -  Reimplemented from vtkAlgorithm to maintain backward
 compatibility for vtkProcessObject.

\end{itemize}
