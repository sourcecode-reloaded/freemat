\section{vtkPropPicker}

\subsection{Usage}

 vtkPropPicker is used to pick an actor/prop given a selection
 point (in display coordinates) and a renderer. This class uses
 graphics hardware/rendering system to pick rapidly (as compared
 to using ray casting as does vtkCellPicker and vtkPointPicker).
 This class determines the actor/prop and pick position in world
 coordinates; point and cell ids are not determined.

To create an instance of class vtkPropPicker, simply
invoke its constructor as follows
\begin{verbatim}
  obj = vtkPropPicker
\end{verbatim}
\subsection{Methods}

The class vtkPropPicker has several methods that can be used.
  They are listed below.
Note that the documentation is translated automatically from the VTK sources,
and may not be completely intelligible.  When in doubt, consult the VTK website.
In the methods listed below, \verb|obj| is an instance of the vtkPropPicker class.
\begin{itemize}
\item  \verb|string = obj.GetClassName ()|

\item  \verb|int = obj.IsA (string name)|

\item  \verb|vtkPropPicker = obj.NewInstance ()|

\item  \verb|vtkPropPicker = obj.SafeDownCast (vtkObject o)|

\item  \verb|int = obj.PickProp (double selectionX, double selectionY, vtkRenderer renderer)| -  Perform the pick and set the PickedProp ivar. If something is picked, a
 1 is returned, otherwise 0 is returned.  Use the GetViewProp() method
 to get the instance of vtkProp that was picked.  Props are picked from
 the renderers list of pickable Props.

\item  \verb|int = obj.PickProp (double selectionX, double selectionY, vtkRenderer renderer, vtkPropCollection pickfrom)| -  Perform a pick from the user-provided list of vtkProps and not from the
 list of vtkProps that the render maintains.

\item  \verb|int = obj.Pick (double selectionX, double selectionY, double selectionZ, vtkRenderer renderer)| -  Overide superclasses' Pick() method.

\item  \verb|int = obj.Pick (double selectionPt[3], vtkRenderer renderer)| -  Overide superclasses' Pick() method.

\end{itemize}
