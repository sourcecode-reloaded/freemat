\section{vtkOctreePointLocatorNode}

\subsection{Usage}

 This class represents a single spatial region in a 3D axis octant 
 partitioning.  It is intended to work efficiently with the 
 vtkOctreePointLocator and is not meant for general use.  It is assumed 
 the region bounds some set of points.  The ordering of the children is 
 (-x,-y,-z),(+x,-y,-z),(-x,+y,-z),(+x,+y,-z),(-x,-y,+z),(+x,-y,+z),
 (-x,+y,+z),(+x,+y,+z).  The portion of the domain assigned to an 
 octant is Min < x <= Max.  
    

To create an instance of class vtkOctreePointLocatorNode, simply
invoke its constructor as follows
\begin{verbatim}
  obj = vtkOctreePointLocatorNode
\end{verbatim}
\subsection{Methods}

The class vtkOctreePointLocatorNode has several methods that can be used.
  They are listed below.
Note that the documentation is translated automatically from the VTK sources,
and may not be completely intelligible.  When in doubt, consult the VTK website.
In the methods listed below, \verb|obj| is an instance of the vtkOctreePointLocatorNode class.
\begin{itemize}
\item  \verb|string = obj.GetClassName ()|

\item  \verb|int = obj.IsA (string name)|

\item  \verb|vtkOctreePointLocatorNode = obj.NewInstance ()|

\item  \verb|vtkOctreePointLocatorNode = obj.SafeDownCast (vtkObject o)|

\item  \verb|obj.SetNumberOfPoints (int numberOfPoints)| -  Set/Get the number of points contained in this region.

\item  \verb|int = obj.GetNumberOfPoints ()| -  Set/Get the number of points contained in this region.

\item  \verb|obj.SetBounds (double xMin, double xMax, double yMin, double yMax, double zMin, double zMax)| -    Set/Get the bounds of the spatial region represented by this node.
   Caller allocates storage for 6-vector in GetBounds.

\item  \verb|obj.SetBounds (double b[6])| -    Set/Get the bounds of the spatial region represented by this node.
   Caller allocates storage for 6-vector in GetBounds.

\item  \verb|obj.GetBounds (double b) const| -    Set/Get the bounds of the spatial region represented by this node.
   Caller allocates storage for 6-vector in GetBounds.

\item  \verb|obj.SetDataBounds (double xMin, double xMax, double yMin, double yMax, double zMin, double zMax)| -    Set/Get the bounds of the points contained in this spatial region.
   This may be smaller than the bounds of the region itself.
   Caller allocates storage for 6-vector in GetDataBounds.

\item  \verb|obj.GetDataBounds (double b) const| -    Set/Get the bounds of the points contained in this spatial region.
   This may be smaller than the bounds of the region itself.
   Caller allocates storage for 6-vector in GetDataBounds.

\item  \verb|obj.SetMinBounds (double minBounds[3])| -    Set the xmax, ymax and zmax value of the bounds of this region

\item  \verb|obj.SetMaxBounds (double maxBounds[3])| -    Set the xmin, ymin and zmin value of the bounds of this 
   data within this region.

\item  \verb|obj.SetMinDataBounds (double minDataBounds[3])| -    Set the xmax, ymax and zmax value of the bounds of this 
   data within this region.

\item  \verb|obj.SetMaxDataBounds (double maxDataBounds[3])| -    Get the ID associated with the region described by this node.  If
   this is not a leaf node, this value should be -1.

\item  \verb|int = obj.GetID ()| -    Get the ID associated with the region described by this node.  If
   this is not a leaf node, this value should be -1.

\item  \verb|int = obj.GetMinID ()| -    If this node is not a leaf node, there are leaf nodes below it whose
   regions represent a partitioning of this region.  The IDs of these
   leaf nodes form a contigous set.  Get the first of the first point's
   ID that is contained in this node.  

\item  \verb|obj.CreateChildNodes ()| -    Add the 8 children.

\item  \verb|obj.DeleteChildNodes ()| -    Delete the 8 children.

\item  \verb|vtkOctreePointLocatorNode = obj.GetChild (int i)| -    Get a pointer to the ith child of this node.

\item  \verb|int = obj.IntersectsRegion (vtkPlanesIntersection pi, int useDataBounds)| -    A vtkPlanesIntersection object represents a convex 3D region bounded
   by planes, and it is capable of computing intersections of
   boxes with itself.  Return 1 if this spatial region intersects
   the spatial region described by the vtkPlanesIntersection object.
   Use the possibly smaller bounds of the points within the region 
   if useDataBounds is non-zero.

\item  \verb|int = obj.ContainsPoint (double x, double y, double z, int useDataBounds)| -    Return 1 if this spatial region entirely contains the given point.
   Use the possibly smaller bounds of the points within the region 
   if useDataBounds is non-zero.

\item  \verb|double = obj.GetDistance2ToBoundary (double x, double y, double z, vtkOctreePointLocatorNode top, int useDataBounds)| -    Calculate the distance squared from any point to the boundary of this
   region.  Use the boundary of the points within the region if useDataBounds
   is non-zero.

\item  \verb|double = obj.GetDistance2ToBoundary (double x, double y, double z, double boundaryPt, vtkOctreePointLocatorNode top, int useDataBounds)| -    Calculate the distance squared from any point to the boundary of this
   region.  Use the boundary of the points within the region if useDataBounds
   is non-zero.  Set boundaryPt to the point on the boundary.

\item  \verb|double = obj.GetDistance2ToInnerBoundary (double x, double y, double z, vtkOctreePointLocatorNode top)| -    Calculate the distance from the specified point (which is required to
   be inside this spatial region) to an interior boundary.  An interior
   boundary is one that is not also an boundary of the entire space
   partitioned by the tree of vtkOctreePointLocatorNode's.

\item  \verb|int = obj.GetSubOctantIndex (double point, int CheckContainment)| -  Return the id of the suboctant that a given point is in.
 If CheckContainment is non-zero then it checks whether
 the point is in the actual bounding box of the suboctant,
 otherwise it only checks which octant the point is in
 that is created from the axis-aligned partitioning of 
 the domain at this octant's center.

\end{itemize}
