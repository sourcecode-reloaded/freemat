\section{vtkThreadedImageAlgorithm}

\subsection{Usage}

 vtkThreadedImageAlgorithm is a filter superclass that hides much of the 
 pipeline  complexity. It handles breaking the pipeline execution 
 into smaller extents so that the vtkImageData limits are observed. It 
 also provides support for multithreading. If you don't need any of this
 functionality, consider using vtkSimpleImageToImageAlgorithm instead.
 .SECTION See also
 vtkSimpleImageToImageAlgorithm

To create an instance of class vtkThreadedImageAlgorithm, simply
invoke its constructor as follows
\begin{verbatim}
  obj = vtkThreadedImageAlgorithm
\end{verbatim}
\subsection{Methods}

The class vtkThreadedImageAlgorithm has several methods that can be used.
  They are listed below.
Note that the documentation is translated automatically from the VTK sources,
and may not be completely intelligible.  When in doubt, consult the VTK website.
In the methods listed below, \verb|obj| is an instance of the vtkThreadedImageAlgorithm class.
\begin{itemize}
\item  \verb|string = obj.GetClassName ()|

\item  \verb|int = obj.IsA (string name)|

\item  \verb|vtkThreadedImageAlgorithm = obj.NewInstance ()|

\item  \verb|vtkThreadedImageAlgorithm = obj.SafeDownCast (vtkObject o)|

\item  \verb|obj.ThreadedExecute (vtkImageData inData, vtkImageData outData, int extent[6], int threadId)|

\item  \verb|obj.SetNumberOfThreads (int )| -  Get/Set the number of threads to create when rendering

\item  \verb|int = obj.GetNumberOfThreadsMinValue ()| -  Get/Set the number of threads to create when rendering

\item  \verb|int = obj.GetNumberOfThreadsMaxValue ()| -  Get/Set the number of threads to create when rendering

\item  \verb|int = obj.GetNumberOfThreads ()| -  Get/Set the number of threads to create when rendering

\item  \verb|int = obj.SplitExtent (int splitExt[6], int startExt[6], int num, int total)| -  Putting this here until I merge graphics and imaging streaming.

\end{itemize}
