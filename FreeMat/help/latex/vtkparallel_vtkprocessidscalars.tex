\section{vtkProcessIdScalars}

\subsection{Usage}

 vtkProcessIdScalars is meant to display which processor owns which cells
 and points.  It is useful for visualizing the partitioning for
 streaming or distributed pipelines.


To create an instance of class vtkProcessIdScalars, simply
invoke its constructor as follows
\begin{verbatim}
  obj = vtkProcessIdScalars
\end{verbatim}
\subsection{Methods}

The class vtkProcessIdScalars has several methods that can be used.
  They are listed below.
Note that the documentation is translated automatically from the VTK sources,
and may not be completely intelligible.  When in doubt, consult the VTK website.
In the methods listed below, \verb|obj| is an instance of the vtkProcessIdScalars class.
\begin{itemize}
\item  \verb|string = obj.GetClassName ()|

\item  \verb|int = obj.IsA (string name)|

\item  \verb|vtkProcessIdScalars = obj.NewInstance ()|

\item  \verb|vtkProcessIdScalars = obj.SafeDownCast (vtkObject o)|

\item  \verb|obj.SetScalarModeToCellData ()| -  Option to centerate cell scalars of points scalars.  Default is point
 scalars.

\item  \verb|obj.SetScalarModeToPointData ()| -  Option to centerate cell scalars of points scalars.  Default is point
 scalars.

\item  \verb|int = obj.GetScalarMode ()|

\item  \verb|obj.SetRandomMode (int )|

\item  \verb|int = obj.GetRandomMode ()|

\item  \verb|obj.RandomModeOn ()|

\item  \verb|obj.RandomModeOff ()|

\item  \verb|obj.SetController (vtkMultiProcessController )| -  By defualt this filter uses the global controller,
 but this method can be used to set another instead.

\item  \verb|vtkMultiProcessController = obj.GetController ()| -  By defualt this filter uses the global controller,
 but this method can be used to set another instead.

\end{itemize}
