\section{vtkArcSource}

\subsection{Usage}

 vtkArcSource is a source object that creates an arc defined by two 
 endpoints and a center. The number of segments composing the polyline is
 controlled by setting the object resolution.

To create an instance of class vtkArcSource, simply
invoke its constructor as follows
\begin{verbatim}
  obj = vtkArcSource
\end{verbatim}
\subsection{Methods}

The class vtkArcSource has several methods that can be used.
  They are listed below.
Note that the documentation is translated automatically from the VTK sources,
and may not be completely intelligible.  When in doubt, consult the VTK website.
In the methods listed below, \verb|obj| is an instance of the vtkArcSource class.
\begin{itemize}
\item  \verb|string = obj.GetClassName ()|

\item  \verb|int = obj.IsA (string name)|

\item  \verb|vtkArcSource = obj.NewInstance ()|

\item  \verb|vtkArcSource = obj.SafeDownCast (vtkObject o)|

\item  \verb|obj.SetPoint1 (double , double , double )| -  Set position of first end point.

\item  \verb|obj.SetPoint1 (double  a[3])| -  Set position of first end point.

\item  \verb|double = obj. GetPoint1 ()| -  Set position of first end point.

\item  \verb|obj.SetPoint2 (double , double , double )| -  Set position of other end point.

\item  \verb|obj.SetPoint2 (double  a[3])| -  Set position of other end point.

\item  \verb|double = obj. GetPoint2 ()| -  Set position of other end point.

\item  \verb|obj.SetCenter (double , double , double )| -  Set position of the center of the circle that define the arc.
 Note: you can use the function vtkMath::Solve3PointCircle to
 find the center from 3 points located on a circle.

\item  \verb|obj.SetCenter (double  a[3])| -  Set position of the center of the circle that define the arc.
 Note: you can use the function vtkMath::Solve3PointCircle to
 find the center from 3 points located on a circle.

\item  \verb|double = obj. GetCenter ()| -  Set position of the center of the circle that define the arc.
 Note: you can use the function vtkMath::Solve3PointCircle to
 find the center from 3 points located on a circle.

\item  \verb|obj.SetResolution (int )| -  Divide line into resolution number of pieces.
 Note: if Resolution is set to 1 (default), the arc is a 
 straight line.

\item  \verb|int = obj.GetResolutionMinValue ()| -  Divide line into resolution number of pieces.
 Note: if Resolution is set to 1 (default), the arc is a 
 straight line.

\item  \verb|int = obj.GetResolutionMaxValue ()| -  Divide line into resolution number of pieces.
 Note: if Resolution is set to 1 (default), the arc is a 
 straight line.

\item  \verb|int = obj.GetResolution ()| -  Divide line into resolution number of pieces.
 Note: if Resolution is set to 1 (default), the arc is a 
 straight line.

\item  \verb|obj.SetNegative (bool )| -  Use the angle that is a negative coterminal of the vectors angle:
 the longest angle.
 Note: false by default.

\item  \verb|bool = obj.GetNegative ()| -  Use the angle that is a negative coterminal of the vectors angle:
 the longest angle.
 Note: false by default.

\item  \verb|obj.NegativeOn ()| -  Use the angle that is a negative coterminal of the vectors angle:
 the longest angle.
 Note: false by default.

\item  \verb|obj.NegativeOff ()| -  Use the angle that is a negative coterminal of the vectors angle:
 the longest angle.
 Note: false by default.

\end{itemize}
