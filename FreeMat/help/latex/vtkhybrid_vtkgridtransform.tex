\section{vtkGridTransform}

\subsection{Usage}

 vtkGridTransform describes a nonlinear warp transformation as a set
 of displacement vectors sampled along a uniform 3D grid.

To create an instance of class vtkGridTransform, simply
invoke its constructor as follows
\begin{verbatim}
  obj = vtkGridTransform
\end{verbatim}
\subsection{Methods}

The class vtkGridTransform has several methods that can be used.
  They are listed below.
Note that the documentation is translated automatically from the VTK sources,
and may not be completely intelligible.  When in doubt, consult the VTK website.
In the methods listed below, \verb|obj| is an instance of the vtkGridTransform class.
\begin{itemize}
\item  \verb|string = obj.GetClassName ()|

\item  \verb|int = obj.IsA (string name)|

\item  \verb|vtkGridTransform = obj.NewInstance ()|

\item  \verb|vtkGridTransform = obj.SafeDownCast (vtkObject o)|

\item  \verb|obj.SetDisplacementGrid (vtkImageData )| -  Set/Get the grid transform (the grid transform must have three 
 components for displacement in x, y, and z respectively).
 The vtkGridTransform class will never modify the data.

\item  \verb|vtkImageData = obj.GetDisplacementGrid ()| -  Set/Get the grid transform (the grid transform must have three 
 components for displacement in x, y, and z respectively).
 The vtkGridTransform class will never modify the data.

\item  \verb|obj.SetDisplacementScale (double )| -  Set scale factor to be applied to the displacements.
 This is used primarily for grids which contain integer
 data types.  Default: 1

\item  \verb|double = obj.GetDisplacementScale ()| -  Set scale factor to be applied to the displacements.
 This is used primarily for grids which contain integer
 data types.  Default: 1

\item  \verb|obj.SetDisplacementShift (double )| -  Set a shift to be applied to the displacements.  The shift
 is applied after the scale, i.e. x = scale*y + shift.
 Default: 0

\item  \verb|double = obj.GetDisplacementShift ()| -  Set a shift to be applied to the displacements.  The shift
 is applied after the scale, i.e. x = scale*y + shift.
 Default: 0

\item  \verb|obj.SetInterpolationMode (int mode)| -  Set interpolation mode for sampling the grid.  Higher-order
 interpolation allows you to use a sparser grid.
 Default: Linear.

\item  \verb|int = obj.GetInterpolationMode ()| -  Set interpolation mode for sampling the grid.  Higher-order
 interpolation allows you to use a sparser grid.
 Default: Linear.

\item  \verb|obj.SetInterpolationModeToNearestNeighbor ()| -  Set interpolation mode for sampling the grid.  Higher-order
 interpolation allows you to use a sparser grid.
 Default: Linear.

\item  \verb|obj.SetInterpolationModeToLinear ()| -  Set interpolation mode for sampling the grid.  Higher-order
 interpolation allows you to use a sparser grid.
 Default: Linear.

\item  \verb|obj.SetInterpolationModeToCubic ()| -  Set interpolation mode for sampling the grid.  Higher-order
 interpolation allows you to use a sparser grid.
 Default: Linear.

\item  \verb|string = obj.GetInterpolationModeAsString ()| -  Set interpolation mode for sampling the grid.  Higher-order
 interpolation allows you to use a sparser grid.
 Default: Linear.

\item  \verb|vtkAbstractTransform = obj.MakeTransform ()| -  Make another transform of the same type.

\item  \verb|long = obj.GetMTime ()| -  Get the MTime.

\end{itemize}
