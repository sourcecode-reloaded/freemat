\section{vtkGenericDataSet}

\subsection{Usage}

 In VTK, spatial-temporal data is defined in terms of a dataset. The
 dataset consists of geometry (e.g., points), topology (e.g., cells), and
 attributes (e.g., scalars, vectors, etc.) vtkGenericDataSet is an abstract
 class defining this abstraction.

 Since vtkGenericDataSet provides a general interface to manipulate data,
 algorithms that process it tend to be slower than those specialized for a
 particular data type. For this reason, there are concrete, non-abstract
 subclasses that represent and provide access to data more efficiently.
 Note that filters to process this dataset type are currently found in the
 VTK/GenericFiltering/ subdirectory.

 Unlike the vtkDataSet class, vtkGenericDataSet provides a more flexible
 interface including support for iterators. vtkGenericDataSet is also
 designed to interface VTK to external simulation packages without the
 penalty of copying memory (see VTK/GenericFiltering/README.html) for
 more information. Thus vtkGenericDataSet plays a central role in the
 adaptor framework.

 Please note that this class introduces the concepts of ''boundary cells''.
 This refers to the boundaries of a cell (e.g., face of a tetrahedron)
 which may in turn be represented as a cell. Boundary cells are derivative
 topological features of cells, and are therefore never explicitly
 represented in the dataset. Often in visualization algorithms, looping
 over boundaries (edges or faces) is employed, while the actual dataset
 cells may not traversed. Thus there are methods to loop over these
 boundary cells.

 Finally, as a point of clarification, points are not the same as vertices.
 Vertices refer to points, and points specify a position is space. Vertices
 are a type of 0-D cell. Also, the concept of a DOFNode, which is where
 coefficients for higher-order cells are kept, is a new concept introduced
 by the adaptor framework (see vtkGenericAdaptorCell for more information).


To create an instance of class vtkGenericDataSet, simply
invoke its constructor as follows
\begin{verbatim}
  obj = vtkGenericDataSet
\end{verbatim}
\subsection{Methods}

The class vtkGenericDataSet has several methods that can be used.
  They are listed below.
Note that the documentation is translated automatically from the VTK sources,
and may not be completely intelligible.  When in doubt, consult the VTK website.
In the methods listed below, \verb|obj| is an instance of the vtkGenericDataSet class.
\begin{itemize}
\item  \verb|string = obj.GetClassName ()| -  Standard VTK type and print macros.

\item  \verb|int = obj.IsA (string name)| -  Standard VTK type and print macros.

\item  \verb|vtkGenericDataSet = obj.NewInstance ()| -  Standard VTK type and print macros.

\item  \verb|vtkGenericDataSet = obj.SafeDownCast (vtkObject o)| -  Standard VTK type and print macros.

\item  \verb|vtkIdType = obj.GetNumberOfPoints ()| -  Return the number of points composing the dataset. See NewPointIterator()
 for more details.
 

\item  \verb|vtkIdType = obj.GetNumberOfCells (int dim)| -  Return the number of cells that explicitly define the dataset. See
 NewCellIterator() for more details.
 
 

\item  \verb|int = obj.GetCellDimension ()| -  Return -1 if the dataset is explicitly defined by cells of varying
 dimensions or if there are no cells. If the dataset is explicitly
 defined by cells of a unique dimension, return this dimension.
 

\item  \verb|obj.GetCellTypes (vtkCellTypes types)| -  Get a list of types of cells in a dataset. The list consists of an array
 of types (not necessarily in any order), with a single entry per type.
 For example a dataset 5 triangles, 3 lines, and 100 hexahedra would
 result a list of three entries, corresponding to the types VTK\_TRIANGLE,
 VTK\_LINE, and VTK\_HEXAHEDRON.
 THIS METHOD IS THREAD SAFE IF FIRST CALLED FROM A SINGLE THREAD AND
 THE DATASET IS NOT MODIFIED
 

\item  \verb|vtkGenericCellIterator = obj.NewCellIterator (int dim)| -  Return an iterator to traverse cells of dimension `dim' (or all
 dimensions if -1) that explicitly define the dataset. For instance, it
 will return only tetrahedra if the mesh is defined by tetrahedra. If the
 mesh is composed of two parts, one with tetrahedra and another part with
 triangles, it will return both, but will not return the boundary edges
 and vertices of these cells. The user is responsible for deleting the
 iterator.
 
 

\item  \verb|vtkGenericCellIterator = obj.NewBoundaryIterator (int dim, int exteriorOnly)| -  Return an iterator to traverse cell boundaries of dimension `dim' (or
 all dimensions if -1) of the dataset.  If `exteriorOnly' is true, only
 the exterior cell boundaries of the dataset will be returned, otherwise
 it will return exterior and interior cell boundaries. The user is
 responsible for deleting the iterator.
 
 

\item  \verb|vtkGenericPointIterator = obj.NewPointIterator ()| -  Return an iterator to traverse the points composing the dataset; they
 can be points that define a cell (corner points) or isolated points.
 The user is responsible for deleting the iterator.
 

\item  \verb|obj.FindPoint (double x[3], vtkGenericPointIterator p)| -  Locate the closest point `p' to position `x' (global coordinates).
 
 

\item  \verb|long = obj.GetMTime ()| -  Datasets are composite objects and need to check each part for their
 modified time.

\item  \verb|obj.ComputeBounds ()| -  Compute the geometry bounding box.

\item  \verb|obj.GetBounds (double bounds[6])| -  Return the geometry bounding box in global coordinates in
 the form (xmin,xmax, ymin,ymax, zmin,zmax) in the `bounds' array.

\item  \verb|obj.GetCenter (double center[3])| -  Get the center of the bounding box in global coordinates.

\item  \verb|double = obj.GetLength ()| -  Return the length of the diagonal of the bounding box.
 

\item  \verb|vtkGenericAttributeCollection = obj.GetAttributes ()| -  Get the collection of attributes associated with this dataset.

\item  \verb|vtkDataSetAttributes = obj.GetAttributes (int type)| -  Set/Get a cell tessellator if cells must be tessellated during
 processing.
 

\item  \verb|obj.SetTessellator (vtkGenericCellTessellator tessellator)| -  Set/Get a cell tessellator if cells must be tessellated during
 processing.
 

\item  \verb|vtkGenericCellTessellator = obj.GetTessellator ()| -  Set/Get a cell tessellator if cells must be tessellated during
 processing.
 

\item  \verb|long = obj.GetActualMemorySize ()| -  Actual size of the data in kilobytes; only valid after the pipeline has
 updated. It is guaranteed to be greater than or equal to the memory
 required to represent the data.

\item  \verb|int = obj.GetDataObjectType ()| -  Return the type of data object.

\item  \verb|vtkIdType = obj.GetEstimatedSize ()| -  Estimated size needed after tessellation (or special operation)

\end{itemize}
