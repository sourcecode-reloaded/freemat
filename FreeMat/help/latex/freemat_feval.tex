\section{FEVAL Evaluate a Function}

\subsection{Usage}

The \verb|feval| function executes a function using its name.
The syntax of \verb|feval| is
\begin{verbatim}
  [y1,y2,...,yn] = feval(f,x1,x2,...,xm)
\end{verbatim}
where \verb|f| is the name of the function to evaluate, and
\verb|xi| are the arguments to the function, and \verb|yi| are the
return values.

Alternately, \verb|f| can be a function handle to a function
(see the section on \verb|function handles| for more information).

Finally, FreeMat also supports \verb|f| being a user defined class
in which case it will atttempt to invoke the \verb|subsref| method
of the class.
\subsection{Example}

Here is an example of using \verb|feval| to call the \verb|cos| 
function indirectly.
\begin{verbatim}
--> feval('cos',pi/4)

ans = 
    0.7071 
\end{verbatim}
Now, we call it through a function handle
\begin{verbatim}
--> c = @cos

c = 
 @cos
--> feval(c,pi/4)

ans = 
    0.7071 
\end{verbatim}
Here we construct an inline object (which is a user-defined class)
and use \verb|feval| to call it
\begin{verbatim}
--> afunc = inline('cos(t)+sin(t)','t')

afunc = 
  inline function object
  f(t) = cos(t)+sin(t)
--> feval(afunc,pi)

ans = 
   -1.0000 

--> afunc(pi)

ans = 
   -1.0000 
\end{verbatim}
In both cases, (the \verb|feval| call and the direct invokation), FreeMat
calls the \verb|subsref| method of the class, which computes the requested 
function.
