% Copyright (c) 2002, 2003 Samit Basu
%
% Permission is hereby granted, free of charge, to any person obtaining a 
% copy of this software and associated documentation files (the "Software"), 
% to deal in the Software without restriction, including without limitation 
% the rights to use, copy, modify, merge, publish, distribute, sublicense, 
% and/or sell copies of the Software, and to permit persons to whom the 
% Software is furnished to do so, subject to the following conditions:
%
% The above copyright notice and this permission notice shall be included 
% in all copies or substantial portions of the Software.
%
% THE SOFTWARE IS PROVIDED "AS IS", WITHOUT WARRANTY OF ANY KIND, EXPRESS 
% OR IMPLIED, INCLUDING BUT NOT LIMITED TO THE WARRANTIES OF MERCHANTABILITY, 
% FITNESS FOR A PARTICULAR PURPOSE AND NONINFRINGEMENT. IN NO EVENT SHALL 
% THE AUTHORS OR COPYRIGHT HOLDERS BE LIABLE FOR ANY CLAIM, DAMAGES OR OTHER 
% LIABILITY, WHETHER IN AN ACTION OF CONTRACT, TORT OR OTHERWISE, ARISING
% FROM, OUT OF OR IN CONNECTION WITH THE SOFTWARE OR THE USE OR OTHER 
% DEALINGS IN THE SOFTWARE.
\subsection{SERVER Open A Server Socket}
\subsubsection{Usage}
Opens up a server socket on the given portnumber and returns a handle
to the resulting socket:
\begin{verbatim}
  [handle,portnumber] = server(portnumber)
\end{verbatim}
The portnumber is an integer in the range $[1024-65535]$.  If you
do not wish to choose a portnumber, you can specify an argument of
$0$ (or omit the argument completely), and one will be selected 
automatically.  The resulting number is returned as the second argument.  
The \verb|handle| argument can then be passed to \verb|accept| to 
get a connection from the server.  Each call to \verb|accept|
will then return a handle to an active communication link over which 
arrays can be sent and received using \verb|send| and \verb|receive|.  Make sure
to call \verb|closeserver| on the handle when you are finished with it.
\subsubsection{Example}
Here are two examples.  In the first case, we use a fixed portnumber $36230$.
On session one, we execute the following:
\begin{verbatim}
--> a = server(36230)
a =
  <uint32>  - size: [1 1]
            1
--> b = accept(a)
\end{verbatim}
At this point the first session will suspend.  In a second session, we execute the following:
\begin{verbatim}
--> c = connect('127.0.0.1',36230)
c =
  <uint32>  - size: [1 1]
            1
\end{verbatim}
In the first session, we get a return from \verb|accept|:
\begin{verbatim}
--> a = server(36230)
a =
  <uint32>  - size: [1 1]
            1
--> b = accept(a)
b =
  <uint32>  - size: [1 1]
            1
\end{verbatim}
At this point a link is set up between the two sessions.  We can now \verb|send| and \verb|receive| arrays over this link.

The second approach uses an anonymous server.
\begin{verbatim}
--> [a,portnum] = server(0)
a =
  <uint32>  - size: [1 1]
            3
portnum =
  <uint32>  - size: [1 1]
        35258
\end{verbatim}
We now pass the number $35258$ to the \verb|connect| call in session two.
\begin{verbatim}
--> c = connect('127.0.0.1',35258)
c =
  <uint32>  - size: [1 1]
            2
\end{verbatim}
which results in the call to \verb|accept| in session one returning.
\begin{verbatim}
--> b = accept(a)
b =
  <uint32>  - size: [1 1]
            2
\end{verbatim}
