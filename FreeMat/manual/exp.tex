% Copyright (c) 2002, 2003 Samit Basu
%
% Permission is hereby granted, free of charge, to any person obtaining a 
% copy of this software and associated documentation files (the "Software"), 
% to deal in the Software without restriction, including without limitation 
% the rights to use, copy, modify, merge, publish, distribute, sublicense, 
% and/or sell copies of the Software, and to permit persons to whom the 
% Software is furnished to do so, subject to the following conditions:
%
% The above copyright notice and this permission notice shall be included 
% in all copies or substantial portions of the Software.
%
% THE SOFTWARE IS PROVIDED "AS IS", WITHOUT WARRANTY OF ANY KIND, EXPRESS 
% OR IMPLIED, INCLUDING BUT NOT LIMITED TO THE WARRANTIES OF MERCHANTABILITY, 
% FITNESS FOR A PARTICULAR PURPOSE AND NONINFRINGEMENT. IN NO EVENT SHALL 
% THE AUTHORS OR COPYRIGHT HOLDERS BE LIABLE FOR ANY CLAIM, DAMAGES OR OTHER 
% LIABILITY, WHETHER IN AN ACTION OF CONTRACT, TORT OR OTHERWISE, ARISING
% FROM, OUT OF OR IN CONNECTION WITH THE SOFTWARE OR THE USE OR OTHER 
% DEALINGS IN THE SOFTWARE.
\subsection{EXP Exponential Function}
\subsubsection{Usage}
Computes the $\exp$ function for its argument.  The general
syntax for its use is
\begin{verbatim}
   y = exp(x)
\end{verbatim}
where $x$ is an $n$-dimensional array of numerical type.
Integer types are promoted to the \verb|double| type prior to
calculation of the \verb|exp| function.  Output $y$ is of the
same size and type as the input $x$, (unless $x$ is an
integer, in which case $y$ is a \verb|double| type).
\subsubsection{Function Internals}
Mathematically, the $\exp$ function is defined for all real
valued arguments $x$ as
\[
  \exp x \equiv e^{x},
\]
where
\[
  e = \sum_{0}^{\infty} \frac{1}{k!}
\]
and is approximately $2.718281828459045$ (returned by the function 
\verb|e|).  For complex values
$z$, the famous Euler formula is used to calculate the 
exponential
\[
  e^{z} = e^{|z|} \left[ \cos \Re z + i \sin \Re z \right]
\]
\subsubsection{Example}
The following piece of code plots the real-valued $\exp$
function over the interval $[-1,1]$:
\begin{verbatim}
--> x = linspace(-1,1);
--> plot(x,exp(x))
\end{verbatim}
\doplot{width=8cm}{expplot1}
In the second example, we plot the unit circle in the complex plane $e^{\imath 2 \pi x}$ for $x \in [-1,1]$.
\begin{verbatim}
--> x = linspace(-1,1);
--> plot(exp(-i*x*2*pi))
\end{verbatim}
\doplot{width=8cm}{expplot2}
