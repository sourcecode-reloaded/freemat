% Copyright (c) 2002, 2003 Samit Basu
%
% Permission is hereby granted, free of charge, to any person obtaining a 
% copy of this software and associated documentation files (the "Software"), 
% to deal in the Software without restriction, including without limitation 
% the rights to use, copy, modify, merge, publish, distribute, sublicense, 
% and/or sell copies of the Software, and to permit persons to whom the 
% Software is furnished to do so, subject to the following conditions:
%
% The above copyright notice and this permission notice shall be included 
% in all copies or substantial portions of the Software.
%
% THE SOFTWARE IS PROVIDED "AS IS", WITHOUT WARRANTY OF ANY KIND, EXPRESS 
% OR IMPLIED, INCLUDING BUT NOT LIMITED TO THE WARRANTIES OF MERCHANTABILITY, 
% FITNESS FOR A PARTICULAR PURPOSE AND NONINFRINGEMENT. IN NO EVENT SHALL 
% THE AUTHORS OR COPYRIGHT HOLDERS BE LIABLE FOR ANY CLAIM, DAMAGES OR OTHER 
% LIABILITY, WHETHER IN AN ACTION OF CONTRACT, TORT OR OTHERWISE, ARISING
% FROM, OUT OF OR IN CONNECTION WITH THE SOFTWARE OR THE USE OR OTHER 
% DEALINGS IN THE SOFTWARE.
\subsection{ATAN Inverse Trigonometric Arctangent Function}
\subsubsection{Usage}
Computes the \verb|atan| function for its argument.  The general
syntax for its use is
\begin{verbatim}
  y = atan(x)
\end{verbatim}
where $x$ is an $n$-dimensional array of numerical type.
Integer types are promoted to the \verb|double| type prior to
calculation of the \verb|atan| function.  Output $y$ is of the
same size and type as the input $x$, (unless $x$ is an
integer, in which case $y$ is a \verb|double| type).  
\subsubsection{Function Internals}
Mathematically, the \verb|atan| function is defined for all 
arguments $x$ as
\[ 
   \mathrm{atan} x \equiv \frac{i}{2}\left(\log(1-i x) - \log(i x + 1)\right).
\]
For real valued variables $x$, the function is computed directly using 
the standard C library's numerical \verb|atan| function. For both 
real and complex arguments $x$, note that generally
$\mathrm{atan}(\tan(x)) \neq x$, due to the periodicity of $\tan(x)$.
\subsubsection{Example}
The following code demonstates the \verb|atan| function over the range 
$[-1,1]$.
\begin{verbatim}
--> t = linspace(-1,1);
--> plot(t,atan(t))
\end{verbatim}
\doplot{width=8cm}{atanplot}
