% Copyright (c) 2002, 2003 Samit Basu
%
% Permission is hereby granted, free of charge, to any person obtaining a 
% copy of this software and associated documentation files (the "Software"), 
% to deal in the Software without restriction, including without limitation 
% the rights to use, copy, modify, merge, publish, distribute, sublicense, 
% and/or sell copies of the Software, and to permit persons to whom the 
% Software is furnished to do so, subject to the following conditions:
%
% The above copyright notice and this permission notice shall be included 
% in all copies or substantial portions of the Software.
%
% THE SOFTWARE IS PROVIDED "AS IS", WITHOUT WARRANTY OF ANY KIND, EXPRESS 
% OR IMPLIED, INCLUDING BUT NOT LIMITED TO THE WARRANTIES OF MERCHANTABILITY, 
% FITNESS FOR A PARTICULAR PURPOSE AND NONINFRINGEMENT. IN NO EVENT SHALL 
% THE AUTHORS OR COPYRIGHT HOLDERS BE LIABLE FOR ANY CLAIM, DAMAGES OR OTHER 
% LIABILITY, WHETHER IN AN ACTION OF CONTRACT, TORT OR OTHERWISE, ARISING
% FROM, OUT OF OR IN CONNECTION WITH THE SOFTWARE OR THE USE OR OTHER 
% DEALINGS IN THE SOFTWARE.
\subsection{FPRINTF Formated File Output Function (C-Style)}
\subsubsection{Usage}
Prints values to a file.  The general syntax for its use is
\begin{verbatim}
  fprintf(fp,format,a1,a2,...).
\end{verbatim}
Here \verb|format| is the format string, which is a string that
controls the format of the output.  The values of the variables
$a_i$ are substituted into the output as required.  It is
an error if there are not enough variables to satisfy the format
string.  Note that this \verb|fprintf| command is not vectorized!  Each
variable must be a scalar.  The value \verb|fp| is the file handle.  For
more details on the format string, see \verb|printf|.  Note also that \verb|fprintf| to the file handle $1$ is effectively equivalent to \verb|printf|.
\subsubsection{Examples}
A number of examples are present in the Examples section of the \verb|printf| command.
