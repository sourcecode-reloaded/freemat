% Copyright (c) 2002, 2003 Samit Basu
%
% Permission is hereby granted, free of charge, to any person obtaining a 
% copy of this software and associated documentation files (the "Software"), 
% to deal in the Software without restriction, including without limitation 
% the rights to use, copy, modify, merge, publish, distribute, sublicense, 
% and/or sell copies of the Software, and to permit persons to whom the 
% Software is furnished to do so, subject to the following conditions:
%
% The above copyright notice and this permission notice shall be included 
% in all copies or substantial portions of the Software.
%
% THE SOFTWARE IS PROVIDED "AS IS", WITHOUT WARRANTY OF ANY KIND, EXPRESS 
% OR IMPLIED, INCLUDING BUT NOT LIMITED TO THE WARRANTIES OF MERCHANTABILITY, 
% FITNESS FOR A PARTICULAR PURPOSE AND NONINFRINGEMENT. IN NO EVENT SHALL 
% THE AUTHORS OR COPYRIGHT HOLDERS BE LIABLE FOR ANY CLAIM, DAMAGES OR OTHER 
% LIABILITY, WHETHER IN AN ACTION OF CONTRACT, TORT OR OTHERWISE, ARISING
% FROM, OUT OF OR IN CONNECTION WITH THE SOFTWARE OR THE USE OR OTHER 
% DEALINGS IN THE SOFTWARE.
\subsection{TYPEOF Determine the Type of an Argument}
\subsubsection{Usage}
Returns a string describing the type of an array.  The syntax for its use is
\begin{verbatim}
   y = typeof(x),
\end{verbatim}
The returned string is one of
\begin{itemize}
\item \verb|'cell'| for cell-arrays
\item \verb|'struct'| for structure-arrays
\item \verb|'logical'| for logical arrays
\item \verb|'uint8'| for unsigned 8-bit integers
\item \verb|'int8'| for signed 8-bit integers
\item \verb|'uint16'| for unsigned 16-bit integers
\item \verb|'int16'| for signed 16-bit integers
\item \verb|'uint32'| for unsigned 32-bit integers
\item \verb|'int32'| for signed 32-bit integers
\item \verb|'float'| for 32-bit floating point numbers
\item \verb|'double'| for 64-bit floating point numbers
\item \verb|'complex'| for complex floating point numbers with 32-bits per field
\item \verb|'dcomplex'| for complex floating point numbers with 64-bits per field
\item \verb|'string'| for string arrays
\end{itemize}
\subsubsection{Example}
The following piece of code demonstrates the output of the \verb|typeof| command for each possible type.  The first example is with a simple cell array.
\begin{verbatim}
--> typeof({1})
ans =
  <string>  - size: [1 4]
 cell
\end{verbatim}
The next example uses the \verb|struct| constructor to make a simple scalar struct.
\begin{verbatim}
--> typeof(struct('foo',3))
ans =
  <string>  - size: [1 6]
 struct
\end{verbatim}
The next example uses a comparison between two scalar integers to generate a scalar logical type.
\begin{verbatim}
--> typeof(3>5)
ans =
  <string>  - size: [1 7]
 logical
\end{verbatim}
For the smaller integers, and the 32-bit unsigned integer types, the typecast operations are used to generate the arguments.
\begin{verbatim}
--> typeof(uint8(3))
ans =
  <string>  - size: [1 5]
 uint8
--> typeof(int8(8))
ans =
  <string>  - size: [1 4]
 int8
--> typeof(uint16(3))
ans =
  <string>  - size: [1 6]
 uint16
--> typeof(int16(8))
ans =
  <string>  - size: [1 5]
 int16
--> typeof(uint32(3))
ans =
  <string>  - size: [1 6]
 uint32
\end{verbatim}
The 32-bit signed integer type is the default for integer arguments.
\begin{verbatim}
--> typeof(-3)
ans =
  <string>  - size: [1 5]
 int32
--> typeof(8)
ans =
  <string>  - size: [1 5]
 int32
\end{verbatim}
Float, double, complex and double-precision complex types can be created using the suffixes.
\begin{verbatim}
--> typeof(1.0f)
ans =
  <string>  - size: [1 5]
 float
--> typeof(1.0D)
ans =
  <string>  - size: [1 6]
 double
--> typeof(1.0f+i)
ans =
  <string>  - size: [1 7]
 complex
--> typeof(1.0D+2.0D*i)
ans =
  <string>  - size: [1 8]
 dcomplex
-->
\end{verbatim}
