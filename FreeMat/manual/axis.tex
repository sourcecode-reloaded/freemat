% Copyright (c) 2002, 2003 Samit Basu
%
% Permission is hereby granted, free of charge, to any person obtaining a 
% copy of this software and associated documentation files (the "Software"), 
% to deal in the Software without restriction, including without limitation 
% the rights to use, copy, modify, merge, publish, distribute, sublicense, 
% and/or sell copies of the Software, and to permit persons to whom the 
% Software is furnished to do so, subject to the following conditions:
%
% The above copyright notice and this permission notice shall be included 
% in all copies or substantial portions of the Software.
%
% THE SOFTWARE IS PROVIDED "AS IS", WITHOUT WARRANTY OF ANY KIND, EXPRESS 
% OR IMPLIED, INCLUDING BUT NOT LIMITED TO THE WARRANTIES OF MERCHANTABILITY, 
% FITNESS FOR A PARTICULAR PURPOSE AND NONINFRINGEMENT. IN NO EVENT SHALL 
% THE AUTHORS OR COPYRIGHT HOLDERS BE LIABLE FOR ANY CLAIM, DAMAGES OR OTHER 
% LIABILITY, WHETHER IN AN ACTION OF CONTRACT, TORT OR OTHERWISE, ARISING
% FROM, OUT OF OR IN CONNECTION WITH THE SOFTWARE OR THE USE OR OTHER 
% DEALINGS IN THE SOFTWARE.
\subsection{AXIS Plot Axis Set/Get Function}
\subsubsection{Usage}
Changes the axis configuration for the currently active plot,
or returns the currently active limits for the axis,  The
general syntax for its use is either
\begin{verbatim}
   [x1,x2,y1,y2] = axis
\end{verbatim}
to get the current axis limits, or
\begin{verbatim}
   axis([x1,x2,y1,y2])
\end{verbatim}
to set the axis limits.  There are also two additional \verb|axis|
commands:
\begin{verbatim}
   axis('tight')
\end{verbatim}
which sets the axis boundaries to be tight as possible, and
\begin{verbatim}
   axis('auto')
\end{verbatim}
which uses a heuristic algorithm to choose a ``reasonable'' set of
axis limits.
\subsubsection{Function Internals}
The \verb|axis| command is used to change the ranges of the \verb|x|
and \verb|y| axes on the current plot.  This permits ``zooming'' of
plots.  By default, when a \verb|plot| command is issued, a heuristic
algorithm adjusts the ranges on the \verb|x| and \verb|y| axes so that
the increments on the axes are ``reasonable'' values.  It also adjusts
the start and stop values on each axis (by enlarging the range and
domain of the plot).  You can reset a plot to this state using the 
\verb|'auto'| argument to the axis.  

Another option is to choose the axes so that they tightly fit the
domain and range of the current datasets.  This is accomplished
using the \verb|'tight'| argument to the axis command.  It will
set the axes to $[x_{\min},x_{\max},y_{\min},y_{\max}]$, where
$x_{\min}$ is the minimum $x$ value over all datasets in the current
plot series, etc.

You can also retrieve the current ranges of the plot axes by issuing
an \verb|axis| command with no arguments.
\subsubsection{Example}
We start by plotting a sinusoid of amplitude $\sqrt{2}$ over the 
range $[-\pi,pi]$, which is not a ``nice'' range, and thus the
auto axis heuristic shrinks the plot to make the range nicer.
\begin{verbatim}
--> x = linspace(-pi,pi);
--> y = sqrt(2)*sin(3*x);
--> plot(x,y,'r-');
--> grid on
\end{verbatim}

\doplot{width=8cm}{axis1}

Suppose we now want to make the axis fit the plot exactly.  We can issue an
\verb|axis('tight')| command, which results in the following plot.
\begin{verbatim}
--> axis tight
\end{verbatim}

\doplot{width=8cm}{axis2}

We can now use the \verb|axis| command to retrieve the current axis values.  By
modifying only the first two entries ($x_{\min}$ and $x_{\max}$), we can
zoom in on one period of the sinusoid.
\begin{verbatim}
--> a = axis
a = 
  <double>  - size: [1 4]
 
Columns 1 to 2
   -3.141592653589793         3.141592653589793      
 
Columns 3 to 4
   -1.412611738084137         1.412611738084136      
--> a(1) = -pi/3; a(2) = pi/3;
--> axis(a);
\end{verbatim}

\doplot{width=8cm}{axis3}

Finally, we can restore the original plot by issuing an \verb|axis('auto')|
command.
\begin{verbatim}
--> axis auto
\end{verbatim}

\doplot{width=8cm}{axis4}
