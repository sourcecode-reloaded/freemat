% Copyright (c) 2002, 2003 Samit Basu
%
% Permission is hereby granted, free of charge, to any person obtaining a 
% copy of this software and associated documentation files (the "Software"), 
% to deal in the Software without restriction, including without limitation 
% the rights to use, copy, modify, merge, publish, distribute, sublicense, 
% and/or sell copies of the Software, and to permit persons to whom the 
% Software is furnished to do so, subject to the following conditions:
%
% The above copyright notice and this permission notice shall be included 
% in all copies or substantial portions of the Software.
%
% THE SOFTWARE IS PROVIDED "AS IS", WITHOUT WARRANTY OF ANY KIND, EXPRESS 
% OR IMPLIED, INCLUDING BUT NOT LIMITED TO THE WARRANTIES OF MERCHANTABILITY, 
% FITNESS FOR A PARTICULAR PURPOSE AND NONINFRINGEMENT. IN NO EVENT SHALL 
% THE AUTHORS OR COPYRIGHT HOLDERS BE LIABLE FOR ANY CLAIM, DAMAGES OR OTHER 
% LIABILITY, WHETHER IN AN ACTION OF CONTRACT, TORT OR OTHERWISE, ARISING
% FROM, OUT OF OR IN CONNECTION WITH THE SOFTWARE OR THE USE OR OTHER 
% DEALINGS IN THE SOFTWARE.
\subsection{MIN Minimum Function}
\subsubsection{Usage}
Computes the minimum of an array along a given dimension, or alternately, 
computes two arrays (entry-wise) and keeps the smaller value for each array.
As a result, the \verb|min| function has a number of syntaxes.  The first
one computes the minimum of an array along a given dimension.
The first general syntax for its use is either
\begin{verbatim}
   [y,n] = min(x,[],d)
\end{verbatim}
where $x$ is a multidimensional array of numerical type, in which case the
output $y$ is the minimum of $x$ along dimension $d$.  
The second argument $n$ is the index that results in the minimum.
In the event that multiple minima are present with the same value,
the index of the first minimum is used. 
The second general syntax for the use of the \verb|min| function is
\begin{verbatim}
   [y,n] = min(x)
\end{verbatim} 
In this case, the minimum is taken along the first non-singleton 
dimension of $x$.  For complex data types,
the minimum is based on the magnitude of the numbers.  NaN's are
ignored in the calculations.
The third general syntax for the use of the \verb|min| function is as 
a comparison function for pairs of arrays.  Here, the general syntax is
\begin{verbatim}
   y = min(x,z)
\end{verbatim}
where $x$ and $z$ are either both numerical arrays of the same dimensions,
or one of the two is a scalar.  In the first case, the output is the 
same size as both arrays, and is defined elementwise by the smaller of the
two arrays.  In the second case, the output is defined elementwise by the 
smaller of the array entries and the scalar.

\subsubsection{Function Internals}
In the general version of the \verb|min| function which is applied to
a single array (using the \verb|min(x,[],d)| or \verb|min(x)| syntaxes),
The output is computed via
\[
y(m_1,\ldots,m_{d-1},1,m_{d+1},\ldots,m_{p}) = 
\min_{k} x(m_1,\ldots,m_{d-1},k,m_{d+1},\ldots,m_{p}),
\]
and the output array $n$ of indices is calculated via
\[
n(m_1,\ldots,m_{d-1},1,m_{d+1},\ldots,m_{p}) = \arg
\min_{k} x(m_1,\ldots,m_{d-1},k,m_{d+1},\ldots,m_{p})
\]
In the two-array version (\verb|min(x,z)|), the single output is computed as
\[
  y(m_1,\ldots,m_{d-1},1,m_{d+1},\ldots,m_{p}) = 
\begin{cases}
  x(m_1,\ldots,m_{d-1},k,m_{d+1},\ldots,m_{p}) & x(\cdots) \leq z(\cdots) \\
  z(m_1,\ldots,m_{d-1},k,m_{d+1},\ldots,m_{p}) & z(\cdots) < x(\cdots).
\end{cases}
\]

\subsubsection{Example}
The following piece of code demonstrates various uses of the minimum
function.  We start with the one-array version.
\begin{verbatim}
--> A = [5,1,3;3,2,1;0,3,1]
A =
  <int32>  - size: [3 3]
  
Columns 1 to 3
             5              1              3
             3              2              1
             0              3              1
\end{verbatim}
We first take the minimum along the columns, resulting in a row vector.
\begin{verbatim}
--> min(A)
ans =
  <int32>  - size: [1 3]
  
Columns 1 to 3
             0              1              1
\end{verbatim}
Next, we take the minimum along the rows, resulting in a column vector.
\begin{verbatim}
--> min(A,[],2)
ans =
  <int32>  - size: [3 1]
  
Columns 1 to 1
             1
             1
             0
\end{verbatim}
When the dimension argument is not supplied, \verb|min| acts along the first non-singular dimension.  For a row vector, this is the column direction:
\begin{verbatim}
--> min([5,3,2,9])
ans =
  <int32>  - size: [1 1]
             2
\end{verbatim}

For the two-argument version, we can compute the smaller of two arrays,
as in this example:
\begin{verbatim}
--> a = int8(100*randn(4))
a =
  <int8>  - size: [4 4]

Columns 1 to 4
   -3    59    -5   110
  -14    70   -16    -3
   69   -93     1   118
  -23     0    16   -74
--> b = int8(100*randn(4))
b =
  <int8>  - size: [4 4]

Columns 1 to 4
   64   -51    74    84
  -40   -62   -84  -126
 -102   -12    43   -54
   69    50   -56    29
--> min(a,b)
ans =
  <int8>  - size: [4 4]

Columns 1 to 4
   -3   -51    -5    84
  -40   -62   -84  -126
 -102   -93     1   -54
  -23     0   -56   -74
\end{verbatim}
Or alternately, we can compare an array with a scalar
\begin{verbatim}
--> a = randn(2)
a =
  <double>  - size: [2 2]

Columns 1 to 2
    0.153277151376321        -0.844375672481831
    0.0125905206377705       -0.0776334436133293
--> min(a,0)
ans =
  <double>  - size: [2 2]

Columns 1 to 2
    0.000000000000000        -0.844375672481831
    0.000000000000000        -0.0776334436133293
\end{verbatim}
