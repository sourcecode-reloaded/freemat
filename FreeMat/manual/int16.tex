% Copyright (c) 2002, 2003 Samit Basu
%
% Permission is hereby granted, free of charge, to any person obtaining a 
% copy of this software and associated documentation files (the "Software"), 
% to deal in the Software without restriction, including without limitation 
% the rights to use, copy, modify, merge, publish, distribute, sublicense, 
% and/or sell copies of the Software, and to permit persons to whom the 
% Software is furnished to do so, subject to the following conditions:
%
% The above copyright notice and this permission notice shall be included 
% in all copies or substantial portions of the Software.
%
% THE SOFTWARE IS PROVIDED "AS IS", WITHOUT WARRANTY OF ANY KIND, EXPRESS 
% OR IMPLIED, INCLUDING BUT NOT LIMITED TO THE WARRANTIES OF MERCHANTABILITY, 
% FITNESS FOR A PARTICULAR PURPOSE AND NONINFRINGEMENT. IN NO EVENT SHALL 
% THE AUTHORS OR COPYRIGHT HOLDERS BE LIABLE FOR ANY CLAIM, DAMAGES OR OTHER 
% LIABILITY, WHETHER IN AN ACTION OF CONTRACT, TORT OR OTHERWISE, ARISING
% FROM, OUT OF OR IN CONNECTION WITH THE SOFTWARE OR THE USE OR OTHER 
% DEALINGS IN THE SOFTWARE.
\subsection{INT16 Convert to Signed 16-bit Integer}
\subsubsection{Usage}
Converts the argument to an signed 16-bit Integer.  The syntax
for its use is
\begin{verbatim}
   y = int16(x)
\end{verbatim}
where $x$ is an $n$-dimensional numerical array.  Conversion
follows the general C rules (e.g., if $x$ is outside the normal
range for a signed 16-bit integer of $[-32768,32767]$, the least significant
16 bits of $x$ are used after conversion to a signed integer).  Note that
both \verb|NaN| and \verb|Inf| both map to 0.
\subsubsection{Example}
The following piece of code demonstrates several uses of \verb|int16|.  First, the routine uses
\begin{verbatim}
--> int16(100)
ans =
  <int16>  - size: [1 1]
  100
--> int16(-100)
ans =
  <int16>  - size: [1 1]
 -100
\end{verbatim}
In the next example, an integer outside the range  of the type is passed in.  The result is the 16 least significant bits of the argument.
\begin{verbatim}
--> int16(40000)
ans =
  <int16>  - size: [1 1]
 -25536
\end{verbatim}
In the next example, a positive double precision argument is passed in.  The result is the signed integer that is closest to the argument.
\begin{verbatim}
--> int16(pi)
ans =
  <int16>  - size: [1 1]
   3
\end{verbatim}
In the next example, a complex argument is passed in.  The result is the signed integer that is closest to the real part of the argument.
\begin{verbatim}
--> int16(5+2*i)
ans =
  <int16>  - size: [1 1]
   5
\end{verbatim}
In the next example, a string argument is passed in.  The string argument is converted into an integer array corresponding to the ASCII values of each character.
\begin{verbatim}
--> int16('helo')
ans =
  <int16>  - size: [1 4]
  
Columns 1 to 4
 104  101  108  111
\end{verbatim}
In the last example, a cell-array is passed in.  For cell-arrays and structure arrays, the result is an error.
\begin{verbatim}
--> int16({4})
Error: Cannot convert cell-arrays to any other type.
\end{verbatim}
