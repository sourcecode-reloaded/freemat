% Copyright (c) 2002, 2003 Samit Basu
%
% Permission is hereby granted, free of charge, to any person obtaining a 
% copy of this software and associated documentation files (the "Software"), 
% to deal in the Software without restriction, including without limitation 
% the rights to use, copy, modify, merge, publish, distribute, sublicense, 
% and/or sell copies of the Software, and to permit persons to whom the 
% Software is furnished to do so, subject to the following conditions:
%
% The above copyright notice and this permission notice shall be included 
% in all copies or substantial portions of the Software.
%
% THE SOFTWARE IS PROVIDED "AS IS", WITHOUT WARRANTY OF ANY KIND, EXPRESS 
% OR IMPLIED, INCLUDING BUT NOT LIMITED TO THE WARRANTIES OF MERCHANTABILITY, 
% FITNESS FOR A PARTICULAR PURPOSE AND NONINFRINGEMENT. IN NO EVENT SHALL 
% THE AUTHORS OR COPYRIGHT HOLDERS BE LIABLE FOR ANY CLAIM, DAMAGES OR OTHER 
% LIABILITY, WHETHER IN AN ACTION OF CONTRACT, TORT OR OTHERWISE, ARISING
% FROM, OUT OF OR IN CONNECTION WITH THE SOFTWARE OR THE USE OR OTHER 
% DEALINGS IN THE SOFTWARE.
\subsection{MEAN Mean Function}
\subsubsection{Usage}
Computes the mean of an array along a given dimension.  The general
syntax for its use is
\begin{verbatim}
  y = mean(x,{d})
\end{verbatim}
where $x$ is an $n$-dimensions array of numerical type.
The output is of the same numerical type as the input, except for 
integer types, which are promoted to double precision prior to taking
the mean.  The argument $d$ is optional, and denotes the dimension 
along which to take the mean.  The output is computed via
\[
y(m_1,\ldots,m_{d-1},1,m_{d+1},\ldots,m_{p}) = \frac{1}{N}
\sum_{k} x(m_1,\ldots,m_{d-1},k,m_{d+1},\ldots,m_{p})
\]
where $N$ is the length of the array $x$ in dimension $d$.
If $d$ is omitted, then the mean is taken along the 
first non-singleton dimension of $x$. 
\subsubsection{Example}
The following piece of code demonstrates various uses of the mean
function
\begin{verbatim}
--> A = [5,1,3;3,2,1;0,3,1]
A =
  <int32>  - size: [3 3]
  
Columns 1 to 3
             5              1              3
             3              2              1
             0              3              1
\end{verbatim}
We start by calling \verb|mean| without a dimension argument, in which case it defaults to the first nonsingular dimension (in this case, along the columns or $d = 1$).
\begin{verbatim}
--> mean(A)
ans =
  <double>  - size: [1 3]
  
Columns 1 to 2
    2.666666666666667         2.000000000000000
  
Columns 3 to 3
    1.666666666666667
\end{verbatim}
Next, we take the mean along the rows.
\begin{verbatim}
--> mean(A,2)
ans =
  <double>  - size: [3 1]
  
Columns 1 to 1
    3.000000000000000
    2.000000000000000
    1.333333333333333
\end{verbatim}
