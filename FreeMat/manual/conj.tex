% Copyright (c) 2002, 2003 Samit Basu
%
% Permission is hereby granted, free of charge, to any person obtaining a 
% copy of this software and associated documentation files (the "Software"), 
% to deal in the Software without restriction, including without limitation 
% the rights to use, copy, modify, merge, publish, distribute, sublicense, 
% and/or sell copies of the Software, and to permit persons to whom the 
% Software is furnished to do so, subject to the following conditions:
%
% The above copyright notice and this permission notice shall be included 
% in all copies or substantial portions of the Software.
%
% THE SOFTWARE IS PROVIDED "AS IS", WITHOUT WARRANTY OF ANY KIND, EXPRESS 
% OR IMPLIED, INCLUDING BUT NOT LIMITED TO THE WARRANTIES OF MERCHANTABILITY, 
% FITNESS FOR A PARTICULAR PURPOSE AND NONINFRINGEMENT. IN NO EVENT SHALL 
% THE AUTHORS OR COPYRIGHT HOLDERS BE LIABLE FOR ANY CLAIM, DAMAGES OR OTHER 
% LIABILITY, WHETHER IN AN ACTION OF CONTRACT, TORT OR OTHERWISE, ARISING
% FROM, OUT OF OR IN CONNECTION WITH THE SOFTWARE OR THE USE OR OTHER 
% DEALINGS IN THE SOFTWARE.
\subsection{CONJ Conjugate Function}
\subsubsection{Usage}
Returns the complex conjugate of the input array for all elements.  The 
general syntax for its use is
\begin{verbatim}
   y = conj(x)
\end{verbatim}
where $x$ is an $n$-dimensional array of numerical type.  The output 
is the same numerical type as the input.  The \verb|conj| function does
nothing to real and integer types.
\subsubsection{Example}
The following demonstrates the complex conjugate applied to a complex scalar.
\begin{verbatim}
--> conj(3+4*i)
ans =
  <complex>  - size: [1 1]
    4.0000000         3.0000000     i
\end{verbatim}
The \verb|conj| function has no effect on real arguments:
\begin{verbatim}
--> conj([2,3,4])
ans =
  <int32>  - size: [1 3]
  
Columns 1 to 3
             2              3              4
\end{verbatim}
For a double-precision complex array,
\begin{verbatim}
--> conj([2.0+3.0*i,i])
ans =
  <dcomplex>  - size: [1 2]
  
Columns 1 to 1
    3.000000000000000        2.000000000000000    i
  
Columns 2 to 2
    1.000000000000000        0.000000000000000    i
\end{verbatim}
