% Copyright (c) 2002, 2003 Samit Basu
%
% Permission is hereby granted, free of charge, to any person obtaining a 
% copy of this software and associated documentation files (the "Software"), 
% to deal in the Software without restriction, including without limitation 
% the rights to use, copy, modify, merge, publish, distribute, sublicense, 
% and/or sell copies of the Software, and to permit persons to whom the 
% Software is furnished to do so, subject to the following conditions:
%
% The above copyright notice and this permission notice shall be included 
% in all copies or substantial portions of the Software.
%
% THE SOFTWARE IS PROVIDED "AS IS", WITHOUT WARRANTY OF ANY KIND, EXPRESS 
% OR IMPLIED, INCLUDING BUT NOT LIMITED TO THE WARRANTIES OF MERCHANTABILITY, 
% FITNESS FOR A PARTICULAR PURPOSE AND NONINFRINGEMENT. IN NO EVENT SHALL 
% THE AUTHORS OR COPYRIGHT HOLDERS BE LIABLE FOR ANY CLAIM, DAMAGES OR OTHER 
% LIABILITY, WHETHER IN AN ACTION OF CONTRACT, TORT OR OTHERWISE, ARISING
% FROM, OUT OF OR IN CONNECTION WITH THE SOFTWARE OR THE USE OR OTHER 
% DEALINGS IN THE SOFTWARE.
\subsection{LOGICAL OPERATORS Logical Array Operators}
\subsubsection{Usage}
There are three Boolean operators available in FreeMat.  The syntax for their use is:
\begin{verbatim}
  y = ~x
  y = a & b
  y = a | b
\end{verbatim}
where $x$, $a$ and $b$ are \verb|logical| arrays.  The operators are
\begin{itemize}
\item NOT ($\sim$) - output $y$ is true if the corresponding element of $x$ is false, and ouput $y$ is false if the corresponding element of $x$ is true.
\item OR ($|$) - output $y$ is true if corresponding element of $a$ is true or if corresponding element of $b$ is true (or if both are true).
\item AND ($\&$) - output $y$ is true only if both the corresponding elements of $a$ and $b$ are both true.
\end{itemize}
The binary operators AND and OR can take scalar arguments as well as vector arguments, in which case, the scalar is operated on with each element of the vector.
\subsubsection{Examples}
Some simple examples of logical operators.  Suppose we want to calculate the exclusive-or (XOR) of two vectors of logical variables.  First, we create a pair of vectors to perform the XOR operation on:
\begin{verbatim}
--> a = (randn(1,6)>0)
a =
  <logical>  - size: [1 6]
  
Columns 1 to 6
 0  0  0  0  1  1
--> b = (randn(1,6)>0)
b =
  <logical>  - size: [1 6]
  
Columns 1 to 6
 0  0  1  0  1  0
\end{verbatim}
Next, we can compute the OR of $a$ and $b$:
\begin{verbatim}
--> c = a | b
c =
  <logical>  - size: [1 6]
  
Columns 1 to 6
 0  0  1  0  1  1
\end{verbatim}
However, the XOR and OR operations differ on the fifth entry - the XOR would be false, since it is true if and only if exactly one of the two inputs is true.  To isolate this case, we can AND the two vectors, to find exactly those entries that appear as true in both $a$ and $b$:
\begin{verbatim}
--> d = a & b
d =
  <logical>  - size: [1 6]
  
Columns 1 to 6
 0  0  0  0  1  0
\end{verbatim}
At this point, we can modify the contents of $c$ in two ways -- the Boolean way is to AND $\sim d$ with $c$, like so
\begin{verbatim}
--> xor = c & (~d)
xor =
  <logical>  - size: [1 6]
  
Columns 1 to 6
 0  0  1  0  0  1
\end{verbatim}
The other way to do this is simply force $c(d) = 0$, which uses the logical indexing mode of FreeMat (see the chapter on indexing for more details).  This, however, will cause $c$ to become an \verb|int32| type, as opposed to a logical type.
\begin{verbatim}
--> c(d) = 0
c =
  <int32>  - size: [1 6]
  
Columns 1 to 5
             0              0              1              0              0
  
Columns 6 to 6
             1
\end{verbatim}
