% Copyright (c) 2002, 2003 Samit Basu
%
% Permission is hereby granted, free of charge, to any person obtaining a 
% copy of this software and associated documentation files (the "Software"), 
% to deal in the Software without restriction, including without limitation 
% the rights to use, copy, modify, merge, publish, distribute, sublicense, 
% and/or sell copies of the Software, and to permit persons to whom the 
% Software is furnished to do so, subject to the following conditions:
%
% The above copyright notice and this permission notice shall be included 
% in all copies or substantial portions of the Software.
%
% THE SOFTWARE IS PROVIDED "AS IS", WITHOUT WARRANTY OF ANY KIND, EXPRESS 
% OR IMPLIED, INCLUDING BUT NOT LIMITED TO THE WARRANTIES OF MERCHANTABILITY, 
% FITNESS FOR A PARTICULAR PURPOSE AND NONINFRINGEMENT. IN NO EVENT SHALL 
% THE AUTHORS OR COPYRIGHT HOLDERS BE LIABLE FOR ANY CLAIM, DAMAGES OR OTHER 
% LIABILITY, WHETHER IN AN ACTION OF CONTRACT, TORT OR OTHERWISE, ARISING
% FROM, OUT OF OR IN CONNECTION WITH THE SOFTWARE OR THE USE OR OTHER 
% DEALINGS IN THE SOFTWARE.
\subsection{CD Change Working Directory Function}
\subsubsection{Usage}
Changes the current working directory to the one specified as the argument.  The general syntax for its use is
\begin{verbatim}
  cd('dirname')
\end{verbatim}
but this can also be expressed as
\begin{verbatim}
  cd 'dirname'
\end{verbatim}
or 
\begin{verbatim}
  cd dirname
\end{verbatim}
Examples of all three usages are given below.
Generally speaking, \verb|dirname| is any string that would be accepted by the underlying OS as a valid directory name.  For example, on most systems, \verb|'.'| refers to the current directory, and \verb|'..'| refers to the parent directory.  Also, depending on the OS, it may be necessary to ``escape'' the directory seperators.  In particular, if directories are seperated with the backwards-slash character \verb|'\\'|, then the path specification must use double-slashes \verb|'\\\\'|. \emph{Note: to get file-name completion to work at this time, you must use one of the first two forms of the command.}
\subsubsection{Example}
The \verb|pwd| command returns the current directory location.  First, we use the simplest form of the \verb|cd| command, in which the directory name argument is given unquoted.
\begin{verbatim}
--> pwd
ans =
  <string>  - size: [1 28]
 /home/basu/FreeMat/build/src
--> cd ..
--> pwd
ans =
  <string>  - size: [1 24]
 /home/basu/FreeMat/build
\end{verbatim}
Next, we use the ``traditional'' form of the function call, using both the parenthesis and the quoted string.
\begin{verbatim}
--> cd('/home/basu/FreeMat')
--> pwd
ans =
  <string>  - size: [1 18]
 /home/basu/FreeMat
\end{verbatim}
In the third version, we use only the quoted string argument without parenthesis.  We also use the filename-completion feature (usually by pressing the \verb|TAB| key) to get a list of possible completions.
\begin{verbatim}
--> cd 'build/
Makefile          doc/              libtecla/         wxBase-2.4.0/
config.log        libCore/          manual/           wxWindows-2.4.0/
config.status     libFFTPack/       src/
--> cd 'build/src'
--> pwd
ans =
  <string>  - size: [1 28]
 /home/basu/FreeMat/build/src
\end{verbatim}
