% Copyright (c) 2002, 2003 Samit Basu
%
% Permission is hereby granted, free of charge, to any person obtaining a 
% copy of this software and associated documentation files (the "Software"), 
% to deal in the Software without restriction, including without limitation 
% the rights to use, copy, modify, merge, publish, distribute, sublicense, 
% and/or sell copies of the Software, and to permit persons to whom the 
% Software is furnished to do so, subject to the following conditions:
%
% The above copyright notice and this permission notice shall be included 
% in all copies or substantial portions of the Software.
%
% THE SOFTWARE IS PROVIDED "AS IS", WITHOUT WARRANTY OF ANY KIND, EXPRESS 
% OR IMPLIED, INCLUDING BUT NOT LIMITED TO THE WARRANTIES OF MERCHANTABILITY, 
% FITNESS FOR A PARTICULAR PURPOSE AND NONINFRINGEMENT. IN NO EVENT SHALL 
% THE AUTHORS OR COPYRIGHT HOLDERS BE LIABLE FOR ANY CLAIM, DAMAGES OR OTHER 
% LIABILITY, WHETHER IN AN ACTION OF CONTRACT, TORT OR OTHERWISE, ARISING
% FROM, OUT OF OR IN CONNECTION WITH THE SOFTWARE OR THE USE OR OTHER 
% DEALINGS IN THE SOFTWARE.
\subsection{FFT (Inverse) Fast Fourier Transform Function}
\subsubsection{Usage}
Computes the Discrete Fourier Transform (DFT) of a vector using the
Fast Fourier Transform technique.  The general syntax for its use is
\begin{verbatim}
  y = fft(x,n,d)
\end{verbatim}
where $x$ is an $n$-dimensional array of numerical type.
Integer types are promoted to the \verb double type prior to 
calculation of the DFT. The argument $n$ is the length of the
FFT, and $d$ is the dimension along which to take the DFT.  If
$n$ is larger than the length of $x$ along dimension $d$,
then $x$ is zero-padded (by appending zeros) prior to calculation
of the DFT.  If $n$ is smaller than the length of $x$  along
the given dimension, then $x$ is truncated (by removing elements
at the end) to length $n$.  Then, the output is computed via
\[
y(m_1,\ldots,m_{d-1},l,m_{d+1},\ldots,m_{p}) = 
\sum_{k=1}^{n} x(m_1,\ldots,m_{d-1},k,m_{d+1},\ldots,m_{p})
e^{-\frac{2\pi(k-1)l}{n}}.
\]
If $d$ is omitted, then the DFT is taken along the first 
non-singleton dimension of $x$.  If $n$ is omitted, then
the DFT length is chosen to match of the length of $x$ along
dimension $d$.  

For the inverse DFT, the calculation is similar, and the arguments
have the same meanings as the DFT:
\[
y(m_1,\ldots,m_{d-1},l,m_{d+1},\ldots,m_{p}) = 
\frac{1}{n} \sum_{k=1}^{n} x(m_1,\ldots,m_{d-1},k,m_{d+1},\ldots,m_{p})
e^{\frac{2\pi(k-1)l}{n}}.
\]
 \subsubsection{Computation}
 The FFT is computed using the FFTPack library, available from 
 netlib at \href{http://www.netlib.org}.  Generally speaking, the 
 computational cost for a FFT is (in worst case) $O(n^2)$.
 However, if $n$ is composite, and can be factored as
 \[
 n = \prod_{k=1}^{p} m_k,
 \]
 then the DFT can be computed in 
 \[
 O(n \sum_{k=1}^{p} m_k)
 \]
 operations.  If $n$ is a power of 2, then the FFT can be
 calculated in $O(n \log_2 n)$.  The calculations for the
 inverse FFT are identical.
 \subsubsection{Example}
 The following piece of code plots the FFT for a sinusoidal signal:
\begin{verbatim}
  --> t = linspace(0,2*pi,128);
  --> x = cos(15*t);
  --> y = fft(x);
  --> plot(t,abs(y));
\end{verbatim}
The resulting plot is:

\doplot{width=8cm}{fft1}

The FFT can also be taken along different dimensions, and with padding 
and/or truncation.  The following example demonstrates the Fourier
Transform being computed along each column, and then along each row.
\begin{verbatim}
--> A = [2,5;3,6]
A =
  <int32>  - size: [2 2]
  
Columns 1 to 2
             2              5
             3              6
--> real(fft(A,[],1))
ans =
  <float>  - size: [2 2]
  
Columns 1 to 2
    5.0000000         11.000000
   -1.0000000         -1.0000000
--> real(fft(A,[],2))
ans =
  <float>  - size: [2 2]
  
Columns 1 to 2
    7.0000000         -3.0000000
    9.0000000         -3.0000000
\end{verbatim}
Fourier transforms can also be padded using the \verb|n| argument.  This
pads the signal with zeros prior to taking the Fourier transform.  Zero
padding in the time domain results in frequency interpolation.  The
following example demonstrates the FFT of a pulse (consisting of 10 ones)
with (red line) and without (green circles) padding.
\begin{verbatim}
--> delta(1:10) = 1;
--> plot((0:255)/256*pi*2,real(fft(delta,256)),'r-');
--> hold on
--> plot((0:9)/10*pi*2,real(fft(delta)),'go');
\end{verbatim}
The resulting plot is:

\doplot{width=8cm}{fft2}
