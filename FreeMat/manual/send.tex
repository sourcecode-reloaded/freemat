% Copyright (c) 2002, 2003 Samit Basu
%
% Permission is hereby granted, free of charge, to any person obtaining a 
% copy of this software and associated documentation files (the "Software"), 
% to deal in the Software without restriction, including without limitation 
% the rights to use, copy, modify, merge, publish, distribute, sublicense, 
% and/or sell copies of the Software, and to permit persons to whom the 
% Software is furnished to do so, subject to the following conditions:
%
% The above copyright notice and this permission notice shall be included 
% in all copies or substantial portions of the Software.
%
% THE SOFTWARE IS PROVIDED "AS IS", WITHOUT WARRANTY OF ANY KIND, EXPRESS 
% OR IMPLIED, INCLUDING BUT NOT LIMITED TO THE WARRANTIES OF MERCHANTABILITY, 
% FITNESS FOR A PARTICULAR PURPOSE AND NONINFRINGEMENT. IN NO EVENT SHALL 
% THE AUTHORS OR COPYRIGHT HOLDERS BE LIABLE FOR ANY CLAIM, DAMAGES OR OTHER 
% LIABILITY, WHETHER IN AN ACTION OF CONTRACT, TORT OR OTHERWISE, ARISING
% FROM, OUT OF OR IN CONNECTION WITH THE SOFTWARE OR THE USE OR OTHER 
% DEALINGS IN THE SOFTWARE.
\subsection{SEND Send An Array Over A Socket Link}
\subsubsection{Usage}
Sends an array across a connected socket link (returned
either by \verb|accept| or \verb|connect|).  The call syntax is
\begin{verbatim}
  send(handle,x)
\end{verbatim}
where $x$ is the array to send, and \verb|handle| is the 
connected socket handle.
\subsubsection{Example}
Here is an example of how an array is sent over a socket link on two sessions of FreeMat.  On session one, we execute the following:
\begin{verbatim}
--> a = server(36230)
a =
  <uint32>  - size: [1 1]
            1
--> b = accept(a)
\end{verbatim}
At this point the first session will suspend.  In a second session, we execute the following:
\begin{verbatim}
--> c = connect('127.0.0.1',36230)
c =
  <uint32>  - size: [1 1]
            1
\end{verbatim}
In the first session, we get a return from \verb|accept|:
\begin{verbatim}
--> a = server(36230)
a =
  <uint32>  - size: [1 1]
            1
--> b = accept(a)
b =
  <uint32>  - size: [1 1]
            1
\end{verbatim}
At this point a link is set up between the two sessions.  If we execute a \verb|send| in session two:
\begin{verbatim}
--> send(c,'Hello world!')
\end{verbatim}
Then a \verb|receive| in session one results in
\begin{verbatim}
--> y = receive(b)
y =
  <string>  - size: [1 12]
 Hello world!
\end{verbatim}
