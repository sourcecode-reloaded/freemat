% Copyright (c) 2002, 2003 Samit Basu
%
% Permission is hereby granted, free of charge, to any person obtaining a 
% copy of this software and associated documentation files (the "Software"), 
% to deal in the Software without restriction, including without limitation 
% the rights to use, copy, modify, merge, publish, distribute, sublicense, 
% and/or sell copies of the Software, and to permit persons to whom the 
% Software is furnished to do so, subject to the following conditions:
%
% The above copyright notice and this permission notice shall be included 
% in all copies or substantial portions of the Software.
%
% THE SOFTWARE IS PROVIDED "AS IS", WITHOUT WARRANTY OF ANY KIND, EXPRESS 
% OR IMPLIED, INCLUDING BUT NOT LIMITED TO THE WARRANTIES OF MERCHANTABILITY, 
% FITNESS FOR A PARTICULAR PURPOSE AND NONINFRINGEMENT. IN NO EVENT SHALL 
% THE AUTHORS OR COPYRIGHT HOLDERS BE LIABLE FOR ANY CLAIM, DAMAGES OR OTHER 
% LIABILITY, WHETHER IN AN ACTION OF CONTRACT, TORT OR OTHERWISE, ARISING
% FROM, OUT OF OR IN CONNECTION WITH THE SOFTWARE OR THE USE OR OTHER 
% DEALINGS IN THE SOFTWARE.
\subsection{MINUS Subtraction Operator}
\subsubsection{Usage}
Subtracts two numerical arrays (elementwise).  There are two forms
for its use, both with the same general syntax:
\begin{verbatim}
  y = a - b
\end{verbatim}
where $a$ and $b$ are $n$-dimensional arrays of numerical type.  In the
first case, the two arguments are the same size, in which case, the 
output $y$ is the same size as the inputs, and is the element-wise
difference of $a$ and $b$.  In the second case, either $a$ or $b$ is a scalar, 
in which case $y$ is the same size as the larger argument,
and is the difference of the scalar to each element of the other argument.

The type of $y$ depends on the types of $a$ and $b$ using the type 
promotion rules.  The types are ordered as:
\begin{enumerate}
\item \verb|uint8| - unsigned, 8-bit integers range $[0,255]$
\item \verb|int8| - signed, 8-bit integers $[-127,128]$
\item \verb|uint16| - unsigned, 16-bit integers $[0,65535]$
\item \verb|int16| - signed, 16-bit integers $[-32768,32767]$
\item \verb|uint32| - unsigned, 32-bit integers $[0,4294967295]$
\item \verb|int32| - signed, 32-bit integers $[-2147483648,2147483647]$
\item \verb|float| - 32-bit floating point
\item \verb|double| - 64-bit floating point
\item \verb|complex| - 32-bit complex floating point
\item \verb|dcomplex| - 64-bit complex floating point
\end{enumerate}
Note that the type promotion and combination rules work similar to 
\verb|C|.  Numerical overflow rules are also the same as \verb|C|.
\subsubsection{Function Internals}
There are three formulae for the subtraction operator, depending on the
sizes of the three arguments.  In the most general case, in which 
the two arguments are the same size, the output is computed via:
\[
y(m_1,\ldots,m_d) = a(m_1,\ldots,m_d) - b(m_1,\ldots,m_d)
\]
If $a$ is a scalar, then the output is computed via
\[
y(m_1,\ldots,m_d) = a - b(m_1,\ldots,m_d).
\]
On the other hand, if $b$ is a scalar, then the output is computed via
\[
y(m_1,\ldots,m_d) = a(m_1,\ldots,m_d) - b.
\]
\subsubsection{Examples}
Here are some examples of using the subtraction operator.  First, a 
straight-forward usage of the minus operator.  The first example
is straightforward - the \verb|int32| is the default type used for
integer constants (same as in \verb|C|), hence the output is the
same type:
\begin{verbatim}
--> 3 - 8
ans =
  <int32>  - size: [1 1]
            -5
\end{verbatim}
Next, we use the floating point syntax to force one of the arguments
to be a \verb|double|, which results in the output being \verb|double|:
\begin{verbatim}
--> 3.1 - 2
ans =
  <double>  - size: [1 1]
    1.100000000000000
\end{verbatim}
Note that if one of the arguments is complex-valued, the output will be
complex also.
\begin{verbatim}
--> a = 3 + 4*i
a =
  <complex>  - size: [1 1]
    3.0000000         4.0000000     i
--> b = a - 2.0f
b =
  <complex>  - size: [1 1]
    1.0000000         4.0000000     i
\end{verbatim}
If a \verb|double| value is subtracted from a \verb|complex|, the result is 
promoted to \verb|dcomplex|.
\begin{verbatim}
--> b = a - 2.0
b =
  <dcomplex>  - size: [1 1]
    1.000000000000000        4.000000000000000    i
\end{verbatim}
We can also demonstrate the three forms of the subtraction operator.  First
the element-wise version:
\begin{verbatim}
--> a = [1,2;3,4]
a =
  <int32>  - size: [2 2]
  
Columns 1 to 2
             1              2
             3              4
--> b = [2,3;6,7]
b =
  <int32>  - size: [2 2]
  
Columns 1 to 2
             2              3
             6              7
--> c = a - b
c =
  <int32>  - size: [2 2]
  
Columns 1 to 2
            -1             -1
            -3             -3
\end{verbatim}
Then the scalar versions
\begin{verbatim}
--> c = a - 1
c =
  <int32>  - size: [2 2]
  
Columns 1 to 2
             0              1
             2              3
--> c = 1 - b
c =
  <int32>  - size: [2 2]
  
Columns 1 to 2
            -1             -2
            -5             -6
\end{verbatim}
