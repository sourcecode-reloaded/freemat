% Copyright (c) 2002, 2003 Samit Basu
%
% Permission is hereby granted, free of charge, to any person obtaining a 
% copy of this software and associated documentation files (the "Software"), 
% to deal in the Software without restriction, including without limitation 
% the rights to use, copy, modify, merge, publish, distribute, sublicense, 
% and/or sell copies of the Software, and to permit persons to whom the 
% Software is furnished to do so, subject to the following conditions:
%
% The above copyright notice and this permission notice shall be included 
% in all copies or substantial portions of the Software.
%
% THE SOFTWARE IS PROVIDED "AS IS", WITHOUT WARRANTY OF ANY KIND, EXPRESS 
% OR IMPLIED, INCLUDING BUT NOT LIMITED TO THE WARRANTIES OF MERCHANTABILITY, 
% FITNESS FOR A PARTICULAR PURPOSE AND NONINFRINGEMENT. IN NO EVENT SHALL 
% THE AUTHORS OR COPYRIGHT HOLDERS BE LIABLE FOR ANY CLAIM, DAMAGES OR OTHER 
% LIABILITY, WHETHER IN AN ACTION OF CONTRACT, TORT OR OTHERWISE, ARISING
% FROM, OUT OF OR IN CONNECTION WITH THE SOFTWARE OR THE USE OR OTHER 
% DEALINGS IN THE SOFTWARE.
\subsection{I-J Square Root of Negative One}
\subsubsection{Usage}
Returns a \verb|complex| value that represents $\sqrt{-1}$.  There are two
functions that return the same value:
\begin{verbatim}
   y = i
\end{verbatim}
and 
\begin{verbatim}
   y = j.
\end{verbatim}
This allows $i$ and $j$ to be used as loop indices and still permit access to $\sqrt{-1}$.  The returned value is a 32-bit complex value.
\subsubsection{Example}
The following examples demonstrate a few calculations with $i$.
\begin{verbatim}
--> i
ans =
  <complex>  - size: [1 1]
    0.00000000        1.0000000     i
--> i^2
ans =
  <complex>  - size: [1 1]
   -1.0000000         0.00000000    i
\end{verbatim}
The same calculations with $j$:
\begin{verbatim}
--> j
ans =
  <complex>  - size: [1 1]
    0.00000000        1.0000000     i
--> j^2
ans =
  <complex>  - size: [1 1]
   -1.0000000         0.00000000    i
\end{verbatim}
Here is an example of how $i$ can be used as a loop index and then recovered as $\sqrt{-1}$.
\begin{verbatim}
--> accum = 0; for i=1:100; accum = accum + i; end; accum
ans =
  <int32>  - size: [1 1]
          5050
--> i
ans =
  <int32>  - size: [1 1]
           100
--> clear i
--> i
ans =
  <complex>  - size: [1 1]
    0.00000000        1.0000000     i
\end{verbatim}
