% Copyright (c) 2002, 2003 Samit Basu
%
% Permission is hereby granted, free of charge, to any person obtaining a 
% copy of this software and associated documentation files (the "Software"), 
% to deal in the Software without restriction, including without limitation 
% the rights to use, copy, modify, merge, publish, distribute, sublicense, 
% and/or sell copies of the Software, and to permit persons to whom the 
% Software is furnished to do so, subject to the following conditions:
%
% The above copyright notice and this permission notice shall be included 
% in all copies or substantial portions of the Software.
%
% THE SOFTWARE IS PROVIDED "AS IS", WITHOUT WARRANTY OF ANY KIND, EXPRESS 
% OR IMPLIED, INCLUDING BUT NOT LIMITED TO THE WARRANTIES OF MERCHANTABILITY, 
% FITNESS FOR A PARTICULAR PURPOSE AND NONINFRINGEMENT. IN NO EVENT SHALL 
% THE AUTHORS OR COPYRIGHT HOLDERS BE LIABLE FOR ANY CLAIM, DAMAGES OR OTHER 
% LIABILITY, WHETHER IN AN ACTION OF CONTRACT, TORT OR OTHERWISE, ARISING
% FROM, OUT OF OR IN CONNECTION WITH THE SOFTWARE OR THE USE OR OTHER 
% DEALINGS IN THE SOFTWARE.
\subsection{WHO Describe Currently Defined Variables}
\subsubsection{Usage}
Reports information on either all variables in the current context
or on a specified set of variables.  For each variable, the \verb|who|
function indicates the size and type of the variable as well as 
if it is a global or persistent.  There are two formats for the 
function call.  The first is the explicit form, in which a list
of variables are provided:
\begin{verbatim}
  who a1 a2 ...
\end{verbatim}
In the second form
\begin{verbatim}
  who
\end{verbatim}
the \verb|who| function lists all variables defined in the current 
context (as well as global and persistent variables). Note that
there are two alternate forms for calling the \verb|who| function:
\begin{verbatim}
  who 'a1' 'a2' ...
\end{verbatim}
and
\begin{verbatim}
  who('a1','a2',...)
\end{verbatim}
\subsubsection{Example}
Here is an example of the general use of \verb|who|, which lists all of the variables defined.
\begin{verbatim}
--> who
  Variable Name      Type   Flags   Size
              a    double           [3 3]
              c      cell           [1 1]
            ans    double           []
              d     int32  global   [1 5]
              m     int32 persist   [1 1]
\end{verbatim}
In the second case, we examine only a specific variable:
\begin{verbatim}
--> who c
  Variable Name      Type   Flags   Size
              c      cell           [1 1]
--> who('c')
  Variable Name      Type   Flags   Size
              c      cell           [1 1]
\end{verbatim}

