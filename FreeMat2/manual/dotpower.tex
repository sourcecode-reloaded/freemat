% Copyright (c) 2002, 2003 Samit Basu
%
% Permission is hereby granted, free of charge, to any person obtaining a 
% copy of this software and associated documentation files (the "Software"), 
% to deal in the Software without restriction, including without limitation 
% the rights to use, copy, modify, merge, publish, distribute, sublicense, 
% and/or sell copies of the Software, and to permit persons to whom the 
% Software is furnished to do so, subject to the following conditions:
%
% The above copyright notice and this permission notice shall be included 
% in all copies or substantial portions of the Software.
%
% THE SOFTWARE IS PROVIDED "AS IS", WITHOUT WARRANTY OF ANY KIND, EXPRESS 
% OR IMPLIED, INCLUDING BUT NOT LIMITED TO THE WARRANTIES OF MERCHANTABILITY, 
% FITNESS FOR A PARTICULAR PURPOSE AND NONINFRINGEMENT. IN NO EVENT SHALL 
% THE AUTHORS OR COPYRIGHT HOLDERS BE LIABLE FOR ANY CLAIM, DAMAGES OR OTHER 
% LIABILITY, WHETHER IN AN ACTION OF CONTRACT, TORT OR OTHERWISE, ARISING
% FROM, OUT OF OR IN CONNECTION WITH THE SOFTWARE OR THE USE OR OTHER 
% DEALINGS IN THE SOFTWARE.
\subsection{DOTPOWER Element-wise Power Operator}
\subsubsection{Usage}
Raises one numerical array to another array (elementwise).  There are three operators all with the same general syntax:
\begin{verbatim}
  y = a .^ b
\end{verbatim}
The result $y$ depends on which of the following three situations applies to the arguments $a$ and $b$:
\begin{enumerate}
  \item $a$ is a scalar, $b$ is an arbitrary $n$-dimensional numerical array, in which case the output is $a$ raised to the power of each element of $b$, and the output is the same size as $b$.
  \item $a$ is an $n$-dimensional numerical array, and $b$ is a scalar, then the output is the same size as $a$, and is defined by each element of $a$ raised to the power $b$.
  \item $a$ and $b$ are both $n$-dimensional numerical arrays of \emph{the same size}.  In this case, each element of the output is the corresponding element of $a$ raised to the power defined by the corresponding element of $b$.
\end{enumerate}
The output follows the standard type promotion rules, although types are not generally preserved under the power operation.  In particular, integers are automatically converted to \verb|double| type, and negative numbers raised to fractional powers can return complex values.
\subsubsection{Function Internals}
There are three formulae for this operator.  For the first form
\[
y(m_1,\ldots,m_d) = a^{b(m_1,\ldots,m_d)},
\]
and the second form
\[
y(m_1,\ldots,m_d) = a(m_1,\ldots,m_d)^b,
\]
and in the third form
\[
y(m_1,\ldots,m_d) = a(m_1,\ldots,m_d)^{b(m_1,\ldots,m_d)}.
\]
\subsubsection{Examples}
We demonstrate the three forms of the dot-power operator using some simple examples.  First, the case of a scalar raised to a series of values.
\begin{verbatim}
--> a = 2
a =
  <int32>  - size: [1 1]
             2
--> b = 1:4
b =
  <int32>  - size: [1 4]
  
Columns 1 to 4
             1              2              3              4
--> c = a.^b
c =
  <double>  - size: [1 4]
  
Columns 1 to 2
    2.000000000000000         4.000000000000000
  
Columns 3 to 4
    8.000000000000000        16.0000000000000
\end{verbatim}
The second case shows a vector raised to a scalar.
\begin{verbatim}
--> c = b.^a
c =
  <double>  - size: [1 4]
  
Columns 1 to 2
    1.000000000000000         4.000000000000000
  
Columns 3 to 4
    9.000000000000000        16.0000000000000
\end{verbatim}
The third case shows the most general use of the dot-power operator.
\begin{verbatim}
--> A = [1,2;3,2]
A =
  <int32>  - size: [2 2]
  
Columns 1 to 2
             1              2
             3              2
--> B = [2,1.5;0.5,0.6]
B =
  <double>  - size: [2 2]
  
Columns 1 to 2
    2.000000000000000         1.500000000000000
    0.500000000000000         0.600000000000000
--> C = A.^B
C =
  <double>  - size: [2 2]
  
Columns 1 to 2
    1.000000000000000         2.828427124746190
    1.732050807568877         1.515716566510398
\end{verbatim}
