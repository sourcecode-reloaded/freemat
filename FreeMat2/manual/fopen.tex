% Copyright (c) 2002, 2003 Samit Basu
%
% Permission is hereby granted, free of charge, to any person obtaining a 
% copy of this software and associated documentation files (the "Software"), 
% to deal in the Software without restriction, including without limitation 
% the rights to use, copy, modify, merge, publish, distribute, sublicense, 
% and/or sell copies of the Software, and to permit persons to whom the 
% Software is furnished to do so, subject to the following conditions:
%
% The above copyright notice and this permission notice shall be included 
% in all copies or substantial portions of the Software.
%
% THE SOFTWARE IS PROVIDED "AS IS", WITHOUT WARRANTY OF ANY KIND, EXPRESS 
% OR IMPLIED, INCLUDING BUT NOT LIMITED TO THE WARRANTIES OF MERCHANTABILITY, 
% FITNESS FOR A PARTICULAR PURPOSE AND NONINFRINGEMENT. IN NO EVENT SHALL 
% THE AUTHORS OR COPYRIGHT HOLDERS BE LIABLE FOR ANY CLAIM, DAMAGES OR OTHER 
% LIABILITY, WHETHER IN AN ACTION OF CONTRACT, TORT OR OTHERWISE, ARISING
% FROM, OUT OF OR IN CONNECTION WITH THE SOFTWARE OR THE USE OR OTHER 
% DEALINGS IN THE SOFTWARE.
\subsection{FOPEN File Open Function}
\subsubsection{Usage}
Opens a file and returns a handle which can be used for subsequent
file manipulations.  The general syntax for its use is
\begin{verbatim}
  fp = fopen(fname,mode,byteorder)
\end{verbatim}
Here \verb|fname| is a string containing the name of the file to be 
opened.  \verb|mode| is the mode string for the file open command.
The first character of the mode string is one of the following:
\begin{itemize}
  \item \texttt{'r'}  Open  file  for  reading.  The file pointer is placed at
          the beginning of the file.  The file can be read from, but
	  not written to.
  \item \texttt{'r+'}   Open for reading and writing.  The file pointer is
          placed at the beginning of the file.  The file can be read
	  from and written to, but must exist at the outset.
  \item \texttt{'w'}    Open file for writing.  If the file already exists, it is
          truncated to zero length.  Otherwise, a new file is
	  created.  The file pointer is placed at the beginning of
	  the file.
  \item \texttt{'w+'}   Open for reading and writing.  The file is created  if  
          it  does not  exist, otherwise it is truncated to zero
	  length.  The file pointer placed at the beginning of the file.
  \item \texttt{'a'}    Open for appending (writing at end of file).  The file  is  
          created  if it does not exist.  The file pointer is placed at
	  the end of the file.
  \item \texttt{'a+'}   Open for reading and appending (writing at end of file).   The
          file  is created if it does not exist.  The file pointer is
	  placed at the end of the file.
\end{itemize}
On some platforms (e.g. Win32) it is necessary to add a 'b' for 
binary files to avoid the operating system's 'CR/LF<->CR' translation.

Finally, FreeMat has the ability to read and write files of any
byte-sex (endian).  The third (optional) input indicates the 
byte-endianness of the file.  If it is omitted, the native endian-ness
of the machine running FreeMat is used.  Otherwise, the third
argument should be one of the following strings:
\begin{itemize}
   \item \texttt{'le','ieee-le','little-endian','littleEndian','little'}
   \item \texttt{'be','ieee-be','big-endian','bigEndian','big'}
\end{itemize}
	
If the file cannot be opened, or the file mode is illegal, then
an error occurs. Otherwise, a file handle is returned (which is
an integer).  This file handle can then be used with \verb|fread|,
\verb|fwrite|, or \verb|fclose| for file access.

Note that three handles are assigned at initialization time:
\begin{itemize}
\item Handle $0$ - is assigned to standard input
\item Handle $1$ - is assigned to standard output
\item Handle $2$ - is assigned to standard error
\end{itemize}
These handles cannot be closed, so that user created file handles start at $3$.

\subsubsection{Examples}
Here are some examples of how to use \verb|fopen|.  First, we create a new file, which we want to be little-endian, regardless of the type of the machine.
\begin{verbatim}
--> fp = fopen('test.dat','wb','ieee-le')
fp =
  <uint32>  - size: [1 1]
            3
\end{verbatim}
