% Copyright (c) 2002, 2003 Samit Basu
%
% Permission is hereby granted, free of charge, to any person obtaining a 
% copy of this software and associated documentation files (the "Software"), 
% to deal in the Software without restriction, including without limitation 
% the rights to use, copy, modify, merge, publish, distribute, sublicense, 
% and/or sell copies of the Software, and to permit persons to whom the 
% Software is furnished to do so, subject to the following conditions:
%
% The above copyright notice and this permission notice shall be included 
% in all copies or substantial portions of the Software.
%
% THE SOFTWARE IS PROVIDED "AS IS", WITHOUT WARRANTY OF ANY KIND, EXPRESS 
% OR IMPLIED, INCLUDING BUT NOT LIMITED TO THE WARRANTIES OF MERCHANTABILITY, 
% FITNESS FOR A PARTICULAR PURPOSE AND NONINFRINGEMENT. IN NO EVENT SHALL 
% THE AUTHORS OR COPYRIGHT HOLDERS BE LIABLE FOR ANY CLAIM, DAMAGES OR OTHER 
% LIABILITY, WHETHER IN AN ACTION OF CONTRACT, TORT OR OTHERWISE, ARISING
% FROM, OUT OF OR IN CONNECTION WITH THE SOFTWARE OR THE USE OR OTHER 
% DEALINGS IN THE SOFTWARE.
\subsection{GRID Plot Grid Toggle Function}
\subsubsection{Usage}
Toggles the drawing of grid lines on the currently active plot.  The
general syntax for its use is
\begin{verbatim}
   grid(state)
\end{verbatim}
where \verb|state| is either
\begin{verbatim}
   grid('on')
\end{verbatim}
to activate the grid lines, or
\begin{verbatim}
   grid('off')
\end{verbatim}
to deactivate the grid lines.
\subsubsection{Example}
Here is a simple plot without grid lines.
\begin{verbatim}
--> x = linspace(-1,1);
--> y = cos(3*pi*x);
--> plot(x,y,'r-');
\end{verbatim}

\doplot{width=8cm}{grid1}

Next, we activate the grid lines.
\begin{verbatim}
--> grid on
\end{verbatim}

\doplot{width=8cm}{grid2}
