% Copyright (c) 2002, 2003 Samit Basu
%
% Permission is hereby granted, free of charge, to any person obtaining a 
% copy of this software and associated documentation files (the "Software"), 
% to deal in the Software without restriction, including without limitation 
% the rights to use, copy, modify, merge, publish, distribute, sublicense, 
% and/or sell copies of the Software, and to permit persons to whom the 
% Software is furnished to do so, subject to the following conditions:
%
% The above copyright notice and this permission notice shall be included 
% in all copies or substantial portions of the Software.
%
% THE SOFTWARE IS PROVIDED "AS IS", WITHOUT WARRANTY OF ANY KIND, EXPRESS 
% OR IMPLIED, INCLUDING BUT NOT LIMITED TO THE WARRANTIES OF MERCHANTABILITY, 
% FITNESS FOR A PARTICULAR PURPOSE AND NONINFRINGEMENT. IN NO EVENT SHALL 
% THE AUTHORS OR COPYRIGHT HOLDERS BE LIABLE FOR ANY CLAIM, DAMAGES OR OTHER 
% LIABILITY, WHETHER IN AN ACTION OF CONTRACT, TORT OR OTHERWISE, ARISING
% FROM, OUT OF OR IN CONNECTION WITH THE SOFTWARE OR THE USE OR OTHER 
% DEALINGS IN THE SOFTWARE.
\subsection{FCLOSE File Close Function}
\subsubsection{Usage}
Closes a file handle, or all open file handles.  The general syntax
for its use is either
\begin{verbatim}
  fclose(handle)
\end{verbatim}
or
\begin{verbatim}
  fclose('all')
\end{verbatim}
In the first case a specific file is closed,  In the second, all open
files are closed.  Note that until a file is closed the file buffers
are not flushed.
\subsubsection{Example}
A simple example of a file being opened with \verb|fopen| and then closed with \verb|fclose|.
\begin{verbatim}
--> fp = fopen('test.dat','wb','ieee-le')
fp =
  <uint32>  - size: [1 1]
            3
--> fclose(fp)
\end{verbatim}

