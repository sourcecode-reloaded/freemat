% Copyright (c) 2002, 2003 Samit Basu
%
% Permission is hereby granted, free of charge, to any person obtaining a 
% copy of this software and associated documentation files (the "Software"), 
% to deal in the Software without restriction, including without limitation 
% the rights to use, copy, modify, merge, publish, distribute, sublicense, 
% and/or sell copies of the Software, and to permit persons to whom the 
% Software is furnished to do so, subject to the following conditions:
%
% The above copyright notice and this permission notice shall be included 
% in all copies or substantial portions of the Software.
%
% THE SOFTWARE IS PROVIDED "AS IS", WITHOUT WARRANTY OF ANY KIND, EXPRESS 
% OR IMPLIED, INCLUDING BUT NOT LIMITED TO THE WARRANTIES OF MERCHANTABILITY, 
% FITNESS FOR A PARTICULAR PURPOSE AND NONINFRINGEMENT. IN NO EVENT SHALL 
% THE AUTHORS OR COPYRIGHT HOLDERS BE LIABLE FOR ANY CLAIM, DAMAGES OR OTHER 
% LIABILITY, WHETHER IN AN ACTION OF CONTRACT, TORT OR OTHERWISE, ARISING
% FROM, OUT OF OR IN CONNECTION WITH THE SOFTWARE OR THE USE OR OTHER 
% DEALINGS IN THE SOFTWARE.
\subsection{ATAN2 Inverse Trigonometric 4-Quadrant Arctangent Function}
\subsubsection{Usage}
Computes the \verb|atan2| function for its argument.  The general
syntax for its use is
\begin{verbatim}
  y = atan2(y,x)
\end{verbatim}
where $x$ and $y$ are $n$-dimensional arrays of numerical type.
Integer types are promoted to the \verb|double| type prior to
calculation of the $\mathrm{atan2}$ function. The size of the output depends
on the size of $x$ and $y$.  If $x$ is a scalar, then $z$
is the same size as $y$, and if $y$ is a scalar, then $z$
is the same size as $x$.  The type of the output is equal to the type of
$y/x$.  
\subsubsection{Function Internals}
The function is defined (for real values) to return an 
angle between $-\pi$ and $\pi$.  The signs of $x$ and $y$
are used to find the correct quadrant for the solution.  For complex
arguments, the two-argument arctangent is computed via
\[
  \mathrm{atan2}(y,x) \equiv -i \log\left(\frac{x+i y}{\sqrt{x^2+y^2}} \right)
\]
For real valued arguments $x,y$, the function is computed directly using 
the standard C library's numerical \verb|atan2| function. For both 
real and complex arguments $x$, note that generally
$\mathrm{atan2}(\sin(x),\cos(x)) \neq x$, due to the periodicities of 
$\cos(x)$ and $\sin(x)$.
\subsubsection{Example}
The following code demonstates the difference between the \verb|atan2| 
function and the \verb|atan| function over the range $[-\pi,\pi]$.
\begin{verbatim}
--> x = linspace(-pi,pi);
--> sx = sin(x); cx = cos(x);
--> plot(x,atan(sx./cx),x,atan2(sx,cx))
\end{verbatim}
\doplot{width=8cm}{atan2plot}
Note how the two-argument \verb|atan2| function (green line) 
correctly ``unwraps'' the phase of the angle, while the \verb|atan| 
function (red line) wraps the angle to the interval $[-\pi/2,\pi/2]$.
