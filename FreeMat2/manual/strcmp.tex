% Copyright (c) 2002, 2003 Samit Basu
%
% Permission is hereby granted, free of charge, to any person obtaining a 
% copy of this software and associated documentation files (the "Software"), 
% to deal in the Software without restriction, including without limitation 
% the rights to use, copy, modify, merge, publish, distribute, sublicense, 
% and/or sell copies of the Software, and to permit persons to whom the 
% Software is furnished to do so, subject to the following conditions:
%
% The above copyright notice and this permission notice shall be included 
% in all copies or substantial portions of the Software.
%
% THE SOFTWARE IS PROVIDED "AS IS", WITHOUT WARRANTY OF ANY KIND, EXPRESS 
% OR IMPLIED, INCLUDING BUT NOT LIMITED TO THE WARRANTIES OF MERCHANTABILITY, 
% FITNESS FOR A PARTICULAR PURPOSE AND NONINFRINGEMENT. IN NO EVENT SHALL 
% THE AUTHORS OR COPYRIGHT HOLDERS BE LIABLE FOR ANY CLAIM, DAMAGES OR OTHER 
% LIABILITY, WHETHER IN AN ACTION OF CONTRACT, TORT OR OTHERWISE, ARISING
% FROM, OUT OF OR IN CONNECTION WITH THE SOFTWARE OR THE USE OR OTHER 
% DEALINGS IN THE SOFTWARE.
\subsection{STRCMP String Compare Function}
\subsubsection{Usage}
Compares two strings $x$ and $y$ for equality.  The general
syntax for its use is
\begin{verbatim}
  p = strcmp(x,y)
\end{verbatim}
where $x$ and $y$ are two strings.  Returns \verb|true| if $x$
and $y$ are the same size, and are equal (as strings).  Otherwise,
it returns \verb|false|.
\subsubsection{Example}
The following piece of code compares two strings:
\begin{verbatim}
--> x1 = 'astring';
--> x2 = 'bstring';
--> x3 = 'astring';
--> strcmp(x1,x2)
ans =
  <logical>  - size: [1 1]
 0
--> strcmp(x1,x3)
ans =
  <logical>  - size: [1 1]
 1
\end{verbatim}

