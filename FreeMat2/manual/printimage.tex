% Copyright (c) 2002, 2003 Samit Basu
%
% Permission is hereby granted, free of charge, to any person obtaining a 
% copy of this software and associated documentation files (the "Software"), 
% to deal in the Software without restriction, including without limitation 
% the rights to use, copy, modify, merge, publish, distribute, sublicense, 
% and/or sell copies of the Software, and to permit persons to whom the 
% Software is furnished to do so, subject to the following conditions:
%
% The above copyright notice and this permission notice shall be included 
% in all copies or substantial portions of the Software.
%
% THE SOFTWARE IS PROVIDED "AS IS", WITHOUT WARRANTY OF ANY KIND, EXPRESS 
% OR IMPLIED, INCLUDING BUT NOT LIMITED TO THE WARRANTIES OF MERCHANTABILITY, 
% FITNESS FOR A PARTICULAR PURPOSE AND NONINFRINGEMENT. IN NO EVENT SHALL 
% THE AUTHORS OR COPYRIGHT HOLDERS BE LIABLE FOR ANY CLAIM, DAMAGES OR OTHER 
% LIABILITY, WHETHER IN AN ACTION OF CONTRACT, TORT OR OTHERWISE, ARISING
% FROM, OUT OF OR IN CONNECTION WITH THE SOFTWARE OR THE USE OR OTHER 
% DEALINGS IN THE SOFTWARE.
\subsection{PRINTIMAGE Print an Image To A File}
\subsubsection{Usage}
This function ``prints'' the currently active image to a file.  The 
generic syntax for its use is
\begin{verbatim}
  printimage(filename)
\end{verbatim}
or, alternately,
\begin{verbatim}
  printimage filename
\end{verbatim}
where \verb|filename| is the (string) filename of the destined file.  The current
image is then saved to the output file using a format that is determined
by the extension of the filename.  The exact output formats may vary on
different platforms, but generally speaking, the following extensions
should be supported cross-platform:
\begin{itemize}
\item \verb|jpg|, \verb|jpeg|  --  JPEG file 
\item \verb|ps|, \verb|eps| -- Encapsulated Postscript file 
\item \verb|png| -- Portable Net Graphics file
\end{itemize}
Note that only the image is printed, \emph{not} the window displaying
the image.  If you want something like that (essentially a window-capture)
use a seperate utility or your operating system's built in screen
capture ability.
\subsubsection{Example}
Here is a simple example of how the figures in this manual are generated.
\begin{verbatim}
--> x = linspace(-1,1,512)'*ones(1,512);
--> y = x';
--> Z = exp(-(x.^2+y.^2)/0.3);
--> image(Z);
--> printimage printimage1.eps
--> printimage printimage1.jpg
\end{verbatim}
which creates two images \verb|printimage1.eps|, which is an Encapsulated
Postscript file, and \verb|printimage1.jpg| which is a JPEG file.

\doplot{width=6cm}{printimage1}
