% Copyright (c) 2002, 2003 Samit Basu
%
% Permission is hereby granted, free of charge, to any person obtaining a 
% copy of this software and associated documentation files (the "Software"), 
% to deal in the Software without restriction, including without limitation 
% the rights to use, copy, modify, merge, publish, distribute, sublicense, 
% and/or sell copies of the Software, and to permit persons to whom the 
% Software is furnished to do so, subject to the following conditions:
%
% The above copyright notice and this permission notice shall be included 
% in all copies or substantial portions of the Software.
%
% THE SOFTWARE IS PROVIDED "AS IS", WITHOUT WARRANTY OF ANY KIND, EXPRESS 
% OR IMPLIED, INCLUDING BUT NOT LIMITED TO THE WARRANTIES OF MERCHANTABILITY, 
% FITNESS FOR A PARTICULAR PURPOSE AND NONINFRINGEMENT. IN NO EVENT SHALL 
% THE AUTHORS OR COPYRIGHT HOLDERS BE LIABLE FOR ANY CLAIM, DAMAGES OR OTHER 
% LIABILITY, WHETHER IN AN ACTION OF CONTRACT, TORT OR OTHERWISE, ARISING
% FROM, OUT OF OR IN CONNECTION WITH THE SOFTWARE OR THE USE OR OTHER 
% DEALINGS IN THE SOFTWARE.
\subsection{RESHAPE Reshape An Array}
\subsubsection{Usage}
Reshapes an array from one size to another. Two seperate 
syntaxes are possible.  The first syntax specifies the array 
dimensions as a sequence of scalar dimensions:
\begin{verbatim}
   y = reshape(x,d1,d2,...,dn).
\end{verbatim}
The resulting array has the given dimensions, and is filled with
the contents of $x$.  The type of $y$ is the same as $x$.  The
\verb|reshape| function requires that the length of $x$ equal
\[
    \prod_{i=1}^{n} di
\]
    
The second syntax specifies the array dimensions as a vector,
where each element in the vector specifies a dimension length:
\begin{verbatim}
   y = reshape(x,[d1,d2,...,dn]).
\end{verbatim}
This syntax is more convenient for calling \verb|reshape| using a 
variable for the argument. Note that arrays are stored in column format, 
which means that elements in $x$ are read in the following order:
\doplot{width=8cm}{reshape}

\subsubsection{Example}
Here are several examples of the use of \verb|reshape| applied to
various arrays.  The first example reshapes a row vector into a 
matrix.
\begin{verbatim}
--> a = uint8(1:6)
a =
  <uint8>  - size: [1 6]
  
Columns 1 to 6
   1    2    3    4    5    6
--> reshape(a,2,3)
ans =
  <uint8>  - size: [2 3]
  
Columns 1 to 3
   1    3    5
   2    4    6
\end{verbatim}
The second example reshapes a longer row vector into a volume with 
two planes.
\begin{verbatim}
--> a = uint8(1:12)
a =
  <uint8>  - size: [1 12]
  
Columns 1 to 12
   1    2    3    4    5    6    7    8    9   10   11   12
--> reshape(a,[2,3,2])
ans =
  <uint8>  - size: [2 3 2]
(:,:,1) =
  
Columns 1 to 3
   1    3    5
   2    4    6
(:,:,2) =
  
Columns 1 to 3
   7    9   11
   8   10   12
\end{verbatim}
The third example reshapes a matrix into another matrix.
\begin{verbatim}
--> a = [1,6,7;3,4,2]
a =
  <int32>  - size: [2 3]
  
Columns 1 to 3
             1              6              7
             3              4              2
--> reshape(a,3,2)
ans =
  <int32>  - size: [3 2]
  
Columns 1 to 2
             1              4
             3              7
             6              2  
\end{verbatim}
