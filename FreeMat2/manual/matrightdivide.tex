% Copyright (c) 2002, 2003 Samit Basu
%
% Permission is hereby granted, free of charge, to any person obtaining a 
% copy of this software and associated documentation files (the "Software"), 
% to deal in the Software without restriction, including without limitation 
% the rights to use, copy, modify, merge, publish, distribute, sublicense, 
% and/or sell copies of the Software, and to permit persons to whom the 
% Software is furnished to do so, subject to the following conditions:
%
% The above copyright notice and this permission notice shall be included 
% in all copies or substantial portions of the Software.
%
% THE SOFTWARE IS PROVIDED "AS IS", WITHOUT WARRANTY OF ANY KIND, EXPRESS 
% OR IMPLIED, INCLUDING BUT NOT LIMITED TO THE WARRANTIES OF MERCHANTABILITY, 
% FITNESS FOR A PARTICULAR PURPOSE AND NONINFRINGEMENT. IN NO EVENT SHALL 
% THE AUTHORS OR COPYRIGHT HOLDERS BE LIABLE FOR ANY CLAIM, DAMAGES OR OTHER 
% LIABILITY, WHETHER IN AN ACTION OF CONTRACT, TORT OR OTHERWISE, ARISING
% FROM, OUT OF OR IN CONNECTION WITH THE SOFTWARE OR THE USE OR OTHER 
% DEALINGS IN THE SOFTWARE.
\subsection{RIGHTDIVIDE Matrix Equation Solver/Divide Operator}
\subsubsection{Usage}
The divide operator \verb|/| is really a combination of three
operators, all of which have the same general syntax:
\begin{verbatim}
  Y = A / B
\end{verbatim}
where $A$ and $B$ are arrays of numerical type.  The result $Y$ depends
on which of the following three situations applies to the arguments
$A$ and $B$:
\begin{enumerate}
  \item $A$ is a scalar, $B$ is an arbitrary $n$-dimensional numerical array, in which case the output is the scalar $A$ divided into each element of $B$.
  \item $B$ is a scalar, $A$ is an arbitrary $n$-dimensional numerical array, in which case the output is each element of $A$ divided by the scalar $B$.
  \item $A,B$ are matrices with the same number of columns, i.e., $A$ is of size $K \times M$, and $B$ is of size $L \times M$, in which case the output is of size $K \times L$.
\end{enumerate}
The output follows the standard type promotion rules, although in the first two cases, if $A$ and $B$ are integers, the output is an integer also, while in the third case if $A$ and $B$ are integers, the output is of type \verb|double|.

\subsubsection{Function Internals}
There are three formulae for the times operator.  For the first form
\[
Y(m_1,\ldots,m_d) = \frac{A}{B(m_1,\ldots,m_d)},
\]
and the second form
\[
Y(m_1,\ldots,m_d) = \frac{A(m_1,\ldots,m_d)}{B}.
\]
In the third form, the output is defined as:
\[
  Y = (B' \backslash A')'
\]
and is used in the equation $Y B = A$.
\subsubsection{Examples}
The right-divide operator is much less frequently used than the left-divide operator, but the concepts are similar.  It can be used to find least-squares and minimum norm solutions.  It can also be used to solve systems of equations in much the same way.  Here's a simple example:
\begin{verbatim}
--> B = [1,1;0,1];
--> A = [4,5]
A =
  <int32>  - size: [1 2]
  
Columns 1 to 2
             4              5
--> A/B
ans =
  <double>  - size: [1 2]
  
Columns 1 to 2
    4.000000000000000         1.000000000000000
\end{verbatim}
