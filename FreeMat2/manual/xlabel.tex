% Copyright (c) 2002, 2003 Samit Basu
%
% Permission is hereby granted, free of charge, to any person obtaining a 
% copy of this software and associated documentation files (the "Software"), 
% to deal in the Software without restriction, including without limitation 
% the rights to use, copy, modify, merge, publish, distribute, sublicense, 
% and/or sell copies of the Software, and to permit persons to whom the 
% Software is furnished to do so, subject to the following conditions:
%
% The above copyright notice and this permission notice shall be included 
% in all copies or substantial portions of the Software.
%
% THE SOFTWARE IS PROVIDED "AS IS", WITHOUT WARRANTY OF ANY KIND, EXPRESS 
% OR IMPLIED, INCLUDING BUT NOT LIMITED TO THE WARRANTIES OF MERCHANTABILITY, 
% FITNESS FOR A PARTICULAR PURPOSE AND NONINFRINGEMENT. IN NO EVENT SHALL 
% THE AUTHORS OR COPYRIGHT HOLDERS BE LIABLE FOR ANY CLAIM, DAMAGES OR OTHER 
% LIABILITY, WHETHER IN AN ACTION OF CONTRACT, TORT OR OTHERWISE, ARISING
% FROM, OUT OF OR IN CONNECTION WITH THE SOFTWARE OR THE USE OR OTHER 
% DEALINGS IN THE SOFTWARE.
\subsection{XLABEL Plot X-axis Label Function}
\subsubsection{Usage}
This command adds a label to the x-axis of the plot.  The general syntax
for its use is
\begin{verbatim}
  xlabel('label')
\end{verbatim}
or in the alternate form
\begin{verbatim}
  xlabel 'label'
\end{verbatim}
or simply
\begin{verbatim}
  xlabel label
\end{verbatim}
Here \verb|label| is a string variable.
\subsubsection{Example}
Here is an example of a simple plot with a label on the \verb|x|-axis.
\begin{verbatim}
--> x = linspace(-1,1);
--> y = cos(2*pi*x);
--> plot(x,y,'r-');
--> xlabel('time');
\end{verbatim}
which results in the following plot.

\doplot{width=8cm}{xlabel1}
