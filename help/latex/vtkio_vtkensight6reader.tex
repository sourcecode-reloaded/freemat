\section{vtkEnSight6Reader}

\subsection{Usage}

 vtkEnSight6Reader is a class to read EnSight6 files into vtk.
 Because the different parts of the EnSight data can be of various data
 types, this reader produces multiple outputs, one per part in the input
 file.
 All variable information is being stored in field data.  The descriptions
 listed in the case file are used as the array names in the field data.
 For complex vector variables, the description is appended with \_r (for the
 array of real values) and \_i (for the array if imaginary values).  Complex
 scalar variables are stored as a single array with 2 components, real and
 imaginary, listed in that order.

To create an instance of class vtkEnSight6Reader, simply
invoke its constructor as follows
\begin{verbatim}
  obj = vtkEnSight6Reader
\end{verbatim}
\subsection{Methods}

The class vtkEnSight6Reader has several methods that can be used.
  They are listed below.
Note that the documentation is translated automatically from the VTK sources,
and may not be completely intelligible.  When in doubt, consult the VTK website.
In the methods listed below, \verb|obj| is an instance of the vtkEnSight6Reader class.
\begin{itemize}
\item  \verb|string = obj.GetClassName ()|

\item  \verb|int = obj.IsA (string name)|

\item  \verb|vtkEnSight6Reader = obj.NewInstance ()|

\item  \verb|vtkEnSight6Reader = obj.SafeDownCast (vtkObject o)|

\end{itemize}
