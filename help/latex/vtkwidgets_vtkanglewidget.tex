\section{vtkAngleWidget}

\subsection{Usage}

 The vtkAngleWidget is used to measure the angle between two rays (defined
 by three points). The three points (two end points and a center)
 can be positioned independently, and when they are released, a special
 PlacePointEvent is invoked so that special operations may be take to
 reposition the point (snap to grid, etc.) The widget has two different
 modes of interaction: when initially defined (i.e., placing the three
 points) and then a manipulate mode (adjusting the position of the 
 three points).
 
 To use this widget, specify an instance of vtkAngleWidget and a
 representation (a subclass of vtkAngleRepresentation). The widget is
 implemented using three instances of vtkHandleWidget which are used to
 position the three points. The representations for these handle widgets
 are provided by the vtkAngleRepresentation.

 .SECTION Event Bindings
 By default, the widget responds to the following VTK events (i.e., it
 watches the vtkRenderWindowInteractor for these events):
 \begin{verbatim}
   LeftButtonPressEvent - add a point or select a handle 
   MouseMoveEvent - position the second or third point, or move a handle
   LeftButtonReleaseEvent - release the selected handle
 \end{verbatim}

 Note that the event bindings described above can be changed using this
 class's vtkWidgetEventTranslator. This class translates VTK events 
 into the vtkAngleWidget's widget events:
 \begin{verbatim}
   vtkWidgetEvent::AddPoint -- add one point; depending on the state
                               it may the first, second or third point 
                               added. Or, if near a handle, select the handle.
   vtkWidgetEvent::Move -- position the second or third point, or move the
                           handle depending on the state.
   vtkWidgetEvent::EndSelect -- the handle manipulation process has completed.
 \end{verbatim}

 This widget invokes the following VTK events on itself (which observers
 can listen for):
 \begin{verbatim}
   vtkCommand::StartInteractionEvent (beginning to interact)
   vtkCommand::EndInteractionEvent (completing interaction)
   vtkCommand::InteractionEvent (moving a handle)
   vtkCommand::PlacePointEvent (after a point is positioned; 
                                call data includes handle id (0,1,2))
 \end{verbatim}

To create an instance of class vtkAngleWidget, simply
invoke its constructor as follows
\begin{verbatim}
  obj = vtkAngleWidget
\end{verbatim}
\subsection{Methods}

The class vtkAngleWidget has several methods that can be used.
  They are listed below.
Note that the documentation is translated automatically from the VTK sources,
and may not be completely intelligible.  When in doubt, consult the VTK website.
In the methods listed below, \verb|obj| is an instance of the vtkAngleWidget class.
\begin{itemize}
\item  \verb|string = obj.GetClassName ()| -  Standard methods for a VTK class.

\item  \verb|int = obj.IsA (string name)| -  Standard methods for a VTK class.

\item  \verb|vtkAngleWidget = obj.NewInstance ()| -  Standard methods for a VTK class.

\item  \verb|vtkAngleWidget = obj.SafeDownCast (vtkObject o)| -  Standard methods for a VTK class.

\item  \verb|obj.SetEnabled (int )| -  The method for activiating and deactiviating this widget. This method
 must be overridden because it is a composite widget and does more than
 its superclasses' vtkAbstractWidget::SetEnabled() method.

\item  \verb|obj.SetRepresentation (vtkAngleRepresentation r)| -  Create the default widget representation if one is not set. 

\item  \verb|obj.CreateDefaultRepresentation ()| -  Create the default widget representation if one is not set. 

\item  \verb|int = obj.IsAngleValid ()| -  A flag indicates whether the angle is valid. The angle value only becomes
 valid after two of the three points are placed.

\item  \verb|obj.SetProcessEvents (int )| -  Methods to change the whether the widget responds to interaction.
 Overridden to pass the state to component widgets.

\end{itemize}
