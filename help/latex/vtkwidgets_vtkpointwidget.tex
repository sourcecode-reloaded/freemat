\section{vtkPointWidget}

\subsection{Usage}

 This 3D widget allows the user to position a point in 3D space using a 3D
 cursor. The cursor has an outline bounding box, axes-aligned cross-hairs,
 and axes shadows. (The outline and shadows can be turned off.) Any of
 these can be turned off. A nice feature of the object is that the
 vtkPointWidget, like any 3D widget, will work with the current interactor
 style. That is, if vtkPointWidget does not handle an event, then all other
 registered observers (including the interactor style) have an opportunity
 to process the event. Otherwise, the vtkPointWidget will terminate the
 processing of the event that it handles.

 To use this object, just invoke SetInteractor() with the argument of the
 method a vtkRenderWindowInteractor.  You may also wish to invoke
 ''PlaceWidget()'' to initially position the widget. The interactor will act
 normally until the ''i'' key (for ''interactor'') is pressed, at which point
 the vtkPointWidget will appear. (See superclass documentation for
 information about changing this behavior.) To move the point, the user can
 grab (left mouse) on any widget line and ''slide'' the point into
 position. Scaling is achieved by using the right mouse button ''up'' the
 render window (makes the widget bigger) or ''down'' the render window (makes
 the widget smaller). To translate the widget use the middle mouse button.
 (Note: all of the translation interactions can be constrained to one of
 the x-y-z axes by using the ''shift'' key.) The vtkPointWidget produces as
 output a polydata with a single point and a vertex cell.

 Some additional features of this class include the ability to control the
 rendered properties of the widget. You can set the properties of the
 selected and unselected representations of the parts of the widget. For
 example, you can set the property of the 3D cursor in its normal and
 selected states.

To create an instance of class vtkPointWidget, simply
invoke its constructor as follows
\begin{verbatim}
  obj = vtkPointWidget
\end{verbatim}
\subsection{Methods}

The class vtkPointWidget has several methods that can be used.
  They are listed below.
Note that the documentation is translated automatically from the VTK sources,
and may not be completely intelligible.  When in doubt, consult the VTK website.
In the methods listed below, \verb|obj| is an instance of the vtkPointWidget class.
\begin{itemize}
\item  \verb|string = obj.GetClassName ()|

\item  \verb|int = obj.IsA (string name)|

\item  \verb|vtkPointWidget = obj.NewInstance ()|

\item  \verb|vtkPointWidget = obj.SafeDownCast (vtkObject o)|

\item  \verb|obj.SetEnabled (int )| -  Methods that satisfy the superclass' API.

\item  \verb|obj.PlaceWidget (double bounds[6])| -  Methods that satisfy the superclass' API.

\item  \verb|obj.PlaceWidget ()| -  Methods that satisfy the superclass' API.

\item  \verb|obj.PlaceWidget (double xmin, double xmax, double ymin, double ymax, double zmin, double zmax)| -  Grab the polydata (including points) that defines the point. A
 single point and a vertex compose the vtkPolyData.

\item  \verb|obj.GetPolyData (vtkPolyData pd)| -  Grab the polydata (including points) that defines the point. A
 single point and a vertex compose the vtkPolyData.

\item  \verb|obj.SetPosition (double x, double y, double z)| -  Set/Get the position of the point. Note that if the position is set
 outside of the bounding box, it will be clamped to the boundary of
 the bounding box.

\item  \verb|obj.SetPosition (double x[3])| -  Set/Get the position of the point. Note that if the position is set
 outside of the bounding box, it will be clamped to the boundary of
 the bounding box.

\item  \verb|double = obj.GetPosition ()| -  Set/Get the position of the point. Note that if the position is set
 outside of the bounding box, it will be clamped to the boundary of
 the bounding box.

\item  \verb|obj.GetPosition (double xyz[3])| -  Turn on/off the wireframe bounding box.

\item  \verb|obj.SetOutline (int o)| -  Turn on/off the wireframe bounding box.

\item  \verb|int = obj.GetOutline ()| -  Turn on/off the wireframe bounding box.

\item  \verb|obj.OutlineOn ()| -  Turn on/off the wireframe bounding box.

\item  \verb|obj.OutlineOff ()| -  Turn on/off the wireframe x-shadows.

\item  \verb|obj.SetXShadows (int o)| -  Turn on/off the wireframe x-shadows.

\item  \verb|int = obj.GetXShadows ()| -  Turn on/off the wireframe x-shadows.

\item  \verb|obj.XShadowsOn ()| -  Turn on/off the wireframe x-shadows.

\item  \verb|obj.XShadowsOff ()| -  Turn on/off the wireframe y-shadows.

\item  \verb|obj.SetYShadows (int o)| -  Turn on/off the wireframe y-shadows.

\item  \verb|int = obj.GetYShadows ()| -  Turn on/off the wireframe y-shadows.

\item  \verb|obj.YShadowsOn ()| -  Turn on/off the wireframe y-shadows.

\item  \verb|obj.YShadowsOff ()| -  Turn on/off the wireframe z-shadows.

\item  \verb|obj.SetZShadows (int o)| -  Turn on/off the wireframe z-shadows.

\item  \verb|int = obj.GetZShadows ()| -  Turn on/off the wireframe z-shadows.

\item  \verb|obj.ZShadowsOn ()| -  Turn on/off the wireframe z-shadows.

\item  \verb|obj.ZShadowsOff ()| -  If translation mode is on, as the widget is moved the bounding box,
 shadows, and cursor are all translated simultaneously as the point
 moves.

\item  \verb|obj.SetTranslationMode (int mode)| -  If translation mode is on, as the widget is moved the bounding box,
 shadows, and cursor are all translated simultaneously as the point
 moves.

\item  \verb|int = obj.GetTranslationMode ()| -  If translation mode is on, as the widget is moved the bounding box,
 shadows, and cursor are all translated simultaneously as the point
 moves.

\item  \verb|obj.TranslationModeOn ()| -  If translation mode is on, as the widget is moved the bounding box,
 shadows, and cursor are all translated simultaneously as the point
 moves.

\item  \verb|obj.TranslationModeOff ()| -  Convenience methods to turn outline and shadows on and off.

\item  \verb|obj.AllOn ()| -  Convenience methods to turn outline and shadows on and off.

\item  \verb|obj.AllOff ()| -  Get the handle properties (the little balls are the handles). The 
 properties of the handles when selected and normal can be 
 set.

\item  \verb|vtkProperty = obj.GetProperty ()| -  Get the handle properties (the little balls are the handles). The 
 properties of the handles when selected and normal can be 
 set.

\item  \verb|vtkProperty = obj.GetSelectedProperty ()| -  Get the handle properties (the little balls are the handles). The 
 properties of the handles when selected and normal can be 
 set.

\item  \verb|obj.SetHotSpotSize (double )| -  Set the ''hot spot'' size; i.e., the region around the focus, in which the
 motion vector is used to control the constrained sliding action. Note the
 size is specified as a fraction of the length of the diagonal of the 
 point widget's bounding box.

\item  \verb|double = obj.GetHotSpotSizeMinValue ()| -  Set the ''hot spot'' size; i.e., the region around the focus, in which the
 motion vector is used to control the constrained sliding action. Note the
 size is specified as a fraction of the length of the diagonal of the 
 point widget's bounding box.

\item  \verb|double = obj.GetHotSpotSizeMaxValue ()| -  Set the ''hot spot'' size; i.e., the region around the focus, in which the
 motion vector is used to control the constrained sliding action. Note the
 size is specified as a fraction of the length of the diagonal of the 
 point widget's bounding box.

\item  \verb|double = obj.GetHotSpotSize ()| -  Set the ''hot spot'' size; i.e., the region around the focus, in which the
 motion vector is used to control the constrained sliding action. Note the
 size is specified as a fraction of the length of the diagonal of the 
 point widget's bounding box.

\end{itemize}
