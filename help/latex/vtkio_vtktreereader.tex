\section{vtkTreeReader}

\subsection{Usage}

 vtkTreeReader is a source object that reads ASCII or binary 
 vtkTree data files in vtk format. (see text for format details).
 The output of this reader is a single vtkTree data object.
 The superclass of this class, vtkDataReader, provides many methods for
 controlling the reading of the data file, see vtkDataReader for more
 information.

To create an instance of class vtkTreeReader, simply
invoke its constructor as follows
\begin{verbatim}
  obj = vtkTreeReader
\end{verbatim}
\subsection{Methods}

The class vtkTreeReader has several methods that can be used.
  They are listed below.
Note that the documentation is translated automatically from the VTK sources,
and may not be completely intelligible.  When in doubt, consult the VTK website.
In the methods listed below, \verb|obj| is an instance of the vtkTreeReader class.
\begin{itemize}
\item  \verb|string = obj.GetClassName ()|

\item  \verb|int = obj.IsA (string name)|

\item  \verb|vtkTreeReader = obj.NewInstance ()|

\item  \verb|vtkTreeReader = obj.SafeDownCast (vtkObject o)|

\item  \verb|vtkTree = obj.GetOutput ()| -  Get the output of this reader.

\item  \verb|vtkTree = obj.GetOutput (int idx)| -  Get the output of this reader.

\item  \verb|obj.SetOutput (vtkTree output)| -  Get the output of this reader.

\end{itemize}
