\section{FORMAT Control the Format of Matrix Display}

\subsection{Usage}

FreeMat supports several modes for displaying matrices (either through the
\verb|disp| function or simply by entering expressions on the command line.  
There are several options for the format command.  The default mode is equivalent
to
\begin{verbatim}
   format short
\end{verbatim}
which generally displays matrices with 4 decimals, and scales matrices if the entries
have magnitudes larger than roughly \verb|1e2| or smaller than \verb|1e-2|.   For more 
information you can use 
\begin{verbatim}
   format long
\end{verbatim}
which displays roughly 7 decimals for \verb|float| and \verb|complex| arrays, and 14 decimals
for \verb|double| and \verb|dcomplex|.  You can also use
\begin{verbatim}
   format short e
\end{verbatim}
to get exponential format with 4 decimals.  Matrices are not scaled for exponential 
formats.  Similarly, you can use
\begin{verbatim}
   format long e
\end{verbatim}
which displays the same decimals as \verb|format long|, but in exponential format.
You can also use the \verb|format| command to retrieve the current format:
\begin{verbatim}
   s = format
\end{verbatim}
where \verb|s| is a string describing the current format.
\subsection{Example}

We start with the short format, and two matrices, one of double precision, and the
other of single precision.
\begin{verbatim}
--> format short
--> a = randn(4)

a = 
    1.9819   -2.4403    0.3082   -0.8708 
    0.3885    0.6777   -2.1203   -0.6899 
   -0.8923   -1.0126   -1.1448   -0.3336 
   -0.5528   -0.2117   -0.6066    0.1530 

--> b = float(randn(4))

b = 
    1.6282   -0.9987   -0.0002    0.7346 
   -0.1461   -0.4450    0.3260    0.0591 
   -0.1927   -0.2583   -0.3209   -1.7827 
   -0.4694   -0.2961   -0.3487   -0.1476 
\end{verbatim}
Note that in the short format, these two matrices are displayed with the same format.
In \verb|long| format, however, they display differently
\begin{verbatim}
--> format long
--> a

ans = 
   1.98194242245660  -2.44033999910309   0.30822105452542  -0.87083520854217 
   0.38848412098646   0.67772654122050  -2.12029702950896  -0.68985792035578 
  -0.89231827506021  -1.01256221976480  -1.14477420632547  -0.33359041318909 
  -0.55283586680694  -0.21170713821002  -0.60660544623052   0.15300924745427 

--> b

ans = 
   1.6282195  -0.9986902  -0.0002282   0.7346091 
  -0.1460750  -0.4449911   0.3259999   0.0591399 
  -0.1926918  -0.2583237  -0.3208777  -1.7827009 
  -0.4693597  -0.2961315  -0.3487136  -0.1476461 
\end{verbatim}
Note also that we we scale the contents of the matrices, FreeMat rescales the entries
with a scale premultiplier.
\begin{verbatim}
--> format short
--> a*1e4

ans = 

   1.0e+04 * 
    1.9819   -2.4403    0.3082   -0.8708 
    0.3885    0.6777   -2.1203   -0.6899 
   -0.8923   -1.0126   -1.1448   -0.3336 
   -0.5528   -0.2117   -0.6066    0.1530 

--> a*1e-4

ans = 

   1.0e-04 * 
    1.9819   -2.4403    0.3082   -0.8708 
    0.3885    0.6777   -2.1203   -0.6899 
   -0.8923   -1.0126   -1.1448   -0.3336 
   -0.5528   -0.2117   -0.6066    0.1530 

--> b*1e4

ans = 

   1.0e+04 * 
    1.6282   -0.9987   -0.0002    0.7346 
   -0.1461   -0.4450    0.3260    0.0591 
   -0.1927   -0.2583   -0.3209   -1.7827 
   -0.4694   -0.2961   -0.3487   -0.1476 

--> b*1e-4

ans = 

   1.0e-04 * 
    1.6282   -0.9987   -0.0002    0.7346 
   -0.1461   -0.4450    0.3260    0.0591 
   -0.1927   -0.2583   -0.3209   -1.7827 
   -0.4694   -0.2961   -0.3487   -0.1476 
\end{verbatim}
Next, we use the exponential formats:
\begin{verbatim}
--> format short e
--> a*1e4

ans = 
  1.9819e+04 -2.4403e+04  3.0822e+03 -8.7084e+03 
  3.8848e+03  6.7773e+03 -2.1203e+04 -6.8986e+03 
 -8.9232e+03 -1.0126e+04 -1.1448e+04 -3.3359e+03 
 -5.5284e+03 -2.1171e+03 -6.0661e+03  1.5301e+03 

--> a*1e-4

ans = 
  1.9819e-04 -2.4403e-04  3.0822e-05 -8.7084e-05 
  3.8848e-05  6.7773e-05 -2.1203e-04 -6.8986e-05 
 -8.9232e-05 -1.0126e-04 -1.1448e-04 -3.3359e-05 
 -5.5284e-05 -2.1171e-05 -6.0661e-05  1.5301e-05 

--> b*1e4

ans = 
  1.6282e+04 -9.9869e+03 -2.2825e+00  7.3461e+03 
 -1.4608e+03 -4.4499e+03  3.2600e+03  5.9140e+02 
 -1.9269e+03 -2.5832e+03 -3.2088e+03 -1.7827e+04 
 -4.6936e+03 -2.9613e+03 -3.4871e+03 -1.4765e+03 

--> b*1e-4

ans = 
  1.6282e-04 -9.9869e-05 -2.2825e-08  7.3461e-05 
 -1.4608e-05 -4.4499e-05  3.2600e-05  5.9140e-06 
 -1.9269e-05 -2.5832e-05 -3.2088e-05 -1.7827e-04 
 -4.6936e-05 -2.9613e-05 -3.4871e-05 -1.4765e-05 
\end{verbatim}
Finally, if we assign the \verb|format| function to a variable, we can retrieve the 
current format:
\begin{verbatim}
--> format short
--> t = format

t = 
short
\end{verbatim}
