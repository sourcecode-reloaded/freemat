\section{vtkLinearTransform}

\subsection{Usage}

 vtkLinearTransform provides a generic interface for linear 
 (affine or 12 degree-of-freedom) geometric transformations. 

To create an instance of class vtkLinearTransform, simply
invoke its constructor as follows
\begin{verbatim}
  obj = vtkLinearTransform
\end{verbatim}
\subsection{Methods}

The class vtkLinearTransform has several methods that can be used.
  They are listed below.
Note that the documentation is translated automatically from the VTK sources,
and may not be completely intelligible.  When in doubt, consult the VTK website.
In the methods listed below, \verb|obj| is an instance of the vtkLinearTransform class.
\begin{itemize}
\item  \verb|string = obj.GetClassName ()|

\item  \verb|int = obj.IsA (string name)|

\item  \verb|vtkLinearTransform = obj.NewInstance ()|

\item  \verb|vtkLinearTransform = obj.SafeDownCast (vtkObject o)|

\item  \verb|obj.TransformNormal (float in[3], float out[3])| -  Apply the transformation to a normal.
 You can use the same array to store both the input and output.

\item  \verb|obj.TransformNormal (double in[3], double out[3])| -  Apply the transformation to a double-precision normal.
 You can use the same array to store both the input and output.

\item  \verb|double = obj.TransformNormal (double x, double y, double z)| -  Synonymous with TransformDoubleNormal(x,y,z).
 Use this if you are programming in python, tcl or Java.

\item  \verb|double = obj.TransformNormal (double normal[3])| -  Synonymous with TransformDoubleNormal(x,y,z).
 Use this if you are programming in python, tcl or Java.

\item  \verb|float = obj.TransformFloatNormal (float x, float y, float z)| -  Apply the transformation to an (x,y,z) normal.
 Use this if you are programming in python, tcl or Java.

\item  \verb|float = obj.TransformFloatNormal (float normal[3])| -  Apply the transformation to an (x,y,z) normal.
 Use this if you are programming in python, tcl or Java.

\item  \verb|double = obj.TransformDoubleNormal (double x, double y, double z)| -  Apply the transformation to a double-precision (x,y,z) normal.
 Use this if you are programming in python, tcl or Java.

\item  \verb|double = obj.TransformDoubleNormal (double normal[3])| -  Apply the transformation to a double-precision (x,y,z) normal.
 Use this if you are programming in python, tcl or Java.

\item  \verb|double = obj.TransformVector (double x, double y, double z)| -  Synonymous with TransformDoubleVector(x,y,z).
 Use this if you are programming in python, tcl or Java.

\item  \verb|double = obj.TransformVector (double normal[3])| -  Synonymous with TransformDoubleVector(x,y,z).
 Use this if you are programming in python, tcl or Java.

\item  \verb|obj.TransformVector (float in[3], float out[3])| -  Apply the transformation to a vector.
 You can use the same array to store both the input and output.

\item  \verb|obj.TransformVector (double in[3], double out[3])| -  Apply the transformation to a double-precision vector.
 You can use the same array to store both the input and output.

\item  \verb|float = obj.TransformFloatVector (float x, float y, float z)| -  Apply the transformation to an (x,y,z) vector.
 Use this if you are programming in python, tcl or Java.

\item  \verb|float = obj.TransformFloatVector (float vec[3])| -  Apply the transformation to an (x,y,z) vector.
 Use this if you are programming in python, tcl or Java.

\item  \verb|double = obj.TransformDoubleVector (double x, double y, double z)| -  Apply the transformation to a double-precision (x,y,z) vector.
 Use this if you are programming in python, tcl or Java.

\item  \verb|double = obj.TransformDoubleVector (double vec[3])| -  Apply the transformation to a double-precision (x,y,z) vector.
 Use this if you are programming in python, tcl or Java.

\item  \verb|obj.TransformPoints (vtkPoints inPts, vtkPoints outPts)| -  Apply the transformation to a series of points, and append the
 results to outPts.  

\item  \verb|obj.TransformNormals (vtkDataArray inNms, vtkDataArray outNms)| -  Apply the transformation to a series of normals, and append the
 results to outNms.  

\item  \verb|obj.TransformVectors (vtkDataArray inVrs, vtkDataArray outVrs)| -  Apply the transformation to a series of vectors, and append the
 results to outVrs.  

\item  \verb|obj.TransformPointsNormalsVectors (vtkPoints inPts, vtkPoints outPts, vtkDataArray inNms, vtkDataArray outNms, vtkDataArray inVrs, vtkDataArray outVrs)| -  Apply the transformation to a combination of points, normals
 and vectors.  

\item  \verb|vtkLinearTransform = obj.GetLinearInverse ()| -  This will calculate the transformation without calling Update.
 Meant for use only within other VTK classes.

\item  \verb|obj.InternalTransformPoint (float in[3], float out[3])| -  This will calculate the transformation without calling Update.
 Meant for use only within other VTK classes.

\item  \verb|obj.InternalTransformPoint (double in[3], double out[3])| -  This will calculate the transformation without calling Update.
 Meant for use only within other VTK classes.

\item  \verb|obj.InternalTransformNormal (float in[3], float out[3])| -  This will calculate the transformation without calling Update.
 Meant for use only within other VTK classes.

\item  \verb|obj.InternalTransformNormal (double in[3], double out[3])| -  This will calculate the transformation without calling Update.
 Meant for use only within other VTK classes.

\item  \verb|obj.InternalTransformVector (float in[3], float out[3])| -  This will calculate the transformation without calling Update.
 Meant for use only within other VTK classes.

\item  \verb|obj.InternalTransformVector (double in[3], double out[3])| -  This will calculate the transformation without calling Update.
 Meant for use only within other VTK classes.

\end{itemize}
