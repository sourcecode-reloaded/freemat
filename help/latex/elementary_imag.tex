\section{IMAG Imaginary Function}

\subsection{Usage}

Returns the imaginary part of the input array for all elements.  The 
general syntax for its use is
\begin{verbatim}
   y = imag(x)
\end{verbatim}
where \verb|x| is an \verb|n|-dimensional array of numerical type.  The output 
is the same numerical type as the input, unless the input is \verb|complex|
or \verb|dcomplex|.  For \verb|complex| inputs, the imaginary part is a floating
point array, so that the return type is \verb|float|.  For \verb|dcomplex|
inputs, the imaginary part is a double precision floating point array, so that
the return type is \verb|double|.  The \verb|imag| function returns zeros for 
real and integer types.
\subsection{Example}

The following demonstrates \verb|imag| applied to a complex scalar.
\begin{verbatim}
--> imag(3+4*i)

ans = 

 4 
\end{verbatim}
The imaginary part of real and integer arguments is a vector of zeros, the
same type and size of the argument.
\begin{verbatim}
--> imag([2,4,5,6])

ans = 

 0 0 0 0 
\end{verbatim}
For a double-precision complex array,
\begin{verbatim}
--> imag([2.0+3.0*i,i])

ans = 

 3 1 
\end{verbatim}
