\section{vtkPolygonalSurfaceContourLineInterpolator}

\subsection{Usage}

 vtkPolygonalSurfaceContourLineInterpolator interpolates and places
 contour points on polygonal surfaces. The class interpolates nodes by
 computing a \em graph \em geodesic lying on the polygonal data. By \em 
 graph \em Geodesic, we mean that the line interpolating the two end
 points traverses along on the mesh edges so as to form the shortest
 path. A Dijkstra algorithm is used to compute the path. See
 vtkDijkstraGraphGeodesicPath.
 
 The class is mean to be used in conjunction with 
 vtkPolygonalSurfacePointPlacer. The reason for this weak coupling is a 
 performance issue, both classes need to perform a cell pick, and 
 coupling avoids multiple cell picks (cell picks are slow).
 

To create an instance of class vtkPolygonalSurfaceContourLineInterpolator, simply
invoke its constructor as follows
\begin{verbatim}
  obj = vtkPolygonalSurfaceContourLineInterpolator
\end{verbatim}
\subsection{Methods}

The class vtkPolygonalSurfaceContourLineInterpolator has several methods that can be used.
  They are listed below.
Note that the documentation is translated automatically from the VTK sources,
and may not be completely intelligible.  When in doubt, consult the VTK website.
In the methods listed below, \verb|obj| is an instance of the vtkPolygonalSurfaceContourLineInterpolator class.
\begin{itemize}
\item  \verb|string = obj.GetClassName ()| -  Standard methods for instances of this class.

\item  \verb|int = obj.IsA (string name)| -  Standard methods for instances of this class.

\item  \verb|vtkPolygonalSurfaceContourLineInterpolator = obj.NewInstance ()| -  Standard methods for instances of this class.

\item  \verb|vtkPolygonalSurfaceContourLineInterpolator = obj.SafeDownCast (vtkObject o)| -  Standard methods for instances of this class.

\item  \verb|int = obj.InterpolateLine (vtkRenderer ren, vtkContourRepresentation rep, int idx1, int idx2)| -  Subclasses that wish to interpolate a line segment must implement this.
 For instance vtkBezierContourLineInterpolator adds nodes between idx1
 and idx2, that allow the contour to adhere to a bezier curve.

\item  \verb|int = obj.UpdateNode (vtkRenderer , vtkContourRepresentation , double , int )| -  The interpolator is given a chance to update the node.
 vtkImageContourLineInterpolator updates the idx'th node in the contour, 
 so it automatically sticks to edges in the vicinity as the user 
 constructs the contour. 
 Returns 0 if the node (world position) is unchanged.

\item  \verb|obj.SetDistanceOffset (double )| -  Height offset at which points may be placed on the polygonal surface.
 If you specify a non-zero value here, be sure to have computed vertex
 normals on your input polygonal data. (easily done with 
 vtkPolyDataNormals).

\item  \verb|double = obj.GetDistanceOffset ()| -  Height offset at which points may be placed on the polygonal surface.
 If you specify a non-zero value here, be sure to have computed vertex
 normals on your input polygonal data. (easily done with 
 vtkPolyDataNormals).

\end{itemize}
