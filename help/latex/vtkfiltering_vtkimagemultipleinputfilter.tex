\section{vtkImageMultipleInputFilter}

\subsection{Usage}

 vtkImageMultipleInputFilter is a super class for filters that 
 have any number of inputs. Streaming is not available in this class yet.

To create an instance of class vtkImageMultipleInputFilter, simply
invoke its constructor as follows
\begin{verbatim}
  obj = vtkImageMultipleInputFilter
\end{verbatim}
\subsection{Methods}

The class vtkImageMultipleInputFilter has several methods that can be used.
  They are listed below.
Note that the documentation is translated automatically from the VTK sources,
and may not be completely intelligible.  When in doubt, consult the VTK website.
In the methods listed below, \verb|obj| is an instance of the vtkImageMultipleInputFilter class.
\begin{itemize}
\item  \verb|string = obj.GetClassName ()|

\item  \verb|int = obj.IsA (string name)|

\item  \verb|vtkImageMultipleInputFilter = obj.NewInstance ()|

\item  \verb|vtkImageMultipleInputFilter = obj.SafeDownCast (vtkObject o)|

\item  \verb|obj.SetInput (int num, vtkImageData input)| -  Set an Input of this filter. 

\item  \verb|obj.AddInput (vtkImageData input)| -  Adds an input to the first null position in the input list.
 Expands the list memory if necessary

\item  \verb|obj.RemoveInput (vtkImageData input)| -  Adds an input to the first null position in the input list.
 Expands the list memory if necessary

\item  \verb|vtkImageData = obj.GetInput (int num)| -  Get one input to this filter.

\item  \verb|vtkImageData = obj.GetInput ()| -  Get one input to this filter.

\item  \verb|obj.SetBypass (int )| -  Turning bypass on will cause the filter to turn off and
 simply pass the data from the first input (input0) through.  
 It is implemented for consistency with vtkImageToImageFilter.

\item  \verb|int = obj.GetBypass ()| -  Turning bypass on will cause the filter to turn off and
 simply pass the data from the first input (input0) through.  
 It is implemented for consistency with vtkImageToImageFilter.

\item  \verb|obj.BypassOn ()| -  Turning bypass on will cause the filter to turn off and
 simply pass the data from the first input (input0) through.  
 It is implemented for consistency with vtkImageToImageFilter.

\item  \verb|obj.BypassOff ()| -  Turning bypass on will cause the filter to turn off and
 simply pass the data from the first input (input0) through.  
 It is implemented for consistency with vtkImageToImageFilter.

\item  \verb|obj.SetNumberOfThreads (int )| -  Get/Set the number of threads to create when rendering

\item  \verb|int = obj.GetNumberOfThreadsMinValue ()| -  Get/Set the number of threads to create when rendering

\item  \verb|int = obj.GetNumberOfThreadsMaxValue ()| -  Get/Set the number of threads to create when rendering

\item  \verb|int = obj.GetNumberOfThreads ()| -  Get/Set the number of threads to create when rendering

\item  \verb|int = obj.SplitExtent (int splitExt[6], int startExt[6], int num, int total)| -  Putting this here until I merge graphics and imaging streaming.

\end{itemize}
