\section{vtkTensorGlyph}

\subsection{Usage}

 vtkTensorGlyph is a filter that copies a geometric representation
 (specified as polygonal data) to every input point. The geometric
 representation, or glyph, can be scaled and/or rotated according to
 the tensor at the input point. Scaling and rotation is controlled
 by the eigenvalues/eigenvectors of the tensor as follows. For each
 tensor, the eigenvalues (and associated eigenvectors) are sorted to
 determine the major, medium, and minor eigenvalues/eigenvectors.

 If the boolean variable ThreeGlyphs is not set the major eigenvalue 
 scales the glyph in the x-direction, the medium in the y-direction, 
 and the minor in the  z-direction. Then, the glyph is rotated so 
 that the glyph's local x-axis lies along the major eigenvector, 
 y-axis along the medium eigenvector, and z-axis along the minor. 

 If the boolean variable ThreeGlyphs is set three glyphs are produced, 
 each of them oriented along an eigenvector and scaled according to the 
 corresponding eigenvector.

 If the boolean variable Symmetric is set each glyph is mirrored (2 or 6 
 glyphs will be produced)

 The x-axis of the source glyph will correspond to the eigenvector 
 on output. Point (0,0,0) in the source will be placed in the data point.
 Variable Length will normally correspond to the distance from the 
 origin to the tip of the source glyph along the x-axis, 
 but can be changed to produce other results when Symmetric is on,
 e.g. glyphs that do not touch or that overlap.

 Please note that when Symmetric is false it will generally be better 
 to place the source glyph from (-0.5,0,0) to (0.5,0,0), i.e. centred
 at the origin. When symmetric is true the placement from (0,0,0) to
 (1,0,0) will generally be more convenient.

 A scale factor is provided to control the amount of scaling. Also, you 
 can turn off scaling completely if desired. The boolean variable 
 ClampScaling controls the maximum scaling (in conjunction with
 MaxScaleFactor.) This is useful in certain applications where 
 singularities or large order of magnitude differences exist in 
 the eigenvalues.

 If the boolean variable ColorGlyphs is set to true the glyphs are
 colored.  The glyphs can be colored using the input scalars
 (SetColorModeToScalars), which is the default, or colored using the
 eigenvalues (SetColorModeToEigenvalues).

 Another instance variable, ExtractEigenvalues, has been provided to
 control extraction of eigenvalues/eigenvectors. If this boolean is
 false, then eigenvalues/eigenvectors are not extracted, and the
 columns of the tensor are taken as the eigenvectors (the norm of
 column, always positive, is the eigenvalue).  This allows
 additional capability over the vtkGlyph3D object. That is, the
 glyph can be oriented in three directions instead of one.

To create an instance of class vtkTensorGlyph, simply
invoke its constructor as follows
\begin{verbatim}
  obj = vtkTensorGlyph
\end{verbatim}
\subsection{Methods}

The class vtkTensorGlyph has several methods that can be used.
  They are listed below.
Note that the documentation is translated automatically from the VTK sources,
and may not be completely intelligible.  When in doubt, consult the VTK website.
In the methods listed below, \verb|obj| is an instance of the vtkTensorGlyph class.
\begin{itemize}
\item  \verb|string = obj.GetClassName ()|

\item  \verb|int = obj.IsA (string name)|

\item  \verb|vtkTensorGlyph = obj.NewInstance ()|

\item  \verb|vtkTensorGlyph = obj.SafeDownCast (vtkObject o)|

\item  \verb|obj.SetSource (vtkPolyData source)| -  Specify the geometry to copy to each point. Old style. See
 SetSourceConnection.

\item  \verb|vtkPolyData = obj.GetSource ()| -  Specify the geometry to copy to each point. Old style. See
 SetSourceConnection.

\item  \verb|obj.SetSourceConnection (int id, vtkAlgorithmOutput algOutput)| -  Specify a source object at a specified table location. New style.
 Source connection is stored in port 1. This method is equivalent
 to SetInputConnection(1, id, outputPort).

\item  \verb|obj.SetSourceConnection (vtkAlgorithmOutput algOutput)| -  Turn on/off scaling of glyph with eigenvalues.

\item  \verb|obj.SetScaling (int )| -  Turn on/off scaling of glyph with eigenvalues.

\item  \verb|int = obj.GetScaling ()| -  Turn on/off scaling of glyph with eigenvalues.

\item  \verb|obj.ScalingOn ()| -  Turn on/off scaling of glyph with eigenvalues.

\item  \verb|obj.ScalingOff ()| -  Turn on/off scaling of glyph with eigenvalues.

\item  \verb|obj.SetScaleFactor (double )| -  Specify scale factor to scale object by. (Scale factor always affects
 output even if scaling is off.)

\item  \verb|double = obj.GetScaleFactor ()| -  Specify scale factor to scale object by. (Scale factor always affects
 output even if scaling is off.)

\item  \verb|obj.SetThreeGlyphs (int )| -  Turn on/off drawing three glyphs

\item  \verb|int = obj.GetThreeGlyphs ()| -  Turn on/off drawing three glyphs

\item  \verb|obj.ThreeGlyphsOn ()| -  Turn on/off drawing three glyphs

\item  \verb|obj.ThreeGlyphsOff ()| -  Turn on/off drawing three glyphs

\item  \verb|obj.SetSymmetric (int )| -  Turn on/off drawing a mirror of each glyph

\item  \verb|int = obj.GetSymmetric ()| -  Turn on/off drawing a mirror of each glyph

\item  \verb|obj.SymmetricOn ()| -  Turn on/off drawing a mirror of each glyph

\item  \verb|obj.SymmetricOff ()| -  Turn on/off drawing a mirror of each glyph

\item  \verb|obj.SetLength (double )| -  Set/Get the distance, along x, from the origin to the end of the 
 source glyph. It is used to draw the symmetric glyphs.

\item  \verb|double = obj.GetLength ()| -  Set/Get the distance, along x, from the origin to the end of the 
 source glyph. It is used to draw the symmetric glyphs.

\item  \verb|obj.SetExtractEigenvalues (int )| -  Turn on/off extraction of eigenvalues from tensor.

\item  \verb|obj.ExtractEigenvaluesOn ()| -  Turn on/off extraction of eigenvalues from tensor.

\item  \verb|obj.ExtractEigenvaluesOff ()| -  Turn on/off extraction of eigenvalues from tensor.

\item  \verb|int = obj.GetExtractEigenvalues ()| -  Turn on/off extraction of eigenvalues from tensor.

\item  \verb|obj.SetColorGlyphs (int )| -  Turn on/off coloring of glyph with input scalar data or
 eigenvalues. If false, or input scalar data not present, then the
 scalars from the source object are passed through the filter.

\item  \verb|int = obj.GetColorGlyphs ()| -  Turn on/off coloring of glyph with input scalar data or
 eigenvalues. If false, or input scalar data not present, then the
 scalars from the source object are passed through the filter.

\item  \verb|obj.ColorGlyphsOn ()| -  Turn on/off coloring of glyph with input scalar data or
 eigenvalues. If false, or input scalar data not present, then the
 scalars from the source object are passed through the filter.

\item  \verb|obj.ColorGlyphsOff ()| -  Turn on/off coloring of glyph with input scalar data or
 eigenvalues. If false, or input scalar data not present, then the
 scalars from the source object are passed through the filter.

\item  \verb|obj.SetColorMode (int )| -  Set the color mode to be used for the glyphs.  This can be set to
 use the input scalars (default) or to use the eigenvalues at the
 point.  If ThreeGlyphs is set and the eigenvalues are chosen for
 coloring then each glyph is colored by the corresponding
 eigenvalue and if not set the color corresponding to the largest
 eigenvalue is chosen.  The recognized values are:
 COLOR\_BY\_SCALARS = 0 (default)
 COLOR\_BY\_EIGENVALUES = 1

\item  \verb|int = obj.GetColorModeMinValue ()| -  Set the color mode to be used for the glyphs.  This can be set to
 use the input scalars (default) or to use the eigenvalues at the
 point.  If ThreeGlyphs is set and the eigenvalues are chosen for
 coloring then each glyph is colored by the corresponding
 eigenvalue and if not set the color corresponding to the largest
 eigenvalue is chosen.  The recognized values are:
 COLOR\_BY\_SCALARS = 0 (default)
 COLOR\_BY\_EIGENVALUES = 1

\item  \verb|int = obj.GetColorModeMaxValue ()| -  Set the color mode to be used for the glyphs.  This can be set to
 use the input scalars (default) or to use the eigenvalues at the
 point.  If ThreeGlyphs is set and the eigenvalues are chosen for
 coloring then each glyph is colored by the corresponding
 eigenvalue and if not set the color corresponding to the largest
 eigenvalue is chosen.  The recognized values are:
 COLOR\_BY\_SCALARS = 0 (default)
 COLOR\_BY\_EIGENVALUES = 1

\item  \verb|int = obj.GetColorMode ()| -  Set the color mode to be used for the glyphs.  This can be set to
 use the input scalars (default) or to use the eigenvalues at the
 point.  If ThreeGlyphs is set and the eigenvalues are chosen for
 coloring then each glyph is colored by the corresponding
 eigenvalue and if not set the color corresponding to the largest
 eigenvalue is chosen.  The recognized values are:
 COLOR\_BY\_SCALARS = 0 (default)
 COLOR\_BY\_EIGENVALUES = 1

\item  \verb|obj.SetColorModeToScalars ()| -  Set the color mode to be used for the glyphs.  This can be set to
 use the input scalars (default) or to use the eigenvalues at the
 point.  If ThreeGlyphs is set and the eigenvalues are chosen for
 coloring then each glyph is colored by the corresponding
 eigenvalue and if not set the color corresponding to the largest
 eigenvalue is chosen.  The recognized values are:
 COLOR\_BY\_SCALARS = 0 (default)
 COLOR\_BY\_EIGENVALUES = 1

\item  \verb|obj.SetColorModeToEigenvalues ()| -  Set the color mode to be used for the glyphs.  This can be set to
 use the input scalars (default) or to use the eigenvalues at the
 point.  If ThreeGlyphs is set and the eigenvalues are chosen for
 coloring then each glyph is colored by the corresponding
 eigenvalue and if not set the color corresponding to the largest
 eigenvalue is chosen.  The recognized values are:
 COLOR\_BY\_SCALARS = 0 (default)
 COLOR\_BY\_EIGENVALUES = 1

\item  \verb|obj.SetClampScaling (int )| -  Turn on/off scalar clamping. If scalar clamping is on, the ivar
 MaxScaleFactor is used to control the maximum scale factor. (This is
 useful to prevent uncontrolled scaling near singularities.)

\item  \verb|int = obj.GetClampScaling ()| -  Turn on/off scalar clamping. If scalar clamping is on, the ivar
 MaxScaleFactor is used to control the maximum scale factor. (This is
 useful to prevent uncontrolled scaling near singularities.)

\item  \verb|obj.ClampScalingOn ()| -  Turn on/off scalar clamping. If scalar clamping is on, the ivar
 MaxScaleFactor is used to control the maximum scale factor. (This is
 useful to prevent uncontrolled scaling near singularities.)

\item  \verb|obj.ClampScalingOff ()| -  Turn on/off scalar clamping. If scalar clamping is on, the ivar
 MaxScaleFactor is used to control the maximum scale factor. (This is
 useful to prevent uncontrolled scaling near singularities.)

\item  \verb|obj.SetMaxScaleFactor (double )| -  Set/Get the maximum allowable scale factor. This value is compared to the
 combination of the scale factor times the eigenvalue. If less, the scale
 factor is reset to the MaxScaleFactor. The boolean ClampScaling has to 
 be ''on'' for this to work.

\item  \verb|double = obj.GetMaxScaleFactor ()| -  Set/Get the maximum allowable scale factor. This value is compared to the
 combination of the scale factor times the eigenvalue. If less, the scale
 factor is reset to the MaxScaleFactor. The boolean ClampScaling has to 
 be ''on'' for this to work.

\end{itemize}
