\section{vtkXMLHierarchicalBoxDataReader}

\subsection{Usage}

 vtkXMLHierarchicalBoxDataReader reads the VTK XML hierarchical data file
 format. XML hierarchical data files are meta-files that point to a list
 of serial VTK XML files. When reading in parallel, it will distribute
 sub-blocks among processor. If the number of sub-blocks is less than
 the number of processors, some processors will not have any sub-blocks
 for that level. If the number of sub-blocks is larger than the
 number of processors, each processor will possibly have more than
 1 sub-block.

To create an instance of class vtkXMLHierarchicalBoxDataReader, simply
invoke its constructor as follows
\begin{verbatim}
  obj = vtkXMLHierarchicalBoxDataReader
\end{verbatim}
\subsection{Methods}

The class vtkXMLHierarchicalBoxDataReader has several methods that can be used.
  They are listed below.
Note that the documentation is translated automatically from the VTK sources,
and may not be completely intelligible.  When in doubt, consult the VTK website.
In the methods listed below, \verb|obj| is an instance of the vtkXMLHierarchicalBoxDataReader class.
\begin{itemize}
\item  \verb|string = obj.GetClassName ()|

\item  \verb|int = obj.IsA (string name)|

\item  \verb|vtkXMLHierarchicalBoxDataReader = obj.NewInstance ()|

\item  \verb|vtkXMLHierarchicalBoxDataReader = obj.SafeDownCast (vtkObject o)|

\end{itemize}
