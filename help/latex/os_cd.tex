\section{CD Change Working Directory Function}

\subsection{Usage}

Changes the current working directory to the one specified as the argument.  The general syntax for its use is
\begin{verbatim}
  cd('dirname')
\end{verbatim}
but this can also be expressed as
\begin{verbatim}
  cd 'dirname'
\end{verbatim}
or 
\begin{verbatim}
  cd dirname
\end{verbatim}
Examples of all three usages are given below.
Generally speaking, \verb|dirname| is any string that would be accepted 
by the underlying OS as a valid directory name.  For example, on most 
systems, \verb|'.'| refers to the current directory, and \verb|'..'| refers 
to the parent directory.  Also, depending on the OS, it may be necessary 
to ``escape'' the directory seperators.  In particular, if directories 
are seperated with the backwards-slash character \verb|'\\'|, then the 
path specification must use double-slashes \verb|'\\\\'|. Note: to get 
file-name completion to work at this time, you must use one of the 
first two forms of the command.

\subsection{Example}

The \verb|pwd| command returns the current directory location.  First, 
we use the simplest form of the \verb|cd| command, in which the directory 
name argument is given unquoted.
\begin{verbatim}
--> pwd

ans = 
/home/basu/dev/branches/FreeMat4/help/tmp
--> cd ..
--> pwd

ans = 
/home/basu/dev/branches/FreeMat4/help
\end{verbatim}
Next, we use the ``traditional'' form of the function call, using 
both the parenthesis and a variable to store the quoted string.
\begin{verbatim}
--> a = pwd;
--> cd(a)
--> pwd

ans = 
/home/basu/dev/branches/FreeMat4/help/tmp
\end{verbatim}
