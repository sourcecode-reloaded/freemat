\section{SUBSASGN Overloaded Class Assignment}

\subsection{Usage}

This method is called for expressions of the form
\begin{verbatim}
  a(b) = c, a{b} = c, a.b = c
\end{verbatim}
and overloading the \verb|subsasgn| method can allow you
to define the meaning of these expressions for
objects of class \verb|a|.  These expressions are mapped
to a call of the form
\begin{verbatim}
  a = subsasgn(a,s,b)
\end{verbatim}
where \verb|s| is a structure array with two fields.  The
first field is
\begin{itemize}
\item  \verb|type|  is a string containing either \verb|'()'| or
 \verb|'{}'| or \verb|'.'| depending on the form of the call.

\item  \verb|subs| is a cell array or string containing the
 the subscript information.

\end{itemize}
When multiple indexing experssions are combined together
such as \verb|a(5).foo{:} = b|, the \verb|s| array contains
the following entries
\begin{verbatim}
  s(1).type = '()'  s(1).subs = {5}
  s(2).type = '.'   s(2).subs = 'foo'
  s(3).type = '{}'  s(3).subs = ':'
\end{verbatim}
