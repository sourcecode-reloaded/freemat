\section{TIMES Matrix Multiply Operator}

\subsection{Usage}

Multiplies two numerical arrays.  This operator is really a combination
of three operators, all of which have the same general syntax:
\begin{verbatim}
  y = a * b
\end{verbatim}
where \verb|a| and \verb|b| are arrays of numerical type.  The result \verb|y| depends
on which of the following three situations applies to the arguments
\verb|a| and \verb|b|:
\begin{enumerate}
\item  \verb|a| is a scalar, \verb|b| is an arbitrary \verb|n|-dimensional numerical array, in which case the output is the element-wise product of \verb|b| with the scalar \verb|a|.

\item  \verb|b| is a scalar, \verb|a| is an arbitrary \verb|n|-dimensional numerical array, in which case the output is the element-wise product of \verb|a| with the scalar \verb|b|.

\item  \verb|a,b| are conformant matrices, i.e., \verb|a| is of size \verb|M x K|, and \verb|b| is of size \verb|K x N|, in which case the output is of size \verb|M x N| and is the matrix product of \verb|a|, and \verb|b|.

\end{enumerate}
Matrix multiplication is only defined for matrices of type \verb|double| 
and \verb|single|.
\subsection{Function Internals}

There are three formulae for the times operator.  For the first form
\[
y(m_1,\ldots,m_d) = a \times b(m_1,\ldots,m_d),
\]
and the second form
\[
y(m_1,\ldots,m_d) = a(m_1,\ldots,m_d) \times b.
\]
In the third form, the output is the matrix product of the arguments
\[
y(m,n) = \sum_{k=1}^{K} a(m,k) b(k,n)
\]
\subsection{Examples}

Here are some examples of using the matrix multiplication operator.  First,
the scalar examples (types 1 and 2 from the list above):
@>
The matrix form, where the first argument is \verb|2 x 3|, and the
second argument is \verb|3 x 1|, so that the product is size 
\verb|2 x 1|.
@>
Note that the output is double precision.
