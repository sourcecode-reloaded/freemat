\section{vtkImporter}

\subsection{Usage}

 vtkImporter is an abstract class that specifies the protocol for
 importing actors, cameras, lights and properties into a
 vtkRenderWindow. The following takes place:
 1) Create a RenderWindow and Renderer if none is provided.
 2) Call ImportBegin, if ImportBegin returns False, return
 3) Call ReadData, which calls:
  a) Import the Actors
  b) Import the cameras
  c) Import the lights
  d) Import the Properties
 7) Call ImportEnd

 Subclasses optionally implement the ImportActors, ImportCameras,
 ImportLights and ImportProperties or ReadData methods. An ImportBegin and
 ImportEnd can optionally be provided to perform Importer-specific
 initialization and termination.  The Read method initiates the import
 process. If a RenderWindow is provided, its Renderer will contained
 the imported objects. If the RenderWindow has no Renderer, one is
 created. If no RenderWindow is provided, both a RenderWindow and
 Renderer will be created. Both the RenderWindow and Renderer can be
 accessed using Get methods.

To create an instance of class vtkImporter, simply
invoke its constructor as follows
\begin{verbatim}
  obj = vtkImporter
\end{verbatim}
\subsection{Methods}

The class vtkImporter has several methods that can be used.
  They are listed below.
Note that the documentation is translated automatically from the VTK sources,
and may not be completely intelligible.  When in doubt, consult the VTK website.
In the methods listed below, \verb|obj| is an instance of the vtkImporter class.
\begin{itemize}
\item  \verb|string = obj.GetClassName ()|

\item  \verb|int = obj.IsA (string name)|

\item  \verb|vtkImporter = obj.NewInstance ()|

\item  \verb|vtkImporter = obj.SafeDownCast (vtkObject o)|

\item  \verb|vtkRenderer = obj.GetRenderer ()|

\item  \verb|obj.SetRenderWindow (vtkRenderWindow )|

\item  \verb|vtkRenderWindow = obj.GetRenderWindow ()|

\item  \verb|obj.Read ()|

\item  \verb|obj.Update ()|

\end{itemize}
