\section{LU LU Decomposition for Matrices}

\subsection{Usage}

Computes the LU decomposition for a matrix.  The form of the
command depends on the type of the argument.  For full (non-sparse)
matrices, the primary form for \verb|lu| is
\begin{verbatim}
   [L,U,P] = lu(A),
\end{verbatim}
where \verb|L| is lower triangular, \verb|U| is upper triangular, and
\verb|P| is a permutation matrix such that \verb|L*U = P*A|.  The second form is
\begin{verbatim}
   [V,U] = lu(A),
\end{verbatim}
where \verb|V| is \verb|P'*L| (a row-permuted lower triangular matrix), 
and \verb|U| is upper triangular.  For sparse, square matrices,
the LU decomposition has the following form:
\begin{verbatim}
   [L,U,P,Q,R] = lu(A),
\end{verbatim}
where \verb|A| is a sparse matrix of either \verb|double| or \verb|dcomplex| type.
The matrices are such that \verb|L*U=P*R*A*Q|, where \verb|L| is a lower triangular
matrix, \verb|U| is upper triangular, \verb|P| and \verb|Q| are permutation vectors
and \verb|R| is a diagonal matrix of row scaling factors.  The decomposition
 is computed using UMFPACK for sparse matrices, and LAPACK for dense
 matrices.
\subsection{Example}

First, we compute the LU decomposition of a dense matrix.
@>
Now we repeat the exercise with a sparse matrix, and demonstrate
the use of the permutation vectors.
@>
