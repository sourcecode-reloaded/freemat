\section{PERMUTE Array Permutation Function}

\subsection{Usage}

The \verb|permute| function rearranges the contents of an array according
to the specified permutation vector.  The syntax for its use is
\begin{verbatim}
    y = permute(x,p)
\end{verbatim}
where \verb|p| is a permutation vector - i.e., a vector containing the 
integers \verb|1...ndims(x)| each occuring exactly once.  The resulting
array \verb|y| contains the same data as the array \verb|x|, but ordered
according to the permutation.  This function is a generalization of
the matrix transpose operation.
\subsection{Example}

Here we use \verb|permute| to transpose a simple matrix (note that permute
also works for sparse matrices):
\begin{verbatim}
--> A = [1,2;4,5]

A = 
 1 2 
 4 5 

--> permute(A,[2,1])

ans = 
 1 4 
 2 5 

--> A'

ans = 
 1 4 
 2 5 
\end{verbatim}
Now we permute a larger n-dimensional array:
\begin{verbatim}
--> A = randn(13,5,7,2);
--> size(A)

ans = 
 13  5  7  2 

--> B = permute(A,[3,4,2,1]);
--> size(B)

ans = 
  7  2  5 13 
\end{verbatim}
