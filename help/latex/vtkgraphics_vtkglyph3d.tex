\section{vtkGlyph3D}

\subsection{Usage}

 vtkGlyph3D is a filter that copies a geometric representation (called
 a glyph) to every point in the input dataset. The glyph is defined with
 polygonal data from a source filter input. The glyph may be oriented
 along the input vectors or normals, and it may be scaled according to
 scalar data or vector magnitude. More than one glyph may be used by
 creating a table of source objects, each defining a different glyph. If a
 table of glyphs is defined, then the table can be indexed into by using
 either scalar value or vector magnitude.
 
 To use this object you'll have to provide an input dataset and a source
 to define the glyph. Then decide whether you want to scale the glyph and
 how to scale the glyph (using scalar value or vector magnitude). Next
 decide whether you want to orient the glyph, and whether to use the
 vector data or normal data to orient it. Finally, decide whether to use a
 table of glyphs, or just a single glyph. If you use a table of glyphs,
 you'll have to decide whether to index into it with scalar value or with
 vector magnitude.
 

To create an instance of class vtkGlyph3D, simply
invoke its constructor as follows
\begin{verbatim}
  obj = vtkGlyph3D
\end{verbatim}
\subsection{Methods}

The class vtkGlyph3D has several methods that can be used.
  They are listed below.
Note that the documentation is translated automatically from the VTK sources,
and may not be completely intelligible.  When in doubt, consult the VTK website.
In the methods listed below, \verb|obj| is an instance of the vtkGlyph3D class.
\begin{itemize}
\item  \verb|string = obj.GetClassName ()|

\item  \verb|int = obj.IsA (string name)|

\item  \verb|vtkGlyph3D = obj.NewInstance ()|

\item  \verb|vtkGlyph3D = obj.SafeDownCast (vtkObject o)|

\item  \verb|obj.SetSource (vtkPolyData pd)| -  Set the source to use for he glyph. Old style. See SetSourceConnection.

\item  \verb|obj.SetSource (int id, vtkPolyData pd)| -  Specify a source object at a specified table location.
 Old style. See SetSourceConnection.

\item  \verb|obj.SetSourceConnection (int id, vtkAlgorithmOutput algOutput)| -  Specify a source object at a specified table location. New style.
 Source connection is stored in port 1. This method is equivalent
 to SetInputConnection(1, id, outputPort).

\item  \verb|obj.SetSourceConnection (vtkAlgorithmOutput algOutput)| -  Get a pointer to a source object at a specified table location.

\item  \verb|vtkPolyData = obj.GetSource (int id)| -  Get a pointer to a source object at a specified table location.

\item  \verb|obj.SetScaling (int )| -  Turn on/off scaling of source geometry.

\item  \verb|obj.ScalingOn ()| -  Turn on/off scaling of source geometry.

\item  \verb|obj.ScalingOff ()| -  Turn on/off scaling of source geometry.

\item  \verb|int = obj.GetScaling ()| -  Turn on/off scaling of source geometry.

\item  \verb|obj.SetScaleMode (int )| -  Either scale by scalar or by vector/normal magnitude.

\item  \verb|int = obj.GetScaleMode ()| -  Either scale by scalar or by vector/normal magnitude.

\item  \verb|obj.SetScaleModeToScaleByScalar ()| -  Either scale by scalar or by vector/normal magnitude.

\item  \verb|obj.SetScaleModeToScaleByVector ()| -  Either scale by scalar or by vector/normal magnitude.

\item  \verb|obj.SetScaleModeToScaleByVectorComponents ()| -  Either scale by scalar or by vector/normal magnitude.

\item  \verb|obj.SetScaleModeToDataScalingOff ()| -  Either scale by scalar or by vector/normal magnitude.

\item  \verb|string = obj.GetScaleModeAsString ()| -  Either scale by scalar or by vector/normal magnitude.

\item  \verb|obj.SetColorMode (int )| -  Either color by scale, scalar or by vector/normal magnitude.

\item  \verb|int = obj.GetColorMode ()| -  Either color by scale, scalar or by vector/normal magnitude.

\item  \verb|obj.SetColorModeToColorByScale ()| -  Either color by scale, scalar or by vector/normal magnitude.

\item  \verb|obj.SetColorModeToColorByScalar ()| -  Either color by scale, scalar or by vector/normal magnitude.

\item  \verb|obj.SetColorModeToColorByVector ()| -  Either color by scale, scalar or by vector/normal magnitude.

\item  \verb|string = obj.GetColorModeAsString ()| -  Either color by scale, scalar or by vector/normal magnitude.

\item  \verb|obj.SetScaleFactor (double )| -  Specify scale factor to scale object by.

\item  \verb|double = obj.GetScaleFactor ()| -  Specify scale factor to scale object by.

\item  \verb|obj.SetRange (double , double )| -  Specify range to map scalar values into.

\item  \verb|obj.SetRange (double  a[2])| -  Specify range to map scalar values into.

\item  \verb|double = obj. GetRange ()| -  Specify range to map scalar values into.

\item  \verb|obj.SetOrient (int )| -  Turn on/off orienting of input geometry along vector/normal.

\item  \verb|obj.OrientOn ()| -  Turn on/off orienting of input geometry along vector/normal.

\item  \verb|obj.OrientOff ()| -  Turn on/off orienting of input geometry along vector/normal.

\item  \verb|int = obj.GetOrient ()| -  Turn on/off orienting of input geometry along vector/normal.

\item  \verb|obj.SetClamping (int )| -  Turn on/off clamping of ''scalar'' values to range. (Scalar value may be
  vector magnitude if ScaleByVector() is enabled.)

\item  \verb|obj.ClampingOn ()| -  Turn on/off clamping of ''scalar'' values to range. (Scalar value may be
  vector magnitude if ScaleByVector() is enabled.)

\item  \verb|obj.ClampingOff ()| -  Turn on/off clamping of ''scalar'' values to range. (Scalar value may be
  vector magnitude if ScaleByVector() is enabled.)

\item  \verb|int = obj.GetClamping ()| -  Turn on/off clamping of ''scalar'' values to range. (Scalar value may be
  vector magnitude if ScaleByVector() is enabled.)

\item  \verb|obj.SetVectorMode (int )| -  Specify whether to use vector or normal to perform vector operations.

\item  \verb|int = obj.GetVectorMode ()| -  Specify whether to use vector or normal to perform vector operations.

\item  \verb|obj.SetVectorModeToUseVector ()| -  Specify whether to use vector or normal to perform vector operations.

\item  \verb|obj.SetVectorModeToUseNormal ()| -  Specify whether to use vector or normal to perform vector operations.

\item  \verb|obj.SetVectorModeToVectorRotationOff ()| -  Specify whether to use vector or normal to perform vector operations.

\item  \verb|string = obj.GetVectorModeAsString ()| -  Specify whether to use vector or normal to perform vector operations.

\item  \verb|obj.SetIndexMode (int )| -  Index into table of sources by scalar, by vector/normal magnitude, or
 no indexing. If indexing is turned off, then the first source glyph in
 the table of glyphs is used. Note that indexing mode will only use the
 InputScalarsSelection array and not the InputColorScalarsSelection
 as the scalar source if an array is specified.

\item  \verb|int = obj.GetIndexMode ()| -  Index into table of sources by scalar, by vector/normal magnitude, or
 no indexing. If indexing is turned off, then the first source glyph in
 the table of glyphs is used. Note that indexing mode will only use the
 InputScalarsSelection array and not the InputColorScalarsSelection
 as the scalar source if an array is specified.

\item  \verb|obj.SetIndexModeToScalar ()| -  Index into table of sources by scalar, by vector/normal magnitude, or
 no indexing. If indexing is turned off, then the first source glyph in
 the table of glyphs is used. Note that indexing mode will only use the
 InputScalarsSelection array and not the InputColorScalarsSelection
 as the scalar source if an array is specified.

\item  \verb|obj.SetIndexModeToVector ()| -  Index into table of sources by scalar, by vector/normal magnitude, or
 no indexing. If indexing is turned off, then the first source glyph in
 the table of glyphs is used. Note that indexing mode will only use the
 InputScalarsSelection array and not the InputColorScalarsSelection
 as the scalar source if an array is specified.

\item  \verb|obj.SetIndexModeToOff ()| -  Index into table of sources by scalar, by vector/normal magnitude, or
 no indexing. If indexing is turned off, then the first source glyph in
 the table of glyphs is used. Note that indexing mode will only use the
 InputScalarsSelection array and not the InputColorScalarsSelection
 as the scalar source if an array is specified.

\item  \verb|string = obj.GetIndexModeAsString ()| -  Index into table of sources by scalar, by vector/normal magnitude, or
 no indexing. If indexing is turned off, then the first source glyph in
 the table of glyphs is used. Note that indexing mode will only use the
 InputScalarsSelection array and not the InputColorScalarsSelection
 as the scalar source if an array is specified.

\item  \verb|obj.SetGeneratePointIds (int )| -  Enable/disable the generation of point ids as part of the output. The
 point ids are the id of the input generating point. The point ids are
 stored in the output point field data and named ''InputPointIds''. Point
 generation is useful for debugging and pick operations.

\item  \verb|int = obj.GetGeneratePointIds ()| -  Enable/disable the generation of point ids as part of the output. The
 point ids are the id of the input generating point. The point ids are
 stored in the output point field data and named ''InputPointIds''. Point
 generation is useful for debugging and pick operations.

\item  \verb|obj.GeneratePointIdsOn ()| -  Enable/disable the generation of point ids as part of the output. The
 point ids are the id of the input generating point. The point ids are
 stored in the output point field data and named ''InputPointIds''. Point
 generation is useful for debugging and pick operations.

\item  \verb|obj.GeneratePointIdsOff ()| -  Enable/disable the generation of point ids as part of the output. The
 point ids are the id of the input generating point. The point ids are
 stored in the output point field data and named ''InputPointIds''. Point
 generation is useful for debugging and pick operations.

\item  \verb|obj.SetPointIdsName (string )| -  Set/Get the name of the PointIds array if generated. By default the Ids
 are named ''InputPointIds'', but this can be changed with this function.

\item  \verb|string = obj.GetPointIdsName ()| -  Set/Get the name of the PointIds array if generated. By default the Ids
 are named ''InputPointIds'', but this can be changed with this function.

\item  \verb|obj.SetFillCellData (int )| -  Enable/disable the generation of cell data as part of the output.
 The cell data at each cell will match the point data of the input
 at the glyphed point.

\item  \verb|int = obj.GetFillCellData ()| -  Enable/disable the generation of cell data as part of the output.
 The cell data at each cell will match the point data of the input
 at the glyphed point.

\item  \verb|obj.FillCellDataOn ()| -  Enable/disable the generation of cell data as part of the output.
 The cell data at each cell will match the point data of the input
 at the glyphed point.

\item  \verb|obj.FillCellDataOff ()| -  Enable/disable the generation of cell data as part of the output.
 The cell data at each cell will match the point data of the input
 at the glyphed point.

\item  \verb|int = obj.IsPointVisible (vtkDataSet , vtkIdType )| -  This can be overwritten by subclass to return 0 when a point is
 blanked. Default implementation is to always return 1;

\end{itemize}
