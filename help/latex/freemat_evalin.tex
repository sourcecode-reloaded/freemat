\section{EVALIN Evaluate a String in Workspace}

\subsection{Usage}

The \verb|evalin| function is similar to the \verb|eval| function, with an additional
argument up front that indicates the workspace that the expressions are to 
be evaluated in.  The various syntaxes for \verb|evalin| are:
\begin{verbatim}
   evalin(workspace,expression)
   x = evalin(workspace,expression)
   [x,y,z] = evalin(workspace,expression)
   evalin(workspace,try_clause,catch_clause)
   x = evalin(workspace,try_clause,catch_clause)
   [x,y,z] = evalin(workspace,try_clause,catch_clause)
\end{verbatim}
The argument \verb|workspace| must be either 'caller' or 'base'.  If it is
'caller', then the expression is evaluated in the caller's work space.
That does not mean the caller of \verb|evalin|, but the caller of the current
function or script.  On the other hand if the argument is 'base', then
the expression is evaluated in the base work space.   See \verb|eval| for
details on the use of each variation.
