\section{vtkExodusIIWriter}

\subsection{Usage}

     This is a vtkWriter that writes it's vtkUnstructuredGrid 
     input out to an Exodus II file.  Go to http://endo.sandia.gov/SEACAS/
     for more information about the Exodus II format.

     Exodus files contain much information that is not captured
     in a vtkUnstructuredGrid, such as time steps, information
     lines, node sets, and side sets.  This information can be
     stored in a vtkModelMetadata object.

     The vtkExodusReader and vtkPExodusReader can create
     a vtkModelMetadata object and embed it in a vtkUnstructuredGrid
     in a series of field arrays.  This writer searches for these
     field arrays and will use the metadata contained in them
     when creating the new Exodus II file. 

     You can also explicitly give the vtkExodusIIWriter a
     vtkModelMetadata object to use when writing the file.

     In the absence of the information provided by vtkModelMetadata,
     if this writer is not part of a parallel application, we will use
     reasonable defaults for all the values in the output Exodus file.
     If you don't provide a block ID element array, we'll create a
     block for each cell type that appears in the unstructured grid.

     However if this writer is part of a parallel application (hence 
     writing out a distributed Exodus file), then we need at the very 
     least a list of all the block IDs that appear in the file.  And 
     we need the element array of block IDs for the input unstructured grid.

     In the absense of a vtkModelMetadata object, you can also provide
     time step information which we will include in the output Exodus
     file.

  .SECTION Caveats
     If the input floating point field arrays and point locations are all
     floats or all doubles, this class will operate more efficiently.
     Mixing floats and doubles will slow you down, because Exodus II
     requires that we write only floats or only doubles.

     We use the terms ''point'' and ''node'' interchangeably.
     Also, we use the terms ''element'' and ''cell'' interchangeably.

To create an instance of class vtkExodusIIWriter, simply
invoke its constructor as follows
\begin{verbatim}
  obj = vtkExodusIIWriter
\end{verbatim}
\subsection{Methods}

The class vtkExodusIIWriter has several methods that can be used.
  They are listed below.
Note that the documentation is translated automatically from the VTK sources,
and may not be completely intelligible.  When in doubt, consult the VTK website.
In the methods listed below, \verb|obj| is an instance of the vtkExodusIIWriter class.
\begin{itemize}
\item  \verb|string = obj.GetClassName ()|

\item  \verb|int = obj.IsA (string name)|

\item  \verb|vtkExodusIIWriter = obj.NewInstance ()|

\item  \verb|vtkExodusIIWriter = obj.SafeDownCast (vtkObject o)|

\item  \verb|obj.SetModelMetadata (vtkModelMetadata )|

\item  \verb|vtkModelMetadata = obj.GetModelMetadata ()|

\item  \verb|obj.SetFileName (string )|

\item  \verb|string = obj.GetFileName ()|

\item  \verb|obj.SetStoreDoubles (int )|

\item  \verb|int = obj.GetStoreDoubles ()|

\item  \verb|obj.SetGhostLevel (int )|

\item  \verb|int = obj.GetGhostLevel ()|

\item  \verb|obj.SetWriteOutBlockIdArray (int )|

\item  \verb|int = obj.GetWriteOutBlockIdArray ()|

\item  \verb|obj.WriteOutBlockIdArrayOn ()|

\item  \verb|obj.WriteOutBlockIdArrayOff ()|

\item  \verb|obj.SetWriteOutGlobalNodeIdArray (int )|

\item  \verb|int = obj.GetWriteOutGlobalNodeIdArray ()|

\item  \verb|obj.WriteOutGlobalNodeIdArrayOn ()|

\item  \verb|obj.WriteOutGlobalNodeIdArrayOff ()|

\item  \verb|obj.SetWriteOutGlobalElementIdArray (int )|

\item  \verb|int = obj.GetWriteOutGlobalElementIdArray ()|

\item  \verb|obj.WriteOutGlobalElementIdArrayOn ()|

\item  \verb|obj.WriteOutGlobalElementIdArrayOff ()|

\item  \verb|obj.SetWriteAllTimeSteps (int )|

\item  \verb|int = obj.GetWriteAllTimeSteps ()|

\item  \verb|obj.WriteAllTimeStepsOn ()|

\item  \verb|obj.WriteAllTimeStepsOff ()|

\item  \verb|obj.SetBlockIdArrayName (string )|

\item  \verb|string = obj.GetBlockIdArrayName ()|

\end{itemize}
