\section{CELL Cell Array of Empty Matrices}

\subsection{Usage}

Creates a cell array of empty matrix entres.  Two seperate 
syntaxes are possible.  The first syntax specifies the array 
dimensions as a sequence of scalar dimensions:
\begin{verbatim}
   y = cell(d1,d2,...,dn).
\end{verbatim}
The resulting array has the given dimensions, and is filled with
all zeros.  The type of \verb|y| is \verb|cell|, a cell array.  
    
The second syntax specifies the array dimensions as a vector,
where each element in the vector specifies a dimension length:
\begin{verbatim}
   y = cell([d1,d2,...,dn]).
\end{verbatim}
This syntax is more convenient for calling \verb|zeros| using a 
variable for the argument.  In both cases, specifying only one
dimension results in a square matrix output.
\subsection{Example}

The following examples demonstrate generation of some zero arrays 
using the first form.
\begin{verbatim}
--> cell(2,3,2)

ans = 

(:,:,1) = 

 [0] [0] [0] 
 [0] [0] [0] 

(:,:,2) = 

 [0] [0] [0] 
 [0] [0] [0] 

--> cell(1,3)

ans = 

 [0] [0] [0] 
\end{verbatim}
The same expressions, using the second form.
\begin{verbatim}
--> cell([2,6])

ans = 

 [0] [0] [0] [0] [0] [0] 
 [0] [0] [0] [0] [0] [0] 

--> cell([1,3])

ans = 

 [0] [0] [0] 
\end{verbatim}
