\section{vtkLineWidget2}

\subsection{Usage}

 This 3D widget defines a straight line that can be interactively placed in
 a scene. The widget is assumed to consist of two parts: 1) two end points
 and 2) a straight line connecting the two points. (The representation
 paired with this widget determines the actual geometry of the widget.) The
 positioning of the two end points is facilitated by using vtkHandleWidgets
 to position the points.

 To use this widget, you generally pair it with a vtkLineRepresentation
 (or a subclass). Variuos options are available in the representation for 
 controlling how the widget appears, and how the widget functions.

 .SECTION Event Bindings
 By default, the widget responds to the following VTK events (i.e., it
 watches the vtkRenderWindowInteractor for these events):
 \begin{verbatim}
 If one of the two end points are selected:
   LeftButtonPressEvent - activate the associated handle widget
   LeftButtonReleaseEvent - release the handle widget associated with the point
   MouseMoveEvent - move the point
 If the line is selected:
   LeftButtonPressEvent - activate a handle widget accociated with the line 
   LeftButtonReleaseEvent - release the handle widget associated with the line
   MouseMoveEvent - translate the line
 In all the cases, independent of what is picked, the widget responds to the 
 following VTK events:
   MiddleButtonPressEvent - translate the widget
   MiddleButtonReleaseEvent - release the widget
   RightButtonPressEvent - scale the widget's representation
   RightButtonReleaseEvent - stop scaling the widget
   MouseMoveEvent - scale (if right button) or move (if middle button) the widget
 \end{verbatim}

 Note that the event bindings described above can be changed using this
 class's vtkWidgetEventTranslator. This class translates VTK events 
 into the vtkLineWidget2's widget events:
 \begin{verbatim}
   vtkWidgetEvent::Select -- some part of the widget has been selected
   vtkWidgetEvent::EndSelect -- the selection process has completed
   vtkWidgetEvent::Move -- a request for slider motion has been invoked
 \end{verbatim}

 In turn, when these widget events are processed, the vtkLineWidget2
 invokes the following VTK events on itself (which observers can listen for):
 \begin{verbatim}
   vtkCommand::StartInteractionEvent (on vtkWidgetEvent::Select)
   vtkCommand::EndInteractionEvent (on vtkWidgetEvent::EndSelect)
   vtkCommand::InteractionEvent (on vtkWidgetEvent::Move)
 \end{verbatim}


To create an instance of class vtkLineWidget2, simply
invoke its constructor as follows
\begin{verbatim}
  obj = vtkLineWidget2
\end{verbatim}
\subsection{Methods}

The class vtkLineWidget2 has several methods that can be used.
  They are listed below.
Note that the documentation is translated automatically from the VTK sources,
and may not be completely intelligible.  When in doubt, consult the VTK website.
In the methods listed below, \verb|obj| is an instance of the vtkLineWidget2 class.
\begin{itemize}
\item  \verb|string = obj.GetClassName ()| -  Standard vtkObject methods

\item  \verb|int = obj.IsA (string name)| -  Standard vtkObject methods

\item  \verb|vtkLineWidget2 = obj.NewInstance ()| -  Standard vtkObject methods

\item  \verb|vtkLineWidget2 = obj.SafeDownCast (vtkObject o)| -  Standard vtkObject methods

\item  \verb|obj.SetEnabled (int enabling)| -  Override superclasses' SetEnabled() method because the line
 widget must enable its internal handle widgets.

\item  \verb|obj.SetRepresentation (vtkLineRepresentation r)| -  Create the default widget representation if one is not set. 

\item  \verb|obj.CreateDefaultRepresentation ()| -  Create the default widget representation if one is not set. 

\item  \verb|obj.SetProcessEvents (int )| -  Methods to change the whether the widget responds to interaction.
 Overridden to pass the state to component widgets.

\end{itemize}
