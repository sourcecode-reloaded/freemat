\section{vtkBiQuadraticQuadraticHexahedron}

\subsection{Usage}

 vtkBiQuadraticQuadraticHexahedron is a concrete implementation of vtkNonLinearCell to
 represent a three-dimensional, 24-node isoparametric biquadratic
 hexahedron. The interpolation is the standard finite element,
 biquadratic-quadratic
 isoparametric shape function. The cell includes mid-edge and center-face nodes. The
 ordering of the 24 points defining the cell is point ids (0-7,8-19, 20-23)
 where point ids 0-7 are the eight corner vertices of the cube; followed by
 twelve midedge nodes (8-19), nodes 20-23 are the center-face nodes. Note that
 these midedge nodes correspond lie
 on the edges defined by (0,1), (1,2), (2,3), (3,0), (4,5), (5,6), (6,7),
 (7,4), (0,4), (1,5), (2,6), (3,7). The center face nodes lieing in quad
 22-(0,1,5,4), 21-(1,2,6,5), 23-(2,3,7,6) and 22-(3,0,4,7)

 \begin{verbatim}

 top 
  7--14--6
  |      |
 15      13
  |      |
  4--12--5

  middle
 19--23--18
  |      |
 20      21
  |      |
 16--22--17

 bottom
  3--10--2
  |      |
 11      9 
  |      |
  0-- 8--1
  
 \end{verbatim}



To create an instance of class vtkBiQuadraticQuadraticHexahedron, simply
invoke its constructor as follows
\begin{verbatim}
  obj = vtkBiQuadraticQuadraticHexahedron
\end{verbatim}
\subsection{Methods}

The class vtkBiQuadraticQuadraticHexahedron has several methods that can be used.
  They are listed below.
Note that the documentation is translated automatically from the VTK sources,
and may not be completely intelligible.  When in doubt, consult the VTK website.
In the methods listed below, \verb|obj| is an instance of the vtkBiQuadraticQuadraticHexahedron class.
\begin{itemize}
\item  \verb|string = obj.GetClassName ()|

\item  \verb|int = obj.IsA (string name)|

\item  \verb|vtkBiQuadraticQuadraticHexahedron = obj.NewInstance ()|

\item  \verb|vtkBiQuadraticQuadraticHexahedron = obj.SafeDownCast (vtkObject o)|

\item  \verb|int = obj.GetCellType ()| -  Implement the vtkCell API. See the vtkCell API for descriptions
 of these methods.

\item  \verb|int = obj.GetCellDimension ()| -  Implement the vtkCell API. See the vtkCell API for descriptions
 of these methods.

\item  \verb|int = obj.GetNumberOfEdges ()| -  Implement the vtkCell API. See the vtkCell API for descriptions
 of these methods.

\item  \verb|int = obj.GetNumberOfFaces ()| -  Implement the vtkCell API. See the vtkCell API for descriptions
 of these methods.

\item  \verb|vtkCell = obj.GetEdge (int )| -  Implement the vtkCell API. See the vtkCell API for descriptions
 of these methods.

\item  \verb|vtkCell = obj.GetFace (int )| -  Implement the vtkCell API. See the vtkCell API for descriptions
 of these methods.

\item  \verb|int = obj.CellBoundary (int subId, double pcoords[3], vtkIdList pts)|

\item  \verb|obj.Contour (double value, vtkDataArray cellScalars, vtkIncrementalPointLocator locator, vtkCellArray verts, vtkCellArray lines, vtkCellArray polys, vtkPointData inPd, vtkPointData outPd, vtkCellData inCd, vtkIdType cellId, vtkCellData outCd)|

\item  \verb|int = obj.Triangulate (int index, vtkIdList ptIds, vtkPoints pts)|

\item  \verb|obj.Derivatives (int subId, double pcoords[3], double values, int dim, double derivs)|

\item  \verb|obj.Clip (double value, vtkDataArray cellScalars, vtkIncrementalPointLocator locator, vtkCellArray tetras, vtkPointData inPd, vtkPointData outPd, vtkCellData inCd, vtkIdType cellId, vtkCellData outCd, int insideOut)| -  Clip this biquadratic hexahedron using scalar value provided. Like
 contouring, except that it cuts the hex to produce linear
 tetrahedron.

\item  \verb|obj.InterpolateFunctions (double pcoords[3], double weights[24])| -  Compute the interpolation functions/derivatives
 (aka shape functions/derivatives)

\item  \verb|obj.InterpolateDerivs (double pcoords[3], double derivs[72])| -  Return the ids of the vertices defining edge/face (`edgeId`/`faceId').
 Ids are related to the cell, not to the dataset.

\end{itemize}
