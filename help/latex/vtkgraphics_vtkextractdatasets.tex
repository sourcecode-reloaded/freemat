\section{vtkExtractDataSets}

\subsection{Usage}

 vtkExtractDataSets accepts a vtkHierarchicalBoxDataSet as input and extracts
 different datasets from different levels. The output is
 vtkHierarchicalBoxDataSet with same structure as the input with only the
 selected datasets passed through. 

To create an instance of class vtkExtractDataSets, simply
invoke its constructor as follows
\begin{verbatim}
  obj = vtkExtractDataSets
\end{verbatim}
\subsection{Methods}

The class vtkExtractDataSets has several methods that can be used.
  They are listed below.
Note that the documentation is translated automatically from the VTK sources,
and may not be completely intelligible.  When in doubt, consult the VTK website.
In the methods listed below, \verb|obj| is an instance of the vtkExtractDataSets class.
\begin{itemize}
\item  \verb|string = obj.GetClassName ()|

\item  \verb|int = obj.IsA (string name)|

\item  \verb|vtkExtractDataSets = obj.NewInstance ()|

\item  \verb|vtkExtractDataSets = obj.SafeDownCast (vtkObject o)|

\item  \verb|obj.AddDataSet (int level, int idx)| -  Add a dataset to be extracted.

\item  \verb|obj.ClearDataSetList ()| -  Remove all entries from the list of datasets to be extracted.

\end{itemize}
