\section{vtkShrinkPolyData}

\subsection{Usage}

 vtkShrinkPolyData shrinks cells composing a polygonal dataset (e.g., 
 vertices, lines, polygons, and triangle strips) towards their centroid. 
 The centroid of a cell is computed as the average position of the
 cell points. Shrinking results in disconnecting the cells from
 one another. The output dataset type of this filter is polygonal data.

 During execution the filter passes its input cell data to its
 output. Point data attributes are copied to the points created during the
 shrinking process.

To create an instance of class vtkShrinkPolyData, simply
invoke its constructor as follows
\begin{verbatim}
  obj = vtkShrinkPolyData
\end{verbatim}
\subsection{Methods}

The class vtkShrinkPolyData has several methods that can be used.
  They are listed below.
Note that the documentation is translated automatically from the VTK sources,
and may not be completely intelligible.  When in doubt, consult the VTK website.
In the methods listed below, \verb|obj| is an instance of the vtkShrinkPolyData class.
\begin{itemize}
\item  \verb|string = obj.GetClassName ()|

\item  \verb|int = obj.IsA (string name)|

\item  \verb|vtkShrinkPolyData = obj.NewInstance ()|

\item  \verb|vtkShrinkPolyData = obj.SafeDownCast (vtkObject o)|

\item  \verb|obj.SetShrinkFactor (double )| -  Set the fraction of shrink for each cell.

\item  \verb|double = obj.GetShrinkFactorMinValue ()| -  Set the fraction of shrink for each cell.

\item  \verb|double = obj.GetShrinkFactorMaxValue ()| -  Set the fraction of shrink for each cell.

\item  \verb|double = obj.GetShrinkFactor ()| -  Get the fraction of shrink for each cell.

\end{itemize}
