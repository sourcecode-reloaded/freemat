\section{vtkImageHSVToRGB}

\subsection{Usage}

 For each pixel with hue, saturation and value components this filter
 outputs the color coded as red, green, blue.  Output type must be the same
 as input type.

To create an instance of class vtkImageHSVToRGB, simply
invoke its constructor as follows
\begin{verbatim}
  obj = vtkImageHSVToRGB
\end{verbatim}
\subsection{Methods}

The class vtkImageHSVToRGB has several methods that can be used.
  They are listed below.
Note that the documentation is translated automatically from the VTK sources,
and may not be completely intelligible.  When in doubt, consult the VTK website.
In the methods listed below, \verb|obj| is an instance of the vtkImageHSVToRGB class.
\begin{itemize}
\item  \verb|string = obj.GetClassName ()|

\item  \verb|int = obj.IsA (string name)|

\item  \verb|vtkImageHSVToRGB = obj.NewInstance ()|

\item  \verb|vtkImageHSVToRGB = obj.SafeDownCast (vtkObject o)|

\item  \verb|obj.SetMaximum (double )| -  Hue is an angle. Maximum specifies when it maps back to 0.
 HueMaximum defaults to 255 instead of 2PI, because unsigned char
 is expected as input.
 Maximum also specifies the maximum of the Saturation, and R, G, B.

\item  \verb|double = obj.GetMaximum ()| -  Hue is an angle. Maximum specifies when it maps back to 0.
 HueMaximum defaults to 255 instead of 2PI, because unsigned char
 is expected as input.
 Maximum also specifies the maximum of the Saturation, and R, G, B.

\end{itemize}
