\section{IMAGESC Image Display Function}

\subsection{Usage}

The \verb|imagesc| command has the following general syntax
\begin{verbatim}
  handle = imagesc(x,y,C,clim)
\end{verbatim}
where \verb|x| is a two vector containing the \verb|x| coordinates
of the first and last pixels along a column, and \verb|y| is a
two vector containing the \verb|y| coordinates of the first and
last pixels along a row.  The matrix \verb|C| constitutes the
image data.  It must either be a scalar matrix, in which case
the image is colormapped using the  \verb|colormap| for the current
figure.  If the matrix is \verb|M x N x 3|, then \verb|C| is intepreted
as RGB data, and the image is not colormapped.  The \verb|clim|
argument is a pairs [low high] that specifies scaling.  You can 
also omit the \verb|x| and \verb|y|, 
\begin{verbatim}
  handle = imagesc(C, clim)
\end{verbatim}
in which case they default to \verb|x = [1,size(C,2)]| and 
\verb|y = [1,size(C,1)]|.  Finally, you can use the \verb|image| function
with only formal arguments
\begin{verbatim}
  handle = imagesc(properties...)
\end{verbatim}

\subsection{Example}

In this example, we create an image that is \verb|512 x 512| pixels
square, and set the background to a noise pattern.  We set the central
\verb|128 x 256| pixels to be white.
@>

