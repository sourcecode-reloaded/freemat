\section{vtkMCubesReader}

\subsection{Usage}

 vtkMCubesReader is a source object that reads binary marching cubes
 files. (Marching cubes is an isosurfacing technique that generates 
 many triangles.) The binary format is supported by W. Lorensen's
 marching cubes program (and the vtkSliceCubes object). The format 
 repeats point coordinates, so this object will merge the points 
 with a vtkLocator object. You can choose to supply the vtkLocator 
 or use the default.

To create an instance of class vtkMCubesReader, simply
invoke its constructor as follows
\begin{verbatim}
  obj = vtkMCubesReader
\end{verbatim}
\subsection{Methods}

The class vtkMCubesReader has several methods that can be used.
  They are listed below.
Note that the documentation is translated automatically from the VTK sources,
and may not be completely intelligible.  When in doubt, consult the VTK website.
In the methods listed below, \verb|obj| is an instance of the vtkMCubesReader class.
\begin{itemize}
\item  \verb|string = obj.GetClassName ()|

\item  \verb|int = obj.IsA (string name)|

\item  \verb|vtkMCubesReader = obj.NewInstance ()|

\item  \verb|vtkMCubesReader = obj.SafeDownCast (vtkObject o)|

\item  \verb|obj.SetFileName (string )| -  Specify file name of marching cubes file.

\item  \verb|string = obj.GetFileName ()| -  Specify file name of marching cubes file.

\item  \verb|obj.SetLimitsFileName (string )| -  Set / get the file name of the marching cubes limits file.

\item  \verb|string = obj.GetLimitsFileName ()| -  Set / get the file name of the marching cubes limits file.

\item  \verb|obj.SetHeaderSize (int )| -  Specify a header size if one exists. The header is skipped and not used at this time.

\item  \verb|int = obj.GetHeaderSizeMinValue ()| -  Specify a header size if one exists. The header is skipped and not used at this time.

\item  \verb|int = obj.GetHeaderSizeMaxValue ()| -  Specify a header size if one exists. The header is skipped and not used at this time.

\item  \verb|int = obj.GetHeaderSize ()| -  Specify a header size if one exists. The header is skipped and not used at this time.

\item  \verb|obj.SetFlipNormals (int )| -  Specify whether to flip normals in opposite direction. Flipping ONLY
 changes the direction of the normal vector. Contrast this with flipping
 in vtkPolyDataNormals which flips both the normal and the cell point
 order.

\item  \verb|int = obj.GetFlipNormals ()| -  Specify whether to flip normals in opposite direction. Flipping ONLY
 changes the direction of the normal vector. Contrast this with flipping
 in vtkPolyDataNormals which flips both the normal and the cell point
 order.

\item  \verb|obj.FlipNormalsOn ()| -  Specify whether to flip normals in opposite direction. Flipping ONLY
 changes the direction of the normal vector. Contrast this with flipping
 in vtkPolyDataNormals which flips both the normal and the cell point
 order.

\item  \verb|obj.FlipNormalsOff ()| -  Specify whether to flip normals in opposite direction. Flipping ONLY
 changes the direction of the normal vector. Contrast this with flipping
 in vtkPolyDataNormals which flips both the normal and the cell point
 order.

\item  \verb|obj.SetNormals (int )| -  Specify whether to read normals.

\item  \verb|int = obj.GetNormals ()| -  Specify whether to read normals.

\item  \verb|obj.NormalsOn ()| -  Specify whether to read normals.

\item  \verb|obj.NormalsOff ()| -  Specify whether to read normals.

\item  \verb|obj.SetDataByteOrderToBigEndian ()| -  These methods should be used instead of the SwapBytes methods.
 They indicate the byte ordering of the file you are trying
 to read in. These methods will then either swap or not swap
 the bytes depending on the byte ordering of the machine it is
 being run on. For example, reading in a BigEndian file on a
 BigEndian machine will result in no swapping. Trying to read
 the same file on a LittleEndian machine will result in swapping.
 As a quick note most UNIX machines are BigEndian while PC's
 and VAX tend to be LittleEndian. So if the file you are reading
 in was generated on a VAX or PC, SetDataByteOrderToLittleEndian otherwise
 SetDataByteOrderToBigEndian. 

\item  \verb|obj.SetDataByteOrderToLittleEndian ()| -  These methods should be used instead of the SwapBytes methods.
 They indicate the byte ordering of the file you are trying
 to read in. These methods will then either swap or not swap
 the bytes depending on the byte ordering of the machine it is
 being run on. For example, reading in a BigEndian file on a
 BigEndian machine will result in no swapping. Trying to read
 the same file on a LittleEndian machine will result in swapping.
 As a quick note most UNIX machines are BigEndian while PC's
 and VAX tend to be LittleEndian. So if the file you are reading
 in was generated on a VAX or PC, SetDataByteOrderToLittleEndian otherwise
 SetDataByteOrderToBigEndian. 

\item  \verb|int = obj.GetDataByteOrder ()| -  These methods should be used instead of the SwapBytes methods.
 They indicate the byte ordering of the file you are trying
 to read in. These methods will then either swap or not swap
 the bytes depending on the byte ordering of the machine it is
 being run on. For example, reading in a BigEndian file on a
 BigEndian machine will result in no swapping. Trying to read
 the same file on a LittleEndian machine will result in swapping.
 As a quick note most UNIX machines are BigEndian while PC's
 and VAX tend to be LittleEndian. So if the file you are reading
 in was generated on a VAX or PC, SetDataByteOrderToLittleEndian otherwise
 SetDataByteOrderToBigEndian. 

\item  \verb|obj.SetDataByteOrder (int )| -  These methods should be used instead of the SwapBytes methods.
 They indicate the byte ordering of the file you are trying
 to read in. These methods will then either swap or not swap
 the bytes depending on the byte ordering of the machine it is
 being run on. For example, reading in a BigEndian file on a
 BigEndian machine will result in no swapping. Trying to read
 the same file on a LittleEndian machine will result in swapping.
 As a quick note most UNIX machines are BigEndian while PC's
 and VAX tend to be LittleEndian. So if the file you are reading
 in was generated on a VAX or PC, SetDataByteOrderToLittleEndian otherwise
 SetDataByteOrderToBigEndian. 

\item  \verb|string = obj.GetDataByteOrderAsString ()| -  These methods should be used instead of the SwapBytes methods.
 They indicate the byte ordering of the file you are trying
 to read in. These methods will then either swap or not swap
 the bytes depending on the byte ordering of the machine it is
 being run on. For example, reading in a BigEndian file on a
 BigEndian machine will result in no swapping. Trying to read
 the same file on a LittleEndian machine will result in swapping.
 As a quick note most UNIX machines are BigEndian while PC's
 and VAX tend to be LittleEndian. So if the file you are reading
 in was generated on a VAX or PC, SetDataByteOrderToLittleEndian otherwise
 SetDataByteOrderToBigEndian. 

\item  \verb|obj.SetSwapBytes (int )| -  Turn on/off byte swapping.

\item  \verb|int = obj.GetSwapBytes ()| -  Turn on/off byte swapping.

\item  \verb|obj.SwapBytesOn ()| -  Turn on/off byte swapping.

\item  \verb|obj.SwapBytesOff ()| -  Turn on/off byte swapping.

\item  \verb|obj.SetLocator (vtkIncrementalPointLocator locator)| -  Set / get a spatial locator for merging points. By default, 
 an instance of vtkMergePoints is used.

\item  \verb|vtkIncrementalPointLocator = obj.GetLocator ()| -  Set / get a spatial locator for merging points. By default, 
 an instance of vtkMergePoints is used.

\item  \verb|obj.CreateDefaultLocator ()| -  Create default locator. Used to create one when none is specified.

\item  \verb|long = obj.GetMTime ()| -  Return the mtime also considering the locator.

\end{itemize}
