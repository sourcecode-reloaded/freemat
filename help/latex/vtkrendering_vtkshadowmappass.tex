\section{vtkShadowMapPass}

\subsection{Usage}

 Render the opaque polygonal geometry of a scene with shadow maps (a
 technique to render hard shadows in hardware).

 This pass expects an initialized depth buffer and color buffer.
 Initialized buffers means they have been cleared with farest z-value and
 background color/gradient/transparent color.
 An opaque pass may have been performed right after the initialization.



 Its delegate is usually set to a vtkOpaquePass.

 .SECTION Implementation
 The first pass of the algorithm is to generate a shadow map per light
 (depth map from the light point of view) by rendering the opaque objects
 with the OCCLUDER property keys.
 The second pass is to render the opaque objects with the RECEIVER keys.


To create an instance of class vtkShadowMapPass, simply
invoke its constructor as follows
\begin{verbatim}
  obj = vtkShadowMapPass
\end{verbatim}
\subsection{Methods}

The class vtkShadowMapPass has several methods that can be used.
  They are listed below.
Note that the documentation is translated automatically from the VTK sources,
and may not be completely intelligible.  When in doubt, consult the VTK website.
In the methods listed below, \verb|obj| is an instance of the vtkShadowMapPass class.
\begin{itemize}
\item  \verb|string = obj.GetClassName ()|

\item  \verb|int = obj.IsA (string name)|

\item  \verb|vtkShadowMapPass = obj.NewInstance ()|

\item  \verb|vtkShadowMapPass = obj.SafeDownCast (vtkObject o)|

\item  \verb|obj.ReleaseGraphicsResources (vtkWindow w)| -  Release graphics resources and ask components to release their own
 resources.
 

\item  \verb|vtkRenderPass = obj.GetOpaquePass ()| -  Delegate for rendering the opaque polygonal geometry.
 If it is NULL, nothing will be rendered and a warning will be emitted.
 It is usually set to a vtkTranslucentPass.
 Initial value is a NULL pointer.

\item  \verb|obj.SetOpaquePass (vtkRenderPass opaquePass)| -  Delegate for rendering the opaque polygonal geometry.
 If it is NULL, nothing will be rendered and a warning will be emitted.
 It is usually set to a vtkTranslucentPass.
 Initial value is a NULL pointer.

\item  \verb|vtkRenderPass = obj.GetCompositeZPass ()| -  Delegate for rendering the opaque polygonal geometry.
 If it is NULL, nothing will be rendered and a warning will be emitted.
 It is usually set to a vtkTranslucentPass.
 Initial value is a NULL pointer.

\item  \verb|obj.SetCompositeZPass (vtkRenderPass opaquePass)| -  Delegate for rendering the opaque polygonal geometry.
 If it is NULL, nothing will be rendered and a warning will be emitted.
 It is usually set to a vtkTranslucentPass.
 Initial value is a NULL pointer.

\item  \verb|obj.SetResolution (int )| -  Set/Get the number of pixels in each dimension of the shadow maps
 (shadow maps are square). Initial value is 256. The greater the better.
 Resolution does not have to be a power-of-two value.

\item  \verb|int = obj.GetResolution ()| -  Set/Get the number of pixels in each dimension of the shadow maps
 (shadow maps are square). Initial value is 256. The greater the better.
 Resolution does not have to be a power-of-two value.

\item  \verb|obj.SetPolygonOffsetFactor (float )| -  Factor used to scale the maximum depth slope of a polygon (definition
 from OpenGL 2.1 spec section 3.5.5 ''Depth Offset'' page 112). This is
 used during the creation the shadow maps (not during mapping of the
 shadow maps onto the geometry)
 Play with this value and PolygonOffsetUnits to solve self-shadowing.
 Valid values can be either positive or negative.
 Initial value is 1.1f (recommended by the nVidia presentation about
 Shadow Mapping by Cass Everitt). 3.1f works well with the regression test.

\item  \verb|float = obj.GetPolygonOffsetFactor ()| -  Factor used to scale the maximum depth slope of a polygon (definition
 from OpenGL 2.1 spec section 3.5.5 ''Depth Offset'' page 112). This is
 used during the creation the shadow maps (not during mapping of the
 shadow maps onto the geometry)
 Play with this value and PolygonOffsetUnits to solve self-shadowing.
 Valid values can be either positive or negative.
 Initial value is 1.1f (recommended by the nVidia presentation about
 Shadow Mapping by Cass Everitt). 3.1f works well with the regression test.

\item  \verb|obj.SetPolygonOffsetUnits (float )| -  Factor used to scale an implementation dependent constant that relates
 to the usable resolution of the depth buffer (definition from OpenGL 2.1
 spec section 3.5.5 ''Depth Offset'' page 112). This is
 used during the creation the shadow maps (not during mapping of the
 shadow maps onto the geometry)
 Play with this value and PolygonOffsetFactor to solve self-shadowing.
 Valid values can be either positive or negative.
 Initial value is 4.0f (recommended by the nVidia presentation about
 Shadow Mapping by Cass Everitt). 10.0f works well with the regression
 test.

\item  \verb|float = obj.GetPolygonOffsetUnits ()| -  Factor used to scale an implementation dependent constant that relates
 to the usable resolution of the depth buffer (definition from OpenGL 2.1
 spec section 3.5.5 ''Depth Offset'' page 112). This is
 used during the creation the shadow maps (not during mapping of the
 shadow maps onto the geometry)
 Play with this value and PolygonOffsetFactor to solve self-shadowing.
 Valid values can be either positive or negative.
 Initial value is 4.0f (recommended by the nVidia presentation about
 Shadow Mapping by Cass Everitt). 10.0f works well with the regression
 test.

\end{itemize}
