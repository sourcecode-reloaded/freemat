\section{vtkMultiThreshold}

\subsection{Usage}

 This filter can be substituted for a chain of several vtkThreshold filters
 and can also perform more sophisticated subsetting operations.
 It generates a vtkMultiBlockDataSet as its output.
 This multiblock dataset contains a vtkUnstructuredGrid for each thresholded
 subset you request.
 A thresholded subset can be a set defined by an interval over a
 point or cell attribute of the mesh; these subsets are called IntervalSets.
 A thresholded subset can also be a boolean combination of one or more IntervalSets;
 these subsets are called BooleanSets.
 BooleanSets allow complex logic since their output
 can depend on multiple intervals over multiple variables
 defined on the input mesh.
 This is useful because it eliminates the need for thresholding several
 times and then appending the results, as can be required with vtkThreshold
 when one wants to remove some range of values (e.g., a notch filter).
 Cells are not repeated when they belong to more than one interval unless
 those intervals have different output grids.

 Another advantage this filter provides over vtkThreshold is the ability
 to threshold on non-scalar (i.e., vector, tensor, etc.) attributes without
 first computing an array containing some norm of the desired attribute.
 vtkMultiThreshold provides $L_1$, $L_2$, and $L_{\infty}$ norms.

 This filter makes a distinction between intermediate subsets and
 subsets that will be output to a grid.
 Each intermediate subset you create with AddIntervalSet or
 AddBooleanSet is given a unique integer identifier (via the return
 values of these member functions).
 If you wish for a given set to be output, you must call
 OutputSet and pass it one of these identifiers.
 The return of OutputSet is the integer index of the output set
 in the multiblock dataset created by this filter.

 For example, if an input mesh defined three attributes T, P, and s, one might
 wish to find cells that satisfy ''T < 320 [K] \&\& ( P > 101 [kPa] || s < 0.1 [kJ/kg/K] )''.
 To accomplish this with a vtkMultiThreshold filter,
 \begin{verbatim}
 vtkMultiThreshold* thr;
 int intervalSets[3];

 intervalSets[0] = thr->AddIntervalSet( vtkMath::NegInf(), 320., vtkMultiThreshold::CLOSED, vtkMultiThreshold::OPEN,
     vtkDataObject::FIELD\_ASSOCIATION\_POINTS, ''T'', 0, 1 );
 intervalSets[1] = thr->AddIntervalSet( 101., vtkMath::Inf(), vtkMultiThreshold::OPEN, vtkMultiThreshold::CLOSED,
     vtkDataObject::FIELD\_ASSOCIATION\_CELLS, ''P'', 0, 1 );
 intervalSets[2] = thr->AddIntervalSet( vtkMath::NegInf(), 0.1, vtkMultiThreshold::CLOSED, vtkMultiThreshold::OPEN,
     vtkDataObject::FIELD\_ASSOCIATION\_POINTS, ''s'', 0, 1 );

 int intermediate = thr->AddBooleanSet( vtkMultiThreshold::OR, 2, \&intervalSets[1] );

 int intersection[2];
 intersection[0] = intervalSets[0];
 intersection[1] = intermediate;
 int outputSet = thr->AddBooleanSet( vtkMultiThreshold::AND, 2, intersection );

 int outputGridIndex = thr->OutputSet( outputSet );
 thr->Update();
 \end{verbatim}
 The result of this filter will be a multiblock dataset that contains a single child with the desired cells.
 If we had also called <code>thr->OutputSet( intervalSets[0] );</code>, there would be two child meshes and
 one would contain all cells with T < 320 [K].
 In that case, the output can be represented by this graph
 \begin{verbatim}
 digraph MultiThreshold {
   set0 [shape=rect,style=filled,label=''point T(0) in [-Inf,320['']
   set1 [shape=rect,label=''cell P(0) in ]101,Inf]'']
   set2 [shape=rect,label=''point s(0) in [-Inf,0.1['']
   set3 [shape=rect,label=''OR'']
   set4 [shape=rect,style=filled,label=''AND'']
   set0 -> set4
   set1 -> set3
   set2 -> set3
   set3 -> set4
 }
 \end{verbatim}
 The filled rectangles represent sets that are output.

To create an instance of class vtkMultiThreshold, simply
invoke its constructor as follows
\begin{verbatim}
  obj = vtkMultiThreshold
\end{verbatim}
\subsection{Methods}

The class vtkMultiThreshold has several methods that can be used.
  They are listed below.
Note that the documentation is translated automatically from the VTK sources,
and may not be completely intelligible.  When in doubt, consult the VTK website.
In the methods listed below, \verb|obj| is an instance of the vtkMultiThreshold class.
\begin{itemize}
\item  \verb|string = obj.GetClassName ()|

\item  \verb|int = obj.IsA (string name)|

\item  \verb|vtkMultiThreshold = obj.NewInstance ()|

\item  \verb|vtkMultiThreshold = obj.SafeDownCast (vtkObject o)|

\item  \verb|int = obj.AddIntervalSet (double xmin, double xmax, int omin, int omax, int assoc, string arrayName, int component, int allScalars)| -  Add a mesh subset to be computed by thresholding an attribute of the input mesh.
 The subset can then be added to an output mesh with OuputSet() or combined with other sets using AddBooleanSet.
 If you wish to include all cells with values below some number  a, call
 with xmin set to vtkMath::NegInf() and xmax set to  a.
 Similarly, if you wish to include all cells with values above some number  a,
 call with xmin set to  a and xmax set to vtkMath::Inf().
 When specifying Inf() or NegInf() for an endpoint, it does not matter whether
 you specify and open or closed endpoint.
 
 When creating intervals, any integers can be used for the IDs of output meshes.
 All that matters is that the same ID be used if intervals should output to the same mesh.
 The outputs are ordered with ascending IDs in output block 0.

 It is possible to specify an invalid interval, in which case these routines will return -1.
 Invalid intervals occur when
 - an array does not exist,
 -  center is invalid,
 -  xmin ==  xmax and  omin and/or  omax are vtkMultiThreshold::OPEN, or
 -  xmin >  xmax.
 -  xmin or  xmax is not a number (i.e., IEEE NaN). Having both  xmin and  xmax equal NaN is allowed.
 vtkMath provides a portable way to specify IEEE infinities and Nan.
 Note that specifying an interval completely out of the bounds of an attribute is considered valid.
 In fact, it is occasionally useful to create a closed interval with both endpoints set to $\infty$
 or both endpoints set to $-\infty$ in order to locate cells with problematic values.

 @param xmin The minimum attribute value
 @param xmax The maximum attribute value
 @param omin Whether the interval should be open or closed at  xmin. Use vtkMultiThreshold::OPEN or vtkMultiThreshold::CLOSED.
 @param omax Whether the interval should be open or closed at  xmax. Use vtkMultiThreshold::OPEN or vtkMultiThreshold::CLOSED.
 @param assoc One of vtkDataObject::FIELD\_ASSOCIATION\_CELLS or vtkDataObject::FIELD\_ASSOCIATION\_POINTS indicating whether 
               a point or cell array should be used.
 @param arrayName The name of the array to use for thresholding
 @param attribType The attribute to use for thresholding.
                   One of vtkDataSetAttributes::SCALARS, VECTORS, TENSORS, NORMALS, TCOORDS, or GLOBALIDS.
 @param component The number of the component to threshold on or one of the following enumerants for norms:
                  LINFINITY\_NORM, L2\_NORM, L1\_NORM.
 @param allScalars When  center is vtkDataObject::FIELD\_ASSOCIATION\_POINTS, must all scalars be in the interval for
                   the cell to be passed to the output, or just a single point's scalar?
 @return An index used to identify the cells selected by the interval or -1 if the interval specification was invalid.
         If a valid value is returned, you may pass it to OutputSet().

\item  \verb|int = obj.AddIntervalSet (double xmin, double xmax, int omin, int omax, int assoc, int attribType, int component, int allScalars)| -  Add a mesh subset to be computed by thresholding an attribute of the input mesh.
 The subset can then be added to an output mesh with OuputSet() or combined with other sets using AddBooleanSet.
 If you wish to include all cells with values below some number  a, call
 with xmin set to vtkMath::NegInf() and xmax set to  a.
 Similarly, if you wish to include all cells with values above some number  a,
 call with xmin set to  a and xmax set to vtkMath::Inf().
 When specifying Inf() or NegInf() for an endpoint, it does not matter whether
 you specify and open or closed endpoint.
 
 When creating intervals, any integers can be used for the IDs of output meshes.
 All that matters is that the same ID be used if intervals should output to the same mesh.
 The outputs are ordered with ascending IDs in output block 0.

 It is possible to specify an invalid interval, in which case these routines will return -1.
 Invalid intervals occur when
 - an array does not exist,
 -  center is invalid,
 -  xmin ==  xmax and  omin and/or  omax are vtkMultiThreshold::OPEN, or
 -  xmin >  xmax.
 -  xmin or  xmax is not a number (i.e., IEEE NaN). Having both  xmin and  xmax equal NaN is allowed.
 vtkMath provides a portable way to specify IEEE infinities and Nan.
 Note that specifying an interval completely out of the bounds of an attribute is considered valid.
 In fact, it is occasionally useful to create a closed interval with both endpoints set to $\infty$
 or both endpoints set to $-\infty$ in order to locate cells with problematic values.

 @param xmin The minimum attribute value
 @param xmax The maximum attribute value
 @param omin Whether the interval should be open or closed at  xmin. Use vtkMultiThreshold::OPEN or vtkMultiThreshold::CLOSED.
 @param omax Whether the interval should be open or closed at  xmax. Use vtkMultiThreshold::OPEN or vtkMultiThreshold::CLOSED.
 @param assoc One of vtkDataObject::FIELD\_ASSOCIATION\_CELLS or vtkDataObject::FIELD\_ASSOCIATION\_POINTS indicating whether 
               a point or cell array should be used.
 @param arrayName The name of the array to use for thresholding
 @param attribType The attribute to use for thresholding.
                   One of vtkDataSetAttributes::SCALARS, VECTORS, TENSORS, NORMALS, TCOORDS, or GLOBALIDS.
 @param component The number of the component to threshold on or one of the following enumerants for norms:
                  LINFINITY\_NORM, L2\_NORM, L1\_NORM.
 @param allScalars When  center is vtkDataObject::FIELD\_ASSOCIATION\_POINTS, must all scalars be in the interval for
                   the cell to be passed to the output, or just a single point's scalar?
 @return An index used to identify the cells selected by the interval or -1 if the interval specification was invalid.
         If a valid value is returned, you may pass it to OutputSet().

\item  \verb|int = obj.AddLowpassIntervalSet (double xmax, int assoc, string arrayName, int component, int allScalars)| -  These convenience members make it easy to insert closed intervals.
 The ''notch'' interval is accomplished by creating a bandpass interval and applying a NAND operation.
 In this case, the set ID returned in the NAND operation set ID.
 Note that you can pass xmin == xmax when creating a bandpass threshold to retrieve elements matching exactly
 one value (since the intervals created by these routines are closed).

\item  \verb|int = obj.AddHighpassIntervalSet (double xmin, int assoc, string arrayName, int component, int allScalars)| -  These convenience members make it easy to insert closed intervals.
 The ''notch'' interval is accomplished by creating a bandpass interval and applying a NAND operation.
 In this case, the set ID returned in the NAND operation set ID.
 Note that you can pass xmin == xmax when creating a bandpass threshold to retrieve elements matching exactly
 one value (since the intervals created by these routines are closed).

\item  \verb|int = obj.AddBandpassIntervalSet (double xmin, double xmax, int assoc, string arrayName, int component, int allScalars)| -  These convenience members make it easy to insert closed intervals.
 The ''notch'' interval is accomplished by creating a bandpass interval and applying a NAND operation.
 In this case, the set ID returned in the NAND operation set ID.
 Note that you can pass xmin == xmax when creating a bandpass threshold to retrieve elements matching exactly
 one value (since the intervals created by these routines are closed).

\item  \verb|int = obj.AddNotchIntervalSet (double xlo, double xhi, int assoc, string arrayName, int component, int allScalars)| -  These convenience members make it easy to insert closed intervals.
 The ''notch'' interval is accomplished by creating a bandpass interval and applying a NAND operation.
 In this case, the set ID returned in the NAND operation set ID.
 Note that you can pass xmin == xmax when creating a bandpass threshold to retrieve elements matching exactly
 one value (since the intervals created by these routines are closed).

\item  \verb|int = obj.AddBooleanSet (int operation, int numInputs, int inputs)| -  Create a new mesh subset using boolean operations on pre-existing sets.

\item  \verb|int = obj.OutputSet (int setId)| -  Create an output mesh containing a boolean or interval subset of the input mesh.

\item  \verb|obj.Reset ()| -  Remove all the intervals currently defined.

\end{itemize}
