\section{ABS Absolute Value Function}

\subsection{Usage}

Returns the absolute value of the input array for all elements.  The 
general syntax for its use is
\begin{verbatim}
   y = abs(x)
\end{verbatim}
where \verb|x| is an \verb|n|-dimensional array of numerical type.  The output 
is the same numerical type as the input, unless the input is \verb|complex|
or \verb|dcomplex|.  For \verb|complex| inputs, the absolute value is a floating
point array, so that the return type is \verb|float|.  For \verb|dcomplex|
inputs, the absolute value is a double precision floating point array, so that
the return type is \verb|double|.
\subsection{Example}

The following demonstrates the \verb|abs| applied to a complex scalar.
\begin{verbatim}
--> abs(3+4*i)

ans = 
 5 
\end{verbatim}
The \verb|abs| function applied to integer and real values:
\begin{verbatim}
--> abs([-2,3,-4,5])

ans = 
 2 3 4 5 
\end{verbatim}
For a double-precision complex array,
\begin{verbatim}
--> abs([2.0+3.0*i,i])

ans = 
    3.6056    1.0000 
\end{verbatim}
