\section{vtkPointsProjectedHull}

\subsection{Usage}

    a subclass of vtkPoints, it maintains the counter clockwise 
    convex hull of the points (projected orthogonally in the 
    three coordinate directions) and has a method to
    test for intersection of that hull with an axis aligned
    rectangle.  This is used for intersection tests of 3D volumes.

To create an instance of class vtkPointsProjectedHull, simply
invoke its constructor as follows
\begin{verbatim}
  obj = vtkPointsProjectedHull
\end{verbatim}
\subsection{Methods}

The class vtkPointsProjectedHull has several methods that can be used.
  They are listed below.
Note that the documentation is translated automatically from the VTK sources,
and may not be completely intelligible.  When in doubt, consult the VTK website.
In the methods listed below, \verb|obj| is an instance of the vtkPointsProjectedHull class.
\begin{itemize}
\item  \verb|string = obj.GetClassName ()|

\item  \verb|int = obj.IsA (string name)|

\item  \verb|vtkPointsProjectedHull = obj.NewInstance ()|

\item  \verb|vtkPointsProjectedHull = obj.SafeDownCast (vtkObject o)|

\item  \verb|int = obj.RectangleIntersectionX (vtkPoints R)|

\item  \verb|int = obj.RectangleIntersectionX (float ymin, float ymax, float zmin, float zmax)|

\item  \verb|int = obj.RectangleIntersectionX (double ymin, double ymax, double zmin, double zmax)|

\item  \verb|int = obj.RectangleIntersectionY (vtkPoints R)|

\item  \verb|int = obj.RectangleIntersectionY (float zmin, float zmax, float xmin, float xmax)|

\item  \verb|int = obj.RectangleIntersectionY (double zmin, double zmax, double xmin, double xmax)|

\item  \verb|int = obj.RectangleIntersectionZ (vtkPoints R)|

\item  \verb|int = obj.RectangleIntersectionZ (float xmin, float xmax, float ymin, float ymax)|

\item  \verb|int = obj.RectangleIntersectionZ (double xmin, double xmax, double ymin, double ymax)|

\item  \verb|int = obj.GetCCWHullX (float pts, int len)|

\item  \verb|int = obj.GetCCWHullX (double pts, int len)|

\item  \verb|int = obj.GetCCWHullY (float pts, int len)|

\item  \verb|int = obj.GetCCWHullY (double pts, int len)|

\item  \verb|int = obj.GetCCWHullZ (float pts, int len)|

\item  \verb|int = obj.GetCCWHullZ (double pts, int len)|

\item  \verb|int = obj.GetSizeCCWHullX ()|

\item  \verb|int = obj.GetSizeCCWHullY ()|

\item  \verb|int = obj.GetSizeCCWHullZ ()|

\item  \verb|obj.Initialize ()|

\item  \verb|obj.Reset ()|

\item  \verb|obj.Update ()|

\end{itemize}
