\section{vtkTableToSparseArray}

\subsection{Usage}

 Converts a vtkTable into a sparse array.  Use AddCoordinateColumn() to
 designate one-to-many table columns that contain coordinates for each
 array value, and SetValueColumn() to designate the table column that 
 contains array values.

 Thus, the number of dimensions in the output array will equal the number
 of calls to AddCoordinateColumn().

 The coordinate columns will also be used to populate dimension labels
 in the output array.

 .SECTION Thanks
 Developed by Timothy M. Shead (tshead@sandia.gov) at Sandia National Laboratories.

To create an instance of class vtkTableToSparseArray, simply
invoke its constructor as follows
\begin{verbatim}
  obj = vtkTableToSparseArray
\end{verbatim}
\subsection{Methods}

The class vtkTableToSparseArray has several methods that can be used.
  They are listed below.
Note that the documentation is translated automatically from the VTK sources,
and may not be completely intelligible.  When in doubt, consult the VTK website.
In the methods listed below, \verb|obj| is an instance of the vtkTableToSparseArray class.
\begin{itemize}
\item  \verb|string = obj.GetClassName ()|

\item  \verb|int = obj.IsA (string name)|

\item  \verb|vtkTableToSparseArray = obj.NewInstance ()|

\item  \verb|vtkTableToSparseArray = obj.SafeDownCast (vtkObject o)|

\item  \verb|obj.ClearCoordinateColumns ()| -  Specify the set of input table columns that will be mapped to coordinates
 in the output sparse array.

\item  \verb|obj.AddCoordinateColumn (string name)| -  Specify the set of input table columns that will be mapped to coordinates
 in the output sparse array.

\item  \verb|obj.SetValueColumn (string name)| -  Specify the input table column that will be mapped to values in the output array.

\item  \verb|string = obj.GetValueColumn ()| -  Specify the input table column that will be mapped to values in the output array.

\end{itemize}
