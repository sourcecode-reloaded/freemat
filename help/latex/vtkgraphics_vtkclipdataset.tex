\section{vtkClipDataSet}

\subsection{Usage}

 vtkClipDataSet is a filter that clips any type of dataset using either
 any subclass of vtkImplicitFunction, or the input scalar
 data. Clipping means that it actually ''cuts'' through the cells of
 the dataset, returning everything inside of the specified implicit
 function (or greater than the scalar value) including ''pieces'' of
 a cell. (Compare this with vtkExtractGeometry, which pulls out
 entire, uncut cells.) The output of this filter is an unstructured
 grid.

 To use this filter, you must decide if you will be clipping with an
 implicit function, or whether you will be using the input scalar
 data.  If you want to clip with an implicit function, you must:
 1) define an implicit function
 2) set it with the SetClipFunction method
 3) apply the GenerateClipScalarsOn method
 If a ClipFunction is not specified, or GenerateClipScalars is off
 (the default), then the input's scalar data will be used to clip
 the polydata.

 You can also specify a scalar value, which is used to decide what is
 inside and outside of the implicit function. You can also reverse the
 sense of what inside/outside is by setting the InsideOut instance
 variable. (The clipping algorithm proceeds by computing an implicit
 function value or using the input scalar data for each point in the
 dataset. This is compared to the scalar value to determine
 inside/outside.)

 This filter can be configured to compute a second output. The
 second output is the part of the cell that is clipped away. Set the
 GenerateClippedData boolean on if you wish to access this output data.

To create an instance of class vtkClipDataSet, simply
invoke its constructor as follows
\begin{verbatim}
  obj = vtkClipDataSet
\end{verbatim}
\subsection{Methods}

The class vtkClipDataSet has several methods that can be used.
  They are listed below.
Note that the documentation is translated automatically from the VTK sources,
and may not be completely intelligible.  When in doubt, consult the VTK website.
In the methods listed below, \verb|obj| is an instance of the vtkClipDataSet class.
\begin{itemize}
\item  \verb|string = obj.GetClassName ()|

\item  \verb|int = obj.IsA (string name)|

\item  \verb|vtkClipDataSet = obj.NewInstance ()|

\item  \verb|vtkClipDataSet = obj.SafeDownCast (vtkObject o)|

\item  \verb|obj.SetValue (double )| -  Set the clipping value of the implicit function (if clipping with
 implicit function) or scalar value (if clipping with
 scalars). The default value is 0.0. This value is ignored if
 UseValueAsOffset is true and a clip function is defined.

\item  \verb|double = obj.GetValue ()| -  Set the clipping value of the implicit function (if clipping with
 implicit function) or scalar value (if clipping with
 scalars). The default value is 0.0. This value is ignored if
 UseValueAsOffset is true and a clip function is defined.

\item  \verb|obj.SetUseValueAsOffset (bool )| -  If UseValueAsOffset is true, Value is used as an offset parameter to
 the implicit function. Otherwise, Value is used only when clipping
 using a scalar array. Default is true.

\item  \verb|bool = obj.GetUseValueAsOffset ()| -  If UseValueAsOffset is true, Value is used as an offset parameter to
 the implicit function. Otherwise, Value is used only when clipping
 using a scalar array. Default is true.

\item  \verb|obj.UseValueAsOffsetOn ()| -  If UseValueAsOffset is true, Value is used as an offset parameter to
 the implicit function. Otherwise, Value is used only when clipping
 using a scalar array. Default is true.

\item  \verb|obj.UseValueAsOffsetOff ()| -  If UseValueAsOffset is true, Value is used as an offset parameter to
 the implicit function. Otherwise, Value is used only when clipping
 using a scalar array. Default is true.

\item  \verb|obj.SetInsideOut (int )| -  Set/Get the InsideOut flag. When off, a vertex is considered
 inside the implicit function if its value is greater than the
 Value ivar. When InsideOutside is turned on, a vertex is
 considered inside the implicit function if its implicit function
 value is less than or equal to the Value ivar.  InsideOut is off
 by default.

\item  \verb|int = obj.GetInsideOut ()| -  Set/Get the InsideOut flag. When off, a vertex is considered
 inside the implicit function if its value is greater than the
 Value ivar. When InsideOutside is turned on, a vertex is
 considered inside the implicit function if its implicit function
 value is less than or equal to the Value ivar.  InsideOut is off
 by default.

\item  \verb|obj.InsideOutOn ()| -  Set/Get the InsideOut flag. When off, a vertex is considered
 inside the implicit function if its value is greater than the
 Value ivar. When InsideOutside is turned on, a vertex is
 considered inside the implicit function if its implicit function
 value is less than or equal to the Value ivar.  InsideOut is off
 by default.

\item  \verb|obj.InsideOutOff ()| -  Set/Get the InsideOut flag. When off, a vertex is considered
 inside the implicit function if its value is greater than the
 Value ivar. When InsideOutside is turned on, a vertex is
 considered inside the implicit function if its implicit function
 value is less than or equal to the Value ivar.  InsideOut is off
 by default.

\item  \verb|obj.SetClipFunction (vtkImplicitFunction )|

\item  \verb|vtkImplicitFunction = obj.GetClipFunction ()|

\item  \verb|obj.SetGenerateClipScalars (int )| -  If this flag is enabled, then the output scalar values will be 
 interpolated from the implicit function values, and not the 
 input scalar data. If you enable this flag but do not provide an
 implicit function an error will be reported.

\item  \verb|int = obj.GetGenerateClipScalars ()| -  If this flag is enabled, then the output scalar values will be 
 interpolated from the implicit function values, and not the 
 input scalar data. If you enable this flag but do not provide an
 implicit function an error will be reported.

\item  \verb|obj.GenerateClipScalarsOn ()| -  If this flag is enabled, then the output scalar values will be 
 interpolated from the implicit function values, and not the 
 input scalar data. If you enable this flag but do not provide an
 implicit function an error will be reported.

\item  \verb|obj.GenerateClipScalarsOff ()| -  If this flag is enabled, then the output scalar values will be 
 interpolated from the implicit function values, and not the 
 input scalar data. If you enable this flag but do not provide an
 implicit function an error will be reported.

\item  \verb|obj.SetGenerateClippedOutput (int )| -  Control whether a second output is generated. The second output
 contains the polygonal data that's been clipped away.

\item  \verb|int = obj.GetGenerateClippedOutput ()| -  Control whether a second output is generated. The second output
 contains the polygonal data that's been clipped away.

\item  \verb|obj.GenerateClippedOutputOn ()| -  Control whether a second output is generated. The second output
 contains the polygonal data that's been clipped away.

\item  \verb|obj.GenerateClippedOutputOff ()| -  Control whether a second output is generated. The second output
 contains the polygonal data that's been clipped away.

\item  \verb|obj.SetMergeTolerance (double )| -  Set the tolerance for merging clip intersection points that are near
 the vertices of cells. This tolerance is used to prevent the generation
 of degenerate primitives. Note that only 3D cells actually use this
 instance variable.

\item  \verb|double = obj.GetMergeToleranceMinValue ()| -  Set the tolerance for merging clip intersection points that are near
 the vertices of cells. This tolerance is used to prevent the generation
 of degenerate primitives. Note that only 3D cells actually use this
 instance variable.

\item  \verb|double = obj.GetMergeToleranceMaxValue ()| -  Set the tolerance for merging clip intersection points that are near
 the vertices of cells. This tolerance is used to prevent the generation
 of degenerate primitives. Note that only 3D cells actually use this
 instance variable.

\item  \verb|double = obj.GetMergeTolerance ()| -  Set the tolerance for merging clip intersection points that are near
 the vertices of cells. This tolerance is used to prevent the generation
 of degenerate primitives. Note that only 3D cells actually use this
 instance variable.

\item  \verb|vtkUnstructuredGrid = obj.GetClippedOutput ()| -  Return the Clipped output.

\item  \verb|obj.SetLocator (vtkIncrementalPointLocator locator)| -  Specify a spatial locator for merging points. By default, an
 instance of vtkMergePoints is used.

\item  \verb|vtkIncrementalPointLocator = obj.GetLocator ()| -  Specify a spatial locator for merging points. By default, an
 instance of vtkMergePoints is used.

\item  \verb|obj.CreateDefaultLocator ()| -  Create default locator. Used to create one when none is specified. The 
 locator is used to merge coincident points.

\item  \verb|long = obj.GetMTime ()| -  Return the mtime also considering the locator and clip function.

\end{itemize}
