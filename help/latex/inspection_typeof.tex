\section{TYPEOF Determine the Type of an Argument}

\subsection{Usage}

Returns a string describing the type of an array.  The syntax for its use is
\begin{verbatim}
   y = typeof(x),
\end{verbatim}
The returned string is one of
\begin{itemize}
\item  \verb|'cell'| for cell-arrays

\item  \verb|'struct'| for structure-arrays

\item  \verb|'logical'| for logical arrays

\item  \verb|'uint8'| for unsigned 8-bit integers

\item  \verb|'int8'| for signed 8-bit integers

\item  \verb|'uint16'| for unsigned 16-bit integers

\item  \verb|'int16'| for signed 16-bit integers

\item  \verb|'uint32'| for unsigned 32-bit integers

\item  \verb|'int32'| for signed 32-bit integers

\item  \verb|'float'| for 32-bit floating point numbers

\item  \verb|'double'| for 64-bit floating point numbers

\item  \verb|'string'| for string arrays

\end{itemize}
\subsection{Example}

The following piece of code demonstrates the output of the \verb|typeof| 
command for each possible type.  The first example is with a simple cell array.
\begin{verbatim}
--> typeof({1})

ans = 

 cell
\end{verbatim}
The next example uses the \verb|struct| constructor to make a simple scalar struct.
\begin{verbatim}
--> typeof(struct('foo',3))

ans = 

 struct
\end{verbatim}
The next example uses a comparison between two scalar integers to generate 
a scalar logical type.
\begin{verbatim}
--> typeof(3>5)

ans = 

 logical
\end{verbatim}
For the integers, the typecast operations are used to generate the arguments.
\begin{verbatim}
--> typeof(uint8(3))

ans = 

 uint8

--> typeof(int8(8))

ans = 

 int8

--> typeof(uint16(3))

ans = 

 uint16

--> typeof(int16(8))

ans = 

 int16

--> typeof(uint32(3))

ans = 

 uint32

--> typeof(int32(3))

ans = 

 int32

--> typeof(uint64(3))

ans = 

 uint64

--> typeof(int64(3))

ans = 

 int64
\end{verbatim}
Float, and double can be created using the suffixes.
\begin{verbatim}
--> typeof(1.0f)

ans = 

 single

--> typeof(1.0D)

ans = 

 double

--> typeof(1.0f+i)

ans = 

 single

--> typeof(1.0D+2.0D*i)

ans = 

 double
\end{verbatim}
