\section{TYPEOF Determine the Type of an Argument}

\subsection{Usage}

Returns a string describing the type of an array.  The syntax for its use is
\begin{verbatim}
   y = typeof(x),
\end{verbatim}
The returned string is one of
\begin{itemize}
\item  \verb|'cell'| for cell-arrays

\item  \verb|'struct'| for structure-arrays

\item  \verb|'logical'| for logical arrays

\item  \verb|'uint8'| for unsigned 8-bit integers

\item  \verb|'int8'| for signed 8-bit integers

\item  \verb|'uint16'| for unsigned 16-bit integers

\item  \verb|'int16'| for signed 16-bit integers

\item  \verb|'uint32'| for unsigned 32-bit integers

\item  \verb|'int32'| for signed 32-bit integers

\item  \verb|'float'| for 32-bit floating point numbers

\item  \verb|'double'| for 64-bit floating point numbers

\item  \verb|'string'| for string arrays

\end{itemize}
\subsection{Example}

The following piece of code demonstrates the output of the \verb|typeof| 
command for each possible type.  The first example is with a simple cell array.
@>
The next example uses the \verb|struct| constructor to make a simple scalar struct.
@>
The next example uses a comparison between two scalar integers to generate 
a scalar logical type.
@>
For the integers, the typecast operations are used to generate the arguments.
@>
Float, and double can be created using the suffixes.
@>
