\section{vtkStructuredGridAlgorithm}

\subsection{Usage}


To create an instance of class vtkStructuredGridAlgorithm, simply
invoke its constructor as follows
\begin{verbatim}
  obj = vtkStructuredGridAlgorithm
\end{verbatim}
\subsection{Methods}

The class vtkStructuredGridAlgorithm has several methods that can be used.
  They are listed below.
Note that the documentation is translated automatically from the VTK sources,
and may not be completely intelligible.  When in doubt, consult the VTK website.
In the methods listed below, \verb|obj| is an instance of the vtkStructuredGridAlgorithm class.
\begin{itemize}
\item  \verb|string = obj.GetClassName ()|

\item  \verb|int = obj.IsA (string name)|

\item  \verb|vtkStructuredGridAlgorithm = obj.NewInstance ()|

\item  \verb|vtkStructuredGridAlgorithm = obj.SafeDownCast (vtkObject o)|

\item  \verb|vtkStructuredGrid = obj.GetOutput ()| -  Get the output data object for a port on this algorithm.

\item  \verb|vtkStructuredGrid = obj.GetOutput (int )| -  Get the output data object for a port on this algorithm.

\item  \verb|obj.SetOutput (vtkDataObject d)| -  Get the output data object for a port on this algorithm.

\item  \verb|vtkDataObject = obj.GetInput ()|

\item  \verb|vtkDataObject = obj.GetInput (int port)|

\item  \verb|vtkStructuredGrid = obj.GetStructuredGridInput (int port)|

\item  \verb|obj.SetInput (vtkDataObject )| -  Set an input of this algorithm. You should not override these
 methods because they are not the only way to connect a pipeline.
 Note that these methods support old-style pipeline connections.
 When writing new code you should use the more general
 vtkAlgorithm::SetInputConnection().  These methods transform the
 input index to the input port index, not an index of a connection
 within a single port.

\item  \verb|obj.SetInput (int , vtkDataObject )| -  Set an input of this algorithm. You should not override these
 methods because they are not the only way to connect a pipeline.
 Note that these methods support old-style pipeline connections.
 When writing new code you should use the more general
 vtkAlgorithm::SetInputConnection().  These methods transform the
 input index to the input port index, not an index of a connection
 within a single port.

\item  \verb|obj.AddInput (vtkDataObject )| -  Add an input of this algorithm.  Note that these methods support
 old-style pipeline connections.  When writing new code you should
 use the more general vtkAlgorithm::AddInputConnection().  See
 SetInput() for details.

\item  \verb|obj.AddInput (int , vtkDataObject )| -  Add an input of this algorithm.  Note that these methods support
 old-style pipeline connections.  When writing new code you should
 use the more general vtkAlgorithm::AddInputConnection().  See
 SetInput() for details.

\end{itemize}
