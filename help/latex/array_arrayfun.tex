\section{ARRAYFUN Apply a Function To Elements of an Array}

\subsection{Usage}

The \verb|arrayfun| function is used to apply a function handle
to each element of an input array (or arrays), and to collect
the outputs into an array.  The general syntax for its use is
\begin{verbatim}
   y = arrayfun(fun, x)
\end{verbatim}
where \verb|x| is an N-dimensional array.  In this case, each 
element of the output \verb|y\_i| is defined as \verb|fun(x\_i)|.  You can
also supply multiple arguments to \verb|arrayfun|, provided all of the
arguments are the same size
\begin{verbatim}
   y = arrayfun(fun, x, z,...)
\end{verbatim}
in which case each output \verb|y\_i = fun(x\_i,z\_i,...)|.

If the function returns multiple outputs, then \verb|arrayfun| can be
called with multiple outputs, in which case each output goes to
a separate array output
\begin{verbatim}
   [y1,y2,...] = arrayfun(fun, x, z, ...)
\end{verbatim}
The assumption is that the output types for each call to \verb|fun| is
the same across the inputs.

Finally, some hints can be provided to \verb|arrayfun| using the syntax
\begin{verbatim}
   [y1,y2,...] = arrayfun(fun, x, z, ..., 'param', value, 'param', value)
\end{verbatim}
where \verb|param| and \verb|value| take on the following possible values:
\begin{itemize}
\item  \verb|'UniformOutput'| - if the \verb|value| is \verb|true| then each output of \verb|fun|
   must be a scalar, and the outputs are concatenated into an array the same size
   as the input arrays.  If the \verb|value| is \verb|false| then the outputs are encapsulated
   into a cell array, with each entry in the cell array containing the call to 
   \verb|fun(x\_i,z\_i,...)|.

\item  \verb|'ErrorHandler'| - in this case \verb|value| is a function handle that gets called
  when \verb|fun| throws an error.  If \verb|'ErrorHandler'| is not specified, then \verb|arrayfun|
  allows the error to propogate (i.e., and exception is thrown).

\end{itemize}
