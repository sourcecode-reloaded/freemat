\section{vtkDiscretizableColorTransferFunction}

\subsection{Usage}

 This is a cross between a vtkColorTransferFunction and a vtkLookupTable
 selectively combiniting the functionality of both.
 NOTE: One must call Build() after making any changes to the points
 in the ColorTransferFunction to ensure that the discrete and non-discrete
 version match up.

To create an instance of class vtkDiscretizableColorTransferFunction, simply
invoke its constructor as follows
\begin{verbatim}
  obj = vtkDiscretizableColorTransferFunction
\end{verbatim}
\subsection{Methods}

The class vtkDiscretizableColorTransferFunction has several methods that can be used.
  They are listed below.
Note that the documentation is translated automatically from the VTK sources,
and may not be completely intelligible.  When in doubt, consult the VTK website.
In the methods listed below, \verb|obj| is an instance of the vtkDiscretizableColorTransferFunction class.
\begin{itemize}
\item  \verb|string = obj.GetClassName ()|

\item  \verb|int = obj.IsA (string name)|

\item  \verb|vtkDiscretizableColorTransferFunction = obj.NewInstance ()|

\item  \verb|vtkDiscretizableColorTransferFunction = obj.SafeDownCast (vtkObject o)|

\item  \verb|obj.Build ()| -  Generate discretized lookup table, if applicable.
 This method must be called after changes to the ColorTransferFunction
 otherwise the discretized version will be inconsitent with the 
 non-discretized one.

\item  \verb|obj.SetDiscretize (int )| -  Set if the values are to mapped after discretization. The
 number of discrete values is set by using SetNumberOfValues().
 Not set by default, i.e. color value is determined by
 interpolating at the scalar value.

\item  \verb|int = obj.GetDiscretize ()| -  Set if the values are to mapped after discretization. The
 number of discrete values is set by using SetNumberOfValues().
 Not set by default, i.e. color value is determined by
 interpolating at the scalar value.

\item  \verb|obj.DiscretizeOn ()| -  Set if the values are to mapped after discretization. The
 number of discrete values is set by using SetNumberOfValues().
 Not set by default, i.e. color value is determined by
 interpolating at the scalar value.

\item  \verb|obj.DiscretizeOff ()| -  Set if the values are to mapped after discretization. The
 number of discrete values is set by using SetNumberOfValues().
 Not set by default, i.e. color value is determined by
 interpolating at the scalar value.

\item  \verb|obj.SetUseLogScale (int useLogScale)| -  Get/Set if log scale must be used while mapping scalars
 to colors. The default is 0.

\item  \verb|int = obj.GetUseLogScale ()| -  Get/Set if log scale must be used while mapping scalars
 to colors. The default is 0.

\item  \verb|obj.SetNumberOfValues (vtkIdType number)| -  Set the number of values i.e. colors to be generated in the
 discrete lookup table. This has no effect if Discretize is off.
 The default is 256.

\item  \verb|vtkIdType = obj.GetNumberOfValues ()| -  Set the number of values i.e. colors to be generated in the
 discrete lookup table. This has no effect if Discretize is off.
 The default is 256.

\item  \verb|obj.GetColor (double v, double rgb[3])| -  Map one value through the lookup table and return the color as
 an RGB array of doubles between 0 and 1.

\item  \verb|vtkUnsignedCharArray = obj.MapScalars (vtkDataArray scalars, int colorMode, int component)| -  An internal method maps a data array into a 4-component, unsigned char
 RGBA array. The color mode determines the behavior of mapping. If 
 VTK\_COLOR\_MODE\_DEFAULT is set, then unsigned char data arrays are
 treated as colors (and converted to RGBA if necessary); otherwise, 
 the data is mapped through this instance of ScalarsToColors. The offset
 is used for data arrays with more than one component; it indicates 
 which component to use to do the blending.
 When the component argument is -1, then the this object uses its
 own selected technique to change a vector into a scalar to map.

\item  \verb|obj.SetAlpha (double alpha)| -  Specify an additional opacity (alpha) value to blend with. Values
 != 1 modify the resulting color consistent with the requested
 form of the output. This is typically used by an actor in order to
 blend its opacity.
 Overridden to pass the alpha to the internal vtkLookupTable.

\item  \verb|int = obj.UsingLogScale ()|

\end{itemize}
