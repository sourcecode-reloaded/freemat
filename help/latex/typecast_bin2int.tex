\section{BIN2INT Convert Binary Arrays to Integer}

\subsection{Usage}

Converts the binary decomposition of an integer array back
to an integer array.  The general syntax for its use is
\begin{verbatim}
   y = bin2int(x)
\end{verbatim}
where \verb|x| is a multi-dimensional logical array, where the last
dimension indexes the bit planes (see \verb|int2bin| for an example).
By default, the output of \verb|bin2int| is unsigned \verb|uint32|.  To
get a signed integer, it must be typecast correctly.  A second form for
\verb|bin2int| takes a \verb|'signed'| flag
\begin{verbatim}
   y = bin2int(x,'signed')
\end{verbatim}
in which case the output is signed.
\subsection{Example}

The following piece of code demonstrates various uses of the int2bin
function.  First the simplest example:
\begin{verbatim}
--> A = [2;5;6;2]

A = 
 2 
 5 
 6 
 2 

--> B = int2bin(A,8)

B = 
 0 0 0 0 0 0 1 0 
 0 0 0 0 0 1 0 1 
 0 0 0 0 0 1 1 0 
 0 0 0 0 0 0 1 0 

--> bin2int(B)

ans = 
 2 
 5 
 6 
 2 

--> A = [1;2;-5;2]

A = 
  1 
  2 
 -5 
  2 

--> B = int2bin(A,8)

B = 
 0 0 0 0 0 0 0 1 
 0 0 0 0 0 0 1 0 
 1 1 1 1 1 0 1 1 
 0 0 0 0 0 0 1 0 

--> bin2int(B)

ans = 
   1 
   2 
 251 
   2 

--> int32(bin2int(B))

ans = 
   1 
   2 
 251 
   2 
\end{verbatim}
\subsection{Tets}

