\section{vtkImplicitPlaneRepresentation}

\subsection{Usage}

 This class is a concrete representation for the
 vtkImplicitPlaneWidget2. It represents an infinite plane defined by a
 normal and point in the context of a bounding box. Through interaction
 with the widget, the plane can be manipulated by adjusting the plane
 normal or moving the origin point.

 To use this representation, you normally define a (plane) origin and (plane)
 normal. The PlaceWidget() method is also used to intially position the
 representation.

To create an instance of class vtkImplicitPlaneRepresentation, simply
invoke its constructor as follows
\begin{verbatim}
  obj = vtkImplicitPlaneRepresentation
\end{verbatim}
\subsection{Methods}

The class vtkImplicitPlaneRepresentation has several methods that can be used.
  They are listed below.
Note that the documentation is translated automatically from the VTK sources,
and may not be completely intelligible.  When in doubt, consult the VTK website.
In the methods listed below, \verb|obj| is an instance of the vtkImplicitPlaneRepresentation class.
\begin{itemize}
\item  \verb|string = obj.GetClassName ()| -  Standard methods for the class.

\item  \verb|int = obj.IsA (string name)| -  Standard methods for the class.

\item  \verb|vtkImplicitPlaneRepresentation = obj.NewInstance ()| -  Standard methods for the class.

\item  \verb|vtkImplicitPlaneRepresentation = obj.SafeDownCast (vtkObject o)| -  Standard methods for the class.

\item  \verb|obj.SetOrigin (double x, double y, double z)| -  Get the origin of the plane.

\item  \verb|obj.SetOrigin (double x[3])| -  Get the origin of the plane.

\item  \verb|double = obj.GetOrigin ()| -  Get the origin of the plane.

\item  \verb|obj.GetOrigin (double xyz[3])| -  Get the origin of the plane.

\item  \verb|obj.SetNormal (double x, double y, double z)| -  Get the normal to the plane.

\item  \verb|obj.SetNormal (double x[3])| -  Get the normal to the plane.

\item  \verb|double = obj.GetNormal ()| -  Get the normal to the plane.

\item  \verb|obj.GetNormal (double xyz[3])| -  Get the normal to the plane.

\item  \verb|obj.SetNormalToXAxis (int )| -  Force the plane widget to be aligned with one of the x-y-z axes.
 If one axis is set on, the other two will be set off.
 Remember that when the state changes, a ModifiedEvent is invoked.
 This can be used to snap the plane to the axes if it is orginally
 not aligned.

\item  \verb|int = obj.GetNormalToXAxis ()| -  Force the plane widget to be aligned with one of the x-y-z axes.
 If one axis is set on, the other two will be set off.
 Remember that when the state changes, a ModifiedEvent is invoked.
 This can be used to snap the plane to the axes if it is orginally
 not aligned.

\item  \verb|obj.NormalToXAxisOn ()| -  Force the plane widget to be aligned with one of the x-y-z axes.
 If one axis is set on, the other two will be set off.
 Remember that when the state changes, a ModifiedEvent is invoked.
 This can be used to snap the plane to the axes if it is orginally
 not aligned.

\item  \verb|obj.NormalToXAxisOff ()| -  Force the plane widget to be aligned with one of the x-y-z axes.
 If one axis is set on, the other two will be set off.
 Remember that when the state changes, a ModifiedEvent is invoked.
 This can be used to snap the plane to the axes if it is orginally
 not aligned.

\item  \verb|obj.SetNormalToYAxis (int )| -  Force the plane widget to be aligned with one of the x-y-z axes.
 If one axis is set on, the other two will be set off.
 Remember that when the state changes, a ModifiedEvent is invoked.
 This can be used to snap the plane to the axes if it is orginally
 not aligned.

\item  \verb|int = obj.GetNormalToYAxis ()| -  Force the plane widget to be aligned with one of the x-y-z axes.
 If one axis is set on, the other two will be set off.
 Remember that when the state changes, a ModifiedEvent is invoked.
 This can be used to snap the plane to the axes if it is orginally
 not aligned.

\item  \verb|obj.NormalToYAxisOn ()| -  Force the plane widget to be aligned with one of the x-y-z axes.
 If one axis is set on, the other two will be set off.
 Remember that when the state changes, a ModifiedEvent is invoked.
 This can be used to snap the plane to the axes if it is orginally
 not aligned.

\item  \verb|obj.NormalToYAxisOff ()| -  Force the plane widget to be aligned with one of the x-y-z axes.
 If one axis is set on, the other two will be set off.
 Remember that when the state changes, a ModifiedEvent is invoked.
 This can be used to snap the plane to the axes if it is orginally
 not aligned.

\item  \verb|obj.SetNormalToZAxis (int )| -  Force the plane widget to be aligned with one of the x-y-z axes.
 If one axis is set on, the other two will be set off.
 Remember that when the state changes, a ModifiedEvent is invoked.
 This can be used to snap the plane to the axes if it is orginally
 not aligned.

\item  \verb|int = obj.GetNormalToZAxis ()| -  Force the plane widget to be aligned with one of the x-y-z axes.
 If one axis is set on, the other two will be set off.
 Remember that when the state changes, a ModifiedEvent is invoked.
 This can be used to snap the plane to the axes if it is orginally
 not aligned.

\item  \verb|obj.NormalToZAxisOn ()| -  Force the plane widget to be aligned with one of the x-y-z axes.
 If one axis is set on, the other two will be set off.
 Remember that when the state changes, a ModifiedEvent is invoked.
 This can be used to snap the plane to the axes if it is orginally
 not aligned.

\item  \verb|obj.NormalToZAxisOff ()| -  Force the plane widget to be aligned with one of the x-y-z axes.
 If one axis is set on, the other two will be set off.
 Remember that when the state changes, a ModifiedEvent is invoked.
 This can be used to snap the plane to the axes if it is orginally
 not aligned.

\item  \verb|obj.SetTubing (int )| -  Turn on/off tubing of the wire outline of the plane. The tube thickens
 the line by wrapping with a vtkTubeFilter.

\item  \verb|int = obj.GetTubing ()| -  Turn on/off tubing of the wire outline of the plane. The tube thickens
 the line by wrapping with a vtkTubeFilter.

\item  \verb|obj.TubingOn ()| -  Turn on/off tubing of the wire outline of the plane. The tube thickens
 the line by wrapping with a vtkTubeFilter.

\item  \verb|obj.TubingOff ()| -  Turn on/off tubing of the wire outline of the plane. The tube thickens
 the line by wrapping with a vtkTubeFilter.

\item  \verb|obj.SetDrawPlane (int plane)| -  Enable/disable the drawing of the plane. In some cases the plane
 interferes with the object that it is operating on (i.e., the
 plane interferes with the cut surface it produces producing
 z-buffer artifacts.)

\item  \verb|int = obj.GetDrawPlane ()| -  Enable/disable the drawing of the plane. In some cases the plane
 interferes with the object that it is operating on (i.e., the
 plane interferes with the cut surface it produces producing
 z-buffer artifacts.)

\item  \verb|obj.DrawPlaneOn ()| -  Enable/disable the drawing of the plane. In some cases the plane
 interferes with the object that it is operating on (i.e., the
 plane interferes with the cut surface it produces producing
 z-buffer artifacts.)

\item  \verb|obj.DrawPlaneOff ()| -  Enable/disable the drawing of the plane. In some cases the plane
 interferes with the object that it is operating on (i.e., the
 plane interferes with the cut surface it produces producing
 z-buffer artifacts.)

\item  \verb|obj.SetOutlineTranslation (int )| -  Turn on/off the ability to translate the bounding box by grabbing it
 with the left mouse button.

\item  \verb|int = obj.GetOutlineTranslation ()| -  Turn on/off the ability to translate the bounding box by grabbing it
 with the left mouse button.

\item  \verb|obj.OutlineTranslationOn ()| -  Turn on/off the ability to translate the bounding box by grabbing it
 with the left mouse button.

\item  \verb|obj.OutlineTranslationOff ()| -  Turn on/off the ability to translate the bounding box by grabbing it
 with the left mouse button.

\item  \verb|obj.SetOutsideBounds (int )| -  Turn on/off the ability to move the widget outside of the bounds
 specified in the initial PlaceWidget() invocation.

\item  \verb|int = obj.GetOutsideBounds ()| -  Turn on/off the ability to move the widget outside of the bounds
 specified in the initial PlaceWidget() invocation.

\item  \verb|obj.OutsideBoundsOn ()| -  Turn on/off the ability to move the widget outside of the bounds
 specified in the initial PlaceWidget() invocation.

\item  \verb|obj.OutsideBoundsOff ()| -  Turn on/off the ability to move the widget outside of the bounds
 specified in the initial PlaceWidget() invocation.

\item  \verb|obj.SetScaleEnabled (int )| -  Turn on/off the ability to scale the widget with the mouse.

\item  \verb|int = obj.GetScaleEnabled ()| -  Turn on/off the ability to scale the widget with the mouse.

\item  \verb|obj.ScaleEnabledOn ()| -  Turn on/off the ability to scale the widget with the mouse.

\item  \verb|obj.ScaleEnabledOff ()| -  Turn on/off the ability to scale the widget with the mouse.

\item  \verb|obj.GetPolyData (vtkPolyData pd)| -  Grab the polydata that defines the plane. The polydata contains a single
 polygon that is clipped by the bounding box.

\item  \verb|vtkPolyDataAlgorithm = obj.GetPolyDataAlgorithm ()| -  Satisfies superclass API.  This returns a pointer to the underlying
 PolyData (which represents the plane).

\item  \verb|obj.GetPlane (vtkPlane plane)| -  Get the implicit function for the plane. The user must provide the
 instance of the class vtkPlane. Note that vtkPlane is a subclass of
 vtkImplicitFunction, meaning that it can be used by a variety of filters
 to perform clipping, cutting, and selection of data.

\item  \verb|obj.UpdatePlacement (void )| -  Satisfies the superclass API.  This will change the state of the widget
 to match changes that have been made to the underlying PolyDataSource

\item  \verb|vtkProperty = obj.GetNormalProperty ()| -  Get the properties on the normal (line and cone).

\item  \verb|vtkProperty = obj.GetSelectedNormalProperty ()| -  Get the properties on the normal (line and cone).

\item  \verb|vtkProperty = obj.GetPlaneProperty ()| -  Get the plane properties. The properties of the plane when selected 
 and unselected can be manipulated.

\item  \verb|vtkProperty = obj.GetSelectedPlaneProperty ()| -  Get the plane properties. The properties of the plane when selected 
 and unselected can be manipulated.

\item  \verb|vtkProperty = obj.GetOutlineProperty ()| -  Get the property of the outline.

\item  \verb|vtkProperty = obj.GetSelectedOutlineProperty ()| -  Get the property of the outline.

\item  \verb|vtkProperty = obj.GetEdgesProperty ()| -  Get the property of the intersection edges. (This property also
 applies to the edges when tubed.)

\item  \verb|int = obj.ComputeInteractionState (int X, int Y, int modify)| -  Methods to interface with the vtkSliderWidget.

\item  \verb|obj.PlaceWidget (double bounds[6])| -  Methods to interface with the vtkSliderWidget.

\item  \verb|obj.BuildRepresentation ()| -  Methods to interface with the vtkSliderWidget.

\item  \verb|obj.StartWidgetInteraction (double eventPos[2])| -  Methods to interface with the vtkSliderWidget.

\item  \verb|obj.WidgetInteraction (double newEventPos[2])| -  Methods to interface with the vtkSliderWidget.

\item  \verb|obj.EndWidgetInteraction (double newEventPos[2])| -  Methods to interface with the vtkSliderWidget.

\item  \verb|double = obj.GetBounds ()|

\item  \verb|obj.GetActors (vtkPropCollection pc)|

\item  \verb|obj.ReleaseGraphicsResources (vtkWindow )|

\item  \verb|int = obj.RenderOpaqueGeometry (vtkViewport )|

\item  \verb|int = obj.RenderTranslucentPolygonalGeometry (vtkViewport )|

\item  \verb|int = obj.HasTranslucentPolygonalGeometry ()|

\item  \verb|obj.SetInteractionState (int )| -  The interaction state may be set from a widget (e.g.,
 vtkImplicitPlaneWidget2) or other object. This controls how the
 interaction with the widget proceeds. Normally this method is used as
 part of a handshaking process with the widget: First
 ComputeInteractionState() is invoked that returns a state based on
 geometric considerations (i.e., cursor near a widget feature), then
 based on events, the widget may modify this further.

\item  \verb|int = obj.GetInteractionStateMinValue ()| -  The interaction state may be set from a widget (e.g.,
 vtkImplicitPlaneWidget2) or other object. This controls how the
 interaction with the widget proceeds. Normally this method is used as
 part of a handshaking process with the widget: First
 ComputeInteractionState() is invoked that returns a state based on
 geometric considerations (i.e., cursor near a widget feature), then
 based on events, the widget may modify this further.

\item  \verb|int = obj.GetInteractionStateMaxValue ()| -  The interaction state may be set from a widget (e.g.,
 vtkImplicitPlaneWidget2) or other object. This controls how the
 interaction with the widget proceeds. Normally this method is used as
 part of a handshaking process with the widget: First
 ComputeInteractionState() is invoked that returns a state based on
 geometric considerations (i.e., cursor near a widget feature), then
 based on events, the widget may modify this further.

\item  \verb|obj.SetRepresentationState (int )| -  Sets the visual appearance of the representation based on the
 state it is in. This state is usually the same as InteractionState.

\item  \verb|int = obj.GetRepresentationState ()| -  Sets the visual appearance of the representation based on the
 state it is in. This state is usually the same as InteractionState.

\end{itemize}
