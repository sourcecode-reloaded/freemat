\section{vtkInteractorStyleUnicam}

\subsection{Usage}

 UniCam is a camera interactor.  Here, just the primary features of the
 UniCam technique are implemented.  UniCam requires just one mouse button
 and supports context sensitive dollying, panning, and rotation.  (In this
 implementation, it uses the right mouse button, leaving the middle and
 left available for other functions.) For more information, see the paper
 at:
 
    ftp://ftp.cs.brown.edu/pub/papers/graphics/research/unicam.pdf
 
 The following is a brief description of the UniCam Camera Controls.  You
 can perform 3 operations on the camera: rotate, pan, and dolly the camera.
 All operations are reached through the right mouse button \& mouse
 movements.
 
 IMPORTANT: UniCam assumes there is an axis that makes sense as a ''up''
 vector for the world.  By default, this axis is defined to be the
 vector <0,0,1>.  You can set it explicitly for the data you are
 viewing with the 'SetWorldUpVector(..)' method in C++, or similarly
 in Tcl/Tk (or other interpreted languages).
 
 1. ROTATE:
 
 Position the cursor over the point you wish to rotate around and press and
 release the left mouse button.  A 'focus dot' appears indicating the
 point that will be the center of rotation.  To rotate, press and hold the
 left mouse button and drag the mouse.. release the button to complete the
 rotation.
 
 Rotations can be done without placing a focus dot first by moving the
 mouse cursor to within 10% of the window border \& pressing and holding the
 left button followed by dragging the mouse.  The last focus dot position
 will be re-used.
 
 2. PAN:
 
 Click and hold the left mouse button, and initially move the mouse left
 or right.  The point under the initial pick will pick correlate w/ the
 mouse tip-- (i.e., direct manipulation).
 
 3. DOLLY (+ PAN):
 
 Click and hold the left mouse button, and initially move the mouse up or
 down.  Moving the mouse down will dolly towards the picked point, and moving
 the mouse up will dolly away from it.  Dollying occurs relative to the
 picked point which simplifies the task of dollying towards a region of
 interest. Left and right mouse movements will pan the camera left and right.
 

To create an instance of class vtkInteractorStyleUnicam, simply
invoke its constructor as follows
\begin{verbatim}
  obj = vtkInteractorStyleUnicam
\end{verbatim}
\subsection{Methods}

The class vtkInteractorStyleUnicam has several methods that can be used.
  They are listed below.
Note that the documentation is translated automatically from the VTK sources,
and may not be completely intelligible.  When in doubt, consult the VTK website.
In the methods listed below, \verb|obj| is an instance of the vtkInteractorStyleUnicam class.
\begin{itemize}
\item  \verb|string = obj.GetClassName ()|

\item  \verb|int = obj.IsA (string name)|

\item  \verb|vtkInteractorStyleUnicam = obj.NewInstance ()|

\item  \verb|vtkInteractorStyleUnicam = obj.SafeDownCast (vtkObject o)|

\item  \verb|obj.SetWorldUpVector (double a[3])|

\item  \verb|obj.SetWorldUpVector (double x, double y, double z)|

\item  \verb|double = obj. GetWorldUpVector ()|

\item  \verb|obj.OnMouseMove ()| -  Concrete implementation of event bindings

\item  \verb|obj.OnLeftButtonDown ()| -  Concrete implementation of event bindings

\item  \verb|obj.OnLeftButtonUp ()| -  Concrete implementation of event bindings

\item  \verb|obj.OnLeftButtonMove ()| -  Concrete implementation of event bindings

\item  \verb|obj.OnTimer ()| -  OnTimer calls RotateCamera, RotateActor etc which should be overridden by
 style subclasses.

\end{itemize}
