\section{vtkImageContinuousErode3D}

\subsection{Usage}

 vtkImageContinuousErode3D replaces a pixel with the minimum over
 an ellipsoidal neighborhood.  If KernelSize of an axis is 1, no processing
 is done on that axis.

To create an instance of class vtkImageContinuousErode3D, simply
invoke its constructor as follows
\begin{verbatim}
  obj = vtkImageContinuousErode3D
\end{verbatim}
\subsection{Methods}

The class vtkImageContinuousErode3D has several methods that can be used.
  They are listed below.
Note that the documentation is translated automatically from the VTK sources,
and may not be completely intelligible.  When in doubt, consult the VTK website.
In the methods listed below, \verb|obj| is an instance of the vtkImageContinuousErode3D class.
\begin{itemize}
\item  \verb|string = obj.GetClassName ()| -  Construct an instance of vtkImageContinuousErode3D filter.
 By default zero values are eroded.

\item  \verb|int = obj.IsA (string name)| -  Construct an instance of vtkImageContinuousErode3D filter.
 By default zero values are eroded.

\item  \verb|vtkImageContinuousErode3D = obj.NewInstance ()| -  Construct an instance of vtkImageContinuousErode3D filter.
 By default zero values are eroded.

\item  \verb|vtkImageContinuousErode3D = obj.SafeDownCast (vtkObject o)| -  Construct an instance of vtkImageContinuousErode3D filter.
 By default zero values are eroded.

\item  \verb|obj.SetKernelSize (int size0, int size1, int size2)| -  This method sets the size of the neighborhood.  It also sets the 
 default middle of the neighborhood and computes the elliptical foot print.

\end{itemize}
