\section{vtkModifiedBSPTree}

\subsection{Usage}

 vtkModifiedBSPTree creates an evenly balanced BSP tree using a top down
 implementation. Axis aligned split planes are found which evenly divide
 cells into two buckets. Generally a split plane will intersect some cells
 and these are usually stored in both child nodes of the current parent.
 (Or split into separate cells which we cannot consider in this case).
 Storing cells in multiple buckets creates problems associated with multiple
 tests against rays and increases the required storage as complex meshes
 will have many cells straddling a split plane (and further splits may
 cause multiple copies of these).

 During a discussion with Arno Formella in 1998 he suggested using
 a third child node to store objects which straddle split planes. I've not
 seen this published (Yes! - see below), but thought it worth trying. This
 implementation of the BSP tree creates a third child node for storing cells
 lying across split planes, the third cell may overlap the other two, but the
 two 'proper' nodes otherwise conform to usual BSP rules.

 The advantage of this implementation is cells only ever lie in one node
 and mailbox testing is avoided. All BBoxes are axis aligned and a ray cast
 uses an efficient search strategy based on near/far nodes and rejects
 all BBoxes using simple tests.

 For fast raytracing, 6 copies of cell lists are stored in each leaf node
 each list is in axis sorted order +/- x,y,z and cells are always tested
 in the direction of the ray dominant axis. Once an intersection is found
 any cell or BBox with a closest point further than the I-point can be
 instantly rejected and raytracing stops as soon as no nodes can be closer
 than the current best intersection point.

 The addition of the 'middle' node upsets the optimal balance of the tree,
 but is a minor overhead during the raytrace. Each child node is contracted
 such that it tightly fits all cells inside it, enabling further ray/box
 rejections.

 This class is intented for persons requiring many ray tests and is optimized
 for this purpose. As no cell ever lies in more than one leaf node, and parent
 nodes do not maintain cell lists, the memory overhead of the sorted cell
 lists is 6*num\_cells*4 for 6 lists of ints, each num\_cells in length.
 The memory requirement of the nodes themselves is usually of minor
 significance.

 Subdividision is controlled by MaxCellsPerNode - any node with more than
 this number will be subdivided providing a good split plane can be found and
 the max depth is not exceeded.

 The average cells per leaf will usually be around half the MaxCellsPerNode,
 though the middle node is usually sparsely populated and lowers the average
 slightly. The middle node will not be created when not needed.
 Subdividing down to very small cells per node is not generally suggested
 as then the 6 stored cell lists are effectively redundant.

 Values of MaxcellsPerNode of around 16->128 depending on dataset size will
 usually give good results.

 Cells are only sorted into 6 lists once - before tree creation, each node
 segments the lists and passes them down to the new child nodes whilst
 maintaining sorted order. This makes for an efficient subdivision strategy.

 NB. The following reference has been sent to me
   @Article{formella-1995-ray,
     author =     ''Arno Formella and Christian Gill'',
     title =      ''{Ray Tracing: A Quantitative Analysis and a New
                   Practical Algorithm}'',
     journal =    ''{The Visual Computer}'',
     year =       ''{1995}'',
     month =       dec,
     pages =      ''{465--476}'',
     volume =     ''{11}'',
     number =     ''{9}'',
     publisher =  ''{Springer}'',
     keywords =   ''{ray tracing, space subdivision, plane traversal,
                    octree, clustering, benchmark scenes}'',
     annote =     ''{We present a new method to accelerate the process of
                    finding nearest ray--object intersections in ray
                    tracing. The algorithm consumes an amount of memory
                    more or less linear in the number of objects. The basic
                    ideas can be characterized with a modified BSP--tree
                    and plane traversal. Plane traversal is a fast linear
                    time algorithm to find the closest intersection point
                    in a list of bounding volumes hit by a ray. We use
                    plane traversal at every node of the high outdegree
                    BSP--tree. Our implementation is competitive to fast
                    ray tracing programs. We present a benchmark suite
                    which allows for an extensive comparison of ray tracing
                    algorithms.}'',
   }

 .SECTION Thanks
  John Biddiscombe for developing and contributing this class

 .SECTION ToDo
 -------------
 Implement intersection heap for testing rays against transparent objects

 .SECTION Style
 --------------
 This class is currently maintained by J. Biddiscombe who has specially
 requested that the code style not be modified to the kitware standard.
 Please respect the contribution of this class by keeping the style
 as close as possible to the author's original.


To create an instance of class vtkModifiedBSPTree, simply
invoke its constructor as follows
\begin{verbatim}
  obj = vtkModifiedBSPTree
\end{verbatim}
\subsection{Methods}

The class vtkModifiedBSPTree has several methods that can be used.
  They are listed below.
Note that the documentation is translated automatically from the VTK sources,
and may not be completely intelligible.  When in doubt, consult the VTK website.
In the methods listed below, \verb|obj| is an instance of the vtkModifiedBSPTree class.
\begin{itemize}
\item  \verb|string = obj.GetClassName ()| -  Standard Type-Macro

\item  \verb|int = obj.IsA (string name)| -  Standard Type-Macro

\item  \verb|vtkModifiedBSPTree = obj.NewInstance ()| -  Standard Type-Macro

\item  \verb|vtkModifiedBSPTree = obj.SafeDownCast (vtkObject o)| -  Standard Type-Macro

\item  \verb|obj.FreeSearchStructure ()| -  Free tree memory

\item  \verb|obj.BuildLocator ()| -  Build Tree

\end{itemize}
