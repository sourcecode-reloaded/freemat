\section{vtkVolumeMapper}

\subsection{Usage}

 vtkVolumeMapper is the abstract definition of a volume mapper for regular
 rectilinear data (vtkImageData).  Several  basic types of volume mappers 
 are supported. 

To create an instance of class vtkVolumeMapper, simply
invoke its constructor as follows
\begin{verbatim}
  obj = vtkVolumeMapper
\end{verbatim}
\subsection{Methods}

The class vtkVolumeMapper has several methods that can be used.
  They are listed below.
Note that the documentation is translated automatically from the VTK sources,
and may not be completely intelligible.  When in doubt, consult the VTK website.
In the methods listed below, \verb|obj| is an instance of the vtkVolumeMapper class.
\begin{itemize}
\item  \verb|string = obj.GetClassName ()|

\item  \verb|int = obj.IsA (string name)|

\item  \verb|vtkVolumeMapper = obj.NewInstance ()|

\item  \verb|vtkVolumeMapper = obj.SafeDownCast (vtkObject o)|

\item  \verb|obj.SetInput (vtkImageData )| -  Set/Get the input data

\item  \verb|obj.SetInput (vtkDataSet )| -  Set/Get the input data

\item  \verb|vtkImageData = obj.GetInput ()| -  Set/Get the input data

\item  \verb|obj.SetBlendMode (int )| -  Set/Get the blend mode. Currently this is only supported
 by the vtkFixedPointVolumeRayCastMapper - other mappers
 have different ways to set this (supplying a function
 to a vtkVolumeRayCastMapper) or don't have any options
 (vtkVolumeTextureMapper2D supports only compositing)

\item  \verb|obj.SetBlendModeToComposite ()| -  Set/Get the blend mode. Currently this is only supported
 by the vtkFixedPointVolumeRayCastMapper - other mappers
 have different ways to set this (supplying a function
 to a vtkVolumeRayCastMapper) or don't have any options
 (vtkVolumeTextureMapper2D supports only compositing)

\item  \verb|obj.SetBlendModeToMaximumIntensity ()| -  Set/Get the blend mode. Currently this is only supported
 by the vtkFixedPointVolumeRayCastMapper - other mappers
 have different ways to set this (supplying a function
 to a vtkVolumeRayCastMapper) or don't have any options
 (vtkVolumeTextureMapper2D supports only compositing)

\item  \verb|obj.SetBlendModeToMinimumIntensity ()| -  Set/Get the blend mode. Currently this is only supported
 by the vtkFixedPointVolumeRayCastMapper - other mappers
 have different ways to set this (supplying a function
 to a vtkVolumeRayCastMapper) or don't have any options
 (vtkVolumeTextureMapper2D supports only compositing)

\item  \verb|int = obj.GetBlendMode ()| -  Set/Get the blend mode. Currently this is only supported
 by the vtkFixedPointVolumeRayCastMapper - other mappers
 have different ways to set this (supplying a function
 to a vtkVolumeRayCastMapper) or don't have any options
 (vtkVolumeTextureMapper2D supports only compositing)

\item  \verb|obj.SetCropping (int )| -  Turn On/Off orthogonal cropping. (Clipping planes are
 perpendicular to the coordinate axes.)

\item  \verb|int = obj.GetCroppingMinValue ()| -  Turn On/Off orthogonal cropping. (Clipping planes are
 perpendicular to the coordinate axes.)

\item  \verb|int = obj.GetCroppingMaxValue ()| -  Turn On/Off orthogonal cropping. (Clipping planes are
 perpendicular to the coordinate axes.)

\item  \verb|int = obj.GetCropping ()| -  Turn On/Off orthogonal cropping. (Clipping planes are
 perpendicular to the coordinate axes.)

\item  \verb|obj.CroppingOn ()| -  Turn On/Off orthogonal cropping. (Clipping planes are
 perpendicular to the coordinate axes.)

\item  \verb|obj.CroppingOff ()| -  Turn On/Off orthogonal cropping. (Clipping planes are
 perpendicular to the coordinate axes.)

\item  \verb|obj.SetCroppingRegionPlanes (double , double , double , double , double , double )| -  Set/Get the Cropping Region Planes ( xmin, xmax, ymin, ymax, zmin, zmax )
 These planes are defined in volume coordinates - spacing and origin are
 considered.

\item  \verb|obj.SetCroppingRegionPlanes (double  a[6])| -  Set/Get the Cropping Region Planes ( xmin, xmax, ymin, ymax, zmin, zmax )
 These planes are defined in volume coordinates - spacing and origin are
 considered.

\item  \verb|double = obj. GetCroppingRegionPlanes ()| -  Set/Get the Cropping Region Planes ( xmin, xmax, ymin, ymax, zmin, zmax )
 These planes are defined in volume coordinates - spacing and origin are
 considered.

\item  \verb|double = obj. GetVoxelCroppingRegionPlanes ()| -  Get the cropping region planes in voxels. Only valid during the 
 rendering process

\item  \verb|obj.SetCroppingRegionFlags (int )| -  Set the flags for the cropping regions. The clipping planes divide the
 volume into 27 regions - there is one bit for each region. The regions 
 start from the one containing voxel (0,0,0), moving along the x axis 
 fastest, the y axis next, and the z axis slowest. These are represented 
 from the lowest bit to bit number 27 in the integer containing the 
 flags. There are several convenience functions to set some common 
 configurations - subvolume (the default), fence (between any of the 
 clip plane pairs), inverted fence, cross (between any two of the 
 clip plane pairs) and inverted cross.

\item  \verb|int = obj.GetCroppingRegionFlagsMinValue ()| -  Set the flags for the cropping regions. The clipping planes divide the
 volume into 27 regions - there is one bit for each region. The regions 
 start from the one containing voxel (0,0,0), moving along the x axis 
 fastest, the y axis next, and the z axis slowest. These are represented 
 from the lowest bit to bit number 27 in the integer containing the 
 flags. There are several convenience functions to set some common 
 configurations - subvolume (the default), fence (between any of the 
 clip plane pairs), inverted fence, cross (between any two of the 
 clip plane pairs) and inverted cross.

\item  \verb|int = obj.GetCroppingRegionFlagsMaxValue ()| -  Set the flags for the cropping regions. The clipping planes divide the
 volume into 27 regions - there is one bit for each region. The regions 
 start from the one containing voxel (0,0,0), moving along the x axis 
 fastest, the y axis next, and the z axis slowest. These are represented 
 from the lowest bit to bit number 27 in the integer containing the 
 flags. There are several convenience functions to set some common 
 configurations - subvolume (the default), fence (between any of the 
 clip plane pairs), inverted fence, cross (between any two of the 
 clip plane pairs) and inverted cross.

\item  \verb|int = obj.GetCroppingRegionFlags ()| -  Set the flags for the cropping regions. The clipping planes divide the
 volume into 27 regions - there is one bit for each region. The regions 
 start from the one containing voxel (0,0,0), moving along the x axis 
 fastest, the y axis next, and the z axis slowest. These are represented 
 from the lowest bit to bit number 27 in the integer containing the 
 flags. There are several convenience functions to set some common 
 configurations - subvolume (the default), fence (between any of the 
 clip plane pairs), inverted fence, cross (between any two of the 
 clip plane pairs) and inverted cross.

\item  \verb|obj.SetCroppingRegionFlagsToSubVolume ()| -  Set the flags for the cropping regions. The clipping planes divide the
 volume into 27 regions - there is one bit for each region. The regions 
 start from the one containing voxel (0,0,0), moving along the x axis 
 fastest, the y axis next, and the z axis slowest. These are represented 
 from the lowest bit to bit number 27 in the integer containing the 
 flags. There are several convenience functions to set some common 
 configurations - subvolume (the default), fence (between any of the 
 clip plane pairs), inverted fence, cross (between any two of the 
 clip plane pairs) and inverted cross.

\item  \verb|obj.SetCroppingRegionFlagsToFence ()| -  Set the flags for the cropping regions. The clipping planes divide the
 volume into 27 regions - there is one bit for each region. The regions 
 start from the one containing voxel (0,0,0), moving along the x axis 
 fastest, the y axis next, and the z axis slowest. These are represented 
 from the lowest bit to bit number 27 in the integer containing the 
 flags. There are several convenience functions to set some common 
 configurations - subvolume (the default), fence (between any of the 
 clip plane pairs), inverted fence, cross (between any two of the 
 clip plane pairs) and inverted cross.

\item  \verb|obj.SetCroppingRegionFlagsToInvertedFence ()| -  Set the flags for the cropping regions. The clipping planes divide the
 volume into 27 regions - there is one bit for each region. The regions 
 start from the one containing voxel (0,0,0), moving along the x axis 
 fastest, the y axis next, and the z axis slowest. These are represented 
 from the lowest bit to bit number 27 in the integer containing the 
 flags. There are several convenience functions to set some common 
 configurations - subvolume (the default), fence (between any of the 
 clip plane pairs), inverted fence, cross (between any two of the 
 clip plane pairs) and inverted cross.

\item  \verb|obj.SetCroppingRegionFlagsToCross ()| -  Set the flags for the cropping regions. The clipping planes divide the
 volume into 27 regions - there is one bit for each region. The regions 
 start from the one containing voxel (0,0,0), moving along the x axis 
 fastest, the y axis next, and the z axis slowest. These are represented 
 from the lowest bit to bit number 27 in the integer containing the 
 flags. There are several convenience functions to set some common 
 configurations - subvolume (the default), fence (between any of the 
 clip plane pairs), inverted fence, cross (between any two of the 
 clip plane pairs) and inverted cross.

\item  \verb|obj.SetCroppingRegionFlagsToInvertedCross ()| -  Set the flags for the cropping regions. The clipping planes divide the
 volume into 27 regions - there is one bit for each region. The regions 
 start from the one containing voxel (0,0,0), moving along the x axis 
 fastest, the y axis next, and the z axis slowest. These are represented 
 from the lowest bit to bit number 27 in the integer containing the 
 flags. There are several convenience functions to set some common 
 configurations - subvolume (the default), fence (between any of the 
 clip plane pairs), inverted fence, cross (between any two of the 
 clip plane pairs) and inverted cross.

\end{itemize}
