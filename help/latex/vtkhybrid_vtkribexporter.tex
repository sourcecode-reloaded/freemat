\section{vtkRIBExporter}

\subsection{Usage}

 vtkRIBExporter is a concrete subclass of vtkExporter that writes a
 Renderman .RIB files. The input specifies a vtkRenderWindow. All
 visible actors and lights will be included in the rib file. The
 following file naming conventions apply:
   rib file - FilePrefix.rib
   image file created by RenderMan - FilePrefix.tif
   texture files - TexturePrefix\_0xADDR\_MTIME.tif
 This object does NOT generate an image file. The user must run either
 RenderMan or a RenderMan emulator like Blue Moon Ray Tracer (BMRT).
 vtk properties are convert to Renderman shaders as follows:
   Normal property, no texture map - plastic.sl
   Normal property with texture map - txtplastic.sl
 These two shaders must be compiled by the rendering package being
 used.  vtkRIBExporter also supports custom shaders. The shaders are
 written using the Renderman Shading Language. See ''The Renderman
 Companion'', ISBN 0-201-50868, 1989 for details on writing shaders.
 vtkRIBProperty specifies the declarations and parameter settings for
 custom shaders.
 Tcl Example: generate a rib file for the current rendering.
 vtkRIBExporter myRIB
   myRIB SetInput \$renWin
   myRIB SetFIlePrefix mine
   myRIB Write
 This will create a file mine.rib. After running this file through
 a Renderman renderer a file mine.tif will contain the rendered image.


To create an instance of class vtkRIBExporter, simply
invoke its constructor as follows
\begin{verbatim}
  obj = vtkRIBExporter
\end{verbatim}
\subsection{Methods}

The class vtkRIBExporter has several methods that can be used.
  They are listed below.
Note that the documentation is translated automatically from the VTK sources,
and may not be completely intelligible.  When in doubt, consult the VTK website.
In the methods listed below, \verb|obj| is an instance of the vtkRIBExporter class.
\begin{itemize}
\item  \verb|string = obj.GetClassName ()|

\item  \verb|int = obj.IsA (string name)|

\item  \verb|vtkRIBExporter = obj.NewInstance ()|

\item  \verb|vtkRIBExporter = obj.SafeDownCast (vtkObject o)|

\item  \verb|obj.SetSize (int , int )|

\item  \verb|obj.SetSize (int  a[2])|

\item  \verb|int = obj. GetSize ()|

\item  \verb|obj.SetPixelSamples (int , int )|

\item  \verb|obj.SetPixelSamples (int  a[2])|

\item  \verb|int = obj. GetPixelSamples ()|

\item  \verb|obj.SetFilePrefix (string )| -  Specify the prefix of the files to write out. The resulting file names
 will have .RIB appended to them.

\item  \verb|string = obj.GetFilePrefix ()| -  Specify the prefix of the files to write out. The resulting file names
 will have .RIB appended to them.

\item  \verb|obj.SetTexturePrefix (string )| -  Specify the prefix of any generated texture files.

\item  \verb|string = obj.GetTexturePrefix ()| -  Specify the prefix of any generated texture files.

\item  \verb|obj.SetBackground (int )| -  Set/Get the background flag. Default is 0 (off).
 If set, the rib file will contain an
 image shader that will use the renderer window's background
 color. Normally, RenderMan does generate backgrounds. Backgrounds are
 composited into the scene with the tiffcomp program that comes with
 Pixar's RenderMan Toolkit.  In fact, Pixar's Renderman will accept an
 image shader but only sets the alpha of the background. Images created
 this way will still have a black background but contain an alpha of 1
 at all pixels and CANNOT be subsequently composited with other images
 using tiffcomp.  However, other RenderMan compliant renderers like
 Blue Moon Ray Tracing (BMRT) do allow image shaders and properly set
 the background color. If this sounds too confusing, use the following
 rules: If you are using Pixar's Renderman, leave the Background
 off. Otherwise, try setting BackGroundOn and see if you get the
 desired results.

\item  \verb|int = obj.GetBackground ()| -  Set/Get the background flag. Default is 0 (off).
 If set, the rib file will contain an
 image shader that will use the renderer window's background
 color. Normally, RenderMan does generate backgrounds. Backgrounds are
 composited into the scene with the tiffcomp program that comes with
 Pixar's RenderMan Toolkit.  In fact, Pixar's Renderman will accept an
 image shader but only sets the alpha of the background. Images created
 this way will still have a black background but contain an alpha of 1
 at all pixels and CANNOT be subsequently composited with other images
 using tiffcomp.  However, other RenderMan compliant renderers like
 Blue Moon Ray Tracing (BMRT) do allow image shaders and properly set
 the background color. If this sounds too confusing, use the following
 rules: If you are using Pixar's Renderman, leave the Background
 off. Otherwise, try setting BackGroundOn and see if you get the
 desired results.

\item  \verb|obj.BackgroundOn ()| -  Set/Get the background flag. Default is 0 (off).
 If set, the rib file will contain an
 image shader that will use the renderer window's background
 color. Normally, RenderMan does generate backgrounds. Backgrounds are
 composited into the scene with the tiffcomp program that comes with
 Pixar's RenderMan Toolkit.  In fact, Pixar's Renderman will accept an
 image shader but only sets the alpha of the background. Images created
 this way will still have a black background but contain an alpha of 1
 at all pixels and CANNOT be subsequently composited with other images
 using tiffcomp.  However, other RenderMan compliant renderers like
 Blue Moon Ray Tracing (BMRT) do allow image shaders and properly set
 the background color. If this sounds too confusing, use the following
 rules: If you are using Pixar's Renderman, leave the Background
 off. Otherwise, try setting BackGroundOn and see if you get the
 desired results.

\item  \verb|obj.BackgroundOff ()| -  Set/Get the background flag. Default is 0 (off).
 If set, the rib file will contain an
 image shader that will use the renderer window's background
 color. Normally, RenderMan does generate backgrounds. Backgrounds are
 composited into the scene with the tiffcomp program that comes with
 Pixar's RenderMan Toolkit.  In fact, Pixar's Renderman will accept an
 image shader but only sets the alpha of the background. Images created
 this way will still have a black background but contain an alpha of 1
 at all pixels and CANNOT be subsequently composited with other images
 using tiffcomp.  However, other RenderMan compliant renderers like
 Blue Moon Ray Tracing (BMRT) do allow image shaders and properly set
 the background color. If this sounds too confusing, use the following
 rules: If you are using Pixar's Renderman, leave the Background
 off. Otherwise, try setting BackGroundOn and see if you get the
 desired results.

\item  \verb|obj.SetExportArrays (int )| -  Set or get the ExportArrays. If ExportArrays is set, then
 all point data, field data, and cell data arrays will get 
 exported together with polygons.

\item  \verb|int = obj.GetExportArraysMinValue ()| -  Set or get the ExportArrays. If ExportArrays is set, then
 all point data, field data, and cell data arrays will get 
 exported together with polygons.

\item  \verb|int = obj.GetExportArraysMaxValue ()| -  Set or get the ExportArrays. If ExportArrays is set, then
 all point data, field data, and cell data arrays will get 
 exported together with polygons.

\item  \verb|obj.ExportArraysOn ()| -  Set or get the ExportArrays. If ExportArrays is set, then
 all point data, field data, and cell data arrays will get 
 exported together with polygons.

\item  \verb|obj.ExportArraysOff ()| -  Set or get the ExportArrays. If ExportArrays is set, then
 all point data, field data, and cell data arrays will get 
 exported together with polygons.

\item  \verb|int = obj.GetExportArrays ()| -  Set or get the ExportArrays. If ExportArrays is set, then
 all point data, field data, and cell data arrays will get 
 exported together with polygons.

\end{itemize}
