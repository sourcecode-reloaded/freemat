\section{vtkAbstractPolygonalHandleRepresentation3D}

\subsection{Usage}

 This class serves as the geometrical representation of a vtkHandleWidget. 
 The handle can be represented by an arbitrary polygonal data (vtkPolyData),
 set via SetHandle(vtkPolyData *). The actual position of the handle 
 will be initially assumed to be (0,0,0). You can specify an offset from
 this position if desired. This class differs from 
 vtkPolygonalHandleRepresentation3D in that the handle will always remain
 front facing, ie it maintains a fixed orientation with respect to the 
 camera. This is done by using vtkFollowers internally to render the actors.

To create an instance of class vtkAbstractPolygonalHandleRepresentation3D, simply
invoke its constructor as follows
\begin{verbatim}
  obj = vtkAbstractPolygonalHandleRepresentation3D
\end{verbatim}
\subsection{Methods}

The class vtkAbstractPolygonalHandleRepresentation3D has several methods that can be used.
  They are listed below.
Note that the documentation is translated automatically from the VTK sources,
and may not be completely intelligible.  When in doubt, consult the VTK website.
In the methods listed below, \verb|obj| is an instance of the vtkAbstractPolygonalHandleRepresentation3D class.
\begin{itemize}
\item  \verb|string = obj.GetClassName ()| -  Standard methods for instances of this class.

\item  \verb|int = obj.IsA (string name)| -  Standard methods for instances of this class.

\item  \verb|vtkAbstractPolygonalHandleRepresentation3D = obj.NewInstance ()| -  Standard methods for instances of this class.

\item  \verb|vtkAbstractPolygonalHandleRepresentation3D = obj.SafeDownCast (vtkObject o)| -  Standard methods for instances of this class.

\item  \verb|obj.SetWorldPosition (double p[3])| -  Set the position of the point in world and display coordinates.

\item  \verb|obj.SetDisplayPosition (double p[3])| -  Set the position of the point in world and display coordinates.

\item  \verb|obj.SetHandle (vtkPolyData )| -  Set/get the handle polydata.

\item  \verb|vtkPolyData = obj.GetHandle ()| -  Set/get the handle polydata.

\item  \verb|obj.SetProperty (vtkProperty )| -  Set/Get the handle properties when unselected and selected.

\item  \verb|obj.SetSelectedProperty (vtkProperty )| -  Set/Get the handle properties when unselected and selected.

\item  \verb|vtkProperty = obj.GetProperty ()| -  Set/Get the handle properties when unselected and selected.

\item  \verb|vtkProperty = obj.GetSelectedProperty ()| -  Set/Get the handle properties when unselected and selected.

\item  \verb|vtkAbstractTransform = obj.GetTransform ()| -  Get the transform used to transform the generic handle polydata before
 placing it in the render window

\item  \verb|obj.BuildRepresentation ()| -  Methods to make this class properly act like a vtkWidgetRepresentation.

\item  \verb|obj.StartWidgetInteraction (double eventPos[2])| -  Methods to make this class properly act like a vtkWidgetRepresentation.

\item  \verb|obj.WidgetInteraction (double eventPos[2])| -  Methods to make this class properly act like a vtkWidgetRepresentation.

\item  \verb|int = obj.ComputeInteractionState (int X, int Y, int modify)| -  Methods to make this class properly act like a vtkWidgetRepresentation.

\item  \verb|obj.ShallowCopy (vtkProp prop)| -  Methods to make this class behave as a vtkProp.

\item  \verb|obj.DeepCopy (vtkProp prop)| -  Methods to make this class behave as a vtkProp.

\item  \verb|obj.GetActors (vtkPropCollection )| -  Methods to make this class behave as a vtkProp.

\item  \verb|obj.ReleaseGraphicsResources (vtkWindow )| -  Methods to make this class behave as a vtkProp.

\item  \verb|int = obj.RenderOpaqueGeometry (vtkViewport viewport)| -  Methods to make this class behave as a vtkProp.

\item  \verb|int = obj.RenderTranslucentPolygonalGeometry (vtkViewport viewport)| -  Methods to make this class behave as a vtkProp.

\item  \verb|int = obj.HasTranslucentPolygonalGeometry ()| -  Methods to make this class behave as a vtkProp.

\item  \verb|obj.SetLabelVisibility (int )| -  A label may be associated with the seed. The string can be set via
 SetLabelText. The visibility of the label can be turned on / off.

\item  \verb|int = obj.GetLabelVisibility ()| -  A label may be associated with the seed. The string can be set via
 SetLabelText. The visibility of the label can be turned on / off.

\item  \verb|obj.LabelVisibilityOn ()| -  A label may be associated with the seed. The string can be set via
 SetLabelText. The visibility of the label can be turned on / off.

\item  \verb|obj.LabelVisibilityOff ()| -  A label may be associated with the seed. The string can be set via
 SetLabelText. The visibility of the label can be turned on / off.

\item  \verb|obj.SetLabelText (string label)| -  A label may be associated with the seed. The string can be set via
 SetLabelText. The visibility of the label can be turned on / off.

\item  \verb|string = obj.GetLabelText ()| -  A label may be associated with the seed. The string can be set via
 SetLabelText. The visibility of the label can be turned on / off.

\item  \verb|obj.SetLabelTextScale (double scale[3])| -  Scale text (font size along each dimension).

\item  \verb|vtkFollower = obj.GetLabelTextActor ()| -  Get the label text actor

\item  \verb|obj.SetUniformScale (double scale)| -  The handle may be scaled uniformly in all three dimensions using this 
 API. The handle can also be scaled interactively using the right 
 mouse button.

\item  \verb|obj.SetHandleVisibility (int )| -  Toogle the visibility of the handle on and off

\item  \verb|int = obj.GetHandleVisibility ()| -  Toogle the visibility of the handle on and off

\item  \verb|obj.HandleVisibilityOn ()| -  Toogle the visibility of the handle on and off

\item  \verb|obj.HandleVisibilityOff ()| -  Toogle the visibility of the handle on and off

\end{itemize}
