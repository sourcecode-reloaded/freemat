\section{vtkImageAppendComponents}

\subsection{Usage}

 vtkImageAppendComponents takes the components from two inputs and merges
 them into one output. If Input1 has M components, and Input2 has N 
 components, the output will have M+N components with input1
 components coming first.

To create an instance of class vtkImageAppendComponents, simply
invoke its constructor as follows
\begin{verbatim}
  obj = vtkImageAppendComponents
\end{verbatim}
\subsection{Methods}

The class vtkImageAppendComponents has several methods that can be used.
  They are listed below.
Note that the documentation is translated automatically from the VTK sources,
and may not be completely intelligible.  When in doubt, consult the VTK website.
In the methods listed below, \verb|obj| is an instance of the vtkImageAppendComponents class.
\begin{itemize}
\item  \verb|string = obj.GetClassName ()|

\item  \verb|int = obj.IsA (string name)|

\item  \verb|vtkImageAppendComponents = obj.NewInstance ()|

\item  \verb|vtkImageAppendComponents = obj.SafeDownCast (vtkObject o)|

\item  \verb|obj.ReplaceNthInputConnection (int idx, vtkAlgorithmOutput input)| -  Replace one of the input connections with a new input.  You can
 only replace input connections that you previously created with
 AddInputConnection() or, in the case of the first input,
 with SetInputConnection().

\item  \verb|obj.SetInput (int num, vtkDataObject input)| -  Set an Input of this filter.  This method is only for support of
 old-style pipeline connections.  When writing new code you should
 use SetInputConnection(), AddInputConnection(), and
 ReplaceNthInputConnection() instead.

\item  \verb|obj.SetInput (vtkDataObject input)| -  Set an Input of this filter.  This method is only for support of
 old-style pipeline connections.  When writing new code you should
 use SetInputConnection(), AddInputConnection(), and
 ReplaceNthInputConnection() instead.

\item  \verb|vtkDataObject = obj.GetInput (int num)| -  Get one input to this filter. This method is only for support of
 old-style pipeline connections.  When writing new code you should
 use vtkAlgorithm::GetInputConnection(0, num).

\item  \verb|vtkDataObject = obj.GetInput ()| -  Get one input to this filter. This method is only for support of
 old-style pipeline connections.  When writing new code you should
 use vtkAlgorithm::GetInputConnection(0, num).

\item  \verb|int = obj.GetNumberOfInputs ()| -  Get the number of inputs to this filter. This method is only for
 support of old-style pipeline connections.  When writing new code
 you should use vtkAlgorithm::GetNumberOfInputConnections(0).

\end{itemize}
