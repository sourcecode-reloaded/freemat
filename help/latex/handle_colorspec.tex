\section{COLORSPEC Color Property Description}

\subsection{Usage}

There are a number of ways of specifying a color value for
a color-based property.  Examples include line colors, 
marker colors, and the like.  One option is to specify
color as an RGB triplet
\begin{verbatim}
   set(h,'color',[r,g,b])
\end{verbatim}
where \verb|r,g,b| are between @[0,1]@.  Alternately, you can
use color names to specify a color.
\begin{itemize}
\item  \verb|'none'| - No color.

\item  \verb|'y','yellow'| - The color @[1,1,0]@ in RGB space.

\item  \verb|'m','magenta'| - The color @[1,0,1]@ in RGB space.

\item  \verb|'c','cyan'| - The color @[0,1,1]@ in RGB space.

\item  \verb|'r','red'| - The color @[1,0,0]@ in RGB space.

\item  \verb|'g','green'| - The color @[0,1,0]@ in RGB space.

\item  \verb|'b','blue'| - The color @[0,0,1]@ in RGB space.

\item  \verb|'w','white'| - The color @[1,1,1]@ in RGB space.

\item  \verb|'k','black'| - The color @[0,0,0]@ in RGB space.

\end{itemize}
