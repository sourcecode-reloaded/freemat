\section{vtkVoxelContoursToSurfaceFilter}

\subsection{Usage}

 vtkVoxelContoursToSurfaceFilter is a filter that takes contours and
 produces surfaces. There are some restrictions for the contours:

   - The contours are input as vtkPolyData, with the contours being
     polys in the vtkPolyData.
   - The contours lie on XY planes - each contour has a constant Z
   - The contours are ordered in the polys of the vtkPolyData such 
     that all contours on the first (lowest) XY plane are first, then
     continuing in order of increasing Z value. 
   - The X, Y and Z coordinates are all integer values.
   - The desired sampling of the contour data is 1x1x1 - Aspect can
     be used to control the aspect ratio in the output polygonal
     dataset.

 This filter takes the contours and produces a structured points
 dataset of signed floating point number indicating distance from
 a contour. A contouring filter is then applied to generate 3D
 surfaces from a stack of 2D contour distance slices. This is 
 done in a streaming fashion so as not to use to much memory.

To create an instance of class vtkVoxelContoursToSurfaceFilter, simply
invoke its constructor as follows
\begin{verbatim}
  obj = vtkVoxelContoursToSurfaceFilter
\end{verbatim}
\subsection{Methods}

The class vtkVoxelContoursToSurfaceFilter has several methods that can be used.
  They are listed below.
Note that the documentation is translated automatically from the VTK sources,
and may not be completely intelligible.  When in doubt, consult the VTK website.
In the methods listed below, \verb|obj| is an instance of the vtkVoxelContoursToSurfaceFilter class.
\begin{itemize}
\item  \verb|string = obj.GetClassName ()|

\item  \verb|int = obj.IsA (string name)|

\item  \verb|vtkVoxelContoursToSurfaceFilter = obj.NewInstance ()|

\item  \verb|vtkVoxelContoursToSurfaceFilter = obj.SafeDownCast (vtkObject o)|

\item  \verb|obj.SetMemoryLimitInBytes (int )| -  Set / Get the memory limit in bytes for this filter. This is the limit
 of the size of the structured points data set that is created for
 intermediate processing. The data will be streamed through this volume
 in as many pieces as necessary.

\item  \verb|int = obj.GetMemoryLimitInBytes ()| -  Set / Get the memory limit in bytes for this filter. This is the limit
 of the size of the structured points data set that is created for
 intermediate processing. The data will be streamed through this volume
 in as many pieces as necessary.

\item  \verb|obj.SetSpacing (double , double , double )|

\item  \verb|obj.SetSpacing (double  a[3])|

\item  \verb|double = obj. GetSpacing ()|

\end{itemize}
