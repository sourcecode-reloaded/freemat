\section{vtkMergePoints}

\subsection{Usage}

 vtkMergePoints is a locator object to quickly locate points in 3D.
 The primary difference between vtkMergePoints and its superclass
 vtkPointLocator is that vtkMergePoints merges precisely coincident points
 and is therefore much faster.

To create an instance of class vtkMergePoints, simply
invoke its constructor as follows
\begin{verbatim}
  obj = vtkMergePoints
\end{verbatim}
\subsection{Methods}

The class vtkMergePoints has several methods that can be used.
  They are listed below.
Note that the documentation is translated automatically from the VTK sources,
and may not be completely intelligible.  When in doubt, consult the VTK website.
In the methods listed below, \verb|obj| is an instance of the vtkMergePoints class.
\begin{itemize}
\item  \verb|string = obj.GetClassName ()|

\item  \verb|int = obj.IsA (string name)|

\item  \verb|vtkMergePoints = obj.NewInstance ()|

\item  \verb|vtkMergePoints = obj.SafeDownCast (vtkObject o)|

\item  \verb|vtkIdType = obj.IsInsertedPoint (double x[3])| -  Determine whether point given by x[3] has been inserted into points list.
 Return id of previously inserted point if this is true, otherwise return
 -1.

\item  \verb|vtkIdType = obj.IsInsertedPoint (double x, double y, double z)| -  Determine whether point given by x[3] has been inserted into points list.
 Return id of previously inserted point if this is true, otherwise return
 -1.

\end{itemize}
