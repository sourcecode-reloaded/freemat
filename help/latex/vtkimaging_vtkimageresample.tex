\section{vtkImageResample}

\subsection{Usage}

 This filter produces an output with different spacing (and extent)
 than the input.  Linear interpolation can be used to resample the data.
 The Output spacing can be set explicitly or relative to input spacing
 with the SetAxisMagnificationFactor method.

To create an instance of class vtkImageResample, simply
invoke its constructor as follows
\begin{verbatim}
  obj = vtkImageResample
\end{verbatim}
\subsection{Methods}

The class vtkImageResample has several methods that can be used.
  They are listed below.
Note that the documentation is translated automatically from the VTK sources,
and may not be completely intelligible.  When in doubt, consult the VTK website.
In the methods listed below, \verb|obj| is an instance of the vtkImageResample class.
\begin{itemize}
\item  \verb|string = obj.GetClassName ()|

\item  \verb|int = obj.IsA (string name)|

\item  \verb|vtkImageResample = obj.NewInstance ()|

\item  \verb|vtkImageResample = obj.SafeDownCast (vtkObject o)|

\item  \verb|obj.SetAxisOutputSpacing (int axis, double spacing)| -  Set desired spacing.  
 Zero is a reserved value indicating spacing has not been set.

\item  \verb|obj.SetAxisMagnificationFactor (int axis, double factor)| -  Set/Get Magnification factors.
 Zero is a reserved value indicating values have not been computed.

\item  \verb|double = obj.GetAxisMagnificationFactor (int axis, vtkInformation inInfo)| -  Set/Get Magnification factors.
 Zero is a reserved value indicating values have not been computed.

\item  \verb|obj.SetDimensionality (int )| -  Dimensionality is the number of axes which are considered during
 execution. To process images dimensionality would be set to 2.
 This has the same effect as setting the magnification of the third
 axis to 1.0

\item  \verb|int = obj.GetDimensionality ()| -  Dimensionality is the number of axes which are considered during
 execution. To process images dimensionality would be set to 2.
 This has the same effect as setting the magnification of the third
 axis to 1.0

\end{itemize}
