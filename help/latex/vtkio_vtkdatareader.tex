\section{vtkDataReader}

\subsection{Usage}

 vtkDataReader is a helper superclass that reads the vtk data file header,
 dataset type, and attribute data (point and cell attributes such as
 scalars, vectors, normals, etc.) from a vtk data file.  See text for
 the format of the various vtk file types.


To create an instance of class vtkDataReader, simply
invoke its constructor as follows
\begin{verbatim}
  obj = vtkDataReader
\end{verbatim}
\subsection{Methods}

The class vtkDataReader has several methods that can be used.
  They are listed below.
Note that the documentation is translated automatically from the VTK sources,
and may not be completely intelligible.  When in doubt, consult the VTK website.
In the methods listed below, \verb|obj| is an instance of the vtkDataReader class.
\begin{itemize}
\item  \verb|string = obj.GetClassName ()|

\item  \verb|int = obj.IsA (string name)|

\item  \verb|vtkDataReader = obj.NewInstance ()|

\item  \verb|vtkDataReader = obj.SafeDownCast (vtkObject o)|

\item  \verb|obj.SetFileName (string )| -  Specify file name of vtk data file to read.

\item  \verb|string = obj.GetFileName ()| -  Specify file name of vtk data file to read.

\item  \verb|int = obj.IsFileValid (string dstype)| -  Is the file a valid vtk file of the passed dataset type ?
 The dataset type is passed as a lower case string.

\item  \verb|int = obj.IsFileStructuredPoints ()| -  Is the file a valid vtk file of the passed dataset type ?
 The dataset type is passed as a lower case string.

\item  \verb|int = obj.IsFilePolyData ()| -  Is the file a valid vtk file of the passed dataset type ?
 The dataset type is passed as a lower case string.

\item  \verb|int = obj.IsFileStructuredGrid ()| -  Is the file a valid vtk file of the passed dataset type ?
 The dataset type is passed as a lower case string.

\item  \verb|int = obj.IsFileUnstructuredGrid ()| -  Is the file a valid vtk file of the passed dataset type ?
 The dataset type is passed as a lower case string.

\item  \verb|int = obj.IsFileRectilinearGrid ()| -  Is the file a valid vtk file of the passed dataset type ?
 The dataset type is passed as a lower case string.

\item  \verb|obj.SetInputString (string in)| -  Specify the InputString for use when reading from a character array.
 Optionally include the length for binary strings. Note that a copy
 of the string is made and stored. If this causes exceedingly large
 memory consumption, consider using InputArray instead.

\item  \verb|string = obj.GetInputString ()| -  Specify the InputString for use when reading from a character array.
 Optionally include the length for binary strings. Note that a copy
 of the string is made and stored. If this causes exceedingly large
 memory consumption, consider using InputArray instead.

\item  \verb|obj.SetInputString (string in, int len)| -  Specify the InputString for use when reading from a character array.
 Optionally include the length for binary strings. Note that a copy
 of the string is made and stored. If this causes exceedingly large
 memory consumption, consider using InputArray instead.

\item  \verb|int = obj.GetInputStringLength ()| -  Specify the InputString for use when reading from a character array.
 Optionally include the length for binary strings. Note that a copy
 of the string is made and stored. If this causes exceedingly large
 memory consumption, consider using InputArray instead.

\item  \verb|obj.SetBinaryInputString (string , int len)| -  Specify the InputString for use when reading from a character array.
 Optionally include the length for binary strings. Note that a copy
 of the string is made and stored. If this causes exceedingly large
 memory consumption, consider using InputArray instead.

\item  \verb|obj.SetInputArray (vtkCharArray )| -  Specify the vtkCharArray to be used  when reading from a string.
 If set, this array has precendence over InputString.
 Use this instead of InputString to avoid the extra memory copy.
 It should be noted that if the underlying char* is owned by the
 user ( vtkCharArray::SetArray(array, 1); ) and is deleted before
 the reader, bad things will happen during a pipeline update.

\item  \verb|vtkCharArray = obj.GetInputArray ()| -  Specify the vtkCharArray to be used  when reading from a string.
 If set, this array has precendence over InputString.
 Use this instead of InputString to avoid the extra memory copy.
 It should be noted that if the underlying char* is owned by the
 user ( vtkCharArray::SetArray(array, 1); ) and is deleted before
 the reader, bad things will happen during a pipeline update.

\item  \verb|string = obj.GetHeader ()| -  Get the header from the vtk data file.

\item  \verb|obj.SetReadFromInputString (int )| -  Enable reading from an InputString or InputArray instead of the default, 
 a file.

\item  \verb|int = obj.GetReadFromInputString ()| -  Enable reading from an InputString or InputArray instead of the default, 
 a file.

\item  \verb|obj.ReadFromInputStringOn ()| -  Enable reading from an InputString or InputArray instead of the default, 
 a file.

\item  \verb|obj.ReadFromInputStringOff ()| -  Enable reading from an InputString or InputArray instead of the default, 
 a file.

\item  \verb|int = obj.GetFileType ()| -  Get the type of file (ASCII or BINARY). Returned value only valid
 after file has been read.

\item  \verb|int = obj.GetNumberOfScalarsInFile ()| -  How many attributes of various types are in this file? This 
 requires reading the file, so the filename must be set prior 
 to invoking this operation. (Note: file characteristics are
 cached, so only a single read is necessary to return file
 characteristics.)

\item  \verb|int = obj.GetNumberOfVectorsInFile ()| -  How many attributes of various types are in this file? This 
 requires reading the file, so the filename must be set prior 
 to invoking this operation. (Note: file characteristics are
 cached, so only a single read is necessary to return file
 characteristics.)

\item  \verb|int = obj.GetNumberOfTensorsInFile ()| -  How many attributes of various types are in this file? This 
 requires reading the file, so the filename must be set prior 
 to invoking this operation. (Note: file characteristics are
 cached, so only a single read is necessary to return file
 characteristics.)

\item  \verb|int = obj.GetNumberOfNormalsInFile ()| -  How many attributes of various types are in this file? This 
 requires reading the file, so the filename must be set prior 
 to invoking this operation. (Note: file characteristics are
 cached, so only a single read is necessary to return file
 characteristics.)

\item  \verb|int = obj.GetNumberOfTCoordsInFile ()| -  How many attributes of various types are in this file? This 
 requires reading the file, so the filename must be set prior 
 to invoking this operation. (Note: file characteristics are
 cached, so only a single read is necessary to return file
 characteristics.)

\item  \verb|int = obj.GetNumberOfFieldDataInFile ()| -  What is the name of the ith attribute of a certain type
 in this file? This requires reading the file, so the filename 
 must be set prior to invoking this operation.

\item  \verb|string = obj.GetScalarsNameInFile (int i)| -  What is the name of the ith attribute of a certain type
 in this file? This requires reading the file, so the filename 
 must be set prior to invoking this operation.

\item  \verb|string = obj.GetVectorsNameInFile (int i)| -  What is the name of the ith attribute of a certain type
 in this file? This requires reading the file, so the filename 
 must be set prior to invoking this operation.

\item  \verb|string = obj.GetTensorsNameInFile (int i)| -  What is the name of the ith attribute of a certain type
 in this file? This requires reading the file, so the filename 
 must be set prior to invoking this operation.

\item  \verb|string = obj.GetNormalsNameInFile (int i)| -  What is the name of the ith attribute of a certain type
 in this file? This requires reading the file, so the filename 
 must be set prior to invoking this operation.

\item  \verb|string = obj.GetTCoordsNameInFile (int i)| -  What is the name of the ith attribute of a certain type
 in this file? This requires reading the file, so the filename 
 must be set prior to invoking this operation.

\item  \verb|string = obj.GetFieldDataNameInFile (int i)| -  What is the name of the ith attribute of a certain type
 in this file? This requires reading the file, so the filename 
 must be set prior to invoking this operation.

\item  \verb|obj.SetScalarsName (string )| -  Set the name of the scalar data to extract. If not specified, first 
 scalar data encountered is extracted.

\item  \verb|string = obj.GetScalarsName ()| -  Set the name of the scalar data to extract. If not specified, first 
 scalar data encountered is extracted.

\item  \verb|obj.SetVectorsName (string )| -  Set the name of the vector data to extract. If not specified, first 
 vector data encountered is extracted.

\item  \verb|string = obj.GetVectorsName ()| -  Set the name of the vector data to extract. If not specified, first 
 vector data encountered is extracted.

\item  \verb|obj.SetTensorsName (string )| -  Set the name of the tensor data to extract. If not specified, first 
 tensor data encountered is extracted.

\item  \verb|string = obj.GetTensorsName ()| -  Set the name of the tensor data to extract. If not specified, first 
 tensor data encountered is extracted.

\item  \verb|obj.SetNormalsName (string )| -  Set the name of the normal data to extract. If not specified, first 
 normal data encountered is extracted.

\item  \verb|string = obj.GetNormalsName ()| -  Set the name of the normal data to extract. If not specified, first 
 normal data encountered is extracted.

\item  \verb|obj.SetTCoordsName (string )| -  Set the name of the texture coordinate data to extract. If not specified,
 first texture coordinate data encountered is extracted.

\item  \verb|string = obj.GetTCoordsName ()| -  Set the name of the texture coordinate data to extract. If not specified,
 first texture coordinate data encountered is extracted.

\item  \verb|obj.SetLookupTableName (string )| -  Set the name of the lookup table data to extract. If not specified, uses 
 lookup table named by scalar. Otherwise, this specification supersedes.

\item  \verb|string = obj.GetLookupTableName ()| -  Set the name of the lookup table data to extract. If not specified, uses 
 lookup table named by scalar. Otherwise, this specification supersedes.

\item  \verb|obj.SetFieldDataName (string )| -  Set the name of the field data to extract. If not specified, uses 
 first field data encountered in file.

\item  \verb|string = obj.GetFieldDataName ()| -  Set the name of the field data to extract. If not specified, uses 
 first field data encountered in file.

\item  \verb|obj.SetReadAllScalars (int )| -  Enable reading all scalars. 

\item  \verb|int = obj.GetReadAllScalars ()| -  Enable reading all scalars. 

\item  \verb|obj.ReadAllScalarsOn ()| -  Enable reading all scalars. 

\item  \verb|obj.ReadAllScalarsOff ()| -  Enable reading all scalars. 

\item  \verb|obj.SetReadAllVectors (int )| -  Enable reading all vectors. 

\item  \verb|int = obj.GetReadAllVectors ()| -  Enable reading all vectors. 

\item  \verb|obj.ReadAllVectorsOn ()| -  Enable reading all vectors. 

\item  \verb|obj.ReadAllVectorsOff ()| -  Enable reading all vectors. 

\item  \verb|obj.SetReadAllNormals (int )| -  Enable reading all normals. 

\item  \verb|int = obj.GetReadAllNormals ()| -  Enable reading all normals. 

\item  \verb|obj.ReadAllNormalsOn ()| -  Enable reading all normals. 

\item  \verb|obj.ReadAllNormalsOff ()| -  Enable reading all normals. 

\item  \verb|obj.SetReadAllTensors (int )| -  Enable reading all tensors. 

\item  \verb|int = obj.GetReadAllTensors ()| -  Enable reading all tensors. 

\item  \verb|obj.ReadAllTensorsOn ()| -  Enable reading all tensors. 

\item  \verb|obj.ReadAllTensorsOff ()| -  Enable reading all tensors. 

\item  \verb|obj.SetReadAllColorScalars (int )| -  Enable reading all color scalars. 

\item  \verb|int = obj.GetReadAllColorScalars ()| -  Enable reading all color scalars. 

\item  \verb|obj.ReadAllColorScalarsOn ()| -  Enable reading all color scalars. 

\item  \verb|obj.ReadAllColorScalarsOff ()| -  Enable reading all color scalars. 

\item  \verb|obj.SetReadAllTCoords (int )| -  Enable reading all tcoords. 

\item  \verb|int = obj.GetReadAllTCoords ()| -  Enable reading all tcoords. 

\item  \verb|obj.ReadAllTCoordsOn ()| -  Enable reading all tcoords. 

\item  \verb|obj.ReadAllTCoordsOff ()| -  Enable reading all tcoords. 

\item  \verb|obj.SetReadAllFields (int )| -  Enable reading all fields. 

\item  \verb|int = obj.GetReadAllFields ()| -  Enable reading all fields. 

\item  \verb|obj.ReadAllFieldsOn ()| -  Enable reading all fields. 

\item  \verb|obj.ReadAllFieldsOff ()| -  Enable reading all fields. 

\item  \verb|int = obj.OpenVTKFile ()| -  Open a vtk data file. Returns zero if error.

\item  \verb|int = obj.ReadHeader ()| -  Read the header of a vtk data file. Returns 0 if error.

\item  \verb|int = obj.ReadCellData (vtkDataSet ds, int numCells)| -  Read the cell data of a vtk data file. The number of cells (from the 
 dataset) must match the number of cells defined in cell attributes (unless
 no geometry was defined).

\item  \verb|int = obj.ReadPointData (vtkDataSet ds, int numPts)| -  Read the point data of a vtk data file. The number of points (from the
 dataset) must match the number of points defined in point attributes
 (unless no geometry was defined).

\item  \verb|int = obj.ReadPoints (vtkPointSet ps, int numPts)| -  Read point coordinates. Return 0 if error.

\item  \verb|int = obj.ReadPoints (vtkGraph g, int numPts)| -  Read point coordinates. Return 0 if error.

\item  \verb|int = obj.ReadVertexData (vtkGraph g, int numVertices)| -  Read the vertex data of a vtk data file. The number of vertices (from the
 graph) must match the number of vertices defined in vertex attributes
 (unless no geometry was defined).

\item  \verb|int = obj.ReadEdgeData (vtkGraph g, int numEdges)| -  Read the edge data of a vtk data file. The number of edges (from the
 graph) must match the number of edges defined in edge attributes
 (unless no geometry was defined).

\item  \verb|int = obj.ReadRowData (vtkTable t, int numEdges)| -  Read the row data of a vtk data file.

\item  \verb|int = obj.ReadCells (int size, int data)| -  Read a bunch of ''cells''. Return 0 if error.

\item  \verb|int = obj.ReadCells (int size, int data, int skip1, int read2, int skip3)| -  Read a piece of the cells (for streaming compliance)

\item  \verb|int = obj.ReadCoordinates (vtkRectilinearGrid rg, int axes, int numCoords)| -  Read the coordinates for a rectilinear grid. The axes parameter specifies
 which coordinate axes (0,1,2) is being read.

\item  \verb|vtkAbstractArray = obj.ReadArray (string dataType, int numTuples, int numComp)| -  Helper functions for reading data.

\item  \verb|vtkFieldData = obj.ReadFieldData ()| -  Helper functions for reading data.

\item  \verb|obj.CloseVTKFile ()| -  Close the vtk file.

\item  \verb|int = obj.ReadMetaData (vtkInformation )|

\end{itemize}
