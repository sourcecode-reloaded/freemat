\section{COMPARISONOPS Array Comparison Operators}

\subsection{Usage}

There are a total of six comparison operators available in FreeMat, all of which are binary operators with the following syntax
\begin{verbatim}
  y = a < b
  y = a <= b
  y = a > b
  y = a >= b
  y = a ~= b
  y = a == b
\end{verbatim}
where \verb|a| and \verb|b| are numerical arrays or scalars, and \verb|y| is a \verb|logical| array of the appropriate size.  Each of the operators has three modes of operation, summarized in the following list:
\begin{enumerate}
\item  \verb|a| is a scalar, \verb|b| is an n-dimensional array - the output is then the same size as \verb|b|, and contains the result of comparing each element in \verb|b| to the scalar \verb|a|.

\item  \verb|a| is an n-dimensional array, \verb|b| is a scalar - the output is the same size as \verb|a|, and contains the result of comparing each element in \verb|a| to the scalar \verb|b|.

\item  \verb|a| and \verb|b| are both n-dimensional arrays of the same size - the output is then the same size as both \verb|a| and \verb|b|, and contains the result of an element-wise comparison between \verb|a| and \verb|b|.

\end{enumerate}
The operators behave the same way as in \verb|C|, with unequal types being promoted using the standard type promotion rules prior to comparisons.  The only difference is that in FreeMat, the not-equals operator is \verb|~=| instead of \verb|!=|.
\subsection{Examples}

Some simple examples of comparison operations.  First a comparison with a scalar:
\begin{verbatim}
--> a = randn(1,5)

a = 
   -0.1760    0.0212    0.0095    2.0556   -1.1627 

--> a>0

ans = 
 0 1 1 1 0 
\end{verbatim}
Next, we construct two vectors, and test for equality:
\begin{verbatim}
--> a = [1,2,5,7,3]

a = 
 1 2 5 7 3 

--> b = [2,2,5,9,4]

b = 
 2 2 5 9 4 

--> c = a == b

c = 
 0 1 1 0 0 
\end{verbatim}
