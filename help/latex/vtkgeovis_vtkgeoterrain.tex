\section{vtkGeoTerrain}

\subsection{Usage}

 vtkGeoTerrain contains a multi-resolution tree of geometry representing
 the globe. It uses a vtkGeoSource subclass to generate the terrain, such
 as vtkGeoGlobeSource. This source must be set before using the terrain in
 a vtkGeoView. The terrain also contains an AddActors() method which
 will update the set of actors representing the globe given the current
 camera position.

To create an instance of class vtkGeoTerrain, simply
invoke its constructor as follows
\begin{verbatim}
  obj = vtkGeoTerrain
\end{verbatim}
\subsection{Methods}

The class vtkGeoTerrain has several methods that can be used.
  They are listed below.
Note that the documentation is translated automatically from the VTK sources,
and may not be completely intelligible.  When in doubt, consult the VTK website.
In the methods listed below, \verb|obj| is an instance of the vtkGeoTerrain class.
\begin{itemize}
\item  \verb|string = obj.GetClassName ()|

\item  \verb|int = obj.IsA (string name)|

\item  \verb|vtkGeoTerrain = obj.NewInstance ()|

\item  \verb|vtkGeoTerrain = obj.SafeDownCast (vtkObject o)|

\item  \verb|vtkGeoSource = obj.GetSource ()| -  The source used to obtain geometry patches.

\item  \verb|obj.SetSource (vtkGeoSource source)| -  The source used to obtain geometry patches.

\item  \verb|obj.SaveDatabase (string path, int depth)| -  Save the set of patches up to a given maximum depth.

\item  \verb|obj.AddActors (vtkRenderer ren, vtkAssembly assembly, vtkCollection imageReps)| -  Update the actors in an assembly used to render the globe.
 ren is the current renderer, and imageReps holds the collection of
 vtkGeoAlignedImageRepresentations that will be blended together to
 form the image on the globe.

\item  \verb|obj.SetOrigin (double , double , double )| -  The world-coordinate origin offset used to eliminate precision errors
 when zoomed in to a particular region of the globe.

\item  \verb|obj.SetOrigin (double  a[3])| -  The world-coordinate origin offset used to eliminate precision errors
 when zoomed in to a particular region of the globe.

\item  \verb|double = obj. GetOrigin ()| -  The world-coordinate origin offset used to eliminate precision errors
 when zoomed in to a particular region of the globe.

\item  \verb|obj.SetMaxLevel (int )| -  The maximum level of the terrain tree.

\item  \verb|int = obj.GetMaxLevelMinValue ()| -  The maximum level of the terrain tree.

\item  \verb|int = obj.GetMaxLevelMaxValue ()| -  The maximum level of the terrain tree.

\item  \verb|int = obj.GetMaxLevel ()| -  The maximum level of the terrain tree.

\end{itemize}
