\section{WHERE Get Information on Program Stack}

\subsection{Usage}

Returns information on the current stack.  The usage is
\begin{verbatim}
   where
\end{verbatim}
The result is a kind of stack trace that indicates the state
of the current call stack, and where you are relative to the
stack.
\subsection{Example}

Suppose we have the following chain of functions.
\begin{verbatim}
    chain1.m
function chain1
  a = 32;
  b = a + 5;
  chain2(b)
\end{verbatim}
\begin{verbatim}
    chain2.m
function chain2(d)
  d = d + 5;
  chain3
\end{verbatim}
\begin{verbatim}
    chain3.m
function chain3
  g = 54;
  f = g + 1;
  keyboard
\end{verbatim}
The execution of the \verb|where| command shows the stack trace.
\begin{verbatim}
--> chain1
[chain3,4]--> where
In /home/sbasu/Devel/FreeMat4/help/tmp/chain3.m(chain3) at line 4
    In /home/sbasu/Devel/FreeMat4/help/tmp/chain2.m(chain2) at line 4
    In /home/sbasu/Devel/FreeMat4/help/tmp/chain1.m(chain1) at line 4
    In scratch() at line 2
    In base(base)
    In base()
    In global()
[chain3,4]
\end{verbatim}
