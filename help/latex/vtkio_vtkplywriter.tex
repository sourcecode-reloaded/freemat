\section{vtkPLYWriter}

\subsection{Usage}

 vtkPLYWriter writes polygonal data in Stanford University PLY format
 (see http://graphics.stanford.edu/data/3Dscanrep/). The data can be
 written in either binary (little or big endian) or ASCII representation.
 As for PointData and CellData, vtkPLYWriter cannot handle normals or
 vectors. It only handles RGB PointData and CellData. You need to set the
 name of the array (using SetName for the array and SetArrayName for the
 writer). If the array is not a vtkUnsignedCharArray with 3 components,
 you need to specify a vtkLookupTable to map the scalars to RGB.

To create an instance of class vtkPLYWriter, simply
invoke its constructor as follows
\begin{verbatim}
  obj = vtkPLYWriter
\end{verbatim}
\subsection{Methods}

The class vtkPLYWriter has several methods that can be used.
  They are listed below.
Note that the documentation is translated automatically from the VTK sources,
and may not be completely intelligible.  When in doubt, consult the VTK website.
In the methods listed below, \verb|obj| is an instance of the vtkPLYWriter class.
\begin{itemize}
\item  \verb|string = obj.GetClassName ()|

\item  \verb|int = obj.IsA (string name)|

\item  \verb|vtkPLYWriter = obj.NewInstance ()|

\item  \verb|vtkPLYWriter = obj.SafeDownCast (vtkObject o)|

\item  \verb|obj.SetDataByteOrder (int )| -  If the file type is binary, then the user can specify which
 byte order to use (little versus big endian).

\item  \verb|int = obj.GetDataByteOrderMinValue ()| -  If the file type is binary, then the user can specify which
 byte order to use (little versus big endian).

\item  \verb|int = obj.GetDataByteOrderMaxValue ()| -  If the file type is binary, then the user can specify which
 byte order to use (little versus big endian).

\item  \verb|int = obj.GetDataByteOrder ()| -  If the file type is binary, then the user can specify which
 byte order to use (little versus big endian).

\item  \verb|obj.SetDataByteOrderToBigEndian ()| -  If the file type is binary, then the user can specify which
 byte order to use (little versus big endian).

\item  \verb|obj.SetDataByteOrderToLittleEndian ()| -  These methods enable the user to control how to add color into the PLY
 output file. The default behavior is as follows. The user provides the
 name of an array and a component number. If the type of the array is
 three components, unsigned char, then the data is written as three
 separate ''red'', ''green'' and ''blue'' properties. If the type is not
 unsigned char, and a lookup table is provided, then the array/component
 are mapped through the table to generate three separate ''red'', ''green''
 and ''blue'' properties in the PLY file. The user can also set the
 ColorMode to specify a uniform color for the whole part (on a vertex
 colors, face colors, or both. (Note: vertex colors or cell colors may be
 written, depending on where the named array is found. If points and
 cells have the arrays with the same name, then both colors will be
 written.)

\item  \verb|obj.SetColorMode (int )| -  These methods enable the user to control how to add color into the PLY
 output file. The default behavior is as follows. The user provides the
 name of an array and a component number. If the type of the array is
 three components, unsigned char, then the data is written as three
 separate ''red'', ''green'' and ''blue'' properties. If the type is not
 unsigned char, and a lookup table is provided, then the array/component
 are mapped through the table to generate three separate ''red'', ''green''
 and ''blue'' properties in the PLY file. The user can also set the
 ColorMode to specify a uniform color for the whole part (on a vertex
 colors, face colors, or both. (Note: vertex colors or cell colors may be
 written, depending on where the named array is found. If points and
 cells have the arrays with the same name, then both colors will be
 written.)

\item  \verb|int = obj.GetColorMode ()| -  These methods enable the user to control how to add color into the PLY
 output file. The default behavior is as follows. The user provides the
 name of an array and a component number. If the type of the array is
 three components, unsigned char, then the data is written as three
 separate ''red'', ''green'' and ''blue'' properties. If the type is not
 unsigned char, and a lookup table is provided, then the array/component
 are mapped through the table to generate three separate ''red'', ''green''
 and ''blue'' properties in the PLY file. The user can also set the
 ColorMode to specify a uniform color for the whole part (on a vertex
 colors, face colors, or both. (Note: vertex colors or cell colors may be
 written, depending on where the named array is found. If points and
 cells have the arrays with the same name, then both colors will be
 written.)

\item  \verb|obj.SetColorModeToDefault ()| -  These methods enable the user to control how to add color into the PLY
 output file. The default behavior is as follows. The user provides the
 name of an array and a component number. If the type of the array is
 three components, unsigned char, then the data is written as three
 separate ''red'', ''green'' and ''blue'' properties. If the type is not
 unsigned char, and a lookup table is provided, then the array/component
 are mapped through the table to generate three separate ''red'', ''green''
 and ''blue'' properties in the PLY file. The user can also set the
 ColorMode to specify a uniform color for the whole part (on a vertex
 colors, face colors, or both. (Note: vertex colors or cell colors may be
 written, depending on where the named array is found. If points and
 cells have the arrays with the same name, then both colors will be
 written.)

\item  \verb|obj.SetColorModeToUniformCellColor ()| -  These methods enable the user to control how to add color into the PLY
 output file. The default behavior is as follows. The user provides the
 name of an array and a component number. If the type of the array is
 three components, unsigned char, then the data is written as three
 separate ''red'', ''green'' and ''blue'' properties. If the type is not
 unsigned char, and a lookup table is provided, then the array/component
 are mapped through the table to generate three separate ''red'', ''green''
 and ''blue'' properties in the PLY file. The user can also set the
 ColorMode to specify a uniform color for the whole part (on a vertex
 colors, face colors, or both. (Note: vertex colors or cell colors may be
 written, depending on where the named array is found. If points and
 cells have the arrays with the same name, then both colors will be
 written.)

\item  \verb|obj.SetColorModeToUniformPointColor ()| -  These methods enable the user to control how to add color into the PLY
 output file. The default behavior is as follows. The user provides the
 name of an array and a component number. If the type of the array is
 three components, unsigned char, then the data is written as three
 separate ''red'', ''green'' and ''blue'' properties. If the type is not
 unsigned char, and a lookup table is provided, then the array/component
 are mapped through the table to generate three separate ''red'', ''green''
 and ''blue'' properties in the PLY file. The user can also set the
 ColorMode to specify a uniform color for the whole part (on a vertex
 colors, face colors, or both. (Note: vertex colors or cell colors may be
 written, depending on where the named array is found. If points and
 cells have the arrays with the same name, then both colors will be
 written.)

\item  \verb|obj.SetColorModeToUniformColor ()| -  These methods enable the user to control how to add color into the PLY
 output file. The default behavior is as follows. The user provides the
 name of an array and a component number. If the type of the array is
 three components, unsigned char, then the data is written as three
 separate ''red'', ''green'' and ''blue'' properties. If the type is not
 unsigned char, and a lookup table is provided, then the array/component
 are mapped through the table to generate three separate ''red'', ''green''
 and ''blue'' properties in the PLY file. The user can also set the
 ColorMode to specify a uniform color for the whole part (on a vertex
 colors, face colors, or both. (Note: vertex colors or cell colors may be
 written, depending on where the named array is found. If points and
 cells have the arrays with the same name, then both colors will be
 written.)

\item  \verb|obj.SetColorModeToOff ()| -  Specify the array name to use to color the data.

\item  \verb|obj.SetArrayName (string )| -  Specify the array name to use to color the data.

\item  \verb|string = obj.GetArrayName ()| -  Specify the array name to use to color the data.

\item  \verb|obj.SetComponent (int )| -  Specify the array component to use to color the data.

\item  \verb|int = obj.GetComponentMinValue ()| -  Specify the array component to use to color the data.

\item  \verb|int = obj.GetComponentMaxValue ()| -  Specify the array component to use to color the data.

\item  \verb|int = obj.GetComponent ()| -  Specify the array component to use to color the data.

\item  \verb|obj.SetLookupTable (vtkScalarsToColors )| -  A lookup table can be specified in order to convert data arrays to
 RGBA colors.

\item  \verb|vtkScalarsToColors = obj.GetLookupTable ()| -  A lookup table can be specified in order to convert data arrays to
 RGBA colors.

\item  \verb|obj.SetColor (char , char , char )| -  Set the color to use when using a uniform color (either point or cells,
 or both). The color is specified as a triplet of three unsigned chars
 between (0,255). This only takes effect when the ColorMode is set to
 uniform point, uniform cell, or uniform color.

\item  \verb|obj.SetColor (char  a[3])| -  Set the color to use when using a uniform color (either point or cells,
 or both). The color is specified as a triplet of three unsigned chars
 between (0,255). This only takes effect when the ColorMode is set to
 uniform point, uniform cell, or uniform color.

\item  \verb|char = obj. GetColor ()| -  Set the color to use when using a uniform color (either point or cells,
 or both). The color is specified as a triplet of three unsigned chars
 between (0,255). This only takes effect when the ColorMode is set to
 uniform point, uniform cell, or uniform color.

\end{itemize}
