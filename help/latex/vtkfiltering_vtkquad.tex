\section{vtkQuad}

\subsection{Usage}

 vtkQuad is a concrete implementation of vtkCell to represent a 2D
 quadrilateral. vtkQuad is defined by the four points (0,1,2,3) in
 counterclockwise order. vtkQuad uses the standard isoparametric
 interpolation functions for a linear quadrilateral.

To create an instance of class vtkQuad, simply
invoke its constructor as follows
\begin{verbatim}
  obj = vtkQuad
\end{verbatim}
\subsection{Methods}

The class vtkQuad has several methods that can be used.
  They are listed below.
Note that the documentation is translated automatically from the VTK sources,
and may not be completely intelligible.  When in doubt, consult the VTK website.
In the methods listed below, \verb|obj| is an instance of the vtkQuad class.
\begin{itemize}
\item  \verb|string = obj.GetClassName ()|

\item  \verb|int = obj.IsA (string name)|

\item  \verb|vtkQuad = obj.NewInstance ()|

\item  \verb|vtkQuad = obj.SafeDownCast (vtkObject o)|

\item  \verb|int = obj.GetCellType ()| -  See the vtkCell API for descriptions of these methods.

\item  \verb|int = obj.GetCellDimension ()| -  See the vtkCell API for descriptions of these methods.

\item  \verb|int = obj.GetNumberOfEdges ()| -  See the vtkCell API for descriptions of these methods.

\item  \verb|int = obj.GetNumberOfFaces ()| -  See the vtkCell API for descriptions of these methods.

\item  \verb|vtkCell = obj.GetEdge (int edgeId)| -  See the vtkCell API for descriptions of these methods.

\item  \verb|vtkCell = obj.GetFace (int )| -  See the vtkCell API for descriptions of these methods.

\item  \verb|int = obj.CellBoundary (int subId, double pcoords[3], vtkIdList pts)| -  See the vtkCell API for descriptions of these methods.

\item  \verb|obj.Contour (double value, vtkDataArray cellScalars, vtkIncrementalPointLocator locator, vtkCellArray verts, vtkCellArray lines, vtkCellArray polys, vtkPointData inPd, vtkPointData outPd, vtkCellData inCd, vtkIdType cellId, vtkCellData outCd)| -  See the vtkCell API for descriptions of these methods.

\item  \verb|int = obj.Triangulate (int index, vtkIdList ptIds, vtkPoints pts)| -  See the vtkCell API for descriptions of these methods.

\item  \verb|obj.Derivatives (int subId, double pcoords[3], double values, int dim, double derivs)| -  See the vtkCell API for descriptions of these methods.

\item  \verb|int = obj.GetParametricCenter (double pcoords[3])| -  Return the center of the triangle in parametric coordinates.

\item  \verb|obj.Clip (double value, vtkDataArray cellScalars, vtkIncrementalPointLocator locator, vtkCellArray polys, vtkPointData inPd, vtkPointData outPd, vtkCellData inCd, vtkIdType cellId, vtkCellData outCd, int insideOut)| -  Clip this quad using scalar value provided. Like contouring, except
 that it cuts the quad to produce other quads and/or triangles.

\item  \verb|obj.InterpolateFunctions (double pcoords[3], double sf[4])| -  Compute the interpolation functions/derivatives
 (aka shape functions/derivatives)

\item  \verb|obj.InterpolateDerivs (double pcoords[3], double derivs[8])| -  Return the ids of the vertices defining edge (`edgeId`).
 Ids are related to the cell, not to the dataset.

\end{itemize}
