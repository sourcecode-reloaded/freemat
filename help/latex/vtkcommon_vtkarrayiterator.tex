\section{vtkArrayIterator}

\subsection{Usage}

 vtkArrayIterator is used to iterate over elements in any vtkAbstractArray
 subclass.
 The vtkArrayIteratorTemplateMacro is used to centralize the set of types
 supported by Execute methods.  It also avoids duplication of long
 switch statement case lists.
 Note that in this macro VTK\_TT is defined to be the type of the iterator
 for the given type of array. One must include the 
 vtkArrayIteratorIncludes.h header file to provide for extending of this macro
 by addition of new iterators.

 Example usage:
 \begin{verbatim}
 vtkArrayIter* iter = array->NewIterator();
 switch(array->GetDataType())
   {
   vtkArrayIteratorTemplateMacro(myFunc(static\_cast<VTK\_TT*>(iter), arg2));
   }
 iter->Delete();
 \end{verbatim}

To create an instance of class vtkArrayIterator, simply
invoke its constructor as follows
\begin{verbatim}
  obj = vtkArrayIterator
\end{verbatim}
\subsection{Methods}

The class vtkArrayIterator has several methods that can be used.
  They are listed below.
Note that the documentation is translated automatically from the VTK sources,
and may not be completely intelligible.  When in doubt, consult the VTK website.
In the methods listed below, \verb|obj| is an instance of the vtkArrayIterator class.
\begin{itemize}
\item  \verb|string = obj.GetClassName ()|

\item  \verb|int = obj.IsA (string name)|

\item  \verb|vtkArrayIterator = obj.NewInstance ()|

\item  \verb|vtkArrayIterator = obj.SafeDownCast (vtkObject o)|

\item  \verb|obj.Initialize (vtkAbstractArray array)| -  Set the array this iterator will iterate over.
 After Initialize() has been called, the iterator is valid
 so long as the Array has not been modified 
 (except using the iterator itself).
 If the array is modified, the iterator must be re-intialized.

\item  \verb|int = obj.GetDataType ()|

\end{itemize}
