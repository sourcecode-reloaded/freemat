\section{vtkPolyDataConnectivityFilter}

\subsection{Usage}

 vtkPolyDataConnectivityFilter is a filter that extracts cells that
 share common points and/or satisfy a scalar threshold
 criterion. (Such a group of cells is called a region.) The filter
 works in one of six ways: 1) extract the largest connected region
 in the dataset; 2) extract specified region numbers; 3) extract all
 regions sharing specified point ids; 4) extract all regions sharing
 specified cell ids; 5) extract the region closest to the specified
 point; or 6) extract all regions (used to color regions).

 This filter is specialized for polygonal data. This means it runs a bit 
 faster and is easier to construct visualization networks that process
 polygonal data.

 The behavior of vtkPolyDataConnectivityFilter can be modified by turning
 on the boolean ivar ScalarConnectivity. If this flag is on, the
 connectivity algorithm is modified so that cells are considered connected
 only if 1) they are geometrically connected (share a point) and 2) the
 scalar values of one of the cell's points falls in the scalar range
 specified. This use of ScalarConnectivity is particularly useful for
 selecting cells for later processing.

To create an instance of class vtkPolyDataConnectivityFilter, simply
invoke its constructor as follows
\begin{verbatim}
  obj = vtkPolyDataConnectivityFilter
\end{verbatim}
\subsection{Methods}

The class vtkPolyDataConnectivityFilter has several methods that can be used.
  They are listed below.
Note that the documentation is translated automatically from the VTK sources,
and may not be completely intelligible.  When in doubt, consult the VTK website.
In the methods listed below, \verb|obj| is an instance of the vtkPolyDataConnectivityFilter class.
\begin{itemize}
\item  \verb|string = obj.GetClassName ()|

\item  \verb|int = obj.IsA (string name)|

\item  \verb|vtkPolyDataConnectivityFilter = obj.NewInstance ()|

\item  \verb|vtkPolyDataConnectivityFilter = obj.SafeDownCast (vtkObject o)|

\item  \verb|obj.SetScalarConnectivity (int )| -  Turn on/off connectivity based on scalar value. If on, cells are connected
 only if they share points AND one of the cells scalar values falls in the
 scalar range specified.

\item  \verb|int = obj.GetScalarConnectivity ()| -  Turn on/off connectivity based on scalar value. If on, cells are connected
 only if they share points AND one of the cells scalar values falls in the
 scalar range specified.

\item  \verb|obj.ScalarConnectivityOn ()| -  Turn on/off connectivity based on scalar value. If on, cells are connected
 only if they share points AND one of the cells scalar values falls in the
 scalar range specified.

\item  \verb|obj.ScalarConnectivityOff ()| -  Turn on/off connectivity based on scalar value. If on, cells are connected
 only if they share points AND one of the cells scalar values falls in the
 scalar range specified.

\item  \verb|obj.SetScalarRange (double , double )| -  Set the scalar range to use to extract cells based on scalar connectivity.

\item  \verb|obj.SetScalarRange (double  a[2])| -  Set the scalar range to use to extract cells based on scalar connectivity.

\item  \verb|double = obj. GetScalarRange ()| -  Set the scalar range to use to extract cells based on scalar connectivity.

\item  \verb|obj.SetExtractionMode (int )| -  Control the extraction of connected surfaces.

\item  \verb|int = obj.GetExtractionModeMinValue ()| -  Control the extraction of connected surfaces.

\item  \verb|int = obj.GetExtractionModeMaxValue ()| -  Control the extraction of connected surfaces.

\item  \verb|int = obj.GetExtractionMode ()| -  Control the extraction of connected surfaces.

\item  \verb|obj.SetExtractionModeToPointSeededRegions ()| -  Control the extraction of connected surfaces.

\item  \verb|obj.SetExtractionModeToCellSeededRegions ()| -  Control the extraction of connected surfaces.

\item  \verb|obj.SetExtractionModeToLargestRegion ()| -  Control the extraction of connected surfaces.

\item  \verb|obj.SetExtractionModeToSpecifiedRegions ()| -  Control the extraction of connected surfaces.

\item  \verb|obj.SetExtractionModeToClosestPointRegion ()| -  Control the extraction of connected surfaces.

\item  \verb|obj.SetExtractionModeToAllRegions ()| -  Control the extraction of connected surfaces.

\item  \verb|string = obj.GetExtractionModeAsString ()| -  Control the extraction of connected surfaces.

\item  \verb|obj.InitializeSeedList ()| -  Initialize list of point ids/cell ids used to seed regions.

\item  \verb|obj.AddSeed (int id)| -  Add a seed id (point or cell id). Note: ids are 0-offset.

\item  \verb|obj.DeleteSeed (int id)| -  Delete a seed id (point or cell id). Note: ids are 0-offset.

\item  \verb|obj.InitializeSpecifiedRegionList ()| -  Initialize list of region ids to extract.

\item  \verb|obj.AddSpecifiedRegion (int id)| -  Add a region id to extract. Note: ids are 0-offset.

\item  \verb|obj.DeleteSpecifiedRegion (int id)| -  Delete a region id to extract. Note: ids are 0-offset.

\item  \verb|obj.SetClosestPoint (double , double , double )| -  Use to specify x-y-z point coordinates when extracting the region 
 closest to a specified point.

\item  \verb|obj.SetClosestPoint (double  a[3])| -  Use to specify x-y-z point coordinates when extracting the region 
 closest to a specified point.

\item  \verb|double = obj. GetClosestPoint ()| -  Use to specify x-y-z point coordinates when extracting the region 
 closest to a specified point.

\item  \verb|int = obj.GetNumberOfExtractedRegions ()| -  Obtain the number of connected regions.

\item  \verb|obj.SetColorRegions (int )| -  Turn on/off the coloring of connected regions.

\item  \verb|int = obj.GetColorRegions ()| -  Turn on/off the coloring of connected regions.

\item  \verb|obj.ColorRegionsOn ()| -  Turn on/off the coloring of connected regions.

\item  \verb|obj.ColorRegionsOff ()| -  Turn on/off the coloring of connected regions.

\end{itemize}
