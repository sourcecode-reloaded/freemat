\section{vtkSphericalDirectionEncoder}

\subsection{Usage}

 vtkSphericalDirectionEncoder is a direction encoder which uses spherical
 coordinates for mapping (nx, ny, nz) into an azimuth, elevation pair.


To create an instance of class vtkSphericalDirectionEncoder, simply
invoke its constructor as follows
\begin{verbatim}
  obj = vtkSphericalDirectionEncoder
\end{verbatim}
\subsection{Methods}

The class vtkSphericalDirectionEncoder has several methods that can be used.
  They are listed below.
Note that the documentation is translated automatically from the VTK sources,
and may not be completely intelligible.  When in doubt, consult the VTK website.
In the methods listed below, \verb|obj| is an instance of the vtkSphericalDirectionEncoder class.
\begin{itemize}
\item  \verb|string = obj.GetClassName ()|

\item  \verb|int = obj.IsA (string name)|

\item  \verb|vtkSphericalDirectionEncoder = obj.NewInstance ()|

\item  \verb|vtkSphericalDirectionEncoder = obj.SafeDownCast (vtkObject o)|

\item  \verb|int = obj.GetEncodedDirection (float n[3])| -  Given a normal vector n, return the encoded direction  

\item  \verb|float = obj.GetDecodedGradient (int value)| - / Given an encoded value, return a pointer to the normal vector

\item  \verb|int = obj.GetNumberOfEncodedDirections (void )| -  Get the decoded gradient table. There are 
 this->GetNumberOfEncodedDirections() entries in the table, each
 containing a normal (direction) vector. This is a flat structure - 
 3 times the number of directions floats in an array.

\end{itemize}
