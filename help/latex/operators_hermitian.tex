\section{HERMITIAN Matrix Hermitian (Conjugate Transpose) Operator}

\subsection{Usage}

Computes the Hermitian of the argument (a 2D matrix).  The syntax for its use is
\begin{verbatim}
  y = a';
\end{verbatim}
where \verb|a| is a \verb|M x N| numerical matrix.  The output \verb|y| is a numerical matrix
of the same type of size \verb|N x M|.  This operator is the conjugating transpose,
which is different from the transpose operator \verb|.'| (which does not 
conjugate complex values).
\subsection{Function Internals}

The Hermitian operator is defined simply as
\[
  y_{i,j} = \overline{a_{j,i}}
\]
where \verb|y_ij| is the element in the \verb|i|th row and \verb|j|th column of the output matrix \verb|y|.
\subsection{Examples}

A simple transpose example:
@>
Here, we use a complex matrix to demonstrate how the Hermitian operator conjugates the entries.
@>
