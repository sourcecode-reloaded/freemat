\section{vtkEarthSource}

\subsection{Usage}

 vtkEarthSource creates a spherical rendering of the geographical shapes
 of the major continents of the earth. The OnRatio determines
 how much of the data is actually used. The radius defines the radius
 of the sphere at which the continents are placed. Obtains data from
 an imbedded array of coordinates.

To create an instance of class vtkEarthSource, simply
invoke its constructor as follows
\begin{verbatim}
  obj = vtkEarthSource
\end{verbatim}
\subsection{Methods}

The class vtkEarthSource has several methods that can be used.
  They are listed below.
Note that the documentation is translated automatically from the VTK sources,
and may not be completely intelligible.  When in doubt, consult the VTK website.
In the methods listed below, \verb|obj| is an instance of the vtkEarthSource class.
\begin{itemize}
\item  \verb|string = obj.GetClassName ()|

\item  \verb|int = obj.IsA (string name)|

\item  \verb|vtkEarthSource = obj.NewInstance ()|

\item  \verb|vtkEarthSource = obj.SafeDownCast (vtkObject o)|

\item  \verb|obj.SetRadius (double )| -  Set radius of earth.

\item  \verb|double = obj.GetRadiusMinValue ()| -  Set radius of earth.

\item  \verb|double = obj.GetRadiusMaxValue ()| -  Set radius of earth.

\item  \verb|double = obj.GetRadius ()| -  Set radius of earth.

\item  \verb|obj.SetOnRatio (int )| -  Turn on every nth entity. This controls how much detail the model
 will have. The maximum ratio is sixteen. (The smaller OnRatio, the more
 detail there is.)

\item  \verb|int = obj.GetOnRatioMinValue ()| -  Turn on every nth entity. This controls how much detail the model
 will have. The maximum ratio is sixteen. (The smaller OnRatio, the more
 detail there is.)

\item  \verb|int = obj.GetOnRatioMaxValue ()| -  Turn on every nth entity. This controls how much detail the model
 will have. The maximum ratio is sixteen. (The smaller OnRatio, the more
 detail there is.)

\item  \verb|int = obj.GetOnRatio ()| -  Turn on every nth entity. This controls how much detail the model
 will have. The maximum ratio is sixteen. (The smaller OnRatio, the more
 detail there is.)

\item  \verb|obj.SetOutline (int )| -  Turn on/off drawing continents as filled polygons or as wireframe outlines.
 Warning: some graphics systems will have trouble with the very large, concave 
 filled polygons. Recommend you use OutlienOn (i.e., disable filled polygons) 
 for now.

\item  \verb|int = obj.GetOutline ()| -  Turn on/off drawing continents as filled polygons or as wireframe outlines.
 Warning: some graphics systems will have trouble with the very large, concave 
 filled polygons. Recommend you use OutlienOn (i.e., disable filled polygons) 
 for now.

\item  \verb|obj.OutlineOn ()| -  Turn on/off drawing continents as filled polygons or as wireframe outlines.
 Warning: some graphics systems will have trouble with the very large, concave 
 filled polygons. Recommend you use OutlienOn (i.e., disable filled polygons) 
 for now.

\item  \verb|obj.OutlineOff ()| -  Turn on/off drawing continents as filled polygons or as wireframe outlines.
 Warning: some graphics systems will have trouble with the very large, concave 
 filled polygons. Recommend you use OutlienOn (i.e., disable filled polygons) 
 for now.

\end{itemize}
