\section{POLY Convert Roots To Polynomial Coefficients}

\subsection{Usage}

This function calculates the polynomial coefficients for given roots
\begin{verbatim}
    p = poly(r)
\end{verbatim}
when \verb|r| is a vector, is a vector whose elements are the coefficients 
of the polynomial whose roots are the elements of \verb|r|.  Alternately,
you can provide a matrix
\begin{verbatim}
    p = poly(A)
\end{verbatim}
when \verb|A| is an \verb|N x N| square matrix, is a row vector with 
\verb|N+1| elements which are the coefficients of the
characteristic polynomial, \verb|det(lambda*eye(size(A))-A)|.

Contributed by Paulo Xavier Candeias under GPL.
\subsection{Example}

Here are some examples of the use of \verb|poly|
\begin{verbatim}
--> A = [1,2,3;4,5,6;7,8,0]

A = 
 1 2 3 
 4 5 6 
 7 8 0 

--> p = poly(A)

p = 
    1.0000   -6.0000  -72.0000  -27.0000 

--> r = roots(p)

r = 
   12.1229 
   -5.7345 
   -0.3884 
\end{verbatim}
