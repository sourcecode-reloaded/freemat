\section{vtkLineWidget}

\subsection{Usage}

 This 3D widget defines a line that can be interactively placed in a
 scene. The line has two handles (at its endpoints), plus the line can be
 picked to translate it in the scene.  A nice feature of the object is that
 the vtkLineWidget, like any 3D widget, will work with the current
 interactor style and any other widgets present in the scene. That is, if
 vtkLineWidget does not handle an event, then all other registered
 observers (including the interactor style) have an opportunity to process
 the event. Otherwise, the vtkLineWidget will terminate the processing of
 the event that it handles.

 To use this object, just invoke SetInteractor() with the argument of the
 method a vtkRenderWindowInteractor.  You may also wish to invoke
 ''PlaceWidget()'' to initially position the widget. The interactor will act
 normally until the ''i'' key (for ''interactor'') is pressed, at which point
 the vtkLineWidget will appear. (See superclass documentation for
 information about changing this behavior.) By grabbing one of the two end
 point handles (use the left mouse button), the line can be oriented and
 stretched (the other end point remains fixed). By grabbing the line
 itself, or using the middle mouse button, the entire line can be
 translated.  Scaling (about the center of the line) is achieved by using
 the right mouse button. By moving the mouse ''up'' the render window the
 line will be made bigger; by moving ''down'' the render window the widget
 will be made smaller. Turn off the widget by pressing the ''i'' key again
 (or invoke the Off() method). (Note: picking the line or either one of the
 two end point handles causes a vtkPointWidget to appear.  This widget has
 the ability to constrain motion to an axis by pressing the ''shift'' key
 while moving the mouse.)

 The vtkLineWidget has several methods that can be used in conjunction with
 other VTK objects. The Set/GetResolution() methods control the number of
 subdivisions of the line; the GetPolyData() method can be used to get the
 polygonal representation and can be used for things like seeding
 streamlines. Typical usage of the widget is to make use of the
 StartInteractionEvent, InteractionEvent, and EndInteractionEvent
 events. The InteractionEvent is called on mouse motion; the other two
 events are called on button down and button up (either left or right
 button).

 Some additional features of this class include the ability to control the
 properties of the widget. You can set the properties of the selected and
 unselected representations of the line. For example, you can set the
 property for the handles and line. In addition there are methods to
 constrain the line so that it is aligned along the x-y-z axes.

To create an instance of class vtkLineWidget, simply
invoke its constructor as follows
\begin{verbatim}
  obj = vtkLineWidget
\end{verbatim}
\subsection{Methods}

The class vtkLineWidget has several methods that can be used.
  They are listed below.
Note that the documentation is translated automatically from the VTK sources,
and may not be completely intelligible.  When in doubt, consult the VTK website.
In the methods listed below, \verb|obj| is an instance of the vtkLineWidget class.
\begin{itemize}
\item  \verb|string = obj.GetClassName ()|

\item  \verb|int = obj.IsA (string name)|

\item  \verb|vtkLineWidget = obj.NewInstance ()|

\item  \verb|vtkLineWidget = obj.SafeDownCast (vtkObject o)|

\item  \verb|obj.SetEnabled (int )| -  Methods that satisfy the superclass' API.

\item  \verb|obj.PlaceWidget (double bounds[6])| -  Methods that satisfy the superclass' API.

\item  \verb|obj.PlaceWidget ()| -  Methods that satisfy the superclass' API.

\item  \verb|obj.PlaceWidget (double xmin, double xmax, double ymin, double ymax, double zmin, double zmax)| -  Set/Get the resolution (number of subdivisions) of the line.

\item  \verb|obj.SetResolution (int r)| -  Set/Get the resolution (number of subdivisions) of the line.

\item  \verb|int = obj.GetResolution ()| -  Set/Get the position of first end point.

\item  \verb|obj.SetPoint1 (double x, double y, double z)| -  Set/Get the position of first end point.

\item  \verb|obj.SetPoint1 (double x[3])| -  Set/Get the position of first end point.

\item  \verb|double = obj.GetPoint1 ()| -  Set/Get the position of first end point.

\item  \verb|obj.GetPoint1 (double xyz[3])| -  Set position of other end point.

\item  \verb|obj.SetPoint2 (double x, double y, double z)| -  Set position of other end point.

\item  \verb|obj.SetPoint2 (double x[3])| -  Set position of other end point.

\item  \verb|double = obj.GetPoint2 ()| -  Set position of other end point.

\item  \verb|obj.GetPoint2 (double xyz[3])| -  Force the line widget to be aligned with one of the x-y-z axes.
 Remember that when the state changes, a ModifiedEvent is invoked.
 This can be used to snap the line to the axes if it is orginally
 not aligned.

\item  \verb|obj.SetAlign (int )| -  Force the line widget to be aligned with one of the x-y-z axes.
 Remember that when the state changes, a ModifiedEvent is invoked.
 This can be used to snap the line to the axes if it is orginally
 not aligned.

\item  \verb|int = obj.GetAlignMinValue ()| -  Force the line widget to be aligned with one of the x-y-z axes.
 Remember that when the state changes, a ModifiedEvent is invoked.
 This can be used to snap the line to the axes if it is orginally
 not aligned.

\item  \verb|int = obj.GetAlignMaxValue ()| -  Force the line widget to be aligned with one of the x-y-z axes.
 Remember that when the state changes, a ModifiedEvent is invoked.
 This can be used to snap the line to the axes if it is orginally
 not aligned.

\item  \verb|int = obj.GetAlign ()| -  Force the line widget to be aligned with one of the x-y-z axes.
 Remember that when the state changes, a ModifiedEvent is invoked.
 This can be used to snap the line to the axes if it is orginally
 not aligned.

\item  \verb|obj.SetAlignToXAxis ()| -  Force the line widget to be aligned with one of the x-y-z axes.
 Remember that when the state changes, a ModifiedEvent is invoked.
 This can be used to snap the line to the axes if it is orginally
 not aligned.

\item  \verb|obj.SetAlignToYAxis ()| -  Force the line widget to be aligned with one of the x-y-z axes.
 Remember that when the state changes, a ModifiedEvent is invoked.
 This can be used to snap the line to the axes if it is orginally
 not aligned.

\item  \verb|obj.SetAlignToZAxis ()| -  Force the line widget to be aligned with one of the x-y-z axes.
 Remember that when the state changes, a ModifiedEvent is invoked.
 This can be used to snap the line to the axes if it is orginally
 not aligned.

\item  \verb|obj.SetAlignToNone ()| -  Enable/disable clamping of the point end points to the bounding box
 of the data. The bounding box is defined from the last PlaceWidget()
 invocation, and includes the effect of the PlaceFactor which is used
 to gram/shrink the bounding box.

\item  \verb|obj.SetClampToBounds (int )| -  Enable/disable clamping of the point end points to the bounding box
 of the data. The bounding box is defined from the last PlaceWidget()
 invocation, and includes the effect of the PlaceFactor which is used
 to gram/shrink the bounding box.

\item  \verb|int = obj.GetClampToBounds ()| -  Enable/disable clamping of the point end points to the bounding box
 of the data. The bounding box is defined from the last PlaceWidget()
 invocation, and includes the effect of the PlaceFactor which is used
 to gram/shrink the bounding box.

\item  \verb|obj.ClampToBoundsOn ()| -  Enable/disable clamping of the point end points to the bounding box
 of the data. The bounding box is defined from the last PlaceWidget()
 invocation, and includes the effect of the PlaceFactor which is used
 to gram/shrink the bounding box.

\item  \verb|obj.ClampToBoundsOff ()| -  Enable/disable clamping of the point end points to the bounding box
 of the data. The bounding box is defined from the last PlaceWidget()
 invocation, and includes the effect of the PlaceFactor which is used
 to gram/shrink the bounding box.

\item  \verb|obj.GetPolyData (vtkPolyData pd)| -  Grab the polydata (including points) that defines the line.  The
 polydata consists of n+1 points, where n is the resolution of the
 line. These point values are guaranteed to be up-to-date when either the
 InteractionEvent or EndInteraction events are invoked. The user provides
 the vtkPolyData and the points and polyline are added to it.

\item  \verb|vtkProperty = obj.GetHandleProperty ()| -  Get the handle properties (the little balls are the handles). The
 properties of the handles when selected and normal can be
 manipulated.

\item  \verb|vtkProperty = obj.GetSelectedHandleProperty ()| -  Get the handle properties (the little balls are the handles). The
 properties of the handles when selected and normal can be
 manipulated.

\item  \verb|vtkProperty = obj.GetLineProperty ()| -  Get the line properties. The properties of the line when selected
 and unselected can be manipulated.

\item  \verb|vtkProperty = obj.GetSelectedLineProperty ()| -  Get the line properties. The properties of the line when selected
 and unselected can be manipulated.

\end{itemize}
