\section{vtkBivariateStatisticsAlgorithm}

\subsection{Usage}

 This class specializes statistics algorithms to the bivariate case, where
 a number of pairs of columns of interest can be selected in the input data set.
 This is done by the means of the following functions:

 ResetColumns() - reset the list of columns of interest.
 Add/RemoveColum( namColX, namColY ) - try to add/remove column pair ( namColX,
 namColy ) to/from the list.
 SetColumnStatus ( namCol, status ) - mostly for UI wrapping purposes, try to 
 add/remove (depending on status) namCol from a list of buffered columns, from
 which all possible pairs are generated.
 The verb ''try'' is used in the sense that neither attempting to 
 repeat an existing entry nor to remove a non-existent entry will work.
 
 .SECTION Thanks
 Thanks to Philippe Pebay and David Thompson from Sandia National Laboratories 
 for implementing this class.

To create an instance of class vtkBivariateStatisticsAlgorithm, simply
invoke its constructor as follows
\begin{verbatim}
  obj = vtkBivariateStatisticsAlgorithm
\end{verbatim}
\subsection{Methods}

The class vtkBivariateStatisticsAlgorithm has several methods that can be used.
  They are listed below.
Note that the documentation is translated automatically from the VTK sources,
and may not be completely intelligible.  When in doubt, consult the VTK website.
In the methods listed below, \verb|obj| is an instance of the vtkBivariateStatisticsAlgorithm class.
\begin{itemize}
\item  \verb|string = obj.GetClassName ()|

\item  \verb|int = obj.IsA (string name)|

\item  \verb|vtkBivariateStatisticsAlgorithm = obj.NewInstance ()|

\item  \verb|vtkBivariateStatisticsAlgorithm = obj.SafeDownCast (vtkObject o)|

\item  \verb|obj.AddColumnPair (string namColX, string namColY)| -  Convenience method to create a request with a single column name pair 
  ( namColX,  namColY) in a single call; this is the preferred method to select 
 columns pairs, ensuring selection consistency (a pair of columns per request).

 Unlike SetColumnStatus(), you need not call RequestSelectedColumns() after AddColumnPair().

 Warning:  namColX and  namColY are only checked for their validity as strings;
 no check is made that either are valid column names.

\item  \verb|int = obj.RequestSelectedColumns ()| -  Use the current column status values to produce a new request for statistics
 to be produced when RequestData() is called.
 Unlike the superclass implementation, this version adds a new request for every
 possible pairing of the selected columns
 instead of a single request containing all the columns.

\end{itemize}
