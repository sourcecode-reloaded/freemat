\section{vtkGraphHierarchicalBundle}

\subsection{Usage}

 This algorithm creates a vtkPolyData from a vtkGraph.  As opposed to
 vtkGraphToPolyData, which converts each arc into a straight line, each arc
 is converted to a polyline, following a tree structure.  The filter requires
 both a vtkGraph and vtkTree as input.  The tree vertices must be a
 superset of the graph vertices.  A common example is when the graph vertices
 correspond to the leaves of the tree, but the internal vertices of the tree
 represent groupings of graph vertices.  The algorithm matches the vertices
 using the array ''PedigreeId''.  The user may alternately set the
 DirectMapping flag to indicate that the two structures must have directly
 corresponding offsets (i.e. node i in the graph must correspond to node i in
 the tree).

 The vtkGraph defines the topology of the output vtkPolyData (i.e.
 the connections between nodes) while the vtkTree defines the geometry (i.e.
 the location of nodes and arc routes).  Thus, the tree must have been
 assigned vertex locations, but the graph does not need locations, in fact
 they will be ignored.  The edges approximately follow the path from the
 source to target nodes in the tree.  A bundling parameter controls how
 closely the edges are bundled together along the tree structure.
 
 You may follow this algorithm with vtkSplineFilter in order to make nicely
 curved edges.

To create an instance of class vtkGraphHierarchicalBundle, simply
invoke its constructor as follows
\begin{verbatim}
  obj = vtkGraphHierarchicalBundle
\end{verbatim}
\subsection{Methods}

The class vtkGraphHierarchicalBundle has several methods that can be used.
  They are listed below.
Note that the documentation is translated automatically from the VTK sources,
and may not be completely intelligible.  When in doubt, consult the VTK website.
In the methods listed below, \verb|obj| is an instance of the vtkGraphHierarchicalBundle class.
\begin{itemize}
\item  \verb|string = obj.GetClassName ()|

\item  \verb|int = obj.IsA (string name)|

\item  \verb|vtkGraphHierarchicalBundle = obj.NewInstance ()|

\item  \verb|vtkGraphHierarchicalBundle = obj.SafeDownCast (vtkObject o)|

\item  \verb|obj.SetBundlingStrength (double )| -  The level of arc bundling in the graph.
 A strength of 0 creates straight lines, while a strength of 1
 forces arcs to pass directly through hierarchy node points.
 The default value is 0.8.

\item  \verb|double = obj.GetBundlingStrengthMinValue ()| -  The level of arc bundling in the graph.
 A strength of 0 creates straight lines, while a strength of 1
 forces arcs to pass directly through hierarchy node points.
 The default value is 0.8.

\item  \verb|double = obj.GetBundlingStrengthMaxValue ()| -  The level of arc bundling in the graph.
 A strength of 0 creates straight lines, while a strength of 1
 forces arcs to pass directly through hierarchy node points.
 The default value is 0.8.

\item  \verb|double = obj.GetBundlingStrength ()| -  The level of arc bundling in the graph.
 A strength of 0 creates straight lines, while a strength of 1
 forces arcs to pass directly through hierarchy node points.
 The default value is 0.8.

\item  \verb|obj.SetDirectMapping (bool )| -  If on, uses direct mapping from tree to graph vertices.
 If off, both the graph and tree must contain PedigreeId arrays
 which are used to match graph and tree vertices.
 Default is off.

\item  \verb|bool = obj.GetDirectMapping ()| -  If on, uses direct mapping from tree to graph vertices.
 If off, both the graph and tree must contain PedigreeId arrays
 which are used to match graph and tree vertices.
 Default is off.

\item  \verb|obj.DirectMappingOn ()| -  If on, uses direct mapping from tree to graph vertices.
 If off, both the graph and tree must contain PedigreeId arrays
 which are used to match graph and tree vertices.
 Default is off.

\item  \verb|obj.DirectMappingOff ()| -  If on, uses direct mapping from tree to graph vertices.
 If off, both the graph and tree must contain PedigreeId arrays
 which are used to match graph and tree vertices.
 Default is off.

\item  \verb|int = obj.FillInputPortInformation (int port, vtkInformation info)| -  Set the input type of the algorithm to vtkGraph.

\end{itemize}
