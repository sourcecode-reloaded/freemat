\section{vtkCell3D}

\subsection{Usage}

 vtkCell3D is an abstract class that extends the interfaces for 3D data 
 cells, and implements methods needed to satisfy the vtkCell API. The 
 3D cells include hexehedra, tetrahedra, wedge, pyramid, and voxel.

To create an instance of class vtkCell3D, simply
invoke its constructor as follows
\begin{verbatim}
  obj = vtkCell3D
\end{verbatim}
\subsection{Methods}

The class vtkCell3D has several methods that can be used.
  They are listed below.
Note that the documentation is translated automatically from the VTK sources,
and may not be completely intelligible.  When in doubt, consult the VTK website.
In the methods listed below, \verb|obj| is an instance of the vtkCell3D class.
\begin{itemize}
\item  \verb|string = obj.GetClassName ()|

\item  \verb|int = obj.IsA (string name)|

\item  \verb|vtkCell3D = obj.NewInstance ()|

\item  \verb|vtkCell3D = obj.SafeDownCast (vtkObject o)|

\item  \verb|obj.Contour (double value, vtkDataArray cellScalars, vtkIncrementalPointLocator locator, vtkCellArray verts, vtkCellArray lines, vtkCellArray polys, vtkPointData inPd, vtkPointData outPd, vtkCellData inCd, vtkIdType cellId, vtkCellData outCd)|

\item  \verb|obj.Clip (double value, vtkDataArray cellScalars, vtkIncrementalPointLocator locator, vtkCellArray connectivity, vtkPointData inPd, vtkPointData outPd, vtkCellData inCd, vtkIdType cellId, vtkCellData outCd, int insideOut)| -  Cut (or clip) the cell based on the input cellScalars and the specified
 value. The output of the clip operation will be one or more cells of the
 same topological dimension as the original cell.  The flag insideOut
 controls what part of the cell is considered inside - normally cell
 points whose scalar value is greater than ''value'' are considered
 inside. If insideOut is on, this is reversed. Also, if the output cell
 data is non-NULL, the cell data from the clipped cell is passed to the
 generated contouring primitives. (Note: the CopyAllocate() method must
 be invoked on both the output cell and point data. The cellId refers to
 the cell from which the cell data is copied.)  (Satisfies vtkCell API.)

\item  \verb|int = obj.GetCellDimension ()| -  Set the tolerance for merging clip intersection points that are near
 the vertices of cells. This tolerance is used to prevent the generation
 of degenerate tetrahedra during clipping.

\item  \verb|obj.SetMergeTolerance (double )| -  Set the tolerance for merging clip intersection points that are near
 the vertices of cells. This tolerance is used to prevent the generation
 of degenerate tetrahedra during clipping.

\item  \verb|double = obj.GetMergeToleranceMinValue ()| -  Set the tolerance for merging clip intersection points that are near
 the vertices of cells. This tolerance is used to prevent the generation
 of degenerate tetrahedra during clipping.

\item  \verb|double = obj.GetMergeToleranceMaxValue ()| -  Set the tolerance for merging clip intersection points that are near
 the vertices of cells. This tolerance is used to prevent the generation
 of degenerate tetrahedra during clipping.

\item  \verb|double = obj.GetMergeTolerance ()| -  Set the tolerance for merging clip intersection points that are near
 the vertices of cells. This tolerance is used to prevent the generation
 of degenerate tetrahedra during clipping.

\end{itemize}
