\section{vtkKdTreeSelector}

\subsection{Usage}

 If SetKdTree is used, the filter ignores the input and selects based on that
 kd-tree.  If SetKdTree is not used, the filter builds a kd-tree using the
 input point set and uses that tree for selection.  The output is a
 vtkSelection containing the ids found in the kd-tree using the specified
 bounds.

To create an instance of class vtkKdTreeSelector, simply
invoke its constructor as follows
\begin{verbatim}
  obj = vtkKdTreeSelector
\end{verbatim}
\subsection{Methods}

The class vtkKdTreeSelector has several methods that can be used.
  They are listed below.
Note that the documentation is translated automatically from the VTK sources,
and may not be completely intelligible.  When in doubt, consult the VTK website.
In the methods listed below, \verb|obj| is an instance of the vtkKdTreeSelector class.
\begin{itemize}
\item  \verb|string = obj.GetClassName ()|

\item  \verb|int = obj.IsA (string name)|

\item  \verb|vtkKdTreeSelector = obj.NewInstance ()|

\item  \verb|vtkKdTreeSelector = obj.SafeDownCast (vtkObject o)|

\item  \verb|obj.SetKdTree (vtkKdTree tree)| -  The kd-tree to use to find selected ids.
 The kd-tree must be initialized with the desired set of points.
 When this is set, the optional input is ignored.

\item  \verb|vtkKdTree = obj.GetKdTree ()| -  The kd-tree to use to find selected ids.
 The kd-tree must be initialized with the desired set of points.
 When this is set, the optional input is ignored.

\item  \verb|obj.SetSelectionBounds (double , double , double , double , double , double )| -  The bounds of the form (xmin,xmax,ymin,ymax,zmin,zmax).
 To perform a search in 2D, use the bounds
 (xmin,xmax,ymin,ymax,VTK\_DOUBLE\_MIN,VTK\_DOUBLE\_MAX).

\item  \verb|obj.SetSelectionBounds (double  a[6])| -  The bounds of the form (xmin,xmax,ymin,ymax,zmin,zmax).
 To perform a search in 2D, use the bounds
 (xmin,xmax,ymin,ymax,VTK\_DOUBLE\_MIN,VTK\_DOUBLE\_MAX).

\item  \verb|double = obj. GetSelectionBounds ()| -  The bounds of the form (xmin,xmax,ymin,ymax,zmin,zmax).
 To perform a search in 2D, use the bounds
 (xmin,xmax,ymin,ymax,VTK\_DOUBLE\_MIN,VTK\_DOUBLE\_MAX).

\item  \verb|obj.SetSelectionFieldName (string )| -  The field name to use when generating the selection.
 If set, creates a VALUES selection.
 If not set (or is set to NULL), creates a INDICES selection.
 By default this is not set.

\item  \verb|string = obj.GetSelectionFieldName ()| -  The field name to use when generating the selection.
 If set, creates a VALUES selection.
 If not set (or is set to NULL), creates a INDICES selection.
 By default this is not set.

\item  \verb|obj.SetSelectionAttribute (int )| -  The field attribute to use when generating the selection.
 If set, creates a PEDIGREEIDS or GLOBALIDS selection.
 If not set (or is set to -1), creates a INDICES selection.
 By default this is not set.
 NOTE: This should be set a constant in vtkDataSetAttributes,
 not vtkSelection.

\item  \verb|int = obj.GetSelectionAttribute ()| -  The field attribute to use when generating the selection.
 If set, creates a PEDIGREEIDS or GLOBALIDS selection.
 If not set (or is set to -1), creates a INDICES selection.
 By default this is not set.
 NOTE: This should be set a constant in vtkDataSetAttributes,
 not vtkSelection.

\item  \verb|obj.SetSingleSelection (bool )| -  Whether to only allow up to one value in the result.
 The item selected is closest to the center of the bounds,
 if there are any points within the selection threshold.
 Default is off.

\item  \verb|bool = obj.GetSingleSelection ()| -  Whether to only allow up to one value in the result.
 The item selected is closest to the center of the bounds,
 if there are any points within the selection threshold.
 Default is off.

\item  \verb|obj.SingleSelectionOn ()| -  Whether to only allow up to one value in the result.
 The item selected is closest to the center of the bounds,
 if there are any points within the selection threshold.
 Default is off.

\item  \verb|obj.SingleSelectionOff ()| -  Whether to only allow up to one value in the result.
 The item selected is closest to the center of the bounds,
 if there are any points within the selection threshold.
 Default is off.

\item  \verb|obj.SetSingleSelectionThreshold (double )| -  The threshold for the single selection.
 A single point is added to the selection if it is within
 this threshold from the bounds center.
 Default is 1.

\item  \verb|double = obj.GetSingleSelectionThreshold ()| -  The threshold for the single selection.
 A single point is added to the selection if it is within
 this threshold from the bounds center.
 Default is 1.

\item  \verb|long = obj.GetMTime ()|

\end{itemize}
