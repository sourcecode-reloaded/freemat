\section{CLEAR Clear or Delete a Variable}

\subsection{Usage}

Clears a set of variables from the current context, or alternately, 
delete all variables defined in the current context.  There are
several formats for the function call.  The first is the explicit form
in which a list of variables are provided:
\begin{verbatim}
   clear a1 a2 ...
\end{verbatim}
The variables can be persistent or global, and they will be deleted.
The second form
\begin{verbatim}
   clear 'all'
\end{verbatim}
clears all variables and libraries from the current context.  Alternately, you can
use the form:
\begin{verbatim}
   clear 'libs'
\end{verbatim}
which will unload any libraries or DLLs that have been \verb|import|ed. 
Optionally, you can specify that persistent variables should be cleared via:
\begin{verbatim}
   clear 'persistent'
\end{verbatim}
and similarly for global variables:
\begin{verbatim}
   clear 'global'
\end{verbatim}
You can use
\begin{verbatim}
   clear 'classes'
\end{verbatim}
to clear all definitions of user-defined classes.
With no arguments, \verb|clear| defaults to clearing \verb|'all'|.
\subsection{Example}

Here is a simple example of using \verb|clear| to delete a variable.  First, we create a variable called \verb|a|:
\begin{verbatim}
--> a = 53

a = 
 53 
\end{verbatim}
Next, we clear \verb|a| using the \verb|clear| function, and verify that it is deleted.
\begin{verbatim}
--> clear a
--> a
Error: Undefined function or variable a
\end{verbatim}
