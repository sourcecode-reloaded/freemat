\section{vtkDirectionEncoder}

\subsection{Usage}

 Given a direction, encode it into an integer value. This value should
 be less than 65536, which is the maximum number of encoded directions
 supported by this superclass. A direction encoded is used to encode
 normals in a volume for use during volume rendering, and the
 amount of space that is allocated per normal is 2 bytes.
 This is an abstract superclass - see the subclasses for specific 
 implementation details.


To create an instance of class vtkDirectionEncoder, simply
invoke its constructor as follows
\begin{verbatim}
  obj = vtkDirectionEncoder
\end{verbatim}
\subsection{Methods}

The class vtkDirectionEncoder has several methods that can be used.
  They are listed below.
Note that the documentation is translated automatically from the VTK sources,
and may not be completely intelligible.  When in doubt, consult the VTK website.
In the methods listed below, \verb|obj| is an instance of the vtkDirectionEncoder class.
\begin{itemize}
\item  \verb|string = obj.GetClassName ()| -  Get the name of this class

\item  \verb|int = obj.IsA (string name)| -  Get the name of this class

\item  \verb|vtkDirectionEncoder = obj.NewInstance ()| -  Get the name of this class

\item  \verb|vtkDirectionEncoder = obj.SafeDownCast (vtkObject o)| -  Get the name of this class

\item  \verb|int = obj.GetEncodedDirection (float n[3])| -  Given a normal vector n, return the encoded direction

\item  \verb|float = obj.GetDecodedGradient (int value)| - / Given an encoded value, return a pointer to the normal vector

\item  \verb|int = obj.GetNumberOfEncodedDirections (void )| -  Return the number of encoded directions

\end{itemize}
