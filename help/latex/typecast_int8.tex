\section{INT8 Convert to Signed 8-bit Integer}

\subsection{Usage}

Converts the argument to an signed 8-bit Integer.  The syntax
for its use is
\begin{verbatim}
   y = int8(x)
\end{verbatim}
where \verb|x| is an \verb|n|-dimensional numerical array.  Conversion
follows the saturation rules (e.g., if \verb|x| is outside the normal
range for a signed 8-bit integer of \verb|[-127,127]|, it is truncated to that
range.  Note that
both \verb|NaN| and \verb|Inf| both map to 0.
\subsection{Example}

The following piece of code demonstrates several uses of \verb|int8|.  First, the routine uses
@>
In the next example, an integer outside the range  of the type is passed in.  
The result is truncated to the range of the type.
@>
In the next example, a positive double precision argument is passed in.  
The result is the signed integer that is closest to the argument.
@>
In the next example, a complex argument is passed in.  The result is the 
signed complex integer that is closest to the argument.
@>
In the next example, a string argument is passed in.  The string argument 
is converted into an integer array corresponding to the ASCII values of each character.
@>
In the last example, a cell-array is passed in.  For cell-arrays and 
structure arrays, the result is an error.
@>
