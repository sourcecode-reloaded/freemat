\section{vtkFastNumericConversion}

\subsection{Usage}

 vtkFastNumericConversion uses a portable (assuming IEEE format) method for
 converting single and double precision floating point values to a fixed
 point representation. This allows fast integer floor operations on
 platforms, such as Intel X86, in which CPU floating point conversion
 algorithms are very slow. It is based on the techniques described in Chris
 Hecker's article, ''Let's Get to the (Floating) Point'', in Game Developer
 Magazine, Feb/Mar 1996, and the techniques described in Michael Herf's
 website, http://www.stereopsis.com/FPU.html.  The Hecker article can be
 found at http://www.d6.com/users/checker/pdfs/gdmfp.pdf.  Unfortunately,
 each of these techniques is incomplete, and doesn't convert properly, in a
 way that depends on how many bits are reserved for fixed point fractional
 use, due to failing to properly account for the default round-towards-even
 rounding mode of the X86. Thus, my implementation incorporates some
 rounding correction that undoes the rounding that the FPU performs during
 denormalization of the floating point value. Note that the rounding affect
 I'm talking about here is not the effect on the fistp instruction, but
 rather the effect that occurs during the denormalization of a value that
 occurs when adding it to a much larger value. The bits must be shifted to
 the right, and when a ''1'' bit falls off the edge, the rounding mode
 determines what happens next, in order to avoid completely ''losing'' the
 1-bit. Furthermore, my implementation works on Linux, where the default
 precision mode is 64-bit extended precision.

To create an instance of class vtkFastNumericConversion, simply
invoke its constructor as follows
\begin{verbatim}
  obj = vtkFastNumericConversion
\end{verbatim}
\subsection{Methods}

The class vtkFastNumericConversion has several methods that can be used.
  They are listed below.
Note that the documentation is translated automatically from the VTK sources,
and may not be completely intelligible.  When in doubt, consult the VTK website.
In the methods listed below, \verb|obj| is an instance of the vtkFastNumericConversion class.
\begin{itemize}
\item  \verb|string = obj.GetClassName ()|

\item  \verb|int = obj.IsA (string name)|

\item  \verb|vtkFastNumericConversion = obj.NewInstance ()|

\item  \verb|vtkFastNumericConversion = obj.SafeDownCast (vtkObject o)|

\item  \verb|int = obj.TestQuickFloor (double val)| -  Wrappable method for script-testing of correct cross-platform
 functionality

\item  \verb|int = obj.TestSafeFloor (double val)| -  Wrappable method for script-testing of correct cross-platform
 functionality

\item  \verb|int = obj.TestRound (double val)| -  Wrappable method for script-testing of correct cross-platform
 functionality

\item  \verb|int = obj.TestConvertFixedPointIntPart (double val)| -  Wrappable method for script-testing of correct cross-platform
 functionality

\item  \verb|int = obj.TestConvertFixedPointFracPart (double val)| -  Wrappable method for script-testing of correct cross-platform
 functionality

\item  \verb|obj.SetReservedFracBits (int bits)| -  Set the number of bits reserved for fractional precision that are
 maintained as part of the flooring process. This number affects the
 flooring arithmetic. It may be useful if the factional part is to be
 used to index into a lookup table of some sort. However, if you are only
 interested in knowing the fractional remainder after flooring, there
 doesn't appear to be any advantage to using these bits, either in terms
 of a lookup table, or by directly multiplying by some unit fraction,
 over simply subtracting the floored value from the original value. Note
 that since only 32 bits are used for the entire fixed point
 representation, increasing the number of reserved fractional bits
 reduces the range of integer values that can be floored to.
 Add one to the requested number of fractional bits, to make
 the conversion safe with respect to rounding mode. This is the
 same as the difference between QuickFloor and SafeFloor.

\item  \verb|obj.PerformanceTests (void )| -  Conduct timing tests so that the usefulness of this class can be
 ascertained on whatever platform it is being used. Output can be
 retrieved via Print method.

\end{itemize}
