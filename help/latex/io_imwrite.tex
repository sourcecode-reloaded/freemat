\section{IMWRITE Write Matrix to Image File}

\subsection{Usage}

Write the image data from the matrix into a given file.  Note that
FreeMat's support for \verb|imwrite| is not complete.
You can write images in the \verb|jpg,png,xpm,ppm| and some other formats.
The syntax for its use is
\begin{verbatim}
  imwrite(A, filename)
  imwrite(A, map, filename)
  imwrite(A, map, filename, 'Alpha', alpha)

or Octave-style syntax:
  imwrite(filename, A)
  imwrite(filename, A, map)
  imwrite(filename, A, map, alpha)
\end{verbatim}
where \verb|filename| is the name of the file to write to.  The input array 
\verb|A| contains the image data (2D for gray or indexed, and 3D for color).  
If \verb|A| is an integer array (int8, uint8, int16, uint16, int32, uint32), 
the values of its elements should be within 0-255.  If \verb|A| is a 
floating-point array (float or double), the value of its elements should
be in the range [0,1].  \verb|map| contains the colormap information
(for indexed images), and \verb|alpha| the alphamap (transparency).
\subsection{Example}

Here is a simple example of \verb|imread|/\verb|imwrite|.  First, we generate
a grayscale image and save it to an image file.
\begin{verbatim}
--> a =  uint8(255*rand(64));
--> figure(1), image(a), colormap(gray)
--> title('image to save')
Warning: Newly defined variable nargin shadows a function of the same name.  Use clear nargin to recover access to the function
--> imwrite(a, 'test.bmp')
\end{verbatim}
Then, we read image file and show it:
\begin{verbatim}
--> b = imread('test.bmp');
--> figure(2), image(b), colormap(gray)
--> title('loaded image')
Warning: Newly defined variable nargin shadows a function of the same name.  Use clear nargin to recover access to the function
\end{verbatim}
