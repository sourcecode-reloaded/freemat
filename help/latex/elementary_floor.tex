\section{FLOOR Floor Function}

\subsection{Usage}

Computes the floor of an n-dimensional array elementwise.  The
floor of a number is defined as the smallest integer that is
less than or equal to that number. The general syntax for its use
is
\begin{verbatim}
   y = floor(x)
\end{verbatim}
where \verb|x| is a multidimensional array of numerical type.  The \verb|floor| 
function preserves the type of the argument.  So integer arguments 
are not modified, and \verb|float| arrays return \verb|float| arrays as 
outputs, and similarly for \verb|double| arrays.  The \verb|floor| function 
is not defined for complex types.
\subsection{Example}

The following demonstrates the \verb|floor| function applied to various
(numerical) arguments.  For integer arguments, the floor function has
no effect:
\begin{verbatim}
--> floor(3)

ans = 
 3 

--> floor(-3)

ans = 
 -3 
\end{verbatim}
Next, we take the \verb|floor| of a floating point value:
\begin{verbatim}
--> floor(3.023)

ans = 
 3 

--> floor(-2.341)

ans = 
 -3 
\end{verbatim}
