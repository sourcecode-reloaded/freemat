\section{ZEROS Array of Zeros}

\subsection{Usage}

Creates an array of zeros of the specified size.  Two seperate 
syntaxes are possible.  The first syntax specifies the array 
dimensions as a sequence of scalar dimensions:
\begin{verbatim}
   y = zeros(d1,d2,...,dn).
\end{verbatim}
The resulting array has the given dimensions, and is filled with
all zeros.  The type of \verb|y| is \verb|double|, a 64-bit floating
point array.  To get arrays of other types, use the typecast 
functions (e.g., \verb|uint8|, \verb|int8|, etc.).  An alternative syntax
is to use the following notation:
\begin{verbatim}
   y = zeros(d1,d2,...,dn,classname)
\end{verbatim}
where \verb|classname| is one of 'double', 'single', 'int8', 'uint8',
'int16', 'uint16', 'int32', 'uint32', 'int64', 'uint64', 'float', 'logical'.  
    
The second syntax specifies the array dimensions as a vector,
where each element in the vector specifies a dimension length:
\begin{verbatim}
   y = zeros([d1,d2,...,dn]),
\end{verbatim}
or
\begin{verbatim}
   y = zeros([d1,d2,...,dn],classname).
\end{verbatim}
This syntax is more convenient for calling \verb|zeros| using a 
variable for the argument.  In both cases, specifying only one
dimension results in a square matrix output.
\subsection{Example}

The following examples demonstrate generation of some zero arrays 
using the first form.
\begin{verbatim}
--> zeros(2,3,2)

ans = 

(:,:,1) = 

 0 0 0 
 0 0 0 

(:,:,2) = 

 0 0 0 
 0 0 0 

--> zeros(1,3)

ans = 

 0 0 0 
\end{verbatim}
The same expressions, using the second form.
\begin{verbatim}
--> zeros([2,6])

ans = 

 0 0 0 0 0 0 
 0 0 0 0 0 0 

--> zeros([1,3])

ans = 

 0 0 0 
\end{verbatim}
Finally, an example of using the type casting function \verb|uint16| to generate an array of 16-bit unsigned integers with zero values.
\begin{verbatim}
--> uint16(zeros(3))

ans = 

 0 0 0 
 0 0 0 
 0 0 0 
\end{verbatim}
Here we use the second syntax where the class of the output is specified 
explicitly
\begin{verbatim}
--> zeros(3,'int16')

ans = 

 0 0 0 
 0 0 0 
 0 0 0 
\end{verbatim}
