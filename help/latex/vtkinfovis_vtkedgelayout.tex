\section{vtkEdgeLayout}

\subsection{Usage}

 This class is a shell for many edge layout strategies which may be set
 using the SetLayoutStrategy() function.  The layout strategies do the
 actual work.

To create an instance of class vtkEdgeLayout, simply
invoke its constructor as follows
\begin{verbatim}
  obj = vtkEdgeLayout
\end{verbatim}
\subsection{Methods}

The class vtkEdgeLayout has several methods that can be used.
  They are listed below.
Note that the documentation is translated automatically from the VTK sources,
and may not be completely intelligible.  When in doubt, consult the VTK website.
In the methods listed below, \verb|obj| is an instance of the vtkEdgeLayout class.
\begin{itemize}
\item  \verb|string = obj.GetClassName ()|

\item  \verb|int = obj.IsA (string name)|

\item  \verb|vtkEdgeLayout = obj.NewInstance ()|

\item  \verb|vtkEdgeLayout = obj.SafeDownCast (vtkObject o)|

\item  \verb|obj.SetLayoutStrategy (vtkEdgeLayoutStrategy strategy)| -  The layout strategy to use during graph layout.

\item  \verb|vtkEdgeLayoutStrategy = obj.GetLayoutStrategy ()| -  The layout strategy to use during graph layout.

\item  \verb|long = obj.GetMTime ()| -  Get the modification time of the layout algorithm.

\end{itemize}
