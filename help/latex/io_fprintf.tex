\section{FPRINTF Formated File Output Function (C-Style)}

\subsection{Usage}

Prints values to a file.  The general syntax for its use is
\begin{verbatim}
  fprintf(fp,format,a1,a2,...).
\end{verbatim}
or, 
\begin{verbatim}
  fprintf(format,a1,a2,...).
\end{verbatim}
Here \verb|format| is the format string, which is a string that
controls the format of the output.  The values of the variables
\verb|ai| are substituted into the output as required.  It is
an error if there are not enough variables to satisfy the format
string.  Note that this \verb|fprintf| command is not vectorized!  Each
variable must be a scalar.  The value \verb|fp| is the file handle.  If \verb|fp| is omitted,
file handle \verb|1| is assumed, and the behavior of \verb|fprintf| is effectively equivalent to \verb|printf|. For
more details on the format string, see \verb|printf|.
\subsection{Examples}

A number of examples are present in the Examples section of the \verb|printf| command.
