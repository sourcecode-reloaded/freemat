\section{vtkSpiderPlotActor}

\subsection{Usage}

 vtkSpiderPlotActor generates a spider plot from an input field (i.e.,
 vtkDataObject). A spider plot represents N-dimensional data by using a set
 of N axes that originate from the center of a circle, and form the spokes
 of a wheel (like a spider web).  Each N-dimensional point is plotted as a
 polyline that forms a closed polygon; the vertices of the polygon
 are plotted against the radial axes.

 To use this class, you must specify an input data object. You'll probably
 also want to specify the position of the plot be setting the Position and
 Position2 instance variables, which define a rectangle in which the plot
 lies. Another important parameter is the IndependentVariables ivar, which
 tells the instance how to interpret the field data (independent variables
 as the rows or columns of the field). There are also many other instance
 variables that control the look of the plot includes its title and legend.

 Set the text property/attributes of the title and the labels through the 
 vtkTextProperty objects associated with these components.

To create an instance of class vtkSpiderPlotActor, simply
invoke its constructor as follows
\begin{verbatim}
  obj = vtkSpiderPlotActor
\end{verbatim}
\subsection{Methods}

The class vtkSpiderPlotActor has several methods that can be used.
  They are listed below.
Note that the documentation is translated automatically from the VTK sources,
and may not be completely intelligible.  When in doubt, consult the VTK website.
In the methods listed below, \verb|obj| is an instance of the vtkSpiderPlotActor class.
\begin{itemize}
\item  \verb|string = obj.GetClassName ()| -  Standard methods for type information and printing.

\item  \verb|int = obj.IsA (string name)| -  Standard methods for type information and printing.

\item  \verb|vtkSpiderPlotActor = obj.NewInstance ()| -  Standard methods for type information and printing.

\item  \verb|vtkSpiderPlotActor = obj.SafeDownCast (vtkObject o)| -  Standard methods for type information and printing.

\item  \verb|obj.SetInput (vtkDataObject )| -  Set the input to the spider plot actor.

\item  \verb|vtkDataObject = obj.GetInput ()| -  Get the input data object to this actor.

\item  \verb|obj.SetIndependentVariables (int )| -  Specify whether to use the rows or columns as independent variables.
 If columns, then each row represents a separate point. If rows, then 
 each column represents a separate point.

\item  \verb|int = obj.GetIndependentVariablesMinValue ()| -  Specify whether to use the rows or columns as independent variables.
 If columns, then each row represents a separate point. If rows, then 
 each column represents a separate point.

\item  \verb|int = obj.GetIndependentVariablesMaxValue ()| -  Specify whether to use the rows or columns as independent variables.
 If columns, then each row represents a separate point. If rows, then 
 each column represents a separate point.

\item  \verb|int = obj.GetIndependentVariables ()| -  Specify whether to use the rows or columns as independent variables.
 If columns, then each row represents a separate point. If rows, then 
 each column represents a separate point.

\item  \verb|obj.SetIndependentVariablesToColumns ()| -  Specify whether to use the rows or columns as independent variables.
 If columns, then each row represents a separate point. If rows, then 
 each column represents a separate point.

\item  \verb|obj.SetIndependentVariablesToRows ()| -  Specify whether to use the rows or columns as independent variables.
 If columns, then each row represents a separate point. If rows, then 
 each column represents a separate point.

\item  \verb|obj.SetTitleVisibility (int )| -  Enable/Disable the display of a plot title.

\item  \verb|int = obj.GetTitleVisibility ()| -  Enable/Disable the display of a plot title.

\item  \verb|obj.TitleVisibilityOn ()| -  Enable/Disable the display of a plot title.

\item  \verb|obj.TitleVisibilityOff ()| -  Enable/Disable the display of a plot title.

\item  \verb|obj.SetTitle (string )| -  Set/Get the title of the spider plot.

\item  \verb|string = obj.GetTitle ()| -  Set/Get the title of the spider plot.

\item  \verb|obj.SetTitleTextProperty (vtkTextProperty p)| -  Set/Get the title text property.

\item  \verb|vtkTextProperty = obj.GetTitleTextProperty ()| -  Set/Get the title text property.

\item  \verb|obj.SetLabelVisibility (int )|

\item  \verb|int = obj.GetLabelVisibility ()|

\item  \verb|obj.LabelVisibilityOn ()|

\item  \verb|obj.LabelVisibilityOff ()|

\item  \verb|obj.SetLabelTextProperty (vtkTextProperty p)| -  Enable/Disable the creation of a legend. If on, the legend labels will
 be created automatically unless the per plot legend symbol has been
 set.

\item  \verb|vtkTextProperty = obj.GetLabelTextProperty ()| -  Enable/Disable the creation of a legend. If on, the legend labels will
 be created automatically unless the per plot legend symbol has been
 set.

\item  \verb|obj.SetNumberOfRings (int )| -  Specify the number of circumferential rings. If set to zero, then
 none will be shown; othewise the specified number will be shown.

\item  \verb|int = obj.GetNumberOfRingsMinValue ()| -  Specify the number of circumferential rings. If set to zero, then
 none will be shown; othewise the specified number will be shown.

\item  \verb|int = obj.GetNumberOfRingsMaxValue ()| -  Specify the number of circumferential rings. If set to zero, then
 none will be shown; othewise the specified number will be shown.

\item  \verb|int = obj.GetNumberOfRings ()| -  Specify the number of circumferential rings. If set to zero, then
 none will be shown; othewise the specified number will be shown.

\item  \verb|obj.SetAxisLabel (int i, string )| -  Specify the names of the radial spokes (i.e., the radial axes). If
 not specified, then an integer number is automatically generated.

\item  \verb|string = obj.GetAxisLabel (int i)| -  Specify the names of the radial spokes (i.e., the radial axes). If
 not specified, then an integer number is automatically generated.

\item  \verb|obj.SetAxisRange (int i, double min, double max)| -  Specify the range of data on each radial axis. If not specified,
 then the range is computed automatically. 

\item  \verb|obj.SetAxisRange (int i, double range[2])| -  Specify the range of data on each radial axis. If not specified,
 then the range is computed automatically. 

\item  \verb|obj.GetAxisRange (int i, double range[2])| -  Specify the range of data on each radial axis. If not specified,
 then the range is computed automatically. 

\item  \verb|obj.SetPlotColor (int i, double r, double g, double b)| -  Specify colors for each plot. If not specified, they are automatically generated.

\item  \verb|obj.SetPlotColor (int i, double color[3])| -  Specify colors for each plot. If not specified, they are automatically generated.

\item  \verb|obj.SetLegendVisibility (int )| -  Enable/Disable the creation of a legend. If on, the legend labels will
 be created automatically unless the per plot legend symbol has been
 set.

\item  \verb|int = obj.GetLegendVisibility ()| -  Enable/Disable the creation of a legend. If on, the legend labels will
 be created automatically unless the per plot legend symbol has been
 set.

\item  \verb|obj.LegendVisibilityOn ()| -  Enable/Disable the creation of a legend. If on, the legend labels will
 be created automatically unless the per plot legend symbol has been
 set.

\item  \verb|obj.LegendVisibilityOff ()| -  Enable/Disable the creation of a legend. If on, the legend labels will
 be created automatically unless the per plot legend symbol has been
 set.

\item  \verb|vtkLegendBoxActor = obj.GetLegendActor ()| -  Retrieve handles to the legend box. This is useful if you would like 
 to manually control the legend appearance.

\item  \verb|int = obj.RenderOverlay (vtkViewport )| -  Draw the spider plot.

\item  \verb|int = obj.RenderOpaqueGeometry (vtkViewport )| -  Draw the spider plot.

\item  \verb|int = obj.RenderTranslucentPolygonalGeometry (vtkViewport )| -  Does this prop have some translucent polygonal geometry?

\item  \verb|int = obj.HasTranslucentPolygonalGeometry ()| -  Does this prop have some translucent polygonal geometry?

\item  \verb|obj.ReleaseGraphicsResources (vtkWindow )| -  Release any graphics resources that are being consumed by this actor.
 The parameter window could be used to determine which graphic
 resources to release.

\end{itemize}
