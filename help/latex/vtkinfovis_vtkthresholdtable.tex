\section{vtkThresholdTable}

\subsection{Usage}

 vtkThresholdTable uses minimum and/or maximum values to threshold
 table rows based on the values in a particular column.
 The column to threshold is specified using SetInputArrayToProcess(0, ...).

To create an instance of class vtkThresholdTable, simply
invoke its constructor as follows
\begin{verbatim}
  obj = vtkThresholdTable
\end{verbatim}
\subsection{Methods}

The class vtkThresholdTable has several methods that can be used.
  They are listed below.
Note that the documentation is translated automatically from the VTK sources,
and may not be completely intelligible.  When in doubt, consult the VTK website.
In the methods listed below, \verb|obj| is an instance of the vtkThresholdTable class.
\begin{itemize}
\item  \verb|string = obj.GetClassName ()|

\item  \verb|int = obj.IsA (string name)|

\item  \verb|vtkThresholdTable = obj.NewInstance ()|

\item  \verb|vtkThresholdTable = obj.SafeDownCast (vtkObject o)|

\item  \verb|obj.SetMode (int )| -  The mode of the threshold filter.  Options are:
 ACCEPT\_LESS\_THAN (0) accepts rows with values < MaxValue;
 ACCEPT\_GREATER\_THAN (1) accepts rows with values > MinValue;
 ACCEPT\_BETWEEN (2) accepts rows with values > MinValue and < MaxValue;
 ACCEPT\_OUTSIDE (3) accepts rows with values < MinValue or > MaxValue.

\item  \verb|int = obj.GetModeMinValue ()| -  The mode of the threshold filter.  Options are:
 ACCEPT\_LESS\_THAN (0) accepts rows with values < MaxValue;
 ACCEPT\_GREATER\_THAN (1) accepts rows with values > MinValue;
 ACCEPT\_BETWEEN (2) accepts rows with values > MinValue and < MaxValue;
 ACCEPT\_OUTSIDE (3) accepts rows with values < MinValue or > MaxValue.

\item  \verb|int = obj.GetModeMaxValue ()| -  The mode of the threshold filter.  Options are:
 ACCEPT\_LESS\_THAN (0) accepts rows with values < MaxValue;
 ACCEPT\_GREATER\_THAN (1) accepts rows with values > MinValue;
 ACCEPT\_BETWEEN (2) accepts rows with values > MinValue and < MaxValue;
 ACCEPT\_OUTSIDE (3) accepts rows with values < MinValue or > MaxValue.

\item  \verb|int = obj.GetMode ()| -  The mode of the threshold filter.  Options are:
 ACCEPT\_LESS\_THAN (0) accepts rows with values < MaxValue;
 ACCEPT\_GREATER\_THAN (1) accepts rows with values > MinValue;
 ACCEPT\_BETWEEN (2) accepts rows with values > MinValue and < MaxValue;
 ACCEPT\_OUTSIDE (3) accepts rows with values < MinValue or > MaxValue.

\item  \verb|obj.SetMinValue (double v)| -  The maximum value for the threshold as a double.

\item  \verb|obj.SetMaxValue (double v)| -  Criterion is rows whose scalars are between lower and upper thresholds
 (inclusive of the end values).

\item  \verb|obj.ThresholdBetween (double lower, double upper)|

\end{itemize}
