\section{DIFF Difference Function}

\subsection{Usage}

\begin{verbatim}
	y=diff(x)
	y=diff(x,k)
	y=diff(x,k,dim)
\end{verbatim}
 Produce difference of successive vector elements.
 
 If \verb|x| is a vector of length n, \verb|diff (x)| is the
 vector of first differences
 \[
  [x_2 - x_1, ..., x_n - x_{n-1}].
 \]

 If \verb|x| is a matrix, \verb|diff (x)| is the matrix of column
 differences along the first non-singleton dimension.

 The second argument is optional.  If supplied, \verb|diff (x,k)|,
 where \verb|k| is a nonnegative integer, returns the
 \verb|k|-th differences. It is possible that \verb|k| is larger than
 then first non-singleton dimension of the matrix. In this case,
 \verb|diff| continues to take the differences along the next
 non-singleton dimension.

 The dimension along which to take the difference can be explicitly
 stated with the optional variable \verb|dim|. In this case the 
 \verb|k|-th order differences are calculated along this dimension.
 In the case where \verb|k| exceeds \verb|size (x, dim)|
 then an empty matrix is returned.

