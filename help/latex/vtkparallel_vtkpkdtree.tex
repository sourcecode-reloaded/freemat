\section{vtkPKdTree}

\subsection{Usage}

      Build, in parallel, a k-d tree decomposition of one or more
      vtkDataSets distributed across processors.  We assume each
      process has read in one portion of a large distributed data set.
      When done, each process has access to the k-d tree structure, 
      can obtain information about which process contains 
      data for each spatial region, and can depth sort the spatial
      regions.

      This class can also assign spatial regions to processors, based
      on one of several region assignment schemes.  By default 
      a contiguous, convex region is assigned to each process.  Several
      queries return information about how many and what cells I have
      that lie in a region assigned to another process.


To create an instance of class vtkPKdTree, simply
invoke its constructor as follows
\begin{verbatim}
  obj = vtkPKdTree
\end{verbatim}
\subsection{Methods}

The class vtkPKdTree has several methods that can be used.
  They are listed below.
Note that the documentation is translated automatically from the VTK sources,
and may not be completely intelligible.  When in doubt, consult the VTK website.
In the methods listed below, \verb|obj| is an instance of the vtkPKdTree class.
\begin{itemize}
\item  \verb|string = obj.GetClassName ()|

\item  \verb|int = obj.IsA (string name)|

\item  \verb|vtkPKdTree = obj.NewInstance ()|

\item  \verb|vtkPKdTree = obj.SafeDownCast (vtkObject o)|

\item  \verb|obj.BuildLocator ()| -    Build the spatial decomposition.  Call this explicitly
   after changing any parameters affecting the build of the
   tree.  It must be called by all processes in the parallel
   application, or it will hang.  

\item  \verb|vtkIdType = obj.GetTotalNumberOfCells ()| -    Create tables of counts of cells per process per region.
   These tables can be accessed with queries like
   ''HasData'', ''GetProcessCellCountForRegion'', and so on.
   You must have called BuildLocator() beforehand.  This
   method must be called by all processes or it will hang.
   Returns 1 on error, 0 when no error.

\item  \verb|int = obj.CreateProcessCellCountData ()| -    Create tables of counts of cells per process per region.
   These tables can be accessed with queries like
   ''HasData'', ''GetProcessCellCountForRegion'', and so on.
   You must have called BuildLocator() beforehand.  This
   method must be called by all processes or it will hang.
   Returns 1 on error, 0 when no error.

\item  \verb|int = obj.CreateGlobalDataArrayBounds ()| -    A convenience function which compiles the global 
   bounds of the data arrays across processes.  
   These bounds can be accessed with 
   ''GetCellArrayGlobalRange'' and ''GetPointArrayGlobalRange''.
   This method must be called by all processes or it will hang.
   Returns 1 on error, 0 when no error.

\item  \verb|obj.SetController (vtkMultiProcessController c)| -    Set/Get the communicator object

\item  \verb|vtkMultiProcessController = obj.GetController ()| -    Set/Get the communicator object

\item  \verb|int = obj.GetRegionAssignment ()| -    The PKdTree class can assign spatial regions to processors after
   building the k-d tree, using one of several partitioning criteria.
   These functions Set/Get whether this assignment is computed. 
   The default is ''Off'', no assignment is computed.   If ''On'', and
   no assignment scheme is specified, contiguous assignment will be
   computed.  Specifying an assignment scheme (with AssignRegions*())
   automatically turns on RegionAssignment.

\item  \verb|int = obj.AssignRegions (int map, int numRegions)| -    Assign spatial regions to processes via a user defined map.
   The user-supplied map is indexed by region ID, and provides a
   process ID for each region. 

\item  \verb|int = obj.AssignRegionsRoundRobin ()| -    Let the PKdTree class assign a process to each region in a
   round robin fashion.  If the k-d tree has not yet been
   built, the regions will be assigned after BuildLocator executes.

\item  \verb|int = obj.AssignRegionsContiguous ()| -     Let the PKdTree class assign a process to each region
    by assigning contiguous sets of spatial regions to each
    process.  The set of regions assigned to each process will
    always have a union that is a convex space (a box).
    If the k-d tree has not yet been built, the regions
    will be assigned after BuildLocator executes.

\item  \verb|int = obj.GetRegionAssignmentList (int procId, vtkIntArray list)| -     Writes the list of region IDs assigned to the specified
    process.  Regions IDs start at 0 and increase by 1 from there.
    Returns the number of regions in the list. 

\item  \verb|obj.GetAllProcessesBorderingOnPoint (float x, float y, float z, vtkIntArray list)| -     The k-d tree spatial regions have been assigned to processes.
    Given a point on the boundary of one of the regions, this
    method creates a list of all processes whose region
    boundaries include that point.  This may be required when
    looking for processes that have cells adjacent to the cells
    of a given process.

\item  \verb|int = obj.GetProcessAssignedToRegion (int regionId)| -     Returns the ID of the process assigned to the region.

\item  \verb|int = obj.HasData (int processId, int regionId)| -    Returns 1 if the process has data for the given region,
   0 otherwise. 

\item  \verb|int = obj.GetProcessCellCountForRegion (int processId, int regionId)| -    Returns the number of cells the specified process has in the
   specified region.  

\item  \verb|int = obj.GetTotalProcessesInRegion (int regionId)| -    Returns the total number of processes that have data
   falling within this spatial region. 

\item  \verb|int = obj.GetProcessListForRegion (int regionId, vtkIntArray processes)| -    Adds the list of processes having data for the given
   region to the supplied list, returns the number of
   processes added.

\item  \verb|int = obj.GetProcessesCellCountForRegion (int regionId, int count, int len)| -    Writes the number of cells each process has for the region
   to the supplied list of length len.  Returns the number of
   cell counts written.  The order of the cell counts corresponds
   to the order of process IDs in the process list returned by
   GetProcessListForRegion.

\item  \verb|int = obj.GetTotalRegionsForProcess (int processId)| -    Returns the total number of spatial regions that a given
   process has data for. 

\item  \verb|int = obj.GetRegionListForProcess (int processId, vtkIntArray regions)| -    Adds the region IDs for which this process has data to
   the supplied vtkIntArray.  Retruns the number of regions.

\item  \verb|int = obj.GetRegionsCellCountForProcess (int ProcessId, int count, int len)| -    Writes to the supplied integer array the number of cells this
   process has for each region.  Returns the number of
   cell counts written.  The order of the cell counts corresponds
   to the order of region IDs in the region list returned by
   GetRegionListForProcess.

\item  \verb|vtkIdType = obj.GetCellListsForProcessRegions (int ProcessId, int set, vtkIdList inRegionCells, vtkIdList onBoundaryCells)| -    After regions have been assigned to processes, I may want to know
   which cells I have that are in the regions assigned to a particular
   process.

   This method takes a process ID and two vtkIdLists.  It
   writes to the first list the IDs of the cells
   contained in the process' regions.  (That is, their cell
   centroid is contained in the region.)  To the second list it
   write the IDs of the cells which intersect the process' regions 
   but whose cell centroid lies elsewhere.

   The total number of cell IDs written to both lists is returned.  
   Either list pointer passed in can be NULL, and it will be ignored. 
   If there are multiple data sets, you must specify which data set
   you wish cell IDs for.  

   The caller should delete these two lists when done.  This method 
   uses the cell lists created in vtkKdTree::CreateCellLists().
   If the cell lists for the process' regions do not exist, this
   method will first build the cell lists for all regions by calling
   CreateCellLists().  You must remember to DeleteCellLists() when 
   done with all calls to this method, as cell lists can require a 
   great deal of memory.  

\item  \verb|vtkIdType = obj.GetCellListsForProcessRegions (int ProcessId, vtkDataSet set, vtkIdList inRegionCells, vtkIdList onBoundaryCells)| -    After regions have been assigned to processes, I may want to know
   which cells I have that are in the regions assigned to a particular
   process.

   This method takes a process ID and two vtkIdLists.  It
   writes to the first list the IDs of the cells
   contained in the process' regions.  (That is, their cell
   centroid is contained in the region.)  To the second list it
   write the IDs of the cells which intersect the process' regions 
   but whose cell centroid lies elsewhere.

   The total number of cell IDs written to both lists is returned.  
   Either list pointer passed in can be NULL, and it will be ignored. 
   If there are multiple data sets, you must specify which data set
   you wish cell IDs for.  

   The caller should delete these two lists when done.  This method 
   uses the cell lists created in vtkKdTree::CreateCellLists().
   If the cell lists for the process' regions do not exist, this
   method will first build the cell lists for all regions by calling
   CreateCellLists().  You must remember to DeleteCellLists() when 
   done with all calls to this method, as cell lists can require a 
   great deal of memory.  

\item  \verb|vtkIdType = obj.GetCellListsForProcessRegions (int ProcessId, vtkIdList inRegionCells, vtkIdList onBoundaryCells)| -    After regions have been assigned to processes, I may want to know
   which cells I have that are in the regions assigned to a particular
   process.

   This method takes a process ID and two vtkIdLists.  It
   writes to the first list the IDs of the cells
   contained in the process' regions.  (That is, their cell
   centroid is contained in the region.)  To the second list it
   write the IDs of the cells which intersect the process' regions 
   but whose cell centroid lies elsewhere.

   The total number of cell IDs written to both lists is returned.  
   Either list pointer passed in can be NULL, and it will be ignored. 
   If there are multiple data sets, you must specify which data set
   you wish cell IDs for.  

   The caller should delete these two lists when done.  This method 
   uses the cell lists created in vtkKdTree::CreateCellLists().
   If the cell lists for the process' regions do not exist, this
   method will first build the cell lists for all regions by calling
   CreateCellLists().  You must remember to DeleteCellLists() when 
   done with all calls to this method, as cell lists can require a 
   great deal of memory.  

\item  \verb|int = obj.DepthOrderAllProcesses (double directionOfProjection, vtkIntArray orderedList)| -  DO NOT CALL.  Deprecated in VTK 5.2.  Use ViewOrderAllProcessesInDirection
 or ViewOrderAllProcessesFromPosition.

\item  \verb|int = obj.ViewOrderAllProcessesInDirection (double directionOfProjection[3], vtkIntArray orderedList)| -  Return a list of all processes in order from front to back given a
 vector direction of projection.  Use this to do visibility sorts
 in parallel projection mode. `orderedList' will be resized to the number
 of processes. The return value is the number of processes.
 

\item  \verb|int = obj.ViewOrderAllProcessesFromPosition (double cameraPosition[3], vtkIntArray orderedList)| -  Return a list of all processes in order from front to back given a
 camera position.  Use this to do visibility sorts in perspective
 projection mode. `orderedList' will be resized to the number
 of processes. The return value is the number of processes.
 

\item  \verb|int = obj.GetCellArrayGlobalRange (string name, float range[2])|

\item  \verb|int = obj.GetPointArrayGlobalRange (string name, float range[2])|

\item  \verb|int = obj.GetCellArrayGlobalRange (string name, double range[2])|

\item  \verb|int = obj.GetPointArrayGlobalRange (string name, double range[2])|

\item  \verb|int = obj.GetCellArrayGlobalRange (int arrayIndex, double range[2])|

\item  \verb|int = obj.GetPointArrayGlobalRange (int arrayIndex, double range[2])|

\item  \verb|int = obj.GetCellArrayGlobalRange (int arrayIndex, float range[2])|

\item  \verb|int = obj.GetPointArrayGlobalRange (int arrayIndex, float range[2])|

\end{itemize}
