\section{vtkProjectedTexture}

\subsection{Usage}

 vtkProjectedTexture assigns texture coordinates to a dataset as if
 the texture was projected from a slide projected located somewhere in the
 scene.  Methods are provided to position the projector and aim it at a 
 location, to set the width of the projector's frustum, and to set the
 range of texture coordinates assigned to the dataset.  

 Objects in the scene that appear behind the projector are also assigned
 texture coordinates; the projected image is left-right and top-bottom 
 flipped, much as a lens' focus flips the rays of light that pass through
 it.  A warning is issued if a point in the dataset falls at the focus
 of the projector.

To create an instance of class vtkProjectedTexture, simply
invoke its constructor as follows
\begin{verbatim}
  obj = vtkProjectedTexture
\end{verbatim}
\subsection{Methods}

The class vtkProjectedTexture has several methods that can be used.
  They are listed below.
Note that the documentation is translated automatically from the VTK sources,
and may not be completely intelligible.  When in doubt, consult the VTK website.
In the methods listed below, \verb|obj| is an instance of the vtkProjectedTexture class.
\begin{itemize}
\item  \verb|string = obj.GetClassName ()|

\item  \verb|int = obj.IsA (string name)|

\item  \verb|vtkProjectedTexture = obj.NewInstance ()|

\item  \verb|vtkProjectedTexture = obj.SafeDownCast (vtkObject o)|

\item  \verb|obj.SetPosition (double , double , double )| -  Set/Get the position of the focus of the projector.

\item  \verb|obj.SetPosition (double  a[3])| -  Set/Get the position of the focus of the projector.

\item  \verb|double = obj. GetPosition ()| -  Set/Get the position of the focus of the projector.

\item  \verb|obj.SetFocalPoint (double focalPoint[3])| -  Set/Get the focal point of the projector (a point that lies along
 the center axis of the projector's frustum).

\item  \verb|obj.SetFocalPoint (double x, double y, double z)| -  Set/Get the focal point of the projector (a point that lies along
 the center axis of the projector's frustum).

\item  \verb|double = obj. GetFocalPoint ()| -  Set/Get the focal point of the projector (a point that lies along
 the center axis of the projector's frustum).

\item  \verb|obj.SetCameraMode (int )| -  Set/Get the camera mode of the projection -- pinhole projection or
 two mirror projection.

\item  \verb|int = obj.GetCameraMode ()| -  Set/Get the camera mode of the projection -- pinhole projection or
 two mirror projection.

\item  \verb|obj.SetCameraModeToPinhole ()| -  Set/Get the camera mode of the projection -- pinhole projection or
 two mirror projection.

\item  \verb|obj.SetCameraModeToTwoMirror ()| -  Set/Get the mirror separation for the two mirror system.

\item  \verb|obj.SetMirrorSeparation (double )| -  Set/Get the mirror separation for the two mirror system.

\item  \verb|double = obj.GetMirrorSeparation ()| -  Set/Get the mirror separation for the two mirror system.

\item  \verb|double = obj. GetOrientation ()| -  Get the normalized orientation vector of the projector.

\item  \verb|obj.SetUp (double , double , double )|

\item  \verb|obj.SetUp (double  a[3])|

\item  \verb|double = obj. GetUp ()|

\item  \verb|obj.SetAspectRatio (double , double , double )|

\item  \verb|obj.SetAspectRatio (double  a[3])|

\item  \verb|double = obj. GetAspectRatio ()|

\item  \verb|obj.SetSRange (double , double )| -  Specify s-coordinate range for texture s-t coordinate pair.

\item  \verb|obj.SetSRange (double  a[2])| -  Specify s-coordinate range for texture s-t coordinate pair.

\item  \verb|double = obj. GetSRange ()| -  Specify s-coordinate range for texture s-t coordinate pair.

\item  \verb|obj.SetTRange (double , double )| -  Specify t-coordinate range for texture s-t coordinate pair.

\item  \verb|obj.SetTRange (double  a[2])| -  Specify t-coordinate range for texture s-t coordinate pair.

\item  \verb|double = obj. GetTRange ()| -  Specify t-coordinate range for texture s-t coordinate pair.

\end{itemize}
