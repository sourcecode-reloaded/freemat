\section{vtkParametricSuperToroid}

\subsection{Usage}

 vtkParametricSuperToroid generates a supertoroid.  Essentially a
 supertoroid is a torus with the sine and cosine terms raised to a power.
 A supertoroid is a versatile primitive that is controlled by four
 parameters r0, r1, n1 and n2. r0, r1 determine the type of torus whilst
 the value of n1 determines the shape of the torus ring and n2 determines
 the shape of the cross section of the ring. It is the different values of
 these powers which give rise to a family of 3D shapes that are all
 basically toroidal in shape.

 For further information about this surface, please consult the 
 technical description ''Parametric surfaces'' in http://www.vtk.org/documents.php 
 in the ''VTK Technical Documents'' section in the VTk.org web pages.

 Also see: http://astronomy.swin.edu.au/~pbourke/surfaces/.


To create an instance of class vtkParametricSuperToroid, simply
invoke its constructor as follows
\begin{verbatim}
  obj = vtkParametricSuperToroid
\end{verbatim}
\subsection{Methods}

The class vtkParametricSuperToroid has several methods that can be used.
  They are listed below.
Note that the documentation is translated automatically from the VTK sources,
and may not be completely intelligible.  When in doubt, consult the VTK website.
In the methods listed below, \verb|obj| is an instance of the vtkParametricSuperToroid class.
\begin{itemize}
\item  \verb|string = obj.GetClassName ()|

\item  \verb|int = obj.IsA (string name)|

\item  \verb|vtkParametricSuperToroid = obj.NewInstance ()|

\item  \verb|vtkParametricSuperToroid = obj.SafeDownCast (vtkObject o)|

\item  \verb|int = obj.GetDimension ()| -  Set/Get the radius from the center to the middle of the ring of the
 supertoroid.  Default = 1.

\item  \verb|obj.SetRingRadius (double )| -  Set/Get the radius from the center to the middle of the ring of the
 supertoroid.  Default = 1.

\item  \verb|double = obj.GetRingRadius ()| -  Set/Get the radius from the center to the middle of the ring of the
 supertoroid.  Default = 1.

\item  \verb|obj.SetCrossSectionRadius (double )| -  Set/Get the radius of the cross section of ring of the supertoroid.
 Default = 0.5.

\item  \verb|double = obj.GetCrossSectionRadius ()| -  Set/Get the radius of the cross section of ring of the supertoroid.
 Default = 0.5.

\item  \verb|obj.SetXRadius (double )| -  Set/Get the scaling factor for the x-axis. Default = 1.

\item  \verb|double = obj.GetXRadius ()| -  Set/Get the scaling factor for the x-axis. Default = 1.

\item  \verb|obj.SetYRadius (double )| -  Set/Get the scaling factor for the y-axis. Default = 1.

\item  \verb|double = obj.GetYRadius ()| -  Set/Get the scaling factor for the y-axis. Default = 1.

\item  \verb|obj.SetZRadius (double )| -  Set/Get the scaling factor for the z-axis. Default = 1.

\item  \verb|double = obj.GetZRadius ()| -  Set/Get the scaling factor for the z-axis. Default = 1.

\item  \verb|obj.SetN1 (double )| -  Set/Get the shape of the torus ring.  Default = 1.

\item  \verb|double = obj.GetN1 ()| -  Set/Get the shape of the torus ring.  Default = 1.

\item  \verb|obj.SetN2 (double )| -   Set/Get the shape of the cross section of the ring. Default = 1.

\item  \verb|double = obj.GetN2 ()| -   Set/Get the shape of the cross section of the ring. Default = 1.

\item  \verb|obj.Evaluate (double uvw[3], double Pt[3], double Duvw[9])| -  A supertoroid.

 This function performs the mapping $f(u,v) \rightarrow (x,y,x)$, returning it
 as Pt. It also returns the partial derivatives Du and Dv.
 $Pt = (x, y, z), Du = (dx/du, dy/du, dz/du), Dv = (dx/dv, dy/dv, dz/dv)$ .
 Then the normal is $N = Du X Dv$ .

\item  \verb|double = obj.EvaluateScalar (double uvw[3], double Pt[3], double Duvw[9])| -  Calculate a user defined scalar using one or all of uvw, Pt, Duvw.

 uvw are the parameters with Pt being the the cartesian point, 
 Duvw are the derivatives of this point with respect to u, v and w.
 Pt, Duvw are obtained from Evaluate().

 This function is only called if the ScalarMode has the value
 vtkParametricFunctionSource::SCALAR\_FUNCTION\_DEFINED

 If the user does not need to calculate a scalar, then the 
 instantiated function should return zero. 


\end{itemize}
