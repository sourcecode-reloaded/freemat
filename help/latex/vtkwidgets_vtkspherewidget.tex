\section{vtkSphereWidget}

\subsection{Usage}

 This 3D widget defines a sphere that can be interactively placed in a
 scene. 

 To use this object, just invoke SetInteractor() with the argument of the
 method a vtkRenderWindowInteractor.  You may also wish to invoke
 ''PlaceWidget()'' to initially position the widget. The interactor will act
 normally until the ''i'' key (for ''interactor'') is pressed, at which point the
 vtkSphereWidget will appear. (See superclass documentation for information
 about changing this behavior.) 
 Events that occur outside of the widget (i.e., no part of
 the widget is picked) are propagated to any other registered obsevers
 (such as the interaction style).  Turn off the widget by pressing the ''i''
 key again (or invoke the Off() method).

 The vtkSphereWidget has several methods that can be used in conjunction
 with other VTK objects. The Set/GetThetaResolution() and
 Set/GetPhiResolution() methods control the number of subdivisions of the
 sphere in the theta and phi directions; the GetPolyData() method can be
 used to get the polygonal representation and can be used for things like
 seeding streamlines. The GetSphere() method returns a sphere implicit
 function that can be used for cutting and clipping. Typical usage of the
 widget is to make use of the StartInteractionEvent, InteractionEvent, and
 EndInteractionEvent events. The InteractionEvent is called on mouse
 motion; the other two events are called on button down and button up
 (any mouse button).

 Some additional features of this class include the ability to control the
 properties of the widget. You can set the properties of the selected and
 unselected representations of the sphere. 

To create an instance of class vtkSphereWidget, simply
invoke its constructor as follows
\begin{verbatim}
  obj = vtkSphereWidget
\end{verbatim}
\subsection{Methods}

The class vtkSphereWidget has several methods that can be used.
  They are listed below.
Note that the documentation is translated automatically from the VTK sources,
and may not be completely intelligible.  When in doubt, consult the VTK website.
In the methods listed below, \verb|obj| is an instance of the vtkSphereWidget class.
\begin{itemize}
\item  \verb|string = obj.GetClassName ()|

\item  \verb|int = obj.IsA (string name)|

\item  \verb|vtkSphereWidget = obj.NewInstance ()|

\item  \verb|vtkSphereWidget = obj.SafeDownCast (vtkObject o)|

\item  \verb|obj.SetEnabled (int )| -  Methods that satisfy the superclass' API.

\item  \verb|obj.PlaceWidget (double bounds[6])| -  Methods that satisfy the superclass' API.

\item  \verb|obj.PlaceWidget ()| -  Methods that satisfy the superclass' API.

\item  \verb|obj.PlaceWidget (double xmin, double xmax, double ymin, double ymax, double zmin, double zmax)| -  Set the representation of the sphere. Different representations are
 useful depending on the application. The default is 
 VTK\_SPHERE\_WIREFRAME.

\item  \verb|obj.SetRepresentation (int )| -  Set the representation of the sphere. Different representations are
 useful depending on the application. The default is 
 VTK\_SPHERE\_WIREFRAME.

\item  \verb|int = obj.GetRepresentationMinValue ()| -  Set the representation of the sphere. Different representations are
 useful depending on the application. The default is 
 VTK\_SPHERE\_WIREFRAME.

\item  \verb|int = obj.GetRepresentationMaxValue ()| -  Set the representation of the sphere. Different representations are
 useful depending on the application. The default is 
 VTK\_SPHERE\_WIREFRAME.

\item  \verb|int = obj.GetRepresentation ()| -  Set the representation of the sphere. Different representations are
 useful depending on the application. The default is 
 VTK\_SPHERE\_WIREFRAME.

\item  \verb|obj.SetRepresentationToOff ()| -  Set the representation of the sphere. Different representations are
 useful depending on the application. The default is 
 VTK\_SPHERE\_WIREFRAME.

\item  \verb|obj.SetRepresentationToWireframe ()| -  Set the representation of the sphere. Different representations are
 useful depending on the application. The default is 
 VTK\_SPHERE\_WIREFRAME.

\item  \verb|obj.SetRepresentationToSurface ()| -  Set/Get the resolution of the sphere in the Theta direction.
 The default is 16.

\item  \verb|obj.SetThetaResolution (int r)| -  Set/Get the resolution of the sphere in the Theta direction.
 The default is 16.

\item  \verb|int = obj.GetThetaResolution ()| -  Set/Get the resolution of the sphere in the Phi direction.
 The default is 8.

\item  \verb|obj.SetPhiResolution (int r)| -  Set/Get the resolution of the sphere in the Phi direction.
 The default is 8.

\item  \verb|int = obj.GetPhiResolution ()| -  Set/Get the radius of sphere. Default is .5.

\item  \verb|obj.SetRadius (double r)| -  Set/Get the radius of sphere. Default is .5.

\item  \verb|double = obj.GetRadius ()| -  Set/Get the center of the sphere.

\item  \verb|obj.SetCenter (double x, double y, double z)| -  Set/Get the center of the sphere.

\item  \verb|obj.SetCenter (double x[3])| -  Set/Get the center of the sphere.

\item  \verb|double = obj.GetCenter ()| -  Set/Get the center of the sphere.

\item  \verb|obj.GetCenter (double xyz[3])| -  Enable translation and scaling of the widget. By default, the widget
 can be translated and rotated.

\item  \verb|obj.SetTranslation (int )| -  Enable translation and scaling of the widget. By default, the widget
 can be translated and rotated.

\item  \verb|int = obj.GetTranslation ()| -  Enable translation and scaling of the widget. By default, the widget
 can be translated and rotated.

\item  \verb|obj.TranslationOn ()| -  Enable translation and scaling of the widget. By default, the widget
 can be translated and rotated.

\item  \verb|obj.TranslationOff ()| -  Enable translation and scaling of the widget. By default, the widget
 can be translated and rotated.

\item  \verb|obj.SetScale (int )| -  Enable translation and scaling of the widget. By default, the widget
 can be translated and rotated.

\item  \verb|int = obj.GetScale ()| -  Enable translation and scaling of the widget. By default, the widget
 can be translated and rotated.

\item  \verb|obj.ScaleOn ()| -  Enable translation and scaling of the widget. By default, the widget
 can be translated and rotated.

\item  \verb|obj.ScaleOff ()| -  Enable translation and scaling of the widget. By default, the widget
 can be translated and rotated.

\item  \verb|obj.SetHandleVisibility (int )| -  The handle sits on the surface of the sphere and may be moved around
 the surface by picking (left mouse) and then moving. The position
 of the handle can be retrieved, this is useful for positioning cameras
 and lights. By default, the handle is turned off.

\item  \verb|int = obj.GetHandleVisibility ()| -  The handle sits on the surface of the sphere and may be moved around
 the surface by picking (left mouse) and then moving. The position
 of the handle can be retrieved, this is useful for positioning cameras
 and lights. By default, the handle is turned off.

\item  \verb|obj.HandleVisibilityOn ()| -  The handle sits on the surface of the sphere and may be moved around
 the surface by picking (left mouse) and then moving. The position
 of the handle can be retrieved, this is useful for positioning cameras
 and lights. By default, the handle is turned off.

\item  \verb|obj.HandleVisibilityOff ()| -  The handle sits on the surface of the sphere and may be moved around
 the surface by picking (left mouse) and then moving. The position
 of the handle can be retrieved, this is useful for positioning cameras
 and lights. By default, the handle is turned off.

\item  \verb|obj.SetHandleDirection (double , double , double )| -  Set/Get the direction vector of the handle relative to the center of
 the sphere. The direction of the handle is from the sphere center to
 the handle position.

\item  \verb|obj.SetHandleDirection (double  a[3])| -  Set/Get the direction vector of the handle relative to the center of
 the sphere. The direction of the handle is from the sphere center to
 the handle position.

\item  \verb|double = obj. GetHandleDirection ()| -  Set/Get the direction vector of the handle relative to the center of
 the sphere. The direction of the handle is from the sphere center to
 the handle position.

\item  \verb|double = obj. GetHandlePosition ()| -  Get the position of the handle.

\item  \verb|obj.GetPolyData (vtkPolyData pd)| -  Grab the polydata (including points) that defines the sphere.  The
 polydata consists of n+1 points, where n is the resolution of the
 sphere. These point values are guaranteed to be up-to-date when either the
 InteractionEvent or EndInteraction events are invoked. The user provides
 the vtkPolyData and the points and polysphere are added to it.

\item  \verb|obj.GetSphere (vtkSphere sphere)| -  Get the spherical implicit function defined by this widget.  Note that
 vtkSphere is a subclass of vtkImplicitFunction, meaning that it can be
 used by a variety of filters to perform clipping, cutting, and selection
 of data.

\item  \verb|vtkProperty = obj.GetSphereProperty ()| -  Get the sphere properties. The properties of the sphere when selected 
 and unselected can be manipulated.

\item  \verb|vtkProperty = obj.GetSelectedSphereProperty ()| -  Get the sphere properties. The properties of the sphere when selected 
 and unselected can be manipulated.

\item  \verb|vtkProperty = obj.GetHandleProperty ()| -  Get the handle properties (the little ball on the sphere is the
 handle). The properties of the handle when selected and unselected
 can be  manipulated.

\item  \verb|vtkProperty = obj.GetSelectedHandleProperty ()| -  Get the handle properties (the little ball on the sphere is the
 handle). The properties of the handle when selected and unselected
 can be  manipulated.

\end{itemize}
