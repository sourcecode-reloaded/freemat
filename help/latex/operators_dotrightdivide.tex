\section{DOTRIGHTDIVIDE Element-wise Right-Division Operator}

\subsection{Usage}

Divides two numerical arrays (elementwise).  There are two forms
for its use, both with the same general syntax:
\begin{verbatim}
  y = a ./ b
\end{verbatim}
where \verb|a| and \verb|b| are \verb|n|-dimensional arrays of numerical type.  In the
first case, the two arguments are the same size, in which case, the 
output \verb|y| is the same size as the inputs, and is the element-wise
division of \verb|b| by \verb|a|.  In the second case, either \verb|a| or \verb|b| is a scalar, 
in which case \verb|y| is the same size as the larger argument,
and is the division of the scalar with each element of the other argument.

The rules for manipulating types has changed in FreeMat 4.0.  See \verb|typerules|
for more details.

\subsection{Function Internals}

There are three formulae for the dot-right-divide operator, depending on the
sizes of the three arguments.  In the most general case, in which 
the two arguments are the same size, the output is computed via:
\[
y(m_1,\ldots,m_d) = \frac{a(m_1,\ldots,m_d)}{b(m_1,\ldots,m_d)}
\]
If \verb|a| is a scalar, then the output is computed via
\[
y(m_1,\ldots,m_d) = \frac{a}{b(m_1,\ldots,m_d)}
\]
On the other hand, if \verb|b| is a scalar, then the output is computed via
\[
y(m_1,\ldots,m_d) = \frac{a(m_1,\ldots,m_d)}{b}.
\]
\subsection{Examples}

Here are some examples of using the dot-right-divide operator.  First, a 
straight-forward usage of the \verb|./| operator.  The first example
is straightforward:
\begin{verbatim}
--> 3 ./ 8

ans = 
    0.3750 
\end{verbatim}

We can also divide complex arguments:
\begin{verbatim}
--> a = 3 + 4*i

a = 
   3.0000 +  4.0000i 

--> b = 5 + 8*i

b = 
   5.0000 +  8.0000i 

--> c = a ./ b

c = 
   0.5281 -  0.0449i 
\end{verbatim}
We can also demonstrate the three forms of the dot-right-divide operator.  First
the element-wise version:
\begin{verbatim}
--> a = [1,2;3,4]

a = 
 1 2 
 3 4 

--> b = [2,3;6,7]

b = 
 2 3 
 6 7 

--> c = a ./ b

c = 
    0.5000    0.6667 
    0.5000    0.5714 
\end{verbatim}
Then the scalar versions
\begin{verbatim}
--> c = a ./ 3

c = 
    0.3333    0.6667 
    1.0000    1.3333 

--> c = 3 ./ a

c = 
    3.0000    1.5000 
    1.0000    0.7500 
\end{verbatim}
