\section{vtkMFIXReader}

\subsection{Usage}

 vtkMFIXReader creates an unstructured grid dataset. It reads a restart
 file and a set of sp files.  The restart file contains the mesh 
 information.  MFIX meshes are either cylindrical or rectilinear, but 
 this reader will convert them to an unstructured grid.  The sp files 
 contain transient data for the cells.  Each sp file has one or more 
 variables stored inside it.  

To create an instance of class vtkMFIXReader, simply
invoke its constructor as follows
\begin{verbatim}
  obj = vtkMFIXReader
\end{verbatim}
\subsection{Methods}

The class vtkMFIXReader has several methods that can be used.
  They are listed below.
Note that the documentation is translated automatically from the VTK sources,
and may not be completely intelligible.  When in doubt, consult the VTK website.
In the methods listed below, \verb|obj| is an instance of the vtkMFIXReader class.
\begin{itemize}
\item  \verb|string = obj.GetClassName ()|

\item  \verb|int = obj.IsA (string name)|

\item  \verb|vtkMFIXReader = obj.NewInstance ()|

\item  \verb|vtkMFIXReader = obj.SafeDownCast (vtkObject o)|

\item  \verb|obj.SetFileName (string )| -  Specify the file name of the MFIX Restart data file to read.

\item  \verb|string = obj.GetFileName ()| -  Specify the file name of the MFIX Restart data file to read.

\item  \verb|int = obj.GetNumberOfCells ()| -  Get the total number of cells. The number of cells is only valid after a
 successful read of the data file is performed.

\item  \verb|int = obj.GetNumberOfPoints ()| -  Get the total number of nodes. The number of nodes is only valid after a
 successful read of the data file is performed.

\item  \verb|int = obj.GetNumberOfCellFields ()| -  Get the number of data components at the nodes and cells.

\item  \verb|obj.SetTimeStep (int )| -  Which TimeStep to read.

\item  \verb|int = obj.GetTimeStep ()| -  Which TimeStep to read.

\item  \verb|int = obj.GetNumberOfTimeSteps ()| -  Returns the number of timesteps.

\item  \verb|int = obj. GetTimeStepRange ()| -  Which TimeStepRange to read

\item  \verb|obj.SetTimeStepRange (int , int )| -  Which TimeStepRange to read

\item  \verb|obj.SetTimeStepRange (int  a[2])| -  Which TimeStepRange to read

\item  \verb|int = obj.GetNumberOfCellArrays (void )|

\item  \verb|string = obj.GetCellArrayName (int index)| -  Get the name of the  cell array with the given index in
 the input.

\item  \verb|int = obj.GetCellArrayStatus (string name)| -  Get/Set whether the cell array with the given name is to
 be read.

\item  \verb|obj.SetCellArrayStatus (string name, int status)| -  Get/Set whether the cell array with the given name is to
 be read.

\item  \verb|obj.DisableAllCellArrays ()| -  Turn on/off all cell arrays.

\item  \verb|obj.EnableAllCellArrays ()| -  Turn on/off all cell arrays.

\item  \verb|obj.GetCellDataRange (int cellComp, int index, float min, float max)| -  Get the range of cell data.

\end{itemize}
