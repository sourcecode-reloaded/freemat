\section{vtkImageReader2Factory}

\subsection{Usage}

 vtkImageReader2Factory: This class is used to create a vtkImageReader2
 object given a path name to a file.  It calls CanReadFile on all
 available readers until one of them returns true.  The available reader
 list comes from three places.  In the InitializeReaders function of this
 class, built-in VTK classes are added to the list, users can call
 RegisterReader, or users can create a vtkObjectFactory that has
 CreateObject method that returns a new vtkImageReader2 sub class when
 given the string ''vtkImageReaderObject''.  This way applications can be
 extended with new readers via a plugin dll or by calling RegisterReader.
 Of course all of the readers that are part of the vtk release are made
 automatically available.


To create an instance of class vtkImageReader2Factory, simply
invoke its constructor as follows
\begin{verbatim}
  obj = vtkImageReader2Factory
\end{verbatim}
\subsection{Methods}

The class vtkImageReader2Factory has several methods that can be used.
  They are listed below.
Note that the documentation is translated automatically from the VTK sources,
and may not be completely intelligible.  When in doubt, consult the VTK website.
In the methods listed below, \verb|obj| is an instance of the vtkImageReader2Factory class.
\begin{itemize}
\item  \verb|string = obj.GetClassName ()|

\item  \verb|int = obj.IsA (string name)|

\item  \verb|vtkImageReader2Factory = obj.NewInstance ()|

\item  \verb|vtkImageReader2Factory = obj.SafeDownCast (vtkObject o)|

\end{itemize}
