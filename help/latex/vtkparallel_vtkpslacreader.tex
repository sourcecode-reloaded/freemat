\section{vtkPSLACReader}

\subsection{Usage}


 Extends the vtkSLACReader to read in partitioned pieces.  Due to the nature
 of the data layout, this reader only works in a data parallel mode where
 each process in a parallel job simultaneously attempts to read the piece
 corresponding to the local process id.


To create an instance of class vtkPSLACReader, simply
invoke its constructor as follows
\begin{verbatim}
  obj = vtkPSLACReader
\end{verbatim}
\subsection{Methods}

The class vtkPSLACReader has several methods that can be used.
  They are listed below.
Note that the documentation is translated automatically from the VTK sources,
and may not be completely intelligible.  When in doubt, consult the VTK website.
In the methods listed below, \verb|obj| is an instance of the vtkPSLACReader class.
\begin{itemize}
\item  \verb|string = obj.GetClassName ()|

\item  \verb|int = obj.IsA (string name)|

\item  \verb|vtkPSLACReader = obj.NewInstance ()|

\item  \verb|vtkPSLACReader = obj.SafeDownCast (vtkObject o)|

\item  \verb|vtkMultiProcessController = obj.GetController ()| -  The controller used to communicate partition data.  The number of pieces
 requested must agree with the number of processes, the piece requested must
 agree with the local process id, and all process must invoke
 ProcessRequests of this filter simultaneously.

\item  \verb|obj.SetController (vtkMultiProcessController )| -  The controller used to communicate partition data.  The number of pieces
 requested must agree with the number of processes, the piece requested must
 agree with the local process id, and all process must invoke
 ProcessRequests of this filter simultaneously.

\end{itemize}
