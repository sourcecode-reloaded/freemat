\section{vtkFillHolesFilter}

\subsection{Usage}

 vtkFillHolesFilter is a filter that identifies and fills holes in
 input vtkPolyData meshes. Holes are identified by locating
 boundary edges, linking them together into loops, and then 
 triangulating the resulting loops. Note that you can specify 
 an approximate limit to the size of the hole that can be filled.
 

To create an instance of class vtkFillHolesFilter, simply
invoke its constructor as follows
\begin{verbatim}
  obj = vtkFillHolesFilter
\end{verbatim}
\subsection{Methods}

The class vtkFillHolesFilter has several methods that can be used.
  They are listed below.
Note that the documentation is translated automatically from the VTK sources,
and may not be completely intelligible.  When in doubt, consult the VTK website.
In the methods listed below, \verb|obj| is an instance of the vtkFillHolesFilter class.
\begin{itemize}
\item  \verb|string = obj.GetClassName ()| -  Standard methods for instantiation, type information and printing.

\item  \verb|int = obj.IsA (string name)| -  Standard methods for instantiation, type information and printing.

\item  \verb|vtkFillHolesFilter = obj.NewInstance ()| -  Standard methods for instantiation, type information and printing.

\item  \verb|vtkFillHolesFilter = obj.SafeDownCast (vtkObject o)| -  Standard methods for instantiation, type information and printing.

\item  \verb|obj.SetHoleSize (double )| -  Specify the maximum hole size to fill. This is represented as a radius
 to the bounding circumsphere containing the hole.  Note that this is an
 approximate area; the actual area cannot be computed without first
 triangulating the hole.

\item  \verb|double = obj.GetHoleSizeMinValue ()| -  Specify the maximum hole size to fill. This is represented as a radius
 to the bounding circumsphere containing the hole.  Note that this is an
 approximate area; the actual area cannot be computed without first
 triangulating the hole.

\item  \verb|double = obj.GetHoleSizeMaxValue ()| -  Specify the maximum hole size to fill. This is represented as a radius
 to the bounding circumsphere containing the hole.  Note that this is an
 approximate area; the actual area cannot be computed without first
 triangulating the hole.

\item  \verb|double = obj.GetHoleSize ()| -  Specify the maximum hole size to fill. This is represented as a radius
 to the bounding circumsphere containing the hole.  Note that this is an
 approximate area; the actual area cannot be computed without first
 triangulating the hole.

\end{itemize}
