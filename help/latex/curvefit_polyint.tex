\section{POLYINT Polynomial Coefficient Integration}

\subsection{Usage}

 The polyint function returns the polynomial coefficients resulting
 from integration of polynomial p. The syntax for its use is either
\begin{verbatim}
 pint = polyint(p,k)
\end{verbatim}
 or, for a default \verb|k = 0|,
\begin{verbatim}
 pint = polyint(p);
\end{verbatim}
 where \verb|p| is a vector of polynomial coefficients assumed to be in
 decreasing degree and \verb|k| is the integration constant.
 Contributed by Paulo Xavier Candeias under GPL
\subsection{Example}

Here is are some examples of the use of \verb|polyint|.
\begin{verbatim}
--> polyint([2,3,4])

ans = 
    0.6667    1.5000    4.0000         0 
\end{verbatim}
And
\begin{verbatim}
--> polyint([2,3,4],5)

ans = 
    0.6667    1.5000    4.0000    5.0000 
\end{verbatim}
