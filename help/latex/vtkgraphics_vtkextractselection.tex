\section{vtkExtractSelection}

\subsection{Usage}

 vtkExtractSelection extracts some subset of cells and points from
 its input dataset. The dataset is given on its first input port. 
 The subset is described by the contents of the vtkSelection on its 
 second input port. Depending on the content of the vtkSelection,
 this will use either a vtkExtractSelectedIds, vtkExtractSelectedFrustum
 vtkExtractSelectedLocations or a vtkExtractSelectedThreshold to perform
 the extraction.

To create an instance of class vtkExtractSelection, simply
invoke its constructor as follows
\begin{verbatim}
  obj = vtkExtractSelection
\end{verbatim}
\subsection{Methods}

The class vtkExtractSelection has several methods that can be used.
  They are listed below.
Note that the documentation is translated automatically from the VTK sources,
and may not be completely intelligible.  When in doubt, consult the VTK website.
In the methods listed below, \verb|obj| is an instance of the vtkExtractSelection class.
\begin{itemize}
\item  \verb|string = obj.GetClassName ()|

\item  \verb|int = obj.IsA (string name)|

\item  \verb|vtkExtractSelection = obj.NewInstance ()|

\item  \verb|vtkExtractSelection = obj.SafeDownCast (vtkObject o)|

\item  \verb|obj.SetShowBounds (int )| -  When On, this returns an unstructured grid that outlines selection area.
 Off is the default. Applicable only to Frustum selection extraction.

\item  \verb|int = obj.GetShowBounds ()| -  When On, this returns an unstructured grid that outlines selection area.
 Off is the default. Applicable only to Frustum selection extraction.

\item  \verb|obj.ShowBoundsOn ()| -  When On, this returns an unstructured grid that outlines selection area.
 Off is the default. Applicable only to Frustum selection extraction.

\item  \verb|obj.ShowBoundsOff ()| -  When On, this returns an unstructured grid that outlines selection area.
 Off is the default. Applicable only to Frustum selection extraction.

\item  \verb|obj.SetUseProbeForLocations (int )| -  When On, vtkProbeSelectedLocations is used for extracting selections of
 content type vtkSelection::LOCATIONS. Default is off and then
 vtkExtractSelectedLocations is used.

\item  \verb|int = obj.GetUseProbeForLocations ()| -  When On, vtkProbeSelectedLocations is used for extracting selections of
 content type vtkSelection::LOCATIONS. Default is off and then
 vtkExtractSelectedLocations is used.

\item  \verb|obj.UseProbeForLocationsOn ()| -  When On, vtkProbeSelectedLocations is used for extracting selections of
 content type vtkSelection::LOCATIONS. Default is off and then
 vtkExtractSelectedLocations is used.

\item  \verb|obj.UseProbeForLocationsOff ()| -  When On, vtkProbeSelectedLocations is used for extracting selections of
 content type vtkSelection::LOCATIONS. Default is off and then
 vtkExtractSelectedLocations is used.

\end{itemize}
