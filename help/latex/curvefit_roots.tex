\section{ROOTS Find Roots of Polynomial}

\subsection{Usage}

The \verb|roots| routine will return a column vector containing the
roots of a polynomial.  The general syntax is
\begin{verbatim}
   z = roots(p)
\end{verbatim}
where \verb|p| is a vector containing the coefficients of the polynomial
ordered in descending powers.  
\subsection{Function Internals}

Given a vector 
\[
   [p_1, p_2, \dots p_n]
\]
which describes a polynomial
\[
   p_1 x^{n-1} + p_2 x^{n-2} + \dots + p_n
\]
we construct the companion matrix (which has a characteristic polynomial
matching the polynomial described by \verb|p|), and then find the eigenvalues
of it (which are the roots of its characteristic polynomial), and
which are also the roots of the polynomial of interest.  This technique
for finding the roots is described in the help page for \verb|roots| on the Mathworks
website.
\subsection{Example}

Here is an example of finding the roots to the polynomial
\[
   x^3 - 6x^2 - 72x - 27
\]
\begin{verbatim}
--> roots([1 -6 -72 -27])

ans = 
   12.1229 
   -5.7345 
   -0.3884 
\end{verbatim}
