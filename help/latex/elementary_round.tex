\section{ROUND Round Function}

\subsection{Usage}

Rounds an n-dimensional array to the nearest integer elementwise.
The general syntax for its use is
\begin{verbatim}
   y = round(x)
\end{verbatim}
where \verb|x| is a multidimensional array of numerical type.  The \verb|round| 
function preserves the type of the argument.  So integer arguments 
are not modified, and \verb|float| arrays return \verb|float| arrays as 
outputs, and similarly for \verb|double| arrays.  The \verb|round| function 
is not defined for \verb|complex| or \verb|dcomplex| types.
\subsection{Example}

The following demonstrates the \verb|round| function applied to various
(numerical) arguments.  For integer arguments, the round function has
no effect:
\begin{verbatim}
--> round(3)

ans = 
 3 

--> round(-3)

ans = 
 -3 
\end{verbatim}
Next, we take the \verb|round| of a floating point value:
\begin{verbatim}
--> round(3.023f)

ans = 
 3 

--> round(-2.341f)

ans = 
 -2 
\end{verbatim}
Note that the return type is a \verb|float| also.  Finally, for a \verb|double|
type:
\begin{verbatim}
--> round(4.312)

ans = 
 4 

--> round(-5.32)

ans = 
 -5 
\end{verbatim}
