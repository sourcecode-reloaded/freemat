\section{vtkCompositeDataGeometryFilter}

\subsection{Usage}

 vtkCompositeDataGeometryFilter applies vtkGeometryFilter to all
 leaves in vtkCompositeDataSet. Place this filter at the end of a
 pipeline before a polydata consumer such as a polydata mapper to extract
 geometry from all blocks and append them to one polydata object.

To create an instance of class vtkCompositeDataGeometryFilter, simply
invoke its constructor as follows
\begin{verbatim}
  obj = vtkCompositeDataGeometryFilter
\end{verbatim}
\subsection{Methods}

The class vtkCompositeDataGeometryFilter has several methods that can be used.
  They are listed below.
Note that the documentation is translated automatically from the VTK sources,
and may not be completely intelligible.  When in doubt, consult the VTK website.
In the methods listed below, \verb|obj| is an instance of the vtkCompositeDataGeometryFilter class.
\begin{itemize}
\item  \verb|string = obj.GetClassName ()|

\item  \verb|int = obj.IsA (string name)|

\item  \verb|vtkCompositeDataGeometryFilter = obj.NewInstance ()|

\item  \verb|vtkCompositeDataGeometryFilter = obj.SafeDownCast (vtkObject o)|

\end{itemize}
