\section{vtkCachingInterpolatedVelocityField}

\subsection{Usage}

 vtkCachingInterpolatedVelocityField acts as a continuous velocity field
 by performing cell interpolation on the underlying vtkDataSet.
 This is a concrete sub-class of vtkFunctionSet with 
 NumberOfIndependentVariables = 4 (x,y,z,t) and 
 NumberOfFunctions = 3 (u,v,w). Normally, every time an evaluation
 is performed, the cell which contains the point (x,y,z) has to
 be found by calling FindCell. This is a computationally expensive 
 operation. In certain cases, the cell search can be avoided or shortened 
 by providing a guess for the cell id. For example, in streamline
 integration, the next evaluation is usually in the same or a neighbour
 cell. For this reason, vtkCachingInterpolatedVelocityField stores the last
 cell id. If caching is turned on, it uses this id as the starting point.

To create an instance of class vtkCachingInterpolatedVelocityField, simply
invoke its constructor as follows
\begin{verbatim}
  obj = vtkCachingInterpolatedVelocityField
\end{verbatim}
\subsection{Methods}

The class vtkCachingInterpolatedVelocityField has several methods that can be used.
  They are listed below.
Note that the documentation is translated automatically from the VTK sources,
and may not be completely intelligible.  When in doubt, consult the VTK website.
In the methods listed below, \verb|obj| is an instance of the vtkCachingInterpolatedVelocityField class.
\begin{itemize}
\item  \verb|string = obj.GetClassName ()|

\item  \verb|int = obj.IsA (string name)|

\item  \verb|vtkCachingInterpolatedVelocityField = obj.NewInstance ()|

\item  \verb|vtkCachingInterpolatedVelocityField = obj.SafeDownCast (vtkObject o)|

\item  \verb|int = obj.FunctionValues (double x, double f)| -  Evaluate the velocity field, f={u,v,w}, at {x, y, z}.
 returns 1 if valid, 0 if test failed

\item  \verb|int = obj.InsideTest (double x)| -  Evaluate the velocity field, f={u,v,w}, at {x, y, z}.
 returns 1 if valid, 0 if test failed

\item  \verb|obj.SetDataSet (int I, vtkDataSet dataset, bool staticdataset, vtkAbstractCellLocator locator)| -  Add a dataset used by the interpolation function evaluation.

\item  \verb|string = obj.GetVectorsSelection ()| -  If you want to work with an arbitrary vector array, then set its name 
 here. By default this in NULL and the filter will use the active vector 
 array.

\item  \verb|obj.SelectVectors (string fieldName)| -  Return the cell id cached from last evaluation.

\item  \verb|obj.SetLastCellInfo (vtkIdType c, int datasetindex)| -  Return the cell id cached from last evaluation.

\item  \verb|obj.ClearLastCellInfo ()| -  Set the last cell id to -1 so that the next search does not
 start from the previous cell

\item  \verb|int = obj.GetLastWeights (double w)| -  Returns the interpolation weights/pcoords cached from last evaluation
 if the cached cell is valid (returns 1). Otherwise, it does not
 change w and returns 0.

\item  \verb|int = obj.GetLastLocalCoordinates (double pcoords[3])| -  Returns the interpolation weights/pcoords cached from last evaluation
 if the cached cell is valid (returns 1). Otherwise, it does not
 change w and returns 0.

\item  \verb|int = obj.GetCellCacheHit ()| -  Caching statistics.

\item  \verb|int = obj.GetDataSetCacheHit ()| -  Caching statistics.

\item  \verb|int = obj.GetCacheMiss ()| -  Caching statistics.

\end{itemize}
