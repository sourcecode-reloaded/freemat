\section{vtkSurfaceReconstructionFilter}

\subsection{Usage}

 vtkSurfaceReconstructionFilter takes a list of points assumed to lie on
 the surface of a solid 3D object. A signed measure of the distance to the
 surface is computed and sampled on a regular grid. The grid can then be
 contoured at zero to extract the surface. The default values for
 neighborhood size and sample spacing should give reasonable results for
 most uses but can be set if desired. This procedure is based on the PhD
 work of Hugues Hoppe: http://www.research.microsoft.com/~hoppe

To create an instance of class vtkSurfaceReconstructionFilter, simply
invoke its constructor as follows
\begin{verbatim}
  obj = vtkSurfaceReconstructionFilter
\end{verbatim}
\subsection{Methods}

The class vtkSurfaceReconstructionFilter has several methods that can be used.
  They are listed below.
Note that the documentation is translated automatically from the VTK sources,
and may not be completely intelligible.  When in doubt, consult the VTK website.
In the methods listed below, \verb|obj| is an instance of the vtkSurfaceReconstructionFilter class.
\begin{itemize}
\item  \verb|string = obj.GetClassName ()|

\item  \verb|int = obj.IsA (string name)|

\item  \verb|vtkSurfaceReconstructionFilter = obj.NewInstance ()|

\item  \verb|vtkSurfaceReconstructionFilter = obj.SafeDownCast (vtkObject o)|

\item  \verb|int = obj.GetNeighborhoodSize ()| -  Specify the number of neighbors each point has, used for estimating the
 local surface orientation.  The default value of 20 should be OK for
 most applications, higher values can be specified if the spread of
 points is uneven. Values as low as 10 may yield adequate results for
 some surfaces. Higher values cause the algorithm to take longer. Higher
 values will cause errors on sharp boundaries.

\item  \verb|obj.SetNeighborhoodSize (int )| -  Specify the number of neighbors each point has, used for estimating the
 local surface orientation.  The default value of 20 should be OK for
 most applications, higher values can be specified if the spread of
 points is uneven. Values as low as 10 may yield adequate results for
 some surfaces. Higher values cause the algorithm to take longer. Higher
 values will cause errors on sharp boundaries.

\item  \verb|double = obj.GetSampleSpacing ()| -  Specify the spacing of the 3D sampling grid. If not set, a
 reasonable guess will be made.

\item  \verb|obj.SetSampleSpacing (double )| -  Specify the spacing of the 3D sampling grid. If not set, a
 reasonable guess will be made.

\end{itemize}
