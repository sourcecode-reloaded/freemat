\section{vtkQuadraticPyramid}

\subsection{Usage}

 vtkQuadraticPyramid is a concrete implementation of vtkNonLinearCell to
 represent a three-dimensional, 13-node isoparametric parabolic
 pyramid. The interpolation is the standard finite element, quadratic
 isoparametric shape function. The cell includes a mid-edge node. The
 ordering of the thirteen points defining the cell is point ids (0-4,5-12)
 where point ids 0-4 are the five corner vertices of the pyramid; followed
 by eight midedge nodes (5-12). Note that these midedge nodes correspond lie
 on the edges defined by (0,1), (1,2), (2,3), (3,0), (0,4), (1,4), (2,4),
 (3,4).

To create an instance of class vtkQuadraticPyramid, simply
invoke its constructor as follows
\begin{verbatim}
  obj = vtkQuadraticPyramid
\end{verbatim}
\subsection{Methods}

The class vtkQuadraticPyramid has several methods that can be used.
  They are listed below.
Note that the documentation is translated automatically from the VTK sources,
and may not be completely intelligible.  When in doubt, consult the VTK website.
In the methods listed below, \verb|obj| is an instance of the vtkQuadraticPyramid class.
\begin{itemize}
\item  \verb|string = obj.GetClassName ()|

\item  \verb|int = obj.IsA (string name)|

\item  \verb|vtkQuadraticPyramid = obj.NewInstance ()|

\item  \verb|vtkQuadraticPyramid = obj.SafeDownCast (vtkObject o)|

\item  \verb|int = obj.GetCellType ()| -  Implement the vtkCell API. See the vtkCell API for descriptions
 of these methods.

\item  \verb|int = obj.GetCellDimension ()| -  Implement the vtkCell API. See the vtkCell API for descriptions
 of these methods.

\item  \verb|int = obj.GetNumberOfEdges ()| -  Implement the vtkCell API. See the vtkCell API for descriptions
 of these methods.

\item  \verb|int = obj.GetNumberOfFaces ()| -  Implement the vtkCell API. See the vtkCell API for descriptions
 of these methods.

\item  \verb|vtkCell = obj.GetEdge (int edgeId)| -  Implement the vtkCell API. See the vtkCell API for descriptions
 of these methods.

\item  \verb|vtkCell = obj.GetFace (int faceId)| -  Implement the vtkCell API. See the vtkCell API for descriptions
 of these methods.

\item  \verb|int = obj.CellBoundary (int subId, double pcoords[3], vtkIdList pts)|

\item  \verb|obj.Contour (double value, vtkDataArray cellScalars, vtkIncrementalPointLocator locator, vtkCellArray verts, vtkCellArray lines, vtkCellArray polys, vtkPointData inPd, vtkPointData outPd, vtkCellData inCd, vtkIdType cellId, vtkCellData outCd)|

\item  \verb|int = obj.Triangulate (int index, vtkIdList ptIds, vtkPoints pts)|

\item  \verb|obj.Derivatives (int subId, double pcoords[3], double values, int dim, double derivs)|

\item  \verb|obj.Clip (double value, vtkDataArray cellScalars, vtkIncrementalPointLocator locator, vtkCellArray tets, vtkPointData inPd, vtkPointData outPd, vtkCellData inCd, vtkIdType cellId, vtkCellData outCd, int insideOut)| -  Clip this quadratic triangle using scalar value provided. Like
 contouring, except that it cuts the triangle to produce linear
 triangles.

\item  \verb|int = obj.GetParametricCenter (double pcoords[3])| -  Return the center of the quadratic pyramid in parametric coordinates.

\item  \verb|obj.InterpolateFunctions (double pcoords[3], double weights[13])| -  Compute the interpolation functions/derivatives
 (aka shape functions/derivatives)

\item  \verb|obj.InterpolateDerivs (double pcoords[3], double derivs[39])| -  Return the ids of the vertices defining edge/face (`edgeId`/`faceId').
 Ids are related to the cell, not to the dataset.

\end{itemize}
