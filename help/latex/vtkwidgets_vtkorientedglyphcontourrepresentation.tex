\section{vtkOrientedGlyphContourRepresentation}

\subsection{Usage}

 This class provides the default concrete representation for the 
 vtkContourWidget. It works in conjunction with the 
 vtkContourLineInterpolator and vtkPointPlacer. See vtkContourWidget
 for details.

To create an instance of class vtkOrientedGlyphContourRepresentation, simply
invoke its constructor as follows
\begin{verbatim}
  obj = vtkOrientedGlyphContourRepresentation
\end{verbatim}
\subsection{Methods}

The class vtkOrientedGlyphContourRepresentation has several methods that can be used.
  They are listed below.
Note that the documentation is translated automatically from the VTK sources,
and may not be completely intelligible.  When in doubt, consult the VTK website.
In the methods listed below, \verb|obj| is an instance of the vtkOrientedGlyphContourRepresentation class.
\begin{itemize}
\item  \verb|string = obj.GetClassName ()| -  Standard methods for instances of this class.

\item  \verb|int = obj.IsA (string name)| -  Standard methods for instances of this class.

\item  \verb|vtkOrientedGlyphContourRepresentation = obj.NewInstance ()| -  Standard methods for instances of this class.

\item  \verb|vtkOrientedGlyphContourRepresentation = obj.SafeDownCast (vtkObject o)| -  Standard methods for instances of this class.

\item  \verb|obj.SetCursorShape (vtkPolyData cursorShape)| -  Specify the cursor shape. Keep in mind that the shape will be
 aligned with the  constraining plane by orienting it such that
 the x axis of the geometry lies along the normal of the plane.

\item  \verb|vtkPolyData = obj.GetCursorShape ()| -  Specify the cursor shape. Keep in mind that the shape will be
 aligned with the  constraining plane by orienting it such that
 the x axis of the geometry lies along the normal of the plane.

\item  \verb|obj.SetActiveCursorShape (vtkPolyData activeShape)| -  Specify the shape of the cursor (handle) when it is active.
 This is the geometry that will be used when the mouse is
 close to the handle or if the user is manipulating the handle.

\item  \verb|vtkPolyData = obj.GetActiveCursorShape ()| -  Specify the shape of the cursor (handle) when it is active.
 This is the geometry that will be used when the mouse is
 close to the handle or if the user is manipulating the handle.

\item  \verb|vtkProperty = obj.GetProperty ()| -  This is the property used when the handle is not active 
 (the mouse is not near the handle)

\item  \verb|vtkProperty = obj.GetActiveProperty ()| -  This is the property used when the user is interacting
 with the handle.

\item  \verb|vtkProperty = obj.GetLinesProperty ()| -  This is the property used by the lines.

\item  \verb|obj.SetRenderer (vtkRenderer ren)| -  Subclasses of vtkOrientedGlyphContourRepresentation must implement these methods. These
 are the methods that the widget and its representation use to
 communicate with each other.

\item  \verb|obj.BuildRepresentation ()| -  Subclasses of vtkOrientedGlyphContourRepresentation must implement these methods. These
 are the methods that the widget and its representation use to
 communicate with each other.

\item  \verb|obj.StartWidgetInteraction (double eventPos[2])| -  Subclasses of vtkOrientedGlyphContourRepresentation must implement these methods. These
 are the methods that the widget and its representation use to
 communicate with each other.

\item  \verb|obj.WidgetInteraction (double eventPos[2])| -  Subclasses of vtkOrientedGlyphContourRepresentation must implement these methods. These
 are the methods that the widget and its representation use to
 communicate with each other.

\item  \verb|int = obj.ComputeInteractionState (int X, int Y, int modified)| -  Subclasses of vtkOrientedGlyphContourRepresentation must implement these methods. These
 are the methods that the widget and its representation use to
 communicate with each other.

\item  \verb|obj.GetActors (vtkPropCollection )| -  Methods to make this class behave as a vtkProp.

\item  \verb|obj.ReleaseGraphicsResources (vtkWindow )| -  Methods to make this class behave as a vtkProp.

\item  \verb|int = obj.RenderOverlay (vtkViewport viewport)| -  Methods to make this class behave as a vtkProp.

\item  \verb|int = obj.RenderOpaqueGeometry (vtkViewport viewport)| -  Methods to make this class behave as a vtkProp.

\item  \verb|int = obj.RenderTranslucentPolygonalGeometry (vtkViewport viewport)| -  Methods to make this class behave as a vtkProp.

\item  \verb|int = obj.HasTranslucentPolygonalGeometry ()| -  Methods to make this class behave as a vtkProp.

\item  \verb|vtkPolyData = obj.GetContourRepresentationAsPolyData ()| -  Get the points in this contour as a vtkPolyData. 

\item  \verb|obj.SetAlwaysOnTop (int )| -  Controls whether the contour widget should always appear on top
 of other actors in the scene. (In effect, this will disable OpenGL
 Depth buffer tests while rendering the contour).
 Default is to set it to false.

\item  \verb|int = obj.GetAlwaysOnTop ()| -  Controls whether the contour widget should always appear on top
 of other actors in the scene. (In effect, this will disable OpenGL
 Depth buffer tests while rendering the contour).
 Default is to set it to false.

\item  \verb|obj.AlwaysOnTopOn ()| -  Controls whether the contour widget should always appear on top
 of other actors in the scene. (In effect, this will disable OpenGL
 Depth buffer tests while rendering the contour).
 Default is to set it to false.

\item  \verb|obj.AlwaysOnTopOff ()| -  Controls whether the contour widget should always appear on top
 of other actors in the scene. (In effect, this will disable OpenGL
 Depth buffer tests while rendering the contour).
 Default is to set it to false.

\item  \verb|obj.SetLineColor (double r, double g, double b)| -  Convenience method to set the line color.
 Ideally one should use GetLinesProperty()->SetColor().

\item  \verb|obj.SetShowSelectedNodes (int )| -  A flag to indicate whether to show the Selected nodes
 Default is to set it to false.

\end{itemize}
