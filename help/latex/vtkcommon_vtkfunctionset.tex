\section{vtkFunctionSet}

\subsection{Usage}

 vtkFunctionSet specifies an abstract interface for set of functions
 of the form F\_i = F\_i(x\_j) where F (with i=1..m) are the functions
 and x (with j=1..n) are the independent variables.
 The only supported operation is the  function evaluation at x\_j.

To create an instance of class vtkFunctionSet, simply
invoke its constructor as follows
\begin{verbatim}
  obj = vtkFunctionSet
\end{verbatim}
\subsection{Methods}

The class vtkFunctionSet has several methods that can be used.
  They are listed below.
Note that the documentation is translated automatically from the VTK sources,
and may not be completely intelligible.  When in doubt, consult the VTK website.
In the methods listed below, \verb|obj| is an instance of the vtkFunctionSet class.
\begin{itemize}
\item  \verb|string = obj.GetClassName ()|

\item  \verb|int = obj.IsA (string name)|

\item  \verb|vtkFunctionSet = obj.NewInstance ()|

\item  \verb|vtkFunctionSet = obj.SafeDownCast (vtkObject o)|

\item  \verb|int = obj.FunctionValues (double x, double f)| -  Evaluate functions at x\_j.
 x and f have to point to valid double arrays of appropriate 
 sizes obtained with GetNumberOfFunctions() and
 GetNumberOfIndependentVariables.

\item  \verb|int = obj.GetNumberOfFunctions ()| -  Return the number of independent variables. Note that this is 
 constant for a given type of set of functions and can not be changed  
 at run time.

\item  \verb|int = obj.GetNumberOfIndependentVariables ()|

\end{itemize}
