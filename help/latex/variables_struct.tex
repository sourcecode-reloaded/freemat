\section{STRUCT Structure Array Constructor}

\subsection{Usage}

Creates an array of structures from a set of field, value pairs.
The syntax is
\begin{verbatim}
   y = struct(n_1,v_1,n_2,v_2,...)
\end{verbatim}
where \verb|n\_i| are the names of the fields in the structure array, and
\verb|v\_i| are the values.  The values \verb|v\_i| must either all be
scalars, or be cell-arrays of all the same dimensions.  In the latter 
case, the
output structure array will have dimensions dictated by this common
size.  Scalar entries for the \verb|v\_i| are replicated to fill out
their dimensions. An error is raised if the inputs are not properly matched (i.e., are
not pairs of field names and values), or if the size of any two non-scalar
values cell-arrays are different.

Another use of the \verb|struct| function is to convert a class into a 
structure.  This allows you to access the members of the class, directly 
but removes the class information from the object.

\subsection{Example}

This example creates a 3-element structure array with three fields, \verb|foo|
\verb|bar| and \verb|key|, where the contents of \verb|foo| and \verb|bar| are provided 
explicitly as cell arrays of the same size, and the contents of \verb|bar| 
are replicated from a scalar.
\begin{verbatim}
--> y = struct('foo',{1,3,4},'bar',{'cheese','cola','beer'},'key',508)

y = 
1x3 struct array with fields:
    foo
    bar
    key
--> y(1)

ans = 
    foo: 1
    bar: cheese
    key: 508
--> y(2)

ans = 
    foo: 3
    bar: cola
    key: 508
--> y(3)

ans = 
    foo: 4
    bar: beer
    key: 508
\end{verbatim}

An alternate way to create a structure array is to initialize the last
element of each field of the structure
\begin{verbatim}
--> Test(2,3).Type = 'Beer';
--> Test(2,3).Ounces = 12;
--> Test(2,3).Container = 'Can';
--> Test(2,3)

ans = 
    Type: Beer
    Ounces: 12
    Container: Can
--> Test(1,1)

ans = 
    Type: 0
    Ounces: 0
    Container: 0
\end{verbatim}
