\section{vtkFrameBufferObject}

\subsection{Usage}

 Encapsulates an OpenGL Frame Buffer Object.
 For use by vtkOpenGLFBORenderWindow, not to be used directly.

To create an instance of class vtkFrameBufferObject, simply
invoke its constructor as follows
\begin{verbatim}
  obj = vtkFrameBufferObject
\end{verbatim}
\subsection{Methods}

The class vtkFrameBufferObject has several methods that can be used.
  They are listed below.
Note that the documentation is translated automatically from the VTK sources,
and may not be completely intelligible.  When in doubt, consult the VTK website.
In the methods listed below, \verb|obj| is an instance of the vtkFrameBufferObject class.
\begin{itemize}
\item  \verb|string = obj.GetClassName ()|

\item  \verb|int = obj.IsA (string name)|

\item  \verb|vtkFrameBufferObject = obj.NewInstance ()|

\item  \verb|vtkFrameBufferObject = obj.SafeDownCast (vtkObject o)|

\item  \verb|obj.SetContext (vtkRenderWindow context)| -  Get/Set the context. Context must be a vtkOpenGLRenderWindow. 
 This does not increase the reference count of the 
 context to avoid reference loops.
 SetContext() may raise an error is the OpenGL context does not support the
 required OpenGL extensions.

\item  \verb|vtkRenderWindow = obj.GetContext ()| -  Get/Set the context. Context must be a vtkOpenGLRenderWindow. 
 This does not increase the reference count of the 
 context to avoid reference loops.
 SetContext() may raise an error is the OpenGL context does not support the
 required OpenGL extensions.

\item  \verb|bool = obj.Start (int width, int height, bool shaderSupportsTextureInt)| -  User must take care that width/height match the dimensions of 
 the user defined texture attachments.
 This method makes the ''active buffers'' the buffers that will get drawn
 into by subsequent drawing calls.
 Note that this does not clear the render buffers i.e. no glClear() calls
 are made by either of these methods. It's up to the caller to clear the
 buffers if needed.

\item  \verb|bool = obj.StartNonOrtho (int width, int height, bool shaderSupportsTextureInt)| -  User must take care that width/height match the dimensions of 
 the user defined texture attachments.
 This method makes the ''active buffers'' the buffers that will get drawn
 into by subsequent drawing calls.
 Note that this does not clear the render buffers i.e. no glClear() calls
 are made by either of these methods. It's up to the caller to clear the
 buffers if needed.

\item  \verb|obj.RenderQuad (int minX, int maxX, int minY, int maxY)| -  Renders a quad at the given location with pixel coordinates. This method
 is provided as a convenience, since we often render quads in a FBO.
 
 
 
 
 
 

\item  \verb|obj.Bind ()| -  Save the current framebuffer and make the frame buffer active.
 Multiple calls to Bind has no effect.

\item  \verb|obj.UnBind ()| -  Restore the framebuffer saved with the call to Bind().
 Multiple calls to UnBind has no effect.

\item  \verb|obj.SetActiveBuffer (int index)| -  Choose the buffer to render into.
 This is available only if the GL\_ARB\_draw\_buffers extension is supported
 by the card.

\item  \verb|obj.SetActiveBuffers (int numbuffers, int indices[])| -  Choose the buffer to render into.
 This is available only if the GL\_ARB\_draw\_buffers extension is supported
 by the card.

\item  \verb|obj.SetColorBuffer (int index, vtkTextureObject texture, int zslice)|

\item  \verb|vtkTextureObject = obj.GetColorBuffer (int index)|

\item  \verb|obj.RemoveColorBuffer (int index)|

\item  \verb|obj.RemoveAllColorBuffers ()|

\item  \verb|obj.SetDepthBuffer (vtkTextureObject depthTexture)| -  Set the texture to use as depth buffer.

\item  \verb|obj.RemoveDepthBuffer ()| -  Set the texture to use as depth buffer.

\item  \verb|obj.SetDepthBufferNeeded (bool )| -  If true, the frame buffer object will be initialized with a depth buffer.
 Initial value is true.

\item  \verb|bool = obj.GetDepthBufferNeeded ()| -  If true, the frame buffer object will be initialized with a depth buffer.
 Initial value is true.

\item  \verb|obj.SetNumberOfRenderTargets (int )| -  Set/Get the number of render targets to render into at once.

\item  \verb|int = obj.GetNumberOfRenderTargets ()| -  Set/Get the number of render targets to render into at once.

\item  \verb|int = obj.GetMaximumNumberOfActiveTargets ()| -  Returns the maximum number of targets that can be rendered to at one time.
 This limits the active targets set by SetActiveTargets().
 The return value is valid only if GetContext is non-null.

\item  \verb|int = obj.GetMaximumNumberOfRenderTargets ()| -  Returns the maximum number of render targets available. This limits the
 available attachement points for SetColorAttachment().
 The return value is valid only if GetContext is non-null.

\item  \verb|int = obj. GetLastSize ()| -  Dimensions in pixels of the framebuffer.

\end{itemize}
