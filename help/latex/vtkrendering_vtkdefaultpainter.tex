\section{vtkDefaultPainter}

\subsection{Usage}

 This painter does not do any actual rendering.
 Sets up a default pipeline of painters to mimick the behaiour of 
 old vtkPolyDataMapper. The chain is as follows:
 input--> vtkScalarsToColorsPainter --> vtkClipPlanesPainter -->
 vtkDisplayListPainter --> vtkCompositePainter -->
 vtkCoincidentTopologyResolutionPainter -->
 vtkLightingPainter --> vtkRepresentationPainter --> 
 Delegate of vtkDefaultPainter.
 Typically, the delegate of the default painter be one that is capable of r
 rendering graphics primitives or a vtkChooserPainter which can select appropriate
 painters to do the rendering.

To create an instance of class vtkDefaultPainter, simply
invoke its constructor as follows
\begin{verbatim}
  obj = vtkDefaultPainter
\end{verbatim}
\subsection{Methods}

The class vtkDefaultPainter has several methods that can be used.
  They are listed below.
Note that the documentation is translated automatically from the VTK sources,
and may not be completely intelligible.  When in doubt, consult the VTK website.
In the methods listed below, \verb|obj| is an instance of the vtkDefaultPainter class.
\begin{itemize}
\item  \verb|string = obj.GetClassName ()|

\item  \verb|int = obj.IsA (string name)|

\item  \verb|vtkDefaultPainter = obj.NewInstance ()|

\item  \verb|vtkDefaultPainter = obj.SafeDownCast (vtkObject o)|

\item  \verb|obj.SetScalarsToColorsPainter (vtkScalarsToColorsPainter )| -  Get/Set the painter that maps scalars to colors.

\item  \verb|vtkScalarsToColorsPainter = obj.GetScalarsToColorsPainter ()| -  Get/Set the painter that maps scalars to colors.

\item  \verb|obj.SetClipPlanesPainter (vtkClipPlanesPainter )| -  Get/Set the painter that handles clipping.

\item  \verb|vtkClipPlanesPainter = obj.GetClipPlanesPainter ()| -  Get/Set the painter that handles clipping.

\item  \verb|obj.SetDisplayListPainter (vtkDisplayListPainter )| -  Get/Set the painter that builds display lists.

\item  \verb|vtkDisplayListPainter = obj.GetDisplayListPainter ()| -  Get/Set the painter that builds display lists.

\item  \verb|obj.SetCompositePainter (vtkCompositePainter )| -  Get/Set the painter used to handle composite datasets.

\item  \verb|vtkCompositePainter = obj.GetCompositePainter ()| -  Get/Set the painter used to handle composite datasets.

\item  \verb|obj.SetCoincidentTopologyResolutionPainter (vtkCoincidentTopologyResolutionPainter )| -  Painter used to resolve coincident topology.

\item  \verb|vtkCoincidentTopologyResolutionPainter = obj.GetCoincidentTopologyResolutionPainter ()| -  Painter used to resolve coincident topology.

\item  \verb|obj.SetLightingPainter (vtkLightingPainter )| -  Get/Set the painter that controls lighting.

\item  \verb|vtkLightingPainter = obj.GetLightingPainter ()| -  Get/Set the painter that controls lighting.

\item  \verb|obj.SetRepresentationPainter (vtkRepresentationPainter )| -  Painter used to convert polydata to Wireframe/Points representation.

\item  \verb|vtkRepresentationPainter = obj.GetRepresentationPainter ()| -  Painter used to convert polydata to Wireframe/Points representation.

\item  \verb|obj.SetDelegatePainter (vtkPainter )| -  Set/Get the painter to which this painter should propagare its draw calls.
 These methods are overridden so that the delegate is set
 to the end of the Painter Chain.

\item  \verb|vtkPainter = obj.GetDelegatePainter ()| -  Overridden to setup the chain of painter depending on the 
 actor representation. The chain is rebuilt if
 this->MTime has changed 
 since last BuildPainterChain();
 Building of the chain does not depend on input polydata,
 hence it does not check if the input has changed at all.

\item  \verb|obj.Render (vtkRenderer renderer, vtkActor actor, long typeflags, bool forceCompileOnly)| -  Overridden to setup the chain of painter depending on the 
 actor representation. The chain is rebuilt if
 this->MTime has changed 
 since last BuildPainterChain();
 Building of the chain does not depend on input polydata,
 hence it does not check if the input has changed at all.

\item  \verb|obj.ReleaseGraphicsResources (vtkWindow )| -  Release any graphics resources that are being consumed by this painter.
 The parameter window could be used to determine which graphic
 resources to release. 
 The call is propagated to the delegate painter, if any.

\item  \verb|obj.UpdateBounds (double bounds[6])| -  Expand or shrink the estimated bounds based on the geometric
 transformations applied in the painter. The bounds are left unchanged
 if the painter does not change the geometry.

\end{itemize}
