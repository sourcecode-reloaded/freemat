\section{vtkExtractArraysOverTime}

\subsection{Usage}

 vtkExtractArraysOverTime extracts a selection over time.
 The output is a multiblock dataset. If selection content type is  
 vtkSelection::Locations, then each output block corresponds to each probed
 location. Otherwise, each output block corresponds to an extracted cell/point
 depending on whether the selection field type is CELL or POINT.
 Each block is a vtkTable with a column named Time (or TimeData if Time exists
 in the input).
 When extracting point data, the input point coordinates are copied
 to a column named Point Coordinates or Points (if Point Coordinates
 exists in the input).
 This algorithm does not produce a TIME\_STEPS or TIME\_RANGE information
 because it works across time. 
 .Section Caveat
 This algorithm works only with source that produce TIME\_STEPS().
 Continuous time range is not yet supported.

To create an instance of class vtkExtractArraysOverTime, simply
invoke its constructor as follows
\begin{verbatim}
  obj = vtkExtractArraysOverTime
\end{verbatim}
\subsection{Methods}

The class vtkExtractArraysOverTime has several methods that can be used.
  They are listed below.
Note that the documentation is translated automatically from the VTK sources,
and may not be completely intelligible.  When in doubt, consult the VTK website.
In the methods listed below, \verb|obj| is an instance of the vtkExtractArraysOverTime class.
\begin{itemize}
\item  \verb|string = obj.GetClassName ()|

\item  \verb|int = obj.IsA (string name)|

\item  \verb|vtkExtractArraysOverTime = obj.NewInstance ()|

\item  \verb|vtkExtractArraysOverTime = obj.SafeDownCast (vtkObject o)|

\item  \verb|int = obj.GetNumberOfTimeSteps ()| -  Get the number of time steps

\item  \verb|obj.SetSelectionConnection (vtkAlgorithmOutput algOutput)|

\end{itemize}
