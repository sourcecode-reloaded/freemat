\section{vtkTesting}

\subsection{Usage}


  This is a VTK regression testing framework. Looks like this:
 
  vtkTesting* t = vtkTesting::New();

  Two options for setting arguments
  
  Option 1:
  for ( cc = 1; cc < argc; cc ++ )
    {
    t->AddArgument(argv[cc]);
    }

  Option 2:
  t->AddArgument(''-D'');
  t->AddArgument(my\_data\_dir);
  t->AddArgument(''-V'');
  t->AddArgument(my\_valid\_image);

  ...
  
  Two options of doing testing:
  
  Option 1:
  t->SetRenderWindow(renWin);
  int res = t->RegressionTest(threshold);

  Option 2:
  int res = t->RegressionTest(test\_image, threshold);
  
  ...

  if ( res == vtkTesting::PASSED )
    {
    Test passed
    }
  else
    {
    Test failed
    }


To create an instance of class vtkTesting, simply
invoke its constructor as follows
\begin{verbatim}
  obj = vtkTesting
\end{verbatim}
\subsection{Methods}

The class vtkTesting has several methods that can be used.
  They are listed below.
Note that the documentation is translated automatically from the VTK sources,
and may not be completely intelligible.  When in doubt, consult the VTK website.
In the methods listed below, \verb|obj| is an instance of the vtkTesting class.
\begin{itemize}
\item  \verb|string = obj.GetClassName ()|

\item  \verb|int = obj.IsA (string name)|

\item  \verb|vtkTesting = obj.NewInstance ()|

\item  \verb|vtkTesting = obj.SafeDownCast (vtkObject o)|

\item  \verb|obj.SetFrontBuffer (int )| -  Use front buffer for tests. By default use back buffer.

\item  \verb|int = obj.GetFrontBufferMinValue ()| -  Use front buffer for tests. By default use back buffer.

\item  \verb|int = obj.GetFrontBufferMaxValue ()| -  Use front buffer for tests. By default use back buffer.

\item  \verb|obj.FrontBufferOn ()| -  Use front buffer for tests. By default use back buffer.

\item  \verb|obj.FrontBufferOff ()| -  Use front buffer for tests. By default use back buffer.

\item  \verb|int = obj.GetFrontBuffer ()| -  Use front buffer for tests. By default use back buffer.

\item  \verb|int = obj.RegressionTest (double thresh)| -  Perform the test and return result. At the same time the output will be
 written cout

\item  \verb|int = obj.RegressionTest (vtkImageData image, double thresh)| -  Compare the image with the valid image.

\item  \verb|int = obj.CompareAverageOfL2Norm (vtkDataSet pdA, vtkDataSet pdB, double tol)| -  Compute the average L2 norm between all point data data arrays 
 of types float and double present in the data sets ''dsA'' and ''dsB''
 (this includes instances of vtkPoints) Compare the result of 
 each L2 comutation to ''tol''.

\item  \verb|int = obj.CompareAverageOfL2Norm (vtkDataArray daA, vtkDataArray daB, double tol)| -  Compute the average L2 norm between two data arrays ''daA'' and ''daB'' 
 and compare against ''tol''.

\item  \verb|obj.SetRenderWindow (vtkRenderWindow rw)| -  Set and get the render window that will be used for regression testing.

\item  \verb|vtkRenderWindow = obj.GetRenderWindow ()| -  Set and get the render window that will be used for regression testing.

\item  \verb|obj.SetValidImageFileName (string )| -  Set/Get the name of the valid image file

\item  \verb|string = obj.GetValidImageFileName ()| -  Set/Get the name of the valid image file

\item  \verb|double = obj.GetImageDifference ()| -  Get the image difference.

\item  \verb|obj.AddArgument (string argv)| -  Pass the command line arguments into this class to be processed. Many of
 the Get methods such as GetValidImage and GetBaselineRoot rely on the
 arguments to be passed in prior to retrieving these values. Just call
 AddArgument for each argument that was passed into the command line

\item  \verb|obj.CleanArguments ()|

\item  \verb|string = obj.GetDataRoot ()| -  Get some paramters from the command line arguments, env, or defaults

\item  \verb|obj.SetDataRoot (string )| -  Get some paramters from the command line arguments, env, or defaults

\item  \verb|string = obj.GetTempDirectory ()| -  Get some paramters from the command line arguments, env, or defaults

\item  \verb|obj.SetTempDirectory (string )| -  Get some paramters from the command line arguments, env, or defaults

\item  \verb|int = obj.IsValidImageSpecified ()| -  Is a valid image specified on the command line areguments?

\item  \verb|int = obj.IsInteractiveModeSpecified ()| -  Is the interactive mode specified?

\item  \verb|int = obj.IsFlagSpecified (string flag)| -  Is some arbitrary user flag (''-X'', ''-Z'' etc) specified

\item  \verb|obj.SetBorderOffset (int )| -  Number of pixels added as borders to avoid problems with
 window decorations added by some window managers.

\item  \verb|int = obj.GetBorderOffset ()| -  Number of pixels added as borders to avoid problems with
 window decorations added by some window managers.

\item  \verb|obj.SetVerbose (int )| -  Get/Set verbosity level. A level of 0 is quiet.

\item  \verb|int = obj.GetVerbose ()| -  Get/Set verbosity level. A level of 0 is quiet.

\end{itemize}
