\section{vtkParallelopipedRepresentation}

\subsection{Usage}

 This class provides the default geometrical representation for 
 vtkParallelopipedWidget. As a result of interactions of the widget, this
 representation can take on of the following shapes:
 <p>1) A parallelopiped. (8 handles, 6 faces)
 <p>2) Paralleopiped with a chair depression on any one handle. (A chair
 is a depression on one of the handles that carves inwards so as to allow 
 the user to visualize cuts in the volume). (14 handles, 9 faces).


To create an instance of class vtkParallelopipedRepresentation, simply
invoke its constructor as follows
\begin{verbatim}
  obj = vtkParallelopipedRepresentation
\end{verbatim}
\subsection{Methods}

The class vtkParallelopipedRepresentation has several methods that can be used.
  They are listed below.
Note that the documentation is translated automatically from the VTK sources,
and may not be completely intelligible.  When in doubt, consult the VTK website.
In the methods listed below, \verb|obj| is an instance of the vtkParallelopipedRepresentation class.
\begin{itemize}
\item  \verb|string = obj.GetClassName ()| -  Standard methods for instances of this class.

\item  \verb|int = obj.IsA (string name)| -  Standard methods for instances of this class.

\item  \verb|vtkParallelopipedRepresentation = obj.NewInstance ()| -  Standard methods for instances of this class.

\item  \verb|vtkParallelopipedRepresentation = obj.SafeDownCast (vtkObject o)| -  Standard methods for instances of this class.

\item  \verb|obj.GetActors (vtkPropCollection pc)| -  Methods to satisfy the superclass.

\item  \verb|obj.PlaceWidget (double bounds[6])| -  Place the widget in the scene. You can use either of the two APIs : 
 1) PlaceWidget( double bounds[6] ) 
      Creates a cuboid conforming to the said bounds.
 2) PlaceWidget( double corners[8][3] )
      Creates a parallelopiped with corners specified. The order in 
      which corners are specified must obey the following rule:
      Corner 0 - 1 - 2 - 3 - 0  forms a face
      Corner 4 - 5 - 6 - 7 - 4  forms a face
      Corner 0 - 4 - 5 - 1 - 0  forms a face
      Corner 1 - 5 - 6 - 2 - 1  forms a face
      Corner 2 - 6 - 7 - 3 - 2  forms a face
      Corner 3 - 7 - 4 - 0 - 3  forms a face

\item  \verb|obj.SetInteractionState (int )| -  The interaction state may be set from a widget (e.g., PointWidget) 
 or other object. This controls how the interaction with the
 widget proceeds.

\item  \verb|obj.GetBoundingPlanes (vtkPlaneCollection pc)| -  Get the bounding planes of the object. The first 6 planes will
 be bounding planes of the parallelopiped. If in chair mode, three 
 additional planes will be present. The last three planes will be those
 of the chair. The normals of all the planes will point into the object.

\item  \verb|obj.GetPolyData (vtkPolyData pd)| -  The parallelopiped polydata.

\item  \verb|double = obj.GetBounds ()| -  The parallelopiped polydata.

\item  \verb|obj.SetHandleProperty (vtkProperty )| -  Set/Get the handle properties. 

\item  \verb|obj.SetHoveredHandleProperty (vtkProperty )| -  Set/Get the handle properties. 

\item  \verb|obj.SetSelectedHandleProperty (vtkProperty )| -  Set/Get the handle properties. 

\item  \verb|vtkProperty = obj.GetHandleProperty ()| -  Set/Get the handle properties. 

\item  \verb|vtkProperty = obj.GetHoveredHandleProperty ()| -  Set/Get the handle properties. 

\item  \verb|vtkProperty = obj.GetSelectedHandleProperty ()| -  Set/Get the handle properties. 

\item  \verb|obj.SetHandleRepresentation (vtkHandleRepresentation handle)|

\item  \verb|vtkHandleRepresentation = obj.GetHandleRepresentation (int index)|

\item  \verb|obj.HandlesOn ()| -  Turns the visibility of the handles on/off. Sometimes they may get in
 the way of visualization.

\item  \verb|obj.HandlesOff ()| -  Turns the visibility of the handles on/off. Sometimes they may get in
 the way of visualization.

\item  \verb|vtkProperty = obj.GetFaceProperty ()| -  Get the face properties. When a face is being translated, the face gets
 highlighted with the SelectedFaceProperty.

\item  \verb|vtkProperty = obj.GetSelectedFaceProperty ()| -  Get the face properties. When a face is being translated, the face gets
 highlighted with the SelectedFaceProperty.

\item  \verb|vtkProperty = obj.GetOutlineProperty ()| -  Get the outline properties. These are the properties with which the
 parallelopiped wireframe is rendered.

\item  \verb|vtkProperty = obj.GetSelectedOutlineProperty ()| -  Get the outline properties. These are the properties with which the
 parallelopiped wireframe is rendered.

\item  \verb|obj.BuildRepresentation ()| -  This actually constructs the geometry of the widget from the various
 data parameters.

\item  \verb|obj.ReleaseGraphicsResources (vtkWindow w)| -  Methods required by vtkProp superclass.

\item  \verb|int = obj.RenderOverlay (vtkViewport viewport)| -  Methods required by vtkProp superclass.

\item  \verb|int = obj.RenderOpaqueGeometry (vtkViewport viewport)| -  Methods required by vtkProp superclass.

\item  \verb|int = obj.ComputeInteractionState (int X, int Y, int modify)| -  Given and x-y display coordinate, compute the interaction state of 
 the widget.

\item  \verb|obj.Translate (double translation[3])|

\item  \verb|obj.Translate (int X, int Y)|

\item  \verb|obj.Scale (int X, int Y)|

\item  \verb|obj.PositionHandles ()| -  Synchronize the parallelopiped handle positions with the 
 Polygonal datastructure.

\item  \verb|obj.SetMinimumThickness (double )| -  Minimum thickness for the parallelopiped. User interactions cannot make
 any individual axis of the parallopiped thinner than this value.
 Default is 0.05 expressed as a fraction of the diagonal of the bounding
 box used in the PlaceWidget() invocation.

\item  \verb|double = obj.GetMinimumThickness ()| -  Minimum thickness for the parallelopiped. User interactions cannot make
 any individual axis of the parallopiped thinner than this value.
 Default is 0.05 expressed as a fraction of the diagonal of the bounding
 box used in the PlaceWidget() invocation.

\end{itemize}
