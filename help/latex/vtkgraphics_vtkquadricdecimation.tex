\section{vtkQuadricDecimation}

\subsection{Usage}

 vtkQuadricDecimation is a filter to reduce the number of triangles in
 a triangle mesh, forming a good approximation to the original geometry. 
 The input to vtkQuadricDecimation is a vtkPolyData object, and only
 triangles are treated. If you desire to decimate polygonal meshes, first
 triangulate the polygons with vtkTriangleFilter.

 The algorithm is based on repeated edge collapses until the requested mesh
 reduction is achieved. Edges are placed in a priority queue based on the
 ''cost'' to delete the edge. The cost is an approximate measure of error
 (distance to the original surface)--described by the so-called quadric
 error measure. The quadric error measure is associated with each vertex of
 the mesh and represents a matrix of planes incident on that vertex. The
 distance of the planes to the vertex is the error in the position of the
 vertex (originally the vertex error iz zero). As edges are deleted, the
 quadric error measure associated with the two end points of the edge are
 summed (this combines the plane equations) and an optimal collapse point
 can be computed. Edges connected to the collapse point are then reinserted
 into the queue after computing the new cost to delete them. The process
 continues until the desired reduction level is reached or topological
 constraints prevent further reduction. Note that this basic algorithm can
 be extended to higher dimensions by
 taking into account variation in attributes (i.e., scalars, vectors, and
 so on).

 This paper is based on the work of Garland and Heckbert who first
 presented the quadric error measure at Siggraph '97 ''Surface
 Simplification Using Quadric Error Metrics''. For details of the algorithm
 Michael Garland's Ph.D. thesis is also recommended. Hughues Hoppe's Vis
 '99 paper, ''New Quadric Metric for Simplifying Meshes with Appearance
 Attributes'' is also a good take on the subject especially as it pertains
 to the error metric applied to attributes.

 .SECTION Thanks
 Thanks to Bradley Lowekamp of the National Library of Medicine/NIH for
 contributing this class.

To create an instance of class vtkQuadricDecimation, simply
invoke its constructor as follows
\begin{verbatim}
  obj = vtkQuadricDecimation
\end{verbatim}
\subsection{Methods}

The class vtkQuadricDecimation has several methods that can be used.
  They are listed below.
Note that the documentation is translated automatically from the VTK sources,
and may not be completely intelligible.  When in doubt, consult the VTK website.
In the methods listed below, \verb|obj| is an instance of the vtkQuadricDecimation class.
\begin{itemize}
\item  \verb|string = obj.GetClassName ()|

\item  \verb|int = obj.IsA (string name)|

\item  \verb|vtkQuadricDecimation = obj.NewInstance ()|

\item  \verb|vtkQuadricDecimation = obj.SafeDownCast (vtkObject o)|

\item  \verb|obj.SetTargetReduction (double )| -  Set/Get the desired reduction (expressed as a fraction of the original
 number of triangles). The actual reduction may be less depending on
 triangulation and topological constraints.

\item  \verb|double = obj.GetTargetReductionMinValue ()| -  Set/Get the desired reduction (expressed as a fraction of the original
 number of triangles). The actual reduction may be less depending on
 triangulation and topological constraints.

\item  \verb|double = obj.GetTargetReductionMaxValue ()| -  Set/Get the desired reduction (expressed as a fraction of the original
 number of triangles). The actual reduction may be less depending on
 triangulation and topological constraints.

\item  \verb|double = obj.GetTargetReduction ()| -  Set/Get the desired reduction (expressed as a fraction of the original
 number of triangles). The actual reduction may be less depending on
 triangulation and topological constraints.

\item  \verb|obj.SetAttributeErrorMetric (int )| -  Decide whether to include data attributes in the error metric. If off,
 then only geometric error is used to control the decimation. By default
 the attribute errors are off.

\item  \verb|int = obj.GetAttributeErrorMetric ()| -  Decide whether to include data attributes in the error metric. If off,
 then only geometric error is used to control the decimation. By default
 the attribute errors are off.

\item  \verb|obj.AttributeErrorMetricOn ()| -  Decide whether to include data attributes in the error metric. If off,
 then only geometric error is used to control the decimation. By default
 the attribute errors are off.

\item  \verb|obj.AttributeErrorMetricOff ()| -  Decide whether to include data attributes in the error metric. If off,
 then only geometric error is used to control the decimation. By default
 the attribute errors are off.

\item  \verb|obj.SetScalarsAttribute (int )| -  If attribute errors are to be included in the metric (i.e.,
 AttributeErrorMetric is on), then the following flags control which
 attributes are to be included in the error calculation. By default all
 of these are on.

\item  \verb|int = obj.GetScalarsAttribute ()| -  If attribute errors are to be included in the metric (i.e.,
 AttributeErrorMetric is on), then the following flags control which
 attributes are to be included in the error calculation. By default all
 of these are on.

\item  \verb|obj.ScalarsAttributeOn ()| -  If attribute errors are to be included in the metric (i.e.,
 AttributeErrorMetric is on), then the following flags control which
 attributes are to be included in the error calculation. By default all
 of these are on.

\item  \verb|obj.ScalarsAttributeOff ()| -  If attribute errors are to be included in the metric (i.e.,
 AttributeErrorMetric is on), then the following flags control which
 attributes are to be included in the error calculation. By default all
 of these are on.

\item  \verb|obj.SetVectorsAttribute (int )| -  If attribute errors are to be included in the metric (i.e.,
 AttributeErrorMetric is on), then the following flags control which
 attributes are to be included in the error calculation. By default all
 of these are on.

\item  \verb|int = obj.GetVectorsAttribute ()| -  If attribute errors are to be included in the metric (i.e.,
 AttributeErrorMetric is on), then the following flags control which
 attributes are to be included in the error calculation. By default all
 of these are on.

\item  \verb|obj.VectorsAttributeOn ()| -  If attribute errors are to be included in the metric (i.e.,
 AttributeErrorMetric is on), then the following flags control which
 attributes are to be included in the error calculation. By default all
 of these are on.

\item  \verb|obj.VectorsAttributeOff ()| -  If attribute errors are to be included in the metric (i.e.,
 AttributeErrorMetric is on), then the following flags control which
 attributes are to be included in the error calculation. By default all
 of these are on.

\item  \verb|obj.SetNormalsAttribute (int )| -  If attribute errors are to be included in the metric (i.e.,
 AttributeErrorMetric is on), then the following flags control which
 attributes are to be included in the error calculation. By default all
 of these are on.

\item  \verb|int = obj.GetNormalsAttribute ()| -  If attribute errors are to be included in the metric (i.e.,
 AttributeErrorMetric is on), then the following flags control which
 attributes are to be included in the error calculation. By default all
 of these are on.

\item  \verb|obj.NormalsAttributeOn ()| -  If attribute errors are to be included in the metric (i.e.,
 AttributeErrorMetric is on), then the following flags control which
 attributes are to be included in the error calculation. By default all
 of these are on.

\item  \verb|obj.NormalsAttributeOff ()| -  If attribute errors are to be included in the metric (i.e.,
 AttributeErrorMetric is on), then the following flags control which
 attributes are to be included in the error calculation. By default all
 of these are on.

\item  \verb|obj.SetTCoordsAttribute (int )| -  If attribute errors are to be included in the metric (i.e.,
 AttributeErrorMetric is on), then the following flags control which
 attributes are to be included in the error calculation. By default all
 of these are on.

\item  \verb|int = obj.GetTCoordsAttribute ()| -  If attribute errors are to be included in the metric (i.e.,
 AttributeErrorMetric is on), then the following flags control which
 attributes are to be included in the error calculation. By default all
 of these are on.

\item  \verb|obj.TCoordsAttributeOn ()| -  If attribute errors are to be included in the metric (i.e.,
 AttributeErrorMetric is on), then the following flags control which
 attributes are to be included in the error calculation. By default all
 of these are on.

\item  \verb|obj.TCoordsAttributeOff ()| -  If attribute errors are to be included in the metric (i.e.,
 AttributeErrorMetric is on), then the following flags control which
 attributes are to be included in the error calculation. By default all
 of these are on.

\item  \verb|obj.SetTensorsAttribute (int )| -  If attribute errors are to be included in the metric (i.e.,
 AttributeErrorMetric is on), then the following flags control which
 attributes are to be included in the error calculation. By default all
 of these are on.

\item  \verb|int = obj.GetTensorsAttribute ()| -  If attribute errors are to be included in the metric (i.e.,
 AttributeErrorMetric is on), then the following flags control which
 attributes are to be included in the error calculation. By default all
 of these are on.

\item  \verb|obj.TensorsAttributeOn ()| -  If attribute errors are to be included in the metric (i.e.,
 AttributeErrorMetric is on), then the following flags control which
 attributes are to be included in the error calculation. By default all
 of these are on.

\item  \verb|obj.TensorsAttributeOff ()| -  If attribute errors are to be included in the metric (i.e.,
 AttributeErrorMetric is on), then the following flags control which
 attributes are to be included in the error calculation. By default all
 of these are on.

\item  \verb|obj.SetScalarsWeight (double )| -  Set/Get the scaling weight contribution of the attribute. These
 values are used to weight the contribution of the attributes
 towards the error metric.

\item  \verb|obj.SetVectorsWeight (double )| -  Set/Get the scaling weight contribution of the attribute. These
 values are used to weight the contribution of the attributes
 towards the error metric.

\item  \verb|obj.SetNormalsWeight (double )| -  Set/Get the scaling weight contribution of the attribute. These
 values are used to weight the contribution of the attributes
 towards the error metric.

\item  \verb|obj.SetTCoordsWeight (double )| -  Set/Get the scaling weight contribution of the attribute. These
 values are used to weight the contribution of the attributes
 towards the error metric.

\item  \verb|obj.SetTensorsWeight (double )| -  Set/Get the scaling weight contribution of the attribute. These
 values are used to weight the contribution of the attributes
 towards the error metric.

\item  \verb|double = obj.GetScalarsWeight ()| -  Set/Get the scaling weight contribution of the attribute. These
 values are used to weight the contribution of the attributes
 towards the error metric.

\item  \verb|double = obj.GetVectorsWeight ()| -  Set/Get the scaling weight contribution of the attribute. These
 values are used to weight the contribution of the attributes
 towards the error metric.

\item  \verb|double = obj.GetNormalsWeight ()| -  Set/Get the scaling weight contribution of the attribute. These
 values are used to weight the contribution of the attributes
 towards the error metric.

\item  \verb|double = obj.GetTCoordsWeight ()| -  Set/Get the scaling weight contribution of the attribute. These
 values are used to weight the contribution of the attributes
 towards the error metric.

\item  \verb|double = obj.GetTensorsWeight ()| -  Set/Get the scaling weight contribution of the attribute. These
 values are used to weight the contribution of the attributes
 towards the error metric.

\item  \verb|double = obj.GetActualReduction ()| -  Get the actual reduction. This value is only valid after the
 filter has executed.

\end{itemize}
