\section{vtkLabelHierarchy}

\subsection{Usage}

 This class represents labels in a hierarchy used to denote rendering priority.
 A binary tree of labels is maintained that subdivides the bounds of the
 of the label anchors spatially. Which level of the tree a label occupies
 determines its priority; those at higher levels of the tree will be
 more likely to render than those at lower levels of the tree.

 Pass vtkLabelHierarchy objects to a vtkLabelPlacementMapper filter for dynamic,
 non-overlapping, per-frame placement of labels.

 Note that if we have a d-dimensional binary tree and we want a fixed
 number $n$ of labels in each node (all nodes, not just leaves),
 we can compute the depth of tree required assuming a uniform distribution
 of points. Given a total of $N$ points we know that
 $\frac{N}{|T|} = n$, where $|T|$ is the cardinality of the tree (i.e.,
 the number of nodes it contains).
 Because we have a uniform distribution, the tree will be uniformly subdivided
 and thus $|T| = 1 + 2^d + \left(2^d\right)^2 + \cdots + \left(2^d\right)^k$,
 where $d$ is the dimensionality of the input points (fixed at 3 for now).
 As $k$ becomes large, $|T|pprox 2 \left(2^d\right)^k$.
 Using this approximation, we can solve for $k$:
 \[ k = \frac{\log{\frac{N}{2n}}}{\log{2^d}} \]
 Given a set of $N$ input label anchors, we'll compute $k$ and then
 bin the anchors into tree nodes at level $k$ of the tree. After this,
 all the nodes will be in the leaves of the tree and those leaves will be at
 the $k$-th level; no anchors will be in levels $1, 2, \ldots, k-1$.
 To fix that, we'll choose to move some anchors upwards.
 The exact number to move upwards depends on  TargetLabelCount. We'll
 move as many up as required to have  TargetLabelCount at each node.

 You should avoid situations where  MaximumDepth does not allow for
  TargetLabelCount or fewer entries at each node. The  MaximumDepth
 is a hard limit while  TargetLabelCount is a suggested optimum. You will
 end up with many more than  TargetLabelCount entries per node and things
 will be sloooow.

To create an instance of class vtkLabelHierarchy, simply
invoke its constructor as follows
\begin{verbatim}
  obj = vtkLabelHierarchy
\end{verbatim}
\subsection{Methods}

The class vtkLabelHierarchy has several methods that can be used.
  They are listed below.
Note that the documentation is translated automatically from the VTK sources,
and may not be completely intelligible.  When in doubt, consult the VTK website.
In the methods listed below, \verb|obj| is an instance of the vtkLabelHierarchy class.
\begin{itemize}
\item  \verb|string = obj.GetClassName ()|

\item  \verb|int = obj.IsA (string name)|

\item  \verb|vtkLabelHierarchy = obj.NewInstance ()|

\item  \verb|vtkLabelHierarchy = obj.SafeDownCast (vtkObject o)|

\item  \verb|obj.SetPoints (vtkPoints )| -  Override SetPoints so we can reset the hierarchy when the points change.

\item  \verb|obj.ComputeHierarchy ()| -  Fill the hierarchy with the input labels.

\item  \verb|obj.SetTargetLabelCount (int )| -  The number of labels that is ideally present at any octree node.
 It is best if this is a multiple of $2^d$.

\item  \verb|int = obj.GetTargetLabelCount ()| -  The number of labels that is ideally present at any octree node.
 It is best if this is a multiple of $2^d$.

\item  \verb|obj.SetMaximumDepth (int )| -  The maximum depth of the octree.

\item  \verb|int = obj.GetMaximumDepth ()| -  The maximum depth of the octree.

\item  \verb|obj.SetTextProperty (vtkTextProperty tprop)| -  The default text property assigned to labels in this hierarchy.

\item  \verb|vtkTextProperty = obj.GetTextProperty ()| -  The default text property assigned to labels in this hierarchy.

\item  \verb|obj.SetPriorities (vtkDataArray arr)| -  Set/get the array specifying the importance (priority) of each label.

\item  \verb|vtkDataArray = obj.GetPriorities ()| -  Set/get the array specifying the importance (priority) of each label.

\item  \verb|obj.SetLabels (vtkAbstractArray arr)| -  Set/get the array specifying the text of each label.

\item  \verb|vtkAbstractArray = obj.GetLabels ()| -  Set/get the array specifying the text of each label.

\item  \verb|obj.SetOrientations (vtkDataArray arr)| -  Set/get the array specifying the orientation of each label.

\item  \verb|vtkDataArray = obj.GetOrientations ()| -  Set/get the array specifying the orientation of each label.

\item  \verb|obj.SetIconIndices (vtkIntArray arr)| -  Set/get the array specifying the icon index of each label.

\item  \verb|vtkIntArray = obj.GetIconIndices ()| -  Set/get the array specifying the icon index of each label.

\item  \verb|obj.SetSizes (vtkDataArray arr)| -  Set/get the array specifying the size of each label.

\item  \verb|vtkDataArray = obj.GetSizes ()| -  Set/get the array specifying the size of each label.

\item  \verb|obj.SetBoundedSizes (vtkDataArray arr)| -  Set/get the array specifying the maximum width and height in world coordinates of each label.

\item  \verb|vtkDataArray = obj.GetBoundedSizes ()| -  Set/get the array specifying the maximum width and height in world coordinates of each label.

\item  \verb|obj.GetDiscreteNodeCoordinatesFromWorldPoint (int ijk[3], double pt[3], int level)| -  Given a depth in the hierarchy ( level) and a point  pt in world space, compute  ijk.
 This is used to find other octree nodes at the same  level that are within the search radius
 for candidate labels to be placed. It is called with  pt set to the camera eye point and
 pythagorean quadruples increasingly distant from the origin are added to  ijk to identify
 octree nodes whose labels should be placed.
 @param[out] ijk - discrete coordinates of the octree node at  level containing  pt.
 @param[in]  pt - input world point coordinates
 @param[in]  level - input octree level to be considered

\item  \verb|vtkIdType = obj.GetNumberOfCells ()| -  Inherited members (from vtkDataSet)

\item  \verb|vtkCell = obj.GetCell (vtkIdType )| -  Inherited members (from vtkDataSet)

\item  \verb|obj.GetCell (vtkIdType , vtkGenericCell )| -  Inherited members (from vtkDataSet)

\item  \verb|int = obj.GetCellType (vtkIdType )| -  Inherited members (from vtkDataSet)

\item  \verb|obj.GetCellPoints (vtkIdType , vtkIdList )| -  Inherited members (from vtkDataSet)

\item  \verb|obj.GetPointCells (vtkIdType , vtkIdList )| -  Inherited members (from vtkDataSet)

\item  \verb|int = obj.GetMaxCellSize ()| -  Inherited members (from vtkDataSet)

\item  \verb|vtkPoints = obj.GetCenterPts ()| -  Provide access to original coordinates of sets of coincident points

\item  \verb|vtkCoincidentPoints = obj.GetCoincidentPoints ()| -  Provide access to the set of coincident points that have been
 perturbed by the hierarchy in order to render labels for each
 without overlap.

\end{itemize}
