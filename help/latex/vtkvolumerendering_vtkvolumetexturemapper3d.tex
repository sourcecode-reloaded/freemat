\section{vtkVolumeTextureMapper3D}

\subsection{Usage}

 vtkVolumeTextureMapper3D renders a volume using 3D texture mapping. 
 This class is actually an abstract superclass - with all the actual
 work done by vtkOpenGLVolumeTextureMapper3D. 
 
 This mappers currently supports:

 - any data type as input
 - one component, or two or four non-independent components
 - composite blending
 - intermixed opaque geometry
 - multiple volumes can be rendered if they can
   be sorted into back-to-front order (use the vtkFrustumCoverageCuller)
 
 This mapper does not support:
 - more than one independent component
 - maximum intensity projection
 
 Internally, this mapper will potentially change the resolution of the
 input data. The data will be resampled to be a power of two in each
 direction, and also no greater than 128*256*256 voxels (any aspect) 
 for one or two component data, or 128*128*256 voxels (any aspect)
 for four component data. The limits are currently hardcoded after 
 a check using the GL\_PROXY\_TEXTURE3D because some graphics drivers 
 were always responding ''yes'' to the proxy call despite not being
 able to allocate that much texture memory.

 Currently, calculations are computed using 8 bits per RGBA channel.
 In the future this should be expanded to handle newer boards that
 can support 15 bit float compositing.

 This mapper supports two main families of graphics hardware:
 nvidia and ATI. There are two different implementations of
 3D texture mapping used - one based on nvidia's GL\_NV\_texture\_shader2
 and GL\_NV\_register\_combiners2 extension, and one based on
 ATI's GL\_ATI\_fragment\_shader (supported also by some nvidia boards)
 To use this class in an application that will run on various 
 hardware configurations, you should have a back-up volume rendering
 method. You should create a vtkVolumeTextureMapper3D, assign its
 input, make sure you have a current OpenGL context (you've rendered
 at least once), then call IsRenderSupported with a vtkVolumeProperty 
 as an argument. This method will return 0 if the input has more than
 one independent component, or if the graphics hardware does not
 support the set of required extensions for using at least one of
 the two implemented methods (nvidia or ati)

 .SECTION Thanks
 Thanks to Alexandre Gouaillard at the Megason Lab, Department of Systems
 Biology, Harvard Medical School
 https://wiki.med.harvard.edu/SysBio/Megason/
 for the idea and initial patch to speed-up rendering with compressed
 textures.


To create an instance of class vtkVolumeTextureMapper3D, simply
invoke its constructor as follows
\begin{verbatim}
  obj = vtkVolumeTextureMapper3D
\end{verbatim}
\subsection{Methods}

The class vtkVolumeTextureMapper3D has several methods that can be used.
  They are listed below.
Note that the documentation is translated automatically from the VTK sources,
and may not be completely intelligible.  When in doubt, consult the VTK website.
In the methods listed below, \verb|obj| is an instance of the vtkVolumeTextureMapper3D class.
\begin{itemize}
\item  \verb|string = obj.GetClassName ()|

\item  \verb|int = obj.IsA (string name)|

\item  \verb|vtkVolumeTextureMapper3D = obj.NewInstance ()|

\item  \verb|vtkVolumeTextureMapper3D = obj.SafeDownCast (vtkObject o)|

\item  \verb|obj.SetSampleDistance (float )| -  The distance at which to space sampling planes. This
 may not be honored for interactive renders. An interactive
 render is defined as one that has less than 1 second of
 allocated render time.

\item  \verb|float = obj.GetSampleDistance ()| -  The distance at which to space sampling planes. This
 may not be honored for interactive renders. An interactive
 render is defined as one that has less than 1 second of
 allocated render time.

\item  \verb|int = obj. GetVolumeDimensions ()| -  These are the dimensions of the 3D texture

\item  \verb|float = obj. GetVolumeSpacing ()| -  This is the spacing of the 3D texture

\item  \verb|int = obj.IsRenderSupported (vtkVolumeProperty )| -  Based on hardware and properties, we may or may not be able to
 render using 3D texture mapping. This indicates if 3D texture
 mapping is supported by the hardware, and if the other extensions
 necessary to support the specific properties are available.

\item  \verb|int = obj.GetNumberOfPolygons ()| -  Allow access to the number of polygons used for the
 rendering.

\item  \verb|float = obj.GetActualSampleDistance ()| -  Allow access to the actual sample distance used to render
 the image.

\item  \verb|obj.SetPreferredRenderMethod (int )| -  Set the preferred render method. If it is supported, this
 one will be used. Don't allow ATI\_METHOD - it is not actually
 supported.

\item  \verb|int = obj.GetPreferredRenderMethodMinValue ()| -  Set the preferred render method. If it is supported, this
 one will be used. Don't allow ATI\_METHOD - it is not actually
 supported.

\item  \verb|int = obj.GetPreferredRenderMethodMaxValue ()| -  Set the preferred render method. If it is supported, this
 one will be used. Don't allow ATI\_METHOD - it is not actually
 supported.

\item  \verb|obj.SetPreferredMethodToFragmentProgram ()| -  Set the preferred render method. If it is supported, this
 one will be used. Don't allow ATI\_METHOD - it is not actually
 supported.

\item  \verb|obj.SetPreferredMethodToNVidia ()| -  Set the preferred render method. If it is supported, this
 one will be used. Don't allow ATI\_METHOD - it is not actually
 supported.

\item  \verb|int = obj.GetPreferredRenderMethod ()| -  Set the preferred render method. If it is supported, this
 one will be used. Don't allow ATI\_METHOD - it is not actually
 supported.

\item  \verb|obj.SetUseCompressedTexture (bool )| -  Set/Get if the mapper use compressed textures (if supported by the
 hardware). Initial value is false.
 There are two reasons to use compressed textures: 1. rendering can be 4
 times faster. 2. It saves some VRAM.
 There is one reason to not use compressed textures: quality may be lower
 than with uncompressed textures.

\item  \verb|bool = obj.GetUseCompressedTexture ()| -  Set/Get if the mapper use compressed textures (if supported by the
 hardware). Initial value is false.
 There are two reasons to use compressed textures: 1. rendering can be 4
 times faster. 2. It saves some VRAM.
 There is one reason to not use compressed textures: quality may be lower
 than with uncompressed textures.

\end{itemize}
