\section{vtkGeometricErrorMetric}

\subsection{Usage}

 It is a concrete error metric, based on a geometric criterium:
 the variation of the edge from a straight line.


To create an instance of class vtkGeometricErrorMetric, simply
invoke its constructor as follows
\begin{verbatim}
  obj = vtkGeometricErrorMetric
\end{verbatim}
\subsection{Methods}

The class vtkGeometricErrorMetric has several methods that can be used.
  They are listed below.
Note that the documentation is translated automatically from the VTK sources,
and may not be completely intelligible.  When in doubt, consult the VTK website.
In the methods listed below, \verb|obj| is an instance of the vtkGeometricErrorMetric class.
\begin{itemize}
\item  \verb|string = obj.GetClassName ()| -  Standard VTK type and error macros.

\item  \verb|int = obj.IsA (string name)| -  Standard VTK type and error macros.

\item  \verb|vtkGeometricErrorMetric = obj.NewInstance ()| -  Standard VTK type and error macros.

\item  \verb|vtkGeometricErrorMetric = obj.SafeDownCast (vtkObject o)| -  Standard VTK type and error macros.

\item  \verb|double = obj.GetAbsoluteGeometricTolerance ()| -  Return the squared absolute geometric accuracy. See
 SetAbsoluteGeometricTolerance() for details.
 

\item  \verb|obj.SetAbsoluteGeometricTolerance (double value)| -  Set the geometric accuracy with a squared absolute value.
 This is the geometric object-based accuracy.
 Subdivision will be required if the square distance between the real
 point and the straight line passing through the vertices of the edge is
 greater than `value'. For instance 0.01 will give better result than 0.1.
 

\item  \verb|obj.SetRelativeGeometricTolerance (double value, vtkGenericDataSet ds)| -  Set the geometric accuracy with a value relative to the length of the
 bounding box of the dataset. Internally compute the absolute tolerance.
 For instance 0.01 will give better result than 0.1.
 
 

\item  \verb|int = obj.RequiresEdgeSubdivision (double leftPoint, double midPoint, double rightPoint, double alpha)| -  Does the edge need to be subdivided according to the distance between
 the line passing through its endpoints and the mid point?
 The edge is defined by its `leftPoint' and its `rightPoint'.
 `leftPoint', `midPoint' and `rightPoint' have to be initialized before
 calling RequiresEdgeSubdivision().
 Their format is global coordinates, parametric coordinates and
 point centered attributes: xyx rst abc de...
 `alpha' is the normalized abscissa of the midpoint along the edge.
 (close to 0 means close to the left point, close to 1 means close to the
 right point)
 
 
 
 
 
          =GetAttributeCollection()->GetNumberOfPointCenteredComponents()+6

\item  \verb|double = obj.GetError (double leftPoint, double midPoint, double rightPoint, double alpha)| -  Return the error at the mid-point. It will return an error relative to
 the bounding box size if GetRelative() is true, a square absolute error
 otherwise.
 See RequiresEdgeSubdivision() for a description of the arguments.
 
 
 
 
 
          =GetAttributeCollection()->GetNumberOfPointCenteredComponents()+6
 

\item  \verb|int = obj.GetRelative ()| -  Return the type of output of GetError()

\end{itemize}
