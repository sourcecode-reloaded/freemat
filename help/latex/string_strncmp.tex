\section{STRNCMP String Compare Function To Length N }

\subsection{USAGE}

Compares two strings for equality, but only looks at the
first N characters from each string.  The general syntax 
for its use is
\begin{verbatim}
  p = strncmp(x,y,n)
\end{verbatim}
where \verb|x| and \verb|y| are two strings.  Returns \verb|true| if \verb|x|
and \verb|y| are each at least \verb|n| characters long, and if the
first \verb|n| characters from each string are the same.  Otherwise,
it returns \verb|false|.
In the second form, \verb|strncmp| can be applied to a cell array of
strings.  The syntax for this form is
\begin{verbatim}
  p = strncmp(cellstra,cellstrb,n)
\end{verbatim}
where \verb|cellstra| and \verb|cellstrb| are cell arrays of a strings
to compare.  Also, you can also supply a character matrix as
an argument to \verb|strcmp|, in which case it will be converted
via \verb|cellstr| (so that trailing spaces are removed), before being
compared.
\subsection{Example}

The following piece of code compares two strings:
\begin{verbatim}
--> x1 = 'astring';
--> x2 = 'bstring';
--> x3 = 'astring';
--> strncmp(x1,x2,4)

ans = 
 0 

--> strncmp(x1,x3,4)

ans = 
 1 
\end{verbatim}
Here we use a cell array strings
\begin{verbatim}
--> x = {'ast','bst',43,'astr'}

x = 
 [ast] [bst] [43] [astr] 

--> p = strncmp(x,'ast',3)

p = 
 1 0 0 1 
\end{verbatim}
Here we compare two cell arrays of strings
\begin{verbatim}
--> strncmp({'this','is','a','pickle'},{'think','is','to','pickle'},3)

ans = 
 1 0 0 1 
\end{verbatim}
Finally, the case where one of the arguments is a matrix
string
\begin{verbatim}
--> strncmp({'this','is','a','pickle'},['peter ';'piper ';'hated ';'pickle'],4);
\end{verbatim}
