\section{QR QR Decomposition of a Matrix}

\subsection{Usage}

Computes the QR factorization of a matrix.  The \verb|qr| function has
multiple forms, with and without pivoting.  The non-pivot version
has two forms, a compact version and a full-blown decomposition
version.  The compact version of the decomposition of a matrix 
of size \verb|M x N| is
\begin{verbatim}
  [q,r] = qr(a,0)
\end{verbatim}
where \verb|q| is a matrix of size \verb|M x L| and \verb|r| is a matrix of
size \verb|L x N| and \verb|L = min(N,M)|, and \verb|q*r = a|.  The QR decomposition is
such that the columns of \verb|Q| are orthonormal, and \verb|R| is upper
triangular.  The decomposition is computed using the LAPACK 
routine \verb|xgeqrf|, where \verb|x| is the precision of the matrix.  
FreeMat supports decompositions of \verb|single| and \verb|double| types.

The second form of the non-pivot decomposition omits the second \verb|0|
argument:
\begin{verbatim}
  [q,r] = qr(a)
\end{verbatim}
This second form differs from the previous form only for matrices
with more rows than columns (\verb|M > N|).  For these matrices, the
full decomposition is of a matrix \verb|Q| of size \verb|M x M| and 
a matrix \verb|R| of size \verb|M x N|.  The full decomposition is computed
using the same LAPACK routines as the compact decomposition, but
on an augmented matrix \verb|[a 0]|, where enough columns are added to
form a square matrix.

Generally, the QR decomposition will not return a matrix \verb|R| with
diagonal elements in any specific order.  The remaining two forms 
of the \verb|qr| command utilize permutations of the columns of \verb|a|
so that the diagonal elements of \verb|r| are in decreasing magnitude.
To trigger this form of the decomposition, a third argument is
required, which records the permutation applied to the argument \verb|a|.
The compact version is
\begin{verbatim}
  [q,r,e] = qr(a,0)
\end{verbatim}
where \verb|e| is an integer vector that describes the permutation of
the columns of \verb|a| necessary to reorder the diagonal elements of
\verb|r|.  This result is computed using the LAPACK routines \verb|(s,d)geqp3|.
In the non-compact version of the QR decomposition with pivoting,
\begin{verbatim}
  [q,r,e] = qr(a)
\end{verbatim}
the returned matrix \verb|e| is a permutation matrix, such that 
\verb|q*r*e' = a|.
