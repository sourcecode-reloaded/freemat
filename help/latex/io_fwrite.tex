\section{FWRITE File Write Function}

\subsection{Usage}

Writes an array to a given file handle as a block of binary (raw) data.
The general use of the function is
\begin{verbatim}
  n = fwrite(handle,A)
\end{verbatim}
The \verb|handle| argument must be a valid value returned by the fopen 
function, and accessable for writing. The array \verb|A| is written to
the file a column at a time.  The form of the output data depends
on (and is inferred from) the precision of the array \verb|A|.  If the 
write fails (because we ran out of disk space, etc.) then an error
is returned.  The output \verb|n| indicates the number of elements
successfully written.

Note that unlike MATLAB, FreeMat 4 does not default to \verb|uint8| for
writing arrays to files.  Alternately, the type of the data to be
written to the file can be specified with the syntax
\begin{verbatim}
  n = fwrite(handle,A,type)
\end{verbatim}
where \verb|type| is one of the following legal values:
\begin{itemize}
\item  'uint8','uchar','unsigned char' for an unsigned, 8-bit integer.

\item  'int8','char','integer*1' for a signed, 8-bit integer.

\item  'uint16','unsigned short' for an unsigned, 16-bit  integer.

\item  'int16','short','integer*2' for a signed, 16-bit integer.

\item  'uint32','unsigned int' for an unsigned, 32-bit integer.

\item  'int32','int','integer*4' for a signed, 32-bit integer.

\item  'single','float32','float','real*4' for a 32-bit floating point.

\item  'double','float64','real*8' for a 64-bit floating point.

\end{itemize}

\subsection{Example}

Heres an example of writing an array of \verb|512 x 512| Gaussian-distributed \verb|float| random variables, and then writing them to a file called \verb|test.dat|.
\begin{verbatim}
--> A = float(randn(512));
--> fp = fopen('test.dat','w');
--> fwrite(fp,A,'single');
--> fclose(fp);
\end{verbatim}
