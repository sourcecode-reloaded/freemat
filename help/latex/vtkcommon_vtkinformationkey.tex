\section{vtkInformationKey}

\subsection{Usage}

 vtkInformationKey is the superclass for all keys used to access the
 map represented by vtkInformation.  The vtkInformation::Set and
 vtkInformation::Get methods of vtkInformation are accessed by
 information keys.  A key is a pointer to an instance of a subclass
 of vtkInformationKey.  The type of the subclass determines the
 overload of Set/Get that is selected.  This ensures that the type
 of value stored in a vtkInformation instance corresponding to a
 given key matches the type expected for that key.

To create an instance of class vtkInformationKey, simply
invoke its constructor as follows
\begin{verbatim}
  obj = vtkInformationKey
\end{verbatim}
\subsection{Methods}

The class vtkInformationKey has several methods that can be used.
  They are listed below.
Note that the documentation is translated automatically from the VTK sources,
and may not be completely intelligible.  When in doubt, consult the VTK website.
In the methods listed below, \verb|obj| is an instance of the vtkInformationKey class.
\begin{itemize}
\item  \verb|string = obj.GetClassName ()|

\item  \verb|int = obj.IsA (string name)|

\item  \verb|vtkInformationKey = obj.NewInstance ()|

\item  \verb|vtkInformationKey = obj.SafeDownCast (vtkObject o)|

\item  \verb|obj.Register (vtkObjectBase )| -  Prevent normal vtkObject reference counting behavior.

\item  \verb|obj.UnRegister (vtkObjectBase )| -  Prevent normal vtkObject reference counting behavior.

\item  \verb|string = obj.GetName ()| -  Get the name of the key.  This is not the type of the key, but
 the name of the key instance.

\item  \verb|string = obj.GetLocation ()| -  Get the location of the key.  This is the name of the class in
 which the key is defined.

\item  \verb|vtkInformationKey = obj.(string name, string location)| -  Key instances are static data that need to be created and
 destroyed.  The constructor and destructor must be public.  The
 name of the static instance and the class in which it is defined
 should be passed to the constructor.  They must be string
 literals because the strings are not copied.

\item  \verb|~vtkInformationKey = obj.()| -  Key instances are static data that need to be created and
 destroyed.  The constructor and destructor must be public.  The
 name of the static instance and the class in which it is defined
 should be passed to the constructor.  They must be string
 literals because the strings are not copied.

\item  \verb|obj.ShallowCopy (vtkInformation from, vtkInformation to)| -  Copy the entry associated with this key from one information
 object to another.  If there is no entry in the first information
 object for this key, the value is removed from the second.

\item  \verb|obj.DeepCopy (vtkInformation from, vtkInformation to)| -  Check whether this key appears in the given information object.

\item  \verb|int = obj.Has (vtkInformation info)| -  Check whether this key appears in the given information object.

\item  \verb|obj.Remove (vtkInformation info)| -  Remove this key from the given information object.

\item  \verb|obj.Report (vtkInformation info, vtkGarbageCollector collector)| -  Report a reference this key has in the given information object.

\end{itemize}
