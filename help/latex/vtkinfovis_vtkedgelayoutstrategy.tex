\section{vtkEdgeLayoutStrategy}

\subsection{Usage}

 All edge layouts should subclass from this class.  vtkEdgeLayoutStrategy
 works as a plug-in to the vtkEdgeLayout algorithm.

To create an instance of class vtkEdgeLayoutStrategy, simply
invoke its constructor as follows
\begin{verbatim}
  obj = vtkEdgeLayoutStrategy
\end{verbatim}
\subsection{Methods}

The class vtkEdgeLayoutStrategy has several methods that can be used.
  They are listed below.
Note that the documentation is translated automatically from the VTK sources,
and may not be completely intelligible.  When in doubt, consult the VTK website.
In the methods listed below, \verb|obj| is an instance of the vtkEdgeLayoutStrategy class.
\begin{itemize}
\item  \verb|string = obj.GetClassName ()|

\item  \verb|int = obj.IsA (string name)|

\item  \verb|vtkEdgeLayoutStrategy = obj.NewInstance ()|

\item  \verb|vtkEdgeLayoutStrategy = obj.SafeDownCast (vtkObject o)|

\item  \verb|obj.SetGraph (vtkGraph graph)| -  Setting the graph for the layout strategy

\item  \verb|obj.Initialize ()| -  This method allows the layout strategy to
 do initialization of data structures
 or whatever else it might want to do. 

\item  \verb|obj.Layout ()| -  This is the layout method where the graph that was
 set in SetGraph() is laid out.

\item  \verb|obj.SetEdgeWeightArrayName (string )| -  Set/Get the field to use for the edge weights.

\item  \verb|string = obj.GetEdgeWeightArrayName ()| -  Set/Get the field to use for the edge weights.

\end{itemize}
