\section{vtkPerlinNoise}

\subsection{Usage}

 vtkPerlinNoise computes a Perlin noise field as an implicit function.
 vtkPerlinNoise is a concrete implementation of vtkImplicitFunction.
 Perlin noise, originally described by Ken Perlin, is a non-periodic and
 continuous noise function useful for modeling real-world objects.

 The amplitude and frequency of the noise pattern are adjustable.  This
 implementation of Perlin noise is derived closely from Greg Ward's version
 in Graphics Gems II.

To create an instance of class vtkPerlinNoise, simply
invoke its constructor as follows
\begin{verbatim}
  obj = vtkPerlinNoise
\end{verbatim}
\subsection{Methods}

The class vtkPerlinNoise has several methods that can be used.
  They are listed below.
Note that the documentation is translated automatically from the VTK sources,
and may not be completely intelligible.  When in doubt, consult the VTK website.
In the methods listed below, \verb|obj| is an instance of the vtkPerlinNoise class.
\begin{itemize}
\item  \verb|string = obj.GetClassName ()|

\item  \verb|int = obj.IsA (string name)|

\item  \verb|vtkPerlinNoise = obj.NewInstance ()|

\item  \verb|vtkPerlinNoise = obj.SafeDownCast (vtkObject o)|

\item  \verb|double = obj.EvaluateFunction (double x[3])| -  Evaluate PerlinNoise function.

\item  \verb|double = obj.EvaluateFunction (double x, double y, double z)| -  Evaluate PerlinNoise function.

\item  \verb|obj.EvaluateGradient (double x[3], double n[3])| -  Evaluate PerlinNoise gradient.  Currently, the method returns a 0 
 gradient.

\item  \verb|obj.SetFrequency (double , double , double )| -  Set/get the frequency, or physical scale,  of the noise function 
 (higher is finer scale).  The frequency can be adjusted per axis, or
 the same for all axes.

\item  \verb|obj.SetFrequency (double  a[3])| -  Set/get the frequency, or physical scale,  of the noise function 
 (higher is finer scale).  The frequency can be adjusted per axis, or
 the same for all axes.

\item  \verb|double = obj. GetFrequency ()| -  Set/get the frequency, or physical scale,  of the noise function 
 (higher is finer scale).  The frequency can be adjusted per axis, or
 the same for all axes.

\item  \verb|obj.SetPhase (double , double , double )| -  Set/get the phase of the noise function.  This parameter can be used to
 shift the noise function within space (perhaps to avoid a beat with a
 noise pattern at another scale).  Phase tends to repeat about every
 unit, so a phase of 0.5 is a half-cycle shift.

\item  \verb|obj.SetPhase (double  a[3])| -  Set/get the phase of the noise function.  This parameter can be used to
 shift the noise function within space (perhaps to avoid a beat with a
 noise pattern at another scale).  Phase tends to repeat about every
 unit, so a phase of 0.5 is a half-cycle shift.

\item  \verb|double = obj. GetPhase ()| -  Set/get the phase of the noise function.  This parameter can be used to
 shift the noise function within space (perhaps to avoid a beat with a
 noise pattern at another scale).  Phase tends to repeat about every
 unit, so a phase of 0.5 is a half-cycle shift.

\item  \verb|obj.SetAmplitude (double )| -  Set/get the amplitude of the noise function. Amplitude can be negative.
 The noise function varies randomly between -|Amplitude| and |Amplitude|.
 Therefore the range of values is 2*|Amplitude| large.
 The initial amplitude is 1.

\item  \verb|double = obj.GetAmplitude ()| -  Set/get the amplitude of the noise function. Amplitude can be negative.
 The noise function varies randomly between -|Amplitude| and |Amplitude|.
 Therefore the range of values is 2*|Amplitude| large.
 The initial amplitude is 1.

\end{itemize}
