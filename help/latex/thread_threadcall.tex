\section{THREADCALL Call Function In A Thread}

\subsection{Usage}

The \verb|threadcall| function is a convenience function for executing
a function call in a thread.  The syntax for its use is
\begin{verbatim}
   [val1,...,valn] = threadcall(threadid,timeout,funcname,arg1,arg2,...)
\end{verbatim}
where \verb|threadid| is the ID of the thread (as returned by the
\verb|threadnew| function), \verb|funcname| is the name of the function to call,
and \verb|argi| are the arguments to the function, and \verb|timeout| is the
amount of time (in milliseconds) that the function is allowed to take.
\subsection{Example}

Here is an example of executing a simple function in a different thread.
\begin{verbatim}
--> id = threadnew

id = 
 3 

--> d = threadcall(id,1000,'cos',1.02343)

d = 
    0.5204 

--> threadfree(id)
\end{verbatim}
