\section{vtkAssembly}

\subsection{Usage}

 vtkAssembly is an object that groups vtkProp3Ds, its subclasses, and
 other assemblies into a tree-like hierarchy. The vtkProp3Ds and
 assemblies can then be transformed together by transforming just the root
 assembly of the hierarchy.

 A vtkAssembly object can be used in place of an vtkProp3D since it is a
 subclass of vtkProp3D. The difference is that vtkAssembly maintains a list
 of vtkProp3D instances (its ''parts'') that form the assembly. Then, any
 operation that transforms (i.e., scales, rotates, translates) the parent
 assembly will transform all its parts.  Note that this process is
 recursive: you can create groups consisting of assemblies and/or
 vtkProp3Ds to arbitrary depth.

 To add an assembly to the renderer's list of props, you only need to
 add the root of the assembly. During rendering, the parts of the
 assembly are rendered during a hierarchical traversal process.

To create an instance of class vtkAssembly, simply
invoke its constructor as follows
\begin{verbatim}
  obj = vtkAssembly
\end{verbatim}
\subsection{Methods}

The class vtkAssembly has several methods that can be used.
  They are listed below.
Note that the documentation is translated automatically from the VTK sources,
and may not be completely intelligible.  When in doubt, consult the VTK website.
In the methods listed below, \verb|obj| is an instance of the vtkAssembly class.
\begin{itemize}
\item  \verb|string = obj.GetClassName ()|

\item  \verb|int = obj.IsA (string name)|

\item  \verb|vtkAssembly = obj.NewInstance ()|

\item  \verb|vtkAssembly = obj.SafeDownCast (vtkObject o)|

\item  \verb|obj.AddPart (vtkProp3D )| -  Add a part to the list of parts.

\item  \verb|obj.RemovePart (vtkProp3D )| -  Remove a part from the list of parts,

\item  \verb|vtkProp3DCollection = obj.GetParts ()| -  Return the parts (direct descendants) of this assembly.

\item  \verb|obj.GetActors (vtkPropCollection )| -  For some exporters and other other operations we must be
 able to collect all the actors or volumes. These methods
 are used in that process.

\item  \verb|obj.GetVolumes (vtkPropCollection )| -  For some exporters and other other operations we must be
 able to collect all the actors or volumes. These methods
 are used in that process.

\item  \verb|int = obj.RenderOpaqueGeometry (vtkViewport ren)| -  Render this assembly and all its parts. 
 The rendering process is recursive.
 Note that a mapper need not be defined. If not defined, then no geometry 
 will be drawn for this assembly. This allows you to create ''logical''
 assemblies; that is, assemblies that only serve to group and transform
 its parts.

\item  \verb|int = obj.RenderTranslucentPolygonalGeometry (vtkViewport ren)| -  Render this assembly and all its parts. 
 The rendering process is recursive.
 Note that a mapper need not be defined. If not defined, then no geometry 
 will be drawn for this assembly. This allows you to create ''logical''
 assemblies; that is, assemblies that only serve to group and transform
 its parts.

\item  \verb|int = obj.RenderVolumetricGeometry (vtkViewport ren)| -  Render this assembly and all its parts. 
 The rendering process is recursive.
 Note that a mapper need not be defined. If not defined, then no geometry 
 will be drawn for this assembly. This allows you to create ''logical''
 assemblies; that is, assemblies that only serve to group and transform
 its parts.

\item  \verb|int = obj.HasTranslucentPolygonalGeometry ()| -  Does this prop have some translucent polygonal geometry?

\item  \verb|obj.ReleaseGraphicsResources (vtkWindow )| -  Release any graphics resources that are being consumed by this actor.
 The parameter window could be used to determine which graphic
 resources to release.

\item  \verb|obj.InitPathTraversal ()| -  Methods to traverse the parts of an assembly. Each part (starting from
 the root) will appear properly transformed and with the correct
 properties (depending upon the ApplyProperty and ApplyTransform ivars).
 Note that the part appears as an instance of vtkProp. These methods
 should be contrasted to those that traverse the list of parts using
 GetParts().  GetParts() returns a list of children of this assembly, not
 necessarily with the correct transformation or properties. To use the
 methods below - first invoke InitPathTraversal() followed by repeated
 calls to GetNextPath().  GetNextPath() returns a NULL pointer when the
 list is exhausted.

\item  \verb|vtkAssemblyPath = obj.GetNextPath ()| -  Methods to traverse the parts of an assembly. Each part (starting from
 the root) will appear properly transformed and with the correct
 properties (depending upon the ApplyProperty and ApplyTransform ivars).
 Note that the part appears as an instance of vtkProp. These methods
 should be contrasted to those that traverse the list of parts using
 GetParts().  GetParts() returns a list of children of this assembly, not
 necessarily with the correct transformation or properties. To use the
 methods below - first invoke InitPathTraversal() followed by repeated
 calls to GetNextPath().  GetNextPath() returns a NULL pointer when the
 list is exhausted.

\item  \verb|int = obj.GetNumberOfPaths ()| -  Methods to traverse the parts of an assembly. Each part (starting from
 the root) will appear properly transformed and with the correct
 properties (depending upon the ApplyProperty and ApplyTransform ivars).
 Note that the part appears as an instance of vtkProp. These methods
 should be contrasted to those that traverse the list of parts using
 GetParts().  GetParts() returns a list of children of this assembly, not
 necessarily with the correct transformation or properties. To use the
 methods below - first invoke InitPathTraversal() followed by repeated
 calls to GetNextPath().  GetNextPath() returns a NULL pointer when the
 list is exhausted.

\item  \verb|obj.GetBounds (double bounds[6])| -  Get the bounds for the assembly as (Xmin,Xmax,Ymin,Ymax,Zmin,Zmax).

\item  \verb|double = obj.GetBounds ()| -  Get the bounds for the assembly as (Xmin,Xmax,Ymin,Ymax,Zmin,Zmax).

\item  \verb|long = obj.GetMTime ()| -  Override default GetMTime method to also consider all of the
 assembly's parts.

\item  \verb|obj.ShallowCopy (vtkProp prop)| -  Shallow copy of an assembly. Overloads the virtual vtkProp method.

\end{itemize}
