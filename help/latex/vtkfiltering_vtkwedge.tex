\section{vtkWedge}

\subsection{Usage}

 vtkWedge is a concrete implementation of vtkCell to represent a linear 3D
 wedge. A wedge consists of two triangular and three quadrilateral faces
 and is defined by the six points (0-5). vtkWedge uses the standard
 isoparametric shape functions for a linear wedge. The wedge is defined
 by the six points (0-5) where (0,1,2) is the base of the wedge which,
 using the right hand rule, forms a triangle whose normal points outward
 (away from the triangular face (3,4,5)).

To create an instance of class vtkWedge, simply
invoke its constructor as follows
\begin{verbatim}
  obj = vtkWedge
\end{verbatim}
\subsection{Methods}

The class vtkWedge has several methods that can be used.
  They are listed below.
Note that the documentation is translated automatically from the VTK sources,
and may not be completely intelligible.  When in doubt, consult the VTK website.
In the methods listed below, \verb|obj| is an instance of the vtkWedge class.
\begin{itemize}
\item  \verb|string = obj.GetClassName ()|

\item  \verb|int = obj.IsA (string name)|

\item  \verb|vtkWedge = obj.NewInstance ()|

\item  \verb|vtkWedge = obj.SafeDownCast (vtkObject o)|

\item  \verb|int = obj.GetCellType ()| -  See the vtkCell API for descriptions of these methods.

\item  \verb|int = obj.GetCellDimension ()| -  See the vtkCell API for descriptions of these methods.

\item  \verb|int = obj.GetNumberOfEdges ()| -  See the vtkCell API for descriptions of these methods.

\item  \verb|int = obj.GetNumberOfFaces ()| -  See the vtkCell API for descriptions of these methods.

\item  \verb|vtkCell = obj.GetEdge (int edgeId)| -  See the vtkCell API for descriptions of these methods.

\item  \verb|vtkCell = obj.GetFace (int faceId)| -  See the vtkCell API for descriptions of these methods.

\item  \verb|int = obj.CellBoundary (int subId, double pcoords[3], vtkIdList pts)| -  See the vtkCell API for descriptions of these methods.

\item  \verb|obj.Contour (double value, vtkDataArray cellScalars, vtkIncrementalPointLocator locator, vtkCellArray verts, vtkCellArray lines, vtkCellArray polys, vtkPointData inPd, vtkPointData outPd, vtkCellData inCd, vtkIdType cellId, vtkCellData outCd)| -  See the vtkCell API for descriptions of these methods.

\item  \verb|int = obj.Triangulate (int index, vtkIdList ptIds, vtkPoints pts)| -  See the vtkCell API for descriptions of these methods.

\item  \verb|obj.Derivatives (int subId, double pcoords[3], double values, int dim, double derivs)| -  See the vtkCell API for descriptions of these methods.

\item  \verb|int = obj.GetParametricCenter (double pcoords[3])| -  Return the center of the wedge in parametric coordinates.

\item  \verb|obj.InterpolateFunctions (double pcoords[3], double weights[6])| -  Compute the interpolation functions/derivatives
 (aka shape functions/derivatives)

\item  \verb|obj.InterpolateDerivs (double pcoords[3], double derivs[18])| -  Compute the interpolation functions/derivatives
 (aka shape functions/derivatives)

\end{itemize}
