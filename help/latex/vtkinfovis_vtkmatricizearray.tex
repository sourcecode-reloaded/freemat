\section{vtkMatricizeArray}

\subsection{Usage}

 Given a sparse input array of arbitrary dimension, creates a sparse output
 matrix (vtkSparseArray<double>) where each column is a slice along an
 arbitrary dimension from the source.

 .SECTION Thanks
 Developed by Timothy M. Shead (tshead@sandia.gov) at Sandia National Laboratories.

To create an instance of class vtkMatricizeArray, simply
invoke its constructor as follows
\begin{verbatim}
  obj = vtkMatricizeArray
\end{verbatim}
\subsection{Methods}

The class vtkMatricizeArray has several methods that can be used.
  They are listed below.
Note that the documentation is translated automatically from the VTK sources,
and may not be completely intelligible.  When in doubt, consult the VTK website.
In the methods listed below, \verb|obj| is an instance of the vtkMatricizeArray class.
\begin{itemize}
\item  \verb|string = obj.GetClassName ()|

\item  \verb|int = obj.IsA (string name)|

\item  \verb|vtkMatricizeArray = obj.NewInstance ()|

\item  \verb|vtkMatricizeArray = obj.SafeDownCast (vtkObject o)|

\item  \verb|vtkIdType = obj.GetSliceDimension ()| -  Returns the 0-numbered dimension that will be mapped to columns in the output

\item  \verb|obj.SetSliceDimension (vtkIdType )| -  Sets the 0-numbered dimension that will be mapped to columns in the output

\end{itemize}
