\section{vtkGradientFilter}

\subsection{Usage}

 Estimates the gradient of a field in a data set.  The gradient calculation
 is dependent on the input dataset type.  The created gradient array
 is of the same type as the array it is calculated from (e.g. point data
 or cell data) as well as data type (e.g. float, double).  At the boundary 
 the gradient is not central differencing.  The output array has 
 3*number of components of the input data array.  The ordering for the 
 output tuple will be {du/dx, du/dy, du/dz, dv/dx, dv/dy, dv/dz, dw/dx,
 dw/dy, dw/dz} for an input array {u, v, w}.

To create an instance of class vtkGradientFilter, simply
invoke its constructor as follows
\begin{verbatim}
  obj = vtkGradientFilter
\end{verbatim}
\subsection{Methods}

The class vtkGradientFilter has several methods that can be used.
  They are listed below.
Note that the documentation is translated automatically from the VTK sources,
and may not be completely intelligible.  When in doubt, consult the VTK website.
In the methods listed below, \verb|obj| is an instance of the vtkGradientFilter class.
\begin{itemize}
\item  \verb|string = obj.GetClassName ()|

\item  \verb|int = obj.IsA (string name)|

\item  \verb|vtkGradientFilter = obj.NewInstance ()|

\item  \verb|vtkGradientFilter = obj.SafeDownCast (vtkObject o)|

\item  \verb|obj.SetInputScalars (int fieldAssociation, string name)| -  These are basically a convenience method that calls SetInputArrayToProcess
 to set the array used as the input scalars.  The fieldAssociation comes
 from the vtkDataObject::FieldAssocations enum.  The fieldAttributeType
 comes from the vtkDataSetAttributes::AttributeTypes enum.

\item  \verb|obj.SetInputScalars (int fieldAssociation, int fieldAttributeType)| -  These are basically a convenience method that calls SetInputArrayToProcess
 to set the array used as the input scalars.  The fieldAssociation comes
 from the vtkDataObject::FieldAssocations enum.  The fieldAttributeType
 comes from the vtkDataSetAttributes::AttributeTypes enum.

\item  \verb|string = obj.GetResultArrayName ()| -  Get/Set the name of the resulting array to create.  If NULL (the
 default) then the output array will be named ''Gradients''.

\item  \verb|obj.SetResultArrayName (string )| -  Get/Set the name of the resulting array to create.  If NULL (the
 default) then the output array will be named ''Gradients''.

\item  \verb|int = obj.GetFasterApproximation ()| -  When this flag is on (default is off), the gradient filter will provide a
 less accurate (but close) algorithm that performs fewer derivative
 calculations (and is therefore faster).  The error contains some smoothing
 of the output data and some possible errors on the boundary.  This
 parameter has no effect when performing the gradient of cell data.
 This only applies if the input grid is a vtkUnstructuredGrid or a
 vtkPolyData.

\item  \verb|obj.SetFasterApproximation (int )| -  When this flag is on (default is off), the gradient filter will provide a
 less accurate (but close) algorithm that performs fewer derivative
 calculations (and is therefore faster).  The error contains some smoothing
 of the output data and some possible errors on the boundary.  This
 parameter has no effect when performing the gradient of cell data.
 This only applies if the input grid is a vtkUnstructuredGrid or a
 vtkPolyData.

\item  \verb|obj.FasterApproximationOn ()| -  When this flag is on (default is off), the gradient filter will provide a
 less accurate (but close) algorithm that performs fewer derivative
 calculations (and is therefore faster).  The error contains some smoothing
 of the output data and some possible errors on the boundary.  This
 parameter has no effect when performing the gradient of cell data.
 This only applies if the input grid is a vtkUnstructuredGrid or a
 vtkPolyData.

\item  \verb|obj.FasterApproximationOff ()| -  When this flag is on (default is off), the gradient filter will provide a
 less accurate (but close) algorithm that performs fewer derivative
 calculations (and is therefore faster).  The error contains some smoothing
 of the output data and some possible errors on the boundary.  This
 parameter has no effect when performing the gradient of cell data.
 This only applies if the input grid is a vtkUnstructuredGrid or a
 vtkPolyData.

\item  \verb|obj.SetComputeVorticity (int )| -  Set the resultant array to be vorticity/curl of the input
 array.  The input array must have 3 components.

\item  \verb|int = obj.GetComputeVorticity ()| -  Set the resultant array to be vorticity/curl of the input
 array.  The input array must have 3 components.

\item  \verb|obj.ComputeVorticityOn ()| -  Set the resultant array to be vorticity/curl of the input
 array.  The input array must have 3 components.

\item  \verb|obj.ComputeVorticityOff ()| -  Set the resultant array to be vorticity/curl of the input
 array.  The input array must have 3 components.

\end{itemize}
