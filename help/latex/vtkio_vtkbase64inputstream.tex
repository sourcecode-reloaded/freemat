\section{vtkBase64InputStream}

\subsection{Usage}

 vtkBase64InputStream implements base64 decoding with the
 vtkInputStream interface.

To create an instance of class vtkBase64InputStream, simply
invoke its constructor as follows
\begin{verbatim}
  obj = vtkBase64InputStream
\end{verbatim}
\subsection{Methods}

The class vtkBase64InputStream has several methods that can be used.
  They are listed below.
Note that the documentation is translated automatically from the VTK sources,
and may not be completely intelligible.  When in doubt, consult the VTK website.
In the methods listed below, \verb|obj| is an instance of the vtkBase64InputStream class.
\begin{itemize}
\item  \verb|string = obj.GetClassName ()|

\item  \verb|int = obj.IsA (string name)|

\item  \verb|vtkBase64InputStream = obj.NewInstance ()|

\item  \verb|vtkBase64InputStream = obj.SafeDownCast (vtkObject o)|

\item  \verb|obj.StartReading ()| -  Called after the stream position has been set by the caller, but
 before any Seek or Read calls.  The stream position should not be
 adjusted by the caller until after an EndReading call.

\item  \verb|int = obj.Seek (long offset)| -  Seek to the given offset in the input data.  Returns 1 for
 success, 0 for failure.

\item  \verb|long = obj.Read (string data, long length)| -  Read input data of the given length.  Returns amount actually
 read.

\item  \verb|obj.EndReading ()| -  Called after all desired calls to Seek and Read have been made.
 After this call, the caller is free to change the position of the
 stream.  Additional reads should not be done until after another
 call to StartReading.

\end{itemize}
