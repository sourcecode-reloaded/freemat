\section{vtkRenderWindow}

\subsection{Usage}

 vtkRenderWindow is an abstract object to specify the behavior of a
 rendering window. A rendering window is a window in a graphical user
 interface where renderers draw their images. Methods are provided to
 synchronize the rendering process, set window size, and control double
 buffering.  The window also allows rendering in stereo.  The interlaced
 render stereo type is for output to a VRex stereo projector.  All of the
 odd horizontal lines are from the left eye, and the even lines are from
 the right eye.  The user has to make the render window aligned with the
 VRex projector, or the eye will be swapped.

To create an instance of class vtkRenderWindow, simply
invoke its constructor as follows
\begin{verbatim}
  obj = vtkRenderWindow
\end{verbatim}
\subsection{Methods}

The class vtkRenderWindow has several methods that can be used.
  They are listed below.
Note that the documentation is translated automatically from the VTK sources,
and may not be completely intelligible.  When in doubt, consult the VTK website.
In the methods listed below, \verb|obj| is an instance of the vtkRenderWindow class.
\begin{itemize}
\item  \verb|string = obj.GetClassName ()|

\item  \verb|int = obj.IsA (string name)|

\item  \verb|vtkRenderWindow = obj.NewInstance ()|

\item  \verb|vtkRenderWindow = obj.SafeDownCast (vtkObject o)|

\item  \verb|obj.AddRenderer (vtkRenderer )| -  Add a renderer to the list of renderers.

\item  \verb|obj.RemoveRenderer (vtkRenderer )| -  Remove a renderer from the list of renderers.

\item  \verb|int = obj.HasRenderer (vtkRenderer )| -  Query if a renderer is in the list of renderers.

\item  \verb|vtkRendererCollection = obj.GetRenderers ()| -  Return the collection of renderers in the render window.

\item  \verb|obj.Render ()| -  Ask each renderer owned by this RenderWindow to render its image and
 synchronize this process.

\item  \verb|obj.Start ()| -  Initialize the rendering process.

\item  \verb|obj.Finalize ()| -  Finalize the rendering process.

\item  \verb|obj.Frame ()| -  A termination method performed at the end of the rendering process
 to do things like swapping buffers (if necessary) or similar actions.

\item  \verb|obj.WaitForCompletion ()| -  Block the thread until the actual rendering is finished().
 Useful for measurement only.

\item  \verb|obj.CopyResultFrame ()| -  Performed at the end of the rendering process to generate image.
 This is typically done right before swapping buffers.

\item  \verb|vtkRenderWindowInteractor = obj.MakeRenderWindowInteractor ()| -  Create an interactor to control renderers in this window. We need
 to know what type of interactor to create, because we might be in
 X Windows or MS Windows.

\item  \verb|obj.HideCursor ()| -  Hide or Show the mouse cursor, it is nice to be able to hide the
 default cursor if you want VTK to display a 3D cursor instead.
 Set cursor position in window (note that (0,0) is the lower left
 corner).

\item  \verb|obj.ShowCursor ()| -  Hide or Show the mouse cursor, it is nice to be able to hide the
 default cursor if you want VTK to display a 3D cursor instead.
 Set cursor position in window (note that (0,0) is the lower left
 corner).

\item  \verb|obj.SetCursorPosition (int , int )| -  Hide or Show the mouse cursor, it is nice to be able to hide the
 default cursor if you want VTK to display a 3D cursor instead.
 Set cursor position in window (note that (0,0) is the lower left
 corner).

\item  \verb|obj.SetCurrentCursor (int )| -  Change the shape of the cursor.

\item  \verb|int = obj.GetCurrentCursor ()| -  Change the shape of the cursor.

\item  \verb|obj.SetFullScreen (int )| -  Turn on/off rendering full screen window size.

\item  \verb|int = obj.GetFullScreen ()| -  Turn on/off rendering full screen window size.

\item  \verb|obj.FullScreenOn ()| -  Turn on/off rendering full screen window size.

\item  \verb|obj.FullScreenOff ()| -  Turn on/off rendering full screen window size.

\item  \verb|obj.SetBorders (int )| -  Turn on/off window manager borders. Typically, you shouldn't turn the
 borders off, because that bypasses the window manager and can cause
 undesirable behavior.

\item  \verb|int = obj.GetBorders ()| -  Turn on/off window manager borders. Typically, you shouldn't turn the
 borders off, because that bypasses the window manager and can cause
 undesirable behavior.

\item  \verb|obj.BordersOn ()| -  Turn on/off window manager borders. Typically, you shouldn't turn the
 borders off, because that bypasses the window manager and can cause
 undesirable behavior.

\item  \verb|obj.BordersOff ()| -  Turn on/off window manager borders. Typically, you shouldn't turn the
 borders off, because that bypasses the window manager and can cause
 undesirable behavior.

\item  \verb|int = obj.GetStereoCapableWindow ()| -  Prescribe that the window be created in a stereo-capable mode. This
 method must be called before the window is realized. Default is off.

\item  \verb|obj.StereoCapableWindowOn ()| -  Prescribe that the window be created in a stereo-capable mode. This
 method must be called before the window is realized. Default is off.

\item  \verb|obj.StereoCapableWindowOff ()| -  Prescribe that the window be created in a stereo-capable mode. This
 method must be called before the window is realized. Default is off.

\item  \verb|obj.SetStereoCapableWindow (int capable)| -  Prescribe that the window be created in a stereo-capable mode. This
 method must be called before the window is realized. Default is off.

\item  \verb|int = obj.GetStereoRender ()| -  Turn on/off stereo rendering.

\item  \verb|obj.SetStereoRender (int stereo)| -  Turn on/off stereo rendering.

\item  \verb|obj.StereoRenderOn ()| -  Turn on/off stereo rendering.

\item  \verb|obj.StereoRenderOff ()| -  Turn on/off stereo rendering.

\item  \verb|obj.SetAlphaBitPlanes (int )| -  Turn on/off the use of alpha bitplanes.

\item  \verb|int = obj.GetAlphaBitPlanes ()| -  Turn on/off the use of alpha bitplanes.

\item  \verb|obj.AlphaBitPlanesOn ()| -  Turn on/off the use of alpha bitplanes.

\item  \verb|obj.AlphaBitPlanesOff ()| -  Turn on/off the use of alpha bitplanes.

\item  \verb|obj.SetPointSmoothing (int )| -  Turn on/off point smoothing. Default is off.
 This must be applied before the first Render.

\item  \verb|int = obj.GetPointSmoothing ()| -  Turn on/off point smoothing. Default is off.
 This must be applied before the first Render.

\item  \verb|obj.PointSmoothingOn ()| -  Turn on/off point smoothing. Default is off.
 This must be applied before the first Render.

\item  \verb|obj.PointSmoothingOff ()| -  Turn on/off point smoothing. Default is off.
 This must be applied before the first Render.

\item  \verb|obj.SetLineSmoothing (int )| -  Turn on/off line smoothing. Default is off.
 This must be applied before the first Render.

\item  \verb|int = obj.GetLineSmoothing ()| -  Turn on/off line smoothing. Default is off.
 This must be applied before the first Render.

\item  \verb|obj.LineSmoothingOn ()| -  Turn on/off line smoothing. Default is off.
 This must be applied before the first Render.

\item  \verb|obj.LineSmoothingOff ()| -  Turn on/off line smoothing. Default is off.
 This must be applied before the first Render.

\item  \verb|obj.SetPolygonSmoothing (int )| -  Turn on/off polygon smoothing. Default is off.
 This must be applied before the first Render.

\item  \verb|int = obj.GetPolygonSmoothing ()| -  Turn on/off polygon smoothing. Default is off.
 This must be applied before the first Render.

\item  \verb|obj.PolygonSmoothingOn ()| -  Turn on/off polygon smoothing. Default is off.
 This must be applied before the first Render.

\item  \verb|obj.PolygonSmoothingOff ()| -  Turn on/off polygon smoothing. Default is off.
 This must be applied before the first Render.

\item  \verb|int = obj.GetStereoType ()| -  Set/Get what type of stereo rendering to use.  CrystalEyes
 mode uses frame-sequential capabilities available in OpenGL
 to drive LCD shutter glasses and stereo projectors.  RedBlue
 mode is a simple type of stereo for use with red-blue glasses.
 Anaglyph mode is a superset of RedBlue mode, but the color
 output channels can be configured using the AnaglyphColorMask
 and the color of the original image can be (somewhat) maintained
 using AnaglyphColorSaturation;  the default colors for Anaglyph
 mode is red-cyan.  Interlaced stereo mode produces a composite
 image where horizontal lines alternate between left and right
 views.  StereoLeft and StereoRight modes choose one or the other
 stereo view.  Dresden mode is yet another stereoscopic
 interleaving.

\item  \verb|obj.SetStereoType (int )| -  Set/Get what type of stereo rendering to use.  CrystalEyes
 mode uses frame-sequential capabilities available in OpenGL
 to drive LCD shutter glasses and stereo projectors.  RedBlue
 mode is a simple type of stereo for use with red-blue glasses.
 Anaglyph mode is a superset of RedBlue mode, but the color
 output channels can be configured using the AnaglyphColorMask
 and the color of the original image can be (somewhat) maintained
 using AnaglyphColorSaturation;  the default colors for Anaglyph
 mode is red-cyan.  Interlaced stereo mode produces a composite
 image where horizontal lines alternate between left and right
 views.  StereoLeft and StereoRight modes choose one or the other
 stereo view.  Dresden mode is yet another stereoscopic
 interleaving.

\item  \verb|obj.SetStereoTypeToCrystalEyes ()| -  Set/Get what type of stereo rendering to use.  CrystalEyes
 mode uses frame-sequential capabilities available in OpenGL
 to drive LCD shutter glasses and stereo projectors.  RedBlue
 mode is a simple type of stereo for use with red-blue glasses.
 Anaglyph mode is a superset of RedBlue mode, but the color
 output channels can be configured using the AnaglyphColorMask
 and the color of the original image can be (somewhat) maintained
 using AnaglyphColorSaturation;  the default colors for Anaglyph
 mode is red-cyan.  Interlaced stereo mode produces a composite
 image where horizontal lines alternate between left and right
 views.  StereoLeft and StereoRight modes choose one or the other
 stereo view.  Dresden mode is yet another stereoscopic
 interleaving.

\item  \verb|obj.SetStereoTypeToRedBlue ()| -  Set/Get what type of stereo rendering to use.  CrystalEyes
 mode uses frame-sequential capabilities available in OpenGL
 to drive LCD shutter glasses and stereo projectors.  RedBlue
 mode is a simple type of stereo for use with red-blue glasses.
 Anaglyph mode is a superset of RedBlue mode, but the color
 output channels can be configured using the AnaglyphColorMask
 and the color of the original image can be (somewhat) maintained
 using AnaglyphColorSaturation;  the default colors for Anaglyph
 mode is red-cyan.  Interlaced stereo mode produces a composite
 image where horizontal lines alternate between left and right
 views.  StereoLeft and StereoRight modes choose one or the other
 stereo view.  Dresden mode is yet another stereoscopic
 interleaving.

\item  \verb|obj.SetStereoTypeToInterlaced ()| -  Set/Get what type of stereo rendering to use.  CrystalEyes
 mode uses frame-sequential capabilities available in OpenGL
 to drive LCD shutter glasses and stereo projectors.  RedBlue
 mode is a simple type of stereo for use with red-blue glasses.
 Anaglyph mode is a superset of RedBlue mode, but the color
 output channels can be configured using the AnaglyphColorMask
 and the color of the original image can be (somewhat) maintained
 using AnaglyphColorSaturation;  the default colors for Anaglyph
 mode is red-cyan.  Interlaced stereo mode produces a composite
 image where horizontal lines alternate between left and right
 views.  StereoLeft and StereoRight modes choose one or the other
 stereo view.  Dresden mode is yet another stereoscopic
 interleaving.

\item  \verb|obj.SetStereoTypeToLeft ()| -  Set/Get what type of stereo rendering to use.  CrystalEyes
 mode uses frame-sequential capabilities available in OpenGL
 to drive LCD shutter glasses and stereo projectors.  RedBlue
 mode is a simple type of stereo for use with red-blue glasses.
 Anaglyph mode is a superset of RedBlue mode, but the color
 output channels can be configured using the AnaglyphColorMask
 and the color of the original image can be (somewhat) maintained
 using AnaglyphColorSaturation;  the default colors for Anaglyph
 mode is red-cyan.  Interlaced stereo mode produces a composite
 image where horizontal lines alternate between left and right
 views.  StereoLeft and StereoRight modes choose one or the other
 stereo view.  Dresden mode is yet another stereoscopic
 interleaving.

\item  \verb|obj.SetStereoTypeToRight ()| -  Set/Get what type of stereo rendering to use.  CrystalEyes
 mode uses frame-sequential capabilities available in OpenGL
 to drive LCD shutter glasses and stereo projectors.  RedBlue
 mode is a simple type of stereo for use with red-blue glasses.
 Anaglyph mode is a superset of RedBlue mode, but the color
 output channels can be configured using the AnaglyphColorMask
 and the color of the original image can be (somewhat) maintained
 using AnaglyphColorSaturation;  the default colors for Anaglyph
 mode is red-cyan.  Interlaced stereo mode produces a composite
 image where horizontal lines alternate between left and right
 views.  StereoLeft and StereoRight modes choose one or the other
 stereo view.  Dresden mode is yet another stereoscopic
 interleaving.

\item  \verb|obj.SetStereoTypeToDresden ()| -  Set/Get what type of stereo rendering to use.  CrystalEyes
 mode uses frame-sequential capabilities available in OpenGL
 to drive LCD shutter glasses and stereo projectors.  RedBlue
 mode is a simple type of stereo for use with red-blue glasses.
 Anaglyph mode is a superset of RedBlue mode, but the color
 output channels can be configured using the AnaglyphColorMask
 and the color of the original image can be (somewhat) maintained
 using AnaglyphColorSaturation;  the default colors for Anaglyph
 mode is red-cyan.  Interlaced stereo mode produces a composite
 image where horizontal lines alternate between left and right
 views.  StereoLeft and StereoRight modes choose one or the other
 stereo view.  Dresden mode is yet another stereoscopic
 interleaving.

\item  \verb|obj.SetStereoTypeToAnaglyph ()| -  Set/Get what type of stereo rendering to use.  CrystalEyes
 mode uses frame-sequential capabilities available in OpenGL
 to drive LCD shutter glasses and stereo projectors.  RedBlue
 mode is a simple type of stereo for use with red-blue glasses.
 Anaglyph mode is a superset of RedBlue mode, but the color
 output channels can be configured using the AnaglyphColorMask
 and the color of the original image can be (somewhat) maintained
 using AnaglyphColorSaturation;  the default colors for Anaglyph
 mode is red-cyan.  Interlaced stereo mode produces a composite
 image where horizontal lines alternate between left and right
 views.  StereoLeft and StereoRight modes choose one or the other
 stereo view.  Dresden mode is yet another stereoscopic
 interleaving.

\item  \verb|obj.SetStereoTypeToCheckerboard ()|

\item  \verb|string = obj.GetStereoTypeAsString ()|

\item  \verb|obj.StereoUpdate ()| -  Update the system, if needed, due to stereo rendering. For some stereo
 methods, subclasses might need to switch some hardware settings here.

\item  \verb|obj.StereoMidpoint ()| -  Intermediate method performs operations required between the rendering
 of the left and right eye.

\item  \verb|obj.StereoRenderComplete ()| -  Handles work required once both views have been rendered when using
 stereo rendering.

\item  \verb|obj.SetAnaglyphColorSaturation (float )|

\item  \verb|float = obj.GetAnaglyphColorSaturationMinValue ()|

\item  \verb|float = obj.GetAnaglyphColorSaturationMaxValue ()|

\item  \verb|float = obj.GetAnaglyphColorSaturation ()|

\item  \verb|obj.SetAnaglyphColorMask (int , int )|

\item  \verb|obj.SetAnaglyphColorMask (int  a[2])|

\item  \verb|int = obj. GetAnaglyphColorMask ()|

\item  \verb|obj.WindowRemap ()| -  Remap the rendering window. This probably only works on UNIX right now.
 It is useful for changing properties that can't normally be changed
 once the window is up.

\item  \verb|obj.SetSwapBuffers (int )| -  Turn on/off buffer swapping between images.

\item  \verb|int = obj.GetSwapBuffers ()| -  Turn on/off buffer swapping between images.

\item  \verb|obj.SwapBuffersOn ()| -  Turn on/off buffer swapping between images.

\item  \verb|obj.SwapBuffersOff ()| -  Turn on/off buffer swapping between images.

\item  \verb|int = obj.SetPixelData (int x, int y, int x2, int y2, string data, int front)| -  Set/Get the pixel data of an image, transmitted as RGBRGBRGB. The
 front argument indicates if the front buffer should be used or the back
 buffer. It is the caller's responsibility to delete the resulting
 array. It is very important to realize that the memory in this array
 is organized from the bottom of the window to the top. The origin
 of the screen is in the lower left corner. The y axis increases as
 you go up the screen. So the storage of pixels is from left to right
 and from bottom to top.
 (x,y) is any corner of the rectangle. (x2,y2) is its opposite corner on
 the diagonal.

\item  \verb|int = obj.SetPixelData (int x, int y, int x2, int y2, vtkUnsignedCharArray data, int front)| -  Set/Get the pixel data of an image, transmitted as RGBRGBRGB. The
 front argument indicates if the front buffer should be used or the back
 buffer. It is the caller's responsibility to delete the resulting
 array. It is very important to realize that the memory in this array
 is organized from the bottom of the window to the top. The origin
 of the screen is in the lower left corner. The y axis increases as
 you go up the screen. So the storage of pixels is from left to right
 and from bottom to top.
 (x,y) is any corner of the rectangle. (x2,y2) is its opposite corner on
 the diagonal.

\item  \verb|int = obj.GetRGBAPixelData (int x, int y, int x2, int y2, int front, vtkFloatArray data)| -  Same as Get/SetPixelData except that the image also contains an alpha
 component. The image is transmitted as RGBARGBARGBA... each of which is a
 float value. The ''blend'' parameter controls whether the SetRGBAPixelData
 method blends the data with the previous contents of the frame buffer
 or completely replaces the frame buffer data.

\item  \verb|int = obj.SetRGBAPixelData (int x, int y, int x2, int y2, float , int front, int blend)| -  Same as Get/SetPixelData except that the image also contains an alpha
 component. The image is transmitted as RGBARGBARGBA... each of which is a
 float value. The ''blend'' parameter controls whether the SetRGBAPixelData
 method blends the data with the previous contents of the frame buffer
 or completely replaces the frame buffer data.

\item  \verb|int = obj.SetRGBAPixelData (int , int , int , int , vtkFloatArray , int , int blend)| -  Same as Get/SetPixelData except that the image also contains an alpha
 component. The image is transmitted as RGBARGBARGBA... each of which is a
 float value. The ''blend'' parameter controls whether the SetRGBAPixelData
 method blends the data with the previous contents of the frame buffer
 or completely replaces the frame buffer data.

\item  \verb|obj.ReleaseRGBAPixelData (float data)| -  Same as Get/SetPixelData except that the image also contains an alpha
 component. The image is transmitted as RGBARGBARGBA... each of which is a
 float value. The ''blend'' parameter controls whether the SetRGBAPixelData
 method blends the data with the previous contents of the frame buffer
 or completely replaces the frame buffer data.

\item  \verb|int = obj.GetRGBACharPixelData (int x, int y, int x2, int y2, int front, vtkUnsignedCharArray data)| -  Same as Get/SetPixelData except that the image also contains an alpha
 component. The image is transmitted as RGBARGBARGBA... each of which is a
 float value. The ''blend'' parameter controls whether the SetRGBAPixelData
 method blends the data with the previous contents of the frame buffer
 or completely replaces the frame buffer data.

\item  \verb|int = obj.SetRGBACharPixelData (int x, int y, int x2, int y2, string data, int front, int blend)| -  Same as Get/SetPixelData except that the image also contains an alpha
 component. The image is transmitted as RGBARGBARGBA... each of which is a
 float value. The ''blend'' parameter controls whether the SetRGBAPixelData
 method blends the data with the previous contents of the frame buffer
 or completely replaces the frame buffer data.

\item  \verb|int = obj.SetRGBACharPixelData (int x, int y, int x2, int y2, vtkUnsignedCharArray data, int front, int blend)| -  Same as Get/SetPixelData except that the image also contains an alpha
 component. The image is transmitted as RGBARGBARGBA... each of which is a
 float value. The ''blend'' parameter controls whether the SetRGBAPixelData
 method blends the data with the previous contents of the frame buffer
 or completely replaces the frame buffer data.

\item  \verb|int = obj.GetZbufferData (int x, int y, int x2, int y2, float z)| -  Set/Get the zbuffer data from the frame buffer.
 (x,y) is any corner of the rectangle. (x2,y2) is its opposite corner on
 the diagonal.

\item  \verb|int = obj.GetZbufferData (int x, int y, int x2, int y2, vtkFloatArray z)| -  Set/Get the zbuffer data from the frame buffer.
 (x,y) is any corner of the rectangle. (x2,y2) is its opposite corner on
 the diagonal.

\item  \verb|int = obj.SetZbufferData (int x, int y, int x2, int y2, float z)| -  Set/Get the zbuffer data from the frame buffer.
 (x,y) is any corner of the rectangle. (x2,y2) is its opposite corner on
 the diagonal.

\item  \verb|int = obj.SetZbufferData (int x, int y, int x2, int y2, vtkFloatArray z)| -  Set/Get the zbuffer data from the frame buffer.
 (x,y) is any corner of the rectangle. (x2,y2) is its opposite corner on
 the diagonal.

\item  \verb|float = obj.GetZbufferDataAtPoint (int x, int y)| -  Set the number of frames for doing antialiasing. The default is
 zero. Typically five or six will yield reasonable results without
 taking too long.

\item  \verb|int = obj.GetAAFrames ()| -  Set the number of frames for doing antialiasing. The default is
 zero. Typically five or six will yield reasonable results without
 taking too long.

\item  \verb|obj.SetAAFrames (int )| -  Set the number of frames for doing antialiasing. The default is
 zero. Typically five or six will yield reasonable results without
 taking too long.

\item  \verb|int = obj.GetFDFrames ()| -  Set the number of frames for doing focal depth. The default is zero.
 Depending on how your scene is organized you can get away with as
 few as four frames for focal depth or you might need thirty.
 One thing to note is that if you are using focal depth frames,
 then you will not need many (if any) frames for antialiasing.

\item  \verb|obj.SetFDFrames (int )| -  Set the number of frames for doing focal depth. The default is zero.
 Depending on how your scene is organized you can get away with as
 few as four frames for focal depth or you might need thirty.
 One thing to note is that if you are using focal depth frames,
 then you will not need many (if any) frames for antialiasing.

\item  \verb|int = obj.GetSubFrames ()| -  Set the number of sub frames for doing motion blur. The default is zero.
 Once this is set greater than one, you will no longer see a new frame
 for every Render().  If you set this to five, you will need to do
 five Render() invocations before seeing the result. This isn't
 very impressive unless something is changing between the Renders.
 Changing this value may reset the current subframe count.

\item  \verb|obj.SetSubFrames (int subFrames)| -  Set the number of sub frames for doing motion blur. The default is zero.
 Once this is set greater than one, you will no longer see a new frame
 for every Render().  If you set this to five, you will need to do
 five Render() invocations before seeing the result. This isn't
 very impressive unless something is changing between the Renders.
 Changing this value may reset the current subframe count.

\item  \verb|int = obj.GetNeverRendered ()| -  This flag is set if the window hasn't rendered since it was created

\item  \verb|int = obj.GetAbortRender ()| -  This is a flag that can be set to interrupt a rendering that is in
 progress.

\item  \verb|obj.SetAbortRender (int )| -  This is a flag that can be set to interrupt a rendering that is in
 progress.

\item  \verb|int = obj.GetInAbortCheck ()| -  This is a flag that can be set to interrupt a rendering that is in
 progress.

\item  \verb|obj.SetInAbortCheck (int )| -  This is a flag that can be set to interrupt a rendering that is in
 progress.

\item  \verb|int = obj.CheckAbortStatus ()| -  This is a flag that can be set to interrupt a rendering that is in
 progress.

\item  \verb|int = obj.GetIsPicking ()|

\item  \verb|obj.SetIsPicking (int )|

\item  \verb|obj.IsPickingOn ()|

\item  \verb|obj.IsPickingOff ()|

\item  \verb|int = obj.GetEventPending ()| -  Check to see if a mouse button has been pressed.  All other events
 are ignored by this method.  Ideally, you want to abort the render
 on any event which causes the DesiredUpdateRate to switch from
 a high-quality rate to a more interactive rate.

\item  \verb|int = obj.CheckInRenderStatus ()| -  Clear status (after an exception was thrown for example)

\item  \verb|obj.ClearInRenderStatus ()| -  Set/Get the desired update rate. This is used with
 the vtkLODActor class. When using level of detail actors you
 need to specify what update rate you require. The LODActors then
 will pick the correct resolution to meet your desired update rate
 in frames per second. A value of zero indicates that they can use
 all the time they want to.

\item  \verb|obj.SetDesiredUpdateRate (double )| -  Set/Get the desired update rate. This is used with
 the vtkLODActor class. When using level of detail actors you
 need to specify what update rate you require. The LODActors then
 will pick the correct resolution to meet your desired update rate
 in frames per second. A value of zero indicates that they can use
 all the time they want to.

\item  \verb|double = obj.GetDesiredUpdateRate ()| -  Set/Get the desired update rate. This is used with
 the vtkLODActor class. When using level of detail actors you
 need to specify what update rate you require. The LODActors then
 will pick the correct resolution to meet your desired update rate
 in frames per second. A value of zero indicates that they can use
 all the time they want to.

\item  \verb|int = obj.GetNumberOfLayers ()| -  Get the number of layers for renderers.  Each renderer should have
 its layer set individually.  Some algorithms iterate through all layers,
 so it is not wise to set the number of layers to be exorbitantly large
 (say bigger than 100).

\item  \verb|obj.SetNumberOfLayers (int )| -  Get the number of layers for renderers.  Each renderer should have
 its layer set individually.  Some algorithms iterate through all layers,
 so it is not wise to set the number of layers to be exorbitantly large
 (say bigger than 100).

\item  \verb|int = obj.GetNumberOfLayersMinValue ()| -  Get the number of layers for renderers.  Each renderer should have
 its layer set individually.  Some algorithms iterate through all layers,
 so it is not wise to set the number of layers to be exorbitantly large
 (say bigger than 100).

\item  \verb|int = obj.GetNumberOfLayersMaxValue ()| -  Get the number of layers for renderers.  Each renderer should have
 its layer set individually.  Some algorithms iterate through all layers,
 so it is not wise to set the number of layers to be exorbitantly large
 (say bigger than 100).

\item  \verb|vtkRenderWindowInteractor = obj.GetInteractor ()| -  Get the interactor associated with this render window

\item  \verb|obj.SetInteractor (vtkRenderWindowInteractor )| -  Set the interactor to the render window

\item  \verb|obj.UnRegister (vtkObjectBase o)| -  This Method detects loops of RenderWindow<->Interactor,
 so objects are freed properly.

\item  \verb|obj.SetWindowInfo (string )| -  Dummy stubs for vtkWindow API.

\item  \verb|obj.SetNextWindowInfo (string )| -  Dummy stubs for vtkWindow API.

\item  \verb|obj.SetParentInfo (string )| -  Dummy stubs for vtkWindow API.

\item  \verb|obj.MakeCurrent ()| -  Attempt to make this window the current graphics context for the calling
 thread.

\item  \verb|bool = obj.IsCurrent ()| -  Tells if this window is the current graphics context for the calling
 thread.

\item  \verb|obj.SetForceMakeCurrent ()| -  If called, allow MakeCurrent() to skip cache-check when called.
 MakeCurrent() reverts to original behavior of cache-checking
 on the next render.

\item  \verb|string = obj.ReportCapabilities ()| -  Get report of capabilities for the render window

\item  \verb|int = obj.SupportsOpenGL ()| -  Does this render window support OpenGL? 0-false, 1-true

\item  \verb|int = obj.IsDirect ()| -  Is this render window using hardware acceleration? 0-false, 1-true

\item  \verb|int = obj.GetDepthBufferSize ()| -  This method should be defined by the subclass. How many bits of
 precision are there in the zbuffer?

\item  \verb|int = obj.GetColorBufferSizes (int rgba)| -  Get the size of the color buffer.
 Returns 0 if not able to determine otherwise sets R G B and A into buffer.

\item  \verb|vtkPainterDeviceAdapter = obj.GetPainterDeviceAdapter ()| -  Get the vtkPainterDeviceAdapter which can be used to paint on
 this render window.

\item  \verb|obj.SetMultiSamples (int )| -  Set / Get the number of multisamples to use for hardware antialiasing.

\item  \verb|int = obj.GetMultiSamples ()| -  Set / Get the number of multisamples to use for hardware antialiasing.

\item  \verb|obj.SetStencilCapable (int )| -  Set / Get the availability of the stencil buffer.

\item  \verb|int = obj.GetStencilCapable ()| -  Set / Get the availability of the stencil buffer.

\item  \verb|obj.StencilCapableOn ()| -  Set / Get the availability of the stencil buffer.

\item  \verb|obj.StencilCapableOff ()| -  Set / Get the availability of the stencil buffer.

\item  \verb|obj.SetReportGraphicErrors (int )| -  Turn on/off report of graphic errors. Initial value is false (off).
 This flag is used by vtkGraphicErrorMacro.

\item  \verb|int = obj.GetReportGraphicErrors ()| -  Turn on/off report of graphic errors. Initial value is false (off).
 This flag is used by vtkGraphicErrorMacro.

\item  \verb|obj.ReportGraphicErrorsOn ()| -  Turn on/off report of graphic errors. Initial value is false (off).
 This flag is used by vtkGraphicErrorMacro.

\item  \verb|obj.ReportGraphicErrorsOff ()| -  Turn on/off report of graphic errors. Initial value is false (off).
 This flag is used by vtkGraphicErrorMacro.

\item  \verb|obj.CheckGraphicError ()| -  Update graphic error status, regardless of ReportGraphicErrors flag.
 It means this method can be used in any context and is not restricted to
 debug mode.

\item  \verb|int = obj.HasGraphicError ()| -  Return the last graphic error status. Initial value is false.

\item  \verb|string = obj.GetLastGraphicErrorString ()| -  Return a string matching the last graphic error status.

\end{itemize}
