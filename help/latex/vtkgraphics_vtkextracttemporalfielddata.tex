\section{vtkExtractTemporalFieldData}

\subsection{Usage}

 vtkExtractTemporalFieldData extracts arrays from the input vtkFieldData. 
 These arrays are assumed to contain temporal data, where the nth tuple 
 contains the value for the nth timestep. 
 The output is a 1D rectilinear grid where the 
 XCoordinates correspond to time (the same array is also copied to
 a point array named Time or TimeData (if Time exists in the input).
 This algorithm does not produce a TIME\_STEPS or TIME\_RANGE information
 because it works across time. 
 .Section Caveat
 vtkExtractTemporalFieldData puts a vtkOnePieceExtentTranslator in the
 output during RequestInformation(). As a result, the same whole 
 extented is produced independent of the piece request.
 This algorithm works only with source that produce TIME\_STEPS().
 Continuous time range is not yet supported.

To create an instance of class vtkExtractTemporalFieldData, simply
invoke its constructor as follows
\begin{verbatim}
  obj = vtkExtractTemporalFieldData
\end{verbatim}
\subsection{Methods}

The class vtkExtractTemporalFieldData has several methods that can be used.
  They are listed below.
Note that the documentation is translated automatically from the VTK sources,
and may not be completely intelligible.  When in doubt, consult the VTK website.
In the methods listed below, \verb|obj| is an instance of the vtkExtractTemporalFieldData class.
\begin{itemize}
\item  \verb|string = obj.GetClassName ()|

\item  \verb|int = obj.IsA (string name)|

\item  \verb|vtkExtractTemporalFieldData = obj.NewInstance ()|

\item  \verb|vtkExtractTemporalFieldData = obj.SafeDownCast (vtkObject o)|

\item  \verb|int = obj.GetNumberOfTimeSteps ()| -  Get the number of time steps

\end{itemize}
