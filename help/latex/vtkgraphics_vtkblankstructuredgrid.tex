\section{vtkBlankStructuredGrid}

\subsection{Usage}

 vtkBlankStructuredGrid is a filter that sets the blanking field in a 
 vtkStructuredGrid dataset. The blanking field is set by examining a
 specified point attribute data array (e.g., scalars) and converting
 values in the data array to either a ''1'' (visible) or ''0'' (blanked) value
 in the blanking array. The values to be blanked are specified by giving
 a min/max range. All data values in the data array indicated and laying
 within the range specified (inclusive on both ends) are translated to 
 a ''off'' blanking value.

To create an instance of class vtkBlankStructuredGrid, simply
invoke its constructor as follows
\begin{verbatim}
  obj = vtkBlankStructuredGrid
\end{verbatim}
\subsection{Methods}

The class vtkBlankStructuredGrid has several methods that can be used.
  They are listed below.
Note that the documentation is translated automatically from the VTK sources,
and may not be completely intelligible.  When in doubt, consult the VTK website.
In the methods listed below, \verb|obj| is an instance of the vtkBlankStructuredGrid class.
\begin{itemize}
\item  \verb|string = obj.GetClassName ()|

\item  \verb|int = obj.IsA (string name)|

\item  \verb|vtkBlankStructuredGrid = obj.NewInstance ()|

\item  \verb|vtkBlankStructuredGrid = obj.SafeDownCast (vtkObject o)|

\item  \verb|obj.SetMinBlankingValue (double )| -  Specify the lower data value in the data array specified which will be
 converted into a ''blank'' (or off) value in the blanking array.

\item  \verb|double = obj.GetMinBlankingValue ()| -  Specify the lower data value in the data array specified which will be
 converted into a ''blank'' (or off) value in the blanking array.

\item  \verb|obj.SetMaxBlankingValue (double )| -  Specify the upper data value in the data array specified which will be
 converted into a ''blank'' (or off) value in the blanking array.

\item  \verb|double = obj.GetMaxBlankingValue ()| -  Specify the upper data value in the data array specified which will be
 converted into a ''blank'' (or off) value in the blanking array.

\item  \verb|obj.SetArrayName (string )| -  Specify the data array name to use to generate the blanking
 field. Alternatively, you can specify the array id. (If both are set,
 the array name takes precedence.)

\item  \verb|string = obj.GetArrayName ()| -  Specify the data array name to use to generate the blanking
 field. Alternatively, you can specify the array id. (If both are set,
 the array name takes precedence.)

\item  \verb|obj.SetArrayId (int )| -  Specify the data array id to use to generate the blanking
 field. Alternatively, you can specify the array name. (If both are set,
 the array name takes precedence.)

\item  \verb|int = obj.GetArrayId ()| -  Specify the data array id to use to generate the blanking
 field. Alternatively, you can specify the array name. (If both are set,
 the array name takes precedence.)

\item  \verb|obj.SetComponent (int )| -  Specify the component in the data array to use to generate the blanking
 field.

\item  \verb|int = obj.GetComponentMinValue ()| -  Specify the component in the data array to use to generate the blanking
 field.

\item  \verb|int = obj.GetComponentMaxValue ()| -  Specify the component in the data array to use to generate the blanking
 field.

\item  \verb|int = obj.GetComponent ()| -  Specify the component in the data array to use to generate the blanking
 field.

\end{itemize}
