\section{THREADKILL Halt execution of a thread}

\subsection{Usage}

The \verb|threadkill| function stops (or attempts to stop) execution
of the given thread.  It works only for functions defined in M-files
(i.e., not for built in or imported functions), and it works by 
setting a flag that causes the thread to stop execution at the next
available statement.  The syntax for this function is 
\begin{verbatim}
  threadkill(handle)
\end{verbatim}
where \verb|handle| is the value returned by a \verb|threadnew| call.  
Note that the \verb|threadkill| function returns immediately.  It 
is still your responsibility to call \verb|threadfree| to free
the thread you have halted.

You cannot kill the main thread (thread id \verb|1|).
\subsection{Example}

Here is an example of stopping a runaway thread using \verb|threadkill|.
Note that the thread function in this case is an M-file function.
We start by setting up a free running counter, where we can access 
the counter from the global variables.  
\begin{verbatim}
    freecount.m
function freecount
  global count
  if (~exist('count')) count = 0; end  % Initialize the counter
  while (1)
    count = count + 1;                 % Update the counter
  end
\end{verbatim}
We now launch this function in a thread, and use \verb|threadkill| to
stop it:
\begin{verbatim}
--> a = threadnew;
--> global count                   % register the global variable count
--> count = 0;
--> threadstart(a,'freecount',0)   % start the thread
--> count                          % it is counting

ans = 

 3745 

--> sleep(1)                       % Wait a bit
--> count                          % it is still counting

ans = 

 114900 

--> threadkill(a)                  % kill the counter
--> threadwait(a,1000)             % wait for it to finish

ans = 

 1 

--> count                          % The count will no longer increase

ans = 

 114901 

--> sleep(1)
--> count

ans = 

 114901 

--> threadfree(a)
\end{verbatim}
