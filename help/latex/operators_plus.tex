\section{PLUS Addition Operator}

\subsection{Usage}

Adds two numerical arrays (elementwise) together.  There are two forms
for its use, both with the same general syntax:
\begin{verbatim}
  y = a + b
\end{verbatim}
where \verb|a| and \verb|b| are \verb|n|-dimensional arrays of numerical type.  In the
first case, the two arguments are the same size, in which case, the 
output \verb|y| is the same size as the inputs, and is the element-wise the sum 
of \verb|a| and \verb|b|.  In the second case, either \verb|a| or \verb|b| is a scalar, 
in which case \verb|y| is the same size as the larger argument,
and is the sum of the scalar added to each element of the other argument.

The rules for manipulating types has changed in FreeMat 4.0.  See \verb|typerules|
for more details.

\subsection{Function Internals}

There are three formulae for the addition operator, depending on the
sizes of the three arguments.  In the most general case, in which 
the two arguments are the same size, the output is computed via:
\[
y(m_1,\ldots,m_d) = a(m_1,\ldots,m_d) + b(m_1,\ldots,m_d)
\]
If \verb|a| is a scalar, then the output is computed via
\[
y(m_1,\ldots,m_d) = a + b(m_1,\ldots,m_d).
\]
On the other hand, if \verb|b| is a scalar, then the output is computed via
\[
y(m_1,\ldots,m_d) = a(m_1,\ldots,m_d) + b.
\]
\subsection{Examples}

Here are some examples of using the addition operator.  First, a 
straight-forward usage of the plus operator.  The first example
is straightforward:
\begin{verbatim}
--> 3 + 8

ans = 
 11 
\end{verbatim}
Next, we add a scalar to a vector of values:
\begin{verbatim}
--> 3.1 + [2,4,5,6,7]

ans = 
    5.1000    7.1000    8.1000    9.1000   10.1000 
\end{verbatim}
With complex values
\begin{verbatim}
--> a = 3 + 4*i

a = 
   3.0000 +  4.0000i 

--> b = a + 2

b = 
   5.0000 +  4.0000i 
\end{verbatim}
Finally, the element-wise version:
\begin{verbatim}
--> a = [1,2;3,4]

a = 
 1 2 
 3 4 

--> b = [2,3;6,7]

b = 
 2 3 
 6 7 

--> c = a + b

c = 
  3  5 
  9 11 
\end{verbatim}
