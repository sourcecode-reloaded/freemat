\section{vtkNonLinearCell}

\subsection{Usage}

 vtkNonLinearCell is an abstract superclass for non-linear cell types.
 Cells that are a direct subclass of vtkCell or vtkCell3D are linear;
 cells that are a subclass of vtkNonLinearCell have non-linear interpolation
 functions. Non-linear cells require special treatment when tessellating
 or converting to graphics primitives. Note that the linearity of the cell
 is a function of whether the cell needs tessellation, which does not
 strictly correlate with interpolation order (e.g., vtkHexahedron has
 non-linear interpolation functions (a product of three linear functions
 in r-s-t) even thought vtkHexahedron is considered linear.)

To create an instance of class vtkNonLinearCell, simply
invoke its constructor as follows
\begin{verbatim}
  obj = vtkNonLinearCell
\end{verbatim}
\subsection{Methods}

The class vtkNonLinearCell has several methods that can be used.
  They are listed below.
Note that the documentation is translated automatically from the VTK sources,
and may not be completely intelligible.  When in doubt, consult the VTK website.
In the methods listed below, \verb|obj| is an instance of the vtkNonLinearCell class.
\begin{itemize}
\item  \verb|string = obj.GetClassName ()|

\item  \verb|int = obj.IsA (string name)|

\item  \verb|vtkNonLinearCell = obj.NewInstance ()|

\item  \verb|vtkNonLinearCell = obj.SafeDownCast (vtkObject o)|

\item  \verb|int = obj.IsLinear ()|

\end{itemize}
