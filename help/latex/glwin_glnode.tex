\section{GLNODE Create a GL Node}

\subsection{Usage}

Define a GL Node.  A GL Node is an object that can be displayed
in a GL Window.  It is defined by a triangular mesh of vertices.
It must also have a material that defines its appearance (i.e.
color, shininess, etc.).  The syntax for the \verb|glnode| command
is 
\begin{verbatim}
  glnode(name,material,pointset)  
\end{verbatim}
where \verb|material| is the name of a material that has already been
defined with \verb|gldefmaterial|, \verb|pointset| is a \verb|3 x N| matrix
of points that define the geometry of the object.  Note that the points
are assumed to be connected in triangular facts, with the points
defined counter clock-wise as seen from the outside of the facet.
\verb|FreeMat| will compute the normals.  The \verb|name| argument must
be unique.  If you want multiple instances of a given \verb|glnode|
in your GLWindow, that is fine, as instances of a \verb|glnode| are
created through a \verb|glassembly|.  
