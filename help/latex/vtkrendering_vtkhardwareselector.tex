\section{vtkHardwareSelector}

\subsection{Usage}

 vtkHardwareSelector is a helper that orchestrates color buffer based
 selection. This relies on OpenGL. 
 vtkHardwareSelector can be used to select visible cells or points within a
 given rectangle of the RenderWindow.
 To use it, call in order:
  SetRenderer() - to select the renderer in which we
 want to select the cells/points.
  SetArea() - to set the rectangular region in the render window to select
 in.
  SetFieldAssociation() -  to select the attribute to select i.e.
 cells/points etc. 
  Finally, call Select().
 Select will cause the attached vtkRenderer to render in a special color mode,
 where each cell/point is given it own color so that later inspection of the 
 Rendered Pixels can determine what cells are visible. Select() returns a new
 vtkSelection instance with the cells/points selected.

 Limitations:
 Antialiasing will break this class. If your graphics card settings force
 their use this class will return invalid results.

 Currently only cells from PolyDataMappers can be selected from. When 
 vtkRenderer::Selector is non-null vtkPainterPolyDataMapper uses the
 vtkHardwareSelectionPolyDataPainter which make appropriate calls to
 BeginRenderProp(), EndRenderProp(), RenderAttributeId() to render colors
 correctly. Until alternatives to vtkHardwareSelectionPolyDataPainter
 exist that can do a similar coloration of other vtkDataSet types, only
 polygonal data can be selected. If you need to select other data types,
 consider using vtkDataSetMapper and turning on it's PassThroughCellIds 
 feature, or using vtkFrustumExtractor.

 Only Opaque geometry in Actors is selected from. Assemblies and LODMappers 
 are not currently supported. 

 During selection, visible datasets that can not be selected from are
 temporarily hidden so as not to produce invalid indices from their colors.


To create an instance of class vtkHardwareSelector, simply
invoke its constructor as follows
\begin{verbatim}
  obj = vtkHardwareSelector
\end{verbatim}
\subsection{Methods}

The class vtkHardwareSelector has several methods that can be used.
  They are listed below.
Note that the documentation is translated automatically from the VTK sources,
and may not be completely intelligible.  When in doubt, consult the VTK website.
In the methods listed below, \verb|obj| is an instance of the vtkHardwareSelector class.
\begin{itemize}
\item  \verb|string = obj.GetClassName ()|

\item  \verb|int = obj.IsA (string name)|

\item  \verb|vtkHardwareSelector = obj.NewInstance ()|

\item  \verb|vtkHardwareSelector = obj.SafeDownCast (vtkObject o)|

\item  \verb|obj.SetRenderer (vtkRenderer )| -  Get/Set the renderer to perform the selection on.

\item  \verb|vtkRenderer = obj.GetRenderer ()| -  Get/Set the renderer to perform the selection on.

\item  \verb|obj.SetArea (int , int , int , int )| -  Get/Set the area to select as (xmin, ymin, xmax, ymax).

\item  \verb|obj.SetArea (int  a[4])| -  Get/Set the area to select as (xmin, ymin, xmax, ymax).

\item  \verb|int = obj. GetArea ()| -  Get/Set the area to select as (xmin, ymin, xmax, ymax).

\item  \verb|obj.SetFieldAssociation (int )| -  Set the field type to select. Valid values are 
  vtkDataObject::FIELD\_ASSOCIATION\_POINTS
  vtkDataObject::FIELD\_ASSOCIATION\_CELLS
  vtkDataObject::FIELD\_ASSOCIATION\_VERTICES
  vtkDataObject::FIELD\_ASSOCIATION\_EDGES
  vtkDataObject::FIELD\_ASSOCIATION\_ROWS
 Currently only FIELD\_ASSOCIATION\_POINTS and FIELD\_ASSOCIATION\_CELLS are
 supported.

\item  \verb|int = obj.GetFieldAssociation ()| -  Set the field type to select. Valid values are 
  vtkDataObject::FIELD\_ASSOCIATION\_POINTS
  vtkDataObject::FIELD\_ASSOCIATION\_CELLS
  vtkDataObject::FIELD\_ASSOCIATION\_VERTICES
  vtkDataObject::FIELD\_ASSOCIATION\_EDGES
  vtkDataObject::FIELD\_ASSOCIATION\_ROWS
 Currently only FIELD\_ASSOCIATION\_POINTS and FIELD\_ASSOCIATION\_CELLS are
 supported.

\item  \verb|vtkSelection = obj.Select ()| -  Perform the selection. Returns  a new instance of vtkSelection containing
 the selection on success.

\item  \verb|bool = obj.CaptureBuffers ()| -  It is possible to use the vtkHardwareSelector for a custom picking. (Look
 at vtkScenePicker). In that case instead of Select() on can use
 CaptureBuffers() to render the selection buffers and then get information
 about pixel locations suing GetPixelInformation(). Use ClearBuffers() to
 clear buffers after one's done with the scene.
 The optional final parameter maxDist will look for a cell within the specified
 number of pixels from display\_position.

\item  \verb|obj.ClearBuffers ()| -  Called by any vtkMapper or vtkProp subclass to render an attribute's id.

\item  \verb|obj.RenderAttributeId (vtkIdType attribid)| -  Called by any vtkMapper or vtkProp subclass to render an attribute's id.

\item  \verb|obj.BeginRenderProp ()| -  Called by the mapper (vtkHardwareSelectionPolyDataPainter) before and after
 rendering each prop.

\item  \verb|obj.EndRenderProp ()| -  Called by the mapper (vtkHardwareSelectionPolyDataPainter) before and after
 rendering each prop.

\item  \verb|obj.SetProcessID (int )| -  Get/Set the process id. If process id < 0 (default -1), then the
 PROCESS\_PASS is not rendered.

\item  \verb|int = obj.GetProcessID ()| -  Get/Set the process id. If process id < 0 (default -1), then the
 PROCESS\_PASS is not rendered.

\item  \verb|int = obj.GetCurrentPass ()| -  Get the current pass number.

\item  \verb|vtkSelection = obj.GenerateSelection ()| -  Generates the vtkSelection from pixel buffers.
 Requires that CaptureBuffers() has already been called.
 Optionally you may pass a screen region (xmin, ymin, xmax, ymax)
 to generate a selection from. The region must be a subregion
 of the region specified by SetArea(), otherwise it will be
 clipped to that region.

\item  \verb|vtkSelection = obj.GenerateSelection (int r[4])| -  Generates the vtkSelection from pixel buffers.
 Requires that CaptureBuffers() has already been called.
 Optionally you may pass a screen region (xmin, ymin, xmax, ymax)
 to generate a selection from. The region must be a subregion
 of the region specified by SetArea(), otherwise it will be
 clipped to that region.

\item  \verb|vtkSelection = obj.GenerateSelection (int x1, int y1, int x2, int y2)| -  Generates the vtkSelection from pixel buffers.
 Requires that CaptureBuffers() has already been called.
 Optionally you may pass a screen region (xmin, ymin, xmax, ymax)
 to generate a selection from. The region must be a subregion
 of the region specified by SetArea(), otherwise it will be
 clipped to that region.

\end{itemize}
