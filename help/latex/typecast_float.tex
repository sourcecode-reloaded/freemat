\section{FLOAT Convert to 32-bit Floating Point}

\subsection{Usage}

Converts the argument to a 32-bit floating point number.  The syntax
for its use is
\begin{verbatim}
   y = float(x)
\end{verbatim}
where \verb|x| is an \verb|n|-dimensional numerical array.  
Conversion follows the saturation rules.  Note that both 
\verb|NaN| and \verb|Inf| are both preserved under type conversion.
\subsection{Example}

The following piece of code demonstrates several uses of \verb|float|. 
First, we convert from an integer (the argument is an integer 
because no decimal is present):
\begin{verbatim}
--> float(200)

ans = 
 200 
\end{verbatim}
In the next example, a double precision argument is passed 
in
\begin{verbatim}
--> float(400.0)

ans = 
 400 
\end{verbatim}
In the next example, a complex argument is passed in.
\begin{verbatim}
--> float(3.0+4.0*i)

ans = 
   3.0000 +  4.0000i 
\end{verbatim}
In the next example, a string argument is passed in.  The string 
argument is converted into an integer array corresponding to the 
ASCII values of each character.
\begin{verbatim}
--> float('helo')

ans = 
 104 101 108 111 
\end{verbatim}
In the last example, a cell-array is passed in.  For cell-arrays 
and structure arrays, the result is an error.
\begin{verbatim}
--> float({4})
Error: Cannot perform type conversions with this type
\end{verbatim}
