\section{CLOCKTOTIME Convert Clock Vector to Epoch Time}

\subsection{Usage}

Given the output of the \verb|clock| command, this function computes
the epoch time, i.e, the time in seconds since January 1,1970 
at 00:00:00 UTC.  This function is most useful for calculating elapsed
times using the clock, and should be accurate to less than a millisecond
(although the true accuracy depends on accuracy of the argument vector). 
The usage for \verb|clocktotime| is
\begin{verbatim}
   y = clocktotime(x)
\end{verbatim}
where \verb|x| must be in the form of the output of \verb|clock|, that is
\begin{verbatim}
   x = [year month day hour minute seconds]
\end{verbatim}
\subsection{Example}

Here is an example of using \verb|clocktotime| to time a delay of 1 second
\begin{verbatim}
--> x = clock

x = 

   1.0e+03 * 

    2.0090    0.0010    0.0250    0.0010    0.0050    0.0189 

--> sleep(1)
--> y = clock

y = 

   1.0e+03 * 

    2.0090    0.0010    0.0250    0.0010    0.0050    0.0199 

--> clocktotime(y) - clocktotime(x)

ans = 

    1.0020 
\end{verbatim}
