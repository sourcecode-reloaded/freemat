\section{Function Handles}

\subsection{Usage}

Starting with version 1.11, FreeMat now supports \verb|function handles|,
or \verb|function pointers|.  A \verb|function handle| is an alias for a function
or script that is stored in a variable.  First, the way to assign
a function handle is to use the notation
\begin{verbatim}
    handle = @func
\end{verbatim}
where \verb|func| is the name to point to.  The function \verb|func| must exist
at the time we make the call.  It can be a local function (i.e., a
subfunction).  To use the \verb|handle|, we can either pass it to \verb|feval|
via 
\begin{verbatim}
   [x,y] = feval(handle,arg1,arg2).
\end{verbatim}
Alternately, you can the function directly using the notation
\begin{verbatim}
   [x,y] = handle(arg1,arg2)
\end{verbatim}
