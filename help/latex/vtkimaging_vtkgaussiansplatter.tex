\section{vtkGaussianSplatter}

\subsection{Usage}

 vtkGaussianSplatter is a filter that injects input points into a
 structured points (volume) dataset. As each point is injected, it ''splats''
 or distributes values to nearby voxels. Data is distributed using an
 elliptical, Gaussian distribution function. The distribution function is
 modified using scalar values (expands distribution) or normals
 (creates ellipsoidal distribution rather than spherical).

 In general, the Gaussian distribution function f(x) around a given
 splat point p is given by

     f(x) = ScaleFactor * exp( ExponentFactor*((r/Radius)**2) )

 where x is the current voxel sample point; r is the distance |x-p|
 ExponentFactor <= 0.0, and ScaleFactor can be multiplied by the scalar 
 value of the point p that is currently being splatted.

 If points normals are present (and NormalWarping is on), then the splat 
 function becomes elliptical (as compared to the spherical one described
 by the previous equation). The Gaussian distribution function then
 becomes:
 
     f(x) = ScaleFactor * 
               exp( ExponentFactor*( ((rxy/E)**2 + z**2)/R**2) )

 where E is a user-defined eccentricity factor that controls the elliptical
 shape of the splat; z is the distance of the current voxel sample point
 along normal N; and rxy is the distance of x in the direction
 prependicular to N.

 This class is typically used to convert point-valued distributions into
 a volume representation. The volume is then usually iso-surfaced or
 volume rendered to generate a visualization. It can be used to create
 surfaces from point distributions, or to create structure (i.e.,
 topology) when none exists.

To create an instance of class vtkGaussianSplatter, simply
invoke its constructor as follows
\begin{verbatim}
  obj = vtkGaussianSplatter
\end{verbatim}
\subsection{Methods}

The class vtkGaussianSplatter has several methods that can be used.
  They are listed below.
Note that the documentation is translated automatically from the VTK sources,
and may not be completely intelligible.  When in doubt, consult the VTK website.
In the methods listed below, \verb|obj| is an instance of the vtkGaussianSplatter class.
\begin{itemize}
\item  \verb|string = obj.GetClassName ()|

\item  \verb|int = obj.IsA (string name)|

\item  \verb|vtkGaussianSplatter = obj.NewInstance ()|

\item  \verb|vtkGaussianSplatter = obj.SafeDownCast (vtkObject o)|

\item  \verb|obj.SetSampleDimensions (int i, int j, int k)| -  Set / get the dimensions of the sampling structured point set. Higher
 values produce better results but are much slower.

\item  \verb|obj.SetSampleDimensions (int dim[3])| -  Set / get the dimensions of the sampling structured point set. Higher
 values produce better results but are much slower.

\item  \verb|int = obj. GetSampleDimensions ()| -  Set / get the dimensions of the sampling structured point set. Higher
 values produce better results but are much slower.

\item  \verb|obj.SetModelBounds (double , double , double , double , double , double )| -  Set / get the (xmin,xmax, ymin,ymax, zmin,zmax) bounding box in which
 the sampling is performed. If any of the (min,max) bounds values are
 min >= max, then the bounds will be computed automatically from the input
 data. Otherwise, the user-specified bounds will be used.

\item  \verb|obj.SetModelBounds (double  a[6])| -  Set / get the (xmin,xmax, ymin,ymax, zmin,zmax) bounding box in which
 the sampling is performed. If any of the (min,max) bounds values are
 min >= max, then the bounds will be computed automatically from the input
 data. Otherwise, the user-specified bounds will be used.

\item  \verb|double = obj. GetModelBounds ()| -  Set / get the (xmin,xmax, ymin,ymax, zmin,zmax) bounding box in which
 the sampling is performed. If any of the (min,max) bounds values are
 min >= max, then the bounds will be computed automatically from the input
 data. Otherwise, the user-specified bounds will be used.

\item  \verb|obj.SetRadius (double )| -  Set / get the radius of propagation of the splat. This value is expressed
 as a percentage of the length of the longest side of the sampling
 volume. Smaller numbers greatly reduce execution time.

\item  \verb|double = obj.GetRadiusMinValue ()| -  Set / get the radius of propagation of the splat. This value is expressed
 as a percentage of the length of the longest side of the sampling
 volume. Smaller numbers greatly reduce execution time.

\item  \verb|double = obj.GetRadiusMaxValue ()| -  Set / get the radius of propagation of the splat. This value is expressed
 as a percentage of the length of the longest side of the sampling
 volume. Smaller numbers greatly reduce execution time.

\item  \verb|double = obj.GetRadius ()| -  Set / get the radius of propagation of the splat. This value is expressed
 as a percentage of the length of the longest side of the sampling
 volume. Smaller numbers greatly reduce execution time.

\item  \verb|obj.SetScaleFactor (double )| -  Multiply Gaussian splat distribution by this value. If ScalarWarping
 is on, then the Scalar value will be multiplied by the ScaleFactor
 times the Gaussian function.

\item  \verb|double = obj.GetScaleFactorMinValue ()| -  Multiply Gaussian splat distribution by this value. If ScalarWarping
 is on, then the Scalar value will be multiplied by the ScaleFactor
 times the Gaussian function.

\item  \verb|double = obj.GetScaleFactorMaxValue ()| -  Multiply Gaussian splat distribution by this value. If ScalarWarping
 is on, then the Scalar value will be multiplied by the ScaleFactor
 times the Gaussian function.

\item  \verb|double = obj.GetScaleFactor ()| -  Multiply Gaussian splat distribution by this value. If ScalarWarping
 is on, then the Scalar value will be multiplied by the ScaleFactor
 times the Gaussian function.

\item  \verb|obj.SetExponentFactor (double )| -  Set / get the sharpness of decay of the splats. This is the
 exponent constant in the Gaussian equation. Normally this is
 a negative value.

\item  \verb|double = obj.GetExponentFactor ()| -  Set / get the sharpness of decay of the splats. This is the
 exponent constant in the Gaussian equation. Normally this is
 a negative value.

\item  \verb|obj.SetNormalWarping (int )| -  Turn on/off the generation of elliptical splats. If normal warping is
 on, then the input normals affect the distribution of the splat. This
 boolean is used in combination with the Eccentricity ivar.

\item  \verb|int = obj.GetNormalWarping ()| -  Turn on/off the generation of elliptical splats. If normal warping is
 on, then the input normals affect the distribution of the splat. This
 boolean is used in combination with the Eccentricity ivar.

\item  \verb|obj.NormalWarpingOn ()| -  Turn on/off the generation of elliptical splats. If normal warping is
 on, then the input normals affect the distribution of the splat. This
 boolean is used in combination with the Eccentricity ivar.

\item  \verb|obj.NormalWarpingOff ()| -  Turn on/off the generation of elliptical splats. If normal warping is
 on, then the input normals affect the distribution of the splat. This
 boolean is used in combination with the Eccentricity ivar.

\item  \verb|obj.SetEccentricity (double )| -  Control the shape of elliptical splatting. Eccentricity is the ratio
 of the major axis (aligned along normal) to the minor (axes) aligned
 along other two axes. So Eccentricity > 1 creates needles with the
 long axis in the direction of the normal; Eccentricity<1 creates
 pancakes perpendicular to the normal vector.

\item  \verb|double = obj.GetEccentricityMinValue ()| -  Control the shape of elliptical splatting. Eccentricity is the ratio
 of the major axis (aligned along normal) to the minor (axes) aligned
 along other two axes. So Eccentricity > 1 creates needles with the
 long axis in the direction of the normal; Eccentricity<1 creates
 pancakes perpendicular to the normal vector.

\item  \verb|double = obj.GetEccentricityMaxValue ()| -  Control the shape of elliptical splatting. Eccentricity is the ratio
 of the major axis (aligned along normal) to the minor (axes) aligned
 along other two axes. So Eccentricity > 1 creates needles with the
 long axis in the direction of the normal; Eccentricity<1 creates
 pancakes perpendicular to the normal vector.

\item  \verb|double = obj.GetEccentricity ()| -  Control the shape of elliptical splatting. Eccentricity is the ratio
 of the major axis (aligned along normal) to the minor (axes) aligned
 along other two axes. So Eccentricity > 1 creates needles with the
 long axis in the direction of the normal; Eccentricity<1 creates
 pancakes perpendicular to the normal vector.

\item  \verb|obj.SetScalarWarping (int )| -  Turn on/off the scaling of splats by scalar value.

\item  \verb|int = obj.GetScalarWarping ()| -  Turn on/off the scaling of splats by scalar value.

\item  \verb|obj.ScalarWarpingOn ()| -  Turn on/off the scaling of splats by scalar value.

\item  \verb|obj.ScalarWarpingOff ()| -  Turn on/off the scaling of splats by scalar value.

\item  \verb|obj.SetCapping (int )| -  Turn on/off the capping of the outer boundary of the volume
 to a specified cap value. This can be used to close surfaces
 (after iso-surfacing) and create other effects.

\item  \verb|int = obj.GetCapping ()| -  Turn on/off the capping of the outer boundary of the volume
 to a specified cap value. This can be used to close surfaces
 (after iso-surfacing) and create other effects.

\item  \verb|obj.CappingOn ()| -  Turn on/off the capping of the outer boundary of the volume
 to a specified cap value. This can be used to close surfaces
 (after iso-surfacing) and create other effects.

\item  \verb|obj.CappingOff ()| -  Turn on/off the capping of the outer boundary of the volume
 to a specified cap value. This can be used to close surfaces
 (after iso-surfacing) and create other effects.

\item  \verb|obj.SetCapValue (double )| -  Specify the cap value to use. (This instance variable only has effect 
 if the ivar Capping is on.)

\item  \verb|double = obj.GetCapValue ()| -  Specify the cap value to use. (This instance variable only has effect 
 if the ivar Capping is on.)

\item  \verb|obj.SetAccumulationMode (int )| -  Specify the scalar accumulation mode. This mode expresses how scalar
 values are combined when splats are overlapped. The Max mode acts
 like a set union operation and is the most commonly used; the Min
 mode acts like a set intersection, and the sum is just weird.

\item  \verb|int = obj.GetAccumulationModeMinValue ()| -  Specify the scalar accumulation mode. This mode expresses how scalar
 values are combined when splats are overlapped. The Max mode acts
 like a set union operation and is the most commonly used; the Min
 mode acts like a set intersection, and the sum is just weird.

\item  \verb|int = obj.GetAccumulationModeMaxValue ()| -  Specify the scalar accumulation mode. This mode expresses how scalar
 values are combined when splats are overlapped. The Max mode acts
 like a set union operation and is the most commonly used; the Min
 mode acts like a set intersection, and the sum is just weird.

\item  \verb|int = obj.GetAccumulationMode ()| -  Specify the scalar accumulation mode. This mode expresses how scalar
 values are combined when splats are overlapped. The Max mode acts
 like a set union operation and is the most commonly used; the Min
 mode acts like a set intersection, and the sum is just weird.

\item  \verb|obj.SetAccumulationModeToMin ()| -  Specify the scalar accumulation mode. This mode expresses how scalar
 values are combined when splats are overlapped. The Max mode acts
 like a set union operation and is the most commonly used; the Min
 mode acts like a set intersection, and the sum is just weird.

\item  \verb|obj.SetAccumulationModeToMax ()| -  Specify the scalar accumulation mode. This mode expresses how scalar
 values are combined when splats are overlapped. The Max mode acts
 like a set union operation and is the most commonly used; the Min
 mode acts like a set intersection, and the sum is just weird.

\item  \verb|obj.SetAccumulationModeToSum ()| -  Specify the scalar accumulation mode. This mode expresses how scalar
 values are combined when splats are overlapped. The Max mode acts
 like a set union operation and is the most commonly used; the Min
 mode acts like a set intersection, and the sum is just weird.

\item  \verb|string = obj.GetAccumulationModeAsString ()| -  Specify the scalar accumulation mode. This mode expresses how scalar
 values are combined when splats are overlapped. The Max mode acts
 like a set union operation and is the most commonly used; the Min
 mode acts like a set intersection, and the sum is just weird.

\item  \verb|obj.SetNullValue (double )| -  Set the Null value for output points not receiving a contribution from the
 input points. (This is the initial value of the voxel samples.)

\item  \verb|double = obj.GetNullValue ()| -  Set the Null value for output points not receiving a contribution from the
 input points. (This is the initial value of the voxel samples.)

\item  \verb|obj.ComputeModelBounds (vtkDataSet input, vtkImageData output, vtkInformation outInfo)| -  Compute the size of the sample bounding box automatically from the
 input data. This is an internal helper function.

\end{itemize}
