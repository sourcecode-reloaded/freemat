\section{vtkAbstractArray}

\subsection{Usage}


 vtkAbstractArray is an abstract superclass for data array objects.
 This class defines an API that all subclasses must support.  The
 data type must be assignable and copy-constructible, but no other
 assumptions about its type are made.  Most of the subclasses of
 this array deal with numeric data either as scalars or tuples of
 scalars.  A program can use the IsNumeric() method to check whether
 an instance of vtkAbstractArray contains numbers.  It is also
 possible to test for this by attempting to SafeDownCast an array to
 an instance of vtkDataArray, although this assumes that all numeric
 arrays will always be descended from vtkDataArray.

 <p>

 Every array has a character-string name. The naming of the array
 occurs automatically when it is instantiated, but you are free to
 change this name using the SetName() method.  (The array name is
 used for data manipulation.)


To create an instance of class vtkAbstractArray, simply
invoke its constructor as follows
\begin{verbatim}
  obj = vtkAbstractArray
\end{verbatim}
\subsection{Methods}

The class vtkAbstractArray has several methods that can be used.
  They are listed below.
Note that the documentation is translated automatically from the VTK sources,
and may not be completely intelligible.  When in doubt, consult the VTK website.
In the methods listed below, \verb|obj| is an instance of the vtkAbstractArray class.
\begin{itemize}
\item  \verb|string = obj.GetClassName ()|

\item  \verb|int = obj.IsA (string name)|

\item  \verb|vtkAbstractArray = obj.NewInstance ()|

\item  \verb|vtkAbstractArray = obj.SafeDownCast (vtkObject o)|

\item  \verb|int = obj.Allocate (vtkIdType sz, vtkIdType ext)| -  Allocate memory for this array. Delete old storage only if necessary.
 Note that ext is no longer used.

\item  \verb|obj.Initialize ()| -  Release storage and reset array to initial state.

\item  \verb|int = obj.GetDataType ()| -  Return the underlying data type. An integer indicating data type is 
 returned as specified in vtkSetGet.h.

\item  \verb|int = obj.GetDataTypeSize ()| -  Return the size of the underlying data type.  For a bit, 0 is
 returned.  For string 0 is returned. Arrays with variable length
 components return 0.

\item  \verb|int = obj.GetElementComponentSize ()| -  Return the size, in bytes, of the lowest-level element of an
 array.  For vtkDataArray and subclasses this is the size of the
 data type.  For vtkStringArray, this is
 sizeof(vtkStdString::value\_type), which winds up being
 sizeof(char).  

\item  \verb|obj.SetNumberOfComponents (int )| -  Set/Get the dimention (n) of the components. Must be >= 1. Make sure that
 this is set before allocation.

\item  \verb|int = obj.GetNumberOfComponentsMinValue ()| -  Set/Get the dimention (n) of the components. Must be >= 1. Make sure that
 this is set before allocation.

\item  \verb|int = obj.GetNumberOfComponentsMaxValue ()| -  Set/Get the dimention (n) of the components. Must be >= 1. Make sure that
 this is set before allocation.

\item  \verb|int = obj.GetNumberOfComponents ()| -  Set the number of tuples (a component group) in the array. Note that 
 this may allocate space depending on the number of components.
 Also note that if allocation is performed no copy is performed so
 existing data will be lost (if data conservation is sought, one may
 use the Resize method instead).

\item  \verb|obj.SetNumberOfTuples (vtkIdType number)| -  Set the number of tuples (a component group) in the array. Note that 
 this may allocate space depending on the number of components.
 Also note that if allocation is performed no copy is performed so
 existing data will be lost (if data conservation is sought, one may
 use the Resize method instead).

\item  \verb|vtkIdType = obj.GetNumberOfTuples ()| -  Set the tuple at the ith location using the jth tuple in the source array.
 This method assumes that the two arrays have the same type
 and structure. Note that range checking and memory allocation is not 
 performed; use in conjunction with SetNumberOfTuples() to allocate space.

\item  \verb|obj.SetTuple (vtkIdType i, vtkIdType j, vtkAbstractArray source)| -  Set the tuple at the ith location using the jth tuple in the source array.
 This method assumes that the two arrays have the same type
 and structure. Note that range checking and memory allocation is not 
 performed; use in conjunction with SetNumberOfTuples() to allocate space.

\item  \verb|obj.InsertTuple (vtkIdType i, vtkIdType j, vtkAbstractArray source)| -  Insert the jth tuple in the source array, at ith location in this array. 
 Note that memory allocation is performed as necessary to hold the data.

\item  \verb|vtkIdType = obj.InsertNextTuple (vtkIdType j, vtkAbstractArray source)| -  Insert the jth tuple in the source array, at the end in this array. 
 Note that memory allocation is performed as necessary to hold the data.
 Returns the location at which the data was inserted.

\item  \verb|obj.GetTuples (vtkIdList ptIds, vtkAbstractArray output)| -  Given a list of point ids, return an array of tuples.
 You must insure that the output array has been previously 
 allocated with enough space to hold the data.

\item  \verb|obj.GetTuples (vtkIdType p1, vtkIdType p2, vtkAbstractArray output)| -  Get the tuples for the range of points ids specified 
 (i.e., p1->p2 inclusive). You must insure that the output array has 
 been previously allocated with enough space to hold the data.

\item  \verb|obj.DeepCopy (vtkAbstractArray da)| -  Deep copy of data. Implementation left to subclasses, which
 should support as many type conversions as possible given the
 data type.

 Subclasses should call vtkAbstractArray::DeepCopy() so that the
 information object (if one exists) is copied from  da.

\item  \verb|obj.InterpolateTuple (vtkIdType i, vtkIdList ptIndices, vtkAbstractArray source, double weights)| -  Set the ith tuple in this array as the interpolated tuple value,
 given the ptIndices in the source array and associated 
 interpolation weights.
 This method assumes that the two arrays are of the same type
 and strcuture.

\item  \verb|obj.InterpolateTuple (vtkIdType i, vtkIdType id1, vtkAbstractArray source1, vtkIdType id2, vtkAbstractArray source2, double t)|

\item  \verb|obj.Squeeze ()| -  Resize object to just fit data requirement. Reclaims extra memory.

\item  \verb|int = obj.Resize (vtkIdType numTuples)| -  Resize the array while conserving the data.  Returns 1 if
 resizing succeeded and 0 otherwise.

\item  \verb|obj.Reset ()| -  Return the size of the data.

\item  \verb|vtkIdType = obj.GetSize ()| -  What is the maximum id currently in the array.

\item  \verb|vtkIdType = obj.GetMaxId ()| -  This method lets the user specify data to be held by the array.  The 
 array argument is a pointer to the data.  size is the size of 
 the array supplied by the user.  Set save to 1 to keep the class
 from deleting the array when it cleans up or reallocates memory.
 The class uses the actual array provided; it does not copy the data 
 from the supplied array.

\item  \verb|long = obj.GetActualMemorySize ()| -  Return the memory in kilobytes consumed by this data array. Used to
 support streaming and reading/writing data. The value returned is
 guaranteed to be greater than or equal to the memory required to
 actually represent the data represented by this object. The 
 information returned is valid only after the pipeline has 
 been updated.

\item  \verb|obj.SetName (string )| -  Set/get array's name

\item  \verb|string = obj.GetName ()| -  Set/get array's name

\item  \verb|string = obj.GetDataTypeAsString (void )| -  Creates an array for dataType where dataType is one of
 VTK\_BIT, VTK\_CHAR, VTK\_UNSIGNED\_CHAR, VTK\_SHORT,
 VTK\_UNSIGNED\_SHORT, VTK\_INT, VTK\_UNSIGNED\_INT, VTK\_LONG,
 VTK\_UNSIGNED\_LONG, VTK\_DOUBLE, VTK\_DOUBLE, VTK\_ID\_TYPE,
 VTK\_STRING.
 Note that the data array returned has be deleted by the
 user.

\item  \verb|int = obj.IsNumeric ()| -  This method is here to make backward compatibility easier.  It
 must return true if and only if an array contains numeric data.

\item  \verb|vtkArrayIterator = obj.NewIterator ()| -  Subclasses must override this method and provide the right 
 kind of templated vtkArrayIteratorTemplate.

\item  \verb|vtkIdType = obj.GetDataSize ()| -  Tell the array explicitly that the data has changed.
 This is only necessary to call when you modify the array contents
 without using the array's API (i.e. you retrieve a pointer to the
 data and modify the array contents).  You need to call this so that
 the fast lookup will know to rebuild itself.  Otherwise, the lookup
 functions will give incorrect results.

\item  \verb|obj.DataChanged ()| -  Tell the array explicitly that the data has changed.
 This is only necessary to call when you modify the array contents
 without using the array's API (i.e. you retrieve a pointer to the
 data and modify the array contents).  You need to call this so that
 the fast lookup will know to rebuild itself.  Otherwise, the lookup
 functions will give incorrect results.

\item  \verb|obj.ClearLookup ()| -  Delete the associated fast lookup data structure on this array,
 if it exists.  The lookup will be rebuilt on the next call to a lookup
 function.

\item  \verb|vtkInformation = obj.GetInformation ()| -  Get an information object that can be used to annotate the array.
 This will always return an instance of vtkInformation, if one is
 not currently associated with the array it will be created.

\item  \verb|bool = obj.HasInformation ()|

\end{itemize}
