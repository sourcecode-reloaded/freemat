\section{DEC2HEX Convert Decimal Number to Hexadecimal}

\subsection{Usage}

Converts an integer value into its hexadecimal representation.  The syntax
for its use is
\begin{verbatim}
   y = dec2hex(x)
\end{verbatim}
where \verb|x| is an integer (and is promoted to a 64-bit integer if it is not).
The returned value \verb|y| is a string containing the hexadecimal representation
of that integer.  If you require a minimum length for the hexadecimal
representation, you can specify an optional second argument
\begin{verbatim}
   y = dec2hex(x,n)
\end{verbatim}
where \verb|n| indicates the minimum number of digits in the representation.
\subsection{Example}

Here are some simple examples:
\begin{verbatim}
--> dec2hex(1023)

ans = 
3FF
\end{verbatim}
\begin{verbatim}
--> dec2hex(58128493)

ans = 
376F86D
\end{verbatim}
