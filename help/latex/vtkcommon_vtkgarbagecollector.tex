\section{vtkGarbageCollector}

\subsection{Usage}

 vtkGarbageCollector is used by VTK classes that may be involved in
 reference counting loops (such as Algorithm <-> Executive).  It
 detects strongly connected components of the reference graph that
 have been leaked deletes them.  The garbage collector uses the
 ReportReferences method to search the reference graph and construct
 a net reference count for each connected component.  If the net
 reference count is zero the entire set of objects is deleted.
 Deleting each component may leak other components, which are then
 collected recursively.

 To enable garbage collection for a class, add these members:

 \begin{verbatim}

  public:
   virtual void Register(vtkObjectBase* o)
     {
     this->RegisterInternal(o, 1);
     }
   virtual void UnRegister(vtkObjectBase* o)
     {
     this->UnRegisterInternal(o, 1);
     }

  protected:

   virtual void ReportReferences(vtkGarbageCollector* collector)
     {
     // Report references held by this object that may be in a loop.
     this->Superclass::ReportReferences(collector);
     vtkGarbageCollectorReport(collector, this->OtherObject, ''Other Object'');
     }
 \end{verbatim}

 The implementations should be in the .cxx file in practice.
 It is important that the reference be reported using the real
 pointer or smart pointer instance that holds the reference.  When
 collecting the garbage collector will actually set this pointer to
 NULL.  The destructor of the class should be written to deal with
 this.  It is also expected that an invariant is maintained for any
 reference that is reported.  The variable holding the reference
 must always either be NULL or refer to a fully constructed valid
 object.  Therefore code like ''this->Object->UnRegister(this)'' must
 be avoided if ''this->Object'' is a reported reference because it
 is possible that the object is deleted before UnRegister returns
 but then ''this->Object'' will be left as a dangling pointer.  Instead
 use code like

 \begin{verbatim}
   vtkObjectBase* obj = this->Object;
   this->Object = 0;
   obj->UnRegister(this);
 \end{verbatim}

 so that the reported reference maintains the invariant.

 If subclassing from a class that already supports garbage
 collection, one need only provide the ReportReferences method.

To create an instance of class vtkGarbageCollector, simply
invoke its constructor as follows
\begin{verbatim}
  obj = vtkGarbageCollector
\end{verbatim}
\subsection{Methods}

The class vtkGarbageCollector has several methods that can be used.
  They are listed below.
Note that the documentation is translated automatically from the VTK sources,
and may not be completely intelligible.  When in doubt, consult the VTK website.
In the methods listed below, \verb|obj| is an instance of the vtkGarbageCollector class.
\begin{itemize}
\item  \verb|string = obj.GetClassName ()|

\item  \verb|int = obj.IsA (string name)|

\item  \verb|vtkGarbageCollector = obj.NewInstance ()|

\item  \verb|vtkGarbageCollector = obj.SafeDownCast (vtkObject o)|

\end{itemize}
