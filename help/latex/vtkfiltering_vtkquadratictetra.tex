\section{vtkQuadraticTetra}

\subsection{Usage}

 vtkQuadraticTetra is a concrete implementation of vtkNonLinearCell to
 represent a three-dimensional, 10-node, isoparametric parabolic
 tetrahedron. The interpolation is the standard finite element, quadratic
 isoparametric shape function. The cell includes a mid-edge node on each of
 the size edges of the tetrahedron. The ordering of the ten points defining
 the cell is point ids (0-3,4-9) where ids 0-3 are the four tetra
 vertices; and point ids 4-9 are the midedge nodes between (0,1), (1,2),
 (2,0), (0,3), (1,3), and (2,3).


To create an instance of class vtkQuadraticTetra, simply
invoke its constructor as follows
\begin{verbatim}
  obj = vtkQuadraticTetra
\end{verbatim}
\subsection{Methods}

The class vtkQuadraticTetra has several methods that can be used.
  They are listed below.
Note that the documentation is translated automatically from the VTK sources,
and may not be completely intelligible.  When in doubt, consult the VTK website.
In the methods listed below, \verb|obj| is an instance of the vtkQuadraticTetra class.
\begin{itemize}
\item  \verb|string = obj.GetClassName ()|

\item  \verb|int = obj.IsA (string name)|

\item  \verb|vtkQuadraticTetra = obj.NewInstance ()|

\item  \verb|vtkQuadraticTetra = obj.SafeDownCast (vtkObject o)|

\item  \verb|int = obj.GetCellType ()| -  Implement the vtkCell API. See the vtkCell API for descriptions
 of these methods.

\item  \verb|int = obj.GetCellDimension ()| -  Implement the vtkCell API. See the vtkCell API for descriptions
 of these methods.

\item  \verb|int = obj.GetNumberOfEdges ()| -  Implement the vtkCell API. See the vtkCell API for descriptions
 of these methods.

\item  \verb|int = obj.GetNumberOfFaces ()| -  Implement the vtkCell API. See the vtkCell API for descriptions
 of these methods.

\item  \verb|vtkCell = obj.GetEdge (int )| -  Implement the vtkCell API. See the vtkCell API for descriptions
 of these methods.

\item  \verb|vtkCell = obj.GetFace (int )| -  Implement the vtkCell API. See the vtkCell API for descriptions
 of these methods.

\item  \verb|int = obj.CellBoundary (int subId, double pcoords[3], vtkIdList pts)|

\item  \verb|obj.Contour (double value, vtkDataArray cellScalars, vtkIncrementalPointLocator locator, vtkCellArray verts, vtkCellArray lines, vtkCellArray polys, vtkPointData inPd, vtkPointData outPd, vtkCellData inCd, vtkIdType cellId, vtkCellData outCd)|

\item  \verb|int = obj.Triangulate (int index, vtkIdList ptIds, vtkPoints pts)|

\item  \verb|obj.Derivatives (int subId, double pcoords[3], double values, int dim, double derivs)|

\item  \verb|obj.Clip (double value, vtkDataArray cellScalars, vtkIncrementalPointLocator locator, vtkCellArray tetras, vtkPointData inPd, vtkPointData outPd, vtkCellData inCd, vtkIdType cellId, vtkCellData outCd, int insideOut)| -  Clip this edge using scalar value provided. Like contouring, except
 that it cuts the tetra to produce new tetras.

\item  \verb|int = obj.GetParametricCenter (double pcoords[3])| -  Return the center of the quadratic tetra in parametric coordinates.

\item  \verb|double = obj.GetParametricDistance (double pcoords[3])| -  Return the distance of the parametric coordinate provided to the
 cell. If inside the cell, a distance of zero is returned.

\item  \verb|obj.InterpolateFunctions (double pcoords[3], double weights[10])| -  Compute the interpolation functions/derivatives
 (aka shape functions/derivatives)

\item  \verb|obj.InterpolateDerivs (double pcoords[3], double derivs[30])| -  Return the ids of the vertices defining edge/face (`edgeId`/`faceId').
 Ids are related to the cell, not to the dataset.

\end{itemize}
