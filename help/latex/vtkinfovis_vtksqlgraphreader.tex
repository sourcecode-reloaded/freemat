\section{vtkSQLGraphReader}

\subsection{Usage}


 Creates a vtkGraph using one or two vtkSQLQuery's.  The first (required)
 query must have one row for each arc in the graph.
 The query must have two columns which represent the source and target
 node ids.

 The second (optional) query has one row for each node in the graph.
 The table must have a field whose values match those in the arc table.
 If the node table is not given, 
 a node will be created for each unique source or target identifier 
 in the arc table.

 The source, target, and node ID fields must be of the same type, 
 and must be either vtkStringArray or a subclass of vtkDataArray.

 All columns in the queries, including the source, target, and node index
 fields, are copied into the arc data and node data of the resulting
 vtkGraph.  If the node query is not given, the node data will contain
 a single ''id'' column with the same type as the source/target id arrays.

 If parallel arcs are collected, not all the arc data is not copied into 
 the output.  Only the source and target id arrays will be transferred.
 An additional vtkIdTypeArray column called ''weight'' is created which 
 contains the number of times each arc appeared in the input.

 If the node query contains positional data, the user may specify the
 names of these fields.
 These arrays must be data arrays.  The z-coordinate array is optional,
 and if not given the z-coordinates are set to zero.

To create an instance of class vtkSQLGraphReader, simply
invoke its constructor as follows
\begin{verbatim}
  obj = vtkSQLGraphReader
\end{verbatim}
\subsection{Methods}

The class vtkSQLGraphReader has several methods that can be used.
  They are listed below.
Note that the documentation is translated automatically from the VTK sources,
and may not be completely intelligible.  When in doubt, consult the VTK website.
In the methods listed below, \verb|obj| is an instance of the vtkSQLGraphReader class.
\begin{itemize}
\item  \verb|string = obj.GetClassName ()|

\item  \verb|int = obj.IsA (string name)|

\item  \verb|vtkSQLGraphReader = obj.NewInstance ()|

\item  \verb|vtkSQLGraphReader = obj.SafeDownCast (vtkObject o)|

\item  \verb|obj.SetDirected (bool )| -  When set, creates a directed graph, as opposed to an undirected graph.

\item  \verb|bool = obj.GetDirected ()| -  When set, creates a directed graph, as opposed to an undirected graph.

\item  \verb|obj.DirectedOn ()| -  When set, creates a directed graph, as opposed to an undirected graph.

\item  \verb|obj.DirectedOff ()| -  When set, creates a directed graph, as opposed to an undirected graph.

\item  \verb|obj.SetVertexQuery (vtkSQLQuery q)| -  The query that retrieves the node information.

\item  \verb|vtkSQLQuery = obj.GetVertexQuery ()| -  The query that retrieves the node information.

\item  \verb|obj.SetEdgeQuery (vtkSQLQuery q)| -  The query that retrieves the arc information.

\item  \verb|vtkSQLQuery = obj.GetEdgeQuery ()| -  The query that retrieves the arc information.

\item  \verb|obj.SetSourceField (string )| -  The name of the field in the arc query for the source node of each arc.

\item  \verb|string = obj.GetSourceField ()| -  The name of the field in the arc query for the source node of each arc.

\item  \verb|obj.SetTargetField (string )| -  The name of the field in the arc query for the target node of each arc.

\item  \verb|string = obj.GetTargetField ()| -  The name of the field in the arc query for the target node of each arc.

\item  \verb|obj.SetVertexIdField (string )| -  The name of the field in the node query for the node ID.

\item  \verb|string = obj.GetVertexIdField ()| -  The name of the field in the node query for the node ID.

\item  \verb|obj.SetXField (string )| -  The name of the field in the node query for the node's x coordinate.

\item  \verb|string = obj.GetXField ()| -  The name of the field in the node query for the node's x coordinate.

\item  \verb|obj.SetYField (string )| -  The name of the field in the node query for the node's y coordinate.

\item  \verb|string = obj.GetYField ()| -  The name of the field in the node query for the node's y coordinate.

\item  \verb|obj.SetZField (string )| -  The name of the field in the node query for the node's z coordinate.

\item  \verb|string = obj.GetZField ()| -  The name of the field in the node query for the node's z coordinate.

\item  \verb|obj.SetCollapseEdges (bool )| -  When set, creates a graph with no parallel arcs.
 Parallel arcs are combined into one arc.
 No cell fields are passed to the output, except the vtkGhostLevels array if
 it exists, but a new field ''weight'' is created that holds the number of 
 duplicates of that arc in the input.

\item  \verb|bool = obj.GetCollapseEdges ()| -  When set, creates a graph with no parallel arcs.
 Parallel arcs are combined into one arc.
 No cell fields are passed to the output, except the vtkGhostLevels array if
 it exists, but a new field ''weight'' is created that holds the number of 
 duplicates of that arc in the input.

\item  \verb|obj.CollapseEdgesOn ()| -  When set, creates a graph with no parallel arcs.
 Parallel arcs are combined into one arc.
 No cell fields are passed to the output, except the vtkGhostLevels array if
 it exists, but a new field ''weight'' is created that holds the number of 
 duplicates of that arc in the input.

\item  \verb|obj.CollapseEdgesOff ()| -  When set, creates a graph with no parallel arcs.
 Parallel arcs are combined into one arc.
 No cell fields are passed to the output, except the vtkGhostLevels array if
 it exists, but a new field ''weight'' is created that holds the number of 
 duplicates of that arc in the input.

\end{itemize}
