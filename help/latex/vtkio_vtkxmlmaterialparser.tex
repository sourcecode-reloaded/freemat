\section{vtkXMLMaterialParser}

\subsection{Usage}

 vtkXMLMaterialParser parses a VTK Material file and provides that file's
 description of a number of vertex and fragment shaders along with data
 values specified for data members of vtkProperty. This material is to be
 applied to an actor through it's vtkProperty and augments VTK's concept
 of a vtkProperty to include explicitly include vertex and fragment shaders
 and parameter settings for those shaders. This effectively makes reflectance
 models and other shaders  a material property. If no shaders are specified
 VTK should default to standard rendering.

 .SECTION Design
 vtkXMLMaterialParser provides access to 3 distinct types of first-level
 vtkXMLDataElements that describe a VTK material. These elements are as
 follows:

 vtkProperty - describe values for vtkProperty data members

 vtkVertexShader - a vertex shader and enough information to
 install it into the hardware rendering pipeline including values for
 specific shader parameters and structures.

 vtkFragmentShader - a fragment shader and enough information to
 install it into the hardware rendering pipeline including values for
 specific shader parameters and structures.

 The design of the material file closely follows that of vtk's xml
 descriptions of it's data sets. This allows use of the very handy
 vtkXMLDataElement which provides easy access to an xml element's 
 attribute values. Inlined data is currently not handled.

 Ideally this class would be a Facade to a DOM parser, but VTK only
 provides access to expat, a SAX parser. Other vtk classes that parse
 xml files are tuned to read vtkDataSets and don't provide the functionality
 to handle generic xml data. As such they are of little use here.

 This class may be extended for better data  handling or may become a
 Facade to a DOM parser should on become part of the VTK code base.
 .SECTION Thanks
 Shader support in VTK includes key contributions by Gary Templet at 
 Sandia National Labs.

To create an instance of class vtkXMLMaterialParser, simply
invoke its constructor as follows
\begin{verbatim}
  obj = vtkXMLMaterialParser
\end{verbatim}
\subsection{Methods}

The class vtkXMLMaterialParser has several methods that can be used.
  They are listed below.
Note that the documentation is translated automatically from the VTK sources,
and may not be completely intelligible.  When in doubt, consult the VTK website.
In the methods listed below, \verb|obj| is an instance of the vtkXMLMaterialParser class.
\begin{itemize}
\item  \verb|string = obj.GetClassName ()|

\item  \verb|int = obj.IsA (string name)|

\item  \verb|vtkXMLMaterialParser = obj.NewInstance ()|

\item  \verb|vtkXMLMaterialParser = obj.SafeDownCast (vtkObject o)|

\item  \verb|vtkXMLMaterial = obj.GetMaterial ()| -  Set/Get the vtkXMLMaterial representation of the parsed material.

\item  \verb|obj.SetMaterial (vtkXMLMaterial )| -  Set/Get the vtkXMLMaterial representation of the parsed material.

\item  \verb|int = obj.Parse ()| -  Overridden to initialize the internal structures before
 the parsing begins.

\item  \verb|int = obj.Parse (string inputString)| -  Overridden to initialize the internal structures before
 the parsing begins.

\item  \verb|int = obj.Parse (string inputString, int length)| -  Overridden to initialize the internal structures before
 the parsing begins.

\item  \verb|int = obj.InitializeParser ()| -  Overridden to clean up internal structures before the chunk-parsing
 begins.

\end{itemize}
