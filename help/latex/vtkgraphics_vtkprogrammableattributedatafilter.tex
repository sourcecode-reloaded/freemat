\section{vtkProgrammableAttributeDataFilter}

\subsection{Usage}

 vtkProgrammableAttributeDataFilter is a filter that allows you to write a
 custom procedure to manipulate attribute data - either point or cell
 data. For example, you could generate scalars based on a complex formula;
 convert vectors to normals; compute scalar values as a function of
 vectors, texture coords, and/or any other point data attribute; and so
 on. The filter takes multiple inputs (input plus an auxiliary input list),
 so you can write procedures that combine several dataset point
 attributes. Note that the output of the filter is the same type
 (topology/geometry) as the input.

 The filter works as follows. It operates like any other filter (i.e.,
 checking and managing modified and execution times, processing Update()
 and Execute() methods, managing release of data, etc.), but the difference
 is that the Execute() method simply invokes a user-specified function with
 an optional (void *) argument (typically the ''this'' pointer in C++). It is
 also possible to specify a function to delete the argument via
 ExecuteMethodArgDelete().

 To use the filter, you write a procedure to process the input datasets,
 process the data, and generate output data. Typically, this means grabbing
 the input point or cell data (using GetInput() and maybe GetInputList()),
 operating on it (creating new point and cell attributes such as scalars,
 vectors, etc.), and then setting the point and/or cell attributes in the
 output dataset (you'll need to use GetOutput() to access the output).
 (Note: besides C++, it is possible to do the same thing in Tcl, Java, or
 other languages that wrap the C++ core.) Remember, proper filter protocol
 requires that you don't modify the input data - you create new output data
 from the input.


To create an instance of class vtkProgrammableAttributeDataFilter, simply
invoke its constructor as follows
\begin{verbatim}
  obj = vtkProgrammableAttributeDataFilter
\end{verbatim}
\subsection{Methods}

The class vtkProgrammableAttributeDataFilter has several methods that can be used.
  They are listed below.
Note that the documentation is translated automatically from the VTK sources,
and may not be completely intelligible.  When in doubt, consult the VTK website.
In the methods listed below, \verb|obj| is an instance of the vtkProgrammableAttributeDataFilter class.
\begin{itemize}
\item  \verb|string = obj.GetClassName ()|

\item  \verb|int = obj.IsA (string name)|

\item  \verb|vtkProgrammableAttributeDataFilter = obj.NewInstance ()|

\item  \verb|vtkProgrammableAttributeDataFilter = obj.SafeDownCast (vtkObject o)|

\item  \verb|obj.AddInput (vtkDataSet in)| -  Add a dataset to the list of data to process.

\item  \verb|obj.RemoveInput (vtkDataSet in)| -  Remove a dataset from the list of data to process.

\item  \verb|vtkDataSetCollection = obj.GetInputList ()| -  Return the list of inputs.

\end{itemize}
