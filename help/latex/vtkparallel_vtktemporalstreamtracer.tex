\section{vtkTemporalStreamTracer}

\subsection{Usage}

 vtkTemporalStreamTracer is a filter that integrates a vector field to generate
 


To create an instance of class vtkTemporalStreamTracer, simply
invoke its constructor as follows
\begin{verbatim}
  obj = vtkTemporalStreamTracer
\end{verbatim}
\subsection{Methods}

The class vtkTemporalStreamTracer has several methods that can be used.
  They are listed below.
Note that the documentation is translated automatically from the VTK sources,
and may not be completely intelligible.  When in doubt, consult the VTK website.
In the methods listed below, \verb|obj| is an instance of the vtkTemporalStreamTracer class.
\begin{itemize}
\item  \verb|string = obj.GetClassName ()|

\item  \verb|int = obj.IsA (string name)|

\item  \verb|vtkTemporalStreamTracer = obj.NewInstance ()|

\item  \verb|vtkTemporalStreamTracer = obj.SafeDownCast (vtkObject o)|

\item  \verb|obj.SetTimeStep (int )| -  Set/Get the TimeStep. This is the primary means of advancing 
 the particles. The TimeStep should be animated and this will drive
 the pipeline forcing timesteps to be fetched from upstream.

\item  \verb|int = obj.GetTimeStep ()| -  Set/Get the TimeStep. This is the primary means of advancing 
 the particles. The TimeStep should be animated and this will drive
 the pipeline forcing timesteps to be fetched from upstream.

\item  \verb|obj.SetIgnorePipelineTime (int )| -  To get around problems with the Paraview Animation controls
 we can just animate the time step and ignore the TIME\_ requests

\item  \verb|int = obj.GetIgnorePipelineTime ()| -  To get around problems with the Paraview Animation controls
 we can just animate the time step and ignore the TIME\_ requests

\item  \verb|obj.IgnorePipelineTimeOn ()| -  To get around problems with the Paraview Animation controls
 we can just animate the time step and ignore the TIME\_ requests

\item  \verb|obj.IgnorePipelineTimeOff ()| -  To get around problems with the Paraview Animation controls
 we can just animate the time step and ignore the TIME\_ requests

\item  \verb|obj.SetTimeStepResolution (double )| -  If the data source does not have the correct time values 
 present on each time step - setting this value to non unity can
 be used to adjust the time step size from 1s pre step to
 1x\_TimeStepResolution : Not functional in this version. 
 Broke it @todo, put back time scaling

\item  \verb|double = obj.GetTimeStepResolution ()| -  If the data source does not have the correct time values 
 present on each time step - setting this value to non unity can
 be used to adjust the time step size from 1s pre step to
 1x\_TimeStepResolution : Not functional in this version. 
 Broke it @todo, put back time scaling

\item  \verb|obj.SetForceReinjectionEveryNSteps (int )| -  When animating particles, it is nice to inject new ones every Nth step
 to produce a continuous flow. Setting ForceReinjectionEveryNSteps to a 
 non zero value will cause the particle source to reinject particles
 every Nth step even if it is otherwise unchanged.
 Note that if the particle source is also animated, this flag will be
 redundant as the particles will be reinjected whenever the source changes 
 anyway

\item  \verb|int = obj.GetForceReinjectionEveryNSteps ()| -  When animating particles, it is nice to inject new ones every Nth step
 to produce a continuous flow. Setting ForceReinjectionEveryNSteps to a 
 non zero value will cause the particle source to reinject particles
 every Nth step even if it is otherwise unchanged.
 Note that if the particle source is also animated, this flag will be
 redundant as the particles will be reinjected whenever the source changes 
 anyway

\item  \verb|obj.SetTerminationTime (double )| -  Setting TerminationTime to a positive value will cause particles
 to terminate when the time is reached. Use a vlue of zero to
 diable termination. The units of time should be consistent with the 
 primary time variable.

\item  \verb|double = obj.GetTerminationTime ()| -  Setting TerminationTime to a positive value will cause particles
 to terminate when the time is reached. Use a vlue of zero to
 diable termination. The units of time should be consistent with the 
 primary time variable.

\item  \verb|obj.SetTerminationTimeUnit (int )| -  The units of TerminationTime may be actual 'Time' units as described
 by the data, or just TimeSteps of iteration.

\item  \verb|int = obj.GetTerminationTimeUnit ()| -  The units of TerminationTime may be actual 'Time' units as described
 by the data, or just TimeSteps of iteration.

\item  \verb|obj.SetTerminationTimeUnitToTimeUnit ()| -  The units of TerminationTime may be actual 'Time' units as described
 by the data, or just TimeSteps of iteration.

\item  \verb|obj.SetTerminationTimeUnitToStepUnit ()| -  The units of TerminationTime may be actual 'Time' units as described
 by the data, or just TimeSteps of iteration.

\item  \verb|obj.SetStaticSeeds (int )| -  if StaticSeeds is set and the mesh is static, 
 then every time particles are injected we can re-use the same 
 injection information. We classify particles according to
 processor just once before start.
 If StaticSeeds is set and a moving seed source is specified
 the motion will be ignored and results will not be as expected.

\item  \verb|int = obj.GetStaticSeeds ()| -  if StaticSeeds is set and the mesh is static, 
 then every time particles are injected we can re-use the same 
 injection information. We classify particles according to
 processor just once before start.
 If StaticSeeds is set and a moving seed source is specified
 the motion will be ignored and results will not be as expected.

\item  \verb|obj.StaticSeedsOn ()| -  if StaticSeeds is set and the mesh is static, 
 then every time particles are injected we can re-use the same 
 injection information. We classify particles according to
 processor just once before start.
 If StaticSeeds is set and a moving seed source is specified
 the motion will be ignored and results will not be as expected.

\item  \verb|obj.StaticSeedsOff ()| -  if StaticSeeds is set and the mesh is static, 
 then every time particles are injected we can re-use the same 
 injection information. We classify particles according to
 processor just once before start.
 If StaticSeeds is set and a moving seed source is specified
 the motion will be ignored and results will not be as expected.

\item  \verb|obj.SetStaticMesh (int )| -  if StaticMesh is set, many optimizations for cell caching
 can be assumed. if StaticMesh is not set, the algorithm
 will attempt to find out if optimizations can be used, but
 setting it to true will force all optimizations.
 Do not Set StaticMesh to true if a dynamic mesh is being used
 as this will invalidate all results.

\item  \verb|int = obj.GetStaticMesh ()| -  if StaticMesh is set, many optimizations for cell caching
 can be assumed. if StaticMesh is not set, the algorithm
 will attempt to find out if optimizations can be used, but
 setting it to true will force all optimizations.
 Do not Set StaticMesh to true if a dynamic mesh is being used
 as this will invalidate all results.

\item  \verb|obj.StaticMeshOn ()| -  if StaticMesh is set, many optimizations for cell caching
 can be assumed. if StaticMesh is not set, the algorithm
 will attempt to find out if optimizations can be used, but
 setting it to true will force all optimizations.
 Do not Set StaticMesh to true if a dynamic mesh is being used
 as this will invalidate all results.

\item  \verb|obj.StaticMeshOff ()| -  if StaticMesh is set, many optimizations for cell caching
 can be assumed. if StaticMesh is not set, the algorithm
 will attempt to find out if optimizations can be used, but
 setting it to true will force all optimizations.
 Do not Set StaticMesh to true if a dynamic mesh is being used
 as this will invalidate all results.

\item  \verb|obj.SetController (vtkMultiProcessController controller)| -  Set/Get the controller used when sending particles between processes
 The controller must be an instance of vtkMPIController.
 If VTK was compiled without VTK\_USE\_MPI on, then the Controller is simply
 ignored.

\item  \verb|vtkMultiProcessController = obj.GetController ()| -  Set/Get the controller used when sending particles between processes
 The controller must be an instance of vtkMPIController.
 If VTK was compiled without VTK\_USE\_MPI on, then the Controller is simply
 ignored.

\item  \verb|obj.SetParticleWriter (vtkAbstractParticleWriter pw)| -  Set/Get the Writer associated with this Particle Tracer
 Ideally a parallel IO capable vtkH5PartWriter should be used
 which will collect particles from all parallel processes
 and write them to a single HDF5 file.

\item  \verb|vtkAbstractParticleWriter = obj.GetParticleWriter ()| -  Set/Get the Writer associated with this Particle Tracer
 Ideally a parallel IO capable vtkH5PartWriter should be used
 which will collect particles from all parallel processes
 and write them to a single HDF5 file.

\item  \verb|obj.SetParticleFileName (string )| -  Set/Get the filename to be used with the particle writer when
 dumping particles to disk    

\item  \verb|string = obj.GetParticleFileName ()| -  Set/Get the filename to be used with the particle writer when
 dumping particles to disk    

\item  \verb|obj.SetEnableParticleWriting (int )| -  Set/Get the filename to be used with the particle writer when
 dumping particles to disk    

\item  \verb|int = obj.GetEnableParticleWriting ()| -  Set/Get the filename to be used with the particle writer when
 dumping particles to disk    

\item  \verb|obj.EnableParticleWritingOn ()| -  Set/Get the filename to be used with the particle writer when
 dumping particles to disk    

\item  \verb|obj.EnableParticleWritingOff ()| -  Set/Get the filename to be used with the particle writer when
 dumping particles to disk    

\item  \verb|obj.AddSourceConnection (vtkAlgorithmOutput input)| -  Provide support for multiple see sources

\item  \verb|obj.RemoveAllSources ()| -  Provide support for multiple see sources

\end{itemize}
