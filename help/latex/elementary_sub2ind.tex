\section{SUB2IND Convert Multiple Indexing To Linear Indexing}

\subsection{Usage}

The \verb|sub2ind| function converts a multi-dimensional indexing expression
into a linear (or vector) indexing expression.  The syntax for its use
is
\begin{verbatim}
   y = sub2ind(sizevec,d1,d2,...,dn)
\end{verbatim}
where \verb|sizevec| is the size of the array being indexed into, and each
\verb|di| is a vector of the same length, containing index values.  The basic
idea behind \verb|sub2ind| is that it makes
\begin{verbatim}
  [z(d1(1),d2(1),...,dn(1)),...,z(d1(n),d2(n),...,dn(n))]
\end{verbatim}
equivalent to
\begin{verbatim}
  z(sub2ind(size(z),d1,d2,...,dn))
\end{verbatim}
where the later form is using vector indexing, and the former one is using
native, multi-dimensional indexing.
\subsection{Example}

Suppose we have a simple \verb|3 x 4| matrix \verb|A| containing some random integer
elements
\begin{verbatim}
--> A = randi(ones(3,4),10*ones(3,4))

A = 
  2  1  4  2 
  8 10  4  7 
 10  7  4 10 
\end{verbatim}
We can extract the elements \verb|(1,3),(2,3),(3,4)| of \verb|A| via \verb|sub2ind|.
To calculate which elements of \verb|A| this corresponds to, we can use
\verb|sub2ind| as
\begin{verbatim}
--> n = sub2ind(size(A),1:3,2:4)

n = 
  4  8 12 

--> A(n)

ans = 
  1  4 10 
\end{verbatim}
