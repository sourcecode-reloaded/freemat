\section{vtkDataSetEdgeSubdivisionCriterion}

\subsection{Usage}

 This is a subclass of vtkEdgeSubdivisionCriterion that is used for
 tessellating cells of a vtkDataSet, particularly nonlinear
 cells.

 It provides functions for setting the current cell being tessellated and a
 convenience routine,  EvaluateFields() to evaluate field values at a
 point. You should call  EvaluateFields() from inside  EvaluateEdge()
 whenever the result of  EvaluateEdge() will be true. Otherwise, do
 not call  EvaluateFields() as the midpoint is about to be discarded.
 (<i>Implementor's note</i>: This isn't true if UGLY\_ASPECT\_RATIO\_HACK
 has been defined. But in that case, we don't want the exact field values;
 we need the linearly interpolated ones at the midpoint for continuity.)


To create an instance of class vtkDataSetEdgeSubdivisionCriterion, simply
invoke its constructor as follows
\begin{verbatim}
  obj = vtkDataSetEdgeSubdivisionCriterion
\end{verbatim}
\subsection{Methods}

The class vtkDataSetEdgeSubdivisionCriterion has several methods that can be used.
  They are listed below.
Note that the documentation is translated automatically from the VTK sources,
and may not be completely intelligible.  When in doubt, consult the VTK website.
In the methods listed below, \verb|obj| is an instance of the vtkDataSetEdgeSubdivisionCriterion class.
\begin{itemize}
\item  \verb|string = obj.GetClassName ()|

\item  \verb|int = obj.IsA (string name)|

\item  \verb|vtkDataSetEdgeSubdivisionCriterion = obj.NewInstance ()|

\item  \verb|vtkDataSetEdgeSubdivisionCriterion = obj.SafeDownCast (vtkObject o)|

\item  \verb|obj.SetMesh (vtkDataSet )|

\item  \verb|vtkDataSet = obj.GetMesh ()|

\item  \verb|obj.SetCellId (vtkIdType cell)|

\item  \verb|vtkIdType = obj.GetCellId () const|

\item  \verb|vtkCell = obj.GetCell ()|

\item  \verb|bool = obj.EvaluateEdge (double p0, double midpt, double p1, int field\_start)|

\item  \verb|obj.EvaluatePointDataField (double result, double weights, int field)| -  Evaluate either a cell or nodal field.
 This exists because of the funky way that Exodus data will be handled.
 Sure, it's a hack, but what are ya gonna do?

\item  \verb|obj.EvaluateCellDataField (double result, double weights, int field)| -  Evaluate either a cell or nodal field.
 This exists because of the funky way that Exodus data will be handled.
 Sure, it's a hack, but what are ya gonna do?

\item  \verb|obj.SetChordError2 (double )| -  Get/Set the square of the allowable chord error at any edge's midpoint.
 This value is used by EvaluateEdge.

\item  \verb|double = obj.GetChordError2 ()| -  Get/Set the square of the allowable chord error at any edge's midpoint.
 This value is used by EvaluateEdge.

\item  \verb|obj.SetFieldError2 (int s, double err)| -  Get/Set the square of the allowable error magnitude for the
 scalar field  s at any edge's midpoint.
 A value less than or equal to 0 indicates that the field
 should not be used as a criterion for subdivision.

\item  \verb|double = obj.GetFieldError2 (int s) const| -  Get/Set the square of the allowable error magnitude for the
 scalar field  s at any edge's midpoint.
 A value less than or equal to 0 indicates that the field
 should not be used as a criterion for subdivision.

\item  \verb|obj.ResetFieldError2 ()| -  Tell the subdivider not to use any field values as subdivision criteria.
 Effectively calls SetFieldError2( a, -1. ) for all fields.

\item  \verb|int = obj.GetActiveFieldCriteria ()| -  Return a bitfield specifying which FieldError2 criteria are positive (i.e., actively
 used to decide edge subdivisions).
 This is stored as separate state to make subdivisions go faster.

\item  \verb|int = obj.GetActiveFieldCriteria () const|

\end{itemize}
