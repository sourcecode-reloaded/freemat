\section{vtkDijkstraImageContourLineInterpolator}

\subsection{Usage}

 vtkDijkstraImageContourLineInterpolator interpolates and places
 contour points on images. The class interpolates nodes by
 computing a graph lying on the image data. By graph, we mean 
 that the line interpolating the two end points traverses along 
 pixels so as to form a shortest path. A Dijkstra algorithm is 
 used to compute the path.
 
 The class is meant to be used in conjunction with 
 vtkImageActorPointPlacer. One reason for this coupling is a 
 performance issue: both classes need to perform a cell pick, and 
 coupling avoids multiple cell picks (cell picks are slow).  Another
 issue is that the interpolator may need to set the image input to 
 its vtkDijkstraImageGeodesicPath ivar.


To create an instance of class vtkDijkstraImageContourLineInterpolator, simply
invoke its constructor as follows
\begin{verbatim}
  obj = vtkDijkstraImageContourLineInterpolator
\end{verbatim}
\subsection{Methods}

The class vtkDijkstraImageContourLineInterpolator has several methods that can be used.
  They are listed below.
Note that the documentation is translated automatically from the VTK sources,
and may not be completely intelligible.  When in doubt, consult the VTK website.
In the methods listed below, \verb|obj| is an instance of the vtkDijkstraImageContourLineInterpolator class.
\begin{itemize}
\item  \verb|string = obj.GetClassName ()| -  Standard methods for instances of this class.

\item  \verb|int = obj.IsA (string name)| -  Standard methods for instances of this class.

\item  \verb|vtkDijkstraImageContourLineInterpolator = obj.NewInstance ()| -  Standard methods for instances of this class.

\item  \verb|vtkDijkstraImageContourLineInterpolator = obj.SafeDownCast (vtkObject o)| -  Standard methods for instances of this class.

\item  \verb|int = obj.InterpolateLine (vtkRenderer ren, vtkContourRepresentation rep, int idx1, int idx2)| -  Subclasses that wish to interpolate a line segment must implement this.
 For instance vtkBezierContourLineInterpolator adds nodes between idx1
 and idx2, that allow the contour to adhere to a bezier curve.

\item  \verb|obj.SetCostImage (vtkImageData )| -  Set the image data for the vtkDijkstraImageGeodesicPath.
 If not set, the interpolator uses the image data input to the image actor.
 The image actor is obtained from the expected vtkImageActorPointPlacer.

\item  \verb|vtkImageData = obj.GetCostImage ()| -  Set the image data for the vtkDijkstraImageGeodesicPath.
 If not set, the interpolator uses the image data input to the image actor.
 The image actor is obtained from the expected vtkImageActorPointPlacer.

\item  \verb|vtkDijkstraImageGeodesicPath = obj.GetDijkstraImageGeodesicPath ()| -  access to the internal dijkstra path

\end{itemize}
