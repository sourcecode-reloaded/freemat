\section{vtkOutputStream}

\subsection{Usage}

 vtkOutputStream provides a VTK-style interface wrapping around a
 standard output stream.  The access methods are virtual so that
 subclasses can transparently provide encoding of the output.  Data
 lengths for Write calls refer to the length of the data in memory.
 The actual length in the stream may differ for subclasses that
 implement an encoding scheme.

To create an instance of class vtkOutputStream, simply
invoke its constructor as follows
\begin{verbatim}
  obj = vtkOutputStream
\end{verbatim}
\subsection{Methods}

The class vtkOutputStream has several methods that can be used.
  They are listed below.
Note that the documentation is translated automatically from the VTK sources,
and may not be completely intelligible.  When in doubt, consult the VTK website.
In the methods listed below, \verb|obj| is an instance of the vtkOutputStream class.
\begin{itemize}
\item  \verb|string = obj.GetClassName ()|

\item  \verb|int = obj.IsA (string name)|

\item  \verb|vtkOutputStream = obj.NewInstance ()|

\item  \verb|vtkOutputStream = obj.SafeDownCast (vtkObject o)|

\item  \verb|int = obj.StartWriting ()| -  Called after the stream position has been set by the caller, but
 before any Write calls.  The stream position should not be
 adjusted by the caller until after an EndWriting call.

\item  \verb|int = obj.Write (string data, long length)| -  Write output data of the given length.

\item  \verb|int = obj.Write (string data, long length)| -  Write output data of the given length.

\item  \verb|int = obj.EndWriting ()| -  Called after all desired calls to Write have been made.  After
 this call, the caller is free to change the position of the
 stream.  Additional writes should not be done until after another
 call to StartWriting.

\end{itemize}
