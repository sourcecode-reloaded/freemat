\section{vtkScalarBarRepresentation}

\subsection{Usage}


 This class represents a scalar bar for a vtkScalarBarWidget.  This class
 provides support for interactively placing a scalar bar on the 2D overlay
 plane.  The scalar bar is defined by an instance of vtkScalarBarActor.

 One specialty of this class is that if the scalar bar is moved near enough
 to an edge, it's orientation is flipped to match that edge.


To create an instance of class vtkScalarBarRepresentation, simply
invoke its constructor as follows
\begin{verbatim}
  obj = vtkScalarBarRepresentation
\end{verbatim}
\subsection{Methods}

The class vtkScalarBarRepresentation has several methods that can be used.
  They are listed below.
Note that the documentation is translated automatically from the VTK sources,
and may not be completely intelligible.  When in doubt, consult the VTK website.
In the methods listed below, \verb|obj| is an instance of the vtkScalarBarRepresentation class.
\begin{itemize}
\item  \verb|string = obj.GetClassName ()|

\item  \verb|int = obj.IsA (string name)|

\item  \verb|vtkScalarBarRepresentation = obj.NewInstance ()|

\item  \verb|vtkScalarBarRepresentation = obj.SafeDownCast (vtkObject o)|

\item  \verb|vtkScalarBarActor = obj.GetScalarBarActor ()| -  The prop that is placed in the renderer.

\item  \verb|obj.SetScalarBarActor (vtkScalarBarActor )| -  The prop that is placed in the renderer.

\item  \verb|obj.BuildRepresentation ()| -  Satisfy the superclass' API.

\item  \verb|obj.WidgetInteraction (double eventPos[2])| -  Satisfy the superclass' API.

\item  \verb|obj.GetSize (double size[2])| -  These methods are necessary to make this representation behave as
 a vtkProp.

\item  \verb|obj.GetActors2D (vtkPropCollection collection)| -  These methods are necessary to make this representation behave as
 a vtkProp.

\item  \verb|obj.ReleaseGraphicsResources (vtkWindow window)| -  These methods are necessary to make this representation behave as
 a vtkProp.

\item  \verb|int = obj.RenderOverlay (vtkViewport )| -  These methods are necessary to make this representation behave as
 a vtkProp.

\item  \verb|int = obj.RenderOpaqueGeometry (vtkViewport )| -  These methods are necessary to make this representation behave as
 a vtkProp.

\item  \verb|int = obj.RenderTranslucentPolygonalGeometry (vtkViewport )| -  These methods are necessary to make this representation behave as
 a vtkProp.

\item  \verb|int = obj.HasTranslucentPolygonalGeometry ()| -  These methods are necessary to make this representation behave as
 a vtkProp.

\item  \verb|obj.SetOrientation (int orient)| -  Get/Set the orientation.

\item  \verb|int = obj.GetOrientation ()| -  Get/Set the orientation.

\end{itemize}
