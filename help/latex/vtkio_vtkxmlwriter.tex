\section{vtkXMLWriter}

\subsection{Usage}

 vtkXMLWriter provides methods implementing most of the
 functionality needed to write VTK XML file formats.  Concrete
 subclasses provide actual writer implementations calling upon this
 functionality.

To create an instance of class vtkXMLWriter, simply
invoke its constructor as follows
\begin{verbatim}
  obj = vtkXMLWriter
\end{verbatim}
\subsection{Methods}

The class vtkXMLWriter has several methods that can be used.
  They are listed below.
Note that the documentation is translated automatically from the VTK sources,
and may not be completely intelligible.  When in doubt, consult the VTK website.
In the methods listed below, \verb|obj| is an instance of the vtkXMLWriter class.
\begin{itemize}
\item  \verb|string = obj.GetClassName ()|

\item  \verb|int = obj.IsA (string name)|

\item  \verb|vtkXMLWriter = obj.NewInstance ()|

\item  \verb|vtkXMLWriter = obj.SafeDownCast (vtkObject o)|

\item  \verb|obj.SetByteOrder (int )| -  Get/Set the byte order of data written to the file.  The default
 is the machine's hardware byte order.

\item  \verb|int = obj.GetByteOrder ()| -  Get/Set the byte order of data written to the file.  The default
 is the machine's hardware byte order.

\item  \verb|obj.SetByteOrderToBigEndian ()| -  Get/Set the byte order of data written to the file.  The default
 is the machine's hardware byte order.

\item  \verb|obj.SetByteOrderToLittleEndian ()| -  Get/Set the byte order of data written to the file.  The default
 is the machine's hardware byte order.

\item  \verb|obj.SetIdType (int )| -  Get/Set the size of the vtkIdType values stored in the file.  The
 default is the real size of vtkIdType.

\item  \verb|int = obj.GetIdType ()| -  Get/Set the size of the vtkIdType values stored in the file.  The
 default is the real size of vtkIdType.

\item  \verb|obj.SetIdTypeToInt32 ()| -  Get/Set the size of the vtkIdType values stored in the file.  The
 default is the real size of vtkIdType.

\item  \verb|obj.SetIdTypeToInt64 ()| -  Get/Set the size of the vtkIdType values stored in the file.  The
 default is the real size of vtkIdType.

\item  \verb|obj.SetFileName (string )| -  Get/Set the name of the output file.

\item  \verb|string = obj.GetFileName ()| -  Get/Set the name of the output file.

\item  \verb|obj.SetCompressor (vtkDataCompressor )| -  Get/Set the compressor used to compress binary and appended data
 before writing to the file.  Default is a vtkZLibDataCompressor.

\item  \verb|vtkDataCompressor = obj.GetCompressor ()| -  Get/Set the compressor used to compress binary and appended data
 before writing to the file.  Default is a vtkZLibDataCompressor.

\item  \verb|obj.SetCompressorType (int compressorType)| -  Convenience functions to set the compressor to certain known types.

\item  \verb|obj.SetCompressorTypeToNone ()| -  Convenience functions to set the compressor to certain known types.

\item  \verb|obj.SetCompressorTypeToZLib ()| -  Get/Set the block size used in compression.  When reading, this
 controls the granularity of how much extra information must be
 read when only part of the data are requested.  The value should
 be a multiple of the largest scalar data type.

\item  \verb|obj.SetBlockSize (int blockSize)| -  Get/Set the block size used in compression.  When reading, this
 controls the granularity of how much extra information must be
 read when only part of the data are requested.  The value should
 be a multiple of the largest scalar data type.

\item  \verb|int = obj.GetBlockSize ()| -  Get/Set the block size used in compression.  When reading, this
 controls the granularity of how much extra information must be
 read when only part of the data are requested.  The value should
 be a multiple of the largest scalar data type.

\item  \verb|obj.SetDataMode (int )| -  Get/Set the data mode used for the file's data.  The options are
 vtkXMLWriter::Ascii, vtkXMLWriter::Binary, and
 vtkXMLWriter::Appended.

\item  \verb|int = obj.GetDataMode ()| -  Get/Set the data mode used for the file's data.  The options are
 vtkXMLWriter::Ascii, vtkXMLWriter::Binary, and
 vtkXMLWriter::Appended.

\item  \verb|obj.SetDataModeToAscii ()| -  Get/Set the data mode used for the file's data.  The options are
 vtkXMLWriter::Ascii, vtkXMLWriter::Binary, and
 vtkXMLWriter::Appended.

\item  \verb|obj.SetDataModeToBinary ()| -  Get/Set the data mode used for the file's data.  The options are
 vtkXMLWriter::Ascii, vtkXMLWriter::Binary, and
 vtkXMLWriter::Appended.

\item  \verb|obj.SetDataModeToAppended ()| -  Get/Set the data mode used for the file's data.  The options are
 vtkXMLWriter::Ascii, vtkXMLWriter::Binary, and
 vtkXMLWriter::Appended.

\item  \verb|obj.SetEncodeAppendedData (int )| -  Get/Set whether the appended data section is base64 encoded.  If
 encoded, reading and writing will be slower, but the file will be
 fully valid XML and text-only.  If not encoded, the XML
 specification will be violated, but reading and writing will be
 fast.  The default is to do the encoding.

\item  \verb|int = obj.GetEncodeAppendedData ()| -  Get/Set whether the appended data section is base64 encoded.  If
 encoded, reading and writing will be slower, but the file will be
 fully valid XML and text-only.  If not encoded, the XML
 specification will be violated, but reading and writing will be
 fast.  The default is to do the encoding.

\item  \verb|obj.EncodeAppendedDataOn ()| -  Get/Set whether the appended data section is base64 encoded.  If
 encoded, reading and writing will be slower, but the file will be
 fully valid XML and text-only.  If not encoded, the XML
 specification will be violated, but reading and writing will be
 fast.  The default is to do the encoding.

\item  \verb|obj.EncodeAppendedDataOff ()| -  Get/Set whether the appended data section is base64 encoded.  If
 encoded, reading and writing will be slower, but the file will be
 fully valid XML and text-only.  If not encoded, the XML
 specification will be violated, but reading and writing will be
 fast.  The default is to do the encoding.

\item  \verb|obj.SetInput (vtkDataObject )| -  Set/Get an input of this algorithm. You should not override these
 methods because they are not the only way to connect a pipeline

\item  \verb|obj.SetInput (int , vtkDataObject )| -  Set/Get an input of this algorithm. You should not override these
 methods because they are not the only way to connect a pipeline

\item  \verb|vtkDataObject = obj.GetInput (int port)| -  Set/Get an input of this algorithm. You should not override these
 methods because they are not the only way to connect a pipeline

\item  \verb|vtkDataObject = obj.GetInput ()| -  Set/Get an input of this algorithm. You should not override these
 methods because they are not the only way to connect a pipeline

\item  \verb|string = obj.GetDefaultFileExtension ()| -  Get the default file extension for files written by this writer.

\item  \verb|int = obj.Write ()| -  Invoke the writer.  Returns 1 for success, 0 for failure.

\item  \verb|obj.SetTimeStep (int )| -  Which TimeStep to write.

\item  \verb|int = obj.GetTimeStep ()| -  Which TimeStep to write.

\item  \verb|int = obj. GetTimeStepRange ()| -  Which TimeStepRange to write.

\item  \verb|obj.SetTimeStepRange (int , int )| -  Which TimeStepRange to write.

\item  \verb|obj.SetTimeStepRange (int  a[2])| -  Which TimeStepRange to write.

\item  \verb|int = obj.GetNumberOfTimeSteps ()| -  Set the number of time steps

\item  \verb|obj.SetNumberOfTimeSteps (int )| -  Set the number of time steps

\item  \verb|obj.Start ()| -  API to interface an outside the VTK pipeline control

\item  \verb|obj.Stop ()| -  API to interface an outside the VTK pipeline control

\item  \verb|obj.WriteNextTime (double time)| -  API to interface an outside the VTK pipeline control

\end{itemize}
