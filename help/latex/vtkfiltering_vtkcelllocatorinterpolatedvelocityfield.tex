\section{vtkCellLocatorInterpolatedVelocityField}

\subsection{Usage}

  vtkCellLocatorInterpolatedVelocityField acts as a continuous velocity 
  field via cell interpolation on a vtkDataSet, NumberOfIndependentVariables
  = 4 (x,y,z,t) and NumberOfFunctions = 3 (u,v,w). As a concrete sub-class
  of vtkAbstractInterpolatedVelocityField, it adopts vtkAbstractCellLocator's 
  sub-classes, e.g., vtkCellLocator and vtkModifiedBSPTree, without the use
  of vtkPointLocator ( employed by vtkDataSet/vtkPointSet::FindCell() in 
  vtkInterpolatedVelocityField ). vtkCellLocatorInterpolatedVelocityField 
  adopts one level of cell caching. Specifically, if the next point is still 
  within the previous cell, cell location is then simply skipped and vtkCell::
  EvaluatePosition() is called to obtain the new parametric coordinates and 
  weights that are used to interpolate the velocity function values across the
  vertices of this cell. Otherwise a global cell (the target containing the next
  point) location is instead directly invoked, without exploiting the clue that
  vtkInterpolatedVelocityField makes use of from the previous cell (an immediate
  neighbor). Although ignoring the neighbor cell may incur a relatively high 
  computational cost, vtkCellLocatorInterpolatedVelocityField is more robust in
  locating the target cell than its sibling class vtkInterpolatedVelocityField.

To create an instance of class vtkCellLocatorInterpolatedVelocityField, simply
invoke its constructor as follows
\begin{verbatim}
  obj = vtkCellLocatorInterpolatedVelocityField
\end{verbatim}
\subsection{Methods}

The class vtkCellLocatorInterpolatedVelocityField has several methods that can be used.
  They are listed below.
Note that the documentation is translated automatically from the VTK sources,
and may not be completely intelligible.  When in doubt, consult the VTK website.
In the methods listed below, \verb|obj| is an instance of the vtkCellLocatorInterpolatedVelocityField class.
\begin{itemize}
\item  \verb|string = obj.GetClassName ()|

\item  \verb|int = obj.IsA (string name)|

\item  \verb|vtkCellLocatorInterpolatedVelocityField = obj.NewInstance ()|

\item  \verb|vtkCellLocatorInterpolatedVelocityField = obj.SafeDownCast (vtkObject o)|

\item  \verb|vtkAbstractCellLocator = obj.GetLastCellLocator ()| -  Get the cell locator attached to the most recently visited dataset.

\item  \verb|vtkAbstractCellLocator = obj.GetCellLocatorPrototype ()| -  Get the prototype of the cell locator that is used for interpolating the
 velocity field during integration.

\item  \verb|obj.SetCellLocatorPrototype (vtkAbstractCellLocator prototype)| -  Set a prototype of the cell locator that is used for interpolating the 
 velocity field during integration.

\item  \verb|obj.CopyParameters (vtkAbstractInterpolatedVelocityField from)| -  Import parameters. Sub-classes can add more after chaining.

\item  \verb|obj.AddDataSet (vtkDataSet dataset)| -  Add a dataset coupled with a cell locator (of type vtkAbstractCellLocator)
 for vector function evaluation. Note the use of a vtkAbstractCellLocator 
 enables robust cell location. If more than one dataset is added, the 
 evaluation point is searched in all until a match is found. THIS FUNCTION
 DOES NOT CHANGE THE REFERENCE COUNT OF dataset FOR THREAD SAFETY REASONS.

\item  \verb|int = obj.FunctionValues (double x, double f)| -  Evaluate the velocity field f at point (x, y, z).

\item  \verb|obj.SetLastCellId (vtkIdType c, int dataindex)| -  Set the cell id cached by the last evaluation within a specified dataset.

\item  \verb|obj.SetLastCellId (vtkIdType c)|

\end{itemize}
