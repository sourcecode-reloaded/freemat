\section{vtkPolygonalSurfacePointPlacer}

\subsection{Usage}

 vtkPolygonalSurfacePointPlacer places points on polygonal data and is
 meant to be used in conjunction with 
 vtkPolygonalSurfaceContourLineInterpolator.

 .SECTION Usage


To create an instance of class vtkPolygonalSurfacePointPlacer, simply
invoke its constructor as follows
\begin{verbatim}
  obj = vtkPolygonalSurfacePointPlacer
\end{verbatim}
\subsection{Methods}

The class vtkPolygonalSurfacePointPlacer has several methods that can be used.
  They are listed below.
Note that the documentation is translated automatically from the VTK sources,
and may not be completely intelligible.  When in doubt, consult the VTK website.
In the methods listed below, \verb|obj| is an instance of the vtkPolygonalSurfacePointPlacer class.
\begin{itemize}
\item  \verb|string = obj.GetClassName ()| -  Standard methods for instances of this class.

\item  \verb|int = obj.IsA (string name)| -  Standard methods for instances of this class.

\item  \verb|vtkPolygonalSurfacePointPlacer = obj.NewInstance ()| -  Standard methods for instances of this class.

\item  \verb|vtkPolygonalSurfacePointPlacer = obj.SafeDownCast (vtkObject o)| -  Standard methods for instances of this class.

\item  \verb|obj.AddProp (vtkProp )|

\item  \verb|obj.RemoveViewProp (vtkProp prop)|

\item  \verb|obj.RemoveAllProps ()|

\item  \verb|int = obj.ComputeWorldPosition (vtkRenderer ren, double displayPos[2], double worldPos[3], double worldOrient[9])| -  Given a renderer and a display position in pixel coordinates,
 compute the world position and orientation where this point
 will be placed. This method is typically used by the
 representation to place the point initially.
 For the Terrain point placer this computes world points that
 lie at the specified height above the terrain.

\item  \verb|int = obj.ComputeWorldPosition (vtkRenderer ren, double displayPos[2], double refWorldPos[3], double worldPos[3], double worldOrient[9])| -  Given a renderer, a display position, and a reference world
 position, compute the new world position and orientation 
 of this point. This method is typically used by the 
 representation to move the point.

\item  \verb|int = obj.ValidateWorldPosition (double worldPos[3])| -  Given a world position check the validity of this 
 position according to the constraints of the placer

\item  \verb|int = obj.ValidateDisplayPosition (vtkRenderer , double displayPos[2])| -  Given a display position, check the validity of this position.

\item  \verb|int = obj.ValidateWorldPosition (double worldPos[3], double worldOrient[9])| -  Given a world position and a world orientation,
 validate it according to the constraints of the placer.

\item  \verb|vtkCellPicker = obj.GetCellPicker ()| -  Get the Prop picker.

\item  \verb|vtkPolyDataCollection = obj.GetPolys ()| -  Be sure to add polydata on which you wish to place points to this list
 or they will not be considered for placement.

\item  \verb|obj.SetDistanceOffset (double )| -  Height offset at which points may be placed on the polygonal surface.
 If you specify a non-zero value here, be sure to compute cell normals
 on your input polygonal data (easily done with vtkPolyDataNormals).

\item  \verb|double = obj.GetDistanceOffset ()| -  Height offset at which points may be placed on the polygonal surface.
 If you specify a non-zero value here, be sure to compute cell normals
 on your input polygonal data (easily done with vtkPolyDataNormals).

\end{itemize}
