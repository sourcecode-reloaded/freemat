\section{CONJ Conjugate Function}

\subsection{Usage}

Returns the complex conjugate of the input array for all elements.  The 
general syntax for its use is
\begin{verbatim}
   y = conj(x)
\end{verbatim}
where \verb|x| is an \verb|n|-dimensional array of numerical type.  The output 
is the same numerical type as the input.  The \verb|conj| function does
nothing to real and integer types.
\subsection{Example}

The following demonstrates the complex conjugate applied to a complex scalar.
\begin{verbatim}
--> conj(3+4*i)

ans = 

   3.0000 -  4.0000i 
\end{verbatim}
The \verb|conj| function has no effect on real arguments:
\begin{verbatim}
--> conj([2,3,4])

ans = 

 2 3 4 
\end{verbatim}
For a double-precision complex array,
\begin{verbatim}
--> conj([2.0+3.0*i,i])

ans = 

   2.0000 -  3.0000i   0.0000 -  1.0000i 
\end{verbatim}
