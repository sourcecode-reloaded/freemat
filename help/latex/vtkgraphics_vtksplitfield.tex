\section{vtkSplitField}

\subsection{Usage}

 vtkSplitField is used to split a multi-component field (vtkDataArray)
 into multiple single component fields. The new fields are put in
 the same field data as the original field. The output arrays
 are of the same type as the input array. Example:
 @verbatim
 sf->SetInputField(''gradient'', vtkSplitField::POINT\_DATA);
 sf->Split(0, ''firstcomponent'');
 @endverbatim
 tells vtkSplitField to extract the first component of the field
 called gradient and create an array called firstcomponent (the
 new field will be in the output's point data).
 The same can be done from Tcl:
 @verbatim
 sf SetInputField gradient POINT\_DATA
 sf Split 0 firstcomponent

 AttributeTypes: SCALARS, VECTORS, NORMALS, TCOORDS, TENSORS
 Field locations: DATA\_OBJECT, POINT\_DATA, CELL\_DATA
 @endverbatim
 Note that, by default, the original array is also passed through.

To create an instance of class vtkSplitField, simply
invoke its constructor as follows
\begin{verbatim}
  obj = vtkSplitField
\end{verbatim}
\subsection{Methods}

The class vtkSplitField has several methods that can be used.
  They are listed below.
Note that the documentation is translated automatically from the VTK sources,
and may not be completely intelligible.  When in doubt, consult the VTK website.
In the methods listed below, \verb|obj| is an instance of the vtkSplitField class.
\begin{itemize}
\item  \verb|string = obj.GetClassName ()|

\item  \verb|int = obj.IsA (string name)|

\item  \verb|vtkSplitField = obj.NewInstance ()|

\item  \verb|vtkSplitField = obj.SafeDownCast (vtkObject o)|

\item  \verb|obj.SetInputField (int attributeType, int fieldLoc)| -  Use the  given attribute in the field data given
 by fieldLoc as input.

\item  \verb|obj.SetInputField (string name, int fieldLoc)| -  Use the array with given name in the field data given
 by fieldLoc as input.

\item  \verb|obj.SetInputField (string name, string fieldLoc)| -  Helper method used by other language bindings. Allows the caller to
 specify arguments as strings instead of enums.

\item  \verb|obj.Split (int component, string arrayName)| -  Create a new array with the given component.

\end{itemize}
