\section{EVAL Evaluate a String}

\subsection{Usage}

The \verb|eval| function evaluates a string.  The general syntax
for its use is
\begin{verbatim}
   eval(s)
\end{verbatim}
where \verb|s| is the string to evaluate.  If \verb|s| is an expression
(instead of a set of statements), you can assign the output
of the \verb|eval| call to one or more variables, via
\begin{verbatim}
   x = eval(s)
   [x,y,z] = eval(s)
\end{verbatim}

Another form of \verb|eval| allows you to specify an expression or
set of statements to execute if an error occurs.  In this 
form, the syntax for \verb|eval| is
\begin{verbatim}
   eval(try_clause,catch_clause),
\end{verbatim}
or with return values,
\begin{verbatim}
   x = eval(try_clause,catch_clause)
   [x,y,z] = eval(try_clause,catch_clause)
\end{verbatim}
These later forms are useful for specifying defaults.  Note that
both the \verb|try_clause| and \verb|catch_clause| must be expressions,
as the equivalent code is
\begin{verbatim}
  try
    [x,y,z] = try_clause
  catch
    [x,y,z] = catch_clause
  end
\end{verbatim}
so that the assignment must make sense in both cases.
\subsection{Example}

Here are some examples of \verb|eval| being used.
@>
The primary use of the \verb|eval| statement is to enable construction
of expressions at run time.
@>
Here we demonstrate the use of the catch-clause to provide a 
default value
@>
Note that in the second case, \verb|b| takes the value of 33, indicating
that the evaluation of the first expression failed (because \verb|z| is
not defined).
