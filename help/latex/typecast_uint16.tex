\section{UINT16 Convert to Unsigned 16-bit Integer}

\subsection{Usage}

Converts the argument to an unsigned 16-bit Integer.  The syntax
for its use is
\begin{verbatim}
   y = uint16(x)
\end{verbatim}
where \verb|x| is an \verb|n|-dimensional numerical array.  Conversion
follows saturation rules (e.g., if \verb|x| is outside the normal
range for an unsigned 16-bit integer of \verb|[0,65535]|, it is truncated
to that range.  Note that
both \verb|NaN| and \verb|Inf| both map to 0.
\subsection{Example}

The following piece of code demonstrates several uses of \verb|uint16|.
\begin{verbatim}
--> uint16(200)

ans = 
 200 
\end{verbatim}
In the next example, an integer outside the range  of the type is passed in.  
The result is truncated to the maximum value of the data type.
\begin{verbatim}
--> uint16(99400)

ans = 
 65535 
\end{verbatim}
In the next example, a negative integer is passed in.  The result is 
truncated to zero.
\begin{verbatim}
--> uint16(-100)

ans = 
 0 
\end{verbatim}
In the next example, a positive double precision argument is passed in.  
The result is the unsigned integer that is closest to the argument.
\begin{verbatim}
--> uint16(pi)

ans = 
 3 
\end{verbatim}
In the next example, a complex argument is passed in.  The result is the 
complex unsigned integer that is closest to the argument.
\begin{verbatim}
--> uint16(5+2*i)

ans = 
   5.0000 +  2.0000i 
\end{verbatim}
In the next example, a string argument is passed in.  The string argument is converted into an integer array corresponding to the ASCII values of each character.
\begin{verbatim}
--> uint16('helo')

ans = 
 104 101 108 111 
\end{verbatim}
In the last example, a cell-array is passed in.  For cell-arrays and structure arrays, the result is an error.
\begin{verbatim}
--> uint16({4})
Error: Cannot perform type conversions with this type
\end{verbatim}
