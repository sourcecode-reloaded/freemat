\section{vtkTreeLayoutStrategy}

\subsection{Usage}

 Assigns points to the nodes of a tree in either a standard or radial layout.
 The standard layout places each level on a horizontal line, while the
 radial layout places each level on a concentric circle.
 You may specify the sweep angle of the tree which constrains the tree
 to be contained within a wedge. Also, you may indicate the log scale of
 the tree, which diminishes the length of arcs at lower levels of the tree.
 Values near zero give a large proportion of the space to the tree levels
 near the root, while values near one give nearly equal proportions of space
 to all tree levels.

 The user may also specify an array to use to indicate the distance from the
 root, either vertically (for standard layout) or radially
 (for radial layout).  You specify this with SetDistanceArrayName().

To create an instance of class vtkTreeLayoutStrategy, simply
invoke its constructor as follows
\begin{verbatim}
  obj = vtkTreeLayoutStrategy
\end{verbatim}
\subsection{Methods}

The class vtkTreeLayoutStrategy has several methods that can be used.
  They are listed below.
Note that the documentation is translated automatically from the VTK sources,
and may not be completely intelligible.  When in doubt, consult the VTK website.
In the methods listed below, \verb|obj| is an instance of the vtkTreeLayoutStrategy class.
\begin{itemize}
\item  \verb|string = obj.GetClassName ()|

\item  \verb|int = obj.IsA (string name)|

\item  \verb|vtkTreeLayoutStrategy = obj.NewInstance ()|

\item  \verb|vtkTreeLayoutStrategy = obj.SafeDownCast (vtkObject o)|

\item  \verb|obj.Layout ()| -  Perform the tree layout.

\item  \verb|obj.SetAngle (double )| -  The sweep angle of the tree.
 For a standard tree layout, this should be between 0 and 180.
 For a radial tree layout, this can be between 0 and 360.

\item  \verb|double = obj.GetAngleMinValue ()| -  The sweep angle of the tree.
 For a standard tree layout, this should be between 0 and 180.
 For a radial tree layout, this can be between 0 and 360.

\item  \verb|double = obj.GetAngleMaxValue ()| -  The sweep angle of the tree.
 For a standard tree layout, this should be between 0 and 180.
 For a radial tree layout, this can be between 0 and 360.

\item  \verb|double = obj.GetAngle ()| -  The sweep angle of the tree.
 For a standard tree layout, this should be between 0 and 180.
 For a radial tree layout, this can be between 0 and 360.

\item  \verb|obj.SetRadial (bool )| -  If set, the tree is laid out with levels on concentric circles 
 around the root. If unset (default), the tree is laid out with 
 levels on horizontal lines.

\item  \verb|bool = obj.GetRadial ()| -  If set, the tree is laid out with levels on concentric circles 
 around the root. If unset (default), the tree is laid out with 
 levels on horizontal lines.

\item  \verb|obj.RadialOn ()| -  If set, the tree is laid out with levels on concentric circles 
 around the root. If unset (default), the tree is laid out with 
 levels on horizontal lines.

\item  \verb|obj.RadialOff ()| -  If set, the tree is laid out with levels on concentric circles 
 around the root. If unset (default), the tree is laid out with 
 levels on horizontal lines.

\item  \verb|obj.SetLogSpacingValue (double )| -  The spacing of tree levels. Levels near zero give more space
 to levels near the root, while levels near one (the default)
 create evenly-spaced levels. Levels above one give more space
 to levels near the leaves.

\item  \verb|double = obj.GetLogSpacingValue ()| -  The spacing of tree levels. Levels near zero give more space
 to levels near the root, while levels near one (the default)
 create evenly-spaced levels. Levels above one give more space
 to levels near the leaves.

\item  \verb|obj.SetLeafSpacing (double )| -  The spacing of leaves.  Levels near one evenly space leaves
 with no gaps between subtrees.  Levels near zero creates
 large gaps between subtrees.

\item  \verb|double = obj.GetLeafSpacingMinValue ()| -  The spacing of leaves.  Levels near one evenly space leaves
 with no gaps between subtrees.  Levels near zero creates
 large gaps between subtrees.

\item  \verb|double = obj.GetLeafSpacingMaxValue ()| -  The spacing of leaves.  Levels near one evenly space leaves
 with no gaps between subtrees.  Levels near zero creates
 large gaps between subtrees.

\item  \verb|double = obj.GetLeafSpacing ()| -  The spacing of leaves.  Levels near one evenly space leaves
 with no gaps between subtrees.  Levels near zero creates
 large gaps between subtrees.

\item  \verb|obj.SetDistanceArrayName (string )| -  Get/Set the array to use to determine the distance from the
 root.

\item  \verb|string = obj.GetDistanceArrayName ()| -  Get/Set the array to use to determine the distance from the
 root.

\end{itemize}
