\section{vtkUnstructuredGridLinearRayIntegrator}

\subsection{Usage}


 vtkUnstructuredGridLinearRayIntegrator performs piecewise linear ray
 integration.  Considering that transfer functions in VTK are piecewise
 linear, this class should give the ''correct'' integration under most
 circumstances.  However, the computations performed are fairly hefty and
 should, for the most part, only be used as a benchmark for other, faster
 methods.


To create an instance of class vtkUnstructuredGridLinearRayIntegrator, simply
invoke its constructor as follows
\begin{verbatim}
  obj = vtkUnstructuredGridLinearRayIntegrator
\end{verbatim}
\subsection{Methods}

The class vtkUnstructuredGridLinearRayIntegrator has several methods that can be used.
  They are listed below.
Note that the documentation is translated automatically from the VTK sources,
and may not be completely intelligible.  When in doubt, consult the VTK website.
In the methods listed below, \verb|obj| is an instance of the vtkUnstructuredGridLinearRayIntegrator class.
\begin{itemize}
\item  \verb|string = obj.GetClassName ()|

\item  \verb|int = obj.IsA (string name)|

\item  \verb|vtkUnstructuredGridLinearRayIntegrator = obj.NewInstance ()|

\item  \verb|vtkUnstructuredGridLinearRayIntegrator = obj.SafeDownCast (vtkObject o)|

\item  \verb|obj.Initialize (vtkVolume volume, vtkDataArray scalars)|

\item  \verb|obj.Integrate (vtkDoubleArray intersectionLengths, vtkDataArray nearIntersections, vtkDataArray farIntersections, float color[4])|

\end{itemize}
