\section{vtkExtractBlock}

\subsection{Usage}

 vtkExtractBlock is a filter that extracts blocks from a multiblock dataset.
 Each node in the multi-block tree is identified by an \c index. The index can
 be obtained by performing a preorder traversal of the tree (including empty 
 nodes). eg. A(B (D, E), C(F, G)). 
 Inorder traversal yields: A, B, D, E, C, F, G
 Index of A is 0, while index of C is 4.

To create an instance of class vtkExtractBlock, simply
invoke its constructor as follows
\begin{verbatim}
  obj = vtkExtractBlock
\end{verbatim}
\subsection{Methods}

The class vtkExtractBlock has several methods that can be used.
  They are listed below.
Note that the documentation is translated automatically from the VTK sources,
and may not be completely intelligible.  When in doubt, consult the VTK website.
In the methods listed below, \verb|obj| is an instance of the vtkExtractBlock class.
\begin{itemize}
\item  \verb|string = obj.GetClassName ()|

\item  \verb|int = obj.IsA (string name)|

\item  \verb|vtkExtractBlock = obj.NewInstance ()|

\item  \verb|vtkExtractBlock = obj.SafeDownCast (vtkObject o)|

\item  \verb|obj.AddIndex (int index)| -  Select the block indices to extract. 
 Each node in the multi-block tree is identified by an \c index. The index can
 be obtained by performing a preorder traversal of the tree (including empty 
 nodes). eg. A(B (D, E), C(F, G)). 
 Inorder traversal yields: A, B, D, E, C, F, G
 Index of A is 0, while index of C is 4.

\item  \verb|obj.RemoveIndex (int index)| -  Select the block indices to extract. 
 Each node in the multi-block tree is identified by an \c index. The index can
 be obtained by performing a preorder traversal of the tree (including empty 
 nodes). eg. A(B (D, E), C(F, G)). 
 Inorder traversal yields: A, B, D, E, C, F, G
 Index of A is 0, while index of C is 4.

\item  \verb|obj.RemoveAllIndices ()| -  Select the block indices to extract. 
 Each node in the multi-block tree is identified by an \c index. The index can
 be obtained by performing a preorder traversal of the tree (including empty 
 nodes). eg. A(B (D, E), C(F, G)). 
 Inorder traversal yields: A, B, D, E, C, F, G
 Index of A is 0, while index of C is 4.

\item  \verb|obj.SetPruneOutput (int )| -  When set, the output mutliblock dataset will be pruned to remove empty
 nodes. On by default.

\item  \verb|int = obj.GetPruneOutput ()| -  When set, the output mutliblock dataset will be pruned to remove empty
 nodes. On by default.

\item  \verb|obj.PruneOutputOn ()| -  When set, the output mutliblock dataset will be pruned to remove empty
 nodes. On by default.

\item  \verb|obj.PruneOutputOff ()| -  When set, the output mutliblock dataset will be pruned to remove empty
 nodes. On by default.

\item  \verb|obj.SetMaintainStructure (int )| -  This is used only when PruneOutput is ON. By default, when pruning the
 output i.e. remove empty blocks, if node has only 1 non-null child block,
 then that node is removed. To preserve these parent nodes, set this flag to
 true. Off by default.

\item  \verb|int = obj.GetMaintainStructure ()| -  This is used only when PruneOutput is ON. By default, when pruning the
 output i.e. remove empty blocks, if node has only 1 non-null child block,
 then that node is removed. To preserve these parent nodes, set this flag to
 true. Off by default.

\item  \verb|obj.MaintainStructureOn ()| -  This is used only when PruneOutput is ON. By default, when pruning the
 output i.e. remove empty blocks, if node has only 1 non-null child block,
 then that node is removed. To preserve these parent nodes, set this flag to
 true. Off by default.

\item  \verb|obj.MaintainStructureOff ()| -  This is used only when PruneOutput is ON. By default, when pruning the
 output i.e. remove empty blocks, if node has only 1 non-null child block,
 then that node is removed. To preserve these parent nodes, set this flag to
 true. Off by default.

\end{itemize}
