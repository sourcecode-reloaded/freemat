\section{vtkGeoProjection}

\subsection{Usage}

 This class uses the PROJ.4 library to represent geographic coordinate
 projections.

To create an instance of class vtkGeoProjection, simply
invoke its constructor as follows
\begin{verbatim}
  obj = vtkGeoProjection
\end{verbatim}
\subsection{Methods}

The class vtkGeoProjection has several methods that can be used.
  They are listed below.
Note that the documentation is translated automatically from the VTK sources,
and may not be completely intelligible.  When in doubt, consult the VTK website.
In the methods listed below, \verb|obj| is an instance of the vtkGeoProjection class.
\begin{itemize}
\item  \verb|string = obj.GetClassName ()|

\item  \verb|int = obj.IsA (string name)|

\item  \verb|vtkGeoProjection = obj.NewInstance ()|

\item  \verb|vtkGeoProjection = obj.SafeDownCast (vtkObject o)|

\item  \verb|obj.SetName (string )| -  Set/get the short name describing the projection you wish to use.
 This defaults to ''rpoly'' for no reason other than I like it.
 To get a list of valid values, use the GetNumberOfProjections() and
 GetProjectionName(int) static methods.

\item  \verb|string = obj.GetName ()| -  Set/get the short name describing the projection you wish to use.
 This defaults to ''rpoly'' for no reason other than I like it.
 To get a list of valid values, use the GetNumberOfProjections() and
 GetProjectionName(int) static methods.

\item  \verb|int = obj.GetIndex ()| -  Return the index of the current projection's type in the list of all projection types.
 On error, this will return -1. On success, it returns a number in [0,GetNumberOfProjections()[.

\item  \verb|string = obj.GetDescription ()| -  Get the description of a projection.
 This will return NULL if the projection name is invalid.

\item  \verb|obj.SetCentralMeridian (double )| -  Set/get the longitude which corresponds to the central meridian of the projection.
 This defaults to 0, the Greenwich Meridian.

\item  \verb|double = obj.GetCentralMeridian ()| -  Set/get the longitude which corresponds to the central meridian of the projection.
 This defaults to 0, the Greenwich Meridian.

\end{itemize}
