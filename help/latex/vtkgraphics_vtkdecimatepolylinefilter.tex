\section{vtkDecimatePolylineFilter}

\subsection{Usage}

 vtkDecimatePolylineFilter is a filter to reduce the number of lines in a
 polyline. The algorithm functions by evaluating an error metric for each
 vertex (i.e., the distance of the vertex to a line defined from the two
 vertices on either side of the vertex). Then, these vertices are placed
 into a priority queue, and those with larger errors are deleted first.
 The decimation continues until the target reduction is reached.


To create an instance of class vtkDecimatePolylineFilter, simply
invoke its constructor as follows
\begin{verbatim}
  obj = vtkDecimatePolylineFilter
\end{verbatim}
\subsection{Methods}

The class vtkDecimatePolylineFilter has several methods that can be used.
  They are listed below.
Note that the documentation is translated automatically from the VTK sources,
and may not be completely intelligible.  When in doubt, consult the VTK website.
In the methods listed below, \verb|obj| is an instance of the vtkDecimatePolylineFilter class.
\begin{itemize}
\item  \verb|string = obj.GetClassName ()| -  Standard methods for type information and printing.

\item  \verb|int = obj.IsA (string name)| -  Standard methods for type information and printing.

\item  \verb|vtkDecimatePolylineFilter = obj.NewInstance ()| -  Standard methods for type information and printing.

\item  \verb|vtkDecimatePolylineFilter = obj.SafeDownCast (vtkObject o)| -  Standard methods for type information and printing.

\item  \verb|obj.SetTargetReduction (double )| -  Specify the desired reduction in the total number of polygons (e.g., if
 TargetReduction is set to 0.9, this filter will try to reduce the data set
 to 10% of its original size). 

\item  \verb|double = obj.GetTargetReductionMinValue ()| -  Specify the desired reduction in the total number of polygons (e.g., if
 TargetReduction is set to 0.9, this filter will try to reduce the data set
 to 10% of its original size). 

\item  \verb|double = obj.GetTargetReductionMaxValue ()| -  Specify the desired reduction in the total number of polygons (e.g., if
 TargetReduction is set to 0.9, this filter will try to reduce the data set
 to 10% of its original size). 

\item  \verb|double = obj.GetTargetReduction ()| -  Specify the desired reduction in the total number of polygons (e.g., if
 TargetReduction is set to 0.9, this filter will try to reduce the data set
 to 10% of its original size). 

\end{itemize}
