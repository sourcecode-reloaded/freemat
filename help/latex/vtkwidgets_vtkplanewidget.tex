\section{vtkPlaneWidget}

\subsection{Usage}

 This 3D widget defines a finite (bounded) plane that can be interactively
 placed in a scene. The plane has four handles (at its corner vertices), a
 normal vector, and the plane itself. The handles are used to resize the
 plane; the normal vector to rotate it, and the plane can be picked and
 translated. Selecting the plane while pressing CTRL makes it spin around 
 the normal. A nice feature of the object is that the vtkPlaneWidget, like
 any 3D widget, will work with the current interactor style. That is, if
 vtkPlaneWidget does not handle an event, then all other registered
 observers (including the interactor style) have an opportunity to process
 the event. Otherwise, the vtkPlaneWidget will terminate the processing of
 the event that it handles.

 To use this object, just invoke SetInteractor() with the argument of the
 method a vtkRenderWindowInteractor.  You may also wish to invoke
 ''PlaceWidget()'' to initially position the widget. If the ''i'' key (for
 ''interactor'') is pressed, the vtkPlaneWidget will appear. (See superclass
 documentation for information about changing this behavior.) By grabbing
 the one of the four handles (use the left mouse button), the plane can be
 resized.  By grabbing the plane itself, the entire plane can be
 arbitrarily translated. Pressing CTRL while grabbing the plane will spin
 the plane around the normal. If you select the normal vector, the plane can be
 arbitrarily rotated. Selecting any part of the widget with the middle
 mouse button enables translation of the plane along its normal. (Once
 selected using middle mouse, moving the mouse in the direction of the
 normal translates the plane in the direction of the normal; moving in the
 direction opposite the normal translates the plane in the direction
 opposite the normal.) Scaling (about the center of the plane) is achieved
 by using the right mouse button. By moving the mouse ''up'' the render
 window the plane will be made bigger; by moving ''down'' the render window
 the widget will be made smaller. Events that occur outside of the widget
 (i.e., no part of the widget is picked) are propagated to any other
 registered obsevers (such as the interaction style).  Turn off the widget
 by pressing the ''i'' key again (or invoke the Off() method).

 The vtkPlaneWidget has several methods that can be used in conjunction
 with other VTK objects. The Set/GetResolution() methods control the number
 of subdivisions of the plane; the GetPolyData() method can be used to get
 the polygonal representation and can be used for things like seeding
 stream lines. GetPlane() can be used to update a vtkPlane implicit
 function. Typical usage of the widget is to make use of the
 StartInteractionEvent, InteractionEvent, and EndInteractionEvent
 events. The InteractionEvent is called on mouse motion; the other two
 events are called on button down and button up (either left or right
 button).

 Some additional features of this class include the ability to control the
 properties of the widget. You can set the properties of the selected and
 unselected representations of the plane. For example, you can set the
 property for the handles and plane. In addition there are methods to
 constrain the plane so that it is perpendicular to the x-y-z axes.

To create an instance of class vtkPlaneWidget, simply
invoke its constructor as follows
\begin{verbatim}
  obj = vtkPlaneWidget
\end{verbatim}
\subsection{Methods}

The class vtkPlaneWidget has several methods that can be used.
  They are listed below.
Note that the documentation is translated automatically from the VTK sources,
and may not be completely intelligible.  When in doubt, consult the VTK website.
In the methods listed below, \verb|obj| is an instance of the vtkPlaneWidget class.
\begin{itemize}
\item  \verb|string = obj.GetClassName ()|

\item  \verb|int = obj.IsA (string name)|

\item  \verb|vtkPlaneWidget = obj.NewInstance ()|

\item  \verb|vtkPlaneWidget = obj.SafeDownCast (vtkObject o)|

\item  \verb|obj.SetEnabled (int )| -  Methods that satisfy the superclass' API.

\item  \verb|obj.PlaceWidget (double bounds[6])| -  Methods that satisfy the superclass' API.

\item  \verb|obj.PlaceWidget ()| -  Methods that satisfy the superclass' API.

\item  \verb|obj.PlaceWidget (double xmin, double xmax, double ymin, double ymax, double zmin, double zmax)| -  Set/Get the resolution (number of subdivisions) of the plane.

\item  \verb|obj.SetResolution (int r)| -  Set/Get the resolution (number of subdivisions) of the plane.

\item  \verb|int = obj.GetResolution ()| -  Set/Get the resolution (number of subdivisions) of the plane.

\item  \verb|obj.SetOrigin (double x, double y, double z)| -  Set/Get the origin of the plane.

\item  \verb|obj.SetOrigin (double x[3])| -  Set/Get the origin of the plane.

\item  \verb|double = obj.GetOrigin ()| -  Set/Get the origin of the plane.

\item  \verb|obj.GetOrigin (double xyz[3])| -  Set/Get the origin of the plane.

\item  \verb|obj.SetPoint1 (double x, double y, double z)| -  Set/Get the position of the point defining the first axis of the plane.

\item  \verb|obj.SetPoint1 (double x[3])| -  Set/Get the position of the point defining the first axis of the plane.

\item  \verb|double = obj.GetPoint1 ()| -  Set/Get the position of the point defining the first axis of the plane.

\item  \verb|obj.GetPoint1 (double xyz[3])| -  Set/Get the position of the point defining the first axis of the plane.

\item  \verb|obj.SetPoint2 (double x, double y, double z)| -  Set/Get the position of the point defining the second axis of the plane.

\item  \verb|obj.SetPoint2 (double x[3])| -  Set/Get the position of the point defining the second axis of the plane.

\item  \verb|double = obj.GetPoint2 ()| -  Set/Get the position of the point defining the second axis of the plane.

\item  \verb|obj.GetPoint2 (double xyz[3])| -  Set/Get the position of the point defining the second axis of the plane.

\item  \verb|obj.SetCenter (double x, double y, double z)| -  Get the center of the plane.

\item  \verb|obj.SetCenter (double x[3])| -  Get the center of the plane.

\item  \verb|double = obj.GetCenter ()| -  Get the center of the plane.

\item  \verb|obj.GetCenter (double xyz[3])| -  Get the center of the plane.

\item  \verb|obj.SetNormal (double x, double y, double z)| -  Get the normal to the plane.

\item  \verb|obj.SetNormal (double x[3])| -  Get the normal to the plane.

\item  \verb|double = obj.GetNormal ()| -  Get the normal to the plane.

\item  \verb|obj.GetNormal (double xyz[3])| -  Get the normal to the plane.

\item  \verb|obj.SetRepresentation (int )| -  Control how the plane appears when GetPolyData() is invoked.
 If the mode is ''outline'', then just the outline of the plane
 is shown. If the mode is ''wireframe'' then the plane is drawn
 with the outline plus the interior mesh (corresponding to the
 resolution specified). If the mode is ''surface'' then the plane
 is drawn as a surface.

\item  \verb|int = obj.GetRepresentationMinValue ()| -  Control how the plane appears when GetPolyData() is invoked.
 If the mode is ''outline'', then just the outline of the plane
 is shown. If the mode is ''wireframe'' then the plane is drawn
 with the outline plus the interior mesh (corresponding to the
 resolution specified). If the mode is ''surface'' then the plane
 is drawn as a surface.

\item  \verb|int = obj.GetRepresentationMaxValue ()| -  Control how the plane appears when GetPolyData() is invoked.
 If the mode is ''outline'', then just the outline of the plane
 is shown. If the mode is ''wireframe'' then the plane is drawn
 with the outline plus the interior mesh (corresponding to the
 resolution specified). If the mode is ''surface'' then the plane
 is drawn as a surface.

\item  \verb|int = obj.GetRepresentation ()| -  Control how the plane appears when GetPolyData() is invoked.
 If the mode is ''outline'', then just the outline of the plane
 is shown. If the mode is ''wireframe'' then the plane is drawn
 with the outline plus the interior mesh (corresponding to the
 resolution specified). If the mode is ''surface'' then the plane
 is drawn as a surface.

\item  \verb|obj.SetRepresentationToOff ()| -  Control how the plane appears when GetPolyData() is invoked.
 If the mode is ''outline'', then just the outline of the plane
 is shown. If the mode is ''wireframe'' then the plane is drawn
 with the outline plus the interior mesh (corresponding to the
 resolution specified). If the mode is ''surface'' then the plane
 is drawn as a surface.

\item  \verb|obj.SetRepresentationToOutline ()| -  Control how the plane appears when GetPolyData() is invoked.
 If the mode is ''outline'', then just the outline of the plane
 is shown. If the mode is ''wireframe'' then the plane is drawn
 with the outline plus the interior mesh (corresponding to the
 resolution specified). If the mode is ''surface'' then the plane
 is drawn as a surface.

\item  \verb|obj.SetRepresentationToWireframe ()| -  Control how the plane appears when GetPolyData() is invoked.
 If the mode is ''outline'', then just the outline of the plane
 is shown. If the mode is ''wireframe'' then the plane is drawn
 with the outline plus the interior mesh (corresponding to the
 resolution specified). If the mode is ''surface'' then the plane
 is drawn as a surface.

\item  \verb|obj.SetRepresentationToSurface ()| -  Force the plane widget to be aligned with one of the x-y-z axes.
 Remember that when the state changes, a ModifiedEvent is invoked.
 This can be used to snap the plane to the axes if it is orginally
 not aligned.

\item  \verb|obj.SetNormalToXAxis (int )| -  Force the plane widget to be aligned with one of the x-y-z axes.
 Remember that when the state changes, a ModifiedEvent is invoked.
 This can be used to snap the plane to the axes if it is orginally
 not aligned.

\item  \verb|int = obj.GetNormalToXAxis ()| -  Force the plane widget to be aligned with one of the x-y-z axes.
 Remember that when the state changes, a ModifiedEvent is invoked.
 This can be used to snap the plane to the axes if it is orginally
 not aligned.

\item  \verb|obj.NormalToXAxisOn ()| -  Force the plane widget to be aligned with one of the x-y-z axes.
 Remember that when the state changes, a ModifiedEvent is invoked.
 This can be used to snap the plane to the axes if it is orginally
 not aligned.

\item  \verb|obj.NormalToXAxisOff ()| -  Force the plane widget to be aligned with one of the x-y-z axes.
 Remember that when the state changes, a ModifiedEvent is invoked.
 This can be used to snap the plane to the axes if it is orginally
 not aligned.

\item  \verb|obj.SetNormalToYAxis (int )| -  Force the plane widget to be aligned with one of the x-y-z axes.
 Remember that when the state changes, a ModifiedEvent is invoked.
 This can be used to snap the plane to the axes if it is orginally
 not aligned.

\item  \verb|int = obj.GetNormalToYAxis ()| -  Force the plane widget to be aligned with one of the x-y-z axes.
 Remember that when the state changes, a ModifiedEvent is invoked.
 This can be used to snap the plane to the axes if it is orginally
 not aligned.

\item  \verb|obj.NormalToYAxisOn ()| -  Force the plane widget to be aligned with one of the x-y-z axes.
 Remember that when the state changes, a ModifiedEvent is invoked.
 This can be used to snap the plane to the axes if it is orginally
 not aligned.

\item  \verb|obj.NormalToYAxisOff ()| -  Force the plane widget to be aligned with one of the x-y-z axes.
 Remember that when the state changes, a ModifiedEvent is invoked.
 This can be used to snap the plane to the axes if it is orginally
 not aligned.

\item  \verb|obj.SetNormalToZAxis (int )| -  Force the plane widget to be aligned with one of the x-y-z axes.
 Remember that when the state changes, a ModifiedEvent is invoked.
 This can be used to snap the plane to the axes if it is orginally
 not aligned.

\item  \verb|int = obj.GetNormalToZAxis ()| -  Force the plane widget to be aligned with one of the x-y-z axes.
 Remember that when the state changes, a ModifiedEvent is invoked.
 This can be used to snap the plane to the axes if it is orginally
 not aligned.

\item  \verb|obj.NormalToZAxisOn ()| -  Force the plane widget to be aligned with one of the x-y-z axes.
 Remember that when the state changes, a ModifiedEvent is invoked.
 This can be used to snap the plane to the axes if it is orginally
 not aligned.

\item  \verb|obj.NormalToZAxisOff ()| -  Force the plane widget to be aligned with one of the x-y-z axes.
 Remember that when the state changes, a ModifiedEvent is invoked.
 This can be used to snap the plane to the axes if it is orginally
 not aligned.

\item  \verb|obj.GetPolyData (vtkPolyData pd)| -  Grab the polydata (including points) that defines the plane.  The
 polydata consists of (res+1)*(res+1) points, and res*res quadrilateral
 polygons, where res is the resolution of the plane. These point values
 are guaranteed to be up-to-date when either the InteractionEvent or
 EndInteraction events are invoked. The user provides the vtkPolyData and
 the points and polyplane are added to it.

\item  \verb|obj.GetPlane (vtkPlane plane)| -  Get the planes describing the implicit function defined by the plane
 widget. The user must provide the instance of the class vtkPlane. Note
 that vtkPlane is a subclass of vtkImplicitFunction, meaning that it can
 be used by a variety of filters to perform clipping, cutting, and
 selection of data.

\item  \verb|vtkPolyDataAlgorithm = obj.GetPolyDataAlgorithm ()| -  Satisfies superclass API.  This returns a pointer to the underlying
 PolyData.  Make changes to this before calling the initial PlaceWidget()
 to have the initial placement follow suit.  Or, make changes after the
 widget has been initialised and call UpdatePlacement() to realise.

\item  \verb|obj.UpdatePlacement (void )| -  Satisfies superclass API.  This will change the state of the widget to
 match changes that have been made to the underlying PolyDataSource

\item  \verb|vtkProperty = obj.GetHandleProperty ()| -  Get the handle properties (the little balls are the handles). The 
 properties of the handles when selected and normal can be 
 manipulated.

\item  \verb|vtkProperty = obj.GetSelectedHandleProperty ()| -  Get the handle properties (the little balls are the handles). The 
 properties of the handles when selected and normal can be 
 manipulated.

\item  \verb|obj.SetPlaneProperty (vtkProperty )| -  Get the plane properties. The properties of the plane when selected 
 and unselected can be manipulated.

\item  \verb|vtkProperty = obj.GetPlaneProperty ()| -  Get the plane properties. The properties of the plane when selected 
 and unselected can be manipulated.

\item  \verb|vtkProperty = obj.GetSelectedPlaneProperty ()| -  Get the plane properties. The properties of the plane when selected 
 and unselected can be manipulated.

\end{itemize}
