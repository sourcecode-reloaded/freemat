\section{POLYDER Polynomial Coefficient Differentiation}

\subsection{Usage}

The \verb|polyder| function returns the polynomial coefficients resulting
from differentiation of polynomial \verb|p|. The syntax for its use is either
\begin{verbatim}
 pder = polyder(p)
\end{verbatim}
 for the derivitave of polynomial p, or
\begin{verbatim}
 convp1p2der = polyder(p1,p2)
\end{verbatim}
 for the derivitave of polynomial conv(p1,p2), or still
\begin{verbatim}
 [nder,dder] = polyder(n,d)
\end{verbatim}
for the derivative of polynomial \verb|n/d| (\verb|nder| is the numerator
and \verb|dder| is the denominator). In all cases the polynomial 
coefficients are assumed to be in decreasing degree.
Contributed by Paulo Xavier Candeias under GPL
\subsection{Example}

Here are some examples of the use of \verb|polyder|
\begin{verbatim}
--> polyder([2,3,4])

ans = 
 4 3 
\end{verbatim}
\begin{verbatim}
--> polyder([2,3,4],7)

ans = 
 28 21 
\end{verbatim}
\begin{verbatim}
--> [n,d] = polyder([2,3,4],5)
n = 
 -20 -15 

d = 
 25 
\end{verbatim}
