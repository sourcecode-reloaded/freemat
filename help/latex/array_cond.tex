\section{COND Condition Number of a Matrix}

\subsection{Usage}

Calculates the condition number of a matrix.  To compute the
2-norm condition number of a matrix (ratio of largest to smallest
singular values), use the syntax
\begin{verbatim}
   y = cond(A)
\end{verbatim}
where A is a matrix.  If you want to compute the condition number
in a different norm (e.g., the 1-norm), use the second syntax
\begin{verbatim}
   y = cond(A,p)
\end{verbatim}
where \verb|p| is the norm to use when computing the condition number.
The following choices of \verb|p| are supported
\begin{itemize}
\item  \verb|p = 1| returns the 1-norm, or the max column sum of A

\item  \verb|p = 2| returns the 2-norm (largest singular value of A)

\item  \verb|p = inf| returns the infinity norm, or the max row sum of A

\item  \verb|p = 'fro'| returns the Frobenius-norm (vector Euclidean norm, or RMS value)

\end{itemize}
\subsection{Function Internals}

The condition number is defined as
\[
  \frac{\|A\|_p}{\|A^{-1}\|_p}
\]
This equation is precisely how the condition number is computed for
the case \verb|p ~= 2|.  For the \verb|p=2| case, the condition number can 
be computed much more efficiently using the ratio of the largest and
smallest singular values.
\subsection{Example}

The condition number of this matrix is large
@>
You can also (for the case \verb|p=1| use \verb|rcond| to calculate an estimate
of the condition number
@>
