\section{vtkRungeKutta45}

\subsection{Usage}

 This is a concrete sub-class of vtkInitialValueProblemSolver.
 It uses a 5th order Runge-Kutta method with stepsize control to obtain 
 the values of a set of functions at the next time step. The stepsize
 is adjusted by calculating an estimated error using an embedded 4th
 order Runge-Kutta formula:
 Press, W. H. et al., 1992, Numerical Recipes in Fortran, Second
 Edition, Cambridge University Press
 Cash, J.R. and Karp, A.H. 1990, ACM Transactions on Mathematical
 Software, vol 16, pp 201-222

To create an instance of class vtkRungeKutta45, simply
invoke its constructor as follows
\begin{verbatim}
  obj = vtkRungeKutta45
\end{verbatim}
\subsection{Methods}

The class vtkRungeKutta45 has several methods that can be used.
  They are listed below.
Note that the documentation is translated automatically from the VTK sources,
and may not be completely intelligible.  When in doubt, consult the VTK website.
In the methods listed below, \verb|obj| is an instance of the vtkRungeKutta45 class.
\begin{itemize}
\item  \verb|string = obj.GetClassName ()|

\item  \verb|int = obj.IsA (string name)|

\item  \verb|vtkRungeKutta45 = obj.NewInstance ()|

\item  \verb|vtkRungeKutta45 = obj.SafeDownCast (vtkObject o)|

\end{itemize}
