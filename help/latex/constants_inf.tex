\section{INF Infinity Constant}

\subsection{Usage}

Returns a value that represents positive infinity 
for both 32 and 64-bit floating point values.
\begin{verbatim}
   y = inf
\end{verbatim}
The returned type is a 64-bit float, but demotion to
64 bits preserves the infinity.
\subsection{Function Internals}

The infinity constant has
several interesting properties.  In particular:
\[
\begin{array}{ll}
   \infty \times 0 & = \mathrm{NaN} \\                                             \infty \times a & = \infty \, \mathrm{for all} \, a > 0 \\   \infty \times a & = -\infty \, \mathrm{for all} \, a < 0 \\   \infty / \infty & = \mathrm{NaN} \\   \infty / 0 & = \infty 
\end{array}
\]
Note that infinities are not preserved under type conversion to integer types (see the examples below).
\subsection{Example}

The following examples demonstrate the various properties of the infinity constant.
\begin{verbatim}
--> inf*0

ans = 

       nan 

--> inf*2

ans = 

       inf 

--> inf*-2

ans = 

      -inf 

--> inf/inf

ans = 

       nan 

--> inf/0

ans = 

       inf 

--> inf/nan

ans = 

       nan 
\end{verbatim}
Note that infinities are preserved under type conversion to floating point types (i.e., \verb|float|, \verb|double|, \verb|complex| and \verb|dcomplex| types), but not integer  types.
\begin{verbatim}
--> uint32(inf)

ans = 

 4294967295 

--> complex(inf)

ans = 

       inf 
\end{verbatim}
