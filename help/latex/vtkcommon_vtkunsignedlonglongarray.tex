\section{vtkUnsignedLongLongArray}

\subsection{Usage}

 vtkUnsignedLongLongArray is an array of values of type unsigned long long.
 It provides methods for insertion and retrieval of values and will
 automatically resize itself to hold new data.

To create an instance of class vtkUnsignedLongLongArray, simply
invoke its constructor as follows
\begin{verbatim}
  obj = vtkUnsignedLongLongArray
\end{verbatim}
\subsection{Methods}

The class vtkUnsignedLongLongArray has several methods that can be used.
  They are listed below.
Note that the documentation is translated automatically from the VTK sources,
and may not be completely intelligible.  When in doubt, consult the VTK website.
In the methods listed below, \verb|obj| is an instance of the vtkUnsignedLongLongArray class.
\begin{itemize}
\item  \verb|string = obj.GetClassName ()|

\item  \verb|int = obj.IsA (string name)|

\item  \verb|vtkUnsignedLongLongArray = obj.NewInstance ()|

\item  \verb|vtkUnsignedLongLongArray = obj.SafeDownCast (vtkObject o)|

\item  \verb|int = obj.GetDataType ()| -  Copy the tuple value into a user-provided array.

\item  \verb|long = obj.long GetValue (vtkIdType id)| -  Set the data at a particular index. Does not do range checking. Make sure
 you use the method SetNumberOfValues() before inserting data.

\item  \verb|obj.SetValue (vtkIdType id, long long value)| -  Specify the number of values for this object to hold. Does an
 allocation as well as setting the MaxId ivar. Used in conjunction with
 SetValue() method for fast insertion.

\item  \verb|obj.SetNumberOfValues (vtkIdType number)| -  Insert data at a specified position in the array.

\item  \verb|obj.InsertValue (vtkIdType id, long long f)| -  Insert data at the end of the array. Return its location in the array.

\item  \verb|vtkIdType = obj.InsertNextValue (long long f)| -  Get the address of a particular data index. Make sure data is allocated
 for the number of items requested. Set MaxId according to the number of
 data values requested.

\item  \verb|long = obj.long WritePointer (vtkIdType id, vtkIdType number)| -  Get the address of a particular data index. Performs no checks
 to verify that the memory has been allocated etc.

\item  \verb|long = obj.long GetPointer (vtkIdType id)| -  This method lets the user specify data to be held by the array.  The
 array argument is a pointer to the data.  size is the size of
 the array supplied by the user.  Set save to 1 to keep the class
 from deleting the array when it cleans up or reallocates memory.
 The class uses the actual array provided; it does not copy the data
 from the suppled array. 

\end{itemize}
