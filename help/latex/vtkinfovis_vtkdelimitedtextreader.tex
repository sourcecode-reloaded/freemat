\section{vtkDelimitedTextReader}

\subsection{Usage}

 vtkDelimitedTextReader is an interface for pulling in data from a
 flat, delimited ascii or unicode text file (delimiter can be any character).

 The behavior of the reader with respect to ascii or unicode input is controlled
 by the SetUnicodeCharacterSet() method.  By default (without calling SetUnicodeCharacterSet()),
 the reader will expect to read ascii text and will output vtkStdString columns.  Use 
 the Set and Get methods to set delimiters that do not contain UTF8 in the name when operating 
 the reader in default ascii mode.  If the SetUnicodeCharacterSet() method is called, the reader 
 will output vtkUnicodeString columns in the output table.  In addition, it is necessary to use 
 the Set and Get methods that contain UTF8 in the name to specify delimiters when operating in 
 unicode mode.

 This class emits ProgressEvent for every 100 lines it reads.

 .SECTION Thanks
 Thanks to Andy Wilson, Brian Wylie, Tim Shead, and Thomas Otahal
 from Sandia National Laboratories for implementing this class.
 

To create an instance of class vtkDelimitedTextReader, simply
invoke its constructor as follows
\begin{verbatim}
  obj = vtkDelimitedTextReader
\end{verbatim}
\subsection{Methods}

The class vtkDelimitedTextReader has several methods that can be used.
  They are listed below.
Note that the documentation is translated automatically from the VTK sources,
and may not be completely intelligible.  When in doubt, consult the VTK website.
In the methods listed below, \verb|obj| is an instance of the vtkDelimitedTextReader class.
\begin{itemize}
\item  \verb|string = obj.GetClassName ()|

\item  \verb|int = obj.IsA (string name)|

\item  \verb|vtkDelimitedTextReader = obj.NewInstance ()|

\item  \verb|vtkDelimitedTextReader = obj.SafeDownCast (vtkObject o)|

\item  \verb|string = obj.GetFileName ()|

\item  \verb|obj.SetFileName (string )|

\item  \verb|string = obj.GetUnicodeCharacterSet ()| -  Specifies the character set used in the input file.  Valid character set
 names will be drawn from the list maintained by the Internet Assigned Name
 Authority at

   http://www.iana.org/assignments/character-sets

 Where multiple aliases are provided for a character set, the preferred MIME name
 will be used.  vtkUnicodeDelimitedTextReader currently supports ''US-ASCII'', ''UTF-8'',
 ''UTF-16'', ''UTF-16BE'', and ''UTF-16LE'' character sets.

\item  \verb|obj.SetUnicodeCharacterSet (string )| -  Specifies the character set used in the input file.  Valid character set
 names will be drawn from the list maintained by the Internet Assigned Name
 Authority at

   http://www.iana.org/assignments/character-sets

 Where multiple aliases are provided for a character set, the preferred MIME name
 will be used.  vtkUnicodeDelimitedTextReader currently supports ''US-ASCII'', ''UTF-8'',
 ''UTF-16'', ''UTF-16BE'', and ''UTF-16LE'' character sets.

\item  \verb|obj.SetUTF8RecordDelimiters (string delimiters)| -  Specify the character(s) that will be used to separate records.
 The order of characters in the string does not matter.  Defaults
 to ''\r\\n''.

\item  \verb|string = obj.GetUTF8RecordDelimiters ()| -  Specify the character(s) that will be used to separate records.
 The order of characters in the string does not matter.  Defaults
 to ''\r\\n''.

\item  \verb|obj.SetFieldDelimiterCharacters (string )| -  Specify the character(s) that will be used to separate fields.  For
 example, set this to '','' for a comma-separated value file.  Set
 it to ''.:;'' for a file where columns can be separated by a
 period, colon or semicolon.  The order of the characters in the
 string does not matter.  Defaults to a comma.

\item  \verb|string = obj.GetFieldDelimiterCharacters ()| -  Specify the character(s) that will be used to separate fields.  For
 example, set this to '','' for a comma-separated value file.  Set
 it to ''.:;'' for a file where columns can be separated by a
 period, colon or semicolon.  The order of the characters in the
 string does not matter.  Defaults to a comma.

\item  \verb|obj.SetUTF8FieldDelimiters (string delimiters)|

\item  \verb|string = obj.GetUTF8FieldDelimiters ()|

\item  \verb|char = obj.GetStringDelimiter ()| -  Get/set the character that will begin and end strings.  Microsoft
 Excel, for example, will export the following format:

 ''First Field'',''Second Field'',''Field, With, Commas'',''Fourth Field''

 The third field has a comma in it.  By using a string delimiter,
 this will be correctly read.  The delimiter defaults to ''''.

\item  \verb|obj.SetStringDelimiter (char )| -  Get/set the character that will begin and end strings.  Microsoft
 Excel, for example, will export the following format:

 ''First Field'',''Second Field'',''Field, With, Commas'',''Fourth Field''

 The third field has a comma in it.  By using a string delimiter,
 this will be correctly read.  The delimiter defaults to ''''.

\item  \verb|obj.SetUTF8StringDelimiters (string delimiters)|

\item  \verb|string = obj.GetUTF8StringDelimiters ()|

\item  \verb|obj.SetUseStringDelimiter (bool )| -  Set/get whether to use the string delimiter.  Defaults to on.

\item  \verb|bool = obj.GetUseStringDelimiter ()| -  Set/get whether to use the string delimiter.  Defaults to on.

\item  \verb|obj.UseStringDelimiterOn ()| -  Set/get whether to use the string delimiter.  Defaults to on.

\item  \verb|obj.UseStringDelimiterOff ()| -  Set/get whether to use the string delimiter.  Defaults to on.

\item  \verb|bool = obj.GetHaveHeaders ()| -  Set/get whether to treat the first line of the file as headers.

\item  \verb|obj.SetHaveHeaders (bool )| -  Set/get whether to treat the first line of the file as headers.

\item  \verb|obj.SetMergeConsecutiveDelimiters (bool )| -  Set/get whether to merge successive delimiters.  Use this if (for
 example) your fields are separated by spaces but you don't know
 exactly how many.

\item  \verb|bool = obj.GetMergeConsecutiveDelimiters ()| -  Set/get whether to merge successive delimiters.  Use this if (for
 example) your fields are separated by spaces but you don't know
 exactly how many.

\item  \verb|obj.MergeConsecutiveDelimitersOn ()| -  Set/get whether to merge successive delimiters.  Use this if (for
 example) your fields are separated by spaces but you don't know
 exactly how many.

\item  \verb|obj.MergeConsecutiveDelimitersOff ()| -  Set/get whether to merge successive delimiters.  Use this if (for
 example) your fields are separated by spaces but you don't know
 exactly how many.

\item  \verb|vtkIdType = obj.GetMaxRecords ()| -  Specifies the maximum number of records to read from the file.  Limiting the
 number of records to read is useful for previewing the contents of a file.

\item  \verb|obj.SetMaxRecords (vtkIdType )| -  Specifies the maximum number of records to read from the file.  Limiting the
 number of records to read is useful for previewing the contents of a file.

\item  \verb|obj.SetDetectNumericColumns (bool )| -  When set to true, the reader will detect numeric columns and create
 vtkDoubleArray or vtkIntArray for those instead of vtkStringArray. Default
 is off.

\item  \verb|bool = obj.GetDetectNumericColumns ()| -  When set to true, the reader will detect numeric columns and create
 vtkDoubleArray or vtkIntArray for those instead of vtkStringArray. Default
 is off.

\item  \verb|obj.DetectNumericColumnsOn ()| -  When set to true, the reader will detect numeric columns and create
 vtkDoubleArray or vtkIntArray for those instead of vtkStringArray. Default
 is off.

\item  \verb|obj.DetectNumericColumnsOff ()| -  When set to true, the reader will detect numeric columns and create
 vtkDoubleArray or vtkIntArray for those instead of vtkStringArray. Default
 is off.

\item  \verb|obj.SetPedigreeIdArrayName (string )| -  The name of the array for generating or assigning pedigree ids
 (default ''id'').

\item  \verb|string = obj.GetPedigreeIdArrayName ()| -  The name of the array for generating or assigning pedigree ids
 (default ''id'').

\item  \verb|obj.SetGeneratePedigreeIds (bool )| -  If on (default), generates pedigree ids automatically.
 If off, assign one of the arrays to be the pedigree id.

\item  \verb|bool = obj.GetGeneratePedigreeIds ()| -  If on (default), generates pedigree ids automatically.
 If off, assign one of the arrays to be the pedigree id.

\item  \verb|obj.GeneratePedigreeIdsOn ()| -  If on (default), generates pedigree ids automatically.
 If off, assign one of the arrays to be the pedigree id.

\item  \verb|obj.GeneratePedigreeIdsOff ()| -  If on (default), generates pedigree ids automatically.
 If off, assign one of the arrays to be the pedigree id.

\item  \verb|obj.SetOutputPedigreeIds (bool )| -  If on, assigns pedigree ids to output. Defaults to off.

\item  \verb|bool = obj.GetOutputPedigreeIds ()| -  If on, assigns pedigree ids to output. Defaults to off.

\item  \verb|obj.OutputPedigreeIdsOn ()| -  If on, assigns pedigree ids to output. Defaults to off.

\item  \verb|obj.OutputPedigreeIdsOff ()| -  If on, assigns pedigree ids to output. Defaults to off.

\item  \verb|vtkStdString = obj.GetLastError ()| -  Returns a human-readable description of the most recent error, if any.
 Otherwise, returns an empty string.  Note that the result is only valid
 after calling Update().

\end{itemize}
