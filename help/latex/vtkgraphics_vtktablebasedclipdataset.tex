\section{vtkTableBasedClipDataSet}

\subsection{Usage}

  vtkTableBasedClipDataSet is a filter that clips any type of dataset using 
  either any subclass of vtkImplicitFunction or an input scalar point data 
  array. Clipping means that it actually ''cuts'' through the cells of the 
  dataset, returning everything outside the specified implicit function (or 
  greater than the scalar value) including ''pieces'' of a cell (Note to compare 
  this with vtkExtractGeometry, which pulls out entire, uncut cells). The 
  output of this filter is a vtkUnstructuredGrid data.

  To use this filter, you need to decide whether an implicit function or an 
  input scalar point data array is used for clipping. For the former case, 
  1) define an implicit function
  2) provide it to this filter via SetClipFunction()
  If a clipping function is not specified, or GenerateClipScalars is off( the 
  default), the input scalar point data array is then employed for clipping.

  You can also specify a scalar (iso-)value, which is used to decide what is
  inside and outside the implicit function. You can also reverse the sense of 
  what inside/outside is by setting IVAR InsideOut. The clipping algorithm 
  proceeds by computing an implicit function value or using the input scalar 
  point data value for each point in the dataset. This is compared against the 
  scalar (iso-)value to determine the inside/outside status.

  Although this filter sometimes (but rarely) may resort to the sibling class
  vtkClipDataSet for handling some special grids (such as cylinders or cones 
  with capping faces in the form of a vtkPolyData), it itself is able to deal
  with most grids. It is worth mentioning that vtkTableBasedClipDataSet is 
  capable of addressing the artifacts that may occur with vtkClipDataSet due 
  to the possibly inconsistent triagulation modes between neighboring cells. 
  In addition, the former is much faster than the latter. Furthermore, the 
  former produces less cells (with ratio usually being 5~6) than by the latter 
  in the output. In other words, this filter retains the original cells (i.e., 
  without triangulation / tetrahedralization) wherever possible. All these 
  advantages are gained by adopting the unique clipping and triangulation tables
  proposed by VisIt.


To create an instance of class vtkTableBasedClipDataSet, simply
invoke its constructor as follows
\begin{verbatim}
  obj = vtkTableBasedClipDataSet
\end{verbatim}
\subsection{Methods}

The class vtkTableBasedClipDataSet has several methods that can be used.
  They are listed below.
Note that the documentation is translated automatically from the VTK sources,
and may not be completely intelligible.  When in doubt, consult the VTK website.
In the methods listed below, \verb|obj| is an instance of the vtkTableBasedClipDataSet class.
\begin{itemize}
\item  \verb|string = obj.GetClassName ()|

\item  \verb|int = obj.IsA (string name)|

\item  \verb|vtkTableBasedClipDataSet = obj.NewInstance ()|

\item  \verb|vtkTableBasedClipDataSet = obj.SafeDownCast (vtkObject o)|

\item  \verb|long = obj.GetMTime ()| -  Get the MTime for which the point locator and clip function are consdiered.

\item  \verb|obj.SetInsideOut (int )| -  Set/Get the InsideOut flag. With this flag off, a vertex is considered
 inside (the implicit function or the isosurface) if the (function or scalar)
 value is greater than IVAR Value. With this flag on, a vertex is considered
 inside (the implicit function or the isosurface) if the (function or scalar)
 value is less than or equal to IVAR Value. This flag is off by default.

\item  \verb|int = obj.GetInsideOut ()| -  Set/Get the InsideOut flag. With this flag off, a vertex is considered
 inside (the implicit function or the isosurface) if the (function or scalar)
 value is greater than IVAR Value. With this flag on, a vertex is considered
 inside (the implicit function or the isosurface) if the (function or scalar)
 value is less than or equal to IVAR Value. This flag is off by default.

\item  \verb|obj.InsideOutOn ()| -  Set/Get the InsideOut flag. With this flag off, a vertex is considered
 inside (the implicit function or the isosurface) if the (function or scalar)
 value is greater than IVAR Value. With this flag on, a vertex is considered
 inside (the implicit function or the isosurface) if the (function or scalar)
 value is less than or equal to IVAR Value. This flag is off by default.

\item  \verb|obj.InsideOutOff ()| -  Set/Get the InsideOut flag. With this flag off, a vertex is considered
 inside (the implicit function or the isosurface) if the (function or scalar)
 value is greater than IVAR Value. With this flag on, a vertex is considered
 inside (the implicit function or the isosurface) if the (function or scalar)
 value is less than or equal to IVAR Value. This flag is off by default.

\item  \verb|obj.SetValue (double )| -  Set/Get the clipping value of the implicit function (if an implicit function
 is applied) or scalar data array (if a scalar data array is used), with 0.0
 as the default value. This value is ignored if flag UseValueAsOffset is true
 AND a clip function is defined.

\item  \verb|double = obj.GetValue ()| -  Set/Get the clipping value of the implicit function (if an implicit function
 is applied) or scalar data array (if a scalar data array is used), with 0.0
 as the default value. This value is ignored if flag UseValueAsOffset is true
 AND a clip function is defined.

\item  \verb|obj.SetUseValueAsOffset (bool )| -  Set/Get flag UseValueAsOffset, with true as the default value. With this flag
 on, IVAR Value is used as an offset parameter to the implicit function. Value 
 is used only when clipping using a scalar array.

\item  \verb|bool = obj.GetUseValueAsOffset ()| -  Set/Get flag UseValueAsOffset, with true as the default value. With this flag
 on, IVAR Value is used as an offset parameter to the implicit function. Value 
 is used only when clipping using a scalar array.

\item  \verb|obj.UseValueAsOffsetOn ()| -  Set/Get flag UseValueAsOffset, with true as the default value. With this flag
 on, IVAR Value is used as an offset parameter to the implicit function. Value 
 is used only when clipping using a scalar array.

\item  \verb|obj.UseValueAsOffsetOff ()| -  Set/Get flag UseValueAsOffset, with true as the default value. With this flag
 on, IVAR Value is used as an offset parameter to the implicit function. Value 
 is used only when clipping using a scalar array.

\item  \verb|obj.SetClipFunction (vtkImplicitFunction )|

\item  \verb|vtkImplicitFunction = obj.GetClipFunction ()|

\item  \verb|obj.SetGenerateClipScalars (int )| -  Set/Get flag GenerateClipScalars, with 0 as the default value. With this
 flag on, the scalar point data values obtained by evaluating the implicit
 function will be exported to the output. Note that this flag requries that
 an implicit function be provided.

\item  \verb|int = obj.GetGenerateClipScalars ()| -  Set/Get flag GenerateClipScalars, with 0 as the default value. With this
 flag on, the scalar point data values obtained by evaluating the implicit
 function will be exported to the output. Note that this flag requries that
 an implicit function be provided.

\item  \verb|obj.GenerateClipScalarsOn ()| -  Set/Get flag GenerateClipScalars, with 0 as the default value. With this
 flag on, the scalar point data values obtained by evaluating the implicit
 function will be exported to the output. Note that this flag requries that
 an implicit function be provided.

\item  \verb|obj.GenerateClipScalarsOff ()| -  Set/Get flag GenerateClipScalars, with 0 as the default value. With this
 flag on, the scalar point data values obtained by evaluating the implicit
 function will be exported to the output. Note that this flag requries that
 an implicit function be provided.

\item  \verb|obj.SetLocator (vtkIncrementalPointLocator locator)| -  Set/Get a point locator locator for merging duplicate points. By default, 
 an instance of vtkMergePoints is used. Note that this IVAR is provided
 in this class only because this filter may resort to its sibling class
 vtkClipDataSet when processing some special grids (such as cylinders or
 cones with capping faces in the form of a vtkPolyData) while the latter
 requires a point locator. This filter itself does not need a locator.

\item  \verb|vtkIncrementalPointLocator = obj.GetLocator ()| -  Set/Get a point locator locator for merging duplicate points. By default, 
 an instance of vtkMergePoints is used. Note that this IVAR is provided
 in this class only because this filter may resort to its sibling class
 vtkClipDataSet when processing some special grids (such as cylinders or
 cones with capping faces in the form of a vtkPolyData) while the latter
 requires a point locator. This filter itself does not need a locator.

\item  \verb|obj.SetMergeTolerance (double )| -  Set/Get the tolerance used for merging duplicate points near the clipping 
 intersection cells. This tolerance may prevent the generation of degenerate
 primitives. Note that only 3D cells actually use this IVAR.

\item  \verb|double = obj.GetMergeToleranceMinValue ()| -  Set/Get the tolerance used for merging duplicate points near the clipping 
 intersection cells. This tolerance may prevent the generation of degenerate
 primitives. Note that only 3D cells actually use this IVAR.

\item  \verb|double = obj.GetMergeToleranceMaxValue ()| -  Set/Get the tolerance used for merging duplicate points near the clipping 
 intersection cells. This tolerance may prevent the generation of degenerate
 primitives. Note that only 3D cells actually use this IVAR.

\item  \verb|double = obj.GetMergeTolerance ()| -  Set/Get the tolerance used for merging duplicate points near the clipping 
 intersection cells. This tolerance may prevent the generation of degenerate
 primitives. Note that only 3D cells actually use this IVAR.

\item  \verb|obj.CreateDefaultLocator ()| -  Create a default point locator when none is specified. The point locator is
 used to merge coincident points.

\item  \verb|obj.SetGenerateClippedOutput (int )| -  Set/Get whether a second output is generated. The second output contains the
 polygonal data that is clipped away by the iso-surface.

\item  \verb|int = obj.GetGenerateClippedOutput ()| -  Set/Get whether a second output is generated. The second output contains the
 polygonal data that is clipped away by the iso-surface.

\item  \verb|obj.GenerateClippedOutputOn ()| -  Set/Get whether a second output is generated. The second output contains the
 polygonal data that is clipped away by the iso-surface.

\item  \verb|obj.GenerateClippedOutputOff ()| -  Set/Get whether a second output is generated. The second output contains the
 polygonal data that is clipped away by the iso-surface.

\item  \verb|vtkUnstructuredGrid = obj.GetClippedOutput ()| -  Return the clipped output.

\end{itemize}
