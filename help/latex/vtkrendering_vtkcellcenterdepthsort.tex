\section{vtkCellCenterDepthSort}

\subsection{Usage}

 vtkCellCenterDepthSort is a simple and fast implementation of depth
 sort, but it only provides approximate results.  The sorting algorithm
 finds the centroids of all the cells.  It then performs the dot product
 of the centroids against a vector pointing in the direction of the
 camera transformed into object space.  It then performs an ordinary sort
 on the result.


To create an instance of class vtkCellCenterDepthSort, simply
invoke its constructor as follows
\begin{verbatim}
  obj = vtkCellCenterDepthSort
\end{verbatim}
\subsection{Methods}

The class vtkCellCenterDepthSort has several methods that can be used.
  They are listed below.
Note that the documentation is translated automatically from the VTK sources,
and may not be completely intelligible.  When in doubt, consult the VTK website.
In the methods listed below, \verb|obj| is an instance of the vtkCellCenterDepthSort class.
\begin{itemize}
\item  \verb|string = obj.GetClassName ()|

\item  \verb|int = obj.IsA (string name)|

\item  \verb|vtkCellCenterDepthSort = obj.NewInstance ()|

\item  \verb|vtkCellCenterDepthSort = obj.SafeDownCast (vtkObject o)|

\item  \verb|obj.InitTraversal ()|

\item  \verb|vtkIdTypeArray = obj.GetNextCells ()|

\end{itemize}
