\section{vtkClipConvexPolyData}

\subsection{Usage}

 vtkClipConvexPolyData is a filter that clips a convex polydata with a set
 of planes. Its main usage is for clipping a bounding volume with frustum
 planes (used later one in volume rendering).

To create an instance of class vtkClipConvexPolyData, simply
invoke its constructor as follows
\begin{verbatim}
  obj = vtkClipConvexPolyData
\end{verbatim}
\subsection{Methods}

The class vtkClipConvexPolyData has several methods that can be used.
  They are listed below.
Note that the documentation is translated automatically from the VTK sources,
and may not be completely intelligible.  When in doubt, consult the VTK website.
In the methods listed below, \verb|obj| is an instance of the vtkClipConvexPolyData class.
\begin{itemize}
\item  \verb|string = obj.GetClassName ()|

\item  \verb|int = obj.IsA (string name)|

\item  \verb|vtkClipConvexPolyData = obj.NewInstance ()|

\item  \verb|vtkClipConvexPolyData = obj.SafeDownCast (vtkObject o)|

\item  \verb|obj.SetPlanes (vtkPlaneCollection planes)| -  Set all the planes at once using a vtkPlanes implicit function.
 This also sets the D value.

\item  \verb|vtkPlaneCollection = obj.GetPlanes ()| -  Set all the planes at once using a vtkPlanes implicit function.
 This also sets the D value.

\item  \verb|long = obj.GetMTime ()| -  Redefines this method, as this filter depends on time of its components
 (planes)

\end{itemize}
