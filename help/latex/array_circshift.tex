\section{CIRCSHIFT Circularly Shift an Array}

\subsection{USAGE}

Applies a circular shift along each dimension of a given array.  The
syntax for its use is
\begin{verbatim}
   y = circshift(x,shiftvec)
\end{verbatim}
where \verb|x| is an n-dimensional array, and \verb|shiftvec| is a vector of
integers, each of which specify how much to shift \verb|x| along the
corresponding dimension.  
\subsection{Example}

The following examples show some uses of \verb|circshift| on N-dimensional
arrays.
\begin{verbatim}
--> x = int32(rand(4,5)*10)

x = 

  5  1  5  5  9 
  5  7  3  0  5 
  8  4  9  9 10 
  2  9  7  2  8 

--> circshift(x,[1,0])

ans = 

  2  9  7  2  8 
  5  1  5  5  9 
  5  7  3  0  5 
  8  4  9  9 10 

--> circshift(x,[0,-1])

ans = 

  1  5  5  9  5 
  7  3  0  5  5 
  4  9  9 10  8 
  9  7  2  8  2 

--> circshift(x,[2,2])

ans = 

  9 10  8  4  9 
  2  8  2  9  7 
  5  9  5  1  5 
  0  5  5  7  3 

--> x = int32(rand(4,5,3)*10)

x = 

(:,:,1) = 

  4  4  5 10  8 
  5  1  7  4  6 
  8  6  3  5  8 
  1  4  3  3  4 

(:,:,2) = 

  9  8  8  4  7 
  7  7  4  5  8 
  8  5  9  5  6 
  8  2  5  1  3 

(:,:,3) = 

  6  9  2  1 10 
  1  6  7  9  9 
 10 10  2  6  1 
  7  4  0  6  4 

--> circshift(x,[1,0,0])

ans = 

(:,:,1) = 

  1  4  3  3  4 
  4  4  5 10  8 
  5  1  7  4  6 
  8  6  3  5  8 

(:,:,2) = 

  8  2  5  1  3 
  9  8  8  4  7 
  7  7  4  5  8 
  8  5  9  5  6 

(:,:,3) = 

  7  4  0  6  4 
  6  9  2  1 10 
  1  6  7  9  9 
 10 10  2  6  1 

--> circshift(x,[0,-1,0])

ans = 

(:,:,1) = 

  4  5 10  8  4 
  1  7  4  6  5 
  6  3  5  8  8 
  4  3  3  4  1 

(:,:,2) = 

  8  8  4  7  9 
  7  4  5  8  7 
  5  9  5  6  8 
  2  5  1  3  8 

(:,:,3) = 

  9  2  1 10  6 
  6  7  9  9  1 
 10  2  6  1 10 
  4  0  6  4  7 

--> circshift(x,[0,0,-1])

ans = 

(:,:,1) = 

  9  8  8  4  7 
  7  7  4  5  8 
  8  5  9  5  6 
  8  2  5  1  3 

(:,:,2) = 

  6  9  2  1 10 
  1  6  7  9  9 
 10 10  2  6  1 
  7  4  0  6  4 

(:,:,3) = 

  4  4  5 10  8 
  5  1  7  4  6 
  8  6  3  5  8 
  1  4  3  3  4 

--> circshift(x,[2,-3,1])

ans = 

(:,:,1) = 

  6  1 10 10  2 
  6  4  7  4  0 
  1 10  6  9  2 
  9  9  1  6  7 

(:,:,2) = 

  5  8  8  6  3 
  3  4  1  4  3 
 10  8  4  4  5 
  4  6  5  1  7 

(:,:,3) = 

  5  6  8  5  9 
  1  3  8  2  5 
  4  7  9  8  8 
  5  8  7  7  4 
\end{verbatim}
