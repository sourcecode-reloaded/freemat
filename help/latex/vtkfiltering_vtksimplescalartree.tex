\section{vtkSimpleScalarTree}

\subsection{Usage}

 vtkSimpleScalarTree creates a pointerless binary tree that helps search for
 cells that lie within a particular scalar range. This object is used to
 accelerate some contouring (and other scalar-based techniques).
 
 The tree consists of an array of (min,max) scalar range pairs per node in
 the tree. The (min,max) range is determined from looking at the range of
 the children of the tree node. If the node is a leaf, then the range is
 determined by scanning the range of scalar data in n cells in the
 dataset. The n cells are determined by arbitrary selecting cell ids from
 id(i) to id(i+n), and where n is specified using the BranchingFactor
 ivar. Note that leaf node i=0 contains the scalar range computed from
 cell ids (0,n-1); leaf node i=1 contains the range from cell ids (n,2n-1);
 and so on. The implication is that there are no direct lists of cell ids
 per leaf node, instead the cell ids are implicitly known.

To create an instance of class vtkSimpleScalarTree, simply
invoke its constructor as follows
\begin{verbatim}
  obj = vtkSimpleScalarTree
\end{verbatim}
\subsection{Methods}

The class vtkSimpleScalarTree has several methods that can be used.
  They are listed below.
Note that the documentation is translated automatically from the VTK sources,
and may not be completely intelligible.  When in doubt, consult the VTK website.
In the methods listed below, \verb|obj| is an instance of the vtkSimpleScalarTree class.
\begin{itemize}
\item  \verb|string = obj.GetClassName ()| -  Standard type related macros and PrintSelf() method.

\item  \verb|int = obj.IsA (string name)| -  Standard type related macros and PrintSelf() method.

\item  \verb|vtkSimpleScalarTree = obj.NewInstance ()| -  Standard type related macros and PrintSelf() method.

\item  \verb|vtkSimpleScalarTree = obj.SafeDownCast (vtkObject o)| -  Standard type related macros and PrintSelf() method.

\item  \verb|obj.SetBranchingFactor (int )| -  Set the branching factor for the tree. This is the number of
 children per tree node. Smaller values (minimum is 2) mean deeper
 trees and more memory overhead. Larger values mean shallower
 trees, less memory usage, but worse performance.

\item  \verb|int = obj.GetBranchingFactorMinValue ()| -  Set the branching factor for the tree. This is the number of
 children per tree node. Smaller values (minimum is 2) mean deeper
 trees and more memory overhead. Larger values mean shallower
 trees, less memory usage, but worse performance.

\item  \verb|int = obj.GetBranchingFactorMaxValue ()| -  Set the branching factor for the tree. This is the number of
 children per tree node. Smaller values (minimum is 2) mean deeper
 trees and more memory overhead. Larger values mean shallower
 trees, less memory usage, but worse performance.

\item  \verb|int = obj.GetBranchingFactor ()| -  Set the branching factor for the tree. This is the number of
 children per tree node. Smaller values (minimum is 2) mean deeper
 trees and more memory overhead. Larger values mean shallower
 trees, less memory usage, but worse performance.

\item  \verb|int = obj.GetLevel ()| -  Get the level of the scalar tree. This value may change each time the
 scalar tree is built and the branching factor changes.

\item  \verb|obj.SetMaxLevel (int )| -  Set the maximum allowable level for the tree. 

\item  \verb|int = obj.GetMaxLevelMinValue ()| -  Set the maximum allowable level for the tree. 

\item  \verb|int = obj.GetMaxLevelMaxValue ()| -  Set the maximum allowable level for the tree. 

\item  \verb|int = obj.GetMaxLevel ()| -  Set the maximum allowable level for the tree. 

\item  \verb|obj.BuildTree ()| -  Construct the scalar tree from the dataset provided. Checks build times
 and modified time from input and reconstructs the tree if necessary.

\item  \verb|obj.Initialize ()| -  Initialize locator. Frees memory and resets object as appropriate.

\item  \verb|obj.InitTraversal (double scalarValue)| -  Begin to traverse the cells based on a scalar value. Returned cells
 will have scalar values that span the scalar value specified.

\end{itemize}
