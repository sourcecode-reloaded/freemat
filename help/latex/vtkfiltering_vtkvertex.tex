\section{vtkVertex}

\subsection{Usage}

 vtkVertex is a concrete implementation of vtkCell to represent a 3D point.

To create an instance of class vtkVertex, simply
invoke its constructor as follows
\begin{verbatim}
  obj = vtkVertex
\end{verbatim}
\subsection{Methods}

The class vtkVertex has several methods that can be used.
  They are listed below.
Note that the documentation is translated automatically from the VTK sources,
and may not be completely intelligible.  When in doubt, consult the VTK website.
In the methods listed below, \verb|obj| is an instance of the vtkVertex class.
\begin{itemize}
\item  \verb|string = obj.GetClassName ()|

\item  \verb|int = obj.IsA (string name)|

\item  \verb|vtkVertex = obj.NewInstance ()|

\item  \verb|vtkVertex = obj.SafeDownCast (vtkObject o)|

\item  \verb|int = obj.GetCellType ()| -  See the vtkCell API for descriptions of these methods.

\item  \verb|int = obj.GetCellDimension ()| -  See the vtkCell API for descriptions of these methods.

\item  \verb|int = obj.GetNumberOfEdges ()| -  See the vtkCell API for descriptions of these methods.

\item  \verb|int = obj.GetNumberOfFaces ()| -  See the vtkCell API for descriptions of these methods.

\item  \verb|vtkCell = obj.GetEdge (int )| -  See the vtkCell API for descriptions of these methods.

\item  \verb|vtkCell = obj.GetFace (int )| -  See the vtkCell API for descriptions of these methods.

\item  \verb|obj.Clip (double value, vtkDataArray cellScalars, vtkIncrementalPointLocator locator, vtkCellArray pts, vtkPointData inPd, vtkPointData outPd, vtkCellData inCd, vtkIdType cellId, vtkCellData outCd, int insideOut)| -  See the vtkCell API for descriptions of these methods.

\item  \verb|int = obj.CellBoundary (int subId, double pcoords[3], vtkIdList pts)| -  Given parametric coordinates of a point, return the closest cell
 boundary, and whether the point is inside or outside of the cell. The
 cell boundary is defined by a list of points (pts) that specify a vertex
 (1D cell).  If the return value of the method is != 0, then the point is
 inside the cell.

\item  \verb|obj.Contour (double value, vtkDataArray cellScalars, vtkIncrementalPointLocator locator, vtkCellArray verts1, vtkCellArray lines, vtkCellArray verts2, vtkPointData inPd, vtkPointData outPd, vtkCellData inCd, vtkIdType cellId, vtkCellData outCd)| -  Generate contouring primitives. The scalar list cellScalars are
 scalar values at each cell point. The point locator is essentially a
 points list that merges points as they are inserted (i.e., prevents
 duplicates).

\item  \verb|int = obj.GetParametricCenter (double pcoords[3])| -  Return the center of the triangle in parametric coordinates.

\item  \verb|int = obj.Triangulate (int index, vtkIdList ptIds, vtkPoints pts)| -  Triangulate the vertex. This method fills pts and ptIds with information
 from the only point in the vertex.

\item  \verb|obj.Derivatives (int subId, double pcoords[3], double values, int dim, double derivs)| -  Get the derivative of the vertex. Returns (0.0, 0.0, 0.0) for all
 dimensions.

\item  \verb|obj.InterpolateFunctions (double pcoords[3], double weights[1])| -  Compute the interpolation functions/derivatives
 (aka shape functions/derivatives)

\item  \verb|obj.InterpolateDerivs (double pcoords[3], double derivs[3])|

\end{itemize}
