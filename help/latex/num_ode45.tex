\section{ODE45 Numerical Solution of ODEs}

\subsection{Usage}

 function [t,y] = ode45(f,tspan,y0,options,varargin)
 function SOL   = ode45(f,tspan,y0,options,varargin)

 ode45 is a solver for ordinary differential equations and initial value problems.
 To solve the ODE
\begin{verbatim}
      y'(t) =  f(t,y)
      y(0)  =  y0
\end{verbatim}
 over the interval tspan=[t0 t1], you can use ode45. For example, to solve
 the ode

      y'   =  y
      y(0) =  1

 whose exact solution is y(t)=exp(t), over the interval t0=0, t1=3, do
\begin{verbatim}
-->       [t,y]=ode45(@(t,y) y,[0 3],1)
Warning: Newly defined variable error shadows a function of the same name.  Use clear error to recover access to the function

k = 
 2 


y = 
    1.0000    1.0030 


t = 

   1.0e-03 * 
         0    3.0000 


k = 
 3 


y = 
    1.0000    1.0030    1.0182 


t = 

   1.0e-02 * 
         0    0.3000    1.8000 


k = 
 4 


y = 
    1.0000    1.0030    1.0182    1.0975 


t = 

   1.0e-02 * 
         0    0.3000    1.8000    9.3000 


k = 
 5 


y = 
    1.0000    1.0030    1.0182    1.0975    1.4814 


t = 
         0    0.0030    0.0180    0.0930    0.3930 


k = 
 6 


y = 
    1.0000    1.0030    1.0182    1.0975    1.4814    1.9997 


t = 
         0    0.0030    0.0180    0.0930    0.3930    0.6930 


k = 
 7 


y = 

 Columns 1 to 6

    1.0000    1.0030    1.0182    1.0975    1.4814    1.9997 

 Columns 7 to 7

    2.6993 


t = 

 Columns 1 to 6

         0    0.0030    0.0180    0.0930    0.3930    0.6930 

 Columns 7 to 7

    0.9930 


k = 
 8 


y = 

 Columns 1 to 6

    1.0000    1.0030    1.0182    1.0975    1.4814    1.9997 

 Columns 7 to 8

    2.6993    3.6437 


t = 

 Columns 1 to 6

         0    0.0030    0.0180    0.0930    0.3930    0.6930 

 Columns 7 to 8

    0.9930    1.2930 


k = 
 9 


y = 

 Columns 1 to 6

    1.0000    1.0030    1.0182    1.0975    1.4814    1.9997 

 Columns 7 to 9

    2.6993    3.6437    4.9185 


t = 

 Columns 1 to 6

         0    0.0030    0.0180    0.0930    0.3930    0.6930 

 Columns 7 to 9

    0.9930    1.2930    1.5930 


k = 
 10 


y = 

 Columns 1 to 6

    1.0000    1.0030    1.0182    1.0975    1.4814    1.9997 

 Columns 7 to 10

    2.6993    3.6437    4.9185    6.6392 


t = 

 Columns 1 to 6

         0    0.0030    0.0180    0.0930    0.3930    0.6930 

 Columns 7 to 10

    0.9930    1.2930    1.5930    1.8930 


k = 
 11 


y = 

 Columns 1 to 6

    1.0000    1.0030    1.0182    1.0975    1.4814    1.9997 

 Columns 7 to 11

    2.6993    3.6437    4.9185    6.6392    8.9620 


t = 

 Columns 1 to 6

         0    0.0030    0.0180    0.0930    0.3930    0.6930 

 Columns 7 to 11

    0.9930    1.2930    1.5930    1.8930    2.1930 


k = 
 12 


y = 

 Columns 1 to 6

    1.0000    1.0030    1.0182    1.0975    1.4814    1.9997 

 Columns 7 to 12

    2.6993    3.6437    4.9185    6.6392    8.9620   12.0975 


t = 

 Columns 1 to 6

         0    0.0030    0.0180    0.0930    0.3930    0.6930 

 Columns 7 to 12

    0.9930    1.2930    1.5930    1.8930    2.1930    2.4930 


k = 
 13 


y = 

 Columns 1 to 6

    1.0000    1.0030    1.0182    1.0975    1.4814    1.9997 

 Columns 7 to 12

    2.6993    3.6437    4.9185    6.6392    8.9620   12.0975 

 Columns 13 to 13

   16.3299 


t = 

 Columns 1 to 6

         0    0.0030    0.0180    0.0930    0.3930    0.6930 

 Columns 7 to 12

    0.9930    1.2930    1.5930    1.8930    2.1930    2.4930 

 Columns 13 to 13

    2.7930 


k = 
 14 


y = 

 Columns 1 to 6

    1.0000    1.0030    1.0182    1.0975    1.4814    1.9997 

 Columns 7 to 12

    2.6993    3.6437    4.9185    6.6392    8.9620   12.0975 

 Columns 13 to 14

   16.3299   20.0854 


t = 

 Columns 1 to 6

         0    0.0030    0.0180    0.0930    0.3930    0.6930 

 Columns 7 to 12

    0.9930    1.2930    1.5930    1.8930    2.1930    2.4930 

 Columns 13 to 14

    2.7930    3.0000 

t = 

 Columns 1 to 6

         0    0.0030    0.0180    0.0930    0.3930    0.6930 

 Columns 7 to 12

    0.9930    1.2930    1.5930    1.8930    2.1930    2.4930 

 Columns 13 to 14

    2.7930    3.0000 

y = 
    1.0000 
    1.0030 
    1.0182 
    1.0975 
    1.4814 
    1.9997 
    2.6993 
    3.6437 
    4.9185 
    6.6392 
    8.9620 
   12.0975 
   16.3299 
   20.0854 
\end{verbatim}
 If you want a dense output (i.e., an output that also contains an interpolating
 spline), use instead
\begin{verbatim}
-->       SOL=ode45(@(t,y) y,[0 3],1)
Warning: Newly defined variable error shadows a function of the same name.  Use clear error to recover access to the function

k = 
 2 


y = 
    1.0000    1.0030 


t = 

   1.0e-03 * 
         0    3.0000 


k = 
 3 


y = 
    1.0000    1.0030    1.0182 


t = 

   1.0e-02 * 
         0    0.3000    1.8000 


k = 
 4 


y = 
    1.0000    1.0030    1.0182    1.0975 


t = 

   1.0e-02 * 
         0    0.3000    1.8000    9.3000 


k = 
 5 


y = 
    1.0000    1.0030    1.0182    1.0975    1.4814 


t = 
         0    0.0030    0.0180    0.0930    0.3930 


k = 
 6 


y = 
    1.0000    1.0030    1.0182    1.0975    1.4814    1.9997 


t = 
         0    0.0030    0.0180    0.0930    0.3930    0.6930 


k = 
 7 


y = 

 Columns 1 to 6

    1.0000    1.0030    1.0182    1.0975    1.4814    1.9997 

 Columns 7 to 7

    2.6993 


t = 

 Columns 1 to 6

         0    0.0030    0.0180    0.0930    0.3930    0.6930 

 Columns 7 to 7

    0.9930 


k = 
 8 


y = 

 Columns 1 to 6

    1.0000    1.0030    1.0182    1.0975    1.4814    1.9997 

 Columns 7 to 8

    2.6993    3.6437 


t = 

 Columns 1 to 6

         0    0.0030    0.0180    0.0930    0.3930    0.6930 

 Columns 7 to 8

    0.9930    1.2930 


k = 
 9 


y = 

 Columns 1 to 6

    1.0000    1.0030    1.0182    1.0975    1.4814    1.9997 

 Columns 7 to 9

    2.6993    3.6437    4.9185 


t = 

 Columns 1 to 6

         0    0.0030    0.0180    0.0930    0.3930    0.6930 

 Columns 7 to 9

    0.9930    1.2930    1.5930 


k = 
 10 


y = 

 Columns 1 to 6

    1.0000    1.0030    1.0182    1.0975    1.4814    1.9997 

 Columns 7 to 10

    2.6993    3.6437    4.9185    6.6392 


t = 

 Columns 1 to 6

         0    0.0030    0.0180    0.0930    0.3930    0.6930 

 Columns 7 to 10

    0.9930    1.2930    1.5930    1.8930 


k = 
 11 


y = 

 Columns 1 to 6

    1.0000    1.0030    1.0182    1.0975    1.4814    1.9997 

 Columns 7 to 11

    2.6993    3.6437    4.9185    6.6392    8.9620 


t = 

 Columns 1 to 6

         0    0.0030    0.0180    0.0930    0.3930    0.6930 

 Columns 7 to 11

    0.9930    1.2930    1.5930    1.8930    2.1930 


k = 
 12 


y = 

 Columns 1 to 6

    1.0000    1.0030    1.0182    1.0975    1.4814    1.9997 

 Columns 7 to 12

    2.6993    3.6437    4.9185    6.6392    8.9620   12.0975 


t = 

 Columns 1 to 6

         0    0.0030    0.0180    0.0930    0.3930    0.6930 

 Columns 7 to 12

    0.9930    1.2930    1.5930    1.8930    2.1930    2.4930 


k = 
 13 


y = 

 Columns 1 to 6

    1.0000    1.0030    1.0182    1.0975    1.4814    1.9997 

 Columns 7 to 12

    2.6993    3.6437    4.9185    6.6392    8.9620   12.0975 

 Columns 13 to 13

   16.3299 


t = 

 Columns 1 to 6

         0    0.0030    0.0180    0.0930    0.3930    0.6930 

 Columns 7 to 12

    0.9930    1.2930    1.5930    1.8930    2.1930    2.4930 

 Columns 13 to 13

    2.7930 


k = 
 14 


y = 

 Columns 1 to 6

    1.0000    1.0030    1.0182    1.0975    1.4814    1.9997 

 Columns 7 to 12

    2.6993    3.6437    4.9185    6.6392    8.9620   12.0975 

 Columns 13 to 14

   16.3299   20.0854 


t = 

 Columns 1 to 6

         0    0.0030    0.0180    0.0930    0.3930    0.6930 

 Columns 7 to 12

    0.9930    1.2930    1.5930    1.8930    2.1930    2.4930 

 Columns 13 to 14

    2.7930    3.0000 


SOL = 
    x: 1 14 double array
    y: 1 14 double array
    xe: 
    ye: 
    ie: 
    solver: generic_ode_solver
    interpolant: 1 1 functionpointer array
    idata: 1 1 struct array
\end{verbatim}
 You can view the result using
\begin{verbatim}
      plot(0:0.01:3,deval(SOL,0:0.01:3))
\end{verbatim}
 You will notice that this function is available for "every" value of t, while

      plot(t,y,'o-')

 is only available at a few points.

 The optional argument 'options' is a structure. It may contain any of the
 following fields:

 'AbsTol'      - Absolute tolerance, default is 1e-6.
 'RelTol'      - Relative tolerance, default is 1e-3.
 'MaxStep'     - Maximum step size, default is (tspan(2)-tspan(1))/10
 'InitialStep' - Initial step size, default is maxstep/100
 'Stepper'     - To override the default Fehlberg integrator
 'Events'      - To provide an event function
 'Projection'  - To provide a projection function

 The varargin is ignored by this function, but is passed to all your callbacks, i.e.,
 f, the event function and the projection function.

 ==Event Function==

 The event function can be used to detect situations where the integrator should stop,
 possibly because the right-hand-side has changed, because of a collision, etc...

 An event function should look like

    function [val,isterminal,direction]=event(t,y,...)

 The return values are:

 val        - the value of the event function.
 isterminal - whether or not this event should cause termination of the integrator.
 direction  - 1=upcrossings only matter, -1=downcrossings only, 0=both.

 == Projection function ==

 For geometric integration, you can provide a projection function which will be
 called after each time step. The projection function has the following signature:

     function yn=project(t,yn,...);

 If the output yn is very different from the input yn, the quality of interpolation
 may decrease.
